% Arithmetic in Global Fields - A User's Guide
% Copyright (C) 1990 Kevin R. Coombes and David R. Grant
% Distributed under the terms of the GNU General Public License as
% published by the Free Software Foundation, without any warranty.
% See the COPYRIGHT notice with this distribution for details.
%
% layout
%
\magnification = \magstep1
%
\font\title = cmssbx10 scaled \magstep2
\font\subtitle = cmss10 scaled \magstep1
%
\newcount\secno
\secno=0
\def\section#1\par{\vskip0pt plus .2\vsize\penalty-250
    \vskip0pt plus-.2\vsize\bigskip\vskip\parskip
    \advance\secno by 1 \message{#1}
   \centerline{\bf \the\secno:\enspace #1}\nobreak\smallskip}
%
\def\*{\begingroup\tt}
\def\@{\endgroup}
%
% text
%
\vglue 1.0 truein

\centerline{\title Arithmetic in Global Fields}
\smallskip
\centerline{\subtitle A library of C functions for global fields}
\medskip
\centerline{\subtitle Kevin R. Coombes and David R. Grant}

\vskip 1.0 truein

\centerline{\bf Introduction}
\smallskip
Any algebraic number theorist knows that there is no essential
difference between the ring of integers and the ring of polynomials in
one variable over a finite field. The purpose of this document is to
describe a library of C functions which provides a common interface for
doing computations with arbitrary-precision integers (or rational
numbers) or with arbitrary degree polynomials (or rational functions)
over finite fields.

The most elementary things that must be done to use this package
(after installation) are:
\item{(i)} \*\#include\@ the appropriate header file. This consists of
exactly one of the files \*integer.h\@, \*polynomial.h\@,
\*fraction.h\@, \*rational.h\@, or \*quadratic.h\@. We have adopted
the convention that ``fraction'' refers to rational numbers, and
``rational'' refers to rational functions. At this time, arithmetic in
quadratic extensions is confined to the rings of integers in fields of
the form ${\bf Q}(\sqrt{D})$ where $D$ is an integer whose absolute
value is sufficiently small to fit into a \*short\@.
\item{(ii)} Before performing any operations, run the routine
\*Initial($p$)\@, where $p$ is the characteristic of the ring. The
main thing accomplished by this routine is the initialization of
certain storage areas needed during the computations.

\section Basic Arithmetic Operations

The type of an arbitrary precision integer or arbitrary degree
polynomial over a finite field is the same as ``pointer to unsigned
short''. Thus, the library functions which carry out the basic
arithmetic operations: 
\*\obeylines
Zadd(a, b)
Zsub(a, b)
Zmult(a, b)
\@

\noindent are all defined so that their arguments and return values
are of type  \*unsigned short *\@. These routines return, repectively,
the sum, difference, and product of \*a\@ and \*b\@.

Division is somewhat more complicated. It is declared as
\*\obeylines
unsigned short *Zdivalg(a, b, r)
unsigned short *a, *b, *r;
\@

\noindent The return value is the quotient of \*a\@ by \*b\@. If the pointer
\*r\@ is not \*NULL\@, then the remainder is stored at that
location. It is the programmer's responsibility to ensure that enough
space is available to hold all the digits of the
remainder. The conventions adopted for the quotient and remainder of
the division algorithm are: (i) The quotient of two polynomials is
always a monic polynomial. (ii) The remainder when dividing two
integers always has the same sign as the dividend \*a\@. (iii) The
relation $a = bq+r$ is guaranteed to be satisfied.

Both the ring of integers and the ring of polynomials are Euclidean;
that is, the euclidean algorithm can be used to compute greatest
common divisors. The routine which implements this is
\*\obeylines
void Zgcd(a, b, g, aa, bb)
unsigned short *a, *b, *g, *aa, *bb;
\@

\noindent The greatest common divisor of \*a\@ and \*b\@ is stored at
\*g\@; the quotients \*a/g\@ and \*b/g\@ are stored at \*aa\@ and \*bb\@,
respectively. The user must make sure that enough space exists at
these locations to store the answers. The conventions used are that
the greatest common divisor of two polynomials is always monic, and
the greatest common divisor of two integers is always positive.

There are two different versions of each of these functions---one for
integers and one for polynomials. They have a common interface, and
the external definitions needed to use any of them are contained in
the header file \*Ztools.h\@. A file which only uses the basic
operations can do so by \*\#include\@'ing this header; it must be linked
with the correct library at compile time. The library of integer
functions is called \*libZ.a\@; the library of polynomial functions is
called \*libP.a\@. The distribution includes some test files to
illustrate how programs can be written to work regtardless of whether
integers or polynomials are used.

For example, suppose a program to use these routines is contained in
the file \*foo.c\@. Two different executable routines can be
maintained by the commands
\*\obeylines
cc -Iincdir -Llibdir foo.c -lZ -o foo\_Z
cc -Iincdir -Llibdir foo.c -lP -o foo\_P
\@
\noindent where the header files are contained in the directory
\*incdir\@ and the libraries are stored in the directory \*libdir\@.

\section Integer and Polynomial Utilities

Additional utilities have been provided to make it easier to deal with
the integer and polynomial routines described above. These utilities
are kept in yet another library, \*libU.a\@. Most of them
are implemented as a single routine which takes a ``toggle'' as one of
its arguments to determine whether or not the other arguments
represent integers or polynomials. In order to be able to write
programs which don't explicitly access this toggle, header files are
provided which define macros to handle this. The usual method of
writing code which uses any of these routines is to put
\*\obeylines
\#ifdef INTEGERS
\#include ``integer.h''
\#elif defined POLYNOMIALS
\#include ``polynomial.h''
\#else
some line to make the compiler complain
\#endif
\@
\noindent at the top of the file. The header files \*integer.h\@ and 
\*polynomial.h\@ automatically include the common part of the
interface; namely, \*Ztools.h\@. In order to maintain two versions of
a program which uses these utilities, they must be compiled with the
lines 
\*\obeylines
cc -DINTEGERS -Iincdir -Llibdir foo.c -lZ -lU -o foo\_Z
cc -DPOLYNOMIALS -Iincdir -Llibdir foo.c -lP -lU -o foo\_P
\@

\medskip\noindent The declarations of the integer and polynomial
utilities are:

\*\obeylines
char Zeq(a, b);
unsigned short *a, *b;
\qquad
char Zlessoreq(a, b);
unsigned short *a, *b;
\qquad
unsigned short Size(a);
unsigned short *a;
\qquad
unsigned short *Constant(l)
long l;
\qquad
void Shcopy(a, b)
unsigned short *a, *b;
\qquad
\@

The first two routines are boolean functions which compare \*a\@ and
\*b\@. They return a non-zero value for true and the value zero for
false. True means that $a=b$ or $a\le b$, respectively.

The utility \*Size()\@ determines the number of digits occupied by an
integer; it throws away any high-order zeros. If you
don't mind wasting a little space, calls to \*Size()\@ can be replaced
by using the fact that the first location occupied by a very long
integer or polynomial is at least as big as the value returned by
\*Size()\@. It is generally used
when determining how much space is needed to store some integer or
polynomial which is about to be copied into some safe location. The
utility \*Shcopy()\@ does the copying; it copies the 
contents of the integer or polynomial \*a\@ into the location pointed
to by \*b\@. The programmer must take care to ensure that enough space
is available; \*Shcopy()\@ doesn't care, and will blindly go on
copying and destroying data if you haven't provided enough space. One
of the most common uses of \*Shcopy()\@ is to put the value of some
constant in a safe place. The utility \*Constant()\@ takes as its
input a \*long\@ whose absolute value must fit into an \*unsigned short\@;
it returns an integer or polynomial correctly formatted to be
compatible with the multi-precision routines. However, \*Constant()\@
continually re-uses the same storage locations. The next call will
write over the data. So, one usually invokes this in the form
\smallskip
\*\obeylines
unsigned short One[3];
Shcopy(Constant(1), One);
\@
\smallskip
\noindent Note that integer constants, as explained below in the section on
storage, actually require three unsigned shorts to hold the data. The
ratio of useful information to number of storage locations improves as
the size of the numbers increases (fortunately).

\section Input and Output

The desire to keep input and output uniform for both integers and
polynomials, combined with an attempt to conserve space, has lead to
an unusual requirement for the entering of integers. To avoid this
embarrassing disclosure as long as possible, we begin by discussing
polynomials. Polynomials are entered and output as a string of
coefficients separated by white space; the coefficients of the highest
degree terms are entered first, and the constant term is entered last.
Polynomials can be continued onto the next line by using the character
\*\&\@ as the last character of a line. An unusual convention is that
all non-digits entered are treated as white space. This convention is
adopted to be compatible with the output routines, and to make it
easier to search and save relevant lines from a long output file. For,
the output routine will usually write lines of the form

\*\obeylines
mypoly = 1 2 3 4 1 2 3 4 1 2 3 4 \&
(more mypoly) 1 2 3 4
\@

Such lines can be conveniently picked out by searching for the name
\*mypoly\@, and can be used as the input for another program that uses
this package without any modification.

The input routine is declared as:
\smallskip
\*\obeylines
char Collect(msg, a)
char *msg;
unsigned short *a;
\@
\smallskip
\noindent and the output routine by
\smallskip
\*\obeylines
void Showme(msg, a)
char *msg;
unsigned short *a;
\@
The function \*Collect()\@ will store the input in the location
pointed to by \*a\@. It is the user's responsibility to ensure that
enough space is available. There is a limit to the number of
coefficients that can be entered in a single polynomial; currently,
that limit is set at 1024. If it becomes necessary to increase this,
you must change the value of the constant \*MAXDIGIT\@ in the file
\*collect.c\@ and recompile the utility library. The output routine
\*Showme()\@ has no such limitations. For both functions, the
parameter \*msg\@ is a string describing the input or output.

In addition, the output routine comes in a flavor amenable to the use
of a file other than the standard one for output. The routine is
called \*Fshowme(fp, msg,a)\@, and the first parameter is a pointer to
an open \*FILE\@ where the output should be written.

And now for integers. No provision has yet been made for decimal input
or output. In order to use the same routines, we have chosen to
represent integers in the base 65536, which is the size of an
\*unsigned short\@ on our machine. They are entered as for
polynomials, as digits to this base which are separated by white
space, with the most significant digit entered first and the least
signifigant digit entered last. In addition, they can be preceeded by
an optional sign, and continued on the following line using the
ampersand \*\&\@. The routines which do input and output have the same
declarations.

\section Rational Numbers and Rational Functions

There is a library \*libQ.a\@ of functions to do rational arithmatic
with either 
integers or polynomials. These routines are written so that they only
refer to functions defined in \*Ztools.h\@, and thus can be compiled
without explicit reference to either integers or polynomials. However,
before being used they must be linked with the correct library.
Moreover, most applications which use these routines will also use
some utilities that need to know whether integers or polynomials are
intended. Thus, the standard method is similar to that suggested
before. If you want to be able to maintain a file \*foo.c\@ that works
with either integers or polynomials, you have to include code along
the lines of
\*\obeylines
\#ifdef FRACTIONS
\#include ``fraction.h''
\#elif defined RATIONALS
\#include ``rational.h''
\#else
panic
\#endif
\@
\smallskip
\noindent and compile with one of the lines 
\smallskip
\*\obeylines
cc -DFRACTIONS -Iincdir -Llibdir foo.c -lQ -lZ -lU -o foo\_Z
cc -DRATIONALS -Iincdir -Llibdir foo.c -lQ -lP -lU -o foo\_P
\@
\smallskip

\noindent All the rational function routines are written to use the
obvious structure, defined as 
\*\obeylines
typedef struct $\{$
\qquad unsigned short *num;
\qquad unsigned short *den;
$\}$ ratl;
\@
\smallskip\noindent Thus, the functions 
\smallskip
\*\obeylines
Qadd(a,b)
Qsub(a,b)
Qminus(a)
Qmult(a,b)
Qdiv(a,b)
Qinv(a)
\@
\smallskip\noindent all take parameters of type \*ratl\@ and return an
answer of the same type. They return, unsurprisingly, the sum,
difference, negative, product, quotient, and reciprocal, respectively,
of their arguments. There is also a boolean function \*Qeq(a,b)\@
which returns a \*char\@ which has a non-zero value if and only if its
arguments represent the same fraction or rational function.

A similar collection of utilities exists to support rational
operations; these are all contained in the standard utility library
\*libU.a\@. These utilities are
\smallskip\*\obeylines
char Inratl(msg, r, s);
char *msg;
ratl r;
short s;
\qquad
void Ratshowme(msg, r)
char *msg;
ratl r;
\qquad
ratl Ratconst(l)
long l;
\qquad
void Ratcopy(a,b)
ratl a,b;
\@
\medskip
The input routine \*Inratl()\@ makes use of \*Collect()\@ to read in
the numerator and denominator of a rational structure separately.
these will be stored in \*r\@. To permit shortcuts for the frequent
occasions when the inputs are actually integers even though the
computations will make use of rationals, the parameter \*s\@ is
provided to convey this infomation. This parameter can only
legitimately take on the values \*INTEG\_IN\@ or \*FRAC\_IN\@ which are
\*\#define\@d in the header file \*myinput.h\@.

The output routine \*Ratshowme()\@ uses the standard integer output
routine \*Showme()\@ to write out a rational structure in two parts.
It also has a file oriented version \*FRatshowme(fp, msg, r)\@, whose
first parameter is a pointer to an open \*FILE\@ on which the output
is to be written.

The routine \*Ratconst()\@ returns a rational structure representing
the integer \*l\@. It has the same drawbacks as the function
\*Constant()\@ in that it re-uses storage, and so its value should be
copied to a safe place using \*Ratcopy()\@. The latter, in turn, is
equally unsafe, in that it will overwrite data if you have not made
sure that the second argument contains enough available space to hold
all the data in its first argument.

\section Quadratic Number Fields

There is also a library \*libD.a\@ for doing arbitrary precision
computations in the ring of integers of a quadratic number field. This
package is not implemented for polynomials at this time, and it only
works for the rings of integers in a field of the form ${\bf
Q}(\sqrt{D})$ where the absolute value of $D$ is sufficiently small
that it will fit in an \*unsigned short\@. In order to use this
routine, you must (i) include the proper header file via the command
\*\#include ``Dstruct.h''\@, and (ii) link with the libraries
\*libD.a\@, \*libZ.a\@ and \*libU.a\@ at compile time.

These routines make use of a structure declared as
\*\obeylines
typedef struct quad {
\qquad  unsigned short *alpha;
\qquad  unsigned short *beta;
} quad;
\@
Each element of type \*quad\@ represents an integer of the form
$\alpha + \beta\omega$ where $\omega$ is either $(1+\sqrt{D})/2$ or
$\sqrt{D}$ depending on whether or not $D$ is congruent to 1 modulo
$4$.

The available routines are
\smallskip\*\obeylines
Dadd(a,b)
Dsub(a,b)
Dminus(a)
Dmult(a,b)
\@
\smallskip\noindent each of which takes \*quad\@s as arguments and
returns them as answers. They compute (you guessed it) the sum,
difference, negative, and product of their argument, respectively.
There is also a boolean function \*Deq(a,b)\@ as with the other pieces
of the package. Finally, there is a function
\*\obeylines
unsigned short *Norm(a)
quad a;
\@
which returns the very long integer which is the norm of the quadratic
integer \*a\@.

No provision has been made as yet for separate input and output
routines; these must be done using the integer utilities. There are
also no functions that do division, compute g.c.d.'s. or do rational
arithmetic with these numbers.

\section Storage

The basic decision on how to store the data was made based on the
desire to be able to treat integers and polynomials in a similar
manner. The first applications made of these functions were limited
primarily by space, so compact storage was made a premium. Thus, the
following system was decided upon: Data is stored as an array of
unsigned shorts, in little-endian order. Thus, the type of a
polynomial or arbitrary precision integer is the same as ``pointer to
unsigned short''. The value immediately pointed to is the offset of
the most significant digit or the highest coefficient. In the case of
polynomials, this value is one more than the degree. In the case of
integers, it is actually two more than the number of actual digits;
this occurs because the most significant digit is reserved as a
signbit, and so it is required to have one of the values 0 (for
positive) or \*NEGSIGNBIT\@ (for negative). An integer is written base
$2^{16} = 65536$ in order to make the most efficient use of space.

The next storage-related decision was that the user of the library
should have to spend a minimal amount of time worrying about it. Thus,
the basic routines automatically maintain a block of storage in which
to do scratch computations, and, when necessary, create the space
needed to store their return values. In order to do this and still cut
down on the number of calls to \* malloc()\@ and \* free()\@, the
following method was adopted. The program maintains a linked list of
storage blocks, each of which is called a \*warehouse\@. A
\*warehouse\@ is a structure defined as follows: 

\*\obeylines
typedef struct warehouse $\{$
\qquad unsigned short *array;
\qquad int nextin;
\qquad unsigned depth;
\qquad struct warehouse *newer;
$\}$ warehouse;
\@

\noindent The element \*array\@ in a \*warehouse\@ is a pointer to an
array containing a number of \*unsigned short\@'s equal to the value
of \*depth\@. The element \*nextin\@ is the offset from the beginning
of the array of the next available open space. When the warehouse gets
full, \*nextin\@ is set equal to $-1$, and a new warehouse is created
and linked into the end of the list; \*newer\@ points to the next
warehouse.

The principal routines which manipulate warehouses are declared by 
\*\obeylines
warehouse *makeWH(howbig)
unsigned howbig;
\qquad
unsigned short *capture(where, howmany)
warehouse where;
unsigned howmany;
\qquad
void release(where, mark)
warehouse *where;
unsigned short *mark;
\@

Whenever one of the routines that performs integer arithmetic needs
space either for a scratch computation or to store its answer, it
calls the function \*capture()\@ to get storage space from the
warehouse. The warehouses used in these computations are defined by
the call to \*Initial()\@ which must take place before any operations
on arbitrary length integers or polynomials can be performed. The
\*Initial()\@ routine, among other things, calls \*makeWH()\@ which
\*malloc()\@s and initializes a \*warehouse\@. Then \*capture()\@ gets
space for the requested number of \*unsigned short\@s out of the
\*warehouse\@, creating a new one if the old one doesn't have enough
room. When you can safely discard some intermediate computations and
recover some space, you call \*release()\@, which resets \*nextin\@ to
be the offset of the location pointed to by the \*mark\@, and
\*free()\@s any extra \*warehouse\@s.


\section Defects

It is clear that there should be a routine \*Zcmp(a, b)\@ which
returns positive, zero, or negative values depending on whether \*a\@
is greater than, equal to, or less than \*b\@.

It is equally clear that provision should be made for the decimal
input and output of arbitrarily long integers.

\section Installation

It should be possible to do the entire installation simply by typing
\*make\@ in the parent directory where this package was placed. Some
things that you might want to change are (i) the conventions involving
where the \*\#include\@ files eventually get stored, to make them
available for general users applications, (ii) the conventions for
where libraries get installed, if your \*C\@ compiler does not support
the \*-L\@ option, (iii) the definition of \*CFLAGS\@, and, most
importantly, (iv) the definition of \*ARCH\@ which depends heavily on
the fact that this program was developed on a Sun workstation in an
environment that included both Sun3 and Sun4 machines.

\vskip 1.0truein

{\obeylines
Kevin R. Coombes
Department of Mathematics
University of Michigan
Ann Arbor MI 48109
\enspace
kevin@math.lsa.umich.edu
\enspace
\enspace
David R. Grant
Department of Mathematics
University of Colorado
Boulder, CO 80309
\enspace
grant@boulder.colorado.edu
}

\end

