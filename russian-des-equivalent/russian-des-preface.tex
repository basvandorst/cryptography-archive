% russian-des-preface.tex - Cover letter for the translation of
% GOST 28147---89, the Russian equivalent of the U.S. Data Encryption
% Standard.

% Load a variety of files that provide functions not in plain TeX.

\input basic		% useful functions not in plain TeX
\input misc		% miscellaneous (non-fundamental) TeX macros
\input list		% simple list mechanism

% Increase the size of the type to about 14 points.

		\magnification=\magstep2

% In order to facilitate an open style of layout in the TeX
% manuscript, carriage return is made equivalent to a space 

			\catcode'15=10	

% so that double carriage returns don't break paragraphs.  Paragraph
% breaking is done by explicit \par markers.

% Provide extra space between paragaraphs.

		\parskip=1ex plus 1ex

% Use lower case roman numerals in page numbers

\footline={\hss\tenrm\romannumeral\pageno\hss\llap{DRAFT - \datetime}}

% End a line with a horizontall fill.  The effect is much like \rightline,
% but can occur in the middle of text.

		\def\endline{\hfill\break}

% beginning of cover letter
%\null%\vskip 1ex

\centerline{Government Standard of the U.S.S.R. 28147---89}
\centerline{Cryptographic Protection for Data Processing Systems}
\vskip 3ex

\centerline{Translated from the Russian by}
\vskip 1ex
\centerline{Aleksandr Malchik}
\vskip 1ex
\centerline{Sun Microsystems Laboratories}
\centerline{Mountain View, California}
\vskip 2ex
\centerline{with editorial and typographic assistance from}
\vskip 1ex
\centerline{Whitfield Diffie}
\vskip 1ex
\centerline{Sun Microsystems}
\centerline{Mountain View, California.}%
\vskip 6ex
\centerline{Preface to the English Translation}%
\vskip 2ex

    The Government Standard of the U.S.S.R. 28147---89, Cryptographic
Protection for Data Processing Systems, appears to have played a role
in the Soviet Union similar to that played by the U.S. Data Encryption
Standard (FIPS 46).  When issued, it bore the minimal classification
`For Official Use,' but is now said to be widely available in software
both in the Former Soviet Union and elsewhere.  In apparent contrast
to DES's explicit limitation to unclassified information, the
introduction to GOST 28147---89 contains the intriguing remark that
the cryptographic transformation algorithm ``does not place any
limitations on the secrecy level of the protected information.'' \par

    The algorithms are similar in that both operate on 64-bit blocks
by successively modifying half of the bits with a function of the
other half.  Beyond that, the similarity declines and several
differences are visible.

 \vskip 1ex
 \list{\bullets}
  \item The Soviet system has 32 rounds rather than the 16 of DES.

  \item Each round is somewhat simpler than a round of DES.  In the
	`f' function, 32 bits of text are added modulo 32 to 32
	bits of key, transformed by a block of eight, 4-bit to 4-bit
	S-boxes and rotated 11 bits to the left.

  \item In contrast to DES's meager 56 bits of key, GOST 28147---89
	has 256 bits of primary key and 512 bits of secondary key.
	The secondary key is the block of eight S-boxes, which are
	specific to individual networks and are not included in the
	standard.

  \item In place of the complex key schedule of DES, the primary key
	is divided into eight 32-bit words.  For the first twenty-four
	rounds, these are used cyclically in ascending order.  For
	the last eight, they are used in descending order.
  \endlist

\noindent The standard is also somewhat broader that FIPS46.  It 
includes output feedback and cipher feedback modes of operation,
both limited to 64-bit blocks, and a mode for producing message
authentication codes. \par

    The translation is intentionally colloquial, preferring standard
English terms to charmingly quaint cognates of the Russian.  The
objective is to make it easy for English speaking cryptographers to
identify what is novel with minimal effort.  In some places,
particularly the glossary, this has changed the wording considerably
and a few remarks on some of the freeer choices, seems in order. \par

 \list{}
  \item In Russian, the expression `the GAMMA of a cipher' means the
	keystream and the term for an additive keystream cipher is
	`GAMMIROVANIA.'  This has been translated as `output feedback,'
	because that is the way in which their keystream is generated
	and is the name of the corresponding mode in FIPS81.

  \item The Russian term that literally translates `keystream mode
	with feedback' corresponds to `cipher feedback mode' and that
	term has been used throughout.

  \item The Russian term for authentication translates, not unreasonably,
	as `imitation insertion protection.'  The term for authenticator,
	however, is literally the `imitation insertion.'  This has been
	rendered as `message authentication code.'
  \endlist

    Because of the substantial amount of mathematical notation, the
translation has been typeset in \TeX.  The numbers in boxes in the
margins show the approximate page boudaries in the original document.
\par

    Aside from the elimination of errors, which are undoubtedly still
numerous, the translation is complete except for the diagrams, which
will be added when a flowchart program with {\TeX} compatible
Postscript output is located.

    Please send comments and corrections by email to: %\endline
\hbox{whitfield.diffie@Eng.Sun.Com} or on paper to me at:

$$ \vbox{\halign{\lft{#}\cr
	 Sun Microsystems, MTV14-203 \cr
	 2550 Garcia Avenue \cr
	 Mountain View, CA 94043 \cr}} $$

\par\vskip 3ex

\leftline{\hskip 3in Whitfield Diffie}
\leftline{\hskip 3in Distinguished Engineer}
\leftline{\hskip 3in October 1993}

\vskip 6ex
\centerline{References}
\vskip 1ex

 \list{}
  \item Data Encryption Standard, National Bureau of Standards,
	Federal Information Processing Standard Publication
	No. 46, January 1977.

  \item DES Modes of Operation, , National Bureau of Standards, 
	Federal Information Processing Standards Publication
	No. 81, September 1980.
  \endlist

\bye

