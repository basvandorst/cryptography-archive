\chapter{Functions and Operations available in PARI and GP}

The functions and operators available in PARI and in the GP/PARI calculator
are numerous and everexpanding. Here is a description of the ones available
in version \vers. It should be noted that many of these functions accept
quite different types as arguments, but others are more restricted. The list
of acceptable types will be given for each function or class of functions.
Except when stated otherwise, it is understood that a function or operation
which should make natural sense is legal.
In this chapter, we will describe the functions according to a rough
classification. For the functions in alphabetical order, see the general 
index.

\section{Standard monadic or dyadic operators.}

\subsec{$\pm$} $+x$ and $-x$ refer to monadic operators
($+x$ does nothing and $-x$ negates $x$).

The library syntax is $\teb{gneg}(x)$ for $-x$.

\subsec{$+$, $-$} $x+y$ and $x-y$ are the \ref{sum} and \ref{difference} of $x$ and $y$.
Among the prominent impossibilities are addition/subtraction between a scalar
type and a vector or a matrix, between vector/matrices of incompatible sizes
and between an integermod and a real number.

The library syntax is $\teb{gadd}(x,y)$ for $x+y$, $\teb{gsub}(x,y)$
for $x-y$.

\subsec{$*$} $x*y$ is the \ref{product} of $x$ and $y$. Among the
prominent impossibilities are multiplication between vector/matrices of
incompatible sizes, between an integermod and a real number. Note
that because of vector and matrix operations, $*$ is not necessarily
commutative. Note also that since multiplication between two column or two
row
vectors is not allowed, to obtain the \ref{scalar product} of two vectors of the
same
length, you must multiply a line vector by a column vector, if necessary by
transposing one of the vectors (using {\tt \~{}} or {\tt trans$(\,)$}, see section 8).

The library syntax is $\teb{gmul}(x,y)$ for $x*y$.

\subsec{$/$} $x/y$ is the \ref{quotient} of $x$ and $y$. In addition to the
impossibilities for multiplication, note that if the divisor is a matrix, it
must be an invertible square matrix, and in that case the result is $x*y^{-1}$.
Furthermore note that the result is as
exact as possible: in particular, division of two integers always gives a
rational number (which may be an integer if the quotient is exact) and {\sl
not\/} the Euclidean quotient (see $x\backslash y$ for that), and similarly the
quotient of two polynomials is a rational function in general. To obtain the
approximate real value of the quotient of two integers, add $0.$ to the
result; to obtain the approximate $p$-adic value of the quotient of two
integers, add $O(p^k)$ to the result; finally,
to obtain the \ref{taylor series} expansion of the quotient of two polynomials,
add $O(X\hat{\ }k)$ to the result or use the {\tt taylor} function (see section 7).

The library syntax is $\teb{gdiv}(x,y)$ for $x/y$.

\subsec{$\backslash$} $x\backslash y$ is the \ref{euclidean quotient} of $x$ and $y$.
The types must be either
both integer or both polynomials. The result is the Euclidean quotient.
In the case of integer division, the quotient is such that the corresponding
remainder is nonnegative.

The library syntax is $\teb{gdivent}(x,y)$ for $x\backslash y$.

\subsec{$\%$} $x\%y$ is the \ref{euclidean remainder} of $x$ and $y$. The 
modulus $y$ must be of type integer or polynomial.
The result is the remainder, always nonnegative
in the case of integers. Allowed dividend types are scalar exact types when
the modulus is an integer, and polynomials, polymods and rational functions
when the modulus is a polynomial.

The library syntax is $\teb{gmod}(x,y)$ for $x\%y$.

\subsec{\ref{divres}}$(x,y)$ creates a vector with two components,
the first being the Euclidean quotient, the second the Euclidean remainder,
of the division of $x$ by $y$. This avoids the need to do two divisions if
one needs both the quotient and the remainder. The arguments must be both
integers or both polynomials, and in the case of integers the remainder
is of the sign of the dividend.

The library syntax is $\teb{gdiventres}(x,y)$.

\subsec{$\hat{}$} $x\hat{\ }y$ is \ref{powering}. If the exponent is an integer,
then exact
operations are performed using binary powering techniques. In particular, in
this case
the first argument cannot be a vector or matrix unless it is a square matrix
(and moreover invertible if the exponent is negative). If the exponent is
not of type integer, this is treated as a transcendental function (see section 3),
and in particular has the effect of componentwise powering on vector or
matrices.

The library syntax is $\teb{gpui}(x,y,prec)$ for $x\hat{\ }y$
where the precision parameter prec must be given only if $y$ is not
integral but $x$ is an exact (real or complex) type.

In addition, there are two shift functions (i.e. multiplication and division
by powers of 2) which are faster than the general multiplication or division
routines:

\subsecref{shift}$(x,n)$: shift $x$ componentwise left by n bits if $n\ge0$ and 
right by $-n$ bits if $n<0$. A left shift by $n$ corresponds to multiplication
by $2^n$, and a right shift by $-n$ corresponds to Euclidean division of $x$
by $2^{-n}$.

The library syntax is $\teb{gshift}(x,n)$ where $n$ is a C-integer.

\subsecref{shiftmul}$(x,n)$: multiply $x$ by $2^n$. The difference with
{\tt shift} is that when $n<0$, the ordinary division takes place, hence for
example if $x$ is an integer the result may be a fraction, while for 
{\tt shift} Euclidean division takes place when $n<0$ hence if $x$ is an
integer the result is still an integer.

The library syntax is $\teb{gmul2n}(x,n)$ where $n$ is a C-integer.

\subsec{comparison and \ref{boolean operators}}. 
The six standard \ref{comparison operators}
$<=$, $<$, $>=$, $>$, $==$, $!=$ are available in GP, and in library mode 
under the names {\bf \ref{gle}, \ref{glt}, \ref{gge}, \ref{ggt}, \ref{geq}, \ref{gne}} respectively. The 
library syntax is ${\bf co}(x,y)$, where co is the comparison 
operator. The result is 1 if the comparison is true, 0 if it is false.

The standard boolean functions $|\,|$ (\ref{inclusive or}) and $\&\&$ (\ref{and}) \sref{or}are
also available, and the library syntax is $\teb{gor}(x,y)$ and
$\teb{gand}(x,y)$ respectively. Note that to avoid confusion with the
factorial function, there is no $!$ (\ref{not}) operator, but this can easily be circumvented.

In library mode, it is in fact usually preferable to use the two basic 
functions which are $\teb{gcmp}(x,y)$ which gives
the sign (1, 0, or -1) of $x-y$, where $x$ and $y$ must be in $\Bbb R$,
and $\teb{gegal}(x,y)$ which can be applied to any two PARI objects
$x$ and $y$ and gives 1 (i.e. true) if they are equal (but not necessarily
identical), 0 (i.e. false) otherwise. Particular cases of gegal which
should be used are $\teb{gcmp0}(x)$ ($x=0$ ?), $\teb{gcmp1}(x)$
($x=1$ ?), and \sref{gcmp\string\_1} {\bf gcmp\_1} $(x)$ ($x=-1$ ?).

Note that $\teb{gcmp0}(x)$ tests whether $x$ is equal to zero, even if
$x$ is not an exact object. To test whether $x$ is an exact object which
is equal to zero, one must use $\teb{isexactzero}$.
\smallskip
GP accepts the following synonyms for some of the above functions:
since there is no bitwise {\tt and} or bitwise {\tt or}, $|$ and $\&$ are accepted as\sref{bitwise and} \sref{bitwise or}
synonyms of $|\,|$ and $\&\&$ respectively. Also, $<>$ is accepted as a
synonym for $!=$. On the other hand, $=$ is definitely {\sl not} a
synonym for $==$ since it is the assignment statement.

\subsec{lex}$(x,y)$: gives the result of a lexicographic comparison between $x$ and $y$.
This is to be interpreted in quite a wide sense. For example,  the vector $[1,3]$
will be considered smaller than the longer vector $[1,3,-1]$ (but of course larger
than $[1,2,5]$), i.e. {\tt lex([1,3],[1,3,5])} will give -1 as a result.

The library syntax is $\teb{lexcmp}(x,y)$.

\subsec{sign}$(x)$: \ref{sign} of $x$, which must be of type
integer, real or fraction. The result (0, 1 or -1) is a C-integer.

The library syntax is $\teb{gsigne}(x)$.

\subsec{\ref{max}}$(x,y)$ and {\bf \ref{min}}$(x,y)$ create the
maximum and minimum of $x$ and $y$ when they can be compared.

The library syntax is $\teb{gmax}(x,y)$ and $\teb{gmin}(x,y)$.

\subsectitle{Important remark.} In all the above library syntaxes, we have
given only the basic names of the functions. For example ${\tt gadd}(x,y
)$
assumes that $x$ and $y$ are PARI objects (the GEN type) and {\sl creates}
the result $x+y$ on the PARI stack. From most of these basic names, the names
and
effects of other functions can be obtained in the following way. We give
the example for gadd, but the same is true for all the other functions
above, and a few more that we have not given:

$\teb{gaddgs}(x,y)$: here $x$ is a GEN, $y$ is an ordinary 32-bit
integer.

$\teb{gaddsg}(x,y)$: here $x$ is an ordinary 32-bit integer, $y$ is a
GEN.

$\teb{gaddz}(x,y,z),\ \teb{gaddgsz}(x,y,z),\ \teb{gaddsgz}(x,y,z)$: here $z$ is a
preexisting GEN and the result of the corresponding operation is put in $z$.
The size of the PARI stack does not change.

As simplification to programming, for many functions beginning with a {\bf
g}, such
as gadd, one can replace the {\bf g} by an {\bf l} to obtain a {\bf long}
result instead
of a GEN (i.e. pointer to long) result. For instance $\teb{ladd}(x,y)$ is
identical with $({\tt (long)gadd}(x,y))$. These macros are written in
the files listed in {\tt \ref{gencom.h}}.

\section{Conversions and similar elementary functions or commands}

Many of the conversion functions are rounding or truncating operations.
In this case, if the argument is a rational function, the result is the
Euclidean quotient of the numerator by the denominator, and if the argument
is a vector or a matrix, the operation is done componentwise. This will not
be restated for every function.

\subsecref{allocatemem}$(x)$: this is a very special operation which
allows the user to change the stack size {\it after} initalization.
$x$ must be a non-negative integer. If $x>2$, a new stack of size
$16*\lceil x/16\rceil$ bytes will be allocated, all the Pari data on the
old stack will be moved to the new one, and the old stack will be freed
and discarded. If $x=0$, $1$ or $2$, the size of the new stack will be
double that of the old one. Finally, in case $x=0$, after doubling the stack,
a warning message will be printed.

Although it is a function, this must be the {\it only} instruction in a GP
line. The technical reason is that this program terminates by a longjmp
and not by a return.

The library syntax is $\teb{allocatemoremem}(x)$, where $x$ is an unsigned 
long, and the return type is void. GP uses a variant which ends by a longjmp.

\subsecref{binary}$(x)$: outputs the vector of the binary digits of $|x|$.
Here $x$ can be an integer, a real number (in which case the result has two
components, one for the integer part, one for the fractional part) or a
vector/matrix.

The library syntax is $\teb{binaire}(x)$.

\subsecref{bittest}$(x,n)$: outputs the $n^{\text{th}}$ bit of $|x|$ starting
from the right (i.e. the coefficient of $2^n$ in the binary expansion of $x$.
The result is 0 or 1.

The library syntax is $\teb{bittest}(x,n)$ where $n$ as well as the result
are C-integers.

\subsecref{bytesize}$(x)$: outputs the total number of bytes occupied by the
tree representing the Pari object $x$.

The library syntax is $\teb{taille2}(x)$ which returns a C {\tt long}. The
function {\tt taille} returns the number of {\it words} instead.

\subsec{\ref{ceil}}$(x)$: ceiling of $x$. When $x$ is in $\Bbb R$,
the result is the smallest integer greater than or equal to $x$.

The library syntax is $\teb{gceil}(x)$.

\subsecref{centerlift}$(x)$: lifts an element $x=a \bmod n$ of $\Bbb Z/n\Bbb Z$ to
$a$ in $\Bbb Z$, and similarly lifts a polymod to a polynomial. This is the same
as {\tt lift} except that in the particular case of elements of $\Bbb Z/n\Bbb Z$, the
lift $y$ is such that $-n/2<y\le n$. If $x$ is
of type fraction, complex, quadratic, polynomial, power series, rational
function, vector or matrix, the lift is done for each coefficient.
Forbidden types for $x$ are reals and $p$-adics. Note that the main variable
of the lift of a polymod will be variable number 0 (i.e. 'X') if the rules
concerning the creation of polymods explained in chapter 2 are observed.

The library syntax is $\teb{centerlift}(x)$.

\subsec{\ref{changevar}}$(x,y)$: change the variables in the object $x$
according to the permutation specified by the vector $y$.
For example, assume that the variables have been introduced in the
order {\tt x}, {\tt a}, {\tt b}, {\tt c}. Then,
if $y$ is the vector {\tt [x,c,a,b]}, the variable {\tt a} will be
replaced by {\tt c}, {\tt b} by {\tt a}, and {\tt a} by {\tt b}, {\tt x}
being unchanged. Note that the permutation must be completely
specified, e.g. {\tt [c,a,b]} would not work, since this would replace
{\tt x} by {\tt c}, and leave {\tt a} and {\tt b} unchanged.

The library syntax is $\teb{changevar}(x,y)$.

\subsec{\ref{components} of a PARI object}: 

There are essentially three ways to extract the components from a PARI object.

The first and most general, is the function $\teb{compo}(x,n)$ which extracts
the $n^{\text{th}}$-component of $x$. This is to be understood
as follows: every PARI type has one or two initial \ref{code words}. The components
are counted,
starting at 1, after these code words. In particular if $x$ is a vector, this
is indeed
the $n^{\text{th}}$-component of $x$, if $x$ is a matrix, the $n^{\text{th}}$
column,
if $x$ is a polynomial, the $n^{\text{th}}$ coefficient (i.e. of degree
$n-1$),
and for power series, the $n^{\text{th}}$ significant coefficient. The use
of the function {\tt compo} implies the knowledge of the structure of the
different PARI types, which can be recalled by typing $\backslash$t under GP.

The library syntax is $\teb{compo}(x,n)$, where $n$ is a 32-bit C-integer.

The two other methods are more natural but more restricted. First, the
function $\teb{coeff}(x,n)$ gives the coefficient of degree $n$ of the
polynomial or power series $x$, with respect to the main variable of $x$
(to see the order of the variables or to change it, use the function 
{\tt reorder}, see section 11). In particular if $n$ is less than the valuation of $x$ or
in the case of a polynomial, greater than the degree, the result is zero
(contrary to {\tt compo} which would send an error message). If $x$ is a power
series and $n$ is greater than the largest significant degree, then an error
message is issued.

For greater flexibility, vector or matrix types are also accepted for $x$, and
the meaning is then identical with that of {\tt compo}.

Finally note that a scalar type is considered by {\tt coeff} as a polynomial
of degree zero.

The library syntax is $\teb{truecoeff}(x,n)$.

The third method is specific to vectors or matrices under GP. If $x$ is a
(row or column) vector, then {\tt \ref{x[n]}} represents the $n^{\text{th}}$
component of $x$, i.e. {\tt compo(x,n)}. It is more natural and shorter to
write. If $x$ is a matrix, {\tt \ref{x[m,n]}} represents the coefficient of
row {\tt m} and column {\tt n} of the matrix, {\tt \ref{x[m,]}} represents
the $m^{\text{th}}$ {\sl row\/} of $x$, and {\tt \ref{x[,n]}} represents
the $n^{\text{th}}$ {\sl column\/} of $x$.

Finally note that in library mode, the macro {\tt coeff(x,m,n)} exists with
exactly the meaning of {\tt x[m,n]} under GP when $x$ is a matrix. This macro
should not be confused with the two-variable macro {\tt coeff} used primarily
for polynomials and power series under GP.\sref{coeff (in library mode)}

\subsecref{conj}$(x)$ or $x\_$: conjugate of $x$. The meaning of this
is clear, except that for real quadratic numbers, it means conjugation in the
real quadratic field. This function has no effect on integers, reals,
integermods, fractions or $p$-adics. The only forbidden type is polymod
(see {\tt conjvec} for this).

The library syntax is $\teb{gconj}(x)$.

\subsecref{conjvec}$(x)$: conjugate vector representation of $x$. If $x$ is
a polymod {\tt mod(a,q)}, this gives a vector of length degree(q) containing
the complex embeddings of the polymod if {\tt q} has integral or rational
coefficients, and the conjugates of the polymod if {\tt q} has some integermod
coefficients. If $x$ is a integer or a rational number, the result is $x$.
If $x$ is a (row or column) vector, the result is a matrix whose columns are
the conjugate vectors of the individual elements of $x$.

The library syntax is $\teb{conjvec}(x,prec)$, where prec is a C integer.

\subsecref{cvtoi}$(x)$: If $x$ is in $\Bbb R$, truncates $x$ to an
integer.
If the result is not significant because the exponent of $x$ is too large
compared to its precision, no error occurs, but the number of lost
significant
bits is put at the address specified in a second argument. This is transparent
under GP, but this second argument must be given in library mode. If no
digits
are lost, this second argument contains the negative of the number of
significant bits of the fractional part.

The library syntax is $\teb{gcvtoi}(x,\& e)$ where $e$ is a 32-bit
C-integer.

\subsecref{denom}$(x)$: lowest denominator of $x$. The meaning of
this is clear when $x$ is a rational number or function.
When $x$ is an integer or a polynomial, the result is equal to 1. When $x$ is 
a vector or a matrix, the lowest common denominator of the components of $x$
is computed. All other types are forbidden.

The library syntax is $\teb{denom}(x)$.

\subsecref{floor}$(x)$: floor of $x$. When $x$ is in $\Bbb R$,
the result is the largest integer smaller than or equal to $x$.

The library syntax is $\teb{gfloor}(x)$.

\subsecref{frac}$(x)$: fractional part of $x$. Identical to
$x-\text{floor}(x)$. If $x$ is real, the result is in $[0,1[$.

The library syntax is $\teb{gfrac}(x)$.

\subsecref{getrand}$()$: returns the current value of the random number 
seed. Useful mainly for debugging purposes.

The library syntax is $\teb{getrand}()$. The value returned is a GEN.

\subsecref{getstack}$()$: returns the current value of {\tt top$-$avma},
i.e. the number of bytes used up to now on the stack. Useful mainly for
debugging purposes.

The library syntax is $\teb{getstack}()$. The value returned is a GEN.

\subsecref{gettime}$()$: returns the time (in milliseconds) elapsed since 
either the last call to {\tt gettime}, or to the beginning of the containing 
GP instruction (if inside GP), whichever came last.

The library syntax is $\teb{gettime}()$. The value returned is a GEN.

\subsecref{imag}$(x)$: imaginary part of $x$. When
$x$ is a quadratic number, this is the coefficient of $\omega$ in
the ``canonical'' integral basis $(1, \omega)$.

The library syntax is $\teb{gimag}(x)$.

\subsecref{length}$(x)$: number of non-code words in $x$ really used
(i.e. the effective length for integers and polynomials). In particular,
the degree of a polynomial is equal to its length minus 1.

The library syntax is $\teb{glength}(x)$.

\subsecref{lift}$(x)$: lifts an element $x=a \bmod n$ of $\Bbb Z/n\Bbb Z$ to
$a$ in $\Bbb Z$, and similarly lifts a polymod to a polynomial. If $x$ is
of type fraction, complex, quadratic, polynomial, power series, rational
function, vector or matrix, the lift is done for each coefficient.
Forbidden types for $x$ are reals and $p$-adics. Note that the main variable
of the lift of a polymod will be variable number 0 (i.e. 'X') if the rules
concerning the creation of polymods explained in chapter 2 are observed.

The library syntax is $\teb{lift}(x)$.

\subsecref{mod}$(x,y)$: creates the PARI object $(x \mod y)$,
i.e. an integermod or a polymod. $y$ must be an integer or
a polynomial. If $y$ is an integer, $x$ must be an integer. If $y$ is a
polynomial, $x$ must be a scalar or a polynomial. The result is put on
the PARI stack.

This function is not the same as $x\% y$, the result of which is an
integer or a polynomial.

The library syntax is $\teb{gmodulcp}(x,y)$. 

\subsecref{modp}$(x,y)$: same effect as {\tt mod}, except that the created
result is put on the heap and not on the stack, and hence becomes a permanent
copy which cannot be erased later by garbage collecting (see 4.2.4). In 
particular, care should be taken to avoid creating too many such objects,
since the heap is very small (typically a few thousand objects at most).

The library syntax is $\teb{gmodulo}(x,y)$.

\subsecref{norm}$(x)$: algebraic norm of $x$, i.e. the product of $x$
with its conjugate (no square roots are taken), or conjugates for polymods.
For vectors and matrices, the norm is taken componentwise and hence is
not the $L^2$-norm (see {\tt norml2} below). Note that the norm of an element of
$\Bbb R$ is its square, so as to be compatible with the complex norm.

The library syntax is $\teb{gnorm}(x)$.

\subsecref{norml2}$(x)$: square of the $L^2$-norm of $x$. $x$ must
be a (row or column) vector.

The library syntax is $\teb{gnorml2}(x)$.

\subsecref{numer}$(x)$: numerator of $x$. When $x$ is a rational number
or function, the meaning is clear. When $x$ is an integer or a polynomial the
result is $x$ itself. When $x$ is a vector or a matrix, then {\tt numer(x)} is
defined to be {\tt denom(x)*x}. All other types are forbidden.

The library syntax is $\teb{numer}(x)$.

\subsecref{permutation}$(n,k)$: generates the $k$-th permutation (as a vector
of length $n$) of the numbers $1$ to $n$. The number $k$ is taken modulo $n!\,$.

The library syntax is $\teb{permute}(n,k)$, where $n$ is a C-integer.

\subsecref{poly}$(x,v)$: transform the object $x$ into a polynomial with main
variable $v$. If $x$ is a scalar, this gives a constant polynomial. If $x$ is
a power series, the effect is identical to {\tt trunc} (see below), i.e. it
chops off the $O(X^k)$. If $x$ is a vector, this function creates the polynomial
whose coefficients are given in $x$, with $x[1]$ being the leading coefficient
(which can be zero).

Warning: this is {\sl not} a substitution function. It is intended to be quick and
dirty. So if you try {\tt poly(a,y)} on the polynomial {\tt a=x+y}, you will get
{\tt y+y}, which is not a valid PARI/GP object.

The library syntax is $\teb{gtopoly}(x,v)$, where $v$ is a variable number.

\subsecref{polyrev}$(x,v)$: transform the object $x$ into a polynomial with main
variable $v$. If $x$ is a scalar, this gives a constant polynomial. If $x$ is
a power series, the effect is identical to {\tt trunc} (see below), i.e. it
chops off the $O(X^k)$. If $x$ is a vector, this function creates the polynomial
whose coefficients are given in $x$, with $x[1]$ being the constant term. Note that
this is the reverse of {\tt poly} if $x$ is a vector, otherwise it is identical to
{\tt poly}.

The library syntax is $\teb{gtopolyrev}(x,v)$, where $v$ is a variable number.

\subsecref{prec}$(x,n)$: $x$ being a PARI object and $n$ a 32-bit
C-integer, creates a new object equal to $x$ but with a new precision $n$.
This is to be understood as follows:

For exact types, no change. For $x$ a vector or a matrix, the operation
is done componentwise.

For real $x$, $n$ is the number of desired significant {\sl decimal} digits.
If $n$ is smaller than the precision of $x$, $x$ is truncated, otherwise
$x$ is extended with zeros.

For $x$ a $p$-adic or a power series, $n$ is the desired number of
significant $p$-adic or $X$-adic digits, where $X$ is the main variable of $x$.

Note that the function {\tt prec} never changes the type of the result. In
particular
it is not possible to use {\tt prec} to obtain a polynomial from a power series.
For that, see {\tt trunc} (see below).

The library syntax is $\teb{gprec}(x,n)$, where $n$ is a C-integer.

\subsecref{quadgen}$(x)$: creates the quadratic number \sref{omega}
$\omega=(a+\sqrt{x})/2$ where $a=0$ if $x\equiv0\mod4$, 
$a=1$ if $x\equiv1\mod4$, so that $(1,\omega)$ is an integral basis for
the quadratic order of discriminant $x$. $x$ must be an integer congruent to 
0 or 1 modulo 4.

The library syntax is $\teb{quadgen}(x)$.

\subsecref{quadpoly}$(x)$: creates the ``canonical'' quadratic
polynomial corresponding to the discriminant $x$, i.e. the minimal polynomial
of {\tt quadgen(x)}. $x$ must be an integer congruent to 0 or 1 modulo 4.

The library syntax is $\teb{quaddisc}(x)$.

\subsecref{random}$()$: gives a random integer between 0 and $2^{31}-1$.
This is simply the standard {\tt rand()} function of C, hence in particular
the low order bits are not very random.

The library syntax is $\teb{genrand}()$.

\subsecref{real}$(x)$: real part of $x$. In the case where
$x$ is a quadratic number, this is the coefficient of $1$ in
the ``canonical'' integral basis $(1, \omega)$.

The library syntax is $\teb{greal}(x)$.

\subsecref{rndtoi}$(x)$: same as {\tt cvtoi} (see above) except
that truncation is replaced by rounding (see {\tt round} below).

The library syntax is $\teb{grndtoi}(x,\& e)$.

\subsecref{round}$(x)$: If $x$ is in $\Bbb R$, rounds $x$
to the nearest integer. If the exponent of $x$ is too large compared
to its precision, the result is undefined and an error message occurs.
Use {\tt rndtoi} above to suppress this error handling.

Important remark: note that, contrary to the other truncation functions
(except {\tt rndtoi} which is essentially the same), this function operates
on every coefficient at every level of a PARI object. For example
$$\text{trunc}\left(\dfrac{2.4*X^2-1.7}{1.2*X}\right)=2.0*X,$$ while
$$\text{round}\left(\dfrac{2.4*X^2-1.7}{1.2*X}\right)=\dfrac{2*X^2-2}{X}.$$
An important use of {\tt round} is to get exact results after a long
approximate computation, when theory tells you that the coefficients
must be integers.

The library syntax is $\teb{ground}(x)$.

\subsecref{rounderror}$(x)$: maximum error in decimal digits observed when
rounding $x$. This functions allows the user to easily check the precision of
the rounding of a result whose components are supposed to be integers.

The library syntax is $\teb{rounderror}(x)$ and the result is a C-integer.

\subsecref{series}$(x,v)$: transform the object $x$ into a power series with
main variable $v$. If $x$ is a scalar, this gives a constant power series
with precision given by the global variable {\tt precdl} (transparent under GP).
If $x$ is a polynomial, the precision is the greatest of {\tt precdl} and the
degree of the polynomial. If $x$ is a vector, the precision is similarly given,
and the coefficients of the vector are understood to be the coefficients of
the power series starting from the constant term (i.e. the reverse of the
function {\tt poly}, see above).

The warning given for {\tt poly} applies here: this is not a substitution
function.

The library syntax is $\teb{gtoser}(x,v)$, where $v$ is a variable number.

\subsecref{setprecision}$(n)$: sets the current default precision equal to $n$
decimal digits if $n>0$. In any case, it returns the value of the
default precision before the change. Apart from the fact that it returns a value, this is
identical to the $\backslash$precision=$n$ command, with the difference that
it can be used inside any GP expression or program.

The library syntax is $\teb{setprecr}(n)$, where $n$ is a C-integer, as
is the returned value.

\subsecref{setrand}$(n)$: reseeds the random number generator to the value
$n$. The initial seed (in Unix systems at least) is $n=1$.

The library syntax is $\teb{setrand}(n)$ where $n$ is a C-integer. 
The value returned is $n$ once again, but as a GEN.

\subsecref{setserieslength}$(n)$: sets the current default series length equal
to $n$ if $n>0$. In any case, it returns the value of the default
length before the change. Apart from the fact that it returns a value, this is
identical to the $\backslash$serieslength=$n$ command, with the difference that
it can be used inside any GP expression or program.

The library syntax is $\teb{setserieslength}(n)$, where $n$ is a C-integer, as
is the returned value.

\subsecref{settype}$(x,t)$: makes a copy of $x$ and sets its type equal to
type number $t$. This function must be used with extreme caution, otherwise
disasters may occur, but one instance where it is extremely useful is
{\tt settyp(x,14)} when $x$ is a rational function (type 13). In this case,
the created object, as well as the objects created from it, will not be
reduced automatically, making the operations much faster. In fact this function
is the {\it only} way to create reducible rationals (type 5) or rational
functions (type 14) in GP.

The library syntax is $\teb{gsettype}(x,t)$, where $t$ is a long, but need
not be used since the internal function {\tt settyp} is available. Note that
{\tt settyp} does {\it not} create a copy of $x$, while {\tt gsettype} does.

\subsecref{simplify}$(x)$: this function tries to simplify the object $x$ as
much as it can. The simplifications do not concern rational functions (which
Pari automatically tries to simplify), but type changes. Specifically, a 
complex or quadratic number whose imaginary part is exactly equal to 0 (i.e. 
not a real zero) is converted to its real part, and a polynomial of degree zero
is converted to its constant term. For all types, this of course occurs
recursively. This function is useful in any case, but in particular before
the use of arithmetic functions which expect integer arguments, and not for
example a complex number of 0 imaginary part an integer real part (which is
however printed as an integer).

The library syntax is $\teb{simplify}(x)$.

\subsecref{size}$(x)$: gives the maximal number of decimal digits minus one of the
components of $x$.

The library syntax is $\teb{size}(x)$, and the result is a C-integer.

\subsecref{trunc}$(x)$: truncation of $x$. When $x$ is in $\Bbb R$,
this means that the part after the decimal point is chopped away. If the
result is not significant because the exponent is too large compared
to the precision, an error occurs. Use {\tt cvtoi} (see above) to suppress this error
handling.

Note a very special use of {\tt trunc}: when applied to a power series, it
transforms it into a polynomial or a rational function with denominator
a power of $X$, by chopping away the $O(X^k)$. Similarly, when applied to
a $p$-adic number, it transforms it into an integer or a rational number
by chopping away the $O(p^k)$.

The library syntax is $\teb{gtrunc}(x)$.

\subsecref{type}$(x)$: internal type number (from 1 to 19) of the pari object
$x$. This is useful only under GP. See also the function {\tt settype}.

The library syntax is $\teb{gtype}(x)$, but need not be used since the
internal function {\tt typ} is available. Note that {\tt gtype} gives a
GEN while {\tt typ} gives a C-integer.

\subsecref{valuation}$(x,p)$: Computes the highest exponent of $p$
dividing $x$. If $p$ is of type integer, $x$ must be an integer,
an integermod whose modulus is divisible by $p$, a fraction, a
$q$-adic number with $q=p$, or a polynomial or power series 
in which case the valuation is the minimum of the valuation of the coefficients.

If $p$ is of type polynomial, $x$ must
be of type polynomial or rational function, and also a power series
if $x$ is a monomial. If $x=0$, an error is issued except in case of
$p$-adic numbers and power series, in which case the result is the
exponent of the zero. Finally, the valuation of a vector,
complex or quadratic number is the minimum of the component
valuations. Any other type combinations gives an error.

The library syntax is $\teb{ggval}(x)$.

\subsecref{vec}$(x)$: transform the object $x$ into a vector. The vector will
be with one component only, except when $x$ is a vector/matrix or a quadratic
form (in which case the resulting vector is simply the initial object considered
as a row vector), but more importantly when $x$ is a polynomial or a power
series. In the case of a polynomial, the coefficients of the vector starts
with the leading coefficient of the polynomial, while for power series only
the significant coefficients are taken into account, but this time by
increasing order of degree.

The library syntax is $\teb{gtovec}(x)$.

\subsecref{vecmax}$(x)$: if $x$ is a vector or a matrix, returns the maximum
of the elements of $x$, otherwise returns a copy of $x$. Returns zero if $x$
is empty (as it should).

The library syntax is $\teb{vecmax}(x)$.

\subsecref{vecmin}$(x)$: if $x$ is a vector or a matrix, returns the minimum
of the elements of $x$, otherwise returns a copy of $x$. Gives an error if
$x$ is empty.

The library syntax is $\teb{vecmin}(x)$.

\section{Transcendental functions.}

As a general rule, which of course in some cases may have exceptions,
transcendental functions operate in the following way:

\quad If the argument is either an integer, a real, a rational, a complex
or a quadratic number, it is, if necessary, first converted to a real
(or complex) number using the current \ref{precision} held in the variable
{\bf prec}. Under GP this is transparent to the user, but when programming
in library mode, care must be taken to supply the parameter {\tt prec} as 
the last argument of the function if the first argument is an exact object
(see 1.2.5.), otherwise disaster will occur.

Then, the function is computed. Note that even if the argument is real, the
result may be complex (e.g. $\text{acos}(2.0)$ or $\text{acosh}(0.0
)$).
Note also that the principal branch is always chosen.

\quad If the argument is an integermod or a $p$-adic, at present only a
few functions like {\tt sqrt} (square root), {\tt sqr} (square), {\tt log},
{\tt exp}, powering, {\tt teich} (Teichm\"uller character) and {\tt agm}
(arithmetic-geometric mean) are implemented.
Note that in the case of a $2$-adic number,
${\tt sqr}(x)$ is not identical to $x*x$: for example if
$x = 1+O(2^5)$ then $x*x = 1+O(2^5)$ while ${\tt sqr}(x) = 1+O(2^6)$.
(Remark: note that if we wanted to be strictly consistent with the
PARI philosophy, we should have $x*y= (4 \mod 8)$ when both $x$ and $y$ are
congruent to $2$ modulo $4$, or ${\tt sqr}(x)=(4\mod 32)$ when $x$ is
congruent to $2$ modulo $4$. However, since an integermod is an exact object,
PARI assumes that the modulus must not change, and the result is hence
$0 \mod 4$ in both cases. On the other hand, $p$-adics are not exact objects,
hence are treated differently.)

\quad If the argument is a polynomial, power series or rational function,
it is, if necessary, first converted to a power series using the current
precision held in the variable {\bf \ref{precdl}}. Under GP this again is
transparent to the user. When programming in library mode, however, the
global variable {\tt precdl} must be set before calling the function
if the argument has an exact type (i.e. not a power series). Here {\tt precdl} is
not an argument of the function, but a global variable.

Then the Taylor series expansion of the function around $X=0$ (where $X$ is
the main variable) is computed
to a number of terms depending on the number of terms of the argument
and the function being computed.

\quad If the argument is a vector or a matrix, the result is componentwise
evaluation of the function. In particular, transcendental functions on
square matrices, which are not implemented in the present version \vers{}
(see Appendix C however),
will have a slightly different name if they are implemented some day.

\subsec{$\hat{}$} $x\hat{\ }y$: if $y$ is not of type integer, this has the same
effect as $\exp(y*\ln(x))$. It can be applied to $p$-adic numbers as
well as to the more usual types.\sref{powering}

The library syntax is $\teb{gpui}(x,y,prec)$.

\subsecref{abs}$(x)$: absolute value of $x$ (modulus if $x$ is complex).
Polynomials, power series and rational functions are not allowed.

The library syntax is $\teb{gabs}(x,prec)$ where the second argument
$prec$ is necessary only in the complex or imaginary quadratic case.

\subsecref{acos}$(x)$: principal branch of $\text{cos}^{-1}(x)$,
i.e. such that $\text{Re(acos}(x))\in [0,\pi]$. If
$x\in \Bbb R$ and $|x|>1$ then $\text{acos}(x)$ is complex.

The library syntax is $\teb{gacos}(x,prec)$.

\subsecref{acosh}$(x)$: principal branch of $\text{cosh}^{-1}(x)$,
i.e. such that $\text{Im(acosh}(x))\in [0,\pi]$. If
$x\in \Bbb R$ and $x<1$ then $\text{acosh}(x)$ is complex.

The library syntax is $\teb{gach}(x,prec)$.

\subsecref{agm}$(x,y)$: arithmetic-geometric mean of $x$ and $y$. In the
case of complex or negative numbers, the principal square root is always
chosen. $p$-adic or power series arguments are also allowed. Note that
a $p$-adic agm exists only if $x/y$ is congruent to 1 modulo $p$ (modulo
16 for $p=2$). $x$ and $y$ cannot both be vectors or matrices.

The library syntax is $\teb{agm}(x,y,prec)$.

\subsecref{arg}$(x)$: argument of the complex number $x$, such that
$-\pi<\text{arg}(x)\le\pi$.

The library syntax is $\teb{garg}(x,prec)$.

\subsecref{asin}$(x)$: principal branch of $\text{sin}^{-1}(x)$,
i.e. such that $\text{Re(asin}(x))\in [-\pi/2,\pi/2]$. If
$x\in \Bbb R$ and $|x|>1$ then $\text{asin}(x)$ is complex.

The library syntax is $\teb{gasin}(x,prec)$.

\subsecref{asinh}$(x)$: principal branch of $\text{sinh}^{-1}(x)$,
i.e. such that $\text{Im(asinh)}(x)\in [-\pi/2,\pi/2]$.

The library syntax is $\teb{gash}(x,prec)$.

\subsecref{atan}$(x)$: principal branch of $\text{tan}^{-1}(x)$,
i.e. such that $\text{Re(atan}(x))\in ]-\pi/2,\pi/2[$.

The library syntax is $\teb{gatan}(x,prec)$.

\subsecref{atanh}$(x)$: principal branch of $\text{tanh}^{-1}(x)$.
i.e. such that $\text{Im(atanh}(x))\in ]-\pi/2,\pi/2]$. If
$x\in \Bbb R$ and $|x|>1$ then $\text{atanh}(x)$ is complex.

The library syntax is $\teb{gath}(x,prec)$.

\subsecref{bernreal}$(x)$: Bernoulli number $B_x$, where $B_0=1$, $B_1=-1/2$,
$B_2=1/6$,\dots, expressed as a real number with the current precision.

The library syntax is $\teb{bernreal}(x,prec)$.

\subsecref{bernvec}$(x)$: creates a vector containing, as rational numbers, the
Bernoulli numbers $B_0$, $B_2$,\dots, $B_{2x}$. These Bernoulli numbers can
then be used as follows. Assume that this vector has been put into a variable,
say b. Then we define under GP:

{\tt bern(x)=if(x==1,-1/2,if((x<0)||(x\%2),0,b[x/2+1]))}

and then {\tt bern(k)} gives the Bernoulli number of index $k$ as a rational
number, exactly as {\tt bernreal(k)} gives it as a real number.

The library syntax is $\teb{bernvec}(x)$.

\subsecref{cos}$(x)$: cosine of $x$.

The library syntax is $\teb{gcos}(x,prec)$.

\subsecref{cosh}$(x)$: hyperbolic cosine of $x$.

The library syntax is $\teb{gch}(x,prec)$.

\subsecref{dilog}$(x)$: principal branch of the dilogarithm of $x$, i.e.
analytic continuation of the power series $\log_2(x)=\sum_{n\ge1}x^n/n^2$.

The library syntax is $\teb{dilog}(x,prec)$.

\subsecref{eint1}$(x)$: exponential integral 
$\int_x^\infty \dfrac{e^{-t}}{t}\,dt$.

The library syntax is $\teb{eint1}(x,prec)$.

\subsecref{erfc}$(x)$: complementary error function
$(2/\sqrt\pi)\int_x^\infty e^{-t^2}\,dt$.

The library syntax is $\teb{erfc}(x,prec)$.

\subsecref{eta}$(x)$: Dedekind's $\eta$ function, without the
$q^{1/24}$. The meaning is this. If $x$ is a complex number with
positive imaginary part, the result is $\prod_{n=1}^\infty(1-q^n)$,
where $q=e^{2i\pi x}$. If $x$ is a power series (or can be converted
to a power series) with positive valuation, the result is $\prod_{n=1}^\infty(1-x^n)$.

The library syntax is $\teb{eta}(x,prec)$.

\subsecref{euler}: Euler's constant $0.57721\cdots$.

Note that {\tt euler} is one of the few special reserved names which cannot
be used for variables.

The library syntax is $\teb{mpeuler}(prec)$ where $prec$ {\sl must} be
given.
Note that this creates $\gamma$ on the PARI stack. If one does not want to
create it on that stack but be able to use it later under the global
name {\bf \ref{geuler}} (with no parentheses) use instead $\teb{consteuler}(prec
)$.

\subsecref{exp}$(x)$: exponential of $x$.
$p$-adic arguments with positive valuation are accepted.

The library syntax is $\teb{gexp}(x,prec)$.

\subsecref{gamh}$(x)$: gamma function evaluated at the argument
$x+1/2$. When $x$ is an integer, this is much faster than using
$\text{gamma}(x + 1/2)$.

The library syntax is $\teb{ggamd}(x,prec)$.

\subsecref{gamma}$(x)$: gamma function of $x$. In the present version \vers{}
the $p$-adic gamma function is not yet implemented.

The library syntax is $\teb{ggamma}(x,prec)$.

\subsecref{hyperu}$(a,b,x)$: U-confluent hypergeometric function with
parameters $a$ and $b$.

The library syntax is $\teb{hyperu}(a,b,x,prec)$.

\subsecref{incgam}$(x,y)$: incomplete gamma function.

The arguments $x$ and $y$ must be positive. The result returned is
$\int_y^\infty e^{-t}t^{x-1}\,dt$.

The library syntax is $\teb{incgam}(x,y,prec)$.

Note: in addition, there exist also the functions
{\tt \ref{incgam1} and \ref{incgam2}} which are used for internal purposes.

\subsecref{incgam3}$(x,y)$: complementary incomplete gamma function.

The arguments $x$ and $y$ must be positive. The result returned is
$\int_0^y e^{-t}t^{x-1}\,dt$. when $y$ is not too large.

The library syntax is $\teb{incgam3}(x,y,prec)$.

\subsecref{incgam4}$(x,y,z)$: incomplete gamma function with given gamma value.
This function assumes (of course without checking) that $z=\Gamma(x)$.
The arguments $x$ and $y$ must be positive. The result returned is
$\int_y^\infty e^{-t}t^{x-1}\,dt$, as in the ordinary {\tt incgam} function,
but the computation will be much faster when $y$ is small.

The library syntax is $\teb{incgam4}(x,y, z,prec)$.

\subsecref{jbesselh}$(n,x)$: $J$-Bessel function of half integral index.
More precisely, {\tt jbesselh(n,x)} computes $J_{n+1/2}(x)$ where $n$
must be of type integer, and $x$ is any element of $\Bbb C$. In the
present version \vers, this function is not very accurate when $x$ is
small.

The library syntax is $\teb{jbesselh}(n,x,prec)$.

\subsecref{jell}$(x)$: elliptic $j$-invariant. $x$ must be a complex
number with positive imaginary part, or convertible into a power series or
a $p$-adic number with positive valuation.

The library syntax is $\teb{jell}(x,prec)$.

\subsecref{kbessel}$(nu,x)$: K-Bessel function of index $nu$
(which can be complex) and argument $x$. Only real and positive arguments
$x$ are allowed in the present version \vers.

The library syntax is $\teb{kbessel}(nu, x,prec)$.

Note: in addition, another implementation of this function which is often
faster than kbessel is the function {\bf \ref{kbessel2}}.

\subsecref{ln}$(x)$ or {\bf log}$(x)$:\sref{log} principal branch of the
natural logarithm of $x$, i.e. such that $\text{Im(ln}(x))\in ]-\pi,\pi]$.
The result is complex (with imaginary part equal to $\pi$) if 
$x\in \Bbb R$ and $x<0$.

$p$-adic arguments are also accepted for $x$, with the convention that
$\ln(p)=0$. Hence in particular $\exp(\ln(x))/x$ will not in general be
equal to 1 but to a $p-1$-th root of unity (or $\pm1$ if $p=2$)
times a power of $p$.

The library syntax is $\teb{glog}(x,prec)$.

\subsecref{lngamma}$(x)$: principal branch of the logarithm of the gamma
function of $x$. Can have much larger arguments than {\tt gamma} itself.
In the present version \vers{} the $p$-adic lngamma function is not yet 
implemented.

The library syntax is $\teb{glngamma}(x,prec)$.

\subsecref{logagm}$(x)$: principal branch of the natural logarithm of $x$,
computed using an agm formula suggested by Mestre, when $x$ is real, otherwise
identical to {\tt log}.

The library syntax is $\teb{glogagm}(x,prec)$.

\subsecref{pi}: the constant pi ($3.14159\cdots$).

The library syntax is $\teb{mppi}(prec)$ where $prec$ {\sl must} be given.
Note that this creates $\pi$ on the PARI stack. If one does not want to
create it on that stack but be able to use it later under the global
name {\bf \ref{gpi}} (with no parentheses) use instead $\teb{constpi}(prec)$.

\subsecref{polylog}$(m,x)$: $m^\text{th}$ polylogarithm of $x$, i.e. analytic
continuation of the power series $\log_m(x)=\sum_{n\ge1}x^n/n^m$. The program
uses the power series when $|x|^2\le1/2$, and the power series expansion in
$log(x)$ otherwise. It is valid in a large domain (at least $|x|<230$), but should
not be used too far away from the unit circle since it is then better to use
the functional equation linking the value at $x$ to the value at $1/x$, which
takes a trivial form for the {\tt polylogd} and {\tt polylogp} functions
below. Power series, polynomial, rational and vector/matrix arguments are
allowed.

The library syntax is $\teb{gpolylog}(m,x,prec)$.

\subsecref{polylogd}$(m,x)$: modified $m^\text{th}$ polylogarithm of $x$,
called $\tilde D_m(x)$ in Zagier, defined for $|x|\le1$ by
$$\Re_m\left(\sum_{k=0}^{m-1}\dfrac{(-\log|x|)^k}{k!}\log_{m-k}(x)
+\dfrac{(-\log|x|)^{m-1}}{m!}\log|1-x|\right)$$
and such that $D_m(1/x)=(-1)^{m-1}D_m(x)$, where $\Re_m$ denotes $\Re$
or $\Im$ depending whether $m$ is odd or even.

The library syntax is $\teb{polylogd}(m,x,prec)$.

\subsecref{polylogdold}$(m,x)$: modified $m^\text{th}$ polylogarithm of $x$,
called $D_m(x)$ in Zagier, defined for $|x|\le1$ by
$$\Re_m\left(\sum_{k=0}^{m-1}\dfrac{(-\log|x|)^k}{k!}\log_{m-k}(x)
-\dfrac{1}{2}\dfrac{(-\log|x|)^m}{m!}\right)$$
and such that $D_m(1/x)=(-1)^{m-1}D_m(x)$, where $\Re_m$ denotes $\Re$
or $\Im$ depending whether $m$ is odd or even.

The library syntax is $\teb{polylogdold}(m,x,prec)$.

\subsecref{polylogp}$(m,x)$: another modified $m^\text{th}$ polylogarithm
of $x$, called $P_m(x)$ in Zagier, defined for $|x|\le1$ by
$$\Re_m\left(\sum_{k=0}^{m-1}\dfrac{2^kB_k}{k!}(\log|x|)^k\log_{m-k}(x)
-\dfrac{2^{m-1}B_m}{m!}(\log|x|)^m\right)$$
and such that $P_m(1/x)=(-1)^{m-1}P_m(x)$, where $\Re_m$ denotes $\Re$
or $\Im$ depending whether $m$ is odd or even.

The library syntax is $\teb{polylogp}(m,x,prec)$.

\subsecref{psi}$(x)$: the $\psi$-function of $x$, i.e. the
logarithmic derivative $\Gamma'(x)/\Gamma(x)$.

The library syntax is $\teb{gpsi}(x,prec)$.

\subsecref{sin}$(x)$: sine of $x$.

The library syntax is $\teb{gsin}(x,prec)$.

\subsecref{sinh}$(x)$: hyperbolic sine of $x$.

The library syntax is $\teb{gsh}(x,prec)$.

\subsecref{sqr}$(x)$: square of $x$. Not identical to $x*x$ in
the case of $2$-adics (see above), and acts componentwise
on vectors and matrices. In particular, if $x$ is a square matrix, sqr($x$)
is not the square of $x$ (use instead $x*x$ or $x\hat{\ }2$).

The library syntax is $\teb{gsqr}(x)$.

\subsecref{sqrt}$(x)$: principal branch of the square root of $x$,
i.e. such that $\text{Arg(sqrt}(x))\in ]-\pi/2,\pi/2]$, or in other words
such that $\text{Re(sqrt}(x))>0$ or $\text{Re(sqrt}(x))=0$ and
$\text{Im(sqrt}(x))>=0$.
Integermod a prime and $p$-adics are allowed as arguments.
If $x\in \Bbb R$ and $x<0$ the result is complex with positive imaginary part.

The library syntax is $\teb{gsqrt}(x,prec)$.

\subsecref{tan}$(x)$: tangent of $x$.

The library syntax is $\teb{gtan}(x,prec)$.

\subsecref{tanh}$(x)$: hyperbolic tangent of $x$.

The library syntax is $\teb{gth}(x,prec)$.

\subsecref{teich}$(x)$: Teichmuller character of the $p$-adic number
$x$.

The library syntax is $\teb{teich}(x)$.

\subsecref{wf}$(x)$: Weber's $f$ function, i.e. such that
$j=(f^{24}-16)^3/f^{24}$.

The library syntax is $\teb{wf}(x,prec)$.

\subsecref{wf2}$(x)$: Weber's $f_2$ function, i.e. such that
$j=(f_2^{24}+16)^3/f_2^{24}$.

The library syntax is $\teb{wf2}(x,prec)$.

\subsecref{zeta}$(s)$: Riemann's zeta function $\zeta(s)=\sum_{n\ge1}n^{-s}$,
computed using the Euler-Maclaurin summation formula, except when $s$ is of type
integer, in which case it is computed using Bernoulli numbers for $s<=0$ or
$s>0$ and even, and using modular forms for $s>0$ and odd.

The library syntax is $\teb{gzeta}(s,prec)$.

\section{Arithmetic functions.}

These functions are by definition functions whose natural domain
of definition is either $\Bbb Z$ (or $\Bbb N$), or sometimes polynomials over
a base ring. Functions which concern polynomials exclusively will be
explained in the next section.
The way these functions are used is completely different from
transcendental functions: in general only the types integer and
polynomial are accepted as arguments.

In the present version \vers{} a primitive but useful version of the \ref{ECM}
method has been implemented. Also, numbers found to be pseudo-primes after
10 successful trials of the \ref{Rabin-Miller} test are declared primes.

\subsecref{bestappr}$(x,k)$: if $x\in\Bbb R$, finds the best rational
approximation to $x$ with denominator at most equal to $k$ using continued
fractions. If $x$ is a generic type, {\tt bestappr} is used on each coefficient
independently.

The library syntax is $\teb{bestappr}(x,k)$.

\subsecref{bezout}$(x,y)$: finds $u$ and $v$ minimal in a
natural sense such that $x*u+y*v=\text{gcd}(x,y)$. The arguments
must be both integers or both polynomials, and the result is a
row vector with three components $u$, $v$, and $\text{gcd}(x,y)$.

The library syntax is either $\teb{vecbezout}(x,y)$ to get the
vector, or $\teb{gbezout}(x,y, \& u, \& v)$ which gives as result the
address of the created gcd, and puts in $u$ and $v$ the addresses of the
corresponding created objects.

\subsecref{bigomega}$(x)$: number of prime
divisors of $x$ counted with multiplicity. $x$ must be an integer,
and the result is a 32-bit C-integer.

The library syntax is $\teb{bigomega}(x)$.

\subsecref{bin}$(x,y)$: \ref{binomial coefficient} $\binom x y$.
Here $y$ must be an integer, but $x$ can be any PARI object.

The library syntax is $\teb{binome}(x,y)$, where $y$ must be a C-integer.

\subsecref{boundcf}$(x,lmax)$: creates the row vector whose components are
the partial quotients of the \ref{continued fraction} expansion of $x$, the
number of partial quotients being limited to lmax.
If $x$ is a real number, the expansion stops at the last significant partial
quotient or after lmax terms, whichever is smallest. $x$ can also be a rational
function or a power series.

The library syntax is $\teb{gboundcf}(x,lmax)$, where $lmax$ is a C integer.

\subsecref{boundfact}$(x,p)$: For integer $x$, finds only the prime factors
up to $p$ or to the default bound of the prime table created upon 
initialization of GP, whichever is lowest (except when $p<=1$ where the
effect is identical to {\tt smallfact}. The remaining part is hence not
necessarily prime. 

The library syntax is $\teb{boundfact}(x,p)$.

\subsecref{buchimag}$(D,c,c2)$: McCurley's sub-exponential algorithm for
computing the class group of an imaginary quadratic field of discriminant $D$.
Should be used instead of {\tt classno} for $|D|>10^{25}$ or when the 
{\sl structure} is wanted. The numbers $c$ and $c2$ are positive real numbers
which control the execution time and the stack size. To get maximum speed,
set $c2=c$. To get a rigorous result (under GRH) you must take $c2=6$.
Reasonable values for $c$ are between $0.1$ and $2$. 

The result of this function is a vector with 4 components. The first is the
class number, the second is a vector giving the structure of the class group
as a product of cyclic groups, and the third is a vector giving as binary
quadratic forms generators of those cyclic groups. The 
last component is a measure of the correctness of the result. If it is close
to 1, the result is correct (under GRH). If it is close to a larger integer,
this shows that the class number is off by a factor equal to this integer,
and you must start again with a larger value for $c$ or a different
random seed. In this case, a warning message is printed.

Note: some installations may have this function disabled.

The library syntax is $\teb{buchimag}(D,c,c2)$.

\subsecref{buchreal}$(D,n,c,c2)$: Buchmann's sub-exponential algorithm for
computing the class group and the regulator of a real quadratic field of
discriminant $D$. Should be used instead of {\tt classno} and {\tt regula} for
$D>10^{10}$ or when the {\sl structure} is wanted. If the integer $n$ is equal
to 0 the function computes the ordinary (or wide) class group and regulator,
while if $n$ is non-zero it computes the narrow class group and regulator
(this case is not yet implemented in version \vers).
The numbers $c$ and $c2$ are positive real numbers which control the execution
time and the stack size. To get maximum speed,
set $c2=c$. To get a rigorous result (under GRH) you must take $c2=6$.
Reasonable values for $c$ are between $0.1$ and $2$.

The result of this function is a vector with 5 components. The first is the 
class number, the second is a vector giving the structure of the class 
group as a product of cyclic groups, and the third is a vector giving as binary
quadratic forms generators of those cyclic groups. The fourth component is
the regulator, computed to an accuracy which is the maximum of an internal 
accuracy determined by the program and the current default (note that once the
regulator is known to a small accuracy it is 
trivial to compute it to very high accuracy. I leave this as an exercise to
the reader, or see the tutorial). The last component is a measure of
the correctness of the result. If it is close to 1, the results are correct 
(under GRH). If it is close to a larger integer, this shows that the product
of the class number by the regulator is off by a factor equal to this integer,
and you must start again with a larger value for $c$ or a different
random seed. In this case, a warning message is printed.

Note: some installations may have this function disabled.

The library syntax is $\teb{buchreal}(D,n,c,c2)$.

\subsecref{cf}$(x)$: creates the row vector whose components are
the partial quotients of the \ref{continued fraction} expansion of $x$.
If $x$ is a real number, the expansion stops at the last significant partial
quotient. $x$ can also be a rational function.

The library syntax is $\teb{gcf}(x)$.

\subsecref{cf2}$(b,x)$: creates the row vector whose components are
the partial quotients of the continued fraction expansion of $x$, the
numerators being equal to the coefficients of the vector $b$. The length
of the result is equal to the length of $b$ unless a partial remainder is
encountered which is equal to zero, in which case the expansion stops.
In the case of real numbers, the stopping criterion in thus different
from that of {\tt cf} since if $b$ is too long, some partial quotients may
not be significant. $x$ can also be a rational function or a power series.

The library syntax is $\teb{gcf2}(b,x)$.

\subsecref{chinese}$(x,y)$: $x$ and $y$ being both integermods or
polymods, creates (with the same type) a $z$ in the same residue class as $x$
and in the same residue class as $y$, if it is possible.

The library syntax is $\teb{chinois}(x,y)$.

\subsecref{classno}$(x)$: class number of the quadratic field of discriminant
$x$. In the present version \vers, a simple algorithm is used for $x>0$, so
$x$ should not be too large (say $x<10^7$) for the time to be reasonable. On
the other hand, for $x<0$ one can reasonably compute classno($x$) for 
$|x|<10^{25}$, since the method used is Shanks' method which is in
$O(|x|^{1/4})$. For larger values of $|D|$, see {\tt buchimag} for $D<0$ and
{\tt buchreal} for $D>0$.

The library syntax is $\teb{classno}(x)$.

There also exists the function {\tt \ref{classno2}}, which computes the class
number using the functional equation. However, it is in $O(|x|^{1/2})$. 
Finally, in library mode only there exists the function {\tt \ref{classno3}} 
which computes the class number of an imaginary quadratic field by counting 
reduced forms, a $O(|x|)$ algorithm. See also {\tt hclassno} below.

\subsecref{compimag}$(x,y)$ composition of the binary quadratic forms
$x$ and $y$. This is of course a special case of {\tt gmul(x,y)}. See also
{\tt nucomp} and {\tt nudupl}.

The library syntax is $\teb{compimag}(x,y)$.

\subsecref{comprealraw}$(x,y)$ composition of the binary quadratic
forms $x$ and $y$ with positive discriminant, without reduction of the
result. This is useful e.g. to compute a generating element of an
ideal.

The library syntax is $\teb{comprealraw}(x,y)$.

\subsecref{content}$(x)$: computes the gcd of all the coefficients of $x$,
when this gcd makes sense. If $x$ is a scalar, this simply gives $x$. If
$x$ is a polynomial (and by extension a power series), it gives the usual
content of $x$. If $x$ is a rational function, it gives the ratio of the
contents of the numerator and the denominator. Finally, if $x$ is a
vector or a matrix, it gives the gcd of all the entries.

The library syntax is $\teb{content}(x)$.

\subsecref{divisors}$(x)$: creates a row vector whose components are the
positive divisors of the integer $x$ in increasing order.

The library syntax is $\teb{divisors}(x)$.

\subsecref{fact}$(x)$ or $x!$: factorial of $x$. The expression $x!$
gives a result which is an integer, while $\text{fact}(x)$ gives a real
number.

The library syntax is $\teb{mpfact}(x)$ for $x!$ and $\teb{mpfactr}(x,prec)$
for $\text{fact}(x)$. $x$ must be a 32-bit C-integer and not a PARI integer.

\subsecref{factfq}$(x,p,a)$: factorization in the field $\Bbb F_q$ defined by
the irreducible polynomial $a$ over $\Bbb F_p$ of the polynomial
$x$. The coefficients of $x$ must be operation-compatible with $\Bbb Z/p\Bbb
Z$. The result is a two column matrix, the first column being the irreducible
polynomials dividing $x$, and the second the exponents. It is advised to use
for the variable of $a$ (which will be used as variable of a polymod), a
name distinct from the other variables used, so that a {\tt lift()} of the
result will be legible.

The library syntax is $\teb{factmod9}(x,p,a)$.

\subsecref{factmod}$(x,p)$: factorization mod $p$ of the polynomial
$x$. The coefficients of $x$ must be operation-compatible with $\Bbb Z/p\Bbb
Z$. The result is a two column matrix, the first column being the irreducible
polynomials dividing $x$, and the second the exponents. If you want only the
{\sl degrees} of the irreducible polynomials (for example for computing an
$L$-function), use the function {\tt simplefactmod}.

The library syntax is $\teb{factmod}(x,p)$.

\subsecref{factor}$(x)$: general factorization function. If $x$ is
of type integer, rational, polynomial or rational function,
the result is a two column matrix, the first column being the irreducibles
dividing $x$ (prime numbers or polynomials), and the second the exponents.
If $x$ is a vector or a matrix, the factoring is done componentwise (hence
the result is a vector or matrix of two column matrices).

The polynomials or rational functions to be factored must have coefficients
either integers or integermods. Note that PARI does {\sl not}
yet know how to factor multivariate polynomials.

The library syntax is $\teb{factor}(x)$. See also {\tt \ref{factpol}} 
{\tt \ref{factpol2}} and {\tt \ref{factornf}}.

\subsecref{fibo}$(x)$: $x^{\text{th}}$ Fibonacci number.

The library syntax is $\teb{fibo}(x)$. $x$ must be a
32-bit C-integer and not a PARI integer.

\subsecref{gcd}$(x,y)$: creates the greatest common divisor of $x$
and $y$. $x$ and $y$ can be of quite general types; for instance both
rational numbers. Vector/matrix types are also accepted. Note that for these
types, {\tt gcd} is not commutative.

The library syntax is $\teb{ggcd}(x,y)$.

\subsecref{hclassno}$(x)$: \ref{Hurwitz class number} of $x$, where $x$ is
nonnegative and congruent to 0 or 3 modulo 4. See also {\tt classno} above.

The library syntax is $\teb{hclassno}(x)$.

\subsecref{hilb}$(x,y,p)$ or {\bf hilbp}$(x,y)$: \ref{Hilbert symbol} of
$x$ and $y$. If $x$ and $y$ are of type integer or fraction, the function
{\tt hilb} must be used with an explicit third parameter $p$, $p=0$ meaning the place
at infinity. Otherwise, $p$ need not be given, and $x$ and $y$ can be of 
compatible types integer, fraction, real, integermod or $p$-adic.

The library syntax is in both cases $\teb{hil}(x,y,p)$ or
$\teb{hil}(x,y)$.

\subsecref{isfund}$(x)$: true (1) if $x$ is equal to 1 or to the
discriminant of a quadratic field, false (0) otherwise.

The library syntax is $\teb{isfundamental}(x)$.

\subsecref{isprime}$(x)$: true (1) if $x$ is a strong pseudo-prime
for 10 randomly chosen bases, false (0) otherwise.

The library syntax is $\teb{isprime}(x)$.

\subsecref{ispsp}$(x)$: true (1) if $x$ is a strong 
pseudo-prime for a randomly chosen base, false (0) otherwise.

The library syntax is $\teb{ispsp}(x)$.

\subsecref{isqrt}$(x)$: integer square root of $x$, which must be
of PARI type integer. A negative $x$ is allowed, and the result in that case
is {\tt i*isqrt(-x)}.

The library syntax is $\teb{racine}(x)$.

\subsecref{issqfree}$(x)$: true (1) if $x$ is squarefree, false if not.
Here $x$ can be an integer or a polynomial.

The library syntax is $\teb{issquarefree}(x)$.

\subsecref{issquare}$(x)$: true (1) if $x$ is square, false if not. $x$ can
be of any type (for vector/matrix types the function is applied to the 
components).

The library syntax is $\teb{gcarreparfait}(x)$.

\subsecref{kro}$(x,y)$: Kronecker \sref{Legendre symbol}
(i.e. generalized Legendre) symbol $\left(\dfrac{x}{y}\right)$. $x$ and $y$
must be of type integer and the result ($0$ or $\pm 1$) is a 32-bit C-integer.

The library syntax is $\teb{kronecker}(x,y)$.

\subsecref{lcm}$(x,y)$: least common multiple of $x$ and $y$, i.e.
such that $\text{lcm}(x,y)*\text{gcd}(x,y)=x*y$.

The library syntax is $\teb{glcm}(x,y)$.

\subsecref{mu}$(x)$: M\"obius $\mu$-function of $x$. $x$ must be of
type integer and the result ($0$ or $\pm 1$) is a 32-bit C-integer.

The library syntax is $\teb{mu}(x)$.

\subsecref{nextprime}$(x)$: finds the smallest prime greater than or
equal to $x$. $x$ must be of type integer.

The library syntax is $\teb{bigprem}(x)$.

\subsecref{nucomp}$(x,y,l)$ composition of the primitive positive definite
binary quadratic forms $x$ and $y$ using the NUCOMP and NUDUPL algorithms of
Shanks (a la Atkin). $l$ is any positive constant, but for optimal speed, one
should take $l=|D|^{1/4}$ where $D$ is the common discriminant of $x$ and $y$.

The library syntax is $\teb{nucomp}(x,y,l)$. The auxiliary function
$\teb{nudupl}(x,l)$ also exists when $x=y$.

\subsecref{numdiv}$(x)$: number of divisors of $x$. $x$ must be of type
integer, and the result is a 32-bit C-integer.

The library syntax is $\teb{numbdiv}(x)$.

\subsecref{nupow}$(x,n)$: $n$-th power of the primitive positive definite
binary quadratic form $x$ using the NUCOMP and NUDUPL algorithms (see
{\tt nucomp} above).

The library syntax is $\teb{nupow}(x,n)$.

\subsecref{omega}$(x)$: number of distinct prime
divisors of $x$. $x$ must be of type integer, and the result is a 32-bit
C-integer.

The library syntax is $\teb{omega}(x)$.

\subsecref{order}$(x)$: $x$ must be an integer mod $n$, and the result
is the order of $x$ in the multiplicative group $(\Bbb Z/n\Bbb Z)^*$. Error if
$x$ is not invertible.

The library syntax is $\teb{order}(x)$.

\subsecref{pf}$(x,p)$: prime binary quadratic form of discriminant $x$
whose first coefficient is the prime number $p$. Error if $x$ is not a
quadratic residue mod $p$. In the case where $x>0$, the ``distance''
component of the form is set equal to zero to the current precision.

The library syntax is $\teb{primeform}(x,p,prec)$, where the third
variable $prec$ is a C-integer, but is necessary only when $x>0$.

\subsecref{phi}$(x)$: Euler's $\phi$-function of $x$. $x$ must be of
type integer.

The library syntax is $\teb{phi}(x)$.

\subsecref{pnqn}$(x)$: When $x$ is a vector or a one row matrix, $x$ is 
considered as the list of partial quotients $[a_0,a_1,\dots,a_n]$ of a rational
number, and the result is the 2 by 2 matrix $[p_n,p_{n-1};q_n,q_{n-1}]$ in the
standard notation of continued fractions, so
$p_n/q_n=a_0+1/(a_1+\dots+1/a_n)\dots)$. If $x$ is a matrix with two rows
$[b_0,b_1,\dots,b_n]$ and $[a_0,a_1,\dots,a_n]$,
this is then considered as a generalized continued fraction and we have
similarly $p_n/q_n=1/b_0(a_0+b_1/(a_1+\dots+b_n/a_n)\dots)$. Note that in this
case one usually has $b_0=1$.

The library syntax is $\teb{pnqn}(x)$.

\subsecref{powrealraw}$(x,n)$ $n$-th power of the binary quadratic
form $x$ of positive discriminant, computed without doing any
reduction (i.e. using {\tt comprealraw} above). Here $n$ must be non-negative
and $n<2^{31}$.

The library syntax is $\teb{powrealraw}(x,n)$ where $n$ must be a C-integer.

\subsecref{prime}$(x)$: the $x^{\text{th}}$ prime number. 

The library syntax is $\teb{prime}(x)$. $x$ must be a 32-bit C-integer.

\subsecref{primes}$(x)$: creates a row vector whose components
are the first $x$ prime numbers. 

The library syntax is $\teb{primes}(x)$. $x$ must be a 32-bit C-integer.

\subsecref{primroot}$(x)$: returns a primitive root of $x$, where $x$
is a prime power.

The library syntax is $\teb{gener}(x)$.

\subsecref{qfi}$(a,b,c)$: creates the binary quadratic form $ax^2+bxy+cy^2$
with $b^2-4ac<0$.

The library syntax is $\teb{qfi}(a,b,c)$.
\sref{Binary quadratic form}

\subsecref{qfr}$(a,b,c,d)$: creates the binary quadratic form $ax^2+bxy+cy^2$
with $b^2-4ac>0$ and distance function $d$.

The library syntax is $\teb{qfr}(a,b,c,d)$.

\subsecref{quaddisc}$(x)$: discriminant of the quadratic field 
$\Bbb Q(\sqrt(x))$, where $x\in\Bbb Q$. Up to trivial 2-factors, this function
can also be used to compute the squarefree part of $x$.

The library syntax is $\teb{quaddisc}(x)$.

\subsecref{redimag}$(x)$: reduce the binary quadratic form $x$ with negative
discriminant.

The library syntax is $\teb{redimag}(x)$.

\subsecref{redreal}$(x)$: reduce the binary quadratic form $x$ with positive
discriminant.

The library syntax is $\teb{redreal}(x)$.

\subsecref{redrealnod}$(x,sq)$: reduce the binary quadratic form $x$ with
positive discriminant $D$ without computing the distance function, where
$sq$ is the integer square root of $D$ (no checking is done of this fact).

The library syntax is $\teb{redrealnod}(x,sq)$.

\subsecref{regula}$(x)$: regulator of the quadratic field of positive 
discriminant $x$. Error if $x$ is not a discriminant (fundamental or not) or
if $x$ is a square. See also {\tt buchreal} if $x$ is large.

The library syntax is $\teb{regula}(x,prec)$.

\subsecref{rhoreal}$(x)$: perform one reduction step on the binary quadratic
form $x$ with positive discriminant.

The library syntax is $\teb{rhoreal}(x)$.

\subsecref{rhorealnod}$(x,sq)$: perform one reduction step on the binary
quadratic form $x$ with positive discriminant $D$ without computing the
distance function, where $sq$ is the integer square root of $D$ (no checking
is done of this fact).

The library syntax is $\teb{rhorealnod}(x,sq)$.

\subsecref{sigma}$(x)$: sum of the positive divisors of $x$. $x$ must
be of type integer.

The library syntax is $\teb{sumdiv}(x)$.

\subsecref{sigmak}$(k,x)$: sum of the $k^{\text{th}}$ powers of the
positive divisors of $x$. $x$ must be of type integer, but (in library mode)
$k$ must be a 32-bit C-integer.

The library syntax is $\teb{sumdivk}(k,x)$.

\subsecref{simplefactmod}$(x,p)$: factorization mod $p$ of the polynomial
$x$. The coefficients of $x$ must be operation-compatible with $\Bbb Z/p\Bbb
Z$. The result is a two column matrix, the first column being the degrees of
the irreducible polynomials dividing $x$, and the second the exponents. If 
you want the irreducible polynomials themselves, use the function
{\tt factmod}.

The library syntax is $\teb{simplefactmod}(x,p)$.

\subsecref{smallfact}$(x)$: For integer $x$, finds only the prime factors
up to the default table created when GP was started (usually $500000$). The 
remaining part is hence not necessarily prime. This never takes more than a few
seconds, and is particularly advantageous with respect to {\tt factor} when
one needs only the small prime divisors and not the complete factorization.

The library syntax is $\teb{smallfact}(x)$.

\subsecref{unit}$(x)$: \ref{fundamental unit} of the real quadratic field
$\Bbb Q(\sqrt x)$ where  $x$ is the positive discriminant of the field.
If $x$ is not a fundamental discriminant, this
probably gives the fundamental unit of the corresponding order.
$x$ must be of type integer, and the result is a quadratic number.

The library syntax is $\teb{fundunit}(x)$.

\section{Functions related to elliptic curves.}

We have implemented a number of functions which are useful for number
theorists working on elliptic curves. We always use Tate's notations.

The functions assume that the curve is given by a general Weierstrass model

$$y^2+a_1xy+a_3y=x^3+a_2x^2+a_4x+a_6,$$

where a priori the $a_i$ can be of any scalar type. This curve can be 
considered as a five component vector {\tt e=[a1,a2,a3,a4,a6]}, but
for most functions it is useful to have at one's disposal more
information. This is given either by the function {\tt initell} (see below),
which gives a 19 component vector (which we will call a long vector in this
section), or by the faster function {\tt smallinitell}
which gives a 13 component vector (which we will call a medium vector), all 
these vectors starting in the same way.
Consequently, in functions which do not use the extra information given
by {\tt initell}, the curve can be given either as a five component
vector, or by one of the longer vectors computed by {\tt initell} or
{\tt smallinitell}.

Other functions, in particular those relative to height computations
(see {\tt hell} below) require also that the curve be in minimal
Weierstrass form. This is achieved by the function {\tt globalred}
below.

Points on elliptic curves are represented as two component vectors
{\tt [x,y]}, except for the point at infinity, i.e. the identity element
of the group law, represented by the one-component vector {\tt [0]}.

\subsecref{addell}$(e,z1,z2)$: sum of the points $z1$ and $z2$ on the
elliptic curve corresponding to the vector $e$.

The library syntax is $\teb{addell}(e,z1,z2)$.

\subsecref{akell}$(e,n)$: compute the coefficient $a_n$ of the $L$-function
of the elliptic curve $e$, i.e. in principle coefficients of
a newform of weight 2 assuming Taniyama-Weil. $e$ must be a medium or long
vector of the type given by {\tt smallinitell} or {\tt initell}. For this
function to work for every $n$ and not just those prime to the conductor,
$e$ must be a minimal Weierstrass equation. If this is not the case,
use the function {\tt globalred} first before using {\tt akell}.

The library syntax is $\teb{akell}(e,n)$.

\subsecref{anell}$(e,n)$: compute the vector of the first $n$ $a_k$
corresponding to the elliptic curve $e$, i.e. in principle coefficients of
a newform of weight 2 assuming Taniyama-Weil. $e$ must be a medium or long
vector of the type given by {\tt smallinitell} or {\tt initell}. For this
function to work for every $n$ and not just those prime to the conductor,
$e$ must be a minimal Weierstrass equation. If this is not the case,
use the function {\tt globalred} first before using {\tt anell}.

The library syntax is $\teb{anell}(e,n)$, where $n$ is a C integer.

\subsecref{apell}$(e,p)$: compute the $a_p$ corresponding to the
elliptic curve $e$ and the prime number $p$, using the baby-step giant-step
method and a trick due to Mestre. No checking is done that
$p$ is indeed prime. The number of points of $e$ over $\Bbb F_p$ is
$p+1-a_p$. $e$ must be a medium or long vector of the type given by 
{\tt smallinitell} or {\tt initell}.

The library syntax is $\teb{apell}(e,p)$.

\subsecref{apell2}$(e,p)$: compute the $a_p$ corresponding to the
elliptic curve $e$ and the prime number $p$ as a sum of Legendre symbols.
This is slower than {\tt apell} as soon as $p$ is greater than 100, say.
$e$ must be a medium or long vector of the type given by {\tt smallinitell}
or {\tt initell}.

The library syntax is $\teb{apell2}(e,p)$.

\subsecref{bilhell}$(e,z1,z2)$: If $z1$ and $z2$ are points on the elliptic
curve $e$, this function computes the value of the canonical bilinear form
on $z1$, $z2$, i.e. {\tt hell(e,z1$+$z2)-hell(e,z1)-hell(e,z2)}, where $+$
denotes of course addition on $e$. In addition, $z1$ or $z2$ (but not both)
can be vectors or matrices.

The library syntax is $\teb{bilhell}(e,z1,z2,prec)$.

\subsecref{chell}$(e,v)$: change the data for the elliptic curve $e$
by changing the coordinates using the vector {\tt v=[u,r,s,t]}, i.e.
if $x'$ and $y'$ are the new coordinates, then $x=u^2x'+r$, 
$y=u^3y'+su^2x'+t$. The vector $e$ must be a medium or long vector of the type
given by {\tt smallinitell} or {\tt initell}.

The library syntax is $\teb{coordch}(e,v)$.

\subsecref{chptell}$(x,v)$: change the coordinates of the point or vector of
points $x$ using the vector {\tt v=[u,r,s,t]}, i.e.
if $x'$ and $y'$ are the new coordinates, then $x=u^2x'+r$, 
$y=u^3y'+su^2x'+t$ (see also {\tt chell} above).

The library syntax is $\teb{pointch}(x,v)$.

\subsecref{globalred}$(e)$: calculate the arithmetic conductor, the global
minimal model of $e$ and the global Tamagawa number $c$. Here $e$ is an 
elliptic curve given by a medium or long vector of the type given by 
{\tt smallinitell} or {\tt initell}, and is supposed to have all its
coefficients $a_i$ in $\Bbb Q$. The result is a 3 component vector $[N,v,c]$.
$N$ is the arithmetic conductor of the curve, $v$ is itself a vector 
$[u,r,s,t]$ with rational components. It gives a coordinate change for $e$ 
over $\Bbb Q$ such that the resulting model has integral coefficients, is 
everywhere minimal, $a_1$ is 0 or 1, $a_2$ is 0, 1 or $-1$ and $a_3$ is 0 
or 1. Such a model is unique, and the vector $v$ is unique if we specify that
$u$ is positive. To get the new model, simply type {\tt chell(e,v)}.
Finally $c$ is the product of the local Tamagawa numbers $c_p$, which is a
quantity which enters in the Birch and Swinnerton-Dyer conjecture.

The library syntax is  $\teb{globalreduction}(e)$.

\subsecref{hell}$(e,z)$: global \ref{N\'eron-Tate height} of the
point $z$ on the elliptic curve $e$. The vector $e$ must be a long vector
of the type given by {\tt initell}. This computation is done using sigma and 
theta-functions and a trick due to J. Silverman.

The library syntax is $\teb{ghell}(e,z,prec)$. The archimedean contribution
alone is given by the library function $\teb{hell}(e,z,prec)$. 

\subsecref{hell2}$(e,z)$: same as {\tt hell}, except that the algorithm used
is Tate's $4^n$ algorithm, and is much slower. 

The library syntax is $\teb{ghell2}(e,z,prec)$.

\subsecref{initell}$(e)$: compute some fixed data concerning the elliptic
curve given by the five component vector $e$, which will be essential for
most further computations on the curve. The result is a 19-component
vector E (called a long vector in this section), containing the following
information:

The first 13 components contain

$$a_1,a_2,a_3,a_4,a_6,b_2,b_4,b_6,b_8,c_4,c_6,\Delta,j.$$

In particular, the discriminant is $E[12]$, and the $j$-invariant is
$E[13]$. 

For the other six components, their content depends on whether the curve
is defined over $\Bbb R$ or not.
\smallskip
When $e$ is defined over $\Bbb R$, $E[14]$ is a vector with three
components containing the roots of the associated Weierstrass equation.
If the roots are all real, then they are ordered by decreasing value.
If only one is real, it is the first component of $E[14]$.

$E[15]$ is the real period of $E$ (integral of $dx/(2y+a_1x+a_3)$ over the
connected component of the identity element of the real points of the
curve), and $E[16]$ is a complex period. In other words, $\omega_1=E[15]$
and $\omega_2=E[16]$ form a basis of the complex lattice defining $e$,
with $\tau=\dfrac{\omega_2}{\omega_1}$ having positive imaginary part.

$E[17]$ and $E[18]$ are the corresponding values $\eta_1$ and $\eta_2$
such that $\eta_1\omega_2-\eta_2\omega_1=i\pi$.

Finally, $E[19]$ is the volume of the complex lattice defining $e$.
\smallskip
When $e$ is defined over $\Bbb Q_p$, the $p$-adic valuation of $j$ must
be negative. Then $E[14]$ is the vector with a single
component equal to the $p$-adic root of the associated Weierstrass
equation corresponding to $-1$ under the Tate parametrization.

$E[15]$ is equal to the square of the $u$-value, in the notation of Tate.

$E[16]$ is the $u$-value itself, if it belongs to $\Bbb Q_p$, otherwise zero.

$E[17]$ is the value of Tate's $q$ for the curve $e$.

$E[18]$ is the value of Mestre's $w$ (this is technical), and
$E[19]$ is arbitrarily set equal to zero.
\smallskip
For all other base fields or rings, the last six components are
arbitrarily set equal to zero.

Note: temporarily, an older version of {\tt initell} called 
{\tt initell2} is installed, but will disappear in the near future.

The library syntax is $\teb{initell}(e,prec)$.

\subsecref{isoncurve}$(e,z)$: gives 1 (i.e. true) if the point $z$ is on the
elliptic curve $e$, 0 otherwise. Here $e$ can be a five-component vector or
a long vector.

The library syntax is $\teb{oncurve}(e,z)$, and the result is a
C-integer.

\subsecref{localred}$(e,p)$: calculate the Kodaira type of the local fiber of
the elliptic curve $e$ at the prime $p$. $e$ must be given by a medium or long 
vector of the type given by {\tt smallinitell} or {\tt initell}, and is 
assumed to have all its coefficients $a_i$ in $\Bbb Z$. The result is a 4 
component vector $[f,kod,v,c]$. Here $f$ is the exponent of $p$ in the 
arithmetic conductor of $e$, $kod$ is the Kodaira type which is coded as
follows:

1 means good reduction (type I$_0$), 2, 3 and 4 mean types II, III and IV 
respectively, $4+\nu$ with $\nu>0$ means type I$_\nu$;
finally the opposite values -1, -2, etc. refer to the starred types 
I$_0^*$, II$^*$, etc. The third component $v$ is itself a vector $[u,r,s,t]$
giving the coordinate changes done during the local reduction. Normally, this
has no use if u is 1, that is, if the given equation was already minimal.
Finally, the last component $c$ is the local Tamagawa number $c_p$.

The library syntax is $\teb{localreduction}(e,p)$.

\subsecref{lseriesell}$(e,s,N,A)$: $e$ being a medium or long vector given by
{\tt smallinitell} or {\tt initell}, this computes the value of the L-series
of $e$ at $s$. It is assumed that $e$ is a minimal model over $\Bbb Z$,
$N$ is equal to plus or minus the conductor depending on the sign of the
functional equation, that the curve is a Weil curve, and $A$ is a cutoff point
for the integral, which must be chosen close to 1 for best speed. The result
being independent of $A$, this function can be used to compute the sign
of the functional equation, and even $N$ if one wants, although it is always
much faster to apply the function {\tt globalred}. 

Note that in the present version \vers, this function is accessible externally
in GP, but in future versions it may only be accessible in library mode
since it will be superseded.

Note also that if the conductor of the curve is large, say greater than 
$10^{12}$, this function will take an unreasonable amount of time since it uses
a $O(N^{1/2})$ algoorithm.

The library syntax is $\teb{lseriesell}(e,s,N,A,prec)$ where $prec$ is a
C-integer.

\subsecref{mathell}$(e,x)$: $x$ being a vector of points, this
function outputs the Gram matrix of $x$ with respect to the N\'eron-Tate
height, in other words, the {\tt (i,j)} component of the matrix is
equal to {\tt hell(e,x[i]+x[j])-hell(e,x[i])-hell(e,x[j])}, where
{\tt x[i]+x[j]} denotes of course the sum of the points on the curve $e$.
The rank of this matrix, at least in some approximate sense, gives the
rank of the set of points, and if $x$ is a basis of the Mordell-Weil
group of $e$, its determinant is equal to the regulator of $e$.

The library syntax is $\teb{mathell}(e,x,prec)$.

\subsecref{ordell}$(e,x)$: gives a 0, 1 or 2-component vector containing
the $y$-coordinates of the points of the curve $e$ having $x$ as 
$x$-coordinate.

The library syntax is $\teb{ordell}(e,x)$.

\subsecref{orderell}$(e,z)$: gives the order of the point $z$ on the elliptic
curve $e$ if it is a torsion point, zero otherwise. In the present 
version \vers{}, this is implemented only for elliptic curves defined over
$\Bbb Q$.

The library syntax is $\teb{orderell}(e,z)$, and the result is a GEN.

\subsecref{pointell}$(e,z)$: $e$ being a long vector, computes the coordinates
$[x,y]$ on the curve $e$ corresponding to the complex number $z$. Hence this is
the inverse function of {\tt zell}. In other words, if the curve is put in 
Weierstra\ss\ form, $[x,y]$ represent the Weierstra\ss\ $\wp$-function and its
derivative. If $z$ is in the lattice defining $e$ over $\Bbb C$, the result is
the point at infinity $[0]$.

The library syntax is $\teb{pointell}(e,z,prec)$.

\subsecref{powell}$(e,n,z)$: computes $n$ times the point $z$ for the
group law on the elliptic curve $e$. Here $n$ is in $\Bbb Z$.

The library syntax is $\teb{powell}(e,n,z)$.

\subsecref{smallinitell}$(e)$: compute some fixed data concerning the elliptic
curve given by the five component vector $e$, which may be useful for
further computations on the curve. The result is a 13-component
vector E (called a medium vector in this section), containing the first 13
components of the vector given by {\tt initell}, in other words the vector

$$[a_1,a_2,a_3,a_4,a_6,b_2,b_4,b_6,b_8,c_4,c_6,\Delta,j].$$

In particular, the discriminant is $E[12]$, and the $j$-invariant is
$E[13]$. 

The library syntax is $\teb{smallinitell}(e)$.

\subsecref{subell}$(e,z1,z2)$: difference of the points $z1$ and $z2$ on the
elliptic curve corresponding to the vector $e$.

The library syntax is $\teb{subell}(e,z1,z2)$.

\subsecref{taniyama}$(e)$: computes the modular parametrization of the elliptic
curve $e$ (where $e$ is given in the format output by {\tt initell} or
{\tt smallinitell}), in the form of a two component vector $[u,v]$ of power 
series, given to the current default series precision. This vector is
characterized by the following two properties. First the point $(x,y)=(u,v)$
satisfies the equation of the elliptic curve. Second, the differential
$du/(2v+a_1u+a_3)$ is equal to $f(z)dz$, a differential form on $H/\Gamma_0(N)$
where $N$ is the conductor of the curve. The variable used in the power series
for $u$ and $v$ is $x$, which is implicitly understood to be equal to
$\exp(2*i*\pi*z)$. The equation of the curve $e$ must be minimal (use 
{\tt globalred} to get a minimal equation).

The library syntax is $\teb{taniyama}(e)$, and the precision of the result
is determined by the global variable {\tt precdl}.

\subsecref{torsell}$(e)$: If $e$ is an elliptic curve defined over $\Bbb Q$,
this outputs the torsion subgroup of $e$ as a 3 component vector 
{\tt [t,v1,v2]} with the following meaning. {\tt t} is the order of the
torsion group. {\tt v1} is a 0, 1 or 2-component vector giving the structure
of the torsion group as a product of cyclic groups, with {\tt v1[2]=2} if it
exists. Finally {\tt v2}, which is a vector of the same length as {\tt v1},
gives generators for these cyclic groups. $e$ must be a vector as output by
{\tt initell} or {\tt smallinitell}.

The library syntax is $\teb{torsell}(e)$.

\subsecref{zell}$(e,z)$: If $e$ is an elliptic curve with coefficients in
$\Bbb R$, this computes a complex number $t$ (modulo the lattice defining $e$)
corresponding to the point $z$, i.e. such that, in the standard Weierstrass
model, $\wp(t)=z[1],\wp'(t)=z[2]$. In other words, this is the inverse function
of {\tt pointell}.

If $e$ has coefficients in $\Bbb Q_p$, then either Tate's $u$ is in $\Bbb Q_p$,
in which case the output is a $p$-adic number $t$ corresponding to the point
$z$ under the Tate parametrization, or only its square is, in which case
the output is $t+1/t$. $e$ must be a long vector of the type given by
{\tt initell}.

The library syntax is $\teb{zell}(e,z,prec)$.

\section{Functions related to general number fields.}

In this section can be found functions which are used almost exclusively
for working in general number fields. Other less specific functions can be
found in the next section on polynomials. Functions related to quadratic 
number fields can be found in section 4 above.

Unless otherwise noted, $nf$ will denote a number field, i.e. a 9-component
vector in the format output by {\tt initalg} or {\tt initalg0}. An
{\it ideal} (usually $x$ or $y$) will mean the following:

$x$ can be a $\Z$-basis, in HNF or not. In this case $x$ is a square matrix.

$x$ can be an {\it idele}, i.e. a 2-component vector, the first being an
ideal given as a $\Z$-basis, the second being a $r_1+r_2$-row vector giving
the complex logarithmic archimedean information.

$x$ can be a $\Z_K$-generating system for an ideal.

$x$ can be a prime ideal, i.e. a 5-component vector in the format output
by {\tt primedec}.

$x$ can be a polymod, i.e. an algebraic integer, in which case $x$ is treated
as being the principal idele generated by $x$.

Finally, $x$ can be an integer or a rational number, also treated as a
principal idele.

\subsecref{base}$(x)$: \ref{integral basis} of the number field defined by
the monic irreducible polynomial $x$, using the round 4 algorithm.
For small degrees and coefficient size, {\tt base2} is a little faster.
For larger degrees, {\tt base} is much faster.

This program is the translation in C by D. Letard of a program written 
by David Ford in Maple. If only a local integral basis is desired, use 
{\tt factoredbase}.

The library syntax is $\teb{base}(x,\&d)$, where $d$ will receive the
discriminant of the number field ({\sl not} of the polynomial $x$).

\subsecref{base2}$(x)$: integral basis of the number field defined by
the monic irreducible polynomial $x$, using the round 2 algorithm.
This program is the translation by in C of a program written 
by David Ford in Algeb. 

The library syntax is $\teb{base2}(x,\&d)$, where $d$ will receive the
discriminant of the number field ({\sl not} of the polynomial $x$).

\subsecref{buchgen}$(P,c,c2)$: identical to {\tt buchgenfu} (see below)
except that the fundamental units and roots of unity are not computed.
Hence the result has only 7 components.

The library syntax is $\teb{buchgen}(P,c,c2,prec)$.

\subsecref{buchgenfu}$(P,c,c2)$: Buchmann's sub-exponential algorithm for
computing the class group, the regulator and a system of fundamental units of
the general algebraic number field $K$ defined by the irreducible monic 
polynomial $P$ with integer coefficients. 

{\bf Note that this is a completely experimental program which is constantly
being improved}.

The numbers $c$ and $c2$ are positive real numbers which control
the execution time and the stack size. To get maximum 
speed, set $c2=c$. To get a rigorous result (under GRH and more) you must take
$c2=12$ (or $c2=6$ in the quadratic case, but then you should use the much 
faster functions {\tt buchimag} and {\tt buchreal} below).
Reasonable values for $c$ are between $0.1$ and $2$.

The result of this function is a vector $v$ with 10 components, which for
ease of presentation is in fact output as a one column matrix. $v[1]$ is 
equal to the polynomial $P$. Note that for optimum performance, $P$ should
have gone through {\tt polred} or {\tt initalg}. $v[2]$ is the 2-component
vector $[r1,r2]$ (where $r1$ and $r2$ are as usual the number of real and 
half the number of complex embeddings of the number field $K$). $v[3]$ is the
2-component vector containing the field discriminant and the index, $v[4]$ is
an integral basis in Hermite normal form. $v[5]$ is a 3-component vector
containing the class number, the structure of the class group as a product
of cyclic groups of order $n_i$, and the corresponding generators of the
class group of respective orders $n_i$. $v[6]$ is the regulator computed to
an accuracy which is the maximum of an internally determined accuracy and 
of the default. $v[7]$ is a measure of the correctness of the result. If it
is close to 1, the results are correct (under GRH and a few more
reasonable assumptions). If it is close to a larger integer, this shows that
the product of the class number by the regulator is off by a factor equal to 
this integer, and you must start again with a larger value for $c$
or a different random seed. (Since the computation involves a random process,
starting again with exactly the same parameters may give the correct result.)
In this case a warning message is printed. $v[8]$ a vector with 2 components,
the first being the number $w$ of roots of unity in $K$ and the second a 
primitive $w$-th root of unity expressed as a polynomial.
$v[9]$ is a system of fundamental units also expressed as polynomials.
Finally, $v[10]$ gives a measure of the correctness of the computations of
the fundamental units (not of the regulator), expressed as a number of bits.
If this number is greater than $20$, say, everything is OK. If $v[10]\le0$,
then we have lost all accuracy in computing the units (usually an error
message will be printed and the units not given). In the intermediate
cases, one must proceed with caution (for example by increasing the current
precision).

Note: some installations may have this function disabled.

The library syntax is $\teb{buchgenfu}(P,c,c2,prec)$.

\subsecref{buchinit}$(P,c,c2)$: essentially identical to {\tt buchgen} 
except that the output is much longer, and should not be printed out explicitly
in general. The result of buchinit is used in programs such as 
{\tt isprincipal} and {\tt isunit}. The result is an 8-component vector $v$, 
whose first 6 components are technical and in principle are not used by the
casual user. However, for the sake of completeness, their description is as
follows. We use the notations explained in the book by H. Cohen, {\it A  
Course in Computational Algebraic Number Theory\/}, Graduate Texts in Maths
{\bf 138}, Springer-Verlag, 1993, Section 6.5, and pages 354 and 355 in 
particular.

$v[1]$ contains the matrix $W$ ({\tt mit} in the source code), i.e. the matrix
in Hermite Normal Form giving on prime ideal generators 
$(\bf p_i)_{1\le i\le r}$ relations for the class group.
$v[2]$ contains the matrix $B$ ({\tt matalpha}), i.e. the matrix
containing the expressions of the prime ideal factorbase in terms of the
$\bf p_i$. It is an $r\times c$ matrix. $v[3]$ contains the complex 
logarithmic embeddings of the system of fundamental units which has been found
(although not computed in the case of {\tt buchinit}). It is an 
$(r_1+r_2)\times(r_1+r_2-1)$ matrix. $v[4]$ contains the matrix $M''_C$ of
Archimedean components of the relations of the matrix $M$, except that the
first $r_1+r_2-1$ columns are suppressed since they are already in $v[3]$.
$v[5]$ contains the prime factor base, i.e. the list of $k$ prime ideals used
in finding the relations. Finally, $v[6]$ contains the permutation of the
prime factor base which was necessary to reduce the relation matrix to the
form explained on page 354 (i.e. with a big $c\times c$ identity matrix on
the lower right). Note that in the above mentioned book, the need to permute
the rows of the relation matrices which occur was not stressed.

The less technical components are as follows. $v[7]$ is equal to the number 
field data $nf$ given by {\tt initalg0}, and $v[8]$ is a 4-component vector
containing the 5-th, 6-th, 7-th and 8-th
components of the result of {\tt buchgenfu[,1]}, i.e. the class group, the 
regulator, the ``check'' number which should be close to 1 and the number of
roots of unity and a generator.

The library syntax is $\teb{buchinit}(P,c,c2,t,prec)$.

\subsecref{buchinitfu}$(P,c,c2)$: same as {\tt buchinit} except that the
8-th component contains the last 6 components of {\tt buchgenfu[,1]}, i.e. also
the fundamental units and the check on their computation.

The library syntax is $\teb{buchinitfu}(P,c,c2,prec)$.

\subsecref{discf}$(x)$: \ref{field discriminant} of the number field defined
by the monic irreducible polynomial $x$.

The library syntax is $\teb{discf}(x)$. See also {\tt base} above.

\subsecref{factoredbase}$(x,f)$: \ref{integral basis} of the number field 
defined by the monic irreducible polynomial $x$, using the round 2 algorithm,
where $f$ is the two column matrix of the factorization of the discriminant 
of the polynomial $x$. Note that $f$ does {\it not} need to be a complete
factorization of the discriminant. This is especially useful if only a local 
integral basis at a prime or primes is desired: only factors with exponents
greater or equal to 2 will be considered.

The library syntax is $\teb{factoredbase}(x,p,\&d)$, where $d$ will receive the
discriminant of the number field.

\subsecref{factoreddiscf}$(x,p)$: \ref{field discriminant} of the number field defined by
the monic irreducible polynomial $x$, using the round 2 algorithm, where $p$ is
the two column matrix of the factorization of the discriminant of the polynomial $x$.

The library syntax is $\teb{factoreddiscf}(x,p)$.

\subsecref{factoredpolred}$(x,p)$: same as {\tt polred} (see below) except that
$p$ is the two column matrix of the factorization of the discriminant of
the polynomial $x$.

The library syntax is $\teb{factoredpolred}(x,p,prec)$.

\subsecref{factoredpolred2}$(x,p)$: same as {\tt polred2} (see below) except that
$p$ is the two column matrix of the factorization of the discriminant of the polynomial $x$.

The library syntax is $\teb{factoredpolred2}(x,p,prec)$.

\subsecref{factornf}$(x,t)$: factorization of the univariate
polynomial $x$ over the number field defined by the (univariate)
polynomial $t$. $x$ may have coefficients in $\Bbb Q$ or in the number
field. The main variable of $t$ must be of lower priority than that of $x$
(in other words the variable number of $t$ must be greater than that of $x$).
However if the coefficients of the number field occur explicitly (as polymods)
as coefficients of $x$, the variable of these polymods {\it must} be the same
as the main variable of $t$. For example 
{\tt factornf(x\^{}2+mod(y,y\^{}2+1),y\^{}2+1)} and
{\tt factornf(x\^{}2+1,y\^{}2+1)} are legal but
{\tt factornf(x\^{}2+mod(z,z\^{}2+1),y\^{}2+1)} is not.

The library syntax is $\teb{polfnf}(x,t)$.

\subsecref{galois}$(x)$: Galois group of the nonconstant polynomial
$x\in\Bbb Q[X]$. In the present version \vers, $x$ must be irreducible and the
degree of $x$ must be less than or equal to 7 (for degrees 8,9,10, and 11
we have specific Pari programs available upon request).

The output is a 3-component vector $[n,s,k]$ with the following meaning:
$n$ is the cardinality of the group, $s$ is its signature ($s=1$ if the
group is a subgroup of the alternating group $A_n$, $s=-1$ otherwise),
and $k$ is the number of the group corresponding to a given pair $(n,s)$
($k=1$ except in 2 cases). Specifically, the groups are coded as follows,
using standard notations:
\smallskip
In degree 1: $S_1=[1,-1,1]$.
\smallskip
In degree 2: $S_2=[2,-1,1]$.
\smallskip
In degree 3: $A_3=C_3=[3,1,1]$, $S_3=[6,-1,1]$.
\smallskip
In degree 4: $C_4=[4,-1,1]$, $V_4=[4,1,1]$, $D_4=[8,-1,1]$, $A_4=[12,1,1]$, $S_4=[24,-1,1]$.
\smallskip
In degree 5: $C_5=[5,1,1]$, $D_5=[10,1,1]$, $M_{20}=[20,-1,1]$, $A_5=[60,1,1]$, $S_5=[120,-1,1]$.
\smallskip
In degree 6: $C_6=[6,-1,1]$, $S_3=[6,-1,2]$, $D_6=[12,-1,1]$, $A_4=[12,1,1]$, $G_{18}=[18,-1,1]$,
$S_4^-=[24,-1,1]$, $A_4\times C_2=[24,-1,2]$, $S_4^+=[24,1,1]$, $G_{36}^-=[36,-1,1]$,
$G_{36}^+=[36,1,1]$, $S_4\times C_2=[48,-1,1]$, $A_5=PSL_2(5)=[60,1,1]$, $G_{72}=[72,-1,1]$,
$S_5=PGL_2(5)=[120,-1,1]$, $A_6=[360,1,1]$, $S_6=[720,-1,1]$.
\smallskip
In degree 7: $C_7=[7,1,1]$, $D_7=[14,-1,1]$, $M_{21}=[21,1,1]$, $M_{42}=[42,-1,1]$,
$PSL_2(7)=PSL_3(2)=[168,1,1]$, $A_7=[2520,1,1]$, $S_7=[5040,-1,1]$.
\smallskip
The method used is that of resolvent polynomials. Beware that in the case of 
degree 7 one has to factor polynomials of degree 35 over $\Bbb Z$, and this 
can take a long time. Furthermore there is a risk that the pari stack 
overflows, so it is prudent in this case to start with a large stack.

The library syntax is $\teb{galois}(x,prec)$.

\subsecref{galoisconj}$(x)$: conjugates of a root $r$ of the nonconstant 
polynomial $x$ expressed as polynomials in $r$. This can be used even if the
corresponding number field is not Galois since some conjugates may lie in the
field.

The library syntax is $\teb{galoisconj}(x,prec)$. A slower version but
which has a better chance of finding the conjugates when the index is 
large is $\teb{galoisconj2}(x,prec)$ (not accessible from GP).

\subsecref{idealadd}$(nf,x,y)$: sum of the two ideals $x$ and $y$ in the 
number field $nf$. When $x$ and $y$ are given by $\Z$-bases, this does
not depend on $nf$ and can be used to compute the sum of any two $\Z$-modules.
The result is given in HNF.

The library syntax is $\teb{idealadd}(nf,x,y)$.

\subsecref{idealdiv}$(nf,x,y)$: quotient $x\cdot y^{-1}$ of the two ideals 
$x$ and $y$ in the number field $nf$, where $nf$ is a 9-component vector
given by {\tt initalg} or {\tt initalg0}. The result is given in HNF.

The library syntax is $\teb{idealdiv}(nf,x,y)$.

\subsecref{idealfactor}$(nf,x)$: factors into prime ideal powers the ideal
$x$ in the number field $nf$, where $nf$ is a 9-component vector
given by {\tt initalg} or {\tt initalg0}. The output format is similar to the
{\tt factor} function, and the prime ideals are represented in the form
output by the {\tt primedec} function, i.e. as 5-element vectors.

The library syntax is $\teb{idealfactor}(nf,x)$.

\subsecref{idealintersect}$(nf,x,y)$: sum of the two ideals $x$ and $y$ in the
number field $nf$. When $x$ and $y$ are given by $\Z$-bases, this does not
depend on $nf$ and can be used to compute the intersection of any two 
$\Z$-modules. The result is given in HNF.

The library syntax is $\teb{idealintersect}(nf,x,y)$.

\subsecref{idealinv}$(nf,x)$: inverse of the ideal $x$ in the number 
field $nf$. $nf$ must be a 9-component vector as given by {\tt initalg},
The result is the Hermite normal form of the inverse of the ideal, together
with the opposite of the Archimedean information if it is given.

The library syntax is $\teb{idealinv}(nf,x)$.

\subsecref{ideallllred}$(nf,x,vdir)$: LLL reduction of the ideal $x$
in the number field $nf$, along the direction $vdir$. Here $vdir$ must be
either an $r1+r2$-component vector ($r1$ and $r2$ number of real and complex
places of $nf$ as usual), or the Pari zero, in which case $vdir$ is assumed
to be equal to the vector having only components equal to 1. The notion
of reduction along a direction is technical and cannot be explained here.
Note that this is {\it not\/} the same as the LLL
reduction of the lattice $x$ since ideal operations are involved. The
result is the Hermite normal form of the LLL-reduced ideal, which is usually,
but not always, a reduced ideal. $x$ may also be a 2-component vector,
the first being as above, and the second containing a matrix of Archimedean
information. In that case, this matrix is suitably updated.

The library syntax is $\teb{ideallllred}(nf,x,vdir,prec)$.

\subsecref{idealmul}$(nf,x,y)$: ideal multiplication of the ideals $x$ and
$y$ in the number field $nf$. $nf$ must be a 9-component vector as given 
by {\tt initalg} or {\tt initalg0}. The result is a generating set for the 
ideal product with at most $n$ elements, and is in Hermite normal form if $x$
and $y$ are, and this is given together with the sum of the Archimedean
information in $x$ and $y$ if both are given.

The library syntax is $\teb{idealmul}(nf,x,y)$.

\subsecref{idealmulred}$(nf,x,y)$: same as {\tt idealmul}, except that the
result is reduced using {\tt ideallllred}.

The library syntax is $\teb{idealmulred}(nf,x,y,prec)$, where $prec$ is a
C-integer representing the precision.

\subsecref{idealpow}$(nf,x,k)$: computes the $k$-th power of the ideal $x$ in
the number field $nf$ as given by {\tt initalg} or {\tt initalg0}. $k$ can be
positive negative or zero. The result is NOT reduced, it is really the $k$-th
ideal power, and is given in HNF. Use {\tt idealpowred} if you want a reduced
result.

The library syntax is $\teb{idealpow}(nf,x,k)$.

\subsecref{idealpowred}$(nf,x,k)$: computes a reduced $k$-th power of the ideal
$x$ in the number field $nf$ as given by {\tt initalg} or {\tt initalg0}. 
To get a non-reduced result, use {\tt idealpow}. Note however that 
{\tt idealpowred} is NOT the same as as {\tt idealpow} followed by reduction
(i.e. {\tt ideallllred}), since the reduction is performed throughout the
powering process.

The library syntax is $\teb{idealpowred}(nf,vp,k,prec)$, where $prec$ is a
C-integer.

\subsecref{idealval}$(nf,x,vp)$: gives the valuation of the ideal $x$
at the prime ideal $vp$ in the number field $nf$. $nf$ must be a 9-component 
vector as given by {\tt initalg} or {\tt initalg0}, and $vp$ must be a
6-component vector as given by {\tt primedec}.

The library syntax is $\teb{idealval}(nf,x,vp)$ and the result is a C-integer.

\subsecref{initalg}$(x)$: $x$ being a nonconstant irreducible polynomial in 
$\Bbb Q[X]$, computes a 9-component vector $v$ useful in working in the number
field $K$ defined by $x$. $v[1]$ contains a polynomial defining $K$ which is
as simple as can easily be found using the {\tt polred} algorithm. $v[2]$ 
contains $[r1,r2]$, the number of real and complex places of $K$. $v[3]$
contains the discriminant of the number field $K$. $v[4]$ contains the 
index of $v[1]$, i.e. $[\Bbb Z_K:\Bbb Z[\theta]]$, where $\theta$ is any
root of $v[1]$. $v[5]$ is a vector containing 5 matrices $M$, $MC$, $T2$, 
$T$ and $MD$ useful for certain computations in the number field $K$.
The matrix $M$ is the $(r1+r2)\times n$ matrix whose
columns represent the numerical values of the conjugates of the elements of
the integral basis. The matrix $MC$ is essentially the conjugate of the
transpose of $M$, except that the last $r2$ columns are also multiplied by 2.
The matrix $T2$ is an $n\times n$ matrix equal to the product $MC\cdot M$,
the so-called T2-matrix. The matrix $T$ is the $n\times n$ matrix whose 
coefficients are $\text{Tr}(\omega_i\omega_j)$ where the $\omega_i$
are the elements of the integral basis. Note that $T=\overline{MC}\cdot M$
and in particular that $T=T_2$ if the field is totally real (in practice
$T_2$ will have real approximate entries and $T$ will have integer entries).
Note also that $\det(T)$ is equal to the discriminant of the field $K$.
Finally, the columns of $MD$ express a $\Bbb Z$-basis of the different of $K$
on the integral basis $v[7]$.

$v[6]$ is the vector containing the $r1+r2$ roots of $v[1]$ corresponding
to the $r1+r2$ embeddings of the number field into $\Bbb C$ (the first $r1$ 
components are real, the next $r2$ have positive imaginary part). $v[7]$ is
an integral basis in Hermite normal form for $\Bbb Z_K$ expressed on the 
powers of $\theta$. $v[8]$ is the $n\times n$ integral matrix expressing
the power basis in terms of the integral basis, and finally $v[9]$ is the
$n\times n^2$ matrix giving the multiplication table of the integral basis.

The library syntax is $\teb{initalg}(x,prec)$.

\subsecref{initalg0}$(x)$: same as {\tt initalg} except that the polynomial
is not changed. This is useful in conjunction with functions expecting a
number field {\tt nf} but where one wants to keep the initial polynomial.

The library syntax is $\teb{initalg0}(x,prec)$.

\subsecref{initalg2}$(P)$: this is a complement to {\tt initalg}. With
the same conditions, it produces a polymod $mod(a,pol)$ equal to
$mod(x,P)$ where $pol$ is the polynomial output by {\tt initalg}.

The library syntax is $\teb{initalg2}(x,prec)$.

\subsecref{isincl}$(x,y)$: tests whether the number field $K$ defined by the
polynomial $x$ is conjugate to a subfield of the field $L$ defined by $y$ 
(where $x$ and $y$ must be in $\Bbb Q[X]$). If they are not, the output is the
number 0. If they are, the output is a vector of polynomials, each polynomial
$a$ representing an isomorphism of $K$ into $L$, i.e. being such that
$y=x\circ a$.

The library syntax is $\teb{nfincl}(x,y)$.

\subsecref{isisom}$(x,y)$: tests whether the two number fields defined by the
polynomials $x$ and $y$ (which must be in $\Bbb Q[X]$) are isomorphic. If they
are not, the output is the number 0. If they are, the output is a vector of
polynomials, each polynomial $a$ representing an isomorphism, i.e. being such
that $y=x\circ a$.

The library syntax is $\teb{nfiso}(x,y)$.

\subsecref{isprincipal}$(bnf,x)$: $bnf$ being the number field data output
by {\tt buchinit}, and $x$ being either a $\Z$-basis of an ideal in the
number field (not necessarily in HNF) or a prime ideal in the format output
by the function {\tt primedec}, this function tests whether the ideal is 
principal or not. The result is much more complete than a simple true/false
answer: it is given as a 3 component vector $[v_1,v_2,e]$ with the following
meaning. $v_1$ is the vector of components $c_i$ of the class of the ideal $x$
in the class group, expressed on the generators $g_i$ given by {\tt buchinit} 
(specifically {\tt buchinit[8][1][3]}). The $c_i$ are chosen so that
$0\le c_i<n_i$ where $n_i$ is the order of $g_i$ (the vector of $n_i$ being 
{\tt buchinit[8][1][2]}).

The second component $v_2$ gives on the integral basis
the components of $\alpha$ such that $x=\alpha\prod_ig_i^{c_i}$.
In particular, $x$ is principal if and only if $v_1$ is equal to the zero 
vector, and if this the case $x=\alpha\Z_K$ where $\alpha$ is given by
$v_2$. Note that if $\alpha$ is too large to be given, a warning message will
be printed and $v_2$ will be set equal to the empty vector. Finally the third
component $e$ is analogous to the last component of {\tt buchgen}: it gives
a check on the accuracy of the result, in bits. $e$ should be at least 
$10$, and preferably much more. If this is not the case, the result
may be incorrect, and a higher default precision should be taken.

The library syntax is $\teb{isprincipal}(bnf,x)$.

\subsecref{isunit}$(bnf,x)$: $bnf$ being the number field data output by
{\tt buchinit} and $x$ being an algebraic number (type integer, rational or
polymod), this outputs the decomposition of $x$ on the fundamental units an
the roots of unity if $x$ is a unit, the empty vector otherwise.
More precisely, if $u_1$,\dots,$u_r$ are the fundamental units found
by {\tt buchgen} and if $\zeta$ is the generator of the group of roots
of unity found by {\tt buchgen}, the output is a vector $[x_1,\dots,x_r,x_{r+1}]$
such that $x=u_1^{x_1}\cdots u_r^{x_r}\cdot\zeta^{x_{r+1}}$. The $x_i$ are
integers for $i\le r$ and is an integer modulo the order of $\zeta$ for
$i=r+1$.

The library syntax is $\teb{isunit}(bnf,x)$.

\subsecref{modreverse}$(a)$: $a$ being a polymod $A(X)$ modulo $T(X)$,
finds the ``reverse polymod'' $B(X)$ modulo $Q(X)$, where $Q$ is the minimal
polynomial of $a$, which must be equal to the degree of $T$, and such that
if $\theta$ is a root of $T$ then $\theta=B(\alpha)$ for a certain root
$\alpha$ of $Q$.

This is very useful when one changes the generating element in algebraic 
extensions.

The library syntax is $\teb{polymodrecip}(x)$.

\subsecref{newtonpoly}$(x,p)$: gives the vector of the slopes of the
Newton polygon of the polynomial $x$ with respect to the prime number $p$.
The $n$ components of the vector are in decreasing order, and $n$ is equal
to the degree of $x$. 

The library syntax is $\teb{newtonpoly}(x,p)$.

\subsecref{ordred}$(x)$: finds polynomials with reasonably small
coefficients and of the same degree as that of $x$ defining suborders
of the order defined by $x$. One of the polynomials always defines
$\Bbb Q$ (hence is equal to $(x-1)^n$, where $n$ is the degree), and
another always defines the same order as $x$ if $x$ is irreducible.

The library syntax is $\teb{ordred}(x)$.

\subsecref{polred}$(x)$: finds polynomials with reasonably small
coefficients defining subfields of the number field defined by $x$. 
One of the polynomials always defines
$\Bbb Q$ (hence is equal to $(x-1)$, where $n$ is the degree), and another
always defines the same number field as $x$ if $x$ is irreducible.

The library syntax is $\teb{polred}(x,prec)$.

\subsecref{polred2}$(x)$: finds polynomials with reasonably small
coefficients defining subfields of the number field defined by $x$. 
The result is a two column matrix, the first column giving the elements,
the second giving the polynomials.

One of the polynomials always defines $\Bbb Q$ (hence is equal to $(x-1)$,
where $n$ is the degree), and another
always defines the same number field as $x$ if $x$ is irreducible.

The library syntax is $\teb{polred2}(x,prec)$.

\subsecref{primedec}$(nf,p)$: computes the prime ideal decomposition of the
prime number $p$ in the number field $nf$. $nf$ must be a 9-component vector
as given by {\tt initalg} and $p$ must be a (positive) prime number. Note that
the fact that $p$ is prime is not checked, so if a non-prime number $p$ is 
given it may lead to unpredictable results.

The result is a vector of 5-component vectors, each representing one of the
prime ideals above $p$ in the number field $nf$. The representation 
$vp=[p,a,e,f,b]$ of a prime ideal means the following. The prime ideal is 
equal to $p\Z_K+\alpha\Z_K$ where $\Z_K$ is the ring of integers of the field
and $\alpha=\sum_i a_i\omega_i$ where the $\omega_i$ form the integral basis
$nf[7]$, $e$ is the ramification index, $f$ is the residual index, and $b$
is an $n$-component column vector representing some $\beta\in\Z_K$ which will
be useful for computing valuations but which the user can ignore.

The library syntax is $\teb{primedec}(nf,p)$.

\subsecref{principalideal}$(nf,x)$: creates the principal ideal generated by
the algebraic number $x$ (which must be of type integer, rational or polymod)
in the number field defined by the 9-component vector $nf$. The result is a
1-column matrix.

The library syntax is $\teb{principalideal}(nf,x)$.

\subsecref{principalidele}$(nf,x)$: creates the principal idele generated by
the algebraic number $x$ (which must be of type integer, rational or polymod)
in the number field defined by the 9-component vector $nf$. The result is a
2-component vector, the first being a 1-column matrix representing the
corresponding principal ideal, and the second being the vector with $r_1+r_2$
components giving the complex logarithmic embedding of $x$.

The library syntax is $\teb{principalidele}(nf,x)$.

\subsecref{rootsof1}$(nf)$: computes the number of roots of unity $w$ and a 
primitive $w$-th root of unity (expressed on the integral basis) belonging to
the number field defined by the vector $nf$ as given by {\tt initalg}.
The result is a two component vector $[w,z]$ where $z$ is a column vector
expressing a primitive $w$-th root of unity on the integral basis $nf[7]$.

The library syntax is $\teb{rootsof1}(nf)$.

\subsecref{signunit}$(bnf)$: $bnf$ being a the number field data output by
{\tt buchinit} and similar programs, this computes an $r_1\times(r_1+r_2-1)$
matrix having $\pm1$ components, giving the signs of the real embeddings of
the fundamental units found by {\tt buchinit}. 

The library syntax is $\teb{signunits}(bnf)$.

\subsecref{smallbase}$(x)$: \ref{integral basis} of the number field defined by
the monic irreducible polynomial $x$, using the round 2 algorithm, where one does
not take into account squares of primes which are not precomputed.

The library syntax is $\teb{smallbase}(x,\&y)$, where $y$ will receive the
discriminant of the number field ({\sl not} of the polynomial $x$).
This program is the translation in C of a program written by David Ford.

\subsecref{smalldiscf}$(x)$: \ref{field discriminant} of the number field defined
by the monic irreducible polynomial $x$, where one does not take into account
squares of primes which are not precomputed.

The library syntax is $\teb{smalldiscf}(x)$. See also {\tt smallbase} above.

\subsecref{smallpolred}$(x)$: same as {\tt polred} above except that
only a suborder of the maximal order may be used.

The library syntax is $\teb{smallpolred}(x,prec)$.

\subsecref{smallpolred2}$(x)$: same as {\tt polred2} above except that
only a suborder of the maximal order may be used.

The library syntax is $\teb{smallpolred2}(x,prec)$.

\subsecref{tschirnhaus}$(x)$: apply a random Tschirnhausen transformation
to the nonconstant separable polynomial $x$ so as to obtain a new equation
for the number field (or more generally the etale algebra) defined by $x$.
This is for instance useful when computing resolvents, hence is used in
the {\tt galois} function above.

The library syntax is $\teb{tschirnhaus(x)}$.

\section{Polynomials and power series.}

We group here all functions which are specific to polynomials or power
series. Many other functions which can be applied on these objects are
described in the other sections. Also, some of the functions described here
can be applied to other types.

\subsecref{apprpadic}$(x,a)$: vector of $p$-adic roots of the polynomial $x$
congruent to the $p$-adic number $a$ modulo $p$ (or modulo 4 if $p=2$), and
with the same $p$-adic precision as $a$. The number $a$ can be an ordinary $p$-adic
number (type 7, i.e. an element of $\Bbb Q_p$) or can be an element of a
finite extension of $\Bbb Q_p$, in which case it is of type 9 (polymod), where
at least one of the coefficients of the polymod is a $p$-adic number. In this case,
the result is the vector of roots belonging to the same extension of $\Bbb Q_p$ as
$a$.

The library syntax is $\teb{apprgen9}(x,a)$, but if $a$ is known to be simply
a $p$-adic number (type 7), the syntax $\teb{apprgen}(x,a)$ can be used.

\subsecref{convol}$(x,y)$: convolution (or \ref{Hadamard product}) of the
two power series $x$ and $y$; in other words if $x=\sum a_k*X^k$ and
$y=\sum b_k*X^k$ then $\text{convol}(x,y)=\sum a_k*b_k*X^k$.

The library syntax is $\teb{convol}(x,y)$.

\subsecref{cyclo}$(n)$: n-th cyclotomic polynomial, where n must be positive.

The library syntax is $\teb{cyclo}(n)$, where $n$ is a C-integer.

\subsecref{deriv}$(x,y)$: derivative of $x$ with respect to the simple
variable $y$. $x$ can be any type except polymod. The derivative of a scalar
type is zero, and the derivative of a vector or matrix is done
componentwise.

The library syntax is $\teb{deriv}(x,v)$, where $v$ is the number of the
variable $y$.

\subsecref{disc}$(x)$: discriminant of $x$. $x$ must be a polynomial.
The algorithm used is the \ref{subresultant algorithm}.

The library syntax is $\teb{discsr}(x)$.

\subsecref{eval}$(x)$: replace in $x$ the formal variables by the
values that have been assigned to them after the creation of $x$.
This is mainly useful in GP, and not in library mode. Do not confuse
this with substitution (see {\tt subst} below).

The library syntax is $\teb{geval}(x)$.

\subsecref{factorpadic}$(x,p,r)$: $p$-adic factorization of the polynomial $x$
to precision $r$, the result being a two column matrix as in {\tt factor}.
$r$ must be in general be strictly larger than the $p$-adic valuation of the
discriminant of $x$ for the result to make any sense. The method used is
Ford-Letard's implementation of the round 4 algorithm of Zassenhaus.
Another version using an algorithm due to Buchmann and Lenstra is
{\tt factorpadic2}, but is less efficient.

The library syntax is $\teb{factorpadic4}(x,p,r)$, where $r$ is a C-integer.

\subsecref{factpol}$(x,l)$: $x$ must be a polynomial with coefficients
in $\Bbb Z$. If $l=0$, find the complete factorization of $x$, and if
$l>0$, search only for irreducible factors of degree less than or equal to
$l$. The result is a two column matrix, the first one containing the 
irreducible factors, the second one the exponents.

The library syntax is $\teb{factpol}(x,l)$. The algorithm used is the standard
Hensel lifting of a mod $p$ factorization. Another implementation using instead
root finding over $\Bbb C$ (instead of implicitly using $\Bbb Q_p$) is
$\teb{factpol2}(x,l)$.

\subsecref{integ}$(x,y)$: \ref{formal integration} of $x$ with respect to
the simple variable $y$. No logarithmic terms
must occur in the result. $x$ can be of any type, but the case where $x$ is a
rational function is not implemented in the present version \vers.

The library syntax is $\teb{integ}(x)$.

\subsecref{laplace}$(x)$: $x$ must be a power series with only
nonnegative exponents. If $x=\sum (a_k/k!)*X^k$ then the result is
$\sum a_k*X^k$.

The library syntax is $\teb{laplace}(x)$.

\subsecref{legendre}$(x)$: creates the $x^{\text{th}}$ \ref{Legendre polynomial}.

The library syntax is $\teb{legendre}(x)$, where $x$ is a 32-bit C-integer.

\subsecref{polint}$(xa,ya,x)$: given the data vectors $xa$ and $ya$
of the same length $n$ ($xa$ containing the $x$-coordinates, and $ya$ the
corresponding $y$-coordinates), this function finds the \ref{interpolating polynomial}
passing through these points and evaluates it at the value $x$.

The library syntax is $\teb{polint}(xa,ya,x,\&er)$, where $er$ will contain an
error estimate on the returned value.

\subsecref{polsym}$(x,n)$: creates the vector of the \ref{symmetric powers}
of the roots of the polynomial $x$ up to power $n$.

The library syntax is $\teb{polsym}(x)$.

\subsecref{recip}$(x)$: reciprocal polynomial of $x$, i.e. the
coefficients are in reverse order. $x$ must be a polynomial.

The library syntax is $\teb{polrecip}(x)$.

\subsecref{resultant}$(x,y)$: resultant of the two polynomials
$x$ and $y$ with exact entries. The algorithm used is the subresultant algorithm.

The library syntax is $\teb{subres}(x,y)$.

\subsecref{resultant2}$(x,y)$: resultant of the two polynomials
$x$ and $y$. The algorithm used is the determinant of Sylvester's matrix.

The library syntax is $\teb{resultant2}(x,y)$.

\subsecref{reverse}$(x)$: reverse power series (i.e. $x^{-1}$, not
$1/x$) of $x$. $x$ must be a power series whose valuation is exactly equal
to one.

The library syntax is $\teb{recip}(x)$.

\subsecref{rootmod}$(x,p)$: roots modulo $p$ of the polynomial $x$. Slightly
slower than {\tt rootmod2} below for $p<100$, but much faster for larger
values. The particular nonprime value $p=4$ is accepted, mainly for $2$-adic
computations. Multiple roots are {\sl not} repeated.

The library syntax is $\teb{rootmod}(x,p)$.

\subsecref{rootmod2}$(x,p)$: roots modulo $p$ of the polynomial $x$. To
be used only when $p$ is small. Multiple roots are repeated with their order
of multiplicity.

The library syntax is $\teb{rootmod2}(x,p)$.

\subsecref{rootpadic}$(x,p,r)$: vector of $p$-adic roots of the polynomial $x$
with $p$-adic precision equal to $r$. Multiple roots are {\sl not} repeated.

The library syntax is $\teb{rootpadic}(x,p,r)$, where $r$ is a C-integer.

\subsecref{roots}$(x)$: complex roots of the polynomial $x$, given
as a column vector where each root is repeated according to its
multiplicity. The precision is given as for transcendental functions:
under GP it is kept in the variable {\tt prec} and is transparent to
the user, but it must be explicitly given as a second argument in library mode.

The algorithm used is a variant of the Newton-Raphson method and is not
guaranteed to converge, but is rather fast. If you get the messages
``too many iterations in roots'' or
``INTERNAL ERROR: incorrect result in roots'', try
modifying your polynomial to get the roots, or try to use the slower function
{\tt rootslong}.

The library syntax is $\teb{roots}(x,prec)$.

\subsecref{rootslong}$(x)$: same as {\tt roots} except that the program is
a little more robust and slower.

The library syntax is $\teb{rootslong}(x,prec)$.

\subsecref{sturm}$(x)$: number of real roots of the real polynomial $x$,
using Sturm's algorithm.

The library syntax is $\teb{sturm}(x)$. The result is a 32-bit C-integer.

\subsecref{sturmpart}$(x,a,b)$: number of real roots of the real polynomial $x$ in
the interval $(a,b]$, using Sturm's algorithm.

The library syntax is $\teb{sturmpart}(x)$. The result is a 32-bit C-integer.

\subsecref{subst}$(x,y,z)$:
replace the simple variable $y$ by the argument $z$ in expression $x$. 
Every non-scalar type is allowed for $x$. If $x$ is a
power series, $z$ must be either a polynomial, a power series, or a
rational function. $y$ must be a simple variable name.

The library syntax is $\teb{gsubst}(x, v,z)$, where $v$ is the number of
the variable $y$.

\subsecref{taylor}$(x,y)$: Taylor expansion around $0$ of $x$ with
respect to the simple variable $y$. $x$ 
can be of any reasonable type, for example a rational function. The number of
terms of the expansion is transparent to the user under GP, but must be
given as a second argument in library mode.

The library syntax is $\teb{tayl}(x,y,n)$, where the 32-bit C-integer $n$
is the desired number of terms in the expansion.

\subsecref{tchebi}$(x)$: creates the $x^{\text{th}}$ Tchebicheff
polynomial.

The library syntax is $\teb{tchebi}(x)$, where $x$ is a 32-bit C-integer.

\section{Vectors, matrices and linear algebra.}

\subsecref{adj}$(x)$: \ref{adjoint matrix} of $x$, i.e. the matrix $y$
of cofactors of $x$ such that $x*y=\text{det}(x)*\text{Id}$. $x$ must be
a (non necessarily invertible) square matrix.

The library syntax is $\teb{adj}(x)$.

\subsecref{algdep}$(x, k)$: 
\sref{algebraic dependence} $x$ being real or complex, finds a
polynomial of degree at most $k$ having $x$ as approximate root.
The algorithm used is a variant of the LLL algorithm due to Hastad,
Lagarias and Schnorr (STACS 1986). Note that the polynomial which is
obtained is not necessarily the ``correct'' one. One can check the
closeness either by a polynomial evaluation or substitution, or by
finding the roots of the polynomial given by algdep.

The library syntax is $\teb{algdep}(x,k,prec)$, where $k$ is a 32-bit
C-integer.

\subsecref{algdep2}$(x, k, dec)$: 
\sref{algebraic dependence} $x$ being real or complex, finds a
polynomial of degree at most $k$ having $x$ as approximate root.
$dec$ is a number which should be between half the number of decimal
digits of precision and that number. The algorithm used is the LLL
algorithm. Note that the polynomial which is
obtained is not necessarily the ``correct'' one. One can check the
closeness either by a polynomial evaluation or substitution, or by
finding the roots of the polynomial given by algdep2.

The library syntax is $\teb{algdep2}(x,k,dec,prec)$, where $k$ and $dec$ are
32-bit C-integers.

\subsecref{char}$(x,y)$: \ref{characteristic polynomial} of $x$ with respect
to the variable $y$, i.e. determinant of $y*I-x$ if $x$ is a square
matrix, determinant of the map multiplication by $x$ if $x$ is a polymod,
error if $x$ is of any other type.
For matrices, the method used is essentially the same as for computing the
adjoint matrix, i.e. computing the traces of the powers of $x$.

The library syntax is $\teb{caradj0}(x, v)$, where $v$ is the variable 
number, or even $\teb{caradj}(x,v,0)$. Note that the function 
{\tt caradj(x,v,pty)} returns the characteristic polynomial of $x$ in the 
variable $v$, and if {\tt pty} is not equal to 0, puts in {\tt pty} the
address of the adjoint matrix of $x$, so both can be obtained at once.

\subsecref{char1}$(x,y)$: \ref{characteristic polynomial} of $x$ with respect
to the variable $y$, i.e. determinant of $y*I-x$ if $x$ is a square
matrix, determinant of the map multiplication by $x$ if $x$ is a polymod,
error if $x$ is of any other type.
For matrices, the method used is that of Lagrange interpolation and is almost
always slower than {\tt char}. Identical to {\tt char} for polymods.

The library syntax is $\teb{caract}(x, v)$, where $v$ is the variable 
number.

\subsecref{char2}$(x,y)$: characteristic polynomial of the square matrix
$x$ with respect to the variable $y$ using the Hessenberg form. This is
faster than {\tt char} when the
coefficients are integermod a prime or real numbers, but is usually slower in
other base rings.

The library syntax is $\teb{carhess}(x,v)$, where $v$ is the variable number.

\subsecref{concat}$(x,y)$: concatenation of $x$ and $y$. If
$x$ or $y$ is not a vector or matrix, it is considered as a one
dimensional vector. All types are allowed for $x$ and $y$, but
the sizes must be compatible. Note that matrices are concatenated
horizontally, i.e. the number of rows stays the same. Using
transpositions, it is easy to concatenate them vertically.

To concatenate vectors sideways (i.e. to obtain a 2-row or two column
matrix), first transform the vector into a 1-row or 1-column matrix
using the function {\tt mat} below.

The library syntax is $\teb{concat}(x,y)$.

\subsecref{det}$(x)$: determinant of $x$. $x$ must be a
square matrix. Another program called {\bf \ref{det2}}$(x)$ is better
when the entries of the matrix are reals or integers for example,
but can be much worse for more complicated entries like multivariate
polynomials.

The library syntax is $\teb{det}(x)$ and $\teb{det2}(x)$.

\subsecref{detint}$(x)$: $x$ being an $m\times n$ matrix with integer
coefficients, this function computes a multiple of the determinant of the 
lattice generated by the columns of $x$ if it is of rank $m$, and returns
zero otherwise. This function can be useful in conjunction with the function
{\tt hermitemod} which needs to know such a multiple. Other ways to obtain
this determinant (assuming the rank is maximal) is 
{\tt det(lllkerim(x)[2]$*$x)} or simply {\tt det(hermite(x))}. 
Try to see which is faster in your case.

The library syntax is $\teb{detint}(x)$.

\subsecref{eigen}$(x)$: gives the eigenvectors of $x$ as columns of
a matrix.

The library syntax is $\teb{eigen}(x)$.

\subsecref{extract}$(x,y)$: extraction of components of the
vector or matrix $x$ according to $y$. $x$ must be a vector or
a matrix. In the case of a matrix, the components are as usual the
{\sl columns} of $x$. $y$ must be either a number of PARI type
integer, in which case it is considered as a mask: The binary bits of $y$ 
are read from right to left, but correspond to taking the components from
left to right. For example, if $y=13=(1101)_2$ then the components 1,3 and 4 are
extracted. 

Or $y$ can be a vector, with integer entries, in which case these
entries correspond to the component number to be extracted, in the order
specified.

In the case of a matrix, the difference between extract and matextract
is that only columns are extracted.

The library syntax is $\teb{extract}(x,y)$.

\subsecref{gauss}$(x,y)$: $x$ being a square matrix and
$y$ a column vector, finds the solution $u$ of $x*u=y$, using
Gaussian elimination. This has the same effect, but is much faster,
than $x^{-1}*y$.

The library syntax is $\teb{gauss}(x,y)$.

\subsecref{hermite}$(x)$: if $x$ is a (not necessarily square) matrix of
maximal rank, finds the {\sl upper triangular} Hermite Normal Form of $x$
If the rank of $x$ is equal to its number of rows, the result is a square
matrix. In general, the columns of the result form a basis of the lattice
spanned by the columns of $x$.

The library syntax is $\teb{hnf}(x)$.

\subsecref{hermitemod}$(x,d)$: if $x$ is a (not necessarily square) matrix of
maximal rank, and $d$ is the (non-zero) determinant of the lattice spanned by
the columns of $x$, finds the {\sl upper triangular} Hermite Normal Form of $x$
If the rank of $x$ is equal to its number of rows, the result is a square
matrix. In general, the columns of the result form a basis of the lattice
spanned by the columns of $x$. This is much faster than {\tt hermite} when
$d$ is known.

The library syntax is $\teb{hnfmod}(x,d)$.

\subsecref{hess}$(x)$: Hessenberg form of the square matrix $x$.

The library syntax is $\teb{hess}(x)$.

\subsec{hilbert}$(x)$: $x$ being a 32-bit C-integer,
creates the \ref{Hilbert matrix} of order $x$, i.e. the matrix whose
coefficient ($i$,$j$) is $1\over {i+j-1}$.

The library syntax is $\teb{hilbert}(x)$.

\subsecref{idmat}$(n)$: creates the $n\times n$ identity matrix.

The library syntax is $\teb{idmat}(n)$ where $n$ is a C-integer. Related
functions are $\teb{gscalmat}(x,n)$ which creates $x$ times the identity
matrix ($x$ being a GEN and $n$ a C-integer) and $\teb{gscalsmat}(x,n)$
which is the same when $x$ is a C-integer.

\subsecref{image}$(x)$: gives a basis for the image of the matrix
$x$ as columns of a matrix. A priori the matrix can have entries of any type.

The library syntax is $\teb{image}(x)$.

\subsecref{imagecompl}$(x)$: gives the vector of the column indices which are 
not extracted by the function {\tt image}. Hence the number of components
of {\tt imagecompl(x)} plus the number of columns of {\tt image(x)} is equal
to the number of columns of the matrix $x$.

The library syntax is $\teb{imagecompl}(x)$.

\subsecref{imager}$(x)$: gives a basis for the image of the matrix $x$ as
columns of a matrix, where $x$ has elements which can be nonexact real or
complex numbers. In that case, the precision of the matrix entries determines
what is meant by an element of the image. In particular, if the matrix is
ill conditioned, the results may not be what you expect.

The library syntax is $\teb{imagereel}(x)$.

\subsecref{indexrank}$(x)$: $x$ being a matrix of rank $r$, gives two vectors
of length $r$ giving a list of rows and columns respectively (starting from 1)
such that the extracted matrix obtained from these 2 vectors using 
{\tt matextract} is invertible.

The library syntax is $\teb{indexrank}(x)$.

\subsecref{indsort}$(x)$: indirect sorting of the vector $x$, i.e. if $x$
is an $n$-dimensional vector, creates the permutation of $[1,2,\dots,n]$ which
applied to the components of $x$ sorts $x$ in increasing order.

The library syntax is $\teb{indexsort}(x)$.

\subsecref{intersect}$(x,y)$: $x$ and $y$ being two matrices with the same number
of rows whose columns are independent, finds a basis of the $\Bbb Q$-vector space
equal to the intersection of the spaces spanned by the columns of $x$ and $y$
respectively. 

The library syntax is $\teb{intersect}(x,y)$.

\subsecref{inverseimage}$(x,y)$: gives one column vector belonging to the 
inverse image of the column vector $y$ by the matrix $x$ if one exists, the
empty vector otherwise. To get the complete inverse image, it suffices to
add to the result any element of the kernel of $x$ obtained for example by
{\tt ker}.

The library syntax is $\teb{inverseimage}(x,y)$.

\subsecref{jacobi}$(x)$: $x$ being a real symmetric matrix, this gives a
 vector having two components: the first one is the vector of eigenvalues of
$x$, the second is the corresponding orthogonal matrix of eigenvectors of
$x$. The method used is Jacobi's method for symmetric matrices.

The library syntax is $\teb{jacobi}(x)$.

\subsecref{ker}$(x)$: gives a basis for the kernel of the matrix
$x$ as columns of a matrix. A priori the matrix can have entries of any type.
See also {\tt keri} and {\tt kerr}.

The library syntax is $\teb{ker}(x)$.

\subsecref{keri}$(x)$: same as {\tt ker}, except that it assumes that the 
matrix has entries of type integer. In that case it is much faster.

The library syntax is $\teb{keri}(x)$.

\subsecref{kerint}$(x)$: gives an LLL-reduced $\Bbb Z$-basis for the lattice
equal to the kernel of the matrix $x$ as columns of the matrix $x$ with
integer entries (rational entries are not permitted) using a modified integer 
LLL algorithm.

The library syntax is $\teb{kerint}(x)$.

\subsecref{kerint1}$(x)$: like {\tt kerint}, but using {\tt matrixqz3},
gives an LLL-reduced $\Bbb Z$-basis for the lattice
equal to the kernel of the matrix $x$ as columns of the matrix $x$ with
integer or rational entries. If LLL reduction of the final result is not
desired, use {\tt matrixqz3(ker(x))} instead.

The library syntax is $\teb{kerint1}(x)$.

\subsecref{kerint2}$(x)$: like {\tt kerint}, but using another modified LLL
algorithm, gives an LLL-reduced $\Bbb Z$-basis for the lattice
equal to the kernel of the matrix $x$ as columns of the matrix $x$ with
integer entries (rational entries are not permitted). In the present version
\vers, only independent rows are allowed. 

The library syntax is $\teb{kerint2}(x)$.

\subsecref{kerr}$(x)$: gives a basis for the kernel of the matrix $x$ as
columns of a matrix, where $x$ has elements which can be nonexact real or
complex numbers. In that case, the precision of the matrix entries determines
what is meant by an element of the kernel. In particular, if the matrix is
ill conditioned, the results may not be what you expect.

The library syntax is $\teb{kerreel}(x)$.

\subsecref{lexsort}$(x)$ sort the elements of the vector $x$ by 
ascending lexicographic order (see {\tt lex} above).

The library syntax is $\teb{lexsort}(x)$.

\subsecref{lindep}$(x)$: 
\sref{linear dependence}$x$ being a vector with real
or complex coefficients, finds a small integral linear combination
among these coefficients using a variant of the LLL algorithm due to
Hastad, Lagarias and Schnorr (STACS 1986).

The library syntax is $\teb{lindep}(x,prec)$.

\subsecref{lindep2}$(x,dec)$: 
\sref{linear dependence}$x$ being a vector with real
or complex coefficients, finds a small integral linear combination
among these coefficients using the LLL algorithm. $dec$ is a parameter which
should be between one half the number of decimal digits of precision
and that number.

The library syntax is $\teb{lindep2}(x,dec,prec)$ where $dec$ is a C-integer.

\subsecref{lll}$(x)$: LLL algorithm applied to the {\bf columns}
of the (non necessarily square) matrix $x$. The columns of $x$ must
however be of maximal rank (otherwise if $x$ has integral coefficients
use {\tt lllint}. The result is a square 
transformation matrix $T$ such that $x\cdot T$ is an LLL-reduced
basis of the lattice generated by the column vectors of $x$. The
computations are done with real numbers (i.e. not with rational numbers) hence
are fast but as presently programmed (version \vers) are numerically unstable.
An older version of this program is {\tt lll1}.

The library syntax is $\teb{lll}(x,prec)$.

\subsecref{lllgram}$(x)$: same as LLL except that the matrix $x$
which is now square is the gram matrix of the lattice vectors, and
not the coordinates of the vectors themselves. The result is again
the transformation matrix $T$ which gives (as columns) the coefficients 
with respect to the initial basis vectors. Same remarks as for LLL about
numerical instability in the present version \vers. An older version of
this program is {\tt lllgram1}.

The library syntax is $\teb{lllgram}(x,prec)$.

\subsecref{lllgramint}$(x)$: The matrix $x$ being the gram matrix of a basis
of some lattice, is assumed to have integral entries. The result is the
unimodular transformation matrix $T$ which gives (as columns) the coefficients
with respect to the initial basis vectors. The computations are done entirely
with integers and is both accurate and quite fast.

The library syntax is $\teb{lllgramint}(x)$.

\subsecref{lllgramkerim}$(x)$: LLL algorithm applied to the gram matrix of
some lattice, which is assumed to have integral entries.
Here the columns can be linearly dependent. The result is a two component
vector of matrices whose concatenation form a unimodular transformation matrix
$T$ which gives as columns the coefficients with respect to the initial basis
vectors. The first matrix will give the kernel, the second will give an 
LLL-reduced basis of the image.

The library syntax is $\teb{lllgramkerim}(x)$.

\subsecref{lllint}$(x)$: LLL algorithm applied to the {\bf columns} of the
(non necessarily square) matrix $x$. It is assumed that the corresponding gram
matrix is integral. The result is a square 
transformation matrix $T$ such that $x\cdot T$ is an LLL-reduced
basis of the lattice generated by the column vectors of $x$. The
computations are done entirely with integers and is both accurate and quite 
fast.

The library syntax is $\teb{lllint}(x)$.

\subsecref{lllkerim}$(x)$: LLL algorithm applied to the columns of the (non
necessarily square) matrix $x$ which is assumed to have integral entries.
Here the columns can be linearly dependent. The result is a two component
vector of matrices, the columns of the first matrix representing a basis
of the integer kernel of $x$ (not necessarily LLL-reduced) and the columns
of the second matrix being an LLL-reduced $\Bbb Z$-basis of the image of the
matrix $x$.

The library syntax is $\teb{lllkerim}(x)$.

\subsecref{lllrat}$(x)$: same as LLL except that the computations
are all done in rational numbers. Hence no risk of numerical instability,
but extremely slow. This function is essentially superseded by {\tt lllint}
so will soon disappear.

The library syntax is $\teb{lllrat}(x)$.

\subsecref{mat}$(x)$: transform the object $x$ into a matrix.
If $x$ is not a vector or a matrix, this creates a $1\times 1$ matrix.
If $x$ is a row (resp. column) vector, this creates a 1-row (resp.
1-column) matrix. If $x$ is already a matrix, a copy of $x$ is created.

This function can be useful in connection with the function {\tt concat}
(see above).

The library syntax is $\teb{gtomat}(x)$.

\subsecref{matextract}$(x,y,z)$: extraction of a matrix
from the matrix $x$, the line mask or vector $y$ and the column mask 
or vector $z$.
$x$ must be a matrix, and $y$ and $z$ either numbers of PARI
type integer or vectors. The extraction is done using the same rules as for
{\tt extract} (see above). The difference with {\tt extract} is that both lines
and columns are extracted.

The library syntax is $\teb{matextract}(x,y,z)$.

\subsecref{matinvr}$(x)$: computes the inverse matrix of $x$ assumed to have
real entries. This may be more efficient than using {\tt ginv} (or $x^{-1}$
under GP).

The library syntax is $\teb{invmatreel}(x)$. A related function is
$\teb{invmulmatreel}(x,y)$ which directly computes $x^{-1}*y$ for matrices
$x$ and $y$.

\subsecref{matsize}$(x)$: $x$ being a vector or matrix, returns a row vector
 with two components, the first being the number of rows (1 for a row vector),
the second the number of columns (1 for a column vector).

The library syntax is $\teb{matsize}(x)$.

\subsecref{matrixqz}$(x,p)$: $x$ being an $m\times n$ matrix with $m\ge n$ with
rational or integer entries and of maximal rank, this function returns a matrix
having only integral entries, having the same image as $x$ (so is which is
obtained from $x$ by right multiplication with an $n\times n$ invertible matrix)
and such that the GCD of all its $n\times n$ subdeterminants is equal to 1 when
$p$ is equal to 0, or non divisible by $p$ otherwise. Here $p$ must be a prime
number (when it is nonzero). However, if the function is used when $p$ has no small
prime factors, it will either work or give the message ``impossible inverse modulo''
and a nontrivial divisor of $p$.

The library syntax is $\teb{matrixqz}(x,p)$.

\subsecref{matrixqz2}$(x)$: $x$ being an $m\times n$ matrix with rational
or integer entries, this function returns a matrix whose columns form a
basis of the lattice equal to $\Bbb Z^n$ intersected with the lattice generated
by the columns of $x$.

The library syntax is $\teb{matrixqz2}(x)$.

\subsecref{matrixqz3}$(x)$: $x$ being an $m\times n$ matrix with rational
or integer entries, this function returns a matrix whose columns form a
basis of the lattice equal to $\Bbb Z^n$ intersected with the 
$\Bbb Q$-vector space generated by the columns of $x$.

The library syntax is $\teb{matrixqz3}(x)$.

\subsecref{minim}$(x,b,m)$: $x$ being a square and symmetric
matrix with integer entries representing a positive definite quadratic form,
gives the number of vectors of square norm less than or equal to $b$ (for
the norm defined by $x$), and at most $m$ of these vectors. The result
is a three component vector, the first component being the number of vectors,
the second being the maximum norm found, and the last vector is a matrix
whose columns are the vectors found (at most $m$).

In the special case where $b=0$, the function searches for the minimal
non-zero vectors.

The library syntax is $\teb{minim}(x,b,m)$

\subsecref{perf}$(x)$: $x$ being a square and symmetric
matrix with integer entries representing a positive definite quadratic form,
looks for the perfection of the form: gives the rank of the family of the
s symetric matices $v_iv_i^t$, the $v_i$ ($1\le i\le s$) beeing the minimal vectors.

The library syntax is $\teb{perf}(x)$

\subsec{pascal}$(x)$: creates
as a matrix the lower triangular \ref{pascal triangle} of order $x+1$
(i.e. with binomial coefficients up to $x$).

The library syntax is $\teb{pascal}(x)$, where $x$ is a 32-bit C-integer.

\subsecref{rank}$(x)$: rank of the matrix $x$.

The library syntax is $\teb{rank}(x)$, and the result is a 32-bit
C-integer.

\subsecref{signat}$(x)$: signature of the quadratic form
represented by the symmetric matrix $x$. The result is a two-component vector.

The library syntax is $\teb{signat}(x)$

\subsecref{smith}$(x)$: if $x$ is a (singular or non-singular) square matrix,
outputs the vector of elementary divisors of $x$ (i.e. the diagonal of the
Smith normal form of $x$)

The library syntax is $\teb{smith}(x)$.

\subsecref{smith2}$(x)$: if $x$ is a (singular or non-singular) square matrix,
outputs a 2-component vector $[u,v]$ where $u$ and $v$ are two unimodular
matrices such that $uxv$ is the diagonal mattrix whose diagonal is 
{\tt smith(x)}.

The library syntax is $\teb{smith2}(x)$.

\subsecref{sort}$(x)$: sort the vector $x$ in ascending order, using the
heapsort method. $x$ must be a vector, and its components integers, reals, or
fractions. A related function is {\tt indsort} (see above) which gives
the indexes of the sorted vector in terms of the initial one. For example, 
{\tt extract(x,indsort(x))} is equivalent to {\tt sort(x)}.

The library syntax is $\teb{sort}(x)$.

\subsecref{sqred}$(x)$: \ref{decomposition into squares} of the quadratic form
represented by the symmetric matrix $x$. The result is a matrix whose diagonal
entries are the coefficients of the squares, and the non diagonal entries
represent the bilinear forms.

The library syntax is $\teb{sqred}(x)$.

\subsecref{supplement}$(x)$: assuming that the columns of the matrix $x$ are
linearly independent (if they are not, an error message is issued), find a
square invertible matrix whose first columns are the columns of $x$, i.e. 
supplement the columns of $x$ to a basis of the whole space.

The library syntax is $\teb{suppl}(x)$.

\subsecref{trace}$(x)$: This applies to quite general $x$.
If $x$ is not a matrix, it is equal to the sum of $x$ and its conjugate,
except for polymods where it is the trace as an algebraic number.

For $x$ a square matrix, it is the ordinary trace. If $x$ is a
non square matrix (but not a vector), an error occurs.

The library syntax is $\teb{trace}(x)$.

\subsecref{trans}$(x)$ or $x\tilde{\ }$: transpose of $x$.
This has an effect only on vectors and matrices.

The library syntax is $\teb{gtrans}(x)$.

\subsecref{vecsort}$(x,k)$: sort the vector of vectors or matrix $x$ in
ascending order of the $k$-th component, using the heapsort method. The 
components being compared must be of type integer, real or fraction.
$k$ can also be a vector, in which case the sorting is done lexicographically
according to the components listed in the vector $k$. For example, if
$k=[2,1,3]$, sorting will be done with respect to the 2nd-component, and when
these are equal with respect to the 1st, and when these are equal with respect
to the 3rd.

The library syntax is $\teb{vecsort}(x,k)$.

\section{Sums, products, integrals and similar functions.}

Although the GP calculator is programmable,
it is useful to have preprogrammed a number of loops, including sums, products,
and a certain number of recursions. Also, a number of functions from numerical
analysis like numerical integration and summation of series will be
described here.

One of the parameters in these loops must be the control variable, hence a
simple variable name. The last parameter must be any legal PARI expression,
including of course expressions using loops. 
Since it is much easier to program directly the loops in library mode, these
functions are mainly useful for GP programming. The use of these functions
in library mode is a little tricky and its explanation will be 
omitted, although the reader can try and figure it out by himself by reading
the source code. A brief explanatory sketch is given in section 11.
Hence in this section we only give the library syntax, with no semantic
explanation.

The letter $X$ will always denote any simple variable name, and represents the
formal parameter used in the function.

\subsecref{divsum}$(n,X,expr)$: sum of expression $expr$ over
the positive divisors of $n$.

In the present version \vers, $n$ is restricted to being less than $2^{31}$.

The library syntax is \tec{divsomme}{(entree *ep, GEN num, char *ch)}.

\subsecref{hvector}$(n,X,expr)$: creates a row vector with $n$ components
whose components are the expression $expr$ evaluated at the integral points
between 1 and $n$.

The library syntax is \tec{vecteur}{(entree *ep, GEN nmax, char *ch)}.

\subsec{(numerical) integration}:\sref{numerical integration}
A number of Romberg-like integration methods are implemented. The user should
not require too much accuracy: 18 or 28 decimal digits is OK, but not much 
more. In addition, analytical cleanup of the integral must have been done: 
there must be no singularities in the interval or at the boundaries. In 
practice this can be accomplished with a simple change of variable. 
Furthermore, for improper integrals, where one or both of the limits of 
integration are plus or minus infinity, the function must decrease 
sufficiently rapidly at infinity. This can often be accomplished
through integration by parts.

Note that \ref{infinity} can be represented with essentially no loss of 
accuracy by 1e4000. However beware of real underflow when dealing with rapidly
decreasing functions. For example, if one wants to compute the
$\int_0^\infty e^{-x^2}\,dx$ to 28 decimal digits, one should set infinity
equal to 10 for example, and certainly not to 1e4000.

The integrand may have values belonging to a vector space over the real
numbers; in particular, it can be complex-valued or vector-valued.

\subsecref{intgen}$(X=a, b, expr)$: general driver routine for doing 
numerical integration from $a$ to $b$ of the expression $expr$, with respect to
the formal variable $X$.

The library syntax is \tec{rombint}{(entree *ep, GEN a, GEN b, char *ch, long int prec)}.

\subsecref{intinf}$(X=a, b, expr)$: numerical integration from $a$ to $b$,
tailored for being used when $a$ or $b$ are infinite. One {\sl must} have 
$ab>0$, and in fact if for example $b=\infty$, then it is preferable to have 
$a$ as large as possible, at least $a\ge1$.

The library syntax is \tec{qromi}{(entree *ep, GEN a, GEN b, char *ch, long int prec)}.

\subsecref{intnum}$(X=a,b,expr)$: simple numerical integration from $a$ to $b$.
This is the fastest method, and should be used when $a$ and $b$ are not too
large, the function is smooth, and can be evaluated exactly everywhere on the
interval $[a,b]$.

The library syntax is \tec{qromb}{(entree *ep, GEN a, GEN b, char *ch, long int prec)}.

\subsecref{intopen}$(X=a,b,expr)$: numerical integration from $a$ to $b$. This
method should be used, again when $a$ and $b$ are not too large, but when the 
function is now allowed to be undefined (but continuous) at $a$ or $b$, for
example the function $\sin(x)/x$ at $x=0$.

The library syntax is \tec{qromo}{(entree *ep, GEN a, GEN b, char *ch, long int prec)}.

Two summation methods are also implemented, for alternating series, and for
series with terms of the same sign (see {\tt sumalt} and {\tt sumpos} below).

\subsecref{matrix}$(m,n,X,Y,expr)$: creation of the $m\times n$ matrix
whose coefficients are given by the expression $expr$. There are two formal
parameters in $expr$, the first one ($X$) corresponding to the rows, the
second ($Y$) to the columns, and $X$ goes from 1 to $m$, $Y$ goes from 1 to
$n$.

The library syntax is \tec{matrice}{(entree *ep1, entree *ep2, GEN nlig, GEN ncol, char *ch)}.

\subsecref{prod}$(x,X=a,b,expr)$: product of expression $expr$, initialized
at $x$, the formal parameter $X$ going from $a$ to $b$. As for {\tt sum}, the
initialization parameter $x$ must be given: its main purpose is to force
the type of the operations being performed. For example if it is put equal
to the integer 1, operations will start being done exactly. If it is put
equal to the real $1.$, they will be done using real numbers having the
default
precision. If it is put equal to the power series $1+O(X^k)$ for a certain
$k$, they will be done using power series of precision at most $k$. These
are the three most common initializations.

As an extreme example, compare $$\text{prod}(1,j=1,100,(1-X\hat{\ }j))$$
with  $$\text{prod}(1+O(X\hat{\ }101),j=1, 100, (1-X\hat{\ }j) ).$$

The library syntax is \tec{produit}{(entree *ep, GEN x, GEN a, GEN b, char *ch)}.

\subsecref{prodeuler}$(X=a,b,expr)$: product of expression $expr$,
initialized at 1., the formal parameter $X$ ranging over the prime numbers
between $a$ and $b$.\sref{Euler product}

The library syntax is \tec{prodeuler}{(entree *ep, GEN a, GEN b, char *ch, long int prec)}.

\subsecref{prodinf}$(X=a,expr)$: \ref{infinite product} of expression $expr$, 
the formal parameter $X$ starting at $a$. The evaluation stops when the 
relative error of the expression minus 1 is less than the default precision. 
The expressions must always evaluate to an element of $\Bbb C$.

The library syntax is \tec{prodinf}{(entree *ep, GEN a, char *ch, long int prec)}.

\subsecref{prodinf1}$(X=a,expr)$: infinite product of expression $1+expr$, the
formal parameter $X$ starting at $a$. The evaluation stops when the relative 
error of the expression is less than the default precision. The expressions
must always evaluate to an element of $\Bbb C$.

The library syntax is \tec{prodinf1}{(entree *ep, GEN a, char *ch, long int prec)}.

\subsecref{solve}$(X=a,b,expr)$: real root of expression $expr$ between $a$ and
$b$, under the condition $expr(X=a)*expr(X=b)\le0$. The method used is 
Brent's method.

The library syntax is \tec{zbrent}{(entree *ep, GEN a, GEN b, char *ch, long int prec)}.

\subsecref{sum}$(x,X=a,b,expr)$: sum of expression $expr$, initialized at
$x$, the formal parameter going from $a$ to $b$. As for {\tt prod}, the
initialization parameter $x$ must be given: its main purpose is to force
the type of the operations being performed. For example if it is put equal
to the integer 0, operations will start being done exactly. If it is put
equal to the real $0.$, they will be done using real numbers having the
default
precision. If it is put equal to the power series $O(X^k)$ for a certain
$k$, they will be done using power series of precision at most $k$. These
are the three most common initializations.

As an extreme example, compare $$\text{sum}(0,j=1, 1000, 1/j)$$
with  $$\text{sum}(0.,j=1, 1000, 1/j ).$$

The library syntax is \tec{somme}{(entree *ep, GEN x, GEN a, GEN b, char *ch)}.

\subsecref{sumalt}$(X=a, expr)$: numerical summation of the series $expr$,
which should be an \ref{alternating series}, the formal variable $X$ 
starting at $a$. The algorithm used is Euler's method of finite differences
combined with a trick due to van Wijngaarden.
Beware that the stopping criterion is that the term gets small enough, hence
terms which are equal to 0 will create problems and should be removed.

Divergent alternating series can sometimes be summed by this method, as well as
series which are not exactly alternating (see for example section 11).

The library syntax is \tec{sumalt}{(entree *ep, GEN a, char *ch, long int prec)}.

\subsecref{suminf}$(X=a,expr)$: \ref{infinite sum} of expression $expr$, the 
formal parameter $X$ starting at $a$. The evaluation stops when the relative
error of the expression is less than the default precision. The expressions
must always evaluate to an element of $\Bbb C$.

The library syntax is \tec{suminf}{(entree *ep, GEN a, char *ch, long int prec)}.

\subsecref{sumpos}$(X=a,expr)$: numerical summation of the series $expr$,
which must be a series of terms having the same sign, the formal variable $X$
starting at $a$. The algorithm used is van Wijngaarden's
trick for converting such a series into an alternating one, and is quite slow.
Beware that the stopping criterion is that the term gets small enough, hence
terms which are equal to 0 will create problems and should be removed.

The library syntax is \tec{sumpos}{(entree *ep, GEN a, char *ch, long int prec)}.

\subsecref{vector}$(n,X,expr)$: creates a row vector with $n$ components
whose components are the expression $expr$ evaluated at the integral points
between 1 and $n$ (identical with {\tt hvector}).

The library syntax is \tec{vecteur}{(entree *ep, GEN nmax, char *ch)}.

\subsecref{vvector}$(n,X,expr)$: creates a column vector with $n$ components
whose components are the expression $expr$ evaluated at the integral points
between 1 and $n$.

The library syntax is \tec{vvecteur}{(entree *ep, GEN nmax, char *ch)}.

\section{Plotting functions.}

Although plotting is not the main purpose of Pari, a number of plotting
functions are provided. These functions are either high-level plotting
functions (all the functions starting with {\tt plot}) in which the user
has little to do but explain what type of plot he wants, and whose syntax
is similar to the one used in the preceding section. Or they can be low-level
plotting functions, where every drawing primitive (point, line, box, etc...)
must be specified by the user. These low-level functions (called rectplot 
functions) work as follows. You have at your disposal 16 virtual windows which
are filled independently, and can then be physically ORed on a single window at
user-defined positions. These rectwindows are numbered from 0 to 15, and must
be initialized before being used by the function {\tt initrect}, which 
specifies the height and width of the virtual window. At all times, a virtual
cursor (initialized at $[0,0]$) is associated to the window, and its current
value can be obtained using the function {\tt cursor}.

A number of primitive graphic objects can then be drawn in these windows, and
only the part of the object which is inside the window will be drawn, with
the exception of polygons and strings which are drawn entirely (but the
virtual cursor can move outside of the window). Finally, the actual physical
drawing is done using the function {\tt draw}. Note that the windows are
preserved so that further drawings using the same windows at different 
positions or different windows can be done without extra work. If you want to
erase a window (and free the corresponding memory), use the function
{\tt killrect}. It is not possible to partially erase a window. Erase it
completely, initialize it again and then fill it with the graphic
objects that you want to keep.

In addition to initializing the window, you may want to have a scaled
window to avoid unnecessary conversions. For this, use the function
{\tt scale} below. As long as this function is not called, the scaling
is simply the number of pixels, the origin being at the upper left and
the $y$-coordinates going downwards. 

Finally, note that in the same way that {\tt texprint} allows you to have
a \TeX{} output corresponding to printed results, the functions starting
with {\tt post} allow you to have {\tt postscript} output of the plots. This
will not be absolutely identical with the screen output, but will be 
sufficiently close. Note that you can use postscript output even if you do
not have the plotting routines enabled. The postscript output is written in
a file called {\tt pari.ps}. Each time a new postscript output is asked for
the postscript output is appended to that file. Hence the user must remove
this file first if he does not want unnecessary drawings from preceding
sessions to appear. On the other hand, in this manner as many plots as desired
can be kept in a single file.
\smallskip
\subsecref{box}$(w,x2,y2)$: let $(x1,y1)$ be the current position of
the virtual cursor. Draw in the rectwindow $w$ the outline of
the rectangle which is such that the points $(x1,y1)$ and $(x2,y2)$ are 
opposite corners. Only the part of the rectangle which is in $w$ is drawn.
The virtual cursor does {\it not} move.

The library syntax is $\teb{rectbox}(w,x2,y2)$ where $w$ is a C-integer.

\subsecref{cursor}$(w)$: give as a 2-component vector the current
(scaled) position of the virtual cursor corresponding to the rectwindow $w$.

The library syntax is $\teb{rectcursor}(w)$ where $w$ is a C-integer.

\subsecref{draw}$(list)$: physically draw the rectwindows given in $list$
which must be a vector whose number of components is divisible by 3. If
$list=[w1,x1,y1,w2,x2,y2,\dots]$, the windows $w1$, $w2$, etc... are physically
placed with their upper left corner at physical position $(x1,y1)$, $(x2,y2)$,
\dots respectively, and are then drawn together. Overlapping regions will thus
be drawn twice, and the windows are considered transparent.

The library syntax is $\teb{rectdraw}(list)$.

Note that in the present version \vers{} this function has been written for the 
X11-window system (hence also for GUI's based on X11 such as
Openwindows and Motif), and for Sunview/Suntools.

\subsecref{initrect}$(w,x,y)$: initialize the rectwindow $w$ to width $x$ and
height $y$, and position the virtual cursor at $(0,0)$.

The library syntax is $\teb{initrect}(w,x,y)$ where all 3 arguments are
C-integers.

\subsecref{killrect}$(w)$: erase rectwindow $w$ and free the corresponding
memory. Note that if you want to use the rectwindow $w$ again, you
must use {\tt initrect} first to specify the new size.

The library syntax is $\teb{killrect}(w)$ where $w$ is a C-integer.

\subsecref{line}$(w,x2,y2)$: let $(x1,y1)$ be the position of the
virtual cursor. Draw in the rectwindow $w$ the part of the
segment $(x1,y1)-(x2,y2)$ which is inside $w$, and move the virtual cursor to
$(x2,y2)$ (even if it is outside the window). If you want to draw a
line from $(x1,y1)$ to $(x2,y2)$ where $(x1,y1)$ is not necessarily
the position of the virtual cursor, use {\tt move(w,x1,y1)} before
using the {\tt line} function.

The library syntax is $\teb{rectline}(w,x2,y2)$ where $w$ is a C-integer.

\subsecref{lines}$(w,listx,listy)$: draw on the rectwindow $w$ the polygon
such that the (x,y)-coordinates of the vertices are in the vectors of equal
length $listx$ and $listy$. The virtual cursor does not move. This is
basically the same function as the function {\tt ploth2}, but with no scaling
factors. Note that to obtain a closed polygon, the first and last point must
be the same. Finally, for simplicity, the whole polygon is drawn, not only the
part of the polygon which is inside the rectwindow.

The library syntax is $\teb{rectlines}(w,listx,listy)$ where $w$ is a
C-integer.

\subsecref{move}$(w,x,y)$: move the virtual cursor of the rectwindow $w$ to
position $(x,y)$.

The library syntax is $\teb{rectmove}(w,x,y)$ where $w$ is a C-integer.

\subsecref{plot}$(X=a, b, expr)$: crude plot of the function represented by
expression $expr$ from $a$ to $b$.

The library syntax is \tec{plot}{(entree *ep, GEN a, GEN b, char *ch)}.

\subsecref{ploth}$(X=a,b,expr)$: high precision plot of the function $y=f(x)$
represented by the expression $expr$, $x$ going from $a$ to $b$. Since this
involves around 1000 function calls, it is advised to keep the current
precision to a minimum (i.e. 9) before calling {\tt ploth}.

Note that in the present version \vers{} this function has been written for the 
Macintosh, the X11-window system (hence also for GUI's based on X11 such as
Openwindows and Motif), and for Sunview/Suntools. An Atari/gem version is also
available upon request.

The library syntax is \tec{ploth}{(entree *ep, GEN a, GEN b, char *ch, long int prec)}.

\subsecref{ploth2}$(T=a,b,expr)$: high precision plot of the curve represented
in parametric form by $x=f(t),y=g(t)$, represented by the expression $expr$,
$t$ going from $a$ to $b$. The expression must be a two component vector.
Since this involves around 2000 function calls, it is advised to keep the 
current precision to a minimum (i.e. 9) before calling {\tt ploth2}.

Note that in the present version \vers{} this function has been written for
the Macintosh, the X11-window system (hence also for GUI's based on X11 such as
Openwindows and Motif), and for Sunview/Suntools. An Atari/gem version is also
available upon request.

The library syntax is \tec{ploth2}{(entree *ep, GEN a, GEN b, char *ch, long int prec)}.

\subsecref{plothraw}$(listx,listy)$: high precision plot of the points whose
(x,y)-coordinates are in the vectors of equal length listx and listy. Automatic
positioning and scaling is done, but with the same scaling factor on $x$ and 
$y$.

The library syntax is $\teb{plothraw}(listx,listy)$.

Note that in the present version \vers{} this function has been written for
the X11-window system (hence also for GUI's based on X11 such as
Openwindows and Motif), and for Sunview/Suntools.

\subsecref{point}$(w,x,y)$: draw the point $(x,y)$ on the rectwindow $w$ (if
it is inside $w$), and in any case move the virtual cursor to position $(x,y)$.

The library syntax is $\teb{rectpoint}(w,x,y)$ where $w$ is a C-integer.

\subsecref{points}$(w,listx,listy)$: draw on the rectwindow $w$ the points
whose (x,y)-coordinates are in the vectors of equal length $listx$ and $listy$
and which are inside $w$. The virtual cursor does not move. This is basically 
the same function as the function {\tt plothraw}, but either with no
scaling factor or with a scale chosen using the function {\tt scale}.

The library syntax is $\teb{rectpoints}(w,listx,listy)$ where $w$ is a
C-integer.

\subsecref{postdraw}$(list)$: same as {\tt draw}, except that the output is
a postscript program appended to the file {\tt pari.ps}.

The library syntax is $\teb{postdraw}(list)$.

\subsecref{postploth}$(X=a,b,expr)$: same as {\tt ploth}, except that the 
output is a postscript program appended to the file {\tt pari.ps}.

The library syntax is \tec{postploth}{(entree *ep, GEN a, GEN b, char *ch)}.

\subsecref{postploth2}$(T=a,b,expr)$: same as {\tt ploth2}, except that the
output is a postscript program appended to the file {\tt pari.ps}.

The library syntax is \tec{postploth2}{(entree *ep, GEN a, GEN b, char *ch)}.

\subsecref{postplothraw}$(listx,listy)$: same as {\tt plothraw}, except that
the output is a postscript program appended to the file {\tt pari.ps}.

The library syntax is $\teb{postplothraw}(listx,listy)$.

\subsecref{rbox}$(w,dx,dy)$: draw in the rectwindow $w$ the outline of
the rectangle which is such that the points $(x1,y1)$ and $(x1+dx,y1+dy)$ are 
opposite corners, where $(x1,y1)$ is the current position of the cursor.
Only the part of the rectangle which is in $w$ is drawn. The virtual cursor
does {\it not} move.

The library syntax is $\teb{rectrbox}(w,dx,dy)$ where $w$ is a C-integer.

\subsecref{rline}$(w,dx,dy)$: draw in the rectwindow $w$ the part of the
segment $(x1,y1)-(x1+dx,y1+dy)$ which is inside $w$, where $(x1,y1)$
is the current position of the virtual cursor, and move the virtual cursor to
$(x1+dx,y1+dy)$ (even if it is outside the window).

The library syntax is $\teb{rectrline}(w,dx,dy)$ where $w$ is a C-integer.

\subsecref{rmove}$(w,dx,dy)$: move the virtual cursor of the rectwindow $w$ to
position $(x1+dx,y1+dy)$, where $(x1,y1)$ is the initial position of the cursor
(i.e. to position $(dx,dy)$ relative to the initial cursor).

The library syntax is $\teb{rectrmove}(w,dx,dy)$ where $w$ is a C-integer.

\subsecref{rpoint}$(w,dx,dy)$: draw the point $(x1+dx,y1+dy)$ on the rectwindow
$w$ (if it is inside $w$), where $(x1,y1)$ is the current position of the
cursor, and in any case move the virtual cursor to position $(x1+dx,y1+dy)$.

The library syntax is $\teb{rectrpoint}(w,dx,dy)$ where $w$ is a C-integer.

\subsecref{scale}$(w,x1,x2,y1,y2)$: scale the local coordinates of the
rectwindow $w$ so that $x$ goes from $x1$ to $x2$ and $y$ goes from $y1$ to
$y2$ ($x2<x1$ and $y2<y1$ being allowed). Initially, after the initialization
of the rectwindow $w$ using the function {\tt initrect}, the default scaling
is the graphic pixel count, and in particular the $y$ axis is oriented 
downwards since the origin is at the upper left. The function {\tt scale} 
allows to change all these defaults and should be used whenever functions are
graphed.

The library syntax is $\teb{rectscale}(w,x1,x2,y1,y2)$ where $w$ is a 
C-integer.

\subsecref{string}$(w,x)$: draw on the rectwindow $w$ at the current position
of the cursor the string corresponding to $x$. $x$ can either be a string
enclosed in quotes (\ "\ ), or $x$ can be a number which can be converted into
a real number. In that case it is printed in the format 9.3lf, i.e.~9 
characters before the decimal point, and 3 after.

The library syntax is $\teb{rectstring}(w,x)$ where $w$ is a C-integer and
$x$ is a GEN which either represents a vector of ASCII values, or can be
converted into a real number.

\section{Programming under GP and user-defined functions.}

\sref{programming}

\subsec{\ref{Variables} and symbolic expressions}.

In GP you can use up to 256 variable names. These names
can be any standard identifier names, i.e. they must start with a letter and contain only
alphanumeric characters. To avoid confusion with other symbols, you must not use non
alphanumeric symbols like '\_', '\$', or '.'. In addition to the function names which
you must not use (see the list with $\backslash$c) there are exactly three special 
variable names which you are not allowed to use: {\bf pi} and {\bf euler}, which
represent well
known constants, but also {\bf i} which is always equal to the square root of $-1$.
Since usually you are not going to use the variable name {\tt pi} for anything else but
the number $\pi$, and similarly for {\tt euler}, these two will not create any problems.
On the other hand, you will have to get used to calling a loop control variable
with a name different from {\tt i}. We realize that this may initially be the cause for
some confusion, but we do not have any replacement symbol for the square root of $-1$.

Now the main thing to understand is that PARI/GP is {\sl not} a symbolic manipulation
package, although it shares some of the functionalities. One of the main consequences
of this fact is that all expressions are evaluated as soon as they are written, they
never stay in a purely abstract form. As an important example, consider what happens
when you use a variable name {\sl before} assigning a value into it. This is perfectly
acceptable to GP, who considers this variable in fact as a polynomial of degree 1, with
coefficients 1 in degree 1, 0 in degree 0, whose variable is the variable name you used.

If later you assign a value to that variable, the objects which you have created before
will still be considered as polynomials. If you want to obtain their value, use the
function {\tt eval} (see section 7 above).

Another consequence is that the variables are numbered in the order
that they appear, and the main variable of an expression is always
the lowest numbered variable. Hence if you are working with expressions
involving several variables and want to have them ordered in a specific manner
{\sl in the internal representation}, the simplest is just to write down the variables
one after the other under GP before starting any real computations.
If you already have started working and want to change the names of the variables
in an object, use the function {\tt changevar} (see section 2). If you only want
to have them ordered when the result is printed, you can also use the function
{\tt reorder} (see below).

Finally, note that if $x$ is a vector, you can assign a result to $x[m]$ (i.e. write
something like $x[k]=expr$). If $x$ is a matrix, you can assign a result to $x[m,n]$, but {\sl not} to $x[m]$, which is a vector. If you want to assign an
expression to the $m$-th column of a matrix $x$, use $x[,m]=expr$ instead. Similarly, use $x[m,]=expr$ to assign an expression to the $m$-th row of $x$.

\subsec{\ref{Expression}s and \ref{expression sequence}s}.

An expression is formed by combining the GP operators, functions (including user-defined
functions) and control statements. It may be preceded by an assignment statement '$=$'
into a variable. It always has a value, which can be any PARI object.

Several expressions can be combined on a single line by separating them with semicolons
(';') and also with colons (':') for those who are used to BASIC. Such an expression
sequence will be called simply a $seq$. A $seq$ also has a value, which is the value
of the last nonempty expression in the sequence. Under GP, the value of the $seq$ is
always put on the stack (i.e. it will become the next object $\%n$), and only that value.
The values of the other expressions in the $seq$ are discarded after the execution of
the $seq$ is complete, except of course if they were assigned into variables.
In addition, the value of the $seq$ (or of course of an expression if there is only one)
is printed if the line does not end with a semicolon (';').

\subsecref{Control statements}.

A number of control statements are available under GP. They are simpler and have a
slightly different syntax than their C counterparts, but are quite powerful enough
to write any kind of program. Some of them are specific to GP, since they are made
for number theorists. They are as follows. As usual, $X$ will denote any simple variable
name, and $seq$ will always denote a sequence of expressions, including the empty
sequence.

\subsubsecref{for}$(X=a,b,seq)$: the formal variable $X$ going from $a$ to $b$,
the $seq$ is evaluated. Nothing is done if $a>b$. $a$ and $b$ must be in $\Bbb R$.

\subsubsecref{fordiv}$(n,X,seq)$: the formal variable $X$ ranging through the
positive divisors of $n$, the sequence $seq$ is evaluated. $n$ must be of type integer.

\subsubsecref{forprime}$(X=a,b,seq)$: the formal variable $X$ ranging over the prime
numbers between $a$ to $b$ (including $a$ and $b$ if they are prime),
the $seq$ is evaluated. Nothing is done if $a>b$. Note that $a$ and $b$
must be in $\Bbb R$.

\subsubsecref{forstep}$(X=a,b,s,seq)$: the formal variable $X$ going from $a$ to
$b$, in increments of $s$, the $seq$ is evaluated. Nothing is done if $s>0$ and 
$a>b$ or if $s<0$ and $a<b$. $a$, $b$ and $s$ must be in $\Bbb R$, and $s$ must
be nonzero.

\subsubsecref{goto}$(a)$: $a$ being an integer between 0 and 99, go to the
position in the seq referred to by label $a$. This can be either a forward skip
or a backwards skip. In case of multiple labels, control is transferred in the
following manner. If one such label is to the left of the goto instruction, 
control is transferred to the rightmost one among those which are to the left.
Otherwise, control is transferred to the rightmost label with a given number.
In any case, multiple labels in a given seq should be avoided. 

Also, good programming practice implies that the 
{\tt goto} instruction should be used only in special cases, such as getting
out of a deeply nested set of control instructions in one step.
In fact, in the present version \vers, the {\tt goto} instruction almost always
leads to syntax errors hence should be avoided, except as a means to stop a
program catastrophically without getting out of GP. This may be corrected in a
future release.

Note that usually, it is possible to use the {\tt goto} instruction
with an $a$ which is not an absolute constant, but may vary in the seq
("computed goto").

\subsubsecref{if}$(a,seq1,seq2)$: if $a$ is nonzero, the expression sequence $seq1$
is evaluated, otherwise the expression $seq2$ is evaluated. Of course, $seq1$ or $seq2$
may be empty, but the syntax must stay the same: {\tt if(a,seq,)} evaluates $seq$ if
$a$ is not equal to zero, does nothing otherwise; {\tt if(a,,seq)} evaluates $seq$ if $a$
is equal to zero, does nothing otherwise.

\subsubsecref{label}$(a)$: $a$ being an integer between 0 and 99, label this
position as label number $a$ for use by the {\tt goto} instruction. Labels
are specific to a given seq.

\subsubsecref{until}$(a,seq)$: evaluate expression sequence $seq$ until $a$ is not
equal to 0 (i.e. until $a$ is true). If $a$ is initially not equal to 0, $seq$ is
evaluated once (more generally, the condition on $a$ is tested {\sl after} execution of
the $seq$, not before as in {\tt while}).

\subsubsecref{while}$(a,seq)$: while $a$ is nonzero evaluate the expression sequence
$seq$. The test is made {\sl before} evaluating the $seq$, hence in particular if $a$
is initially equal to zero the $seq$ will not be evaluated at all.
\smallskip
\subsec{Specific functions used in GP programming}.

In addition to the general PARI functions, it is necessary to have some functions
which will be of use specifically for GP. They are as follows:

\subsubsecref{kill}$(x)$: kills the present value of the variable or user-defined
function $x$. After {\tt kill} of a variable, since the variable does not have a value,
as has been explained above it is again considered as a monic polynomial of degree 1
with no constant term.

For the following four printing functions, $list$ represents a list (separated by
commas) either of PARI objects or of character strings between double quotes
'{\tt "}', which are always printed as they are.

\subsubsecref{pprint}$(list)$: output $list$ in prettyprint (beautified) format,
ending with a newline.

\subsubsecref{pprint1}$(list)$: output $list$ in prettyprint (beautified) format,
without ending with a newline.

\subsubsecref{print}$(list)$: output $list$ in raw format, ending with a newline.

\subsubsecref{print1}$(list)$: output $list$ in raw format, without ending with a 
newline.

\subsubsecref{read}$()$: reads a string, interpreted as a GP expression, from
the input file, usually standard input (i.e. the keyboard). If a sequence of
expressions is given, the result is the result of the last expression of the
sequence. When using this instruction, it is useful to prompt for the string
by using the {\tt print1} function. Note that in the present version 2.12 of
{\tt pari.el}, when using GP under gnuemacs (see section 11) one {\sl must\/}
prompt for the string, with a string which ends with the same prompt as the
current one (usually a '?').

\subsubsecref{reorder}$(x)$: $x$ must be a vector. If $x$ is the empty vector,
this gives the vector whose components are the existing variables in increasing
order (i.e. in decreasing importance). If $x$ is nonempty, it must be a permutation
of variable names, and this permutation gives a new order of importance of the variables.
For example, if the existing order is {\tt [x,y,z]}, then after {\tt reorder([z,x])} the
order of importance of the variables will be {\tt [z,y,x]}.

\subsubsecref{texprint}$(list)$: outputs $list$ in \TeX{} format. This
output can then be used in a \TeX{} manuscript. 

Warning: in the present version \vers, the printing is done on the standard
output, hence to be able to get the output into a file, you must be either
in an emacs session or shell (see section 11), in a windowing system (which enables
you to grab the output with the mouse), under a UNIX script command, or with
the logfile enabled (see the command $\backslash$l). This will be improved.

\smallskip
\subsecref{User defined functions}.

It is very easy to define a new function under GP, which can then be used like any other
function. The syntax is as follows:

{\tt name(list of true formal variables, list of local variables)=seq}

where {\tt name} is the name that you want to give to your function (same syntactic
restrictions as for variable names). {\tt list of true formal variables} is the list
of variables corresponding to those which you will actually use when calling your
function, the variables being separated by commas, and the list being allowed to be
empty. {\tt list of local variables} is the list of the additional local variables
which you will use in the definition of the function. If you omit some or all of these
local variable declarations, the function will probably still work, but the nondeclared
variables will become global, hence known outside of the function, and this may have
undesirable side-effects. On the other hand, in some cases it may also be what you
want. Finally, as usual {\tt seq} is any expression sequence.

Once the function is defined using the above syntax, you can use it like any other
function. In addition, you can also recall its definition exactly as you do for
predefined functions, that is by writing {\tt ?name}. One small difference
with predefined functions is that you can never redefine such a function, while
you can redefine a user-defined function as many times as you want, without using the
{\tt kill} instruction.

In a given session you can give any identifier name to a function, {\sl except}
those of predefined functions (of course) but also those of any variables that
have been used, even if they have been killed. On the other hand, if you want
to use as a variable name the name of a user-defined function, it is enough
(and necessary) to kill this function name first.

An amusing example of a user-defined function is the following. It is intended
to illustrate both the use of
user-defined functions and the power of the {\tt sumalt} function. Although
since version 1.34 the \ref{Riemann zeta-function} is included in the standard
functions, let us assume that this is not the case (or that we want another
implementation). One way to define it, which is probably the simplest
(but certainly not the most efficient), is as follows:

\sref{zeta function}
{\tt zet(s,j)=sumalt(j=1,(-1)\^{}(j-1)*j\^{}(-s))/(1-2\^{}(1-s))}

Then this gives reasonably good accuracy and speed as long as you are not too far
from the domain of convergence. Try it for $s$ integral between $-5$ and $5$, say,
or for $s=0.5+i*t$ where $t=14.134\dots$ Of course, the call to the function is
done by {\tt zet(s)}, the variable {\tt j} must not be given.

\subsec{Special \ref{editing character}s}. A GP program can of course have more than
one line. Since GP executes your commands as soon as you have finished typing them, there
must be a way to tell it to wait for the next line or lines of input before doing
anything. There are two ways of doing this.

The first one is simply to use the \ref{backslash character} '$\backslash$' at the end of the
line that you are typing, just before hitting $<$return$>$. This tells GP that what you
will write on the next line is the physical continuation of what you have just written.
In other words, it makes GP forget your newline character. For example if you use this
while defining a function, and if you ask for the definition of the function using
{\tt ?name}, you will see that your backslash has disappeared and that everything is on
the same line. You can type a $\backslash$ anywhere. It will be interpreted as above
only if it is immediately followed by a newline. For example, you can type

{\tt 3+$\backslash$}

{\tt 4}

instead of typing {\tt 3+4}.

The second one cannot be used everywhere, but is in general more useful. It is the use
\sref{brace characters}
of braces  '$\{$' and  '$\}$'. When GP sees an opening brace ('$\{$`) at the beginning
of a line, it understands that you are writing a program, and newlines will be ignored
(but registered, contrary to when you use a backslash) until you type a closing brace
'$\}$'. However, there is an important (but easily obeyed) restriction: inside an
open brace-close brace pair, all newlines must occur after a semicolon (';'), i.e.
between two expressions forming a $seq$.

Of course you can combine the use of backslash and braces.

\subsec{The GP/PARI \ref{programming} language}.

As we have seen, the GP calculator uses a primitive purely interpreted language.
The structure of this language is in fact more reminiscent of LISP with a functional
notation, {\tt f(x,y)} rather than {\tt (f x y)}: all programming constructs,
such as {\tt if, while,} etc... are functions, and the main loop does not really
execute, but rather evaluates (sequences of) expressions.

Of course, it is by no means a true LISP. Function names are distinct from variable
names: once a name has been used as a variable name, this cannot be changed in the
same session. Predefined function names cannot be used for anything else, and the
number of actual parameters must match the declaration given in the online help.
Identifiers that are used for user-defined functions can be freely redefined for other
functions, or totally forgotten (using {\tt kill}) and later reused for a variable,
for example.

Each variable has a stack of values, implemented as a linked list. When a new
scope is entered, during a function call,
the value of the actual parameter is pushed on the stack. If the parameter is not
supplied, a special $0$ value called {\bf \ref{gnil}} is pushed on the stack
(this value is not printed if it is
returned as the result of a GP expression sequence). Upon exit, the stack decreases.
You can {\tt kill} a variable, decreasing the stack yourself. This should be used
only at the top level of GP, to undo the effect of an assignment,
not from a function. However, the stack has a bottom:
the value of a variable is the monomial of degree 1 in
this variable, as is natural for a mathematician.

Note that the iterative constructs which use a variable name ({\tt for, fordiv,
forprime, prod, sum, vector, matrix, plot,} etc.) also consider the given
variable to be local to the construct. A value is pushed on entry and pulled on
exit. So, it is not necessary for a function using such a construct to declare
the variable as a dummy formal parameter. In particular, in our {\tt zet}
example above, the variable {\tt j} need not be declared.

Otherwise, it is strongly recommended to declare in this way all other variables that are used
inside a function: If a function accesses a variable which is not one of its
formal parameters, the value used will be the one which happens to be on top of
the stack at the time of the call. This could be a ``global'' value, or a local
value belonging to any function higher in the call chain. So, be warned.

{\bf Implementation note.} For the curious reader, here is how these stacks are
handled: a \ref{hashing function} is computed from the identifier, and used as an
index in {\tt \ref{hashtable}}, a table of pointers. Each of these pointers begins
a linked list of structures (type {\tt \ref{entree}}). The linked list is searched
linearly for the identifier (each list will have less than 7 components or
so). When the correct {\tt entree} is found, it points to the top of the
stack of values for that identifier if it is a variable, to the function
itself if it is a predefined function, and to a copy of the text of the
function if it is a user-defined function. When an error occurs, all of this
maze (rather a tree, in fact) is searched and (hopefully) replaced in the
situation preceding the last call of the main evaluator.

\section{Using GP under \ref{gnuemacs}.}

Thanks to the help of Annette Hoffman from the university of Saarbr\"ucken, 
and since version 1.34.05 from David Carlisle from the university of
Manchester, it is possible and in fact desirable to use GP as a subprocess of
gnuemacs. This is of course possible only if gnuemacs has been installed on 
your machine.

To use this, you must include in your {\tt .emacs} file the following line:

{\tt (load "\ref{pari}" nil t)}

where {\tt pari.el} is the name of the file that will have to be loaded by gnuemacs
(if you have changed the name, or if you have the file in a different directory, you
must of course supply the correct name). This file is included in the PARI distribution.

Once this is done, under gnuemacs if you type {\tt M-x gp} (where as usual M is the
Meta key, i.e. Escape, or on SUN keyboards, the Left key), a special shell
will be started, which in particular launches GP with the default stack size,
prime limit and input buffer size. If you type instead {\tt C-u M-x gp},
you will be asked for the name of the GP executable, the stack size, the prime limit
and the input buffer size before the execution of GP begins. If for any of these
you simply type return, the default value will be used (on Unix machines it will be
{\tt /usr/local/bin/gp} for the executable, 4000000 for the stack, 500000 for the
prime limit and 30000 for the buffer size).
\smallskip
You can then work as usual under GP, but with two notable advantages.
First and foremost, you have at your
disposal all the facilities of a text editor like emacs, in particular for
correcting or copying blocks. Second, you can have an on-line help which is
much more complete than what you obtain by typing {\tt ?name}.
This is done by typing M-? (where M is the Meta key). In the minibuffer,
emacs asks what function you want to describe, and after your reply you obtain
the description which is in the users manual, including the description of
functions (such as $\backslash$, $\%$) which use special symbols.

This help system can also be menu-driven, by using the command {\tt M-\char`\\ c}
which opens a help menu window which enables you to choose the category
of commands for which you want an explanation.

Finally you can use command completion in the following way. After the prompt,
type the first few letters of the command, then M-TAB where TAB is the TAB key
(of course not the letters T,A,B). If there exists a unique command starting
with the letters you have typed, the command name will be completed. If not,
either the list of commands starting with the letters you typed will be
displayed in a separate window (which you can then kill by typing as usual
C-x 1 or by typing in more letters), or ``no match found'' will be displayed
in the Emacs command line.
\smallskip
You also have at your disposal a few other commands. Read the file
pari.txt for details.

Note that if for some reason the session crashes (due to a bug in your program
or in the PARI system), you will usually stay under emacs, but the GP buffer
will be killed. To recover it, simply type again M-x gp (or C-u M-x gp), and
a new session of GP will be started after the old one, so you can recover
what you have typed. Note that this will of course {\sl not} work if for some
reason you exited emacs before coming back (except for the C-z temporary
stopping command).


\vfill\eject

