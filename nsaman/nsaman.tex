\documentclass[a4]{article}
\parskip 4pt plus 1pt minus 1pt
\begin{document}

\begin{center}
{\Large \bf The NSA Security Manual}
\end{center}

\vspace{3ex}

[NOTE:  This file was retyped from an anonymous photocopied submission.  The
        authenticity of it was not verified.]

\vspace{2ex}

\small

{\large \bf \noindent Security Guidelines}

This handbook is designed to introduce you to some of the basic
security principles and procedures with which all NSA employees must comply.
It highlights some of your security responsibilities, and provides guidelines
for answering questions you may be asked concerning your association with this
Agency.  Although you will be busy during the forthcoming weeks learning your
job, meeting co-workers, and becoming accustomed to a new work environment, you
are urged to become familiar with the security information contained in this
handbook.  Please note that a listing of telephone numbers is provided at the
end of this handbook should you have any questions or concerns.

\vspace{2ex}

{\large \bf \noindent Introduction}

In joining NSA you have been given an opportunity to participate in the
activities of one of the most important intelligence organizations of the
United States Government.  At the same time, you have also assumed a trust
which carries with it a most important individual responsibility--the
safeguarding of sensitive information vital to the security of our nation.

While it is impossible to estimate in actual dollars and cents the value of the
work being conducted by this Agency, the information to which you will have
access at NSA is without question critically important to the defense of the
United States.  Since this information may be useful only if it is kept secret,
it requires a very special measure of protection.  The specific nature of this
protection is set forth in various Agency security regulations and directives.
The total NSA Security Program, however, extends beyond these regulations.  It
is based upon the concept that security begins as a state of mind.  The program
is designed to develop an appreciation of the need to protect information vital
to the national defense, and to foster the development of a level of awareness
which will make security more than routine compliance with regulations.

At times, security practices and procedures cause personal inconvenience.  They
take time and effort and on occasion may make it necessary for you to
voluntarily forego some of your usual personal perogatives.  But your
compensation for the inconvenience is the knowledge that the work you are
accomplishing at NSA, within a framework of sound security practices,
contributes significantly to the defense and continued security of the United
States of America.

I extend to you my very best wishes as you enter upon your chosen career or
assignment with NSA.

\noindent Philip T. Pease\\
Director of Security


\section{INITIAL SECURITY RESPONSIBILITIES}

\subsection{Anonymity}

Perhaps one of the first security practices with which new NSA personnel
should become acquainted is the practice of anonymity.  In an open society
such as ours, this practice is necessary because information which is
generally available to the public is available also to hostile intelligence.
Therefore, the Agency mission is best accomplished apart from public
attention.  Basically, anonymity means that NSA personnel are encouraged not
to draw attention to themselves nor to their association with this Agency.
NSA personnel are also cautioned neither to confirm nor deny any specific
questions about NSA activities directed to them by individuals not affiliated
with the Agency.

The ramifications of the practice of anonymity are rather far reaching, and
its success depends on the cooperation of all Agency personnel.  Described
below you will find some examples of situations that you may encounter
concerning your employment and how you should cope with them.  Beyond the
situations cited, your judgement and discretion will become the deciding
factors in how you respond to questions about your employment.

\subsection{Answering Questions About Your Employment}

Certainly, you may tell your family and friends that you are employed at or
assigned to the National Security Agency.  There is no valid reason to deny
them this information.  However, you may not disclose to them any information
concerning specific aspects of the Agency's mission, activities, and
organization.  You should also ask them not to publicize your association with
NSA.

Should strangers or casual acquaintances question you about your place of
employment, an appropriate reply would be that you work for the Department of
Defense.  If questioned further as to where you are employed within the
Department of Defense, you may reply, "NSA."  When you inform someone that you
work for NSA (or the Department of Defense) you may expect that the next
question will be, "What do you do?"  It is a good idea to anticipate this
question and to formulate an appropriate answer.  Do not act mysteriously about
your employment, as that would only succeed in drawing more attention to
yourself.

If you are employed as a secretary, engineer, computer scientist, or in a
clerical, administrative, technical, or other capacity identifiable by a
general title which in no way indicates how your talents are being applied to
the mission of the Agency, it is suggested that you state this general title.
If you are employed as a linguist, you may say that you are a linguist, if
necessary.  However, you should not indicate the specific language(s) with
which you are involved.

The use of service specialty titles which tend to suggest or reveal the nature
of the Agency's mission or specific aspects of their work.  These professional
titles, such as cryptanalyst, signals collection officer, and intelligence
research analyst, if given verbatim to an outsider, would likely generate
further questions which may touch upon the classified aspects of your work.
Therefore, in conversation with outsiders, it is suggested that such job
titles be generalized.  For example, you might indicate that you are a
"research analyst."  You may not, however, discuss the specific nature of your
analytic work.

\subsection{Answering Questions About Your Agency Training}

During your career or assignment at NSA, there is a good chance that you will
receive some type of job-related training.  In many instances the nature of the
training is not classified.  However, in some situations the specialized
training you receive will relate directly to sensitive Agency functions.  In
such cases, the nature of this training may not be discussed with persons
outside of this Agency.

If your training at the Agency includes language training, your explanation for
the source of your linguistic knowledge should be that you obtained it while
working for the Department of Defense.

You Should not draw undue attention to your language abilities, and you may not
discuss how you apply your language skill at the Agency.

If you are considering part-time employment which requires the use of language
or technical skills similar to those required for the performance of your NSA
assigned duties, you must report (in advance) the anticipated part-time work
through your Staff Security Officer (SSO) to the Office of Security's Clearance
Division (M55).

\subsection{Verifying Your Employment}

On occasion, personnel must provide information concerning their employment to
credit institutions in connection with various types of applications for
credit.  In such situations you may state, if you are a civilian employee,
that you are employed by NSA and indicate your pay grade or salary.  Once
again, generalize your job title.  If any further information is desired by
persons or firms with whom you may be dealing, instruct them to request such
information by correspondence addressed to: Director of Civilian Personnel,
National Security Agency, Fort George G. Meade, Maryland 20755-6000.  Military
personnel should use their support group designator and address when
indicating their current assignment.

If you contemplate leaving NSA for employment elsewhere, you may be required
to submit a resume/job application, or to participate in extensive employment
interviews.  In such circumstances, you should have your resume reviewed by
the Classification Advisory Officer (CAO) assigned to your organization.  Your
CAO will ensure that any classified operational details of your duties have
been excluded and will provide you with an unclassified job description.
Should you leave the Agency before preparing such a resume, you may develop
one and send it by registered mail to the NSA/CSS Information Policy Division
(Q43) for review.  Remember, your obligation to protect sensitive Agency
information extends beyond your employment at NSA.

\subsection{The Agency And Public News Media}

From time to time you may find that the agency is the topic of reports or
articles appearing in public news media--newspapers, magazines, books, radio
and TV.  The NSA/CSS Information Policy Division (Q43) represents the Agency in
matters involving the press and other media.  This office serves at the
Agency's official media center and is the Director's liaison office for public
relations, both in the community and with other government agencies.  The
Information Policy Division must approve the release of all information for and
about NSA, its mission, activities, and personnel.  In order to protect the
aspects of Agency operations, NSA personnel must refrain from either confirming
or denying any information concerning the Agency or its activities which may
appear in the public media.  If you are asked about the activities of NSA, the
best response is "no comment."  You should the notify Q43 of the attempted
inquiry.  For the most part, public references to NSA are based upon educated
guesses.  The Agency does not normally make a practice of issuing public
statements about its activities.

\section{GENERAL RESPONSIBILITIES}

\subsection{Espionage And Terrorism}

During your security indoctrination and throughout your NSA career you will
become increasingly aware of the espionage and terrorist threat to the United
States.  Your vigilance is the best single defense in protecting NSA
information, operations, facilities and people.  Any information that comes to
your attention that suggests to you the existence of, or potential for,
espionage or terrorism against the U.S. or its allies must be promptly reported
by you to the Office of Security.

There should be no doubt in your mind about the reality of the threats.  You
are now affiliated with the most sensitive agency in government and are
expected to exercise vigilance and common sense to protect NSA against these
threats.

\subsection{Classification}

Originators of correspondence, communications, equipment, or documents within
the Agency are responsible for ensuring that the proper classification,
downgrading information and, when appropriate, proper caveat notations are
assigned to such material.  (This includes any handwritten notes which contain
classified information).  The three levels of classification are Confidential,
Secret and Top Secret.  The NSA Classification Manual should be used as
guidance in determining proper classification.  If after review of this
document you need assistance, contact the Classification Advisory Officer
(CAO) assigned to your organization, or the Information Policy Division (Q43).

\subsection{Need-To-Know}

Classified information is disseminated only on a strict "need-to-know" basis.
The "need-to-know" policy means that classified information will be
disseminated only to those individuals who, in addition to possessing a proper
clearance, have a requirement to know this information in order to perform
their official duties (need-to-know).  No person is entitled to classified
information solely by virtue of office, position, rank, or security clearance.

All NSA personnel have the responsibility to assert the "need-to-know" policy
as part of their responsibility to protect sensitive information.
Determination of "need-to-know" is a supervisory responsibility.  This means
that if there is any doubt in your mind as to an individual's "need-to-know,"
you should always check with your supervisor before releasing any classified
material under your control.

\subsection{For Official Use Only}

Separate from classified information is information or material marked "FOR
OFFICIAL USE ONLY" (such as this handbook).  This designation is used to
identify that official information or material which, although unclassified, is
exempt from the requirement for public disclosure of information concerning
government activities and which, for a significant reason, should not be given
general circulation.  Each holder of "FOR OFFICAL USE ONLY" (FOUO) information
or material is authorized to disclose such information or material to persons
in other departments or agencies of the Executive and Judicial branches when it
is determined that the information or material is required to carry our a
government function.  The recipient must be advised that the information or
material is not to be disclosed to the general public.  Material which bears
the "FOR OFFICIAL USE ONLY" caveat does not come under the regulations
governing the protection of classified information.  The unauthorized
disclosure of information marked "FOR OFFICIAL USE ONLY" does not constitute an
unauthorized disclosure of classified defense information.  However, Department
of Defense and NSA regulations prohibit the unauthorized disclosure of
information designated "FOR OFFICIAL USE ONLY."  Appropriate administrative
action will be taken to determine responsibility and to apply corrective and/or
disciplinary measures in cases of unauthorized disclosure of information which
bears the "FOR OFFICIAL USE ONLY" caveat.  Reasonable care must be exercised in
limiting the dissemination of "FOR OFFICIAL USE ONLY" information.  While you
may take this handbook home for further study, remember that is does contain
"FOR OFFICIAL USE ONLY" information which should be protected.

\subsection{Prepublication Review}

All NSA personnel (employees, military assignees, and contractors) must submit
for review any planned articles, books, speeches, resumes, or public statements
that may contain classified, classifiable, NSA-derived, or unclassified
protected information, e.g., information relating to the organization, mission,
functions, or activities of NSA.  Your obligation to protect this sensitive
information is a lifetime one.  Even when you resign, retire, or otherwise end
your affiliation with NSA, you must submit this type of material for
prepublication review.  For additional details, contact the Information Policy
Division (Q43) for an explanation of prepublication review procedures.

\subsection{Personnel Security Responsibilities}

Perhaps you an recall your initial impression upon entering an NSA facility.
Like most people, you probably noticed the elaborate physical security
safeguards--fences, concrete barriers, Security Protective Officers,
identification badges, etc.  While these measures provide a substantial degree
of protection for the information housed within our buildings, they represent
only a portion of the overall Agency security program.  In fact, vast amounts
of information leave our facilities daily in the minds of NSA personnel, and
this is where our greatest vulnerability lies.  Experience has indicated that
because of the vital information we work with at NSA, Agency personnel may
become potential targets for hostile intelligence efforts.  Special safeguards
are therefore necessary to protect our personnel.

Accordingly, the Agency has an extensive personnel security program which
establishes internal policies and guidelines governing employee conduct and
activities.  These policies cover a variety of topics, all of which are
designed to protect both you and the sensitive information you will gain
through your work at NSA.

\subsection{Association With Foreign Nationals}

As a member of the U.S. Intelligence Community and by virtue of your access to
sensitive information, you are a potential target for hostile intelligence
activities carried out by or on behalf of citizens of foreign
countries.  A policy concerning association with foreign nationals has been
established by the Agency to minimize the likelihood that its personnel might
become subject to undue influence or duress or targets of hostile activities
through foreign relationships.

As an NSA affiliate, you are prohibited from initiating or maintaining
associations (regardless of the nature and degree) with citizens or officials
of communist-controlled, or other countries which pose a significant threat to
the security of the United States and its interests.  A comprehensive list of
these designated countries is available from your Staff Security Officer or the
Security Awareness Division.  Any contact with citizens of these countries, no
matter how brief or seemingly innocuous, must be reported as soon as possible
to your Staff Security Officer (SSO).  (Individuals designated as Staff
Security Officers are assigned to every organization; a listing of Staff
Security Officers can be found at the back of this handbook).

Additionally, close and continuing associations with any non-U.S. citizens
which are characterized by ties of kinship, obligation, or affection are
prohibited.  A waiver to this policy may be granted only under the most
exceptional circumstances when there is a truly compelling need for an
individual's services or skills and the security risk is negligible.

In particular, a waiver must be granted in advance of a marriage to or
cohabitation with a foreign national in order to retain one's access to NSA
information.  Accordingly, any intent to cohabitate with or marry a non-U.S.
citizen must be reported immediately to your Staff Security Officer.  If a
waiver is granted, future reassignments both at headquarters and overseas may
be affected.

The marriage or intended marriage of an immediate family member (parents,
siblings, children) to a foreign national must also be reported through your
SSO to the Clearance Division (M55).

Casual social associations with foreign nationals (other than those of the
designated countries mentioned above) which arise from normal living and
working arrangements in the community usually do not have to be reported.
During the course of these casual social associations, you are encouraged to
extend the usual social amenities.  Do not act mysteriously or draw attention
to yourself (and possibly to NSA) by displaying an unusually wary attitude.

Naturally, your affiliation with the Agency and the nature of your work should
not be discussed.  Again, you should be careful not to allow these associations
to become close and continuing to the extent that they are characterized by
ties of kinship, obligation, or affection.

If at any time you feel that a "casual" association is in any way suspicious,
you should report this to your Staff Security Officer immediately.  Whenever
any doubt exists as to whether or not a situation should be reported or made a
matter of record, you should decided in favor of reporting it.  In this way,
the situation can be evaluated on its own merits, and you can be advised as to
your future course of action.

\subsection{Correspondence With Foreign Nationals}

NSA personnel are discouraged from initiating correspondence with individuals
who are citizens of foreign countries.  Correspondence with citizens of
communist-controlled or other designated countries is prohibited.  Casual
social correspondence, including the "penpal" variety, with other foreign
acquaintances is acceptable and need not be reported.  If, however, this
correspondence should escalate in its frequency or nature, you should report
that through your Staff Security Officer to the Clearance Division (M55).

\subsection{Embassy Visits}

Since a significant percentage of all espionage activity is known to be
conducted through foreign embassies, consulates, etc., Agency policy
discourages visits to embassies, consulates or other official establishments of
a foreign government.  Each case, however, must be judged on the circumstances
involved.  Therefore, if you plan to visit a foreign embassy for any reason
(even to obtain a visa), you must consult with, and obtain the prior approval
of, your immediate supervisor and the Security Awareness Division (M56).

\subsection{Amateur Radio Activities}

Amateur radio (ham radio) activities are known to be exploited by hostile
intelligence services to identify individuals with access to classified
information; therefore, all licensed operators are expected to be familiar
with NSA/CSS Regulation 100-1, "Operation of Amateur Radio Stations" (23
October 1986).  The specific limitations on contacts with operators from
communist and designated countries are of particular importance.  If you are
an amateur radio operator you should advise the Security Awareness Division
(M56) of your amateur radio activities so that detailed guidance may be
furnished to you.

\subsection{Unofficial Foreign Travel}

In order to further protect sensitive information from possible compromise
resulting from terrorism, coercion, interrogation or capture of Agency
personnel by hostile nations and/or terrorist groups, the Agency has
established certain policies and procedures concerning unofficial foreign
travel.

All Agency personnel (civilian employees, military assignees, and contractors)
who are planning unofficial foreign travel must have that travel approved by
submitting a proposed itinerary to the Security Awareness Division (M56) at
least 30 working days prior to their planned departure from the United States.
Your itinerary should be submitted on Form K2579 (Unofficial Foreign Travel
Request).  This form provides space for noting the countries to be visited,
mode of travel, and dates of departure and return.  Your immediate supervisor
must sign this form to indicate whether or not your proposed travel poses a
risk to the sensitive information, activities, or projects of which you may
have knowledge due to your current assignment.

After your supervisor's assessment is made, this form should be forwarded to
the Security Awareness Director (M56).  Your itinerary will then be reviewed in
light of the existing situation in the country or countries to be visited, and
a decision for approval or disapproval will be based on this assessment.  The
purpose of this policy is to limit the risk of travel to areas of the world
where a threat may exist to you and to your knowledge of classified Agency
activities.

In this context, travel to communist-controlled and other hazardous activity
areas is prohibited.  A listing of these hazardous activity areas is
prohibited.  A listing of these hazardous activity areas can be found in Annex
A of NSA/CSS Regulation No. 30-31, "Security Requirements for Foreign Travel"
(12 June 1987).  From time to time, travel may also be prohibited to certain
areas where the threat from hostile intelligence services, terrorism, criminal
activity or insurgency poses an unacceptable risk to Agency employees and to
the sensitive information they possess.  Advance travel deposits made without
prior agency approval of the proposed travel may result in financial losses by
the employee should the travel be disapproved, so it is important to obtain
approval prior to committing yourself financially.  Questions regarding which
areas of the world currently pose a threat should be directed to the Security
Awareness Division (M56).

Unofficial foreign travel to Canada, the Bahamas, Bermuda, and Mexico does not
require prior approval, however, this travel must still be reported using Form
K2579.  Travel to these areas may be reported after the fact.

While you do not have to report your foreign travel once you have ended your
affiliation with the Agency, you should be aware that the risk incurred in
travelling to certain areas, from a personal safety and/or counterintelligence
standpoint, remains high.  The requirement to protect the classified
information to which you have had access is a lifetime obligation.

\subsection{Membership In Organizations}

Within the United States there are numerous organizations with memberships
ranging from a few to tens of thousands.  While you may certainly participate
in the activities of any reputable organization, membership in any
international club or professional organization/activity with foreign members
should be reported through your Staff Security Officer to the Clearance
Division (M55).  In most cases there are no security concerns or threats to
our employees or affiliates.  However, the Office of Security needs the
opportunity to research the organization and to assess any possible risk to
you and the information to which you have access.

In addition to exercising prudence in your choice of organizational
affiliations, you should endeavor to avoid participation in public activities
of a conspicuously controversial nature because such activities could focus
undesirable attention upon you and the Agency.  NSA employees may, however,
participate in bona fide public affairs such as local politics, so long as such
activities do not violate the provisions of the statutes and regulations which
govern the political activities of all federal employees.  Additional
information may be obtained from your Personnel Representative.

\subsection{Changes In Marital Status/Cohabitation/Names}

All personnel, either employed by or assigned to NSA, must advise the Office of
Security of any changes in their marital status (either marriage or divorce),
cohabitation arrangements, or legal name changes.  Such changes should be
reported by completing NSA Form G1982 (Report of Marriage/Marital Status
Change/Name Change), and following the instructions printed on the form.

\subsection{Use And Abuse Of Drugs}

It is the policy of the National Security Agency to prevent and eliminate the
improper use of drugs by Agency employees and other personnel associated with
the Agency.  The term "drugs" includes all controlled drugs or substances
identified and listed in the Controlled Substances Act of 1970, as amended,
which includes but is not limited to:  narcotics, depressants, stimulants,
cocaine, hallucinogens ad cannabis (marijuana, hashish, and hashish oil).
The use of illegal drugs or the abuse of prescription drugs by persons employed
by, assigned or detailed to the Agency may adversely affect the national
security; may have a serious damaging effect on the safety and the safety of
others; and may lead to criminal prosecution.  Such use of drugs either within
or outside Agency controlled facilities is prohibited.

\subsection{Physical Security Policies}

The physical security program at NSA provides protection for classified
material and operations and ensures that only persons authorized access to the
Agency's spaces and classified material are permitted such access.  This
program is concerned not only with the Agency's physical plant and facilities,
but also with the internal and external procedures for safeguarding the
Agency's classified material and activities.  Therefore, physical security
safeguards include Security Protective Officers, fences, concrete barriers,
access control points, identification badges, safes, and the
compartmentalization of physical spaces.  While any one of these safeguards
represents only a delay factor against attempts to gain unauthorized access to
NSA spaces and material, the total combination of all these safeguards
represents a formidable barrier against physical penetration of NSA.  Working
together with personnel security policies, they provide "security in depth."

The physical security program depends on interlocking procedures.  The
responsibility for carrying out many of these procedures rests with the
individual.  This means you, and every person employed by, assign, or detailed
to the Agency, must assume the responsibility for protecting classified
material.  Included in your responsibilities are:  challenging visitors in
operational areas; determining "need-to-know;" limiting classified
conversations to approved areas; following established locking and checking
procedures; properly using the secure and non-secure telephone systems;
correctly wrapping and packaging classified data for transmittal; and placing
classified waste in burn bags.

\subsection{The NSA Badge}

Even before you enter an NSA facility, you have a constant reminder of
security--the NSA badge.  Every person who enters an NSA installation is
required to wear an authorized badge.  To enter most NSA facilities your badge
must be inserted into an Access Control Terminal at a building entrance and you
must enter your Personal Identification Number (PIN) on the terminal keyboard.
In the absence of an Access Control Terminal, or when passing an internal
security checkpoint, the badge should be held up for viewing by a Security
Protective Officer.  The badge must be displayed at all times while the
individual remains within any NSA installation.

NSA Badges must be clipped to a beaded neck chain.  If necessary for the safety
of those working in the area of electrical equipment or machinery, rubber
tubing may be used to insulate the badge chain.  For those Agency personnel
working in proximity to other machinery or equipment, the clip may be used to
attach the badge to the wearer's clothing, but it must also remain attached to
the chain.

After you leave an NSA installation, remove your badge from public view, thus
avoiding publicizing your NSA affiliation.  Your badge should be kept in a
safe place which is convenient enough to ensure that you will be reminded to
bring it with you to work.  A good rule of thumb is to afford your badge the
same protection you give your wallet or your credit cards.  DO NOT write your
Personal Identification Number on your badge.

If you plan to be away from the Agency for a period of more than 30 days, your
badge should be left at the main Visitor Control Center which services your
facility.

Should you lose your badge, you must report the facts and circumstances
immediately to the Security Operations Center (SOC) (963-3371s/688-6911b) so
that your badge PIN can be deactivated in the Access Control Terminals.  In the
event that you forget your badge when reporting for duty, you may obtain a
"non-retention" Temporary Badge at the main Visitor Control Center which serves
your facility after a co-worker personally identifies your and your clearance
has been verified.

Your badge is to be used as identification only within NSA facilities or other
government installations where the NSA badge is recognized.  Your badge should
never be used outside of the NSA or other government facilities for the purpose
of personal identification.  You should obtain a Department of Defense
identification card from the Civilian Welfare Fund (CWF) if you need to
identify yourself as a government employee when applying for "government
discounts" offered at various commercial establishments.

Your badge color indicates your particular affiliation with NSA and your level
of clearance.  Listed below are explanations of the badge colors you are most
likely to see:

\begin{description}
\item [Green (*)]       Fully cleared NSA employees and certain military
                        assignees.

\item [Orange (*)]      (or Gold) Fully cleared representative of other
                        government agencies.

\item [Black (*)]       Fully cleared contractors or consultants.

\item [Blue]            Employees who are cleared to the SECRET level while
                        awaiting completion of their processing for full
                        (TS/SI) clearance.  These Limited Interim Clearance
                        (LIC) employees are restricted to certain activities
                        while inside a secure area.

\item [Red]             Clearance level is not specified, so assume the holder
                        is uncleared.
\end{description}

* - Fully cleared status means that the person has been cleared to the Top
Secret (TS) level and indoctrinated for Special Intelligence (SI).

All badges with solid color backgrounds (permanent badges) are kept by
individuals until their NSA employment or assignment ends.  Striped badges
("non-retention" badges) are generally issued to visitors and are returned to
the Security Protective Officer upon departure from an NSA facility.

\subsection{Area Control}

Within NSA installations there are generally two types of areas,
Administrative and Secure.  An Administrative Area is one in which storage of
classified information is not authorized, and in which discussions of a
classified nature are forbidden.  This type of area would include the
corridors, restrooms, cafeterias, visitor control areas, credit union, barber
shop, and drugstore.  Since uncleared, non-NSA personnel are often present in
these areas, all Agency personnel must ensure that no classified information is
discussed in an Administrative Area.

Classified information being transported within Agency facilities must be
placed within envelopes, folders, briefcases, etc. to ensure that its contents
or classification markings are not disclosed to unauthorized persons, or that
materials are not inadvertently dropped enroute.

The normal operational work spaces within an NSA facility are designated Secure
Areas.  These areas are approved for classified discussions and for the storage
of classified material.  Escorts must be provided if it is necessary for
uncleared personnel (repairmen, etc.) to enter Secure Areas, an all personnel
within the areas must be made aware of the presence of uncleared individuals.
All unknown, unescorted visitors to Secure Areas should be immediately
challenged by the personnel within the area, regardless of the visitors'
clearance level (as indicated by their badge color).

The corridor doors of these areas must be locked with a deadbolt and all
classified information in the area must be properly secured after normal
working hours or whenever the area is unoccupied.  When storing classified
material, the most sensitive material must be stored in the most secure
containers.  Deadbolt keys for doors to these areas must be returned to the key
desk at the end of the workday.

For further information regarding Secure Areas, consult the Physical Security
Division (M51) or your staff Security Officer.

\subsection{Items Treated As Classified}

For purposes of transportation, storage and destruction, there are certain
types of items which must be treated as classified even though they may not
contain classified information.  Such items include carbon paper, vu-graphs,
punched machine processing cards, punched paper tape, magnetic tape, computer
floppy disks, film, and used typewriter ribbons.  This special treatment is
necessary since a visual examination does not readily reveal whether the items
contain classified information.

\subsection{Prohibited Items}

Because of the potential security or safety hazards, certain items are
prohibited under normal circumstances from being brought into or removed from
any NSA installation.  These items have been groped into two general classes.
Class I prohibited items are those which constitute a threat to the safety and
security of NSA/CSS personnel and facilities.  Items in this category include:

\begin{enumerate}
\item Firearms and ammunition
\item Explosives, incendiary substances, radioactive materials, highly 
volatile materials, or other hazardous materials
\item Contraband or other illegal substances
\item Personally owned photographic or electronic equipment including
microcomputers, reproduction or recording devices, televisions or radios.
\end{enumerate}

Prescribed electronic medical equipment is normally not prohibited, but
requires coordination with the Physical Security Division (M51) prior to being
brought into any NSA building.

Class II prohibited items are those owned by the government or contractors
which constitute a threat to physical, technical, or TEMPEST security.
Approval by designated organizational officials is required before these items
can be brought into or removed from NSA facilities.  Examples are:

\begin{enumerate}
\item Transmitting and receiving equipment
\item Recording equipment and media
\item Telephone equipment and attachments
\item Computing devices and terminals
\item Photographic equipment and film
\end{enumerate}

A more detailed listing of examples of Prohibited Items may be obtained from
your Staff Security Officer or the Physical Security Division (M51).

Additionally, you may realize that other seemingly innocuous items are also
restricted and should not be brought into any NSA facility.  Some of these
items pose a technical threat; others must be treated as restricted since a
visual inspection does not readily reveal whether they are classified.  These
items include:

\begin{enumerate}
\item Negatives from processed film; slides; vu-graphs
\item Magnetic media such as floppy disks, cassette tapes, and VCR videotapes
\item Remote control devices for telephone answering machines
\item Pagers
\end{enumerate}

\subsection{Exit Inspection}

As you depart NSA facilities, you will note another physical security
safeguard--the inspection of the materials you are carrying.  This inspection
of your materials, conducted by Security Protective Officers, is designed to
preclude the inadvertent removal of classified material.  It is limited to any
articles that you are carrying out of the facility and may include letters,
briefcases, newspapers, notebooks, magazines, gym bags, and other such items.
Although this practice may involve some inconvenience, it is conducted in your
best interest, as well as being a sound security practice.  The inconvenience
can be considerably reduced if you keep to a minimum the number of personal
articles that you remove from the Agency.

\subsection{Removal Of Material From NSA Spaces}

The Agency maintains strict controls regarding the removal of material from its
installations, particularly in the case of classified material.

Only under a very limited and official circumstances classified material be
removed from Agency spaces.  When deemed necessary, specific authorization is
required to permit an individual to hand carry classified material out of an
NSA building to another Secure Area.  Depending on the material and
circumstances involved, there are several ways to accomplish this.

A Courier Badge authorizes the wearer, for official purposes, to transport
classified material, magnetic media, or Class II prohibited items between NSA
facilities.  These badges, which are strictly controlled, are made available by
the Physical Security Division (M51) only to those offices which have specific
requirements justifying their use.

An Annual Security Pass may be issued to individuals whose official duties
require that they transport printed classified materials, information storage
media, or Class II prohibited items to secure locations within the local area.
Materials carried by an individual who displays this pass are subject to spot
inspection by Security Protective Officers or other personnel from the Office
of Security.  It is not permissible to use an Annual Security Pass for personal
convenience to circumvent inspection of your personal property by perimeter
Security Protective Officers.

If you do not have access to a Courier Badge and you have not been issued an
Annual Security Pass, you may obtain a One-Time Security Pass to remove
classified materials/magnetic media or admit or remove prohibited items from an
NSA installation.  These passes may be obtained from designated personnel
in your work element who have been given authority to issue them.  The issuing
official must also contact the Security Operations Center (SOC) to obtain
approval for the admission or removal of a Class I prohibited item.

When there is an official need to remove government property which is not
magnetic media, or a prohibited or classified item, a One-Time Property Pass is
used.  This type of pass (which is not a Security Pass) may be obtained from
your element custodial property officer.  A Property Pass is also to be used
when an individual is removing personal property which might be reasonably be
mistaken for unclassified Government property.  This pass is surrendered to the
Security Protective Officer at the post where the material is being removed.
Use of this pass does not preclude inspection of the item at the perimeter
control point by the Security Protective Officer or Security professionals to
ensure that the pass is being used correctly.

\subsection{External Protection Of Classified Information}

On those occasions when an individual must personally transport classified
material between locations outside of NSA facilities, the individual who is
acting as the courier must ensure that the material receives adequate
protection. Protective measures must include double wrapping and packaging of
classified information, keeping the material under constant control, ensuring
the presence of a second appropriately cleared person when necessary, and
delivering the material to authorized persons only.  If you are designated as a
courier outside the local area, contact the Security Awareness Division (M56)
for your courier briefing.

Even more basic than these procedures is the individual security responsibility
to confine classified conversations to secure areas.  Your home, car pool, and
public places are not authorized areas to conduct classified discussions--even
if everyone involved in he discussion possesses a proper clearance and
"need-to-know."  The possibility that a conversation could be overheard by
unauthorized persons dictates the need to guard against classified discussions
in non-secure areas.

Classified information acquired during the course of your career or assignment
to NSA may not be mentioned directly, indirectly, or by suggestion in personal
diaries, records, or memoirs.

\subsection{Reporting Loss Or Disclosure Of Classified Information}

The extraordinary sensitivity of the NSA mission requires the prompt reporting
of any known, suspected, or possible unauthorized disclosure of classified
information, or the discovery that classified information may be lost, or is
not being afforded proper protection.  Any information coming to your
attention concerning the loss or unauthorized disclosure of classified
information should be reported immediately to your supervisor, your Staff
Security Officer, or the Security Operations Center (SOC).

\subsection{Use Of Secure And Non-Secure Telephones}

Two separate telephone systems have been installed in NSA facilities for use in
the conduct of official Agency business:  the secure telephone system (gray
telephone) and the outside, non-secure telephone system (black telephone).  All
NSA personnel must ensure that use of either telephone system does not
jeopardize the security of classified information.

The secure telephone system is authorized for discussion of classified
information.  Personnel receiving calls on the secure telephone may assume that
the caller is authorized to use the system.  However, you must ensure that the
caller has a "need-to-know" the information you will be discussing.

The outside telephone system is only authorized for unclassified official
Agency business calls.  The discussion of classified information is not
permitted on this system.  Do not attempt to use "double-talk" in order to
discuss classified information over the non-secure telephone system.

In order to guard against the inadvertent transmission of classified
information over a non-secure telephone, and individual using the black
telephone in an area where classified activities are being conducted must
caution other personnel in the area that the non-secure telephone is in use.
Likewise, you should avoid using the non-secure telephone in the vicinity of a
secure telephone which is also in use.

\section{HELPFUL INFORMATION}

\subsection{Security Resources}

In the fulfillment of your security responsibilities, you should be aware that
there are many resources available to assist you.  If you have any questions or
concerns regarding security at NSA or your individual security
responsibilities, your supervisor should be consulted.  Additionally, Staff
Security Officers are appointed to the designated Agency elements to assist
these organizations in carrying out their security responsibilities.  There is
a Staff Security Officer assigned to each organization; their phone numbers are
listed at the back of this handbook.  Staff Security Officers also provide
guidance to and monitor the activities of Security Coordinators and Advisors
(individuals who, in addition to their operational duties within their
respective elements, assist element supervisors or managers in discharging
security responsibilities).

Within the Office of Security, the Physical Security Division (M51) will offer
you assistance in matters such as access control, security passes, clearance
verification, combination locks, keys, identification badges, technical
security, and the Security Protective Force.  The Security Awareness Division
(M56) provides security guidance and briefings regarding unofficial foreign
travel, couriers, special access, TDY/PCS, and amateur radio activities.  The
Industrial and Field Security Division (M52) is available to provide security
guidance concerning NSA contractor and field site matters.

The Security Operations Center (SOC) is operated by two Security Duty Officers
(SDOs), 24 hours a day, 7 days a week.  The SDO, representing the Office of
Security, provides a complete range of security services to include direct
communications with fire and rescue personnel for all Agency area facilities.
The SDO is available to handle any physical or personnel problems that may
arise, and if necessary, can direct your to the appropriate security office
that can assist you.  After normal business hours, weekends, and holidays, the
SOC is the focal point for all security matters for all Agency personnel and
facilities (to include Agency field sites and contractors).  The SOC is located
in Room 2A0120, OPS 2A building and the phone numbers are 688-6911(b),
963-3371(s).

However, keep in mind that you may contact any individual or any division
within the Office of Security directly.  Do not hesitate to report any
information which may affect the security of the Agency's mission, information,
facilities or personnel.

\subsection{Security-Related Services}

In addition to Office of Security resources, there are a number of
professional, security-related services available for assistance in answering
your questions or providing the services which you require.

The Installations and Logistics Organization (L) maintains the system for the
collection and destruction of classified waste, and is also responsible for the
movement and scheduling of material via NSA couriers and the Defense Courier
Service (DCS).  Additionally, L monitors the proper addressing, marking, and
packaging of classified material being transmitted outside of NSA; maintains
records pertaining to receipt and transmission of controlled mail; and issues
property passes for the removal of unclassified property.

The NSA Office of Medical Services (M7) has a staff of physicians, clinical
psychologists and an alcoholism counselor.  All are well trained to help
individuals help themselves in dealing with their problems.  Counseling
services, with referrals to private mental health professionals when
appropriate, are all available to NSA personnel.  Appointments can be obtained
by contacting M7 directly.  When an individual refers himself/herself, the
information discussed in the counseling sessions is regarded as privileged
medical information and is retained exclusively in M7 unless it pertains to the
national security.

Counselling interviews are conducted by the Office of Civilian Personnel (M3)
with any civilian employee regarding both on and off-the-job problems.  M3 is
also available to assist all personnel with the personal problems seriously
affecting themselves or members of their families.  In cases of serious
physical or emotional illness, injury, hospitalization, or other personal
emergencies, M3 informs concerned Agency elements and maintains liaison with
family members in order to provide possible assistance.  Similar counselling
services are available to military assignees through Military Personnel (M2).

\section{A FINAL NOTE}

The information you have just read is designed to serve as a guide to assist
you in the conduct of your security responsibilities.  However, it by no means
describes the extent of your obligation to protect information vital to the
defense of our nation.  Your knowledge of specific security regulations is part
of a continuing process of education and experience.  This handbook is designed
to provide he foundation of this knowledge and serve as a guide to the
development of an attitude of security awareness.

In the final analysis, security is an individual responsibility.  As a
participant in the activities of the National Security Agency organization, you
are urged to be always mindful of the importance of the work being accomplished
by NSA and of the unique sensitivity of the Agency's operations.


\end{document}

