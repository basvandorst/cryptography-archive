\documentstyle[a4]{article}

\begin{document}

\begin{titlepage}
\begin{center}

\vspace*{5cm}

\newfont{\HUGE}{cmr17 scaled 2074}

{\HUGE S F S}
\vspace*{2cm}

{\Large                  Secure FileSystem}

\vspace*{1cm} {\large      Version 1.10}
\vspace*{4cm}

                     \copyright{} Peter C. Gutmann  1993, 1994

\vspace*{4cm}
{\small \LaTeX-version by Peter Conrad}

\end{center}
\end{titlepage}

\parindent 0pt
\parskip 2mm

{\large      ``The right to privacy \dots\/ is the most comprehensive of rights and the
       right most valued by civilized man''

      \begin{flushright} -- Justice Louis Brandeis, US Supreme Court, 1928
      \end{flushright}}

\pagebreak

\tableofcontents

\section{Introduction}

Ever since Julius Caesar used the cipher which now bears his name to try to
hide his military dispatches from prying eyes, people have been working on
various means to keep their confidential information private.  Over the years,
the art of cryptography (that is, of scrambling information so only those in
possession of the correct password can unscramble it) has progressed from
simple pencil-and-paper systems to more sophisticated schemes involving complex
electromechanical devices and eventually computers.  The means of breaking
these schemes has progressed on a similar level.  Today, with the
ever-increasing amount of information stored on computers, good cryptography is
needed more than ever before.

There are two main areas in which privacy protection of data is required:

\begin{itemize}
  \item Protection of bulk data stored on disk or tape.
  \item Protection of messages sent to others.
\end{itemize}

SFS is intended to solve the problem of protecting bulk data stored on disk.
The protection of electronic messages is best solved by software packages such
as PGP (available on sites the world over) or various implementations of PEM
(currently available mainly in the US, although non-US versions are beginning
to appear).

SFS has the following features:

\begin{itemize}

\item The current implementation runs as a standard DOS device driver, and
    therefore works with both plain MSDOS or DRDOS as well as other
    software such as Windows, QEMM, Share, disk cacheing software, Stacker,
    JAM, and so on.

\item Up to five encrypted volumes can be accessed at any one time, chosen 
    from a selection of as many volumes as there is storage for.

\item Volumes can be quickly unmounted with a user-defined hotkey, or
    automatically unmounted after a certain amount of time.  They can also be
    converted back to unencrypted volumes or have their contents destroyed if
    required.

\item The software contains various stealth features to minimise the
    possibility of other programs monitoring or altering its operation.

\item The encryption algorithms used have been selected to be free from any
    patent restrictions, and the software itself is not covered by US export
    restrictions as it was developed entirely outside the US (although once a
    copy is sent into the US it can't be re-exported).

\item SFS complies with a number of national and international data
    encryption standards, among them ANSI X3.106, ANSI X9.30 Part 2,
    Federal Information Processing Standard (FIPS) 180, Australian
    Standard 2805.5.2, ISO 10116:1991 and ISO 10126-2:1991, and is on
    nodding terms with a several other relevant standards.

\item The documentation includes fairly in-depth analyses of various security
    aspects of the software, as well as complete design and programming details
    necessary to both create SFS-compatible software and to verify the
    algorithms used in SFS.

\item The encryption system provides reasonable performance.  One tester has
    reported a throughput of 250 K/s for the basic version of SFS, and 260 K/s
    for the 486+ version on his 486 system, when copying a file with the DOS
    copy command from one location on an SFS volume to another.  Throughput on
    a vanilla 386 system was reported at around 160 K/s.

\item Direct access to IDE and SCSI drives is available for better
    performance and for drives which aren't normally accessible to DOS (for
    example systems with more than 2 hard drives).

\end{itemize}

Although the use of DOS is described throughout this document, SFS is not
limited to any particular operating environment, and can be used to contain
virtually any type of filesystem.  In the future an SFS driver for OS/2 HPFS
filesystems may be developed, and there have been discussions on creating a
Linux SFS driver for Unix machines.  A 68000 version of SFS is also reported to
be under development.


\section{Overview}

This document is organised to give step-by-step instructions on setting up the
SFS driver, creating an encrypted volume, and using the encrypted volume to
store information securely.  The first three sections cover each of these
steps, with a special quick-start section preceding them giving a rapid
introduction to getting an encrypted disk volume up and running.  The next
sections provide extra details on topics such as password management,
incompatibility problems, other encryption software, and the politics of
cryptography and privacy.  The final sections provide an in-depth security
analysis, technical information on the SFS driver, and data formats for those
wishing to write SFS-compatible software or wanting to check the security of
the software for themselves.

The document is divided into sections as follows:

\begin{itemize}

\item Why Use SFS?

        Some reasons why use of security measures like SFS may be
        necessary for your data.

\item Terminology

        An explanation of some of the technical terms use in this
        document.  Experienced users can skip this section.

\item Quick Start

        A quick overview of the use of SFS which summarizes the
        next three sections for people in a hurry

\item Loading the SFS Driver

        How to set up the SFS driver needed to handle encrypted
        volumes.

\item Creating an SFS Volume

        How to prepare an SFS encrypted volume for use.

\item Mounting an SFS Volume

        How to mount a previously prepared SFS encrypted volume
        so the operating system can use it.

\item Advanced SFS Driver Options

        Various advanced options such as how to mount SFS volumes
        at system startup so that they are automatically available when
        the system is booted, and customizing the SFS driver operation
        and user interface.

\item Changing the Characteristics of an SFS Volume

        How to change various characteristics of an SFS volume such
        as the password, volume name, disk access method, and 
        auto-unmount timeout, and how to delete SFS volumes or convert 
        them back to normal DOS volumes.

\item Sharing SFS Volumes Between Multiple Users

        How to securely share a single encrypted SFS volume between
        multiple users.

\item Creating Compressed SFS Volumes

        How to create a compressed drive inside a normal SFS volume

\item WinSFS - Using SFS with Windows

        An overview of the Windows version of SFS.

\item Command Summary

        A summary of the commands available with the various SFS
        programs.

\item Incompatibilities

        Comments on unusual hardware and software combinations which
        may create problems for SFS.

\item Authentication of SFS Software

        How to verify that the SFS distribution you have is indeed the
        real thing.

\item Applications

        Various applications and uses for SFS.

\item The Care and Feeding of Passwords

        Details on how to chose and handle a password to protect
        an SFS volume.

\item Other Software

        An overview of other available security software and the
        weakness and problems present in it.

\item Data Security

        Various issues in data security which should be taken into
        consideration when using SFS and similar encryption software.

\item Politics

        A discussion on the politics of cryptography, the right to
        privacy, and some of the reasons why SFS was written.

\item An Introduction to Encryption Systems

        A brief introduction to encryption systems with an emphasis
        on the methods used in SFS.

\item Security Analysis

        An analysis of the level of security offered by SFS and
        some possible attacks on it.

\item Design Details

        Various in-depth design details not covered in the security
        analysis.

\item SFS Disk Volume Layout

        Details on the disk layout used by SFS.

\item Interfacing with SFS

        How to control the SFS driver through software.

\item Interfacing with mountsfs

        How to control the mountsfs program from external software such
        as graphical front-ends.

\item Selected Source Code

        A walkthrough of selected portions of the source code to allow
        verification and help implementors.

\item Future Work

        Various enhancements which may be incorporated into future
        versions of SFS.

\item Recommended Reading

        A short list of recommended reading material for those wishing
        to know more about the design of SFS and encryption in general.

\item Using SFS

        Conditions and terms for use of SFS.

\item Credits
\item Warranty

\end{itemize}

\section{Why Use SFS?}
Virtually all information stored on computer systems is sensitive to some
degree, and therefore worth protecting.  Exactly how sensitive a piece of data
is is unique to each environment.  In some cases the data may be much more
sensitive to errors or omissions, or to unavailability, or to fraudulent
manipulation, than to the problems SFS is designed to guard against.  SFS helps
guard against data being disclosed to the wrong people or organisations, and
against some types of fraudulent manipulation.  By making the data being
protected accessible only to those with authorized access, SFS helps protect
the confidentiality of the information, and the privacy of the individuals the
information pertains to.  Preventing access by unauthorized users also helps to
protect the integrity of the data\footnote{
%Footnote [1]: 
		Although inadvertent modification by authorized users is still
              	possible, the risk from deliberate compromise of the data is
              	greatly reduced.
}.

One way to determine whether the data is sensitive enough to require the use of
SFS is to consider the following:

\begin{itemize}
\item  What are the consequences of the data being made available to the wrong
  people or organisations?

\item  What are the consequences of the data being manipulated for fraudulent
  purposes?
\end{itemize}

An additional impetus for security comes from the legal requirement of many
countries for individuals and organisations to maintain the confidentiality of
the information they handle, or to control their assets (such as computer data)
properly.  For example, one of the ``OECD guidelines governing the protection of
privacy and transborder flows of computer data'' states that data should be
protected against ``loss or unauthorized access, destruction, use, modification,
or disclosure''\footnote{
%Footnote [2]: 
		These guidelines are discussed in more detail in ``Computer
              	Networks'', Volume 5, No.2 (April 1981).               
}.  An example of the requirements for the control of assets is
the US Foreign Corrupt Practices Act of 1977.

In summary, if the cost of damage or disclosure of the data is more than the
cost of using a security measure such as SFS (where cost is measured not only
in monetary terms but also in terms of damage to business and loss of privacy)
then the data should be regarded as being sensitive and should have adequate
security controls to prevent or lessen the potential loss.

%Footnote [1]: Although inadvertent modification by authorized users is still
%              possible, the risk from deliberate compromise of the data is
%              greatly reduced.

%Footnote [2]: These guidelines are discussed in more detail in "Computer
%              Networks", Volume 5, No.2 (April 1981).               


\section{Terminology}

Throughout this document a number of specialised terms are used to describe the
operation of the SFS encryption software.  This section provides a brief
explanation of the terms used.  Experienced users can skip this material and go
directly to the ``Loading the SFS Driver'' section below.

\begin{description}

\item[Disk volume]

    An individual logical disk drive, volume, partition, or filesystem.  A
    single physical hard disk can (and usually does) contain more than one
    volume on it.  Under DOS, each of these volumes is assigned its own drive
    letter and appears as a separate drive, even though they all reside on the
    same physical hard disk.  Thus a system might have a single 128MB hard disk
    which contains four 32MB volumes accessed by the drive letters C:, D:, E:,
    and F:.

    This system is rather confusing and dates back fifteen to twenty years. SFS
    refers to these volumes by name rather than an arbitrary letter, so that
    the volumes might be called ``Encrypted data'', ``Personal correspondence'', or
    ``Accounts receivable, March 1993''.  Unfortunately once SFS has set up the
    volume for DOS to access, it's back to the old F: to identify your data.

\item[Password, key]

    The password or encryption key is used to protect the data on an encrypted
    volume.  Despite its name, a password can (and should) be more than just a
    single word.  The SFS software will accept up to 100 characters of
    password, so that perhaps the term ``passphrase'' would be more appropriate.

    For maximum security, each volume should be protected by its own unique
    password.  The SFS software takes the password for a volume, adds extra
    keying information to it, and converts the result into an encryption key
    which is used to encrypt and decrypt data on a given volume.  Great care
    should be taken in the choice of passwords and in keeping them secret. More
    details on this are given in the section ``The Care and Feeding of
    Passwords'' below.

\item[Device driver]

    A device driver is a special piece of software which is used by the
    operating system to access hardware which it wasn't designed to.  Unless
    the device driver is loaded, the operating system generally won't recognise
    that a piece of hardware even exists.  Even the computer's monitor,
    keyboard, and disk drives are accessed through device drivers, although
    their presence is hidden by the operating system.

    An example of a visible device driver is the one used to handle a mouse.
    Networked disk drives may be accessed through a device driver\footnote{
%Footnote [1]: 
		Actually they use a specialised kind of driver called a network
              	redirector.
    }.  RAM disks
    are implemented as device drivers.  CDROM drives are handled via a device
    driver.  Finally, encrypted SFS volumes are accessed through a device
    driver.

\item[Mount point]

    The locations provided by the SFS driver for mounting encrypted volumes---in
    other words the number of encrypted volumes which can be accessed by the
    driver at any one time.  By default the driver provides one mount point,
    which means one encrypted volume can be accessed through it at any given
    time.  The exact number of mount points can be specified when the SFS
    driver is loaded.

\end{description}

\section{Quick Start}

This section contains a condensed version of the next three sections and allows
a quick start for SFS.  Although it is recommended that the full text be read,
it should be possible to install and use a minimal SFS setup using only the
quick-start information.

Initially, the SFS driver must be loaded by adding an entry for it to the
CONFIG.SYS file.  For example if the SFS.SYS driver was located in the DOS
directory on drive C: the following line should be added to the CONFIG.SYS
file:

{\tt \verb|  |DEVICE=C:$\backslash$DOS$\backslash$SFS.SYS}

Alternatively, the DEVICEHIGH option can be used to load the driver into high
memory under those versions of DOS which support it.  The system should now be
rebooted to make sure the driver is installed.

The use of the SFS driver is covered in more detail in the sections ``Loading
the SFS Driver'' and ``Advanced SFS Driver Options'' below.

The encrypted volume can be created with the {\tt mksfs}
program.  This is run with the letter of the drive to encrypt, and the name of
the encrypted volume preceded by the {\tt vol=} option as arguments.  For 
example to encrypt the
E: drive to create a volume with the name ``Encrypted disk'', the command would
be:

{\tt \verb|  |mksfs "vol=Encrypted volume" e:}

Note that that {\tt "vol=\dots"} option is quoted, as the volume name contains a space.
Volume names without a space don't need to be quoted.

mksfs will confirm that the given drive is indeed the one to be encrypted, and
then ask for an encryption password of between 10 and 100 characters.  After
asking for the password a second time to confirm it, it will encrypt the drive.
This will take a few minutes, and the program will display a progress bar as
the encryption takes place.

There are a great many options and special safety checks built into mksfs to
ensure no data is accidentally destroyed, and it is recommended that the
section ``Creating an SFS Volume'' be at least glanced through to provide an
overview of the functioning of mksfs before it is run.

Once the encrypted volume has been created and the SFS driver loaded, it can be 
mounted with the {\tt mountsfs} utility.  Mounting a volume makes it available 
to DOS as a normal
disk volume, with all encryption being done transparently by the SFS driver.
Like mksfs, mountsfs must be told the encrypted volume's name in order to
access it.  The full name doesn't need to be used, mountsfs will accept any
part of the name in upper or lower case.  Using the name from the previous
example, the command to mount the volume would be:

{\tt \verb|  |mountsfs vol=encrypt}

mountsfs will match the partial name ``encrypt'' with the full volume name
``Encrypted volume'', ask for the encryption password for the volume, and mount
it.  The volume will now be accessible as a normal DOS drive.

More details on the use of mountsfs are contained in the section ``Mounting an
SFS Volume'' below.  Other methods for mounting volumes are given in the section
``Advanced SFS Driver Options'' below.


\section{Loading the SFS Driver}

The SFS device driver {\tt SFS.SYS} or {\tt SFS486.SYS} can be loaded in the usual manner
by specifying it in the CONFIG.SYS file:

{\tt
\verb|  |DEVICE=[$drive$:][$path$]SFS.SYS [SILENT] [UNITS=$n$] [NOXMS]\\
\verb|                              |[PROMPT=$xxx$] [READONLY] [READWRITE]\\
\verb|                              |[FIXED] [REMOVABLE] [ECHO] [FAST=$n$]\\
\verb|                              |[HOTKEY=$xxx$] [TIMEOUT=$nn$] [MOUNT=$nnn$]}

It can also be loaded high under those versions of DOS which support this with:

{\tt
\verb|  |DEVICEHIGH=[$drive$:][$path$]SFS.SYS [SILENT] [UNITS=$n$] [NOXMS]\\
\verb|                                  |[PROMPT=$xxx$] [READONLY]\\
\verb|                                  |[READWRITE] [FIXED] [REMOVABLE]\\
\verb|                                  |[ECHO] [FAST=$n$] [HOTKEY=$xxx$]\\
\verb|                                  |[TIMEOUT=$nn$] [MOUNT=$nnn$]}

The SFS486.SYS driver is loaded the same way.  This driver contains code for
'486 and higher processors, and is slightly smaller and a few percent faster
than the equivalent '386 version.

The arguments to SFS are not case-sensitive, and can be given in upper or lower
case.  They may also be optionally preceded by a `/' for compatibilty with
older types of software.  For example if your copy of the SFS.SYS driver was 
located in the DOS directory on drive C: you would add the following line to 
your CONFIG.SYS file:

{\tt \verb|  |DEVICE=C:$\backslash$DOS$\backslash$SFS.SYS}

The driver will only work on systems with an 80386 or higher processor.  This
is because the en/decryption code (over 10,000 lines of assembly language) has
to have a 32-bit processor to run on.  Virtually all recent PC's fulfil these
requirements, and a 16-bit version would both be much slower and require about
three times as much code space to run in\footnote{
%Footnote [1]:
              There have been calls for 286 versions of SFS from countries in
              which 386+ machines are still difficult to obtain.  There may
              eventually be a 16-bit version, although at the current rate by
              the time it's written everyone will be using Pentiums anyway.
}.

If an attempt is made to load SFS.SYS on a machine which doesn't have a 32-bit
CPU, the message:

{\tt \verb|  |Error: Processor must be 386 or higher}

will be displayed and SFS will de-install itself.

The drive currently recognises thirteen options, {\tt ECHO}, {\tt FAST}, {\tt FIXED}, {\tt HOTKEY},
{\tt MOUNT}, {\tt NOXMS}, {\tt PROMPT}, {\tt READONLY}, {\tt READWRITE}, {\tt REMOVABLE}, {\tt SILENT}, {\tt TIMEOUT}, and 
{\tt UNITS}:

\begin{itemize}

\item The {\tt ECHO} option is used in conjunction with the {\tt MOUNT} option to echo the
  password to the screen when asking for the password for the SFS volume to be
  mounted, and is explained in more detail in the section ``Advanced SFS
  Driver Options'' below.

\item The {\tt FAST} option is used in conjunction with the {\tt MOUNT} option to enable
  various high-speed direct disk access modes in the SFS driver.  These can
  significantly affect the overall performance of the driver, and are discussed
  in more detail in the section ``Advanced SFS Driver Options'' below.

\item The {\tt FIXED} option is used in conjunction with the {\tt MOUNT} option to indicate
   that a volume mounted at system startup is to be kept mounted until the
   system is turned off or reset, as opposed to the normal behaviour of allowing
   it to be unmounted at any point.  This is discussed in more detail in the
   section ``Advanced SFS Driver Options'' below.

\item The {\tt HOTKEY} option is used to specify the quick-unmount hotkey which can be
  used to instantly unmount all currently mounted SFS volumes, and is explained
  in more detail in the sections ``Mounting an SFS Volume'' and ``Advanced SFS
  Driver Options'' below.

\item The {\tt MOUNT} option is used to mount SFS volumes at system startup, and is
  explained in more detail in the section ``Advanced SFS Driver Options'' below.
  The older {\tt AUTOMOUNT} form of this command is still supported by this version
  of SFS, but will be discontinued in future versions.

\item The {\tt NOXMS} option is used to disable SFS buffering data in extended memory.
  By default SFS will allocate a 64K write buffer to speed up disk writes.  If
  no extended memory is available or if the {\tt NOXMS} option is used, SFS will
  print:

  {\tt \verb|  |Warning: No XMS buffers available, slow writes will be used}

  The driver will then switch to using slow disk writes which are about half as
  fast as normal reads and writes.  These are necessary to fix buffering
  problems in MSDOS 6.x and some disk utilities.  If an extended memory buffer
  is used, the slow writes aren't necessary.

\item The {\tt PROMPT} option is used in conjunction with the {\tt MOUNT} option to display a
  user-defined prompt when asking for the password for the SFS volume to be 
  mounted, and is explained in more detail in the section ``Advanced SFS Driver
  Options'' below.

\item The {\tt READONLY} and {\tt READWRITE} options are used in conjunction with the {\tt MOUNT}
  option to disable write access to the volumes being mounted. The {\tt READONLY}
  option disables write access to all following mounted volumes; the {\tt READWRITE}
  option enables write access to all following mounted volumes.  The default
  setting is to allow read and write access to all volumes.
  More details on read-only access to SFS volumes is given in the section ``Mounting an SFS
  Volume'' below.

\item The {\tt REMOVABLE} option is used to undo the effects of the {\tt FIXED} option which is
  explained above.


\item The {\tt SILENT} option can be used to suppress the printing of the start-up
  message.

\item The {\tt TIMEOUT} option is used to specify the time in minutes after which SFS
  volumes are automatically unmounted if they haven't been accessed during that
  time, and is explained in more detail in the sections ``Mounting an SFS
  Volume'' and ``Advanced SFS Driver Options'' below.

\item The {\tt UNITS=$n$} option specifies the number of mount points (or number of disk
  volumes) the driver will provide, where $n$ is the number of units and can
  range from 1 to 5.  Each drive mount point requires 384 bytes of extra memory
  storage.  By default, the driver allocates storage for one mount point.

\end{itemize}

As an example, to suppress the printing of the start-up message and to specify
that the driver should handle up to three encrypted volumes, the previously
given example for loading the driver would be changed to:

{\tt \verb|  |DEVICE=C:$\backslash$DOS$\backslash$SFS.SYS SILENT UNITS=3}

The number of mount points can range from 1 to 5.  If a number outside this
range is specified, the message:

{\tt \verb|  |Error: Invalid number of units specified}

will be displayed and SFS will de-install itself.  Finally, if an invalid
option is given (such as a misspelled or badly-formatted parameter) SFS will
again de-install itself after displaying:

{\tt \verb|  |Error: Unknown parameter specified}

All the remaining driver options are covered in the section ``Advanced SFS
Driver Options'' below.

If the driver installs successfully and unless the {\tt SILENT} option is used it
will, after displaying a general message showing that it has been installed,
indicate which drive will be used as the encrypted one.  For example if
the encrypted drive is made available as E:, the message would be:

{\tt \verb|  |Encrypted volume will be mounted as drive E:}

This indicates that once an encrypted volume is mounted, DOS will access it as
drive E:  If more than one mount point is specified, the range of drives which
will be made available is shown, so that if the option {\tt UNITS=3} were used the
message would be:

{\tt \verb|  |Encrypted volumes will be mounted as drives E: - G:}

When installed SFS consumes around 7.5K of memory, most of which is encryption
code.

%Footnote [1]: There have been calls for 286 versions of SFS from countries in
%              which 386+ machines are still difficult to obtain.  There may
%              eventually be a 16-bit version, although at the current rate by
%              the time it's written everyone will be using Pentiums anyway.


\section{Creating an SFS Volume}

Before SFS can use an encrypted volume, it must be converted from a normal DOS
volume to an encrypted SFS one.  The program which performs this task is mksfs,
(Make Secure Filesystem) and is very loosely patterned after the Unix mkfs
utility.  mksfs takes a standard DOS volume (which may be either freshly
formatted or may already contain files) and turns in into an encrypted SFS
volume.  The encryption process is non-destructive, so in general no data will
be lost.  The only case in which a data loss could occur is if there is a power
cut while the volume is being encrypted (this means that power to the system is
removed as the disk is being written to, which would cause problems under
virtually any software).  If the data being encrypted is extremely valuable or
there is a risk of a power cut occurring, the volume should be backed up
completely before being encrypted.  This should only be necessary in
exceptional circumstances.

If used on a fixed disk, mksfs will encrypt an entire disk partition rather
than individual files.  This is necessary because an SFS partition may contain
a DOS filesystem, or an OS/2 one, or a HPFS one, or an NTFS one, or any one of
a dozen other possible filesystems.  However, many people have only a single
large partition on their hard drive which is used entirely for DOS, which would
require a complete backup of the partition before the FDISK utility can be used
to create two smaller partitions, followed by a restore of the backup to one of
the new partitions.  This problem can be avoided by using one of several
programs which will nondestructively split an existing partition into two 
smaller partitions, one of which can be used as an SFS volume\footnote{
%Footnote [1]: 
		One of these is FIPS, currently at version 1.1 and available as
             	fips11.zip from either sunsite.unc.edu in the directory
              	/pub/Linux/system/Install, tsx-11.mit.edu in the directory
              	/pub/linux/dos\_utils, garbo.uwasa.fi and all mirror sites in the
                directory /pc/diskutil, or oak.oakland.edu and all mirror sites
                in the directory simtel/msdos/diskutil.
}.

If the hardware or software setup you are using is somewhat unusual (for
example you have drives which are compressed with DoubleSpace, Stacker, or JAM,
or you have unusual drive hardware which needs special software like SpeedStor 
to manage it), you should read the section ``Incompatibilities'' below.  In
addition, mksfs may, during normal operation, trigger a number of virus
detectors which monitor access to certain critical disk and memory areas which
software would not normally access.  Finally, mksfs will check whether it is
being run under Quarterdeck's DesqView or Microsoft Windows, as it should in
general not be run while DesqView, Windows, or some other multitasking software
is running.  Since mksfs takes an entire disk volume and encrypts it sector by
sector, any other software which tries to simultaneously access the volume
while mksfs is running will come to grief.  If mksfs detects that it is being
run under either DesqView or Windows, it will display a warning message with an
option to quit and re-run it from DOS only.  Only if there is no chance that
any other program will access the disk volume being encrypted is it safe to run
mksfs under multitasking software.

The mksfs program is run in the following manner:

{\tt \verb|  |mksfs [-c] [-o] [-t] [-e] [serial={\em serial number}] [multiuser]\\
\verb|        |[fastaccess={\em mode}] [timeout={\em timeout}] [wipe]\\
\verb|        |[vol={\em volume name}] $drive$}

Since all arguments are named, they can be given in any order.  The order shown
here is merely an example.

The {\tt -c} and {\tt -t} options are present to allow integrity checks on the SFS
encryption code and the operation of mksfs itself, and are covered in more
detail in the sections ``Incompatibilities'' and ``Security Analysis''
respectively.

The $drive$ specifies the DOS drive letter for which the SFS volume will be
created.  For example to create an SFS volume on the disk currently in the A:
drive the command would be:

{\tt \verb|  |mksfs a:}

It is recommended that each SFS volume be given a unique name for
identification purposes.  Although it is possible to create an unnamed (or
anonymous) volume, this practice is strongly discouraged for fixed disks which
may contain multiple SFS volumes.  If the volumes are anonymous then the user
has no easy way of informing SFS which one it should be accessing apart from
using the mount option with the SFS driver, which is explained in more detail 
in the section ``Advanced SFS Driver Options'' below.  mksfs will check for the 
creation of anonymous volumes on fixed disks and display a warning if this 
occurs.

The volume name can be specified with the {\tt vol=} option.  For example if the
volume name ``Secure disk volume'' was to be created on drive D: then the 
command would be:

{\tt \verb|  |mksfs "vol=Secure disk volume" d:}

Note that the volume name, which in this case contains spaces, has been quoted.
This is necessary since DOS will break apart the name into separate words if it
contains spaces.  If the name is a single word, no quoting is necessary.

The volume serial number can be specified with the {\tt serial=} option.  If no
serial number is provided, mksfs will generate one itself.  In normal usage
there is no need for the user to specify a volume serial number, but the option
has been provided in case it is needed.  If a serial number is specified, it 
should be a unique value since SFS may use it to distinguish between different 
volumes. If mksfs is left to chose the serial number it will automagically use 
a unique value.  The serial number is independant of the volume mount 
identifier, which is explained in the section ``Advanced SFS Driver Options'' 
below.  This serial number is not the same as the serial number which some
operating systems may write to a disk for their own use, and is used only by
SFS to identify volumes.

A special option for removable disks only is the {\tt -o} option.  This is
necessary because some (mostly extinct) variants of DOS treat removable disks
in a peculiar manner.  If mksfs cannot determine the disk format due to the
disk having been created with a strange DOS version, it will exit with the
error message\footnote{
%Footnote [2]:
              Certain boot sector viruses also change the information needed by
              mksfs, so mksfs printing this message may be an indication of a
              viral infection.
}:

{\small
\begin{verbatim}
  Error: Disk information reports unusual disk format, won't process disk.
         Use `-o' option to override this check.
\end{verbatim}}

If mksfs is re-run, this time with the {\tt -o} option, it will perform a check on
secondary format information stored on the disk.  If the information checks
out, it will report (assuming the disk being checked is a 1.2 MB 5 1/4'' disk):

{\small
\begin{verbatim}
  Warning: Disk information reports unusual disk format, performing check on
           secondary information...

           Disk appears to be in 1.2 MB DSHD format
\end{verbatim}}

If mksfs still can't be sure of the disk format, it will exit with an error
message.  Otherwise it will ask:

\begin{verbatim}
  Are you sure you want to process the disk in this format [y/n]
\end{verbatim}

If the given disk format is correct then a response of `Y' will continue, while
a response of `N' will exit the program.

If multiple-user access to the volume is required, the {\tt multiuser} option
should be set to enable this.  This option records extra information which may
later be edited with the adminsfs program to allow other users access to the
volume.  More details on multiuser SFS volumes are given in the section
``Sharing SFS Volumes Between Multiple Users'' below.

If the {\tt multiuser} option is used, mksfs will warn:

{\small
\begin{verbatim}
  Warning: You have specified that access to the volume for multiple users
           be enabled.  Are you sure you want to do this [y/n]
\end{verbatim}}

At this point a response of `Y' will continue and a response of `N' will exit
the program.

The SFS driver can automatically unmount volumes if they have not been accessed
for a certain amount of time.  This option is useful if there is a chance that
an interruption may call you away from a system with mounted SFS volumes
allowing others access to the encrypted data, or can simply be used as a
general safety precaution to automatically unmount the volumes after a sizeable
period of inactivity (this option is unavailable under Windows - see the
section ``Incompatibilities'' below).  However, care should be taken to allow a
large enough safety margin for the timeout, as having a volume take itself
offline five seconds before work is saved to it can be annoying.

The easiest way to set an auto-unmount timeout is to associate a timeout value
with the volume when it is created with mksfs, although this setting can be
added or an existing setting modified at a later point with the chsfs program
(this is explained in more detail in the section ``Changing the Characteristics
of an SFS Volume'' below).  When the volume is mounted, the setting of the
timeout is automatically taken care of by the SFS software.

The timeout value in minutes is specified with the use of the {\tt timeout=}
option.  For example to create the volume used in the previous example with an
auto-unmount timeout of half an hour, the command would be:

\begin{verbatim}
  mksfs "vol=Secure disk volume" timeout=30 d:
\end{verbatim}

The drive on which the volume is being created may be able to handle a
different, faster access mode than the one normally used.  SFS supports a
number of these faster access modes, which can be tested for using the {\tt mksfs
-c} option which is explained in more detail in the section ``Incompatibilities''
below.  If the tests are successful, mksfs will report the fast access mode
which can be used to access the drive.  This mode can be specified with the
{\tt fastaccess=} option when a new volume is created, and all accesses to the
volume will then use the alternative, faster method instead of the default,
somewhat slower one.  Alternatively, use of the faster access mode can be
enabled at a later date with the {\tt chsfs newaccess=} command, which is explained
in more detail in the section ``Changing the Characteristics of an SFS Volume''
below.

For example if the {\tt mksfs -c} test reported that a fast access mode of 1 was
possible, then the previous volume creation example could be changed to:

\begin{verbatim}
  mksfs "vol=Secure disk volume" fastaccess=1
\end{verbatim}

When mounted, all accesses to this volume would then be made with the specified
faster access mode.

If the volume being converted already contains files, the encryption process
will overwrite the original files with their encrypted equivalent.  However
this may not be enough to safely wipe all traces of the original data.  In
order to provide a more thorough means of overwriting it, the {\tt wipe} option may
be used to force mksfs to perform multiple overwrite passes over the original
data.  The encrypted data will not be destroyed by performing these wipes, they
simply ensure that the original unencrypted data is removed with a high degree
of certainty.

In total, 30 separate overwrite passes, which have been selected to provide the
best possible chances of destroying data for various disk encoding schemes,
will be used.  The exact details of the overwrite process, and information on
data deletion in general, is given in the section ``Deletion of SFS Volumes''
below.  This process, while very thorough, is {\bf extremely slow}.  If mksfs is 
run on large volumes with the {\tt wipe} option enabled, the encryption with 
overwrite option may take hours to run to completion.  It is recommended that 
this option only be used if the data to be encrypted is of a highly sensitive 
nature.  Use of this option is unnecessary on an unused, freshly-formatted disk 
which has never contained any data.

The program will now check to see whether the chosen volume name and serial
number conflict with the name and serial number of an existing SFS volume.  If 
both the volume name and serial number conflict, this will make future 
manipulation of the volume difficult as there is no real way to uniquely identify 
it, and mksfs will exit with the error message:

{\small
\begin{verbatim}
  Error: An SFS volume with the given name and serial number already exists.
         Either a new name or serial number should be chosen, or no serial
         number at all specified, in which case mksfs will chose a unique
         serial number for the new volume.
\end{verbatim}}

An alternative possibility, if the conflicting volume is on removable media, is
to temporarily remove the disk from the drive until mksfs has been run.
However this still creates the problem of accessing the volume in the future.
A much easier solution is to either choose unique volume names or to let mksfs
choose the volume serial number - it will always chose a number which doesn't
conflict with an existing volume serial number.

If only the volume name clashes, mksfs will warn:
{\small

\begin{verbatim}
  Warning: An SFS volume with the given name already exists.  Are you sure
           you want to create a new volume with the same name [y/n]
\end{verbatim}

}
At this point a response of `Y' will continue and a response of `N' will exit
the program.

If an anonymous volume is to be created on a fixed disk, mksfs will warn:
{\small

\begin{verbatim}
  Warning: You have not specified a name for the volume to be created.
           This may make future manipulation of the volume difficult.  Are
           you sure you want to create an anonymous volume [y/n]
\end{verbatim}

}
At this point a response of `Y' will continue and a response of `N' will exit
the program.

If it's really necessary, these checks can be overridden by using chsfs to
change the volume's characteristics after it has been created.  Unlike mksfs,
chsfs is not particular about what the volume name is set to, as it makes the 
(possibly incorrect) assumption that the user knows what they are doing.

Once the preliminary processing has been done, mksfs will, in the case of a
fixed disk, scan it for the volume which is to be converted into an SFS one.
Along the way it will perform various checks on the volume to make sure the
volume is accessible, is a standard DOS volume, is not marked as being bootable
(booting off an encrypted volume is somewhat difficult), is not the one
currently in use, and can be converted.  Note that the bootability check may
not be completely foolproof, as some disk managers\footnote{
%Footnote [3]: 
		Among them the OS/2 and Windows NT boot managers.
} perform strange tricks with
bootable volumes to handle multiple operating systems on the same disk.

mksfs performs an additional check if the volume specified for encryption is
the C: drive, which is usually the primary DOS drive and which should under
normal circumstances never be encrypted.  If an attempt to encrypt the C: drive
is made, mksfs will prompt:
{\small

\begin{verbatim}
  Warning: You have chosen to encrypt the C: drive which is usually the
           primary DOS drive and shouldn't be encrypted.  Are you sure you 
           want to do this [y/n]
\end{verbatim}

}
At this point a response of `Y' will continue and a response of `N' will exit
the program.

If the various checks succeed, it will display an informational message giving
details on the volume to be created.  An example of the information displayed
for a fixed drive might be:

\begin{verbatim}
  Volume `Encrypted disk' will be created on fixed drive D:
  This drive has a capacity of 75.2 MB and is labelled `Accounting'
  Are you sure you want to encrypt this volume [y/n]
\end{verbatim}

If the volume is the one to be converted, a response of `Y' will proceed with
the creation of the SFS volume, and a response of `N' will abort the operation.

It is vitally important that the information printed by mksfs is checked before
a `yes' response is given.  Due to the vast array of unusual disk systems,
networked drives, compressed disks, device drivers, and other strangeness, it
could be that mksfs and DOS disagree on which volume is to be encrypted.  In
addition it is very easy to specify the wrong drive accidentally when running
mksfs.  Although this situation will hopefully never occur, it is nevertheless
a good idea to stop for a second and make absolutely certain that the volume
being encrypted is the one which should be encrypted.  Treat mksfs the same way
you would treat the DOS `format' command.

For a floppy drive the information is slightly different:

{\tt \verb|  |Volume `Secure backup' will be created on 1.44MB disk in drive B:}

No yes/no prompt is given for removable disks since they contain far less
information than fixed disk volumes, and will typically be freshly-formatted,
blank diskettes.  This allows the quick bulk encryption of quantities of
diskettes without having to answer the same question for each disk.  If
necessary the encryption operation can be aborted at the password-entry stage.

mksfs will now check the volume to be encrypted for bad sectors.  Most newer
fixed disks will automatically map out bad sectors (if there are any) and use
sectors from spare space on the disk instead (all this is invisible to the
system software and is done internally by the drive itself).  However older
drives may still explicitly report bad sectors.  The presence of bad sectors on
a disk may also indicate a virus infection, or may be used by certain kinds of
(hopefully extinct) copy-protection schemes.  If mksfs finds any of these, it
will print an advisory message:

{\small \tt \verb|  |Warning: This disk contains bad sectors which won't be encrypted by SFS.}

If the disk being encrypted is a floppy disk, mksfs will print a message
recommending that another disk be used instead.  If the data is valuable enough
to need encryption, then it should really be stored on an error-free medium
rather than its loss risked with defective floppy disks:
{\small

\begin{verbatim}
  Warning: This disk contains bad sectors.  Use of damaged disks is not
           recommended as recovery of encrypted data could be difficult if
           further bad sectors develop.  Are you sure you want to encrypt
           this disk [y/n]
\end{verbatim}

}
At this point a response of `Y' will continue and a response of `N' will exit
the program.  SFS will encrypt the disk, but will skip any sectors marked as
being defective.  A similar message will be printed if any bad sectors are
found during the encryption process.  Note that if further bad sectors develop
on the floppy disk, recovery of the data stored in the bad sectors will be
difficult.  It is strongly recommended that only error-free floppy disks be
used with SFS\footnote{
%Footnote [4]:
              Although SFS has been written so that if any data does become
              corrupted, only the corrupted sector and no others will be lost,
              if data which is important to the operating system (such as a
              directory or a file allocation table) is lost, the damage may
              (just as it would for a normal non-encrypted disk) be more
              significant.  In this case any standard disk-recovery program can
              be used to make repairs, just as with a normal DOS disk.
}.

Once the disk checks have been completed, mksfs will ask for a password to use
when encrypting the volume.  This password can range in length from 10 to 100
characters, and should be made up of a complete phrase or sentence rather than
just a single word (mksfs will complain if it thinks the password is of an
insecure form and request that another one be used).  More details on choosing
a password are given in the section ``The Care and Feeding of Passwords'' below.

When asking for the password, mksfs will prompt:

{\tt \verb|  |Please enter password (10...100 characters), [ESC] to quit:}

At this point a password in the given length range can be entered.  For
security reasons the password is not echoed to the screen.  Any typing errors
when entering the password can be corrected with the backspace key.  The Esc
key can be used to quit.  The software will check for a password longer than
the maximum of 100 characters or an attempt to backspace past the start of the
password, and beep a warning when either of these conditions occur.

Once the password has been entered, mksfs will again prompt:

{\tt \verb|  |Please reenter password to confirm, [ESC] to quit:}

This confirmation is necessary to eliminate any problems with hitting an
incorrect key when entering the password the first time.  Note that every
single letter, space, and punctuation mark in the password is critical.  Making
a single mistake (getting a letter mixed up, typing a letter in upper case
instead of lower case, or missing a punctuation mark) will completely change
the encryption key.  For this reason, mksfs performs a double-check on the
password to ensure it really is the correct one.

Once the password has been entered, there is a brief delay while mksfs performs
the complex processing needed to turn it into a key suitable for the encryption
system.  When this has been completed, mksfs will begin converting the disk.
As it processes the volume, it prints a progress bar going from 0\% complete to
100\% complete.  The conversion process will take a few minutes on most disks,
and is somewhat slower than a standard disk formatting procedure which only
writes a very small amount of data to the start of the disk and scans for bad
sectors, whereas mksfs has to read, encrypt, and write the entire disk volume.

As the conversion progresses, the progress bar will gradually fill up until it
shows that the conversion is complete.  Once this has finished, mksfs will exit
with the message:

{\small \tt \verb|  |Encrypted volume created.~You can now mount it with the `mountsfs' command}

If the volume is created on a removable disk, mksfs will ask:

{\tt \verb|  |Do you wish to encrypt another disk [y/n]}

At this point a response of `Y' will continue and a response of `N' will exit
the program.  If the `Y' response is chosen, mksfs will prompt:

{\tt \verb|  |Please insert a new disk in the drive and press a key when ready}

and then repeat the disk encryption cycle.

If the volume is created on a fixed disk, DOS will still think the volume it
was created on is a DOS one rather than an encrypted SFS one.  It is strongly
recommended that you reboot your machine at this point to clear any memories of
the old volume from the system, as any attempt by DOS to access the encrypted
volume as a normal DOS volume will cause it to become very confused.  As a
reminder, mksfs will display the message:
{\small
\begin{verbatim}
  Encrypted volume created. You can now mount it with the `mountsfs' command
  or mount it at system startup with the option `MOUNT=<mount id>' in the
  CONFIG.SYS entry for the SFS driver.

  You may wish to reboot your machine to update the status of the SFS volume
  which is now inaccessible from DOS.
\end{verbatim}
}
The {\tt <mount id>} will be the ID needed to mount the encrypted volume when the 
machine is booted.  More details on automounting volumes are given in the section 
``Advanced SFS Driver Options'' below.


\section{Mounting an SFS Volume}

When the operating system first starts, it finds all disk volumes it can
recognise and automatically makes them available as different logical drive
letters.  However it can't do anything with encrypted SFS volumes, and so they
are effectively invisible to it.  In order to make them visible, they must be
mounted using the mountsfs program.  Operating systems such as Unix mount
filesystems in this manner (in fact the general feel of mountsfs is vaguely
like the Unix filesystem mount utility).

When the operating system mounts a disk volume, it uses the rather primitive
mechanism of assigning a letter of the alphabet to it and referring to the
drive by that letter.  SFS, on the other hand, refers to the volume by the name
given when the volume is created with mksfs rather than some arbitrary letter
(although volumes in removable drives can optionally be referred to by the
driver letter).  Therefore if the encrypted volume was named ``Secure disk
volume'', mountsfs would mount ``Secure disk volume'' rather than, say, ``E:''.  A
fixed disk can contain multiple encrypted volumes, mountsfs will chose the
appropriate one based on the volume name.  When searching for volumes to mount,
all fixed disks are checked before any removable disks are checked, so that a
volume with a given name on a fixed disk would override a volume of the same
name on a floppy disk.

Once the volume is mounted, DOS will still refer to it by a drive letter as
usual (there's only so much the SFS software can do), so that ``Secure disk
volume'' will, after being mounted with SFS, appear as just another DOS drive, 
for example E:.  If necessary the drive letter which SFS uses can be swapped
through the use of the JSWAP utility which comes as part of the JAM disk
compression software.  The use of JSWAP rather than the DOS commands ASSIGN,
SUBST, and JOIN, or other third-party utilities such as the one provided with
Stacker are recommended, as JSWAP provides the safest means of swapping drive
letters.  The JAM disk compression software is discussed in more detail in the
section ``Creating Compressed SFS Volumes'' below.

With removable disks it may sometimes be desirable to refer to the volume by
the drive it is in rather than the volume name.  In this case the drive can be
specified by the usual letters A: or B:, and the actual volume name will be
ignored.  As before, once the disk is mounted with SFS, the volume will appear
as another DOS drive, for example E:.  If the disk is accessed as E:, the SFS
driver will encrypt and decrypt data being written and read.  If the disk is
accessed as A: or B:, DOS will either display garbage or report a general
failure error as it doesn't understand the encrypted disk. The A: or B: drive
letters can still be used to read normal DOS disks, however.  In order to
prevent accidental overwriting of disks, the SFS driver will automatically
unmount a volume if it detects that a disk change has occurred since the last
time it accessed the drive.

The mountsfs program is run in the following manner:


{\tt \verb|  |mountsfs [+r] [+rw] [status] [unmount] [info] [information]\\
\verb|           |[hotkey=$Ctrl$-$Alt$-$LeftShift$-$RightShift$-$letter$ {\em or none}]\\
\verb|           |[timeout=$timeout$] [user={\em user name}] [userfile=[{\em user file}]\\
\verb|           |[vol={\em volume name}] [$drive$]}

Since all arguments are named, they can be given in any order.  The order shown
here is merely an example.

When mountsfs starts, it first performs a number of checks on the internal
status of the SFS driver.  If it can't find the driver, it will exit with the
error message:

{\tt  \verb|  |Error: Cannot find SFS driver}

This is due to the driver not being loaded, either because it is not being
specified in the CONFIG.SYS file, or because there was some error when it was
loaded and it de-installed itself.  More information on this is given in the
section ``Loading the SFS Driver'' above.

If the driver reports a general internal consistency check failure or a
consistency check failure for a particular drive unit (in this case drive F:),
mountsfs will exit with the error message:

{\tt  \verb|  |Error: SFS driver internal consistency check failed}

or:

{\tt  \verb|  |Error: SFS driver consistency check failed for unit F:}

A driver check failure is generally due to some other program or system
software corrupting the driver's internal state.  Possible solutions to this
problem can be found in the section ``Incompatibilities'' below.

In general the volume name would be specified with the {\tt vol=} option.  For
example if the volume name was ``Secure disk volume'' then the mount command
would be:

{\tt \verb|  |mountsfs vol=secure}

The volume name can be in upper or lower case, and the full name need not be
given.  mountsfs will match whatever part of the name is given to any SFS
volume names found until it finds a match.  The SFS volumes are checked in the
same order as they are displayed with the {\tt info} or {\tt information} command.

Alternatively, if the SFS volume to be accessed is on a removable disk, the
drive letter can be specified instead of the volume name.  For example if the
disk drive was A: then the command to mount whatever volume it contained would
be:

{\tt \verb|  |mountsfs a:}

mountsfs will not mount volumes using the mount identifier, as this is reserved 
for use with volumes mounted when the SFS driver is loaded.  More information 
on this is given in the section ``Advanced SFS Driver Options'' below.

In order to find all available SFS volumes, the {\tt info} option can be used.
This will by default search the system for available SFS volumes and print a
list of the volume name, creation date, size, and whether the volume is
currently mounted. For example on a system with two SFS volumes the output from
{\tt mountsfs info} might be:

\begin{verbatim}
    Date     Size   Type Mount status  Volume Name
  -------- -------- ---- ------------- -----------------------------
  01/11/93  Floppy  DOS    Unmounted   Data backup
  06/09/93  10.0 MB DOS  Mounted as E: Personal financial records
  12/04/93  42.5 MB DOS    Unmounted   Encrypted data disk
\end{verbatim}

This shows three SFS volumes, an unmounted volume in a floppy drive containing
backup data, a smaller one on a fixed disk containing personal financial
records which is currently mounted as drive E:, and a larger one containing
general encrypted data which is currently unmounted.  Note that removable media
is treated in a special manner and the exact disk size is indeterminate as the
media may change at any time.  The volume creation date is formatted according
to the country setting on the machine being used, so that the datestamp is
day/month/year in Europe and related countries, month/day/year in the US and
related countries, and year/month/day in Japan.  Both volumes shown here are
DOS volumes, but future versions of SFS will support other volume types such as
OS/2 HPFS, Windows NTFS, and Linux Unix ones.

If more information is desired, the longer {\tt information} form of the command
can be used.  This will display extra information such as the volume serial
number, the mount identifier (see the section ``Advanced SFS Driver Options''
below for more information), the volume filesystem type, whether multiuser
volume access is possible, what type of disk access mode is used for the
volume, the volume name character set, and the default auto-unmount timeout
value (which can be overridden when the volume is mounted if required), as well
as the other information displayed by the usual {\tt mountsfs info} command.  If, 
in the previous example, we had used {\tt mountsfs information} instead of {\tt mountsfs info}
the output might have been:
{\small
\begin{verbatim}
  Volume name  : Data backup
  Volume date  : 01/11/93, 10:13:01  Volume serial number  : 1234
  Volume size  : Removable media     Volume filesystem type: DOS
  Mount status : Unmounted           No mount at system startup possible
  Multiuser access  : Disabled       Fast disk access mode : 0 (Default)
  Vol.name char.set : ISO 646/ASCII  Current access mode   : 0 (Default)
  Unmount timeout   : None set
\end{verbatim}
\begin{verbatim}
  Volume name  : Personal financial records
  Volume date  : 06/09/93, 11:22:19  Volume serial number  : 177545
  Volume size  : 10.0 MB             Volume filesystem type: DOS
  Mount status : Mounted as E:       Mount ID              : 03A12F7B
  Multiuser access  : Disabled       Fast disk access mode : 0 (Default)
  Vol.name char.set : ISO 646/ASCII  Current access mode   : 0 (Default)
  Unmount timeout   : 30 minutes

  Volume name  : Encrypted data disk
  Volume date  : 12/04/93, 22:17:00  Volume serial number  : 69231461
  Volume size  : 42.5 MB             Volume filesystem type: DOS
  Mount status : Unmounted           Mount ID              : 42DD2536
  Multiuser access  : Enabled        Fast disk access mode : 1 (IDE direct)
  Vol.name char.set : ISO 646/ASCII  Current access mode   : 1 (IDE direct)
  Unmount timeout   : 10 minutes
\end{verbatim}

}
By default these two commands will display information on all available 
volumes. If information on an individual volume is required, then the volumes' 
name or drive letter can be given in addition to the {\tt info} or {\tt information} 
option. To change the previous use of the {\tt info} command to apply only to the 
volume named ``Data backup'', the command might be:

{\tt \verb|  |mountsfs info vol=backup}

and the output would be as follows:

\begin{verbatim}
    Date     Size   Type Mount status  Volume Name
  -------- -------- ---- ------------- -----------------------------
  01/11/93  Floppy  DOS    Unmounted   Data backup
\end{verbatim}

The {\tt status} option can be used to check whether any volumes are currently
moun\-ted.  As with the {\tt info} and {\tt informaton} options, by default information
on all mounted SFS volumes is displayed.  If information on an individual
volume is required, then the volumes' name or drive letter can be given in
addition to the {\tt status} option.  Thus the command:

{\tt \verb|  |mountsfs status}

will return a list of the status of the volumes on all mount points, as well as
an indication of the current setting of the quick-unmount hotkey and the
auto-unmount time settings for any mounted volumes (the latter are explained in 
more detail below), whereas the command:

{\tt \verb|  |mountsfs status f:}

will return the above status information only on the volume currently mounted
as F:.  An example of the output of the {\tt status} command when run on the setup
shown in the {\tt info} command examples with a total of two mount points available
might be:

\begin{verbatim}
  SFS volume `Personal financial records' is mounted as drive E:,
          and will time out in 18 minutes.
  Drive F: has no volume mounted

  The quick-unmount hotkey is set to `LeftShift-RightShift'.
\end{verbatim}

The {\tt +r} and {\tt +rw} options specify read and write access to the encrypted
volume.  {\tt +r} allows read-only access and {\tt +rw} allows read and write access.
The default is to allow read/write access.  Note that although mounting an SFS
volume read-only will stop all standard software from writing to it, it may not
stop some malicious programs such as viruses which have been specially written
to attack the SFS driver itself, or which are created specifically to destroy
disk data by bypassing the operating system and accessing the disk hardware
or firmware directly\footnote{
% Footnote [1]:
               Viruses capable of doing this are generally called tunneling
               viruses.  Most of them only tunnel down to the the DOS int 21h
               level (which won't affect SFS), but several tunnel down to the
               BIOS int 13h level.  The DIR II virus tunnels down to the block
               device driver request level (which again won't affect SFS).  In
               addition there is a report of a virus which will access an IDE
               hard drive directly through the drive controller ports (which,
               has the side-effect of crashing Windows when using 32-bit disk
               access).  No viruses capable of accessing SCSI drives through the
               ASPI or CAM drivers are known.  In any case an SFS volume creates
               a rather bad target for DOS viruses since the DOS drive it
               corresponds to is only an illusion created by the SFS driver, and
               the underlying data on disk is invisible to DOS and most viruses.
}.  The read-only option is provided mainly to stop any
accidental overwriting of valuable data on encrypted volumes.

Read-only access can also be specified when an SFS volume is mounted at the.
time the SFS driver is loaded into memory.  More details on this and on 
mounting volumes at system startup are given in the section ``Advanced SFS 
Driver Options'' below.

The read/write status of a volume can be changed once it has been mounted by
running mountsfs with only the {\tt +r} or {\tt +rw} option.  This will change the
read/write status of the currently mounted volume as appropriate.  For example
to allow read/write access to the currently mounted SFS volume the command
would be:

{\tt \verb|  |mountsfs +rw}

If the volume allows multiuser access, only the volume administrator can
directly mount it in the manner described above.  Normal volume users must
specify their user name with the {\tt user=$username$} command in addition to the
usual mount parameters in order to mount the volume\footnote{
%Footnote [2]:
              Some versions of SFS will automatically know the user's name when
              a volume is mounted.  Unfortunately the DOS version isn't one of
              these.
}.  The user name is the
name under which access is granted by the system administrator.  Like the
volume name, any portion of the user name can be given and mountsfs will match
whatever part of the name is given to any user names until it finds a match.
Users can also specify the name of the file to search for user access
information using the {\tt userfile={\em user file}} command.

For example if the volume in the previous example allowed multiuser access and
one of the users granted access to the volume was ``Henry Akely'', he could mount
it with the command:

{\tt \verb|  |mountsfs vol=secure user=henry}

If an attempt to mount a volume with no multiuser access capabilities is made,
mountsfs will exit with the error message:

{\tt \verb|  |Error: This volume has multi-user access disabled}

If access information for the given user cannot be found in the user access
file or files, the program will exit with an error message:

{\tt \verb|  |Error: Cannot find access information for user `henry'}

An individual users access rights to the volume, as set by the volume
administrator, may override certain options specified in mountsfs.  More
details on this, and on the operation of shared SFS volumes as a whole, are
given in the section ``Sharing SFS Volumes Between Multiple Users'' below.

If mountsfs is asked to mount a volume, it will first check to see whether
there is room to mount it.  If all available mount points are already occupied,
the program will print:

{\small \tt \verb|  |Error:~All available drives are allocated-unmount an existing volume first}

and exit.  In this case either an existing volume must be unmounted to free up
a mount point and allow the new volume to be mounted, or the number of mount
points must be increased with the {\tt UNITS=$n$} command when the SFS driver is 
loaded. More details on this are given in the section ``Loading the SFS Driver'' 
above.

If mountsfs is asked to mount a volume, it will search all available disks for
the named volume (if the volume is accessed by name), or check the removable
disk for the volume (if the volume is accessed by disk drive letter).  If the
volume is already mounted, mountsfs will print:

{\tt \verb|  |Error: Encrypted volume is already mounted}

and exit.  Otherwise, it will print a summary of the volume giving the
read/write status, the drive type, and the volume name and date if one exists:

{\tt \verb|  |Volume will be mounted as fixed drive E:.\\
     \verb|  |Encrypted volume is `Personal correspondence', created 12/08/93}

Then it will prompt for the encryption password:

{\tt \verb|  |Please enter password (10...100 characters), [ESC] to quit:}

At this point a password in the given length range can be entered.  For
security reasons the password is not echoed to the screen.  Any typing errors
when entering the password can be corrected with the backspace key.  The Esc
key can be used to quit.  The software will check for a password longer than
the maximum of 100 characters or an attempt to backspace past the start of the
password, and beep a warning when either of these conditions occur.  Once the
password has been entered, mountsfs will process it and reprogram the SFS
device driver to reflect the change in status.

If the disk being mounted is a removable one, mountsfs will check that the
drive being used supports disk change checking.  This is necessary to ensure
that the wrong disk isn't accidentally accessed by the driver.  If the disk is
changed without first being unmounted, the SFS driver will automatically
unmount it the next time an attempt is made to access it\footnote{
%Footnote [3]:
              The driver checks for a disk change when a disk read or write
              attempt is made rather than whenever DOS performs a general disk
              check, as DOS may perform up to half a dozen consecutive disk
              checks before doing anything, which leads to a significant loss
              in performance.
}.  However if the
drive doesn't support the disk change check (generally only rather old drives
have this problem), this automatic unmount won't be possible, and mountsfs will
warn:
{\small

\begin{verbatim}
  Warning: The floppy drive this volume is mounted on does not support disk
           change checking.  This means that great care must be taken to
           ensure the existing volume is unmounted (using either the
           `mountsfs' utility or the quick-unmount hotkey) when a new disk
           is inserted.
\end{verbatim}

}
If the drive does not support the disk change check, it is essential that the
volume be unmounted when the disk is changed.  The easiest way to unmount a
volume is through the quick-unmount hotkey, which is explained in more detail
below.

The {\tt unmount} option is used to unmounts SFS volumes.  This is used to remove
any access to volumes after any work which requires them has been completed, or
to free up a mount point so a new volume can be mounted.  If a particular SFS
volume is contained on a removable disk, it is a good idea to unmount the
volume if the disk in the drive is changed, although mounting a new volume will
automatically unmount the old volume.  The unmount operation can also be
performed using a quick-unmount hotkey which the SFS driver checks for (see
below).  Like the {\tt status} and {\tt information} command, the {\tt unmount} command can
either apply to individual mounted volumes which are specified by their drive
letter, or to all volumes if no drive letter is given.

Unmounting a volume also signals the SFS driver software to write all data
still held in system buffers to disk and to erase any information it still
holds in memory.  It is therefore good practice to always unmount volumes as
soon as they are no longer in use in order to destroy any sensitive information
which may still be held by the SFS driver or in a system buffer.  For example
to unmount all currently mounted volumes the command would be:

{\tt \verb|  |mountsfs unmount}

To unmount the volume currently mounted as F: the command would be:

{\tt \verb|  |mountsfs unmount f:}

A faster way to unmount all volumes is to use the quick-unmount hotkey which
the SFS driver checks for and accepts in place of the standard unmount command.
This can be used both as a convenience to quickly and easily unmount all SFS
volumes, or as a safety feature to allow encrypted volumes to be instantly
unmounted if there is a danger of the data on them being compromised (this
option is generally unavailable under Windows - see the section
``Incompatibilities'' below).

If no hotkey is currently set (either from a previous use of the mountsfs
command or through the use of the {\tt HOTKEY=NONE} option when the SFS driver is
loaded), and the {\tt hotkey=none} option is not specified, mountsfs will install a 
default quick-unmount hotkey which is a combination of the left and right shift 
keys.  On most keyboards these keys are fairly large and easy to reach during 
normal typing.  When both shift keys are pressed and released, all mounted SFS 
volumes will be unmounted as if a normal unmount command had been issued via 
mountsfs, and a single beep will sound to indicate that the unmount was 
successful.

Occasionally this default hotkey combination may clash with other software, or
it may be desirable to use another hotkey combination.  This can be set with
the {\tt hotkey=} option, which may be used to specify any combination of the left
shift key, right shift key, control key, alt key, and a letter key\footnote{
%Footnote [4]:
              The letter key is based on the US keyboard since the SFS driver
              must check for keyboard scan codes rather than actual character
              codes, which can differ slightly for some keyboards.
}.  The
keys are specified in the following manner:

\begin{center}
\begin{tabular}{|l|l|}
\hline
    Key            & Specification\\
\hline
    Alt key        & `alt'\\
    Control key    & `ctrl'\\
    Left shift key & `leftShift'\\
    Right shift key& `rightShift'\\
    Letter key     & `a'...`z'\\
\hline
\end{tabular}
\end{center}

Key combinations should be separated by hyphens, `-'.  The key names are not
case sensitive and can be given in upper or lower case, or a mixture of both.
If an unknown key name is used or the key names are not separated with hyphens,
mountsfs will complain:

{\tt \verb|  |Error: Bad quick-unmount hotkey format}

For example, to specify the use of the left shift and right shift keys as the
quick-unmount hotkey (the usual default setting), the command used in the
previous example would be changed to:

{\tt \verb|  |mountsfs hotkey=LeftShift-RightShift vol=secure}

To use the Control, Alt, and Z keys as the quick-unmount hotkey the command
would be:

{\tt \verb|  |mountsfs hotkey=ctrl-alt-Z vol=secure}

The hotkey value can also be altered without mounting any volumes.  This will
merely update the current hotkey without making any other changes.  For example
to set the right Shift, Control, and I keys as the quick-unmount hotkey (a
rather unwieldy combination), the command would be:

{\tt \verb|  |mountsfs hotkey=rightshift-CTRL-I}

The hotkey unmount can be disabled by specifying {\tt hotkey=none} when mountsfs is
run, either as part of a normal mount operation or by simply running mountsfs
with only the hotkey option, which will clear the unmount hotkey without making
any other changes.

Finally, the hotkey can also be specified when the SFS driver is loaded.  More
details on this are given in the section ``Advanced SFS Driver Options'' below.

If the hotkey unmount is performed while the driver is accessing a volume, the
disk access will complete before the volume is unmounted.

The SFS driver can also automatically unmount volumes if they have not been
accessed for a certain amount of time.  This option is useful if there is a
chance that an interruption may call you away from a system with mounted SFS
volumes, which would allow others access to the encrypted data, or can simply 
be used as a general safety precaution to automatically unmount the volumes 
after a sizeable period of inactivity (this option is unavailable under Windows
--- see the section ``Incompatibilities'' below).  However, care should be taken to 
allow a large enough safety margin for the timeout, as having a volume take 
itself offline five seconds before work is saved to it can be annoying.

The easiest way to set a timeout is to associate a timeout value with the
volume, either when it is created with {\tt mksfs} or at a later point with {\tt chsfs}.
If the volume is mounted, the setting of the timeout is automatically taken
care of by the SFS software.  The current timeout setting for a volume or
volumes may be displayed using the {\tt mountsfs information} command.

However it may be desirable to override this setting using the {\tt timeout=} 
option, which is used to specify the delay in minutes until the unmount takes 
place.  If the volume has no timeout associated with it then by default
mountsfs will not set an auto-unmount timer.  For example, using the previous
mount command but to have the volume automatically unmounted after 15 minutes 
of inactivity the command would be:

{\tt \verb|  |mountsfs timeout=15 vol=secure}

The timeout period must be between 1 and 30,000 minutes (this means that the 
upper timeout limit is around three weeks).  If a timeout value of less than 1 
minute or greater than three weeks is given, mountsfs will exit with the error
message:

{\tt \verb|  |Error: Timeout value must be between 1 and 30,000 minutes}

If no accesses are made to the volume within the given time period, it will be
automatically unmounted.  Like the case when a hotkey unmount is made, a single
beep will sound to indicate that the unmount has taken place.  Each volume has
its own timer, so that different volumes can be given different lengths of time
before they unmount, or no auto-unmount time at all.  This is useful when, for
example, one volume containing highly sensitive information needs to have a
very short timeout, while another volume containing less secret information can
have a much longer timeout.  An example might be a series of three SFS volumes:

\begin{verbatim}
  mountsfs timeout=10 vol=Topsecret
  mountsfs timeout=30 vol=Secret
  mountsfs timeout=60 vol=Confidential
\end{verbatim}

in which the ``Topsecret'' volume is given the shortest timeout of only 10
minutes, the ``Secret'' volume is given a timeout of 30 minutes, and the
``Confidential'' volume is given the longest timeout of a full hour.

The timed unmount can be disabled by specifying {\tt timeout=none} when mountsfs is
run, either as part of a normal mount operation which will affect only the
current volume, or by running mountsfs with only the timeout option, which will
clear the timer for all volumes without making any other changes.

If the timed unmount is performed while the driver is accessing a volume, the
disk access will complete before the volume is unmounted.

Finally, if all is OK, mountsfs will print a short summary message for the
action taken.  For example if the command given was one to unmount all volumes,
with two volumes F: and G: of which only F: was currently mounted, the summary
would be:

{\tt \verb|  |Volume F: has been unmounted\\
     \verb|  |Volume G: is already unmounted}


\section{Advanced SFS Driver Options}

The SFS driver supports several advanced options which can be used to customize
the operation of SFS.  These include the ability to mount SFS volumes
automatically when the driver is loaded, the ability to turn echoing of
passwords on, and the ability to change the read/write status, disk access
mode, and auto-unmount timeout of mounted volumes, the quick-unmount hotkey, 
and the password prompt used when automounting volumes.


\subsection{Mounting SFS Volumes at System Startup}

SFS volumes can be automatically mounted when the system is started up rather
than having to be mounted through the mountsfs program.  This can be specified
using the {\tt MOUNT={\em identification number}} option when the SFS driver is loaded, 
in conjunction with the 8- or 14-digit volume identification number displayed 
by mksfs when the encrypted volume is created or by using the {\tt mountsfs 
information} command.  The volume identifier is used to tell the SFS driver 
which volume to load.   In most cases the shorter 8-digit identifier is used, 
but the longer 12-digit form may be necessary for volumes with more complex 
access procedures such as ones on SCSI drives.  mksfs and mountsfs will always 
print the correct type of identifier for the volume in question.

If the volume allows multiuser access, only the volume administrator can mount 
it.  Normal volume users must follow the standard volume mount procedure using 
mountsfs.  The operation of shared SFS volumes is explained in more detail in 
the section ``Sharing SFS Volumes Between Multiple Users'' below.

For example if mksfs displays the 8-digit volume identifier `530A17FD' for a 
particular volume then the command to mount this volume would be:

{\tt \verb|  |DEVICE=SFS.SYS MOUNT=530A17FD}

If it displays a 14-digit volume identifier `C02100142DE0FC' then the command
to mount the volume would be:

{\tt \verb|  |DEVICE=SFS.SYS MOUNT=C02100142DE0FC}

If an incorrect volume identifier is given, the driver will display

{\tt \verb|  |Error: Invalid mount ID, skipping mount}

and skip the mount procedure.  If the volume identifier is correct, the driver 
will locate the required volume on the disk and try to read in the information 
needed to process it.  If this information cannot be read or is incorrect, the 
driver will display:

{\tt \verb|  |Error: Invalid SFS volume information, skipping mount}

and skip the mount procedure.  If the volume is located on a SCSI drive and the 
SCSI manager software needed to access the drive is not present, the driver
will display:

{\tt \verb|  |Error: SCSI manager not found, cannot mount SCSI drive}

and skip the mount procedure.  If all is correct the driver will ask for the 
password exactly as mountsfs would:

{\tt \verb|  |Please enter password (10...100 characters), [ESC] to quit:}

At this point a password in the given length range can be entered.  For
security reasons the password is not echoed to the screen.  Any typing errors
when entering the password can be corrected with the backspace key.  The Esc
key can be used to quit.  The software will check for a password longer than
the maximum of 100 characters or an attempt to backspace past the start of the
password, and beep a warning when either of these conditions occur.  Up to
three attempts at entering a correct password are allowed before the mount is 
skipped.  If the Esc key is pressed the SFS driver will print:

{\tt \verb|  |Mount operation skipped at user request}

and skip the mount procedure.  Otherwise, once the password has been entered, 
the SFS driver will process it and, if an incorrect password is detected, will 
print:

{\tt \verb|  |Error: Incorrect password, skipping mount}

The driver will then perform a quick disk read test to make sure everything is
working correctly.  If this fails, the driver will display:

{\tt \verb|  |Error: Disk read test failed, skipping mount}

and skip the mount procedure.  Otherwise the encrypted volume will be mounted 
ready for use, with the drive letter being the next available DOS drive.  In 
general the mount procedure is the same as the one which mountsfs uses, except 
that the full functionality of mountsfs is not available during the mount.  In 
all cases if the automount procedure is skipped the driver will still be loaded, so 
that volumes can be mounted at a later time if required.

The mount procedure has a built-in timer which expires if no key is hit for
more than 1 minute.  This is to allow unattended machines to automatically
reboot in case of a power failure without waiting forever for a mount
password.  If no key is pressed for more than 1 minute, the SFS driver will
print:

{\tt \verb|  |Password entry timed out, skipping mount}

and skip the mount procedure.  Again, the driver will still be loaded to allow 
volumes to be mounted at a later time.

Once the volume has been mounted and after the usual SFS installation message
has been displayed, the driver will display the DOS drive on which the
encrypted volume is mounted.  For example if the volume was available as drive
G: the message would be:

{\tt \verb|  |Encrypted volume is now mounted as drive G:}

If necessary the drive letter which SFS uses can be swapped through the use of
the JSWAP utility which comes as part of the JAM disk compression software.
The use of JSWAP rather than the DOS commands ASSIGN, SUBST, and JOIN, or other
third-party utilities such as the one provided with Stacker are recommended, as
JSWAP provides the safest means of swapping drive letters.  In particular,
JSWAP won't swap any non-physical drives, it won't reassign physical drives to
leave a hole in series of block devices (as opposed to the way DoubleSpace does
things), and it can fix things so that various badly-designed programs which
don't normally handle drive swapping too well should still work.
 
If multiple volumes are to be mounted then the mount identifiers should be
given in the order in which the mounts are to take place.  For example if a
second SFS volume with the volume identifier `4850414B' were to be mounted then 
the previous example would change to:

{\tt \verb|  |DEVICE=SFS.SYS MOUNT=530A17FD MOUNT=4850414B}

As more volumes are mounted, the driver will automatically increase the mount
point allocation until the maximum number of 5 mount points has been reached,
making use of the {\tt UNITS=$n$} option unnecessary.  If an attempt is made to 
mount more than 5 volumes, the driver will print:

{\tt \verb|  |Error: No more disk units available for mount}

and skip the mount procedure.


\subsection{Setting the Quick-Unmount Hotkey Value}

When a volume is mounted, the quick-unmount hotkey is by default set to a
combination of the left and right shift keys.  However, like the mountsfs
{\tt hotkey=} option, the SFS driver supports user-defined hotkeys with the
{\tt HOTKEY={\em quick-unmount hotkey}} command, as well as allowing the hotkey unmount
option to be disabled with the {\tt HOTKEY=NONE} command.

The {\tt HOTKEY={\em quick-unmount hotkey}} form of the command may be used to specify
any combination of the left shift key, right shift key, control key, alt key,
and a letter key, in the following manner:

\begin{center}
\begin{tabular}{|l|l|}
\hline
    Key            & Specification\\
\hline
    Alt key        & `alt'\\
    Control key    & `ctrl'\\
    Left shift key & `leftShift'\\
    Right shift key& `rightShift'\\
    Letter key     & `a'...`z'\\
\hline
\end{tabular}
\end{center}
%    Alt key        = `alt'          Control key     = `ctrl'
%    Left shift key = `leftShift'    Right shift key = `rightShift'
%    Letter key     = `a'...`z'

Key combinations should be separated by hyphens, `-'.  The key names are not
case sensitive and can be given in upper or lower case, or a mixture of both.
If an unknown key name is used or the key names are not separated with hyphens,
the SFS driver will complain:

{\tt \verb|  |Error: Bad quick-unmount hotkey format}

For example, to specify the use of the left shift and right shift keys as the
quick-unmount hotkey (the usual default setting), the command used in the
previous example would be changed to:

{\tt \verb|  |DEVICE=SFS.SYS MOUNT=530A17FD HOTKEY=LEFTSHIFT-RIGHTSHIFT}

To use the Control, Alt, and Z keys as the quick-unmount hotkey without
mounting any volumes the command would be:

{\tt \verb|  |DEVICE=SFS.SYS HOTKEY=CTRL-ALT-Z}

To disable hotkey unmounting altogether, and without mounting any volumes, the 
command would be:

{\tt \verb|  |DEVICE=SFS.SYS HOTKEY=NONE}


\subsection{Echoing the Mount Password to the Screen}

Normally when the mount password is entered, nothing will be echoed to the
screen.  However it may be desirable to echo the password to the screen as it
is typed in.  The {\tt ECHO} option can be used to turn on the echoing of
passwords. Note that use of this option makes it much easier to eavesdrop on
the password as it is entered, by simply glancing over the shoulder of the
person entering the password, by making use of monitoring facilities installed
for general security purposes, or by using more sophisticated techniques such
as TEMPEST monitoring (these are covered in more detail in the section ``Data
Security'' below).  For these reasons, use of the {\tt ECHO} option is not
recommended.


\subsection{Changing the Mount Password Prompt}

In some environments it may be undesirable to alert others to the fact that
disk encryption is being used.  Using the {\tt SILENT} option with the driver
removes most indications of the presence of SFS, but if volumes are mounted the 
appearance of the password prompt may still give things away.  To correct this 
problem, the SFS driver supports user-definable prompts with the
{\tt PROMPT={\em user-prompt}} command.  This may be used to specify any single-word
prompt, or, if the prompt is surrounded by quotation marks `{\tt "}', any combination
of characters until another `{\tt "}' is encountered.  For example to make the SFS
mount procedure appear like a network login, the previous mount example might 
be changed to:

{\tt \verb|  |DEVICE=SFS.SYS SILENT PROMPT=Login: MOUNT=530A17FD}

Instead of the usual password prompt, the driver would now display:

{\tt \verb|  |Login:}

when the password was required.

If a prompt containing multiple words is required, the prompt itself would be
surrounded with quotation marks:

{\tt \verb|  |DEVICE=SFS.SYS SILENT PROMPT="Please log on:" MOUNT=530A17FD}

Unfortunately some versions of DOS convert all characters to uppercase before
passing them to SFS.SYS.  In order to allow lowercase characters to be used in
prompts, the {\tt PROMPT=} option recognises the escape sequence {\tt $\backslash$s} to mean
``shift to lowercase'', so that all subsequent characters will be converted to
lowercase before being displayed.  Subsequent uses of {\tt $\backslash$s} will toggle the
current shift state, shifting characters back to uppercase or to lowercase
depending on the current shift state.  For example to have the previous
mount example display the prompt ``Enter Code:'', the command would be:

{\tt \verb|  |DEVICE=SFS.SYS SILENT PROMPT="E$\backslash$sNTER $\backslash$sC$\backslash$sODE:" MOUNT=530A17FD}

The initial ``E'' is displayed in uppercase, then the first {\tt $\backslash$s} shifts the
``NTER'' to lowercase, the second {\tt $\backslash$s} shifts the ``C'' back to uppercase, and the
final {\tt $\backslash$s} shifts the remaining ``ODE:'' to lowercase.

The {\tt PROMPT=} option also recognises a number of other escape codes which may
be used to specify characters which cannot be directly entered into the
CONFIG.SYS file such as quote marks, tabs, and line breaks.  These are based on
the ones used in the C programming language, and are as follows:

%    Newline     = \n        Quote mark  = \"        Bell    = \a
%    Tab         = \t        Backspace   = \b        Escape  = \e
\begin{center}
\begin{tabular}{|l|l|}
\hline
	Character  & Escape code \\
\hline
	Newline    & $\backslash${\tt n}\\
	Quote mark & $\backslash${\tt "}\\
	Bell       & $\backslash${\tt a}\\
	Tab        & $\backslash${\tt t}\\
	Backspace  & $\backslash${\tt b}\\
	Escape     & $\backslash${\tt e}\\
\hline
\end{tabular}
\end{center}

For example to print a line break as part of a prompt, the escape code $\backslash${\tt n} may
be used, allowing prompts to be split over multiple lines, or simply to have
blank lines as part of the prompt.  An extended form of the above prompt, split
over two lines, could be given as:

{\tt \verb|  |DEVICE=SFS.SYS SILENT PROMPT="Network logon$\backslash$nPlease enter password\\
     \verb|  |:" MOUNT=530A17FD}

which would be printed as:

{\tt \verb|  |Network logon\\
     \verb|  |Please enter password:}

The $\backslash${\tt e} escape code can be used in combination with an ANSI driver to allow
special actions such as cursor positioning and colour and text attribute
control.  Most of the useful escape sequences begin with $\backslash${\tt e[}, corresponding
to the ``ESC ['' combination.  These codes only work if an ANSI driver is loaded
by specifying

\verb|  |{\tt DEVICE=ANSI.SYS}

or some other ANSI-compatible driver in CONFIG.SYS.  Some of the possible codes
are:

\begin{center}
\begin{tabular}{|l|p{215.4pt}|l|}
\hline
  ANSI sequence   & Action                                    & Default\\
\hline
  $\backslash${\tt e[$row$;$column$H} & Move the cursor to $(row, column)$        & $(1,1)$\\
  $\backslash${\tt e[$row$;$column$f} & Move the cursor to $(row, column)$        & $(1,1)$\\
  $\backslash${\tt e[$row$d}          & Move the cursor to $row$                  & 1\\
  $\backslash${\tt e[$column$G}       & Move the cursor to $column$               & 1\\
  $\backslash${\tt e[$count$A}        & Moves the cursor up $count$ rows          & 1\\
  $\backslash${\tt e[$count$B}        & Moves the cursor down $count$ rows        & 1\\
  $\backslash${\tt e[$count$C}        & Moves the cursor right $count$ columns    & 1\\
  $\backslash${\tt e[$count$D}        & Moves the cursor left $count$ columns     & 1\\
  $\backslash${\tt e[2J}           & Clears the screen and homes the cursor & \\
  $\backslash${\tt e[K}            & Clears from the cursor position to the
                                     end of the current line & \\
  $\backslash${\tt e[M}            & Clears the entire line & \\
  $\backslash${\tt e[s}            & Saves the current cursor position & \\
  $\backslash${\tt e[u}            & Restores the previously saved cursor
                                     position & \\
  $\backslash${\tt e[$count$b}     & Repeat following character $count$ times  & 1\\
  $\backslash${\tt e[$attr;\dots;attr$m} & Set screen attributes based on $attr$.
                    Possible values for the $attr$ settings
                    are:

	{\small \hyphenation{ fore-ground back-ground }
	\begin{tabular}{|r|p{73.5pt}|r|p{75.5pt}|}
	\hline 
                    30 & Black foreground   & 40 & Black background\\
                    31 & Red foreground     & 41 & Red background\\
                    32 & Green foreground   & 42 & Green background\\
                    33 & Yellow foreground  & 43 & Yellow background\\
                    34 & Blue foreground    & 44 & Blue background\\
                    35 & Magenta foregr.    & 45 & Magenta backgr.\\
                    36 & Cyan foreground    & 46 & Cyan background\\
                    37 & White foreground   & 47 & White background\\
	\hline
	\end{tabular} } & \\
\hline
\end{tabular}
\end{center}

For example to clear the screen and home the cursor before printing the
``Login:'' prompt given previously, the command would be:

\begin{verbatim}
  DEVICE=SFS.SYS SILENT PROMPT=\e[2JLogin: MOUNT=530A17FD
\end{verbatim}

To print the prompt in black on a blue background, the command would be:

\begin{verbatim}
  DEVICE=SFS.SYS SILENT PROMPT=\e[30;44mLogin:\e[37;40
  MOUNT=530A17FD
\end{verbatim}

The order in which these arguments are given is important, since an option only
affects the other options following it.  If the {\tt PROMPT=} option is given after 
the {\tt MOUNT=} option instead of before it, the driver won't use the new prompt 
until after the automount has taken place.  This can be used to allow multiple 
independant prompts when several volumes are mounted, so that in the following 
example:

{\tt \verb|  |DEVICE=SFS.SYS SILENT PROMPT="Local server logon:\ " MOUNT=C0EDBABE\\
     \verb|  |PROMPT="Printer server logon:\ " MOUNT=2A1102D3}

the prompt {\tt "Local server logon:\ "} would be used for the first volume to be
mounted and the prompt {\tt "Printer server logon:\ "} would be used for the second 
volume to be mounted.  The {\tt PROMPT=} setting applies for all further mounts 
until another {\tt PROMPT=} option is given.


\subsection{Changing the Mount Read/Write Access Status}

Write access to a mounted volume can be enabled or disabled in the same manner 
as using the {\tt mountsfs +r} and {\tt mountsfs +rw} options would using the
{\tt READONLY} and {\tt READWRITE} options (more information on read-only access to SFS 
volume is given in the section ``Mounting an SFS Volume'' above).  For example to 
mount the volume used in the previous example read-only the command would be:

{\tt \verb|  |DEVICE=SFS.SYS READONLY MOUNT=530A17FD}

Like the other options which affect the mounting of volumes, the {\tt READONLY} or 
{\tt READWRITE} option must be given before the {\tt MOUNT=} which they are to affect.  
These options apply for all further automounts until another {\tt READONLY} or 
{\tt READWRITE} option is given.  For example to mount the first volume in the 
previous example read-only and the second one with normal write access the 
command would be:

{\tt \verb|  |DEVICE=SFS.SYS READONLY MOUNT=530A17FD READWRITE MOUNT=4850414B}


\subsection{Mounting Volumes as Non-Removable}
 
Since SFS volumes may be unmounted at any point through a hotkey unmount, an
auto-unmount timeout, or through use of the mountsfs program, the driver
reports them to the operating system as being removable volumes.  This means
that the system won't become confused when disk volumes suddenly cease to exist
after an unmount.

Unfortunately some software, while handling removable volumes perfectly well,
prefers to work with non-removable or fixed volumes.  An example of this is
Windows, which won't display volume labels for removable volumes.  Disk
buffering and cacheing for fixed volumes is also somewhat better since the
operating system doesn't have to worry about the volume being unmounted
suddenly, leaving it with nothing to write buffered data to.

The SFS driver allows volumes to be mounted as fixed volumes through the use of
the {\tt FIXED} keyword.  By default, or if the {\tt REMOVABLE} keyword is used, they are
mounted as removable volumes.  Volumes mounted as fixed volumes must be mounted
at system startup and will remain mounted until the system is either restarted
or powered down.  Hotkey, timed, and mountsfs unmounts will {\bf not} affect volumes
mounted in this manner --- they will remain mounted at all times.  For example to
mount the volume used in the previous example as a fixed, non-removable volume
the command would be:

{\tt \verb|  |DEVICE=SFS.SYS FIXED MOUNT=530A17FD}

Like the other options which affect the mounting of volumes, the {\tt FIXED} or
\linebreak 
{\tt REMOVABLE} options must be given before the {\tt MOUNT=} which they are to affect.
These options apply for all further mounts until another {\tt FIXED} or {\tt REMOVABLE}
option is given.  For example to mount the first volume in the previous example
as a fixed volume and the second one as a standard removable one the command
would be:

{\tt \verb|  |DEVICE=SFS.SYS FIXED MOUNT=530A17FD REMOVABLE MOUNT=4850414B}


\subsection{Setting the Auto-unmount Timeout value}

The auto-unmount timeout value functions just like the mountsfs {\tt timeout=}
option, and is used to tell the SFS driver to unmount a volume automatically if
it has not been accessed for a certain amount of time.  The time until the
volume is automatically unmounted can be set with the {\tt TIMEOUT=} option, which 
is used to specify the delay in minutes until the unmount takes place.  This 
option can only be used in conjunction with the {\tt MOUNT=} option, and by default 
no auto-unmount timer is set.

The use of this option is only necessary if the volumes to be mounted have no 
timeout values associated with them, either by mksfs when the volume is created 
or by chsfs at a later point in time.  However, if required, the {\tt TIMEOUT=} 
option can also be used to override any existing timeout settings for the 
volume.  In order to return to the default timeout settings, the
{\tt TIMEOUT=DEFAULT} option may be used.  This sets the timeout of any volumes
mounted after this point to either the value associated with the volume, or
none at all if the volume has no timeout setting.  To force the timeout setting
to be disabled, the {\tt TIMEOUT=NONE} option may be used, which ensures no timeout 
is set for any volumes automounted after this point.  As before, this option holds 
until another {\tt TIMEOUT=} setting is used.

The timeout value currently set for a volume can be displayed with the
{\tt mountsfs information} command.

Like the other options which affect the mounting of volumes, the {\tt TIMEOUT=} 
option must be given before the {\tt MOUNT=} which it is to affect.  The {\tt TIMEOUT=} 
option applies for all further mounts until another {\tt TIMEOUT=} option is given.

Using the previous mount example, but to have a volume automatically unmounted 
after 15 minutes of inactivity, the command would be:

{\tt \verb|  |DEVICE=SFS.SYS TIMEOUT=15 MOUNT=530A17FD}

The timeout period must be between 1 and 30,000 minutes (this means the upper
timeout limit is around three weeks).  If a timeout value of less than 1 minute
or greater than three weeks is given, mountsfs will exit with the error
message:

{\tt \verb|  |Error: Timeout value must be between 1 and 30,000 minutes}

If no accesses are made to a volume within the given time period, it will be 
automatically unmounted.  Like the case when a hotkey unmount is made, a single 
beep will sound to indicate that the unmount has taken place.  Each volume has
its own timer, so that different volumes can be given different lengths of time
before they unmount, or no auto-unmount time at all.  This is useful when, for
example, one volume containing highly sensitive information needs to have a
very short timeout, while another volume containing less secret information can
have a much longer timeout.  For example the two volumes used in the previous
example might be mounted as follows:

{\tt \verb|  |DEVICE=SFS.SYS TIMEOUT=10 MOUNT=530A17FD TIMEOUT=30 MOUNT=4850414B}

in which the first volume is given a short timeout of only 10 minutes while the
second volume, which presumably holds less critical information, is given a
longer timeout of half an hour.  If the volumes already have a timeout setting
and the {\tt TIMEOUT=} option is being used to override it, the default behaviour
of using the setting associated with the volume may be restored with the
{\tt TIMEOUT=DEFAULT} option.  For example if, in the previous example, the default 
timeout setting for the second volume were to be used instead of overriding it 
with a 30-minute timeout, the volumes would be mounted as follows:

{\tt \verb|  |DEVICE=SFS.SYS TIMEOUT=10 MOUNT=530A17FD TIMEOUT=DEFAULT\\
     \verb|  |MOUNT=4850414B}

This would mount the first volume as before, and the second volume with
whatever timeout was set for it by mksfs or chsfs.  If no timeout at all is
required for the second volume, the volumes would be mounted as follows:

{\tt \verb|  |DEVICE=SFS.SYS TIMEOUT=10 MOUNT=530A17FD TIMEOUT=NONE\\
     \verb|  |MOUNT=4850414B}


\subsection{Enabling Fast Direct Disk Access Modes}

SFS supports a number of faster disk access modes, which would normally be
specified when the volume is created with mksfs, or set at a later date with
the {\tt chsfs newaccess=} command.  However it may be desirable to override these
settings when volumes are mounted.  This can be done with the {\tt FAST=} option, 
which takes as an argument the fast access mode given by the {\tt mksfs -c}
command, or 0 to specify the normal, somewhat slower access mode.  The access
mode currently set for a volume can be displayed with the {\tt mountsfs
information} command.

For example if the volume in the previous example was created with no fast
access mode set, but it was later discovered that it could be accessed with
fast access mode\ 1, the mount command would be:

\verb|  |{\tt DEVICE=SFS.SYS FAST=1 MOUNT=530A17FD}

The use of this option is only necessary if the volumes to be mounted have no 
fast access mode associated with them, either by mksfs when the volume is
created or by chsfs at a later point in time.  However, if required, the
{\tt FAST=} option can also be used to override any existing fast access mode
settings for the volume.  In order to return to the default access mode
settings, the {\tt FAST=DEFAULT} option may be used.  This sets the access mode of
any volumes mounted after this point to the mode normally associated with the 
volume.  For example if, in the previous two-volume mount example, the default 
access mode setting for the second volume were to be used instead of overriding 
it with an access mode of 1, the volumes would be automounted as follows:

\begin{verbatim}
  DEVICE=SFS.SYS FAST=1 MOUNT=530A17FD FAST=DEFAULT MOUNT=4850414B
\end{verbatim}

This would mount the first volume as before, and the second volume with
whatever access mode was set for it by mksfs or chsfs.

If a certain access mode is required in order to access a volume (for example
some volumes on SCSI drives can only be accessed via SCSI access methods) then
SFS will always use the appropriate access mode and ignore the current setting
of the {\tt FAST=} option.

Like the other options which affect the mounting of volumes, the {\tt FAST=} option 
must be given before the {\tt MOUNT=} which it is to affect.  The {\tt FAST=} option 
applies for all further mounts until another {\tt FAST=} option is given.


\section{Changing the Characteristics of an SFS Volume}

Once an SFS volume has been created, various characteristics of the volume and
the entire volume itself can be altered using the chsfs program.  This allows
the SFS volume password, volume name, disk access mode, and auto-unmount 
timeout to be changed, allows SFS volumes to be quickly deleted, and allows the 
reversion of SFS volumes to their original unencrypted form.

The chsfs program is run in the following manner:

{\tt \verb|  |chsfs [newpass] [newvol={\em new volume name}] [newtimeout=$timeout$]\\
\verb|        |[newaccess={\em new access mode}] [delete] [convert]\\
\verb|        |[vol={\em volume name}] [$drive$]}

Since all arguments are named, they can be given in any order.  The order shown
here is merely an example.

In general the volume name would be specified with the {\tt vol=} option.  For
example if the volume name was ``Secure disk volume'' then the command would be:

{\tt \verb|  |chsfs $command$ vol=secure}

The volume name can be in upper or lower case, and the full name need not be
given.  chsfs will match whatever part of the name is given to any SFS
volume names found until it finds a match.

Alternatively, if the SFS volume to be accessed is on a removable disk, the
drive letter can be specified instead of the volume name.  For example if the
disk drive was A: then the command would be:

{\tt \verb|  |chsfs $command$ a:}

In order to find all available SFS volumes on all disks, the {\tt mountsfs info}
option can be used as outlined in the section ``Mounting an SFS Volume'' above.

The basic characteristics of the SFS volume can be changed with the {\tt newpass},
{\tt newvol}, {\tt newaccess}, and {\tt newtimeout} commands, which set a new password, 
new volume name, new disk access mode, and new auto-unmount timeout respectively.  
These commands can each be used individually, or two, three, or even all four 
may be used together (although they can't be used in conjunction with the 
{\tt delete} or {\tt convert} options).  Their usage is in general similar to their use 
with mksfs.  

\begin{itemize}

\item {\tt newpass} takes no arguments and will prompt for the original password and 
    then the new password, after which it will change the volume password from the 
    original to the new one. 

\item {\tt newvol} takes as an argument the new volume name. 

\item {\tt newaccess} takes as an argument the fast disk access mode obtained by
    running the mksfs program with the {\tt -c} option, with mode 0 being the
    default, slower access mode. 

\item {\tt newtimeout} takes as an argument the auto-unmount timeout setting in
    minutes, or `none' to clear the auto-unmount timer setting for this volume.

\end{itemize}

Since chsfs makes changes to the header record of an encrypted volume, some
anti-virus programs may print a warning about the boot sector of the volume
being changed (despite the fact that the volume is quite clearly not an MSDOS
filesystem).  This warning can be ignored.

As an example, to change the name of the SFS volume ``Personal data'' to 
``Letters'' and the auto-unmount timer setting to 30 minutes, the command would
be:

{\tt \verb|  |chsfs vol=personal newvol=Letters newtimeout=30}

If the newpass option is used, chsfs will first ask for the old poassword:

{\tt \verb|  |Please enter old password (10...100 characters), [ESC] to quit:}

After verifying that the password is correct, chsfs will ask for the new
password:

{\tt \verb|  |Please enter new password (10...100 characters), [ESC] to quit:}

Like mksfs, chsfs will then ask for this password a second time for safety.
Before updating the volume information, chsfs will perform the same
multiple-overwrite operation used by the {\tt delete} option of chsfs (see below)
to erase the original volume header, which is based on the old password.  This
ensures that no trace of the original disk access information remains before it
is replaced by the new access information.

Once the details for the new volume name, auto-unmount timeout, access mode, or 
password have been obtained and the changes made to the volume, chsfs will 
display a message indicating the changes made.  For the above example the 
message would be:

{\tt \verb|  |Volume characteristics successfully updated.

     \verb|  |The new volume name is `Letters'.\\
     \verb|  |The new auto-unmount timeout is set to 30 minutes.}

Note that chsfs doesn't perform the checking for duplicate or nonexistant
volume names and the checking for correct functioning of different disk access 
modes which mksfs does.  This is to allow the safe choices forced by mksfs to 
be subsequently overridden using chsfs if required\footnote{
%Footnote [1]:
              This makes the (possibly incorrect) assumption that the chsfs
              user knows what they are doing.
}.

Changes to the SFS volume itself are made using the {\tt convert} and {\tt delete}
commands.  {\tt convert} converts a volume back to its original unencrypted form,
and {\tt delete} deletes it entirely, leaving behind what appears to the operating
system as an unformatted disk filled with random noise.

Since converting or deleting a volume while it is mounted is rather dangerous,
chsfs checks whether the volume to be converted or deleted is currently
mounted.  If it is mounted and removable, it will prompt:
{\small
\begin{verbatim}
  Warning: This volume is currently mounted. Do you wish to unmount it
           and continue [y/n]
\end{verbatim}
}
At this point a response of `Y' will continue and a response of `N' will exit
the program.  If a `N' response is entered, the volume can be unmounted using
mountsfs or the quick-unmount hotkey before chsfs is re-run.  If the volume has
been mounted as a fixed, non-removable volume, chsfs will exit with the error
message:
{\small
\begin{verbatim}
  Error: This volume has been mounted as a non-removable volume and cannot
         be unmounted. In order for chsfs to be able to work with it, change
         the CONFIG.SYS entry for the SFS driver and reboot the machine.
\end{verbatim}
}
The delete option will first print the name and creation date of the SFS volume
to be deleted.  At this point the exact name and date of the volume should be
checked to ensure that this is indeed the one to be deleted.  In this example
the volume information will be displayed as:

{\tt \verb|  |Encrypted volume is `Incriminating evidence', created 04/11/93}

chsfs will now prompt for the password in the usual manner.  It uses this to
check that access to the volume is legitimate, and is needed for chsfs to
acquire various pieces of information it needs to perform the deletion.  The
program will then prompt:
{\small

\begin{verbatim}
  Warning: The deletion operation will permanently destroy all data on this
           volume. Are you sure you want to continue with the deletion [y/n]
\end{verbatim}

}
At this point a response of `Y' will continue and a response of `N' will exit
the program.

If chsfs is told to continue, it will perform multiple overwrite passes over
the SFS volume header (which contains all the information needed to access the
volume), printing a progress report as it performs the overwriting:

{\tt \verb|  |Overwriting: Pass 1}

In total chsfs will perform 30 separate overwrite passes which have been
selected to provide the best possible chances of destroying data for various
disk encoding schemes (the exact details are given in the section ``Deletion of
SFS Volumes'' below).  Once the multiple overwrites have completed, chsfs will
print an informational message about the deletion operation:

{\tt \verb|  |Encrypted volume `Incriminating evidence' has been destroyed}

If the volume is on a fixed disk, you may wish to reboot your machine to make
the newly-deleted volume visible to DOS.  Volumes on floppy disks will
automatically be visible.  Since the disk volume is now filled with random
garbage, it will need to be formatted in the same way an unformatted disk would
be before it can be used by DOS.

The convert option will, like the delete option, first print the name and
creation date of the SFS volume to be converted.  At this point the exact name
and date of the volume should be checked to ensure that this is indeed the one
to be converted.  In this example the volume information will be displayed as:

{\tt \verb|  |Encrypted volume is `Disk data', created 07/12/93}

chsfs will prompt for the encryption password exactly as mksfs did when it
originally created the SFS volume, and will then prompt:
{\small

\begin{verbatim}
  Warning: You are about to convert this volume from an encrypted SFS one to
           a normal DOS one.  Are you sure you want to continue with the
           conversion [y/n]
\end{verbatim}

}
At this point a response of `Y' will continue and a response of `N' will exit
the program.

Like mksfs, chsfs will then begin converting the disk.  As it processes the
volume, it prints a progress bar going from 0\% complete to 100\% complete.  The
conversion process will take a few minutes on most disks, and is somewhat
slower than a standard disk formatting procedure which only writes a very small
amount of data to the start of the disk and scans for bad sectors, whereas 
chsfs has to read, decrypt, and write the entire disk volume.

As the conversion progresses, the progress bar will gradually fill up until it
shows that the conversion is complete.  Once this has finished, chsfs will
display the message:

{\small \tt \verb|  |Encrypted volume `Disk data' has been converted to a normal DOS volume.}

The converted volume is now ready to be used as a normal DOS disk again. If the
volume is on a fixed disk, DOS will still think it is an encrypted SFS one
rather than a normal DOS one.  It is recommended that you reboot your machine
at this point to clear any memories of the old volume from the system, as DOS
will not be able to see the converted volume until the reboot takes place.  As
a reminder, chsfs will display:

{\tt \verb|  |You may wish to reboot your machine to update the status of the\\
     \verb|  |volume, which will become available as a standard DOS disk.}

before exiting.  If the volume is on a removable disk, no reboot is necessary
and chsfs will simply print:

{\tt \verb|  |The volume is now available as a standard DOS disk.}

%Footnote [1]: This makes the (possibly incorrect) assumption that the chsfs
%              user knows what they are doing.


\section{Sharing SFS Volumes Between Multiple Users}

At times it may be necessary to share a single encrypted SFS volume between
multiple users.  For instance several individuals may require access to a
volume containing confidential business correspondence as part of their
day-to-day duties.  Usually this would require using a common password which is
known to every member of the group of people who require access.  The need to
share passwords is a serious weakness, as the inability to choose individual,
unique passwords increases the chances that a simple, easy-to-remember (and
easy-to-guess) password is chosen, or that at least one person writes it down
if it is too hard to remember.

SFS solves this problem by allowing each member of the group access to an
encryped volume under their own individual password.  The allocation of access
rights to a volume is controlled by an administrator who can grant or revoke
access as required.  The administration process is handled by the adminsfs
program, which is run in the following manner:

{\tt \verb|  |adminsfs [adduser={\em user name}] [deluser={\em user name}] [chuser={\em user name}]\\
\verb|           |[showuser={\em user name}] [showall] [validfrom={\em DDMMYY}]\\
\verb|           |[validto={\em DDMMYY}] [userfile={\em user file}]}

Since all arguments are named, they can be given in any order.  The order shown
here is merely an example.

[!!!! That's all there is at the moment.  adminsfs is still being checked
      out by beta-testers and parts of it are still under review.  If anyone
      has any suggestions for it, let me know !!!!]


\section{Creating Compressed SFS Volumes}

Creating a compressed drive inside an SFS volume provides, apart from the usual
benefit of increasing the apparent disk space, some additional security against
an attack by breaking up the very regular standard filesystem structure
containing large quantities of known data at known locations into a compressed
filesystem whose structure and contents are much harder to ascertain.
This section contains information on using SFS with Stac Electronic's ``Stacker''
and JAM Software's ``JAM''.

\subsection{SFS and Stacker}

The instructions given here are for Stac Electronics ``Stacker'', although it
should be possible to do the same thing with other reasonably advanced disk
compressors.  Stacker allows the compression of an entire DOS drive, or
compression of any remaining free space on the drive.  Creation of a compressed
Stacker volume involves first mounting the SFS volume on which Stacker is to be
installed, either with the system-startup mount option of the SFS driver or 
with the mountsfs utility.  Stacker should then be installed in the usual 
manner onto the mounted volume.  Under Windows, this involves choosing the 
{\tt Compress} option from the ``Stacker Toolbox'', and under DOS it involves running 
the {\tt stac.exe} program and picking the appropriate option, or using the 
{\tt create.com} program on the drive to be compressed.  Stacker will then 
defragment the drive, ask a few questions, and create the Stacker volume.  When 
the Stacker drive has been created, the appropriate mounting parameters for the 
drive will be added to the STACKER.INI file.

Once the installation has completed, the SFS volume will contain the STACKVOL
file in which Stacker stores the compressed disk data.

The Stacker drive can then be mounted in two ways, either from CONFIG.SYS onto
an SFS volume mounted at system startup, or at a later point (which, however, 
means that it loads an extra copy of the environment variables).  It is also 
possible to load the driver without activating a compressed drive by removing 
the specification for the drive to be mounted from the STACKER.INI file (this 
is normally used for floppies, but works for SFS as well since Stacker treats 
SFS volumes as a removable drive).  The drive can then be mounted at a later 
time with the {\tt stacker {\em drive letter}} command.  This avoids the need to mount 
an SFS volume at startup.  Under DOS 6, Stacker 4.0 loads using a device 
driver, but hooks into DOS like DoubleSpace does (or at least did when it was 
available).  The rest of Stacker is then loaded with the STACHIGH.SYS driver.

As SFS uses whatever drive letters DOS allocates to it, the stacked drive will
take over the drive letter used by the SFS volume rather than swapping drive
letters for the stacked and normal drive as it usually does.  This shouldn't
provide any problems with accessing the drive, as the compressed and encrypted
drive will simply replace the encrypted drive.  However, it will provide
problems at a later point because Stacker uses the ``Mount replaced'' option, in
which Stacker manipulates internal DOS data structures to completely replace
the original SFS drive.  This means that mountsfs can no longer find the
mounted SFS drive for the {\tt status}, {\tt info}, {\tt information}, and {\tt unmount}
commands, although timed unmounts and hotkey unmounts performed by the SFS
driver itself will still work.

\subsection{SFS and JAM}

Using JAM Software's ``JAM'' compressor with SFS is somewhat simpler than using
Stacker, and provides an additional benefit of speeding up effective disk
access times since the high-speed JAM software reduces the amount of data which
must be subsequently encrypted.  Creation of a compressed JAM volume involves
first mounting the SFS volume on which the JAM volume is to be installed,
either with the system-startup mount option of the SFS driver or with the
mountsfs utility.  The compressed volume can then be created as described in
the JAM documentation with the JCREATE utility.

Once the creation of the compressed disk volume has completed, it can be
mounted by loading the JAM.SYS driver via the CONFIG.SYS file, and mounting the
JAM volume using the JMOUNT utility, either onto an SFS volume mounted at
system startup by running JMOUNT from the CONFIG.SYS file, or at a later point
by running JMOUNT from the command line.  This avoids the need to mount an SFS
volume at startup.

Unlike Stacker, JAM does not mess with DOS drive letters, allowing both the SFS
volume and the JAM volume it contains to be accessed as normal.  JAM is
available for FTP from garbo.uwasa.fi and all garbo mirrors as
/pc/arcers/jam119sw.zip.


\section{WinSFS - Using SFS with Windows}

WinSFS is a prototype of the Windows version of SFS, and currently runs as a
front-end for mountsfs, which means that the mountsfs program must be either in
the DOS path or in the Windows directory for WinSFS to work.  WinSFS also needs
the Visual Basic library VBRUN200.DLL in order to run.  This file is publicly
available from a number of sources.

When run, WinSFS will display a window containing a list of SFS volumes
available to be mounted, a list of currently mounted volumes, and an icon bar
which is used to control WinSFS.  These icons perform the following functions:

\begin{center}
\begin{tabular}{|l|l|}
\hline
  Icon              & Function\\
\hline
  Cross icon        & Exit WinSFS\\
  Disk icon         & Mount an SFS volume\\
  Crossed disk icon & Unmount an SFS volume\\
  Information icon  & Display detailed information on an SFS volume\\
  Write icon        & Enable read/write access on an SFS volume\\
  Crossed write icon& Enable read-only access on an SFS volume\\
\hline
\end{tabular}
\end{center}

\subsection{Mounting a Volume with WinSFS}

To mount an SFS volume, click on the volume name in the ``Available volumes''
window, and then click on either the ``Mount volume'' icon or the ``Mount'' button
(eventually this function will also be available by dragging the volume name
and dropping it into the ``Mounted'' list).  WinSFS will ask for the volume
password, and then mount the volume.  Once the volume is mounted, its name will
disappear from the ``Available volumes'' list and appear in the ``Mounted'' list.


\subsection{Unmounting a Volume with WinSFS}

To unmount an SFS volume, click on the volume name in the ``Mounted'' window, and
then click on either the ``Unmount volume'' icon or the ``Unmount'' button
(eventually this function will also be available by dragging the volume name
and dropping it into the ``Available volumes'' list).  WinSFS will unmount the
volume, and its name will disappear from the ``Mounted'' list and appear in the
``Available volumes'' list.


\subsection{Getting Information on a Volume with WinSFS}

To get detailed information on a volume, click on its name, and then either
click the right mouse button, or select the ``Information'' icon.  This will
bring up a window giving extra information on the volume such as the creation
time, serial number, size, and mount identifier.


\subsection{Setting a Volume's Read/Write Access with WinSFS}

To change the read/write status of an SFS volume, click on its name in the
``Moun\-ted'' window, and then click on either the ``Read-only'' or ``Read/write'' icon
in the icon bar to change its access status.


\section{Command Summary}

This section serves as a quick-reference for the options available with the
various SFS programs.  The available options for mksfs, mountsfs, chsfs, and
adminsfs are:

\begin{itemize}

\item MakeSFS - Make Secure Filesystem

  \begin{tabular}{l p{222.4pt} }

  -c                         & Perform a confidence test on the volume to be
                               encrypted without actually encrypting it\\

  -o                         & Override the disk boot record sanity check. This may be
                               necessary for some unusual disk formats\\

  -t                         & Test the integrity of the MDC/SHS encryption code
                               used in SFS\\

  -e                         & Display an extended error code if an error occurs.  This provides
                               extra information on the nature of some errors\\

  multiuser                  & Allow multiuser access on the volume to be created\\

  fastaccess=$mode$   & Specify the fast disk access mode (as given by the
                        output of the {\tt -c} option) to use for this volume\\
  timeout=$timeout$   & Specify the auto-unmount timeout for this volume\\
  wipe                & Wipe the original data before overwriting it with
                        encrypted data (this option is very slow)\\

  vol={\em volume name}      & Specify the name of the volume to be created\\

  serial={\em serial number} & Specify the serial number of the volume to be created\\

  {\em drive letter}         & Specify the letter of the drive to create the
                               encrypted volume on\\
  \end{tabular}

\item MountSFS - Mount Secure Filesystem
\nopagebreak

  \begin{tabular}{p{87pt} p{225pt} }

  +r                    & Mount the encrypted volume with read-only access\\

  +rw                   & Mount the encrypted volume with read/write access (default)\\

  info                  & Show brief information on all available SFS volumes\\

  information           & Show detailed information on all available SFS volumes\\

  status                & Show information on mounted volumes only\\

  unmount               & Unmount the volume\\

  hotkey=$hotkeys$      & Set the quick-unmount hotkey combination\\

  timeout=$timeout$     & Set the auto-unmount timer value in minutes\\

  user=$username$       & Specify the user name for a volume with multiuser access\\

  userfile=$filename$   & Specify the path to the information file associated with
                          a volume which allows multiuser access\\

  vol={\em volume name} & Specify the name of encrypted volume to mount\\

  {\em drive letter}    & Specify the drive letter of the volume to mount\\
                        & (For volumes on floppy disks only)\\
  \end{tabular}

\item ChangeSFS - Change Secure Filesystem

  \begin{tabular}{l p{209.2pt} }

  newpass                   & Set a new volume password\\

  newvol={\em volume name}  & Specify the new volume name\\

  newaccess={\em access mode} & Specify the new fast disk access mode\\

  newtimeout=$timeout$        & Specify the new auto-unmount timeout\\

  delete                    & Delete SFS volume\\

  convert                   & Convert volume back to unencrypted form\\

  vol={\em volume name}     & Specify the name of the encrypted volume to change\\

  {\em drive letter}        & Specify the drive letter of the volume to change\\
                            & (For volumes on floppy disks only)\\
  \end{tabular}

\item AdminSFS - Administrate SFS User Database

  \begin{tabular}{l p{208.6pt} }

  adduser={\em user name} & Add a new user with the given name to the database\\

  deluser={\em user name} & Remove user with the given name from the database\\

  chuser={\em user name}  & Change user database entry for the named user\\

  showuser={\em user name}& Show access information for a given user\\

  showall                 & Show access information for all users\\

  validfrom=$DDMMYY$      & Set date after which access for a user is allowed\\

  validto=$DDMMYY$        & Set date at which a users access expires\\

  userfile=$filename$     & Specify the path to the user information file\\

  \end{tabular}

\end{itemize}

\section{Incompatibilities}

Over the years a variety of strange hardware and software setups have been
created in order to get around some of the shortcomings of the PC hardware and
DOS (and occasionally other operating systems) software.  Since SFS accesses
the disk at a level below that normally used by the operating system, it will
bypass special options like compressed volumes and non-local networked drives,
and won't recognise nonstandard hardware like drives with more than 1024
cylinders which require special software patches in order to work with DOS.  For 
example, SFS will recognise the uncompressed volumes used by Stacker and
DoubleSpace, and JAM, but won't see the compressed volumes as these are an 
illusion created in software and visible only to DOS.  It is therefore not 
possible to encrypt compressed volumes (there would be very little point, as 
encryption would render the data completely uncompressible), although it is 
possible to create a compressed volume inside an encrypted volume (this is 
covered in the section ``Creating Compressed SFS Volumes'' above).


\subsection{Checking for Problems with mksfs}

If your system has an unusual setup, or if you're worried about what SFS may
do, you can use a special option with the mksfs command to perform a check on
the drive which is to be encrypted.  This option also bypasses a number of the
usual checks SFS performs relating to duplicate volume names, anonymous
volumes, and so on, to allow all types of volume arrangements to be checked.

If the {\tt -c} option is specified along with the drive letter, mksfs will (if the
volume in question is a fixed disk) first display technical information on all
available fixed disk volumes, so that the command:

{\tt  \verb|  |mksfs -c e:}

would produce the following output:
{\small

\begin{verbatim}
  Drive partition information follows:

  Ph Bt Dr Cyl. Head Sec. Cyl. Head Sec.  Size  ID Type
  -- -- -- ---- ---- ---- ---- ---- ---- ------ -- ----
   0  N  C    0    1    0  379   15   39 121600 06 DOS (16-bit FAT, >= 32M)
   0  Y  -  380    0    0  383   15   39   1280 0A OS/2 boot manager
   0  N  D  384    1    0  594   15   39  67200 06 DOS (16-bit FAT, >= 32M)
   0  N  E  595    1    0 1022   15   39 136640 06 DOS (16-bit FAT, >= 32M)
                                                  This would be the SFS disk
  06  N  -    0    1    0  442   63   31 452608 07 OS/2 HPFS
  06  N  F  443    0    0  571   63   31 131072 06 DOS (16-bit FAT, >= 32M)
  06  N  G  572    0    0  872   63   31 307200 06 DOS (16-bit FAT, >= 32M)
\end{verbatim}

}
This is only displayed for fixed disks, as floppy disks don't contain this 
information.  The values in the various columns are Ph = physical drive number, 
Bt = bootable flag, Dr = DOS drive letter, Cyl,Head,Sec = partition start, 
Cyl,Head,Sec = partition end, Size = size in kbytes, ID = partition ID byte, 
and Type = partition type.  The proposed SFS partition will be marked as such.  
The drive with an apparent \hbox{2-digit} physical drive number is a SCSI drive which
isn't accessible through the BIOS; the first digit is the SCSI target ID, the
second digit is the logical unit number.  If you don't know what these values 
mean, don't worry---this option is mainly useful in providing technical 
information for those who want it.

Once all drives have been checked, more specific information on the actual
volume in question is displayed:

\begin{verbatim}
  Volume will be checked on fixed drive E:
  This drive has a capacity of 136.6 MB and is labelled `Data disk'
  Are you sure you want to check this volume [y/n]
\end{verbatim}

As with the usual mksfs process, typing `Y' will continue with the volume check
and typing `N' will exit.  If you choose to continue, mksfs will first perform
an initial disk confidence test which consists of some general checks on the
volume layout to make sure its format is valid, and will then perform a read
confidence test in which it reads random disk blocks and compares them with the
data reported by the operating system.  If any errors are encountered, it will
print a diagnostic message before continuing.  If all is OK, the sequence of
messages will be:

{\tt
\verb|  |Performing disk confidence test\dots \\
\verb|  |Performing read confidence test\dots \\
\verb|  |{\em various test-in-progress messages} \\
\verb|  |Confidence test successfully concluded
}

If there are problems, the diagnostic message will give more information on the
nature of the problem.  After the basic tests have completed, mksfs may display
specific information about the particular drive on which the SFS volume is to
be created, and ask whether additional tests should be made to determine
whether use of the fast access modes supported by the SFS driver is possible.
A typical message would be:
{\small
\begin{verbatim}
  This drive is a WDC AC2420 with a multi-sector 256K buffer, and appears
  to support the high-speed direct access mode which SFS is capable of.
  mksfs will now test whether this is indeed true.  Are you sure you want to 
  perform the test [y/n]
\end{verbatim}
}
with the exact text depending on the drive type.  At this point a response of
`Y' will run the extended tests, and a response of `N' will exit the program.
The extended tests are similar to the previous tests, and display the same
messages if problems are found.

If problems are detected, mksfs will display:
{\small
\begin{verbatim}
  This drive does not appear to support the high-speed direct-access mode
  used by SFS.  The default slower access mode will be used.
\end{verbatim}
}
Otherwise, the message:
{\small
\begin{verbatim}
  This drive supports the high-speed direct-access mode used by SFS.
  You can enable use of this mode by specifying the `fastaccess=1' option
  when mksfs is run, or enable it at a later date using the `newaccess=1'
  option in chsfs.
\end{verbatim}
}
will be displayed.  If using Windows 3.1 with 32-bit disk access, this access
mode should not be used, as Windows uses the same mode and will detect SFS disk
accesses and block them.

If the drive is a SCSI device which needs a device driver to work with DOS, SFS
will access it directly as a SCSI device rather than simply a standard disk
drive.  SFS will work with drives accessed through ASPI (Adaptec SCSI
Programming Interface) and CAM (Common Access Method) drivers.  ASPI drivers
come with most SCSI drive controllers or can be purchased seperately.  The CAM
driver ASPICAM.SYS is available from NCR\footnote{
%Footnote [1]:
	       \hspace*{5pt} It is also available from the NCR FTP site ftp.ncr.com as part of
               the archive /pub/ncrchips/scsi/drivers/dos\_win/dos\_drv.zip.
}.

If direct SCSI access is possible, mksfs will display additional information on the drive,
typically:
{\small
\begin{verbatim}
  This drive is a MAXTOR XT-8760S SCSI drive attached to an ADAPTEC AHA-1x4x
  host with host ID 0,target ID 2, logical unit number 0. SFS will access it
  as a SCSI device rather than a normal hard drive.
\end{verbatim}
}
If mksfs is used to create an encrypted volume on this drive, it will
automatically access it with a SCSI access mode (equivalent to {\tt fastaccess=2})
without having to be told about it.

Once all tests have finished, mksfs will display the message:

\verb|  |{\tt Confidence test successfully concluded}

or an error count if errors occurred.  In either case, mksfs will exit after 
the tests have concluded without creating the encrypted volume.  If used with 
the {\tt -c} option, mksfs will never modify any information on disk, whether the 
tests are successful or not.  This is important, as it allows a confidence test 
to be performed before an encrypted volume is created.


\subsection{Problems with Windows}

The timed auto-unmount option and quick-unmount hotkey option are generally
unavailable under Windows as Windows disables the standard keyboard and timer
handling when it runs\footnote{
%Footnote [2]:
              Windows virtualizes the keyboard and timer interrupts and locks
              out SFS.  Although it is possible to bypass this, it must be done
              from within Windows itself, which is not possible for a device
              driver like SFS.
}.  In order to unmount a volume from within Windows,
the mountsfs program must be run explicitly.  The one exception to this rule is
that if a quick-unmount hotkey is set from within a DOS session then it will
remain available (but only within the DOS session) while that particular DOS
session is active.


\subsection{Problems with Other Software}

The Mitsumi CDROM device driver, if installed before another block driver like
SFS, will mistakenly try to use the drive letter allocated to the other driver
as its own one.  There have been reports of other CDROM drivers (in particular
the Sony one) which display similar traits (CDROM drivers are strange beasts
which have rather special requirements).  The DTC SCSI driver has a similar
problem in that it grabs more drive letters than DOS allocates to it, which
means that any block drivers loaded after it will be allocated drive letters by
DOS which are already being used by the SCSI driver.  The solution to this
problem is to make sure that the SFS driver is loaded before any problematic
CDROM or SCSI drivers by placing the {\tt DEVICE=SFS.SYS} line before the one which
loads the CDROM or SCSI driver in the CONFIG.SYS file.

The KEYB driver incorrectly handles the keyboard interrupt, which locks out the
SFS driver's quick-unmount hotkey handling if the {\tt HOTKEY={\em quick-unmount hotkey}} 
option is used at the time the driver is loaded.  If the {\tt HOTKEY=NONE} option
is specified when the SFS driver is installed, and the hotkey is set using
mountsfs after the KEYB driver has been loaded, everything works fine\footnote{
%Footnote [3]: 
		The KEYB driver provides a complete replacement for the BIOS int
              	9h keyboard driver.  KEYB is somewhat peculiar in its keyboard
             	handling, and doesn't coexist well with other keyboard handlers.
             	It also disables interrupts for lengthy periods of time while
              	processing keyboard scan codes.
}.  In
addition, SFS always acts as though the keyboard being used has the default
US-style layout, since the SFS software communicates directly with the keyboard
rather then working through driver software (which hasn't been loaded yet when
SFS is activated).  However since all SFS software performs the same keyboard
handling, this will only be noticed by SFS and should be transparent to the end
user.

Some (now very rare) device drivers and TSR's will destroy the contents of
32-bit registers when they are activated, which means that the data in the SFS
driver will become invalid from one machine instruction to the next.  There
have been reports of older versions of the PC-Kwik cache and Novell's
non-dedicated file server version 2.2 doing this.  A program to detect and
possibly fix this problem is available from garbo.uwasa.fi as
/pc/turbopas/trash.zip.

The Lantastic server software, version 6.0, can cause problems with mksfs.  If
running {\tt mksfs -c} reports errors then the {\tt server.exe} program should be
unloaded before mksfs is used to encrypt a DOS volume, and also before chsfs is
used with the {\tt convert} option.

Some of the Borland software development tools don't handle DOS critical errors
very well (they hang either when the error occurs or soon afterwards).  Since
trying to access a non-mounted volume is treated by DOS as an error, it may
cause programs like the IDE and the debugger to hang\footnote{
%Footnote [4]:
               This is the famous recursive footnote \thefootnote.
}.  Trying to read a 
floppy drive without a disk in the drive, and any other action which causes a 
DOS critical error, can have the same effect.

The Always Technology SCSI manager has a bug which makes use of SCSI devices
with logical unit numbers (LUN's) other than the default value of 0 impossible.
The SFS programs will detect this SCSI manager and avoid using devices with
LUN's other than 0.  In practice this will not be a problem since SCSI devices
normally have the LUN set to 0.

SFS will not work with S\&H Computer Systems' TSX multi-tasking operating
system, which doesn't support some disk utilities, DOS device drivers, and
programs which directly access hardware devices (which pretty well covers all
of what SFS does).


\subsection{Problems with hardware}

Some floppy drive and system BIOS combinations aren't terribly reliable.  It
has been reported that a laptop using the Phoenix 1.01 BIOS gives a multitude
of disk errors when encrypting a disk using mksfs.  The exact error type is
uncertain since the error code returned when the disk access fails is an
undefined value.  The Award 3.03 BIOS when used with some floppy drives also
causes problems, especially with newer versions of DOS (version 6.0 and up),
which may have great trouble reliably writing disks.  Microsoft's suggested
solution to the problem is a BIOS upgrade.


\section{Authentication of SFS Software}

There have been several occasions in the past when fake versions of software
have been distributed.  Sometimes these fake release are even wrapped up in a
nice-looking ``security envelope'' guaranteeing their authenticity.  With
encryption software like SFS it is all too tempting for an opponent to simply
create and distribute a compromised version of SFS rather than try to break the
SFS encryption itself.  In order to avoid any problems in this respect, the
distributed SFS driver and executables are accompanied by a digital signature
which can be used to verify that it is indeed an official version.

In order to check the authenticity of the particular version of SFS, you will
need the PGP encryption package, and my public key, which is included in the
standard PGP distribution.  My key is signed by Philip Zimmermann, the original
author of PGP, and several members of the PGP development team.  First, my key
should be checked for authenticity with the command:

{\tt \verb|  |pgp -kc "Peter Gutmann"}

When it performs the key check, PGP should display the following signatures:
{\small

\begin{verbatim}
 Type bits/keyID     Date    User ID
 pub  1024/997D47 1992/08/02 Peter Gutmann <pgut1@cs.aukuni.ac.nz>
 sig!      E722D9 1992/11/26   Branko Lankester <lankeste@fwi.uva.nl>
 sig!      997D47 1992/10/11   Peter Gutmann <pgut1@cs.aukuni.ac.nz>
 sig!      7C02F9 1992/09/07   Felipe Rodriquez <nonsenso@utopia.hacktic.nl>
 sig!      1336F5 1992/09/05   Harry Bush <Harry@castle.riga.lv>
 sig!      67F70B 1992/09/02   Philip R. Zimmermann <prz@sage.cgd.ucar.edu>
\end{verbatim}

}
There may be other signatures on there, but these are the ones from the PGP
development team and are the most important ones.  Version 2.1 and up of PGP
can, in addition, generate a key fingerprint for a key.  This can be calculated
with the command:

{\tt \verb|  |pgp -kvc "Peter Gutmann"}

PGP should display the following:
{\small

\begin{verbatim}
  pub  1024/997D47 1992/08/02 Peter Gutmann <pgut1@cs.aukuni.ac.nz>
         Key fingerprint = 7C 6D 81 DF F2 62 0F 4A  67 0E 86 50 99 7E A6 B1
\end{verbatim}

}
If the keyID or key fingerprint for my key differs from the one shown above or
the signatures don't check out, then the key is a probably a fake and shouldn't
be trusted.  Assuming the key is in order, the authenticity of the device
driver and the support software can be checked with:

{\tt \verb|  |pgp sfs.sig sfs.sys\\
     \verb|  |pgp $program$.sig $program$.exe}

where sfs.sig and $program$.sig are the digital signatures included with SFS as
distributed.  For example to check the authenticity of the mksfs program type:

{\tt \verb|  |pgp mksfs.sig mksfs.exe}

When it performs the check, PGP should display:

{\tt
\verb|  |Good signature from user Peter Gutmann $<$pgut1@cs.aukuni.ac.nz$>$ .\\
\verb|  |Signature made {\em date of signature}}

If PGP reports a bad signature then the executable shouldn't be trusted.  A
new, hopefully untouched, version can be obtained from any archive site, BBS,
or system which carries the standard SFS distribution, or it can be obtained
directly from the author.


\section{Applications}

Apart from the simple use of SFS for personal and business data privacy, there
are a number of other possible applications for which it can be used.  Some of
these are listed below.


\subsection{Secure Information Exchange}

If a communications channel is available between two systems which use SFS,
confidential data can be transferred from one encrypted SFS volume to the other
by using encryption on the communications channel.  For example a businessman
whose work involves a lot of travel could read data off the SFS volume on his
portable computer and encrypt it as it is sent via modem to his place of work.
At work the data could be decrypted and written to another SFS volume.  The
only time the data is available in unencrypted form is while it is being read
off the SFS volume and re-encrypted for transmission, which represents a
minimal risk as interrupting the transmission will involve stopping the program
which will (presumably) contain error handlers which erase any sensitive
information from memory.

Using a package like PGP (Pretty Good Privacy) or a PEM (Privacy-Enhanced Mail)
implementation in conjunction with SFS allows the secure distribution of 
information through an online service like a computer bulletin board.  The 
online system can retrieve the public key of the person requesting the 
information, read the required data off the SFS volume into the encryption 
program where it is encrypted with the recipients public key, and transmit it.  
At the other end the recipient will decrypt the data with their private key and 
write it straight onto their own SFS volume.  Again, the amount of time in 
which unencrypted data is available is minimal, and properly implemented 
software will destroy any sensitive information if interrupted in any way.


\subsection{Defence in Depth}

With the increasing strength of cryptographic software which is becoming
available to the public, means of compromising encryption security which don't
involve breaking the encryption itself are becoming more and more desirable.
This may involve things like creating fake versions of the encryption software
which have trapdoors in them and planting them in a victim's system, planting
versions which save the entered password somewhere and then restore the
original unaltered copies, or similar tricks.  This means that for maximum
security it is necessary to not only protect the password, but also to protect
the encryption software itself, and any software which interacts with it, and
anything which interacts with that, ad nauseum.  If several encryption and
security packages are used, every one of these must be protected separately.

By using SFS, some degree of protection is offered against malicious
manipulation, since an attacker must first get to the software stored on an SFS
volume in order to compromise it.  Storing other security-related software on
an encryption volume takes it out of the reach of any attack, but makes the SFS
software itself more of a target for an attack.  Eventually this problem can be
reduced somewhat through the use of SFS encryption hardware, which is currently
under (very gradual) development.  Another possibility is to store duplicate
copies of the SFS software on an encrypted volume which is initially mounted
read-only.  The versions on the SFS volume can be compared (using software also
stored on the SFS volume) with the unencrypted versions, and if they are
identical to the reference versions, write access to the volume can be enabled
and the volume used as normal.  Another possibility is to simply store
checksums or digital signatures for the SFS programs on the encrypted volume,
and only write-enable it if the checksums or signatures check out.

\subsection{Using SFS for Virus Protection}

SFS can be used as a form of virus protection for large collections of
computers by using it to create a centralised entry point for all data to the
system.  Consider a company operating 1,000 separate machines.  Normally this
would require 1,000 copies of a virus scanner to be installed and updated every
few months as new viruses appear.  In addition, use of the scanner on every one
of the 1,000 machines would have to be enforced rigorously.

An alternative is to install SFS on each of the machines, and make a policy
that only SFS-encrypted disks will be used within the company.  Then a single
scanner can be installed on a single machine, and all disks brought in from the
outside scanned and encrypted on that machine.

If every computer is initially virus-free, and all disks are SFS-encrypted,
then there are two possible means of attack for a virus.  The first is to
infect a file or disk when it is outside the company.  However as disks
originating from within the company are encrypted, no files (or, indeed,
anything) are visible on them, so there is nothing for a virus to infect (in
fact, DOS won't even recognise the disks as being formatted).  All disks
originating from outside the company have to be processed by the single
controlled computer before they can be used (or SFS will refuse to mount them), 
meaning that any known virus on a non-company disk should be picked up before 
the disk is encrypted.

Alternatively, a boot sector virus could infect an SFS-encrypted disk.
However, if an attempt is made to use the infected disk (which involves
mounting it), the mount will fail as the boot sector will contain the virus
rather than the SFS volume header.  The person who tried to mount the volume
will assume the disk has not been ``converted'' yet, and will bring it to the
machine used for processing the disks.  At this point the virus can be found by
the scanner.

This procedure isn't totally error free.  It won't work if there may already be
viruses present on one or more of the machines before SFS is installed.  In
addition, an SFS disk whose volume header is overwritten by a virus is probably
damaged beyond repair.  However it does provide a reasonable amount of
protection, and has the pleasant side effect of keeping all the company records
secure against unauthorized access attempts.

%\documentstyle[a4]{article}

%\begin{document}
%\parindent 0pt
%\parskip 2mm

\section{The Care and Feeding of Passwords}

With the inherent strength of an encryption system like the one used by SFS,
the password used for encryption is becoming more the focus of attack than the
encryption system itself.  The reason for this is that trying to guess an
encryption password is far simpler than trying to break the encryption system.

SFS allows keys of up to 100 characters in length.  These keys can contain
letters, numbers, spaces, punctuation, and most control and extended characters
except backspace (which is used for editing), escape (which is used to abort
the password entry), and carriage return or newline, which are used to signify
the end of the password.  This fact should be made use of to the fullest, with
preferred passwords being entire phrases rather than individual words (in fact
since very few words are longer than the SFS absolute minimum password length
of 10 characters, the complete set of these words can be checked in moments).
There exist programs designed to allow high-speed password cracking of standard
encryption algorithms which can, in a matter of hours (sometimes minutes, even
seconds in the case of very weak algorithms), attempt to use the contents of a
number of very large and complete dictionaries as sample passwords\footnote{
%Footnote [1]:
              	A large collection of dictionaries suitable for this kind of
              	attack can be found on black.ox.ac.uk in the `wordlists'
              	directory.  These dictionaries contain, among other things, 2MB
              	of Dutch words, 2MB of German words, 600KB of Italian words,
              	600KB of Norwegian words, 200KB of Swedish words, 3.3MB of
              	Finnish words, 1MB of Japanese words, 1.1MB of Polish words,
              	700KB of assorted names, and a very large collection of assorted
              	wordlists covering technical terms, jargon, the Koran, the works
              	of Lewis Carrol, characters, actors, and titles from movies,
              	plays, and television, Monty Python, Star Trek, US politics, US 
		postal areas, the CIA world fact book, the contents of several 
		large standard dictionaries and thesaurii, and common terms from 
		Australian, Chinese, Danish, Dutch, English, French, German, 
		Italian, Japanese, Latin, Norwegian, Polish, Russian, Spanish, 
		Swedish, Yiddish, computers, literature, places, religion, and 
		scientific terms.

              	The black.ox.ac.uk site also contains, in the directory
              	/src/security, the file cracklib25.tar.Z, a word dictionary of 
		around 10MB, stored as a 6.4MB compressed tar file.

              	A large dictionary of English words which also contains
              	abbreviations, hyphenations, and misspelled words, is available
              	from wocket.vantage.gte.com (131.131.98.182) in the 
		directory pub/standard\_dictionary as dic-0594.tar, an 
		uncompressed 16.1MB file, dic-0594.tar.Z, a compressed 7.6MB
              	file, dic-0594.tar.gz, a Gzip'ed 5.9MB file, and dic-0594.zip, a
              	Zipped 5.8MB file.  This contains around 1,520,000 entries.  In
              	combination with a Markov model for the English language built
              	from commonly-available texts, this wordlist provides a powerful
              	tool for attacking even full passphrases.

              A Unix password dictionary is available from ftp.spc.edu as
              .unix/password-dictionary.txt.

              Grady Ward $<$grady@netcom.com$>$ has collected very large
              collections of words, phrases, and other items suitable for
              dictionary attacks on cryptosystems.  Even the NSA has used his
              lists in their work.
}.  For
example one recent study of passwords used on Unix systems\footnote{
%Footnote [2]: 
		Daniel Klein, ``Foiling the Cracker: A Survey of, and Improvements
              	to, Password Security'', Software Engineering Institute, Carnegie
              	Mellon University.
} found 25\% of all
passwords simply by using sophisticated guessing techniques.  Of the 25\% total,
nearly 21\% (or around 3,000 passwords) were found within the first week using
only the spare processing power of a few low-end workstations.  368 were found
within the first few minutes.  On an average system with 50 users, the first
password could be found in under 2 minutes, with 5--15 passwords being found by
the end of the first day\footnote{
%Footnote [3]: 
		An improved implementation is approximately 3 times faster on an
              	entry-level 386 system, 4 times faster on an entry-level 486
              	system, and up to 10 times faster on a more powerful workstation
              	such as a Sparcstation 10 or DEC 5000/260, meaning that the first
              	password would be found in just over 10 seconds on such a
              	machine.
}.

Virtually all passwords composed of single words can be broken with ease in
this manner, even in the case of encryption methods like the one which is used
by SFS, which has been specially designed to be resistant to this form of
attack (doing a test of all possible 10-letter passwords assuming a worst-case
situation in which the password contains lowercase letters only, can be
accomplished in 450,000 years on a fast workstation (DEC Alpha) if the attacker
knows the contents of the encrypted volume in advance---or about 4 1/2 years on
a network of 100,000 of these machines).  Of course no attacker would use this
approach, as few people will use every possible combination of 10 letter
passwords.  By using an intelligent dictionary-based cracking program, this
time can be reduced to only a few months.  Complete programs which perform this
task and libraries for incorporation into other software are already widely
available\footnote{
%Footnote [4]: 
		One such program is ``crack'', currently at version 4.1 and
              	available from black.ox.ac.uk in the directory /src/security as
              	crack41.tar.Z.
}. This problem is especially apparent if the encryption algorithm used
is very weak---the encryption used by the popular Pkzip archiver, for example,
can usually be broken in this manner in a few seconds on a cheap personal
computer using the standard wordlist supplied with all Unix systems\footnote{
%Footnote [5]: 
		Actual cryptanalysis of the algorithm, rather than just trying
              	passwords, takes a little longer, usually on the order of a few
              	minutes with a low-end workstation.
}.

Simple modifications to passwords should not be trusted.  Capitalizing some
letters, spelling the words backwards, adding one or two digits to the end, and
so on, present only a slightly more difficult challenge to the average
password-cracker than plain unadorned passwords.  Any phrase which could be
present in any kind of list (song lyrics, movie scripts, books, plays, poetry,
famous sayings, and so on) should not be used---again, these can be easily and
automatically checked by computers.  Using foreign languages offers no extra
security, since it means an attacker merely has to switch to using
foreign-language dictionaries (or phrase lists, song lyrics, and so on).
Relying on an attacker not knowing that a foreign language is being used (``If I
use Swahili they'll never think of checking for it''---the so-called ``Security
through obscurity'' technique) offers no extra security, since the few extra
days or months it will take to check every known language are only a minor
inconvenience.

Probably the most difficult passwords to crack are ones comprising unusual
phrases or sentences, since instead of searching a small body of text like the
contents of a dictionary, book, or phrase list, the cracker must search a much
larger corpus of data, namely all possible phrases in the language being used.
Needless to say, the use of common phrases should be avoided, since these will
be an obvious target for crackers.

Some sample bad passwords might be:

\begin{center}
\begin{tabular}{l l}
    misconception        &     Found in a standard dictionary\\
    noitpecnocsim        &     Reversed standard dictionary word\\
    miskonseption        &     Simple misspelling of a standard word\\
    m1skon53pshun        &     Not-so-simple misspelling of a standard word\\
    MiScONcepTiON        &     Standard word with strange capitalization\\
    misconception1234    &     Standard word with simple numeric code appended\\
    3016886726           &     Simple numeric code, probably a US phone number\\
    YKYBHTLWYS           &     Simple mnemonic\\
\end{tabular}
\end{center}

In general coming up with a secure single-word password is virtually impossible
unless you have a very good memory for things like unique 20-digit numbers.

Some sample bad passphrases might be:

\begin{center}
\begin{tabular}{p{180.2pt} p{157pt}}
    What has it got in its     
     pocketses?                & Found in a common book\\
    Ph'n-glui mgl'w naf'h      
      Cthulhu R'yleh w'gah     & Found in a somewhat less common book\\
    For yesterday the word of  
      Caesar might have stood  & Found in a theatrical work\\
    modify the characteristics 
      of a directory           & Found in a technical manual\\
    T'was brillig, and the     
      slithy toves             & Found in a book of poetry\\
    I've travelled roads that  
      lead to wonder           & Found in a list of music lyrics\\
    azetylenoszilliert in      
      phaenomenaler kugel\-form  & Found in an obscure foreign journal\\
    Arl be back                & Found in several films\\
    I don't recall             & Associated with a famous person (although
                                 it does make a good answer to the question
                                 ``What's the password?'' during an
                                 interrogation)\\
\end{tabular}
\end{center}

Needless to say, a passphrase should never be written down or recorded in any
other way, or communicated to anyone else.

%Footnote [1]: A large collection of dictionaries suitable for this kind of
%              attack can be found on black.ox.ac.uk in the `wordlists'
%              directory.  These dictionaries contain, among other things, 2MB
%              of Dutch words, 2MB of German words, 600KB of Italian words,
%              600KB of Norwegian words, 200KB of Swedish words, 3.3MB of
%              Finnish words, 1MB of Japanese words, 1.1MB of Polish words,
%              700KB of assorted names, and a very large collection of assorted
%              wordlists covering technical terms, jargon, the Koran, the works
%              of Lewis Carrol, characters, actors, and titles from movies,
%              plays, and television, Monty Python, Star Trek, US politics, US postal areas, the
%              CIA world fact book, the contents of several large standard
%              dictionaries and thesaurii, and common terms from Australian,
%              Chinese, Danish, Dutch, English, French, German, Italian, Japanese,
%              Latin, Norwegian, Polish, Russian, Spanish, Swedish, Yiddish, computers,
%              literature, places, religion, and scientific terms.

%              The black.ox.ac.uk site also contains, in the directory
%              /src/security/cracklib25.tar.Z, a word dictionary of around 10MB,
%              stored as a 6MB compressed tar file.

%              A large dictionary of English words which also contains
%              abbreviations, hyphenations, and misspelled words, is available
%              from wocket.vantage.gte.com in the pub/standard_dictionary as
%              dic-0294.tar, an uncompressed 8.9MB file, or dic-0294.tar.Z, a
%              compressed 4.1MB file, and contains around 880,000 entries.  In
%              combination with a Markov model for the English language built
%              from commonly-available texts, this wordlist provides a powerful
%              tool for attacking even full passphrases.

%              A Unix password dictionary is available from ftp.spc.edu as
%              .unix/password-dictionary.txt.

%              Grady Ward <grady@netcom.com> has collected very large
%              collections of words, phrases, and other items suitable for
%              dictionary attacks on cryptosystems.  Even the NSA has used his
%              lists in their work.


\section{Other Software}

There are a small number of other programs available which claim to provide
disk security of the kind provided by SFS.  However by and large these tend to
use badly or incorrectly implemented algorithms, or algorithms which are known
to offer very little security.  One such example is Norton's Diskreet, which
encrypts disks using either a fast proprietary cipher or the US Data Encryption
Standard (DES).  The fast proprietary cipher is very simple to break (it can be
done with pencil and paper), and offers protection only against a casual 
browser.  Certainly anyone with any programming or puzzle-solving skills won't 
be stopped for long by a system as simple as this.

The more secure DES algorithm is also available in Diskreet, but there are
quite a number of implementation errors which greatly reduce the security it
should provide.  Although accepting a password of up to 40 characters, it then
converts this to uppercase-only characters and then reduces the total size to 8
characters of which only a small portion are used for the encryption itself.
This leads to a huge reduction in the number of possible encryption keys, so
that not only are there a finite (and rather small) total number of possible
passwords, there are also a large number of equivalent keys, any of which will
decrypt a file (for example a file encrypted with the key `xxxxxx' can be
decrypted with `xxxxxx', `xxxxyy', `yyyyxx', and a large collection of other
keys, too many to list here).

These fatal flaws mean that a fast dictionary-based attack can be used to check
virtually all possible passwords in a matter of hours in a standard PC.  In
addition the CBC (cipher block chaining) encryption mode used employs a known,
fixed initialisation vector (IV) and restarts the chaining every 512 bytes, 
which means that patterns in the encrypted data are not hidden by the 
encryption.  Using these two implementation errors, a program can be 
constructed which will examine a Diskreet-encrypted disk and produce the 
password used to encrypt it (or at least one of the many, many passwords 
capable of decrypting it) within moments.  In fact, for any data it encrypts,
Diskreet writes a number of constant, fixed data blocks (one of which contains
the name of the programmer who wrote the code, many others are simply runs of
zero bytes) which can be used as the basis of an attack on the encryption.
Even worse, the very weak proprietary scheme used by Diskreet gives away the
encryption key used so that if any two pieces of data are encrypted with the
same password, one with the proprietary scheme and the other with Diskreet's
DES implementation, the proprietary-encrypted data will reveal the encryption
key used for the DES-encrypted data.

These problems are in fact explicitly warned against in any of the documents
covering DES and its modes of operation, such as ISO Standards 10116 and
10126-2, US Government FIPS Publication 81, or basic texts like Denning's
``Cryptography and Data Security''.  It appears that the authors of Diskreet
never bothered to read any of the standard texts on encryption to make sure 
they were doing things right, or never really tested the finished version.  In 
addition the Diskreet encryption code is taken from a code library provided by
another company rather than the people who sell Diskreet, with implementation 
problems in both the encryption code and the rest of Diskreet.

The DES routines used in Da Vinci, a popular groupware product, are similarly
poorly implemented.  Not only is an 8-character password used directly as the
DES key, but the DES encryption method used is the electronic codebook (ECB)
mode, whose use is warned against in even the most basic cryptography texts
and, in a milder form, in various international encryption standards.  For
example, Annex A.1 of ISO 10116:1991 states ``The ECB mode is in general not
recommended''.  ISO 10126-2:1991 doesn't even mention ECB as being useful for
message encryption.  The combination of Da Vinci's very regular file structure
(which provides an attacker with a large amount of known data in every file),
the weak ECB encryption mode, and the extremely limited password range, makes a
precomputed dictionary attack (which involves a single lookup in a pre-set
table of plaintext-ciphertext pairs) very easy (even easier, in fact, than the
previously-discussed attack on Unix system passwords).  In fact, as ECB mode 
has no pattern hiding ability whatsoever, all that is necessary is to encrypt a 
common pattern (such as a string of spaces) with all possible dictionary 
password values, and sort and store the result in a table.  Any password in the 
dictionary can then be broken just as fast as the value can be read out of the 
table.

PC Tools is another example of a software package which offers highly insecure
encryption.  The DES implementation used in this package has had the number of
rounds reduced from the normal 16 to a mere 2, making it trivial to break on
any cheap personal computer.  This very weak implementation is distributed
despite a wide body of research which documents just how insecure 2-round DES
really is\footnote{
%Footnote [1]: 
		A 2-round version is in fact so weak that most attackers never
              	bother with it.  Biham and Shamirs ``Differential Cryptanalysis of
              	the Data Encryption Standard'' only starts at 4 rounds, for which
              	16 encrypted data blocks are needed for a chosen-plaintext
              	attack.  A non-differential, ciphertext-only attack on a 3-round
              	version requires 20 encrypted data blocks.  A known-plaintext
              	attack requires ``several'' encrypted data blocks.  A 2-round
              	version will be significantly weaker than the 3-round version.
              	It has been reported that a university lecturer once gave his
              	students 2-round DES to break as a homework exercise.
}.

Even a correctly-implemented and applied DES encryption system offers only
margi\-nal security against a determined attacker.  It has long been rumoured
that certain government agencies and large corporations (and, no doubt,
criminal organizations) possessed specialized hardware which allowed them to
break the DES encryption.  However only in August of 1993 have complete
constructional details for such a device been published.  This device, for
which the budget version can be built for around \$100,000, can find a DES key
in 3.5 hours for the somewhat more ambitious \$1 million version (the budget
version takes 1 1/2 days to perform the same task). The speed of this device
scales linearly with cost, so that the time taken can be reduced to minutes or
even seconds if enough money is invested.  This is a one-off cost, and once a
DES-breaking machine of this type is built it can sit there day and night
churning out a new DES key every few minutes, hours, or days (depending on the
budget of the attacker).

In the 1980's, the East German company Robotron manufactured hundreds of
thousands of DES chips for the former Soviet Union.  This means one of two
things: Either the Soviet Union used the chips to build a DES cracker, or they
used DES to encrypt their own communications, which means that the US built
one.

The only way around the problem of fast DES crackers is to run DES more than
once over the data to be encrypted, using so-called triple DES (using DES twice
is as easy to attack as single DES, so in practice three iterations must be
used).  DES is inherently slow.  Triple DES is twice as slow\footnote{
%Footnote [2]: 
		There are some clever tricks which can be used to make a triple
              	DES implementation only twice as slow as single DES, rather than
              	three times as slow as would be expected.
}.  A hard
drive which performs like a large-capacity floppy drive may give users a sense
of security, but won't do much for their patience.

The continued use of DES, mainly in the US, has been due more to a lack of any
replacement than to an ongoing belief in its security.  The National Bureau of
Standards (now National Institute of Standards and Technology) has only
reluctantly re-certified DES for further use every five years.  Interestingly
enough, the Australian government, which recently developed its own replacement
for DES called SENECA, now rates DES as being ``inappropriate for protecting
government and privacy information'' (this includes things like taxation
information and social security and other personal data).  Now that an
alternative is available, the Australian government seems unwilling to even
certify DES for information given under an ``in confidence'' classification,
which is a relatively low security rating.

Finally, the add-on ``encryption'' capabilities offered by general software
packages are usually laughable.  Various programs exist which will
automatically break the ``encryption'' offered by software such as Arc, Arj,
Lotus 123, Lotus Symphony, Microsoft Excel, Microsoft Word, Paradox, Pkzip 1.x,
the ``improved encryption'' in Pkzip 2.x, Quattro Pro, Unix crypt(1), Wordperfect
5.x and earlier, the ``improved'' encryption in Wordperfect 6.x, and many others
\footnote{
%Footnote [3]: 
		A package which will break many of these schemes is sold by
              	Access Data, 125 South 1025 East, Lindon, Utah 84042, ph.
              	1-800-658-5199 or 1-801-785-0363, fax 1-801-224-6009.  They
              	provide a free demonstration disk which will decrypt files that
              	have a password of 10 characters or less.  Access Data also have
                a UK distributor based in London called Key Exchange, ph.
                071-498-9005.
} \footnote{
%Footnote [4]: 
		A number of programs (too many to list here) which will break the
              	encryption of all manner of software packages are freely
              	available via the internet.  For example, a WordPerfect
              	encryption cracker is available from garbo.uwasa.fi in the
              	directory /pc/util as wppass2.zip.
}.
Indeed, these systems are often so simple to break that at least one package
which does so adds several delay loops simply to make it look as if there were
actually some work involved in the process. Although the manuals for these
programs make claims such as ``If you forget the password, there is absolutely
no way to retrieve the document'', the ``encryption'' used can often be broken
with such time-honoured tools as a piece of paper, a pencil, and a small amount
of thought.  Some programs which offer ``password protection security'' don't
even try to perform any encryption, but simply do a password check to allow
access to the data.  Two examples of this are Stacker and Fastback, both of
which can either have their code patched or have a few bytes of data changed to
ignore any password check before granting access to data.
\stepcounter{footnote} % I HATE those footnote jokes...
\footnotetext[\thefootnote]{
%Footnote [5]:
              Why are you reading this footnote?  Nowhere in the text is there
              a \thefootnote\/ referring you to this note.  Go back to the start, and
              don't read this footnote again!
}


\section{Data Security}

This section presents an overview of a range of security problems which are, in
general, outside the reach of SFS.  These include relatively simple problems
such as not-quite-deleted files and general computer security, through to
sophisticated electronic monitoring and surveillance of a location in order to
recover confidential data or encryption keys.  The coverage is by no means
complete, and anyone seriously concerned about the possibility of such an
attack should consult a qualified security expert for further advice.  It
should be remembered when seeking advice that an attacker will use any
available means of compromising the security of data, and will attack areas
other than those in which the strongest defense mechanisms have been installed.
All possible means of attack should be considered, since strengthening one area
may merely make another area more appealing to an opponent.


\subsection{Information Leakage}

There are several ways in which information can leak from an encrypted SFS
volume onto other media.  The simplest kind is in the form of temporary files
maintained by application software and operating systems, which are usually
stored in a specific location and which, when recovered, may contain file
fragments or entire files from an encrypted volume.  This is true not only for
the traditional word processors, spreadsheets, editors, graphics packages, and
so on which create temporary files on disk in which to save data, but also for
operating systems such as OS/2, Windows NT, and Unix, which reserve a special
area of a disk to store data which is swapped in and out of memory when more
room is needed.

This information is usually deleted by the application after use, so that the
user isn't even aware that it exists.  Unfortunately ``deletion'' generally
consists of setting a flag which indicates that the file has been deleted,
rather than overwriting the data in any secure way.  Any information which is
``deleted'' in this manner can be trivially recovered using a wide variety of
tools\footnote{
%Footnote [1]: 
		For example, more recent versions of MSDOS and DRDOS come with an
              	{\tt undelete} program which will perform this task.
}.  In the case of a swap file there is no explicit deletion as the swap
area is invisible to the user anyway. In a lightly-loaded system, data may
linger in a swap area for a considerable amount of time.

The only real solution to this is to redirect all temporary files and swap
files either to an encrypted volume or to a RAM disk whose contents will be
lost when power is removed.  Most programs allow this redirection, either as
part of the program configuration options or by setting the TMP or TEMP
environment variables to point to the encrypted volume or RAM disk.

Unfortunately moving the swap area and temporary files to an encrypted volume
results in a slowdown in speed as all data must now be encrypted.  One of the
basic premises behind swapping data to disk is that very fast disk access is
available.  By slowing down the speed of swapping, the overall speed of the
system (once swapping becomes necessary) is reduced.  However once a system
starts swapping there is a significant slowdown anyway (with or without
encryption), so the decision as to whether the swap file should be encrypted or
not is left to the individual user.

The other major form of information leakage with encrypted volumes is when
backing up encrypted data.  Currently there is no generally available secure
backup software (the few applications which offer ``security'' features are
ridiculously easy to circumvent), so that all data stored on an encrypted
volume will be backed up in unencrypted form.  Like the decision on where to
store temporary data and swap files, this is a tradeoff between security and
convenience.  If it were possible to back up an encrypted volume in its
encrypted form, the entire volume would have to be backed up as one solid block
every time a backup was made.  This could mean a daily five-hundred-megabyte
backup instead of only the half megabyte which has changed recently.
Incremental backups would be impossible.  Backing up or restoring individual
files would be impossible.  Any data loss or errors in the middle of a large
encrypted block could be catastrophic (in fact the encryption method used in
SFS has been carefully selected to ensure that even a single encrypted data bit
changed by an attacker will be noticeable when the data is decrypted\footnote{
%Footnote [2]:
              This is not a serious limitation, since it will only affect
              deliberate changes in the data.  Any accidental corruption due to
              disk errors will result in the drive hardware reporting the whole
              sector the data is on as being unreadable.  If the data is
              deliberately changed, the sector will be readable without errors,
              but won't be able to be decrypted.
}).

Since SFS volumes in their encrypted form are usually invisible to the
operating system anyway, the only way in which an encrypted volume can be
backed up is by accessing it through the SFS driver, which means the data is
stored in its unencrypted form.  This has the advantage of allowing standard
backup software and schedules to be used, and the disadvantage of making the
unencrypted data available to anyone who has access to the backups.  User
discretion is advised.

If it is absolutely essential that backups be encrypted, and the time (and
storage space) is available to back up an entire encrypted volume, then the
Rawdisk 1.1 driver, available as ftp.uni-duisburg.de:/pub/pc/misc/rawdsk11.zip,
may be used to make the entire encrypted SFS volume appear as a file on a DOS 
drive which can be backed up using standard DOS backup software.  The 
instructions which come with Rawdisk give details on setting the driver up to 
allow non-DOS volumes to be backed up as standard DOS drives.  The SFS volume 
will appear as a single enormous file RAWDISK.DAT which entirely fills the DOS 
volume.

%Footnote [2]: This is not a serious limitation, since it will only affect
%              deliberate changes in the data.  Any accidental corruption due to
%              disk errors will result in the drive hardware reporting the whole
%              sector the data is on as being unreadable.  If the data is
%              deliberately changed, the sector will be readable without errors,
%              but won't be able to be decrypted.


\subsection{Eavesdropping}

The simplest form of eavesdropping consists of directly overwiewing the system
on which confidential data is being processed.  The easiest defence is to
ensure that no direct line-of-sight path exists from devices such as computer
monitors and printers to any location from which an eavesdropper can view the
equipment in question.  Copying of documents and the contents of computer
monitors is generally possible at up to around 100 metres (300 feet) with
relatively unsophisticated equipment, but is technically possible at greater
distances.  The possibility of monitoring from locations such as
office-corridor windows and nearby rooms should also be considered.  This
problem is particularly acute in open-plan offices and homes.

The next simplest form of eavesdropping is remote eavesdropping, which does not
require access to the building but uses techniques for information collection
at a distance.  The techniques used include taking advantage of open windows or
other noise conveying ducts such as air conditioning and chimneys, using
long-range directional microphones, and using equipment capable of sensing
vibrations from surfaces such as windows which are modulated by sound from the
room they enclose.  By recording the sound of keystrokes when a password or
sensitive data is entered, an attacker can later recreate the password or data,
given either access to the keyboard itself or enough recorded keystrokes to
reconstruct the individual key sound patterns.  Similar attacks are possible
with some output devices such as impact printers.

Another form of eavesdropping involves the exploitation of existing equipment
such as telephones and intercoms for audio monitoring purposes.  In general any
device which handles audio signals and which can allow speech or other sounds
to be transmitted from the place of interest, which can be modified to perform
this task, or which can be used as a host to conceal a monitoring device and
provide power and possibly microphone and transmission capabilites to it (such
as, for example, a radio) can be the target for an attacker.  These devices can
include closed-circuit television systems (which can allow direct overviewing
of confidential information displayed on monitors and printers), office
communication systems such as public address systems, telephones, and intercoms
(which can either be used directly or modified to transmit sound from the
location to be monitored), radios and televisions (which can be easily adapted
to act as transmitters and which already contain power supplies, speakers (to
act as microphones), and antennae), and general electrical and electronic
equipment which can harbour a range of electronic eavesdropping devices and
feed them with their own power.

Another eavesdropping possibility is the recovery of information from hardcopy
and printing equipment.  The simplest form of this consists of searching
through discarded printouts and other rubbish for information.  Even shredding
a document offers only moderate protection against a determined enough
attacker, especially if a low-cost shredder which may perform an inadequate job
of shredding the paper is employed.  The recovery of text from the one-pass 
ribbon used in high-quality impact printers is relatively simple.  Recovery of 
text from multipass ribbons is also possible, albeit with somewhat more 
difficulty.  The last few pages printed on a laser printer can also be
recovered from the drum used to transfer the image onto the paper.

Possibly the ultimate form of eavesdropping currently available, usually
referred to as TEMPEST (or occasionally van Eck) monitoring, consists of
monitoring the signals generated by all electrically-powered equipment.  These
signals can be radiated in the same way as standard radio and television
transmissions, or conducted along wiring or other metal work.  Some of these
signals will be related to information being processed by the equipment, and
can be easily intercepted (even at a significant distance) and used to
reconstruct the information in question.  For example, the radiation from a
typical VDU can be used to recover data with only a receiver at up to 25m (75
feet), with a TV antenna at up to 40m (120 feet), with an antenna and
amplification equipment at up to 80m (240 feet), and at even greater distances
with the use of more specialised equipment\footnote{
%Footnote [1]: 
		These figures are taken from ``Schutzma\ss{}nahmen Gegen
              	Kompromittierende Elektromagnetische Emissionen von
              	Bildschirmsichtger\"aten'', Erhard M\"oller and Lutz Bernstein,
              	Labor f\"ur Nachrichtentechnik, Fachhochschule Aachen.
}.  Information can also be
transmitted back through the power lines used to drive the equipment in
question, with transmission distances of up to 100m (300 feet) being possible.

TEMPEST monitoring is usually relatively expensive in resources, difficult to \linebreak
mount, and unpredictable in outcome.  It is most likely to be carried out where 
other methods of eavesdropping are impractical and where general security 
measures are effective in stopping monitoring.  However, once in place, the 
amount of information available through this form of eavesdropping is immense.  
In general it allows the almost complete recovery of all data being processed 
by a certain device such as a monitor or printer, almost undetectably, and over 
a long period of time.  Protection against TEMPEST monitoring is difficult and 
expensive, and is best left to computer security experts\footnote{
%Footnote [2]: 
		TEMPEST information and shielding measures for protection against
              	TEMPEST monitoring are specified in standards like ``Tempest
              	Fundamentals'', NSA-82-89, NACSIM 5000, National Security Agency,
              	February 1, 1982, ``Tempest Countermeasures for Facilities Within
              	the United States'', National COMSEC Instruction, NACSI 5004,
              	January 1984, ``Tempest Countermeasures for Facilities Outside the
              	United States'', National COMSEC Instruction, NACSI 5005, January
              	1985, and MIL-STD 285 and 461B.  Unfortunately these
              	specifications have been classified by the organisations who are
              	most likely to make use of TEMPEST eavesdropping, and are not
              	available to the public.
} \footnote{
%Footnote [3]:
               A computer centre in Moscow had all its windows shielded with
               reflective aluminium film, which was supposed to provide enough
               protection to stop most forms of TEMPEST eavesdropping.  The
               technique seems to have worked, because a KGB monitoring van
               parked outside apparently didn't notice the fact that the
               equipment had been diverted to printing out Strugatsky's novels.
}.

However, some simple measures are still possible, such as paying attention to
the orientation of VDU's (most of the signal radiated from a VDU is towards the
sides, with very little being emitted to the front and rear), chosing equipment
which already meets standards for low emissions (for example in the US the
``quietest'' standard for computers and peripherals is know as the FCC Class B
standard), using well-shielded cable for all systems interconnections
(unshielded cable such as ribbon cable acts as an antenna for broadcasting
computer signals), using high-quality power line filters which block signals
into the high radio frequency range, and other methods generally used to reduce
or eliminate EMI (electromagnetic interference) from electronic equipment.


\subsection{Trojan Horses}

It may be possible for an attacker to replace the SFS software with a copy
which seems to be identical but which has major weaknesses in it which make an
attack much easier, for example by using only a few characters of the password
to encrypt the disk.  The least likely target is mksfs, since changing the way
this operates would require a similar change to mountsfs and the SFS driver
which would be easily detectable by comparing them with an independant,
original copy.  Since a changed mksfs would require the long-term use of a
similarly changed mountsfs and driver, the chances of detection are greatly
increased.

A much more subtle attack involves changing mountsfs.  By substituting a
version which saves the user's password or encryption key to an unused portion
of the disk and then replaces itself with an unmodified, original copy, an
attacker can return at their leisure and read the password or key off the disk,
with the user none the wiser that their encryption key has been compromised.
The SFS driver may be modified to do this as well, although the task is slighly
more difficult than changing mountsfs.

Detecting this type of attack is very difficult, since although it is possible
to use security software which detects changes, this itself might be modified
to give a false reading.  Software which checks the checking software may in
turn be modified, and so on ad infinitum.  In general someone who is determined
enough can plant an undetectable trojan\footnote{
%Footnote [1]:
              An attacker could employ, for example, what David Farber has
              described as ``supplemental functionality in the keyboard driver''.
}, although precautions like using
modification-detection programs, keeping physically separate copies of the SFS
software, and occasionally checking the installed versions against other,
original copies, may help reduce the risk somewhat.  The eventual creation of a
hardware SFS encryption card will reduce the risk further, although it is still
possible for an attacker to substitute their own fake encryption card.

Another possibility is the creation of a program unrelated to SFS which
monitors the BIOS character write routines for the printing of the password
prompt, or video RAM for the appearance of the prompt, or the BIOS keyboard
handler, or any number of other possibilities, and then reads the password as
it is typed in\footnote{
%Footnote [2]: 
		One program which performs this task is Phantom 2, available from
              	wuarchive.wustl.edu in the directory /pub/msdos/keyboard as
              	ptm228.zip, or from P2 Enterprises, P.O. Box 25, Ben Lomond,
              	California 95005-0025.  This program not only allows the
              	recording of all keystrokes but provides timing information down
              	to fractions of a second, allowing for detailed typing pattern
              	analysis by an attacker.

              Another keystroke recorder is Encore, also available from
              wuarchive.wustl.edu in the directory /pub/msdos/keyboard as
              encore.zip.
}.  This is a generic attack against all types of encryption
software, and doesn't rely on a compromised copy of the software itself.  

The stealth features in SFS are one way of making this kind of monitoring much
more difficult, and are explained in more detail in the section ``Security
Analysis'' below.  However the only really failsafe way to defeat this kind of
attack is to use custom hardware which performs its task before any user
software has time to run, such as the hardware SFS version currently under
development.

%Footnote [1]: An attacker could employ, for example, what David Farber has
%              described as ``supplemental functionality in the keyboard driver''.


\subsection{Dangers of Encryption}

The use of very secure encryption is not without its downsides.  Making the 
data completely inaccessible to anyone but the holder of the correct password 
can be hazardous if the data being protected consists of essential information 
such as the business records for a company which are needed in its day-to-day 
operation.  If the holder of the encryption password is killed in an accident 
(or even just rendered unconscious for a time), the potential complete loss of 
all business records is a serious concern.

Another problem is the question of who the holder of the password(s) should be.
If the system administrator at a particular site routinely encrypts all the
data held there for security purposes, then later access to the entire
encrypted dataset is dependant on the administrator, who may forget the
password, or die suddenly, or move on to another job and have little incentive
to inform their previous employer of the encryption password (for example if
they were fired from the previous job).  Although there are (as yet) no known
cases of this happening, it could occur that the ex-administrator has forgotten
the password used at his previous place of employment and might require a
small, five-figure consideration to help jog his memory.  The difficulty in
prosecuting such a case would be rather high, as proving that the encryption
system wasn't really installed in good faith by the well-intentioned
administrator to protect the company data and that the password wasn't
genuinely forgotten would be well nigh impossible.


\section{Politics}

Many governments throughout the world have an unofficial policy on cryptography
which is to reserve all knowledge and use of encryption to the government in
general and the elite in particular.  This means that encryption is to be used
firstly (in the form of restrictions on its use) for intelligence-gathering,
and secondly for protecting the secret communications of the government.
The government therefore uses encryption to protect its own dealings, but
denies its citizens the right to use it to protect their own privacy, and
denies companies the right to use it to protect their business data.  Only a
very small number of countries have laws guaranteeing citizens' rights to use
encryption\footnote{
%Footnote [1]: 
		One of these is Japan.  Article 21 of the Japanese Consitution
              	states:  ``Freedom of assembly and association as well as speech,
              	press, and all other forms of expression are guaranteed.  No
              	censorship shall be maintained, nor shall the secrecy of any
              	means of communication be violated''.}.

This policy is enforced in many ways.  In the US it is mainly through the use
of the ITAR, the International Traffic in Arms Regulations, a law which was
passed without public debate during the second world war and .  This defines all
encryption material (hardware and software) as ``munitions'', subject to special
governmental control.  France also classifies encryption devices as
munitions\footnote{
%Footnote [2]:
               The ``decret du 18 avril 1939'' defines 8 categories of arms and
               munitions from the most dangerous (1st category) to the least
               dangerous (8th category).  The ``decret 73-364 du 12 mars 1973''
               specifies that encryption equipment belongs to the second
               category.  Any usage of such equipment requires authorization
               from the Prime Minister.  The ``decret 86-250 du 18 fev 1986''
               extends the definition of encryption equipment to include
               software.  It specifies that each request for authorization for
               business or private usage of the equipment must be sent to the
               Minister of Telecommunications.  The request must include a
               complete and detailed description of the ``cryptologic process'',
               and if this is materially possible, of two copies of the
               envisaged equipment (see also Footnote \ref{footnote6}).  The ``loi 90-1170 du
               29 decembre 1990'' states that export or use of encryption
               equipment must be previously declared when used only for
               authentication, and previously authorized by the Prime Minister
               in all other cases, with penalties of fines of up to 500,000F and
               three months in jail.  Import of encryption equipment (but not
               encrypted data) is prohibited by the ``decret du 18 avril 1939'',
               article 11.  However the ``loi du 29 dec 1990'' only restricts use
               or export, not import, of encryption equipment.  There are no
               restrictions on the import of encrypted data.
 
               However these laws appear not to be enforced, with encryption
               software being freely imported, exported, available, and used in
               France.
}.  These ``munitions'' in fact have no violent uses beyond perhaps
beating someone to death with a box of disks.  Their only possible use is to 
protect personal privacy, and the privacy of business data.

In limiting the use (and export) of encryption technology\footnote{
%Footnote [3]:
               The reasoning behind this, as stated by the Permanent Select
               Committee on Intelligence in its commentary of 16 June 1994 on
               the HR.3937 Omnibus Export Administration Act is that ``the
               intelligence community's cryptologic success depends in part on
               controlling the use of encryption [\dots] controlling the
               dissemination of sophisticated encryption has been and will
               continue to be critical to those successes [of the US
               intelligence community] and US national security interests''.
}, the US (and many
other countries which follow the lead of the US) are not only denying their
citizens the means to ensure the privacy of personal information stored on
computer, they are also placing businesses at risk.  With no easy way to
protect their data, companies are losing billions of dollars a year to
industrial espionage which even a simple measure like SFS would help to reduce.
Some real-life examples of what the lack of secure encryption can do are:

\begin{itemize}

\item The head of the French DGSE (Direction Generale de la Securite
      Exterieure) secret service has publicly boasted that his organisation,
      through industrial espionage, helped French companies acquire over a
      billion dollars worth of business deals from foreign competitors\footnote{
%Footnote [4]
              This was reported by cryptographer Martin Hellman at the 1993 RSA
              Data Security conference on 14-15 January 1993.} \footnote{
%Footnote [5]
              Some quotes from FBI Director Louis Freeh from a talk given to
              the Executives' Club of Chicago on 17 March 1994:

              "A nation's power is increasingly measured by economic prosperity
               at home and competitiveness abroad.  And in some ways, the
               United States is a sitting duck for countries and individuals
               who want to take a short cut to power".

               [At least 20 nations are] "actively engaged in economic
               espionage".

              "This kind of information [cost and price structure, research and
               development results, marketing plans, bids and customer lists]
               can be intercepted from fax and satellite communications.  It
               can be monitored from cellular and microwave telephone links.
               It can be retrieved from inadequately protected computer
               systems".
} \footnote{
%Footnote [6]: 
\edef\@currentlabel{\csname p@footnote\endcsname\csname thefootnote\endcsname}
\label{footnote6}
		``Powerful computers scan telephone, fax, and computer data
               	traffic for information from certain sources, to certain
               	destinations, or containing certain keywords, and store any
               	interesting communications for later analysis.  The fear that
               	such monitoring stations will, after the end of the cold war, be
               	used for industrial espionage, has been expressed by DP managers
               	and tacitly confirmed by US security agencies''---Markt und
               	Technik, 18/94, page 49.
}.

\item The book ``Friendly Spies'' by Peter Schweitzer, published by Atlantic
      Monthly Press, gives many accounts about covert intelligence operations
      directed \linebreak against US corporations by cold war allies, with foreign
      governments conspiring with foreign companies to steal US technology and
      economic secrets\footnote{
%Footnote [7]:
               ``France and Germany and many other countries require US companies
                to `register' the encryption key for reasons of national
                security.  All of the American transmissions are monitored and
                the data is passed on to the local competitors.  Companies like
                IBM finally began to routinely transmit false information to
                their French subsidiary just to thwart the French Secret Service
                and by transitive property of economic nationalism, French
                computer companies''---RISKS-Forum Digest, 22 February 1993,
                Volume 14, Issue 34
}.

\item A US company was being consistently underbid by a Japanese competitor
      which was intercepting their electronic traffic.  When they started
      encrypting their messages, the underbidding stopped.  A few days later
      they were requested by the US government to stop using encryption in
      their traffic\footnote{
%Footnote [8]:
              Private communications from one of the people involved.
}.

\item A New Zealand computer dealer acquired 90 used disks which turned out to
      contain sensitive financial records from a large bank.  The evening after
      he turned them over to the bank (for a \$15,000 cash ``finders fee'') he was
      killed in a road accident.  The finders fee was never recovered\footnote{
%Footnote [9]:
              This event received nationwide TV, radio, and newspaper coverage
              at the time.  For example, any New Zealand paper published on 7
              September 1992 should provide more information.
}.

      Despite this major security problem, the bank wouldn't learn from their
      mistakes.  A few weeks later a large-capacity networked disk drive
      originally used by them was supplied to another company as a supposedly
      ``new'' replacement for a drive which had died.  This drive was found to
      contain the complete financial records from one of their branches.  Not
      wanting to be scraped off the side of the road one night, the system
      manager decided to erase the contents of the drive\footnote{
%Footnote [10]:
              Private communications from the system manager involved.
}.

      It isn't known how many more of their confidential financial records this
      bank has handed out to the public over the years.

\item The New Zealand Securities Commission accidentally left a number of
      sensitive files on the hard drive of one of a group of machines which was
      later sold at auction for \$100.  These files were stored without any
      security measures, and related to Securities Commission and Serious Fraud
      Office investigations.  At last report, the files had still not been
      recovered\footnote{
%Footnote [11]:
                This event received nationwide TV, radio, and newspaper coverage
                at the time.  Most New Zealand papers published on 13 August
                1994 contain coverage of the story.
}.

\item The book ``By Way of Deception'' by Victor Ostrovsky and Claire Hoy,
      published by St. Martins Press, New York, in 1990 (ISBN 0-312-05613-3),
      reports in Appendix I that Mossad's computer services routinely monitor
      VISA, AMEX, and Diner's Club transactions, as well as police computer
      traffic.

\end{itemize}

In the latter case the lack of encryption not only had the potential to cause
serious financial harm to the bank involved but resulted in the death of the
main player.  The use of a program like SFS would have made the contents of the
disks unreadable to anyone but the bank.

In 1991 the US Justice Department tried to introduce legislation that would
require all US digital communications systems to be reengineered (at enormous
cost) to support monitoring of message traffic by the FBI.  This measure was
never passed into law.  The next year the FBI tried to introduce a similar
measure, but could find no-one willing to back the bill.  In 1993, yet another
attmempt was made, which is currently being fought by an unusual coalition of
civil libertarians, computer users and companies, and communications providers.
A poll carried out by Time/CNN in March 1994 indicated that 2/3 of Americans
were opposed to the legislation\footnote{
%Footnote [12]: 
\edef\@currentlabel{\csname p@footnote\endcsname\csname thefootnote\endcsname}
\label{footnote12}
		``In a Time/CNN poll of 1,000 Americans conducted last week by
               	Yankelovich Partners, two thirds said it was more important to
               	protect the privacy of phone calls than to preserve the ability
               	of police to conduct wiretaps.  When informed about the Clipper
               	Chip, 80\% said they opposed it''---Philip Elmer-Dewitt, ``Who
               	Should Keep the Keys'', TIME, 14 March 1994.
}.

In April 1993, the US government announced a piece of hardware called the
Clipper Chip.  They proposed that this device, whose details are classified and
which contains self-destruct mechanisms which are activated if attempts are
made to examine it too closely, be built into all telephones.  Each chip has a
special serial number which identifies all communications carried out with that
phone.  At the beginning of each transmission, telephones equipped with a
Clipper Chip negotiate a connection which involves sending identifying
information across the phone network, and setting up a common key to use for
encrypting the conversation.

Built into this setup is a special back door which allows the government, and
anyone who works for the government, and anyone who has a friend who works for
the government, and anyone with enough money or force to bribe or coerce the
aforementioned people, to monitor the conversation\footnote{
%Footnote [13]:
                In June 1994, an AT\&T researcher discovered a means of bypassing
                this monitoring using about 28 minutes of computation time on
                easily-available mass-market Tessera cards.  By precomputing the
                values or employing many such devices running in parallel, the
                time can be reduced to virtually nothing.  This attack also
                opened up a number of other interesting possibilities for
                generally bypassing many of the perceived undesirable ``features''
                of Clipper.
}.  The job is made much
easier by the extra identification information which the Clipper Chip attaches
to the data stream.  The Clipper Chip allows monitoring on a scale even George
Orwell couldn't have imagined when he wrote his novel ``1984''\footnote{
%Footnote [14]:
              It has been claimed that the Clipper proposal is an example of
              the government servicing the people in the sense of the term
              used in the sentence ``The farmer got a bull to service his
              cows''.
}.  The Time/CNN
poll mentioned above found that 80\% of Americans were opposed to the Clipper
Chip$^{\ref{footnote12}}$.


A somewhat less blatant attempt to bypass communications privacy is gradually
appearing in the rest of the world.  The GSM digital telephone system uses a
special encryption algorithm called A5X which is a modified form of a stronger
system called A5.  A5X exists solely as a deliberately crippled A5, and is
relatively easy to bypass for electronic monitoring purposes.  Although the
details of A5 are classified ``for national security purposes''\footnote{
%Footnote [15]:
                In June 1994, the statement that A5 was too strong to disclose
                was suddenly changed so that it now became too weak to disclose,
                and that discussing the details might harm export sales.  This
                is an interesting contrast to the position taken in 1993 that
                sales to the Middle East might end up providing A5-capable
                equipment to the likes of Saddam Hussein.  Apparently there was
                a major debate among the NATO signal agencies in the 1980's over
                whether the GSM encryption should be strong or weak, with the
                weak encryption side eventually winning.
}, various 
sources have commented that even the original unmodified A5 probably provides 
only limited security against a determined attack, and the actual 
implementation exhibits some fundamental flaws (such as a 3-hour key rollover) 
which can greatly aid an attacker\footnote{
%Footnote [16]:
		It has been reported that GCHQ, the UK intelligence agency which
              	requested the A5X changes, regards as ``strong encryption'' (and
              	therefore not suitable for use by the public) anything which
              	can't be broken in real time.
} \footnote{
%Footnote [17]:
                UK cryptographer Ross Anderson has charecterised A5 as being
                ``not much good''.  A simple brute-force attack which searches all
                $2^{40}$ key combinations will break the system in about a week on a
                standard PC, with much faster attacks being possible using
                either better algorithms, custom hardware, or both.
                Interestingly, the low upper limit on the number of possible
                keys would also seem to meet the US government requirements for
                weak exportable encryption.
 
                Attacks faster than the basic brute-force one are also possible,
                and one such attack was to be presented by Dr Simon Shepherd at
                an IEE colloquium in London on 3rd June 1994.  However the talk
                was canceled at the last minute by GCHQ.
}.

It is against this worrying background that SFS was created.  Right from the
start, the idea behind SFS was to provide the strongest possible cryptographic
security.  No compromises were made, there are no back doors or weaknesses
designed into the system, nor will there ever be the deliberate crippling of
the system or undermining of its integrity which some organizations would like.
The algorithms and methods used in SFS have been selected specifically for
their acknowledged strength and general acceptance by the worldwide
cryptographic community, and conform to a wide variety of national and
international standards for secure encryption.  As new algorithms and
cryptographic processes appear, SFS will be updated to always provide the best
possible security available.


\section{An Introduction to Encryption Systems}

For space reasons the following introduction to encryption systems is very
brief.  Anyone requiring more in-depth coverage is urged to consult the texts
mentioned in the references at the end of this document.

Encryption algorithms (ciphers) are generally one of two types, block ciphers
and stream ciphers.  A block cipher takes a block of plaintext and converts the
entire block into ciphertext.  A stream cipher takes a single bit or byte of
plaintext at a time and converts it into ciphertext.  There also exist means of
converting block ciphers to stream ciphers, and vice versa.  Usually a stream
cipher is preferred, as it doesn't require data to be quantised to the block
size of the cipher in use.  Unfortunately, stream ciphers, although more
convenient, are usually harder to get right than block ciphers.  Many practical
stream ciphers are in fact simply block ciphers pretending to be stream
ciphers.

Virtually all good conventional-key ciphers are so-called product ciphers, in
which several (relatively) weak transformations such as substitution,
transposition, modular addition/multiplication, and linear transformation are
iterated over a piece of data, gaining more and more strength with each
iteration (usually referred to as a round).  These types of ciphers have been
extensively studied and are reasonably well understood.  The following table
compares the main parameters of several product ciphers.  Lucifer is the
immediate precursor to the US Data Encryption Standard (DES).  Loki is a
proposed alternative to DES.  FEAL is a fast block cipher designed in Japan.
IDEA is a relatively new Swiss block cipher which has been proposed as a
successor to DES and which has (so far) proven more resistant to attack then
DES.  MDC/SHS is a cipher based on the SHS one-way hash function (more on this
later).

\begin{center}
\begin{tabular}{|c|r|r|r|r|}
\hline
Cipher & Block size & Key size & Number of & Complexity of\\
       &            &  (bits)  &   rounds  &  Best Attack\\
\hline
Lucifer&    128     &    128   &     16    &     2$^{21}$\\
DES    &     64     &     56   &     16    &     2$^{43}$\\
Loki91 &     64     &     64   &     16    &     2$^{48}$\\
FEAL-8 &     64     &    128   &      8    &    10,000\\
IDEA   &     64     &    128   &      8    &     2$^{128}$\\
MDC/SHS&    160     &    512   &     80    &     2$^{512}$\\
\hline
\end{tabular}
\end{center}
%   +-----------+------------+----------+-----------+---------------+
%   |  Cipher   | Block size | Key size | Number of | Complexity of |
%   |           |            |  (bits)  |   rounds  |  Best Attack  |
%   +-----------+------------+----------+-----------+---------------+
%   |  Lucifer  |    128     |    128   |     16    |     2^21      |
%   |    DES    |     64     |     56   |     16    |     2^43      |
%   |  Loki91   |     64     |     64   |     16    |     2^48      |
%   |  FEAL-8   |     64     |    128   |      8    |    10,000     |
%   |   IDEA    |     64     |    128   |      8    |     2^128     |
%   |  MDC/SHS  |    160     |    512   |     80    |     2^512     |
%   +-----------+------------+----------+-----------+---------------+

The complexity of the best known attack is the number of operations necessary
to allow the cipher to be broken.  Note how the block size, key size, and
number of rounds don't necessarily give a good indication of how secure the
algorithm itself is.  Lucifer, although it has twice the block size and over
twice the key size of DES, is rather simple to break (the key size of DES is
discussed later on in the section on insecurities).  DES is the result of
several years of work on improvements to Lucifer.  FEAL has been continually 
changed every year or so when the previous version was broken.  Due to this, 
current versions are treated with some scepticism.  Both IDEA and MDC have so 
far resisted all forms of attack, although recently a class of weak keys have been
discovered in IDEA (and a simple change in the algorithm will eliminate these
weak keys).  Note that in the case of the last two algorithms the given
complexity is for a brute-force attack (explained below), which is the most
pessimistic kind possible.  There may be much better attacks available,
although if anyone knows of any they're not saying anything.  Of the algorithms
listed above, DES has been attacked the hardest, and IDEA and MDC the least,
which may go some way toward explaining the fact that brute force is the best 
known attack.

There are a large number of modes of operation in which these block ciphers can
be used.  The simplest is the electronic codebook (ECB) mode, in which the data
to be encrypted is broken up into seperate subblocks which correspond to the
size of the block cipher being used, and each subblock is encrypted
individually.  Unfortunately ECB has a number of weaknesses (some of which are
outlined below), and should never be used in a well-designed cryptosystem.
Using $P[]$ to denote a plaintext block, $C[]$ to denote a ciphertext block, $e()$ to
denote encryption, $d()$ to denote decryption, and $\oplus$ for the exclusive-or
operation, ECB mode encryption can be given as:
\begin{eqnarray*}
    C[ n ] & = & e( P[ n ] )
\end{eqnarray*}
with decryption being:
\begin{eqnarray*}
    P[ n ] & = & d( C[ n ] )
\end{eqnarray*}
A better encryption mode is cipher block chaining (CBC), in which the first
data subblock is exclusive-ored with an initialization vector (IV) and then
encrypted.  The resulting ciphertext is exclusive-ored with the next data
subblock, and the result encrypted.  This process is repeated until all the
data has been encrypted.  Because the ciphertext form of each subblock is a
function of the IV and all preceding subblocks, many of the problems inherent
in the ECB encryption mode are avoided.  CBC-mode encryption is:
\begin{eqnarray*}
    C[ 1 ] & = & e( P[ 1 ] \oplus IV )\\
    C[ n ] & = & e( P[ n ] \oplus C[ n-1 ] )
\end{eqnarray*}
and decryption is:
\begin{eqnarray*}
    P[ 1 ] & = & d( C[ 1 ] ) \oplus IV\\
    P[ n ] & = & d( C[ n ] ) \oplus C[ n-1 ]
\end{eqnarray*}
Another encryption mode is cipher feedback (CFB), in which the IV is encrypted
and then exclusive-ored with the first data subblock to provide the ciphertext.
The resulting ciphertext is then encrypted and exclusive-ored with the next
data subblock to provide the next ciphertext block.  This process is repeated
until all the data has been encrypted.  Because the ciphertext form of each
subblock is a function of the IV and all preceding subblocks (as is also the
case for CBC-mode encryption), many of the problems inherent in the ECB
encryption mode are avoided.  CFB-mode encryption is:
\begin{eqnarray*}
   C[ 1 ] & = & P[ 1 ] \oplus e( IV )\\
   C[ n ] & = & P[ n ] \oplus e( C[ n-1 ] )
\end{eqnarray*}
and decryption is:
\begin{eqnarray*}
    P[ 1 ] & = & C[ 1 ] \oplus e( IV )\\
    P[ n ] & = & C[ n ] \oplus e( C[ n-1 ] )
\end{eqnarray*}
There are several other modes of operation which are not covered here.  More
details can be found in the texts given in the references.

One point worth noting is that by using a different IV for each message in CBC
and CFB mode, the ciphertext will be different each time, even if the same
piece of data is encrypted with the same key.  This can't be done in ECB mode,
and is one of its many weaknesses.

There are several standard types of attack which can be performed on a
cryptosystem.  The most restricted of these is a ciphertext-only attack, in
which the contents of the message are unknown.  This kind of attack virtually
never occurs, as there is always some regularity or known data in the message
which can be exploited by an attacker.

This leads to the next kind of attack, the known-plaintext attack.  In this
case some (or all) of the plaintext of the message is known.  This type of
attack is fairly easy to mount, since most data consists of well-known,
fixed-format messages containing standard headers, a fixed layout, or data
conforming to a certain probability distribution such as ASCII text.

Finally, in a chosen-plaintext attack the attacker is able to select plaintext
and obtain the corresponding ciphertext.  This attack is also moderately easy
to mount, since it simply involves fooling the victim into transmitting a
message or encrypting a piece of data chosen by the attacker.  This kind of
attack was used to help break the Japanese ``Purple'' cipher during WWII by
including in a news release a certain piece of information which it was known
the Japanese would encrypt and transmit to their superiors.

However attacks of this kind are usually entirely unnecessary.  Too many
cryptosystems in everyday use today are very easy to break, either because the
algorithms themselves are weak, because the implementations are incorrect, or
because the way they are used is incorrect.  Often amateurs think they can
design secure systems, and are not aware of what an expert cryptanalyst could
do.  Sometimes there is insufficient motivation for anybody to invest the work
needed to produce a secure system.  Many implementations contain flaws which
aren't immediately obvious to a non-expert.  Some of the possible problems
include:

\begin{itemize}

\item Use of easily-searched keyspaces.  Some algorithms depend for their security
  on the fact that a search of all possible encryption keys (a so-called brute
  force attack) would take too long to be practical.  Or at least, it took too
  long to be practical when the algorithm was designed.  The Unix password
  encryption algorithm is a prime example of this.  The DES key space is
  another example.  Recent research has indicated that the DES was not in fact
  weakened by having only 56 bits of key material (as has often been claimed),
  since the inherent strength of the algorithm itself only provides this many
  bits of security (that is, that increasing the key size would have no effect
  since other attacks which don't involve knowing the key can be used to break
  the encryption in far less than 2$^{56}$ operations).  The encryption used in the
  Pkzip archiver can usually be broken automatically in less time than it takes
  to type the password in for authorized access to the data since, although it
  allows longer keys than DES, it makes the check for valid decryption keys
  exceedingly easy for an attacker.

\item Use of insecure algorithms designed by amateurs.  This covers the algorithms
  used in the majority of commercial database, spreadsheet, and wordprocessing
  programs such as Lotus 123, Lotus Symphony, Microsoft Excel, Microsoft Word,
  Paradox, Quattro Pro, WordPerfect, and many others.  These systems are so
  simple to break that the author of at least one package which does so added
  several delay loops to his code simply to make it look as if there was
  actually some work involved.

\item Use of insecure algorithms designed by experts.  An example is the standard
  Unix crypt command, which is an implementation of a rotor machine much like
  the German Enigma cipher which was broken during WWII.  There is a program
  called cbw (for `crypt breakers workbench') which can automatically decrypt
  data encrypted with crypt\footnote{
%Footnote [1]: 
		Available from black.ox.ac.uk in the directory /src/security as
		cbw.tar.Z.
  }.  After the war, the US government even sold 
  Enigma cipher machines to third-world governments without telling them that 
  they knew how to break this form of encryption.

\item Use of incorrectly-implemented algorithms.  Some encryption programs use the
  DES algorithm, which consists of a series of complicated and arbitrary-
  seeming bit transformations controlled by complex lookup tables.  These
  transformations and tables are very easy to get wrong.

  A well-known fact about the DES algorithm is that even the slightest
  deviation from the correct implementation significantly weakens the algorithm
  itself.  In other words any implementation which doesn't conform 100% to the
  standard may encrypt and decrypt data perfectly, but is in practice rather
  easier to break than the real thing.

  The US National Bureau of Standards (now the National Institute of Standards
  and Technology) provides a reference standard for DES encryption.  A
  disappointingly large number of commercial implementations fail this test.

\item Use of badly-implemented algorithms.  This is another problem which besets
  many DES implementations.  DES can be used in several modes of operation,
  some of them better than others.  The simplest of these is the Electronic
  Codebook (ECB) mode, in which a block of data is broken up into seperate
  subblocks which correspond to the unit of data which DES can encrypt or
  decrypt in one operation, and each subblock is then encrypted seperately.

  There also exist other modes such as CBC in which one block of encrypted data
  is chained to the next (such that the ciphertext block $n$ depends not only on
  the corresponding plaintext but also on all preceding ciphertext blocks
  \mbox{0..n-1}), and CFB, which is a means of converting a block cipher to a stream
  cipher with similar chaining properties.

  There are several forms of attack which can be used when an encrypted message
  consists of a large number of completely independant message blocks.  It is
  often possible to identify by inspection repeated blocks of data, which may
  correspond to patterns like long strings of spaces in text.  This can be used
  as the basis for a known-plaintext attack.

  ECB mode is also open to so-called message modification attacks.  Lets assume
  that Bob asks his bank to deposit \$10,000 in account number 12-3456-789012-3.
  The bank encrypts the message `Deposit \$10,000 in account number
  12-3456-789012-3' and sends it to its central office.  Encrypted in ECB mode
  this looks as follows:

  {\tt \verb|  |E( Deposit \$10,000 in acct. number 12-3456-789012-3 )}

  Bob intercepts this message, and records it.  The encrypted message looks as
  follows:

  {\tt \verb|     |H+2nx/GHEKgvldSbqGQHbrUfotYFtUk6gS4CpMIuH7e2MPZCe}

  Later on in the day, he intercepts the following a message:

  {\tt \verb|     |H+2nx/GHEKgvldSbqGQHbrUfotYFtUk61Pts2LtOHa8oaNWpj}

  Since each block of text is completely independant of any surrounding block,
  he can simply insert the blocks corresponding to his account number:

  {\tt \verb|     |................................gS4CpMIuH7e2MPZCe}

  in place of the existing blocks, and thus alter the encrypted data without
  any knowledge of the encryption key used.  Bob has since gone on to early
  retirement in his new Hawaiian villa.

  ECB mode, and the more secure modes such as CBC and CFB are described in
  several standards.  Some of these standards make a reference to the
  insecurity of ECB mode, and recommend the use of the stronger CBC or CFB
  modes.  Usually implementors stop reading at the section on ECB, with the
  result being that many commercial packages which use DES and which do manage
  to get it correct end up using it in ECB mode.

\item Protocol errors.  Even if a more secure encryption mode such as CBC or CFB
  mode is used, there can still be problems.  If a standard message format
  (such as the one shown above) is used, modification is still possible, except
  that now instead of changing individual parts of a message the entire message
  must be altered, since each piece of data is dependant on all previous parts.
  This can be avoided by prepending a random initialisation vector (IV) to each
  message, which propagates through the message itself to generate a completely
  different ciphertext each time the same message is encrypted.  The use of the
  salt in Unix password encryption is an example of an IV, although the range
  of only 4096 values is too small to provide real security.
\end{itemize}

In some ways, cryptography is like pharmaceuticals.  Its integrity is
important.  Bad penicillin looks just the same as good penicillin.  Determining
whether most software is correct or not is simple - just look at the output.
However the ciphertext produced by a weak encryption algorithm looks no
different from the ciphertext produced by a strong algorithm.... until an
opponent starts using your supposedly secure data against you, or you find your
money transfers are ending up in an account somewhere in Switzerland, or
financing Hawaiian villas.


\section{Security Analysis}

This section attempts to analyse some of the possible failure modes of SFS and
indicate which strategies have been used to minimise problems.


\subsection{Incorrect Encryption Algorithm Implementations}

When implementing something as complex as most encryption algorithms, it is
very simple to make a minor mistake which greatly weakens the implementation.
It is a well-known fact that making even the smallest change to the DES
algorithm reduces its strength considerably.  There is a body of test data
available as US National Bureau of Standards (NBS) special publication 500-20
which can be used to validate DES implementations.  Unfortunately the
programmers who produce many commercial DES implementations either don't know
it exists or have never bothered using it to verify their code (see the section
``Other Software'' above), leading to the distribution of programs which perform
some sort of encryption which is probably quite close to DES but which
nevertheless has none of the security of the real DES.

In order to avoid this problem, the SHS code used in SFS has a self-test
feature which can be used to test conformance with the data given in Federal
Information Processing Standards (FIPS) publication 180 and ANSI X9.30 part 2, 
which are the specifications for the SHS algorithm\footnote{
%Footnote [1]: 
		The FIPS 180 publication is available from the National Technical
              	Information Service, Springfield, Virginia 22161, for \$22.50 + \$3
              	shipping and handling within the US.  NTIS will take telephone
              	orders on +1 (703) 487-4650 (8:30 AM to 5:30 PM Eastern Time),
              	fax +1 (703) 321-8547.  For assistance, call +1 (703) 487-4679.
}.  This self-test can be 
invoked in the mksfs program by giving it the option {\tt -t} for `test':

{\tt  \verb|  |mksfs -t}

mountsfs and the SFS driver itself use exactly the same code, so testing it in
mksfs is enough to ensure correctness of the other two programs.  The following
tests can take a minute or so to run to completion.

The self-test, when run, produces the following output:

{\small \tt
  Running SHS test 1\dots passed, result=0164B8A914CD2A5E74C4F7FF082C4D97F1EDF880\\
  Running SHS test 2\dots passed, result=D2516EE1ACFA5BAF33DFC1C471E438449EF134C8\\
  Running SHS test 3\dots passed, result=3232AFFA48628A26653B5AAA44541FD90D690603}

The test values can be compared for correctness with the values given in
Appendix 1 of the FIPS publication.  If any of the tests fail, mksfs will exit
with an error message.  Otherwise it will perform a speed test and display a
message along the lines of:

{\small \tt
\verb|  |Testing speed for 10MB data\dots done.  Time = 31 seconds, 323 kbytes/second\\
\verb|  |All SHS tests passed
}

Note that the speed given in this test is only vaguely representative of the
actual speed, as the test code used slows the implementation down somewhat.  In
practice the overall speed should be higher than the figure given, while the
actual disk encryption speed will be lower due to the extra overhead of disk
reading and writing.


\subsection{Weak passwords}

Despite the best efforts of security specialists to educate users about the
need to choose good keys, people persist in using very weak passwords to protect
critical data.  SFS attempts to ameliorate this problem by forcing a minimum
key length of 10 characters and making a few simple checks for insecure
passwords such as single words (since the number of words of length 10 or more
characters is rather small, it would allow a very fast dictionary check on all
possible values).  The checking is only rudimentary, but in combination with
the minimum password length should succeed in weeding out most weak passwords.

Another possible option is to force a key to contain at least one punctuation
character, or at least one digit as some Unix systems do.  Unfortunately this
tends to lead people to simply append a single punctuation character or a fixed
digit to the end of an existing key, with little increase in security.

More password issues are discussed in the section ``The Care and Feeding of
Passwords'' above.


\subsection{Data left in program memory by SFS programs}

Various SFS utilities make use of critical keying information which can be used
to access an SFS volume.  Great care has been taken to ensure that all critical
information stored by these programs is erased from memory at the earliest
possible moment.  All encryption-related information is stored in static
buffers which are accessed through pointers passed to various routines, and is
overwritten as soon as it is no longer needed.  All programs take great care to
acquire keying information from the user at the last possible moment, and
destroy this information as soon as either the disk volume has been encrypted
or the keying information has been passed to the SFS driver.  In addition, they
install default exit handlers on startup which systematically erase all data
areas used by the program, both for normal exits and for error or
special-condition exits such as the user interrupting the programs execution.


\subsection{Data left in program memory by the SFS driver}

The SFS driver, in order to transparently encrypt and decrypt a volume, must
at all times store the keying information needed to encrypt and decrypt the
volume.  It is essential that this information be destroyed as soon as the
encrypted volume is unmounted.  SFS does this by erasing all encryption-related
information held by the driver as soon as it receives an unmount command.  In
addition, the driver's use of a unique disk key for each disk ensures that even
if a complete running SFS system is captured by an opponent, only the keys for
the volumes currently mounted will be compromised, even if several volumes are
encrypted with the same user password (see the section ``Controlled Disclosure
of Encrypted Information'' below for more details on this).


\subsection{Data left in system buffers by mksfs}

mksfs must, when encrypting a volume, read and write data via standard system
disk routines.  This data, consisting of raw disk sectors in either plaintext
or encrypted form, can linger inside system buffers and operating system or
hard disk caches for some time afterwards.  However since none of the
information is critical (the plaintext was available anyway moments before
mksfs was run, and at worst knowledge of the plaintext form of a disk sector
leads to a known plaintext attack, which is possible anyway with the highly
regular disk layout used by most operating systems), and since accessing any of
this information in order to erase it is highly nontrivial, this is not 
regarded as a weakness.


\subsection{Data left in system buffers by mountsfs}

As part of its normal operation, mountsfs must pass certain keying information
to the SFS driver through a DOS system call.  DOS itself does not copy the
information, but simply passes a pointer to it to the SFS driver.  After the
driver has been initialised, mountsfs destroys the information as outlined
above.  This is the only time any keying information is passed outside the
control of mountsfs, and the value is only passed by reference.


\subsection{Data left in system buffers by the SFS driver}

Like mksfs, the SFS driver reads and writes data via standard system disk
routines.  This data, consisting of raw disk sectors in either plaintext or
encrypted form, can linger inside system buffers and operating system or hard
disk caches for some time afterwards.  Once the encrypted volume is unmounted,
it is essential that any plaintext information still held in system buffers be
destroyed.

In order to accomplish this, mountsfs, when issuing an unmount command, 
performs two actions intended to destroy any buffered information. First, it 
issues a cache flush command followed by a cache reset command to any DOS drive 
cacheing software it recognizes, such as older versions of Microsoft's 
SmartDrive (with IOCTL commands), newer versions of SmartDrive (version 4.0 and 
up), the PCTools 5.x cache, Central Point Software's PC-Cache 6.0 and up, the 
more recent PC-Cache 8.x, Norton Utilities' NCache-F, NCache-S, and the newer 
NCache 6.0 and up, the Super PC-Kwik cache 3.0 and up, and Qualitas' QCache 4.0 
and up.  Some other cacheing software can be detected but doesn't support 
external cache flushing.  This step is handled by mountsfs rather than the 
driver due to the rather complex nature of the procedures necessary to handle 
the large variety of cacheing software, and the fact that most cacheing 
software can't be accessed from a device drvier.

After this the SFS driver itself issues a disk reset command which has the
effect of flushing all buffered and cached data scheduled to be written to a
disk, and of marking those cache and buffer entries as being available for
immediate use.  In addition to the explicit flushing performed by the mountsfs
program, many cacheing programs will recognise this as a signal to flush their
internal buffers (quite apart from the automatic flushing performed by the
operating system and the drive controller).

Any subsequent disk accesses will therefore overwrite any data still held in
the cache and system buffers.  While this does not provide a complete guarantee
that the data has gone (some software disk caches will only support a cache
flush, not a complete cache reset), it is the best which can be achieved
without using highly hardware and system-specific code.


\subsection{SFS volumes left mounted}

It is possible that an SFS volume may be unintentionally left mounted on an
unattended system, allowing free access to both the in-memory keying
information and the encrypted volume.  In order to lessen this problem
somewhat, a fast-unmount hotkey has been incorporated into the SFS driver which
allows an unmount command to be issued instantly from the keyboard (see the
sections ``Mounting an SFS Volume'' and ``Advanced SFS Driver Options'' above).
The ease of use of this command (a single keystroke) is intended to encourage
the unmounting of encrypted volumes as soon as they are no longer needed,
rather than whenever the system is next powered down.

As an extra precaution, the driver's use of a unique disk key for each disk
ensures that even if a complete running SFS system with encrypted volumes still 
mounted is captured by an opponent, only the key for the volume currently
mounted will be compromised, even if several volumes are encrypted with the
same user password (see the section ``Controlled Disclosure of Encrypted
Information'' below for more details on this).

Finally, a facility for an automatic timed unmount of volumes left mounted is
provided, so that volumes mistakenly left mounted while the system is
unattended may be automatically unmounted after a given period of time.  This
ensures that, when the inevitable distractions occur, encrypted volumes are
safely unmounted at some point rather than being left indefinitely accessible
to anyone with access to the system.


\subsection{Controlled disclosure of encrypted information}

To date there are no known laws which can be used to enforce disclosure of
encrypted information, a field which is usually covered by safeguards against
self-incrimination.  However there have been moves in both the US and the UK to
pass legislation which would compromise the integrity of encrypted information,
or which would remove protection against self-incrimination.  In either case
this would allow agencies to compel users of cryptographic software to reveal
the very information they are trying to protect, often without the users even
being aware that their privacy is being compromised.

The approach taken in the US, in the form of the Clipper initiative, is to have
all encryption keys held by the government.  If disclosure of the information
is required, the key is retrieved from storage and used to decrypt the
information.  A side effect of this is that any data which has ever been
encrypted with the key, and any data which will ever be encrypted in the 
future, has now been rendered unsafe.  This system is best viewed as 
uncontrolled disclosure.

SFS includes a built-in mechanism for controlled disclosure so that, if
disclosure is ever required by law, only the information for which access is
authorized may be revealed.  All other encrypted data remains as secure as it
was previously.

This is achieved by encrypting each disk volume with a unique disk key which is
completely unrelated to the users passphrase.  When the passphrase is entered,
it is transformed by iterating a one-way hash function over it several hundred
times to create an intermediate key which is used to decrypt the disk key
itself.  The disk volume is then en/decrypted with the disk key rather than the
unmodified encryption key supplied by the user.  There is no correlation
between the user/intermediate key and the disk key

If disclosure of the encrypted information is required, the disk key can be
revealed without compromising the security of the user key (since it is
unrelated to the user key), or the integrity of any other data encrypted with
the user key (since the disk key is unique for each disk volume, so that
knowledge of a particular disk key allows only the one volume it corresponds to
to be decrypted).

The controlled disclosure option is handled by the two mountsfs options {\tt -d}
and {\tt -c}.  The {\tt -d} option will disclose the unique key for a given disk
volume, and the {\tt -c} option will accept this key instead of the usual password.
For example to disclose the disk key for the SFS volume ``Data'', the command
would be:

{\tt \verb|  |mountsfs -d vol=data}

mountsfs will then ask for the password in the usual manner and prompt:

\begin{verbatim}
  You are about to disclose the encryption key for this SFS volume.
  Are you sure you want to do this [y/n]
\end{verbatim}

At this point a response of `Y' will continue and a response of `N' will exit
the program without disclosing the disk key.  A response of `Y' will print the
disk key in the following format:

\begin{verbatim}
 Disk key is

   2D 96 B1 06 6D E8 30 21 D5 DB 1B CC 3E 0E C7 CF D6 D1 0C 97 75 DC
   06 74 BC 2F E6 A0 3C 56 80 5F 0D 30 DA 54 D3 0D 28 F3 14 DE 79 67
   6E 1A 75 DC 33 87 86 29 BA A5 B1 64 5B 79 67 8C 8A 1B D1 27 5B 79
   73 6B 45 7B DA 54 43 6C C1 AB 06 67 7E 94 86 F2 50 22 09 8D 21 D5
   E7 3A 80 5F 0D 30 DA 54 D3 0D 28 FF 6D E8 30 21 33 87 86 29 C0 ED
   4A 22 96 B1 06 6D E8 30 21 D5 1C 2D E6 A0 3C 56 DA 54 43 6C 56 80
\end{verbatim}

This is the unique key needed to access the particular encrypted volume.

To use this key instead of the usual one with mountsfs, the command would be:

{\tt \verb|  |mountsfs -c vol=data}

mountsfs will then ask for this disk key instead of the usual password:

{\tt \verb|  |Please enter 264-digit disk key, [ESC] to exit:}

At this point the disk key should be entered.  mountsfs will automatically
format the data as it is entered to conform to the above layout.  The ESC key
can be used at any point to exit the process.

Once the key has been entered, the program will perform a validity check on the
key.  If this test fails, mountsfs will display the message:

{\tt \verb|  |Error: Incorrect disk key entered}

and exit.  Otherwise it will mount the volume as usual.  Since each volume has
its own unique disk key, revealing the key for one volume gives access to that
volume and no others.  Even if 20 volumes are all encrypted with the same user
password, only one volume can be accessed using a given disk key.


\subsection{Password Lifetimes and Scope}

An SFS password which is used over a long period of time makes a very tempting
target for an attacker.  The longer a particular password is used, the greater
the chance that it has been compromised.  A password used for a year has a far
greater chance of being compromised than one used for a day.  If a password is
used for a long period of time, the temptation for an attacker to spend the
effort necessary to break it is far greater than if the password is only a
short-term one.

The scope of a password is also important.  If a password is used to encrypt a
single drive containing business correspondence, it's compromise is only mildly
serious.  If it is employed to protect dozens of disk volumes or a large file
server holding considerable amounts of confidential information, the compromise
of the password could be devastating.  Again, the temptation to attack the
master password for an entire file server is far greater than for the password
protecting data contained on a single floppy disk.

SFS attacks this problem in two ways.  First, it uses unique disk keys to
protect each SFS volume.  The disk key is a 1024-bit cryptographically strong
random key generated from nondeterministic values acquired from system
hardware, system software, and user input (see the subsection ``Generating
Random Numbers'' below).  The data on each disk volume is encrypted using this
unique disk key, which is stored in encrypted form in the SFS volume header
(this works somewhat like session keys in PEM or PGP, except that a
conventional-key algorithm is used instead of a public-key one).  To access the
disk, the encrypted disk key is first decrypted with the user password, and the
disk key itself is used to access the disk.  This denies an attacker any known
plaintext to work with (as the plaintext consists of the random disk key).  To
check whether a given password is valid, an attacker must use it to decrypt the
disk key, rekey the encryption system with the decrypted disk key, and try to
decrypt the disk data.  Only then will they know whether their password guess 
was correct or not.  This moves the target of an attack from a (possibly
simple) user password to a 1024-bit random disk key\footnote{
%Footnote [1]: 
		SFS may not use the entire 1024 bits---the exact usage depends on
              	the encryption algorithm being used.
}.

The other way in which SFS tries to ameliorate the problem of password
lifetimes and scope is by making the changing of a password a very simple
operation.  Since the only thing which needs to be changed when a password
change is made is the encryption on the disk key, the entire password change 
operation can be made in a matter of seconds, rather than the many minutes it 
would take to decrypt and re-encrypt an entire disk.  It is hoped that the ease 
with which passwords can be changed will encourage the frequent changing of 
passwords by users.


\subsection{Trojan Horses}

The general problem of trojan horses is discussed in the section ``Data
Security'' above.  In general, by planting a program in the target machine which
monitors the password as it is entered or the data as it is read from or
written to disk, an attacker can spare themselves the effort of attacking the
encryption system itself.  In an attempt to remove the threat of password
interception, SFS takes direct control of the keyboard and various other pieces
of system hardware which may be used to help intercept keyboard access, making
it significantly more difficult for any monitoring software to detect passwords
as they are entered.  If any keystroke recording or monitoring routines are
running, the SFS password entry process is entirely invisible to them.  This
feature has been tested by a computer security firm who examined SFS, and found
it was the only program they had encountered (including one rated as secure for
military use) which withstood this form of attack.

Similarly, the fast disk access modes used by SFS can be used to bypass any
monitoring software, as SFS takes direct control of the drive controller
hardware rather than using the easily-intercepted BIOS or DOS disk access
routines.

Although these measures can still be worked around by an attacker, the methods
required are by no means trivial and will probably be highly system-dependant,
making a general encryption-defeating trojan horse difficult to achieve.


\section{Design Details}

This section goes into a few of the more obscure details not covered in the
section on security analysis, such as the encryption algorithm used by SFS, the
generation of random numbers, the handling of initialization vectors (IV's),
and a brief overview on the deletion of sensitive information retained in
memory after a program has terminated (this is covered in more detail in the
section ``Security Analysis'' above).


\subsection{The Encryption Algorithm used in SFS}

Great care must be taken when choosing an encryption algorithm for use in
security software.  For example, the standard Unix crypt(1) command is based on
a software implementation of a rotor machine encryption device of the kind
which was broken by mathematicians using pencil and paper (and, later on, some
of the first electronic computers) in the late 1930's\footnote{
%Footnote [1]: 
		This is covered in a number of books, for example Welchman's ``The
              	Hut Six Story: Breaking the Enigma Codes'', New York, McGraw-Hill
              	1982, and Kahns ``Seizing the Enigma'', Boston, Houghton-Mifflin
              	1991.
}.  Indeed, there exists 
a program called `crypt breaker's workbench' which allows the automated 
breaking of data encrypted using the crypt(1) command\footnote{
%Footnote [2]: 
		Available from black.ox.ac.uk in the directory /src/security as
              	cbw.tar.Z.
}.  The insecurity of 
various other programs has been mentioned elsewhere.  It is therefore 
imperative that a reliable encryption system, based on world-wide security 
standards, and easily verifiable by consulting these standards, be used.

When a block cipher is used as a stream cipher by running it in CFB (cipher
feedback) mode, there is no need for the cipher's block transformation to be a
reversible one as it is only ever run in one direction (generally
encrypt-only).  Therefore the use of a reversible cipher such as DES or IDEA is
unnecessary, and any secure one-way block transformation can be substituted.
This fact allows the use of one-way hash functions, which have much larger
block sizes (128 or more bits) and key spaces (512 or more bits) than most
reversible block ciphers in use today.

The transformation involved in a one-way hash function takes an initial hash
value $H$ and a data block $D$, and hashes it to create a new hash value $H'$:
\begin{eqnarray*}
    hash( H, D ) & \rightarrow & H'
\end{eqnarray*}
or, more specifically, in the function used in SFS:
\begin{eqnarray*}
    H + hash( D ) & \rightarrow & H'
\end{eqnarray*}
This operation is explained in more detail in FIPS Publication 180 and ANSI
X9.30 part 2, which defines the Secure Hash Standard.  By using $H$ as the data 
block to be encrypted and $D$ as the key, we can make the output value $H'$ 
dependant on a user-supplied key.  That is, when $H$ is the plaintext, $D$ is the 
encryption key, and $H'$ is the ciphertext:

%INCLUDE PICTURE HERE!!!

\begin{center}
\unitlength=1mm
\linethickness{0.4pt}
\begin{picture}(31.00,31.00)
\put(10.00,31.00){\makebox(0,0)[cb]{plaintext $H$}}
\put(10.00,30.00){\vector(0,-1){10.00}}
\put(00.00,10.00){\framebox(20.00,10.00)[cc]{SHS}}
\put(30.00,15.00){\vector(-1,0){10.00}}
\put(31.00,15.00){\makebox(0,0)[lc]{key $D$}}
\put(10.00,10.00){\vector(0,-1){10.00}}
\put(10.00,00.00){\makebox(0,0)[ct]{ciphertext $H'$}}
\end{picture}
\end{center}

%     plaintext H
%         |
%         v
%    +---------+
%    |   SHS   |<- key D
%    +---------+
%         |
%         v
%    ciphertext H'

If we regard it as a block cipher, the above becomes:
\begin{eqnarray*}
   H' & = & SHS( H ) 
\end{eqnarray*}
which is actually:
\begin{eqnarray*}
   C  & = &  e( P )
\end{eqnarray*}
Since we can only ever ``encrypt'' using a one-way hash function, we need to run
the ``cipher'' in cipher feedback mode, which doesn't require a reversible
encryption algorithm.

By the properties of the hash function, it is computationally infeasible to
either recover the key $D$ or to control the transformation $H \rightarrow H'$ (in other
words given a value for $H'$ we cannot predict the $H$ which generated it, and
given control over the value $H$ we cannot generate an arbitrary $H'$ from it).

The MDC encryption algorithm is a general encryption system which will take any
one-way hash function and turn it into a stream cipher running in CFB mode.  The 
recommended one-way hash function for MDC is the Secure Hash Standard as
specified in Federal Information Processing Standards (FIPS) Publication 180
and ANSI X9.30 part 2.  SHS is used as the block transformation in a block 
cipher run in CFB mode as detailed in AS 2805.5.2 section 8 and ISO 10116:1991 
section 6, with the two parameters (the size of the feedback and plaintext 
variables) j and k both being set to the SHS block size of 160 bits.  The 
properties of this mode of operation are given in Appendix A3 of AS 2805.5.2 
and Annex A.3 of ISO 10116:1991.  The CFB mode of operation is also detailed in 
a number of other standards such as FIPS Publication 81 and USSR Government 
Standard GOST 28147-89, Section 4.  The use of an initialization vector (IV) is 
as given in ISO 10126-2:1991 section 4.2, except that the size of the IV is 
increased to 160 bits from the 48 or 64 bits value given in the standard. This 
is again detailed in a number of other standards such as GOST 28147-89 Section 
3.1.2.  The derivation of the IV is given in the section ``Encryption 
Considerations'' below.

The key setup for the MDC encryption algorithm is performed by running the
cipher over the encryption key (in effect encrypting the key with MDC using
itself as the key) and using the encrypted value as the new encryption key.
This procedure is then repeated a number of times to make a ``brute-force''
decryption attack more difficult, as per the recommendation in the Public-Key
Cryptography Standard (PKCS), part 1.  This reduces any input key, even one
which contains regular data patterns, to a block of white noise equal in size
to the MDC key data.

The exact key scheduling process for MDC is as follows:

\begin{enumerate}

\item Initialization:

 \begin{itemize}
 \item The SHS hash value $H$ is set to the key IV\footnote{
%Footnote [3]:
              Some sources would refer to this value as a `salt'.  The term
              `key IV' is used here as this is probably a more accurate
              description of its function.
 }.
 \item The SHS data block $D$ is set to all zeroes.
 \item The key data of length 2048 bits is set to a 16-bit big-endian value
   containing the length of the user key in bytes, followed by up to 2032 bits
   of user key.

   SHS hash value $H$ = key IV;\\
   SHS data block $D$ = zeroes;\\
   key\_data [0:15] = length of user key in bytes;\\
   key\_data [16:2047] = user key, zero-padded;
 \end{itemize}

\item Key schedule:

 The following process is iterated a number of times:

   \begin{itemize}
   \item The 2048-bit key data block is encrypted using MDC.
   \item Enough of the encrypted key data is copied from the start of the key data
     block into the SHS data block $D$ to fill it.

   for i = 1 to 200 do\\
\verb|    |encrypted\_key = encrypt(key\_data);\\
\verb|    |$D$ = encrypted\_key;
   \end{itemize}
\end{enumerate}

During the repeated encryptions, the IV is never reset.  This means that the IV
from the end of the n-1 th data block is re-injected into the start of the n th
data block.  After 200 iterations, the ``randomness'' present in the key has been
diffused throughout the entire key data block.

Although the full length of the key data block is 2048 bits, the SHS algorithm
only uses 512 bits of this (corresponding to the SHS data block $D$) per
iteration.  The remaining 1536 bits take part in the computation (by being
carried along via the IV) but are not used directly.  By current estimates
there are around 2$^{256}$ atoms in the universe.  Compiling a table of all 2$^{512}$
possible keys which would be necessary for a brute-force attack on MDC would
therefore be a considerable challenge to an attacker, requiring, at least, the
creation of another $512 \times 2^{256}$ universes to hold all the keys.  Even allowing
for the current best-case estimate of a creation time of 7~days per universe,
the mere creation of storage space for all the keys would take an unimaginably
large amount of time.

The SFS key schedule operation has been deliberately designed to slow down
special hardware implementations, since the encryption algorithm is rekeyed
after each iteration.  Normal high-speed password-cracking hardware would (for
example, with DES) have 16 separate key schedules in a circular buffer, each
being applied to a different stage of a 16-stage pipeline (one stage per DES
round) allowing a new result to be obtained in every clock cycle once the
pipeline is filled.  In MDC the key data is reused multiple times during the 80
rounds of SHS, requiring 80 separate key schedules for the same performance as
the 16 DES ones.  However since the algorithm is rekeyed after every iteration
for a total of 200 iterations, this process must either be repeated 200 times
(for a corresponding slowdown factor of 200), or the full pipeline must be
extended to 16,000 stages to allow the one-result-per-cycle performance which
the 16-stage DES pipeline can deliver (assuming the rekeying operation can be
performed in a single cycle).  Changing the iteration count to a higher value
will further slow down this process.

The number of iterations of key encryption is controlled by the user, and is
generally done some hundreds of times.  The setup process in SFS has been tuned
to take approximately half a second on a workstation rated at around 15 MIPS
(corresponding to 200 iterations of the encryption process), making a
brute-force password attack very time-consuming.  Note that the key IV is
injected at the earliest possible moment in the key schedule rather than at the
very end, making the use of a precomputed data attack impossible.  The standard
method of injecting the encryption IV at the end of the key schedule process
offers very little protection against an attack using precomputed data, as it
is still possible to precompute the key schedules and simply drop in the
encryption IV at the last possible moment.

%Footnote [3]: Some sources would refer to this value as a `salt'.  The term
%              `key IV' is used here as this is probably a more accurate
%              description of its function.


\subsection{Generating Random Numbers}

One thing which cryptosystems consume in large quantities are random numbers.
Not just any old random value, but cryptographically strong random numbers.  A
cryptographically strong random value is one which cannot be predicted by an
attacker (if the attacker can predict the values which are used to set up 
encryption keys, then they can make a guess at the encryption key itself).
This automatically rules out all software means of generating random values,
and means specialised hardware must be used.

Very few PC's are currently equipped with this type of hardware.  However SFS
requires 1024 random bits for each encrypted disk, in the form of the disk key
(see the subsection ``Password Lifetimes and Scope'' above).  SFS therefore uses a
number of sources of random numbers, both ones present in the hardware of the
PC and one external source:

\begin{itemize}

\item Various hardware timers which are read occasionally when the program is
    running (generally after operations which are long and complex and will be
    heavily affected by external influences such as interrupts, video, screen,
    and disk I/O, and other factors.

\item The contents and status information of the keyboard buffer

\item Disk driver controller and status information

\item Mouse data and information

\item Video controller registers and status information

\item The clock skew between two hardware clocks available on the PC.  Due to
    background system activity such as interrupt servicing, disk activity, and
    variable-length instruction execution times, these clocks run out-of-phase.
    SFS uses this phase difference as a source of random numbers. 

    {\bf NB: Not implemented yet.}

\item The timing of keystrokes when the password is entered.  SFS reads the
    high-speed 1.19 MHz hardware timer after each keystroke and uses the timer
    values as a source of random numbers.  This timer is used both to measure
    keystroke latency when the password is entered and read at random times
    during program execution.  Trials have shown that this 16-bit counter
    yields around 8 bits of random information (the exact information content
    is difficult to gauge as background interrupts, video updates, disk
    operations, protected-mode supervisor software, and other factors greatly
    affect any accesses to this counter, but an estimate of 8 bits is probably
    close enough\footnote{
%Footnote [1]:
              If an opponent can obtain several hours of keystroke timings and
              can come up with a good model including serial correlations, they
              may be able to reduce the likely inputs to the random number
              accumulator to a somewhat smaller value, or at least bias their
              guesses to fall within the range of likely values.
}).

\item The timing of disk access latency for random disk reads.  The exact 
    operation is as follows:

    \begin{enumerate}
        \item Read a timer-based random disk sector
        \item Add its contents (8 bits)
        \item Read the high-speed 1.19 MHz hardware timer (13 bits)
        \item Use the two values for the next random sector
    \end{enumerate}

    This is repeated as often as required (in the case of SFS this is 10
    times).  Assuming a (currently rather optimistic) maximum of 5ms to acquire
    a sector this provides about 13 bits of randomness per disk operation.  The
    number of factors which influence this value is rather high, and includes
    among other things the time it takes the BIOS to process the request, the
    time it takes the controller to process the request, time to seek to the
    track on the disk, time to read the data (or, if disk cacheing is used,
    time for the occasional cache hit), time to send it to the PC, time to
    switch in and out of protected mode when putting it in the cache, and of
    course the constant 3-degree background radiation of other interrupts and
    system activity happening at the same time.  If a solid-state disk were
    being used, the hardware latency would be significantly reduced, but
    currently virtually no 386-class PC's have solid-state disks (they're
    reserved for palmtops and the like), so this isn't a major concern.
\end{itemize}

An estimate of the number of random bits available from each source is as
follows:

\begin{center}
\begin{tabular}{l|l}

    Keystroke latency, 8 bits per key     & 80 bits for minimum 10-char key\\
    Second password entry for encryption  & 80 bits for minimum 10-char key\\
    Disk access latency, 13 bits per read &130 bits for 10 reads\\
    Disk sector data, 8 bits              & 80 bits for 10 reads\\
    System clocks and timers              &  3 bits\\
    Video controller information          &  4 bits\\
    Keyboard buffer information           &  4 bits\\
    Disk status information               &  4 bits\\
    General system status                 &  4 bits\\
    Random high-speed timer reads         &120 bits for 15 reads\\
 \hline
    Total                                 &509 bits\\
\end{tabular}
\end{center}

These figures are very conservative estimates only, and are based on timing
experiments with typed-in passwords and a careful scrutiny of the PC's hardware
and system status data.  For example, although the time to access a disk sector
for a particular drive may be 10ms or more, the actual variation on that 10ms
may only be $\pm$2ms.  The figures given above were taken by averaging the
variation in times for large numbers of tests.  In practice (especially with 
longer passwords) the number of random bits is increased somewhat (for example 
with a 30-character password the total may be as high as 829 bits of random 
information).  However even the minimal estimate of 509 bits is adequate for 
the 512-bit key required by MDC.

Each quantum of semi-random information is exclusive-ored into a 1024-bit
buffer which is initially set to all zeroes.  Once 1024 bits of buffer have
been filled, the data is encrypted with MDC to distribute the information, and
the first 512 bits of the 1024-bit buffer is used as the key for the next MDC
encyrption pass.  Then more data is added until, again, 1024 bits of buffer
have been filled, whereupon the data is again mixed by encrypting it with MDC.
This process is repeated several times, with the amount of ``randomness'' in the
buffer increasing with each iteration.

Before being used, this data is encrypted 10 times over with MDC to ensure a
complete diffusion of randomness.  Since the IV for the encryption is reused
for the next pass through the buffer, any information from the end of the
buffer is thus reinjected at the start of the buffer on the next encryption
pass.

Although this method of generating random numbers is not perfect, it seems to
be the best available using the existing hardware.  General estimates of the
exact amount of truly random information which can be acquired in this manner
are in the vicinity of several hundred bits.  Proposed attacks all make the
assumption that an attacker is in possession of what amounts to a complete 
hardware trace of events on the machine in question.  Allowing for a reasonable 
amount of physical system security, it can be assumed that the random data used 
in SFS is unpredictable enough to provide an adequate amount of security 
against all but the most determined attacker.

%Footnote [1]: If an opponent can obtain several hours of keystroke timings and
%              can come up with a good model including serial correlations, they
%              may be able to reduce the likely inputs to the random number
%              accumulator to a somewhat smaller value, or at least bias their
%              guesses to fall within the range of likely values.


\subsection{Encryption Considerations}

When a block cipher is converted to handle units of data larger than its
intrinsic block size, a number of weaknesses can be introduced, depending on
the mode of operation which is chosen for the block cipher.  For example, if
two identical ciphertext blocks are present in different locations in a file,
this may be used to determine the plaintext.  If we can find two identical
blocks of ciphertext when cipher block chaining (CBC) is used, then we know
that:
\begin{eqnarray*}
    P[ i ] & = & d( C[ i ] ) \oplus C[ i-1 ]\\
    P[ j ] & = & d( C[ j ] ) \oplus C[ j-1 ]
\end{eqnarray*}
where $C$ is the ciphertext, $P$ is the plaintext, and $e()$ and $d()$ are encryption
and decryption respectively.  Now if $C[ i ] = C[ j ]$, then $d( C[ i ] ) =
d( C[ j ] )$, which cancel out when xor'd so that:
\begin{eqnarray*}
    P[ i ] \oplus C[ i-1 ] & = & P[ j ] \oplus C[ j-1 ]
\end{eqnarray*}
or:
\begin{eqnarray*}
    P[ j ] & = & P[ i ] \oplus C[ i-1 ] \oplus C[ j-1 ]
\end{eqnarray*}
Knowing $C[ i ]$ and $C[ j ]$ we can determine $P[ i ]$ and $P[ j ]$, and knowing
either $P[ i ]$ or $P[ j ]$ we can determine the other.

Something similar holds when cipher feedback (CFB) mode is used, except that
now the decryption operation is:
\begin{eqnarray*}
    P[ i ] & = & e( C[ i-1 ] ) \oplus C[ i ]\\
    P[ j ] & = & e( C[ j-1 ] ) \oplus C[ j ]
\end{eqnarray*}
Now if $C[ i ] = C[ j ]$ then $e( C[ i ] ) = e( C[ j ] )$ (recall that in CFB mode
the block cipher is only ever used for encryption), so that they again cancel
out, so:
\begin{eqnarray*}
    P[ i ] \oplus e( C[ i-1 ] ) & = & P[ j ] \oplus e( C[ j-1 ] )
\end{eqnarray*}
or:
\begin{eqnarray*}
   P[ i ] & = & P[ j ] \oplus e( C[ i-1 ] ) \oplus e( C[ j-1 ] )
\end{eqnarray*}
In general this problem is of little consequence since the probability of
finding two equal blocks of ciphertext when using a 160-bit block cipher on a
dataset of any practical size is negligible.  More esoteric modes of operation
such as plaintext feedback (PFB) and ones whose acronyms have more letters than
Welsh place names tend to have their own special problems and aren't considered
here.  

The problem does become serious, however, in the case of sector-level
encryption, where the initialization vector cannot be varied.  Although the IV
may be unique for each sector, it remains constant unless special measures such
as reserving extra storage for sector IV's which are updated with each sector
write are taken.  If a sector is read from disk, a small change made to part of
it (for example changing a word in a text file), and the sector written out to
disk again, several unchanged ciphertext/plaintext pairs will be present,
allowing the above attack to be applied.  However, there are cases in which
this can be a problem.  For example, running a program such as a disk
defragmenter will rewrite a large number of sectors while leaving the IV
unchanged, allowing an opponent access to large quantities of XOR'd plaintext
blocks simply by recording the disk contents before and after the
defragmentation process.  Normally this problem would be avoided by using a
different IV for each encrypted message, but most disk systems don't have the
room to store an entire sectors worth of data as well as the IV needed to
en/decrypt it.

An additional disadvantage of the CFB encryption mode is that the data in the
last block of a dataset may be altered by an attacker to give different
plaintext without it affecting the rest of the block, since the altered
ciphertext in the last block never enters the feedback loop.  This type of
attack requires that an opponent possess at least two copies of the ciphertext,
and that they differ only in the contents of the last block.  In this case the
last ciphertext block from one copy can be subsituted for the last ciphertext
block in the other copy, allowing a subtle form of message modification attack.
In fact in combination with the previously mentioned weakness of CFB, an
attacker can determine the XOR of the plaintexts in the last block and
substitute an arbitrary piece of ``encrypted'' plaintext to replace the existing
data.

There are several approaches to tackling this problem.  The most simplistic one
is to permute the plaintext in a key-dependant manner before encryption and
after decryption.  This solution is unsatisfactory as it simply shuffles the
data around without necessarily affecting any particular plaintext or
ciphertext block.  The desired goal of a change in any part of the plaintext
affecting the entire dataset is not achieved.

A better solution is to encrypt data twice, once from front to back and then
from back to front\footnote{
%Footnote [1]:
               To be precise, you need some sort of feedback from the end of
               a block on the first encryption pass to the start of the block
               on the next encryption pass.  A block can be encrypted forwards
               twice as long as the IV is wrapped back to the start of the 
               block for the second encryption pass.

}.  The front-to-back pass propagates any dependencies to
the back of the dataset, and the back-to-front pass propagates dependencies
back to the front again.  In this way a single change in any part of the
plaintext affects the entire dataset.  The disadvantage of this approach is
that it at least halves the speed of the encryption, as all data must be
encrypted twice. If the encryption is done in software, this may create an
unacceptable loss of throughput.  Even with hardware assistance there is a
noticeable slowdown, as no hardware implementations easily support backwards
encryption, requiring the data to be reversed in software before the second
pass is possible.

The best solution is probably to use a word-wise scrambler polynomial like the
one used in SHA.  With a block of plaintext P this is:
\begin{eqnarray*}
    P[ i ] & = & P[ i ] \oplus P[ i-K_1 ] \oplus P[ i-K_2 ]
\end{eqnarray*}
with suitable values for the constants $K_1$ and $K_2$.  If $K_2$ is chosen to be 5 (the
SHA block size in words) then the initial values of the 5 words (which can be
thought of as as $P[ -5 ]...P[ -1 ]$) are simply the sectorIV.  The value of $K_1$
is arbitrary, SFS uses a value of 4.

This technique is used by first setting the initial values of the 5 words to
the sectorIV.  The scrambler function is then run over the entire data block,
propagating all dependencies to the last 5 words in the block.  These last 5
words are then used as the IV for the CFB encryption of the entire block.  In
this way the encryption IV depends on all the bits in the block, and the
scrambling does a moderately good job of breaking up statistical patterns in
the plaintext.  No information is lost, so no randomness in the sectorIV is
misplaced.

This also provides resistance to the selective-modification attack which allows
an attacker to change selected bits in the last block of a CFB-encrypted
dataset without damage.  By destroying the IV used in the CFB encryption, the
first block is completely corrupted, which is unlikely to go unnoticed.

To decrypt a dataset encrypted in this manner, the first 5 words of ciphertext
are shifted into the feedback path, and the remainder of the dataset is
decrypted in the standard manner.  The last 5 decrypted words are then used as
the IV to decrypt the first encrypted block.  Finally, the scrambler is run
over the recovered plaintext to undo the changes made during the encryption
scrambling.

The overall en/decryption process used by SFS, in the case of 512-byte sectors
and 32-bit words (so that each sector contains 128 words), is:

\begin{itemize}
\item Encryption:
\begin{enumerate}
  \item using $sectorIV[ 0 ]...sectorIV[ 4 ]$ as the scrambler IV\\
            scramble $data[ 0 ]...data[ 127 ]$

  \item using $data[ 127-5 ]...data[ 127-1 ]$ as the encryption IV\\
            encrypt $data[ 0 ]...data[ 127 ]$
\end{enumerate}
\item Decryption:
\begin{enumerate}
  \item using $data[ 0 ]...data[ 4 ]$ as the encryption IV\\
            decrypt $data[ 5 ]...data[ 127 ]$

  \item using $data[ 127-5 ]...data[ 127-1 ]$ as the encryption IV\\
            decrypt $data[ 0 ]...data[ 4 ]$

  \item using $sectorIV[ 0 ]...sectorIV[ 4 ]$ as the scrambler IV\\
            scramble $data[ 0 ]...data[ 127 ]$
\end{enumerate}
\end{itemize}

where the scrambling operation is:
\begin{eqnarray*}
       data[ i ] & = & data[ i ] \oplus data[ i-4 ] \oplus data[ i-5 ]
\end{eqnarray*}
as outlined above.  Note that the i-4 and i-5 th values referred to here are
the original, scrambled values, not the descrambled values.  The easiest way to
implement this is to cache the last 5 scrambled values and cyclically overwrite
them as each word in the data buffer is processed.

%Footnote [1]:  To be precise, you need some sort of feedback from the end of
%               a block on the first encryption pass to the start of the block
%               on the next encryption pass.  A block can be encrypted forwards
%               twice as long as the IV is wrapped around back to the start of
%               the block for the second encryption pass.


\subsection{A Discussion of the MDC Encryption Algorithm\protect\footnote{
%Footnote [1]:
              Most of this analysis was contributed by Stephan Neuhaus,
              $<$neuhaus@informatik.uni-kl.de$>$
}}

(A word on notation:  The notation \{0,1\}$^k$ is used to mean the set of all bit
strings of length $k$, and \{0,1\}$^*$ means the set of all bit strings, including the
empty string.  Any message can be viewed as a bit string by means of a suitable
encoding).

The encryption method used by SFS is somewhat unusual, and in some respects is
similar to Merkle's ``Meta Method'' for obtaining cryptosystems\footnote{
%Footnote [2]: 
		This is discussed further in Ralph Merkle's paper ``One Way Hash
              	Functions and DES'', Crypto '89 Proceedings, Springer-Verlag,
              	1989 (volume 435 of the Lecture Notes in Computer Science
              	series).
}.  The method
relies on the existence of secure one-way hash functions.  A hash function is a
function that takes as input an arbitrary number of bits and produces a
fixed-sized output called the ``message digest''.  In other words, hash functions
have the form

    $h : \{0,1\}^* \rightarrow \{0,1\}^k              \mbox{ for some fixed $k$,}$

and the hash of a message $M$ is defined to be $h( M )$.  A secure one-way hash
function is a hash function with the following properties:

\begin{enumerate}

    \item For each message $M$, it is easy to compute $h( M )$.

    \item Given $M$, it is computationally infeasible to compute $M'$ with
       $h( M ) = h( M' )$ (secure against forgery).

    \item It is computationally infeasible to compute $M$ and $M'$ with
       $h( M ) = h( M' )$ (secure against collisions).
\end{enumerate}

For a good, but rather technical, discussion of hash functions, see
``Contemporary Cryptology. The Science of Information Integrity'' edited by 
Gustavus Simmons, IEEE Press, 1992 (ISBN 0-87942-277-7).

The terms ``easy to compute'' and ``infeasible to compute'' can be given more
precise definitions, but we'll settle for this informal terminology for now.  
Roughly speaking, ``easy to compute'' means that it will take a tolerable amount 
of time to compute the answer, even on a rather small machine; ``infeasible to 
compute'' means that it should take eons to find out a particular result, even 
when using millions of computers of the fastest conceivable technology in 
parallel.

Examples of hash functions include the MD2, MD4, and MD5 hash functions,
developed by Ron Rivest of RSA Data Security, Inc., which have been (at least
in the case of MD4 and MD5) placed in the public domain, and the Secure Hash
Standard SHS, developed by NIST (with significant input from the NSA).  The
existence of secure one-way hash functions has not been proven, although there
exist some strong candidates, including MD5 and SHS.

The reference implementations of the above hashing functions include one
interesting aspect which makes it possible to use them as encryption functions.
Since the hashing of a very large amount of data in one sweep is not desirable
(because all the data would have to be in memory at the time of hashing), most
hashing algorithms allow data to be hashed incrementally.  This is made 
possible by augmenting the definition of a hash function to include the state 
of the last hashing operation.  In other words, a hash function now has the 
form

    $h : \{0,1\}^k \times \{0,1\}^* \rightarrow \{0,1\}^k,$

where the first argument is the previous hash value, and the hash of a message
$M = ( M_1, M_2, ..., M_n )$ is defined to be
$h( h( ...( h( h_0 , M_1 ), M_2 ), ... ), M_n )$.

(The value of all the $h$ evaluations must not change if the message is broken up
into blocks of different lengths, but all of the previously mentioned hash
functions have that property).  Here, $h_0$ is a fixed, known initial value that
is used in all hashing calculations.

This is not the way ``real'' hash functions behave, but it is close enough.  For
example, the MD5 hashing function has ``initialization'', ``updating'', and
``finalization'' parts, where the finalization part appends the number of hashed
bytes to the message, hashes one final time, and returns the final hash value.
This means that the hashing ``context'' must include the number of bytes hashed
so far, without it being a part of the hash value.  The hash function can be
said to have ``memory''.

If we assume that $h$ is a secure one-way hashing function, we can now use such
an $h$ as a scrambling device.  For example, if we set $E( M ) = h( h_0, M )$ for
every message $M$, $M$ will not be recoverable from $E( M )$, because $h$ is secure by
definition.  Another method would be to supply $M$ to any standard MSDOS or UNIX
utility and use the resulting error message as the ciphertext (remembering that 
a computer is a device for turning input into error messages).  However, there
are still two problems to be solved before we can use hash functions as
encryption functions:

\begin{enumerate}
    \item The scrambling process is not controlled by a key.

    \item The scrambling process is not invertible, so there is no way to
       decrypt the ciphertext.
\end{enumerate}

Both problems can be solved by interchanging the roles of hash and data and by
using CFB mode in the encryption process.  In other words, let $K$ be an
arbitrarily long key, let $M = ( M_1, ..., M_n )$ be a message, broken up into 
chunks of $k$ bits, let IV be an initialization vector, and set
\begin{eqnarray*}
  C_1 & = & M_1 \oplus h( IV, K )\\
  C_i & = & M_i \oplus h( C( i-1 ), K )        \mbox{ for $1 < i \leq n$.}
\end{eqnarray*}
This is sent to the recipient, who easily recovers the plaintext by
\begin{eqnarray*}
  P_1 & = & C_1 \oplus h( IV, K )\\
  P_i & = & C_i \oplus h( C( i-1 ), K )        \mbox{ for $1 < i \leq n$,}
\end{eqnarray*}
since we have
\begin{eqnarray*}
    P_1 & = & ( M_1 \oplus h( IV, K ) ) \oplus h( IV, K )\\
        & = & M_1 \oplus ( h( IV, K ) \oplus h( IV,K ) ),\mbox{ because $\oplus$ is associative,}\\
        & = & M_1 \oplus 0,                              \mbox{ because $x \oplus x = 0$,}\\
        & = & M_1,                                       \mbox{ because $x \oplus 0 = x$,}
\end{eqnarray*}
and similarly for the $P_i$'s.  This method of encryption also offers more
security than using ECB mode, assuming that this were possible with hash
functions, since the plaintext is diffused over the entire ciphertext,
destroying plaintext statistics, and thus foiling straightforward ciphertext
correlation attacks.

This method can clearly be used for any hash function which can hash
incrementally.  Thus, it is a ``Meta Method'' for turning hash functions into
encryption functions.  This is called the Message Digest Cipher (MDC) method of
encryption.  Specific instances of the method have the name of the hash
function added as a suffix.  For example, the MDC method applied to the MD5
hash function would be referred to as MDC/MD5.  SFS uses MDC/SHS.

Having analysed the inner workings of MDC, at least one theoretical attack on
the algorithm should be mentioned.  There are certain properties of hash
functions which may make them unsuitable for use as encryption algorithms.  For
example suppose knowledge of a 160-bit input/output pair to SHS leaks a
fraction of a bit of information about the data being hashed, maybe a quarter
of a bit.  This allows a search of $2^{159.75}$ data blocks to find another data
block that causes the given input-output transformation, and thus finds a
second message which produces the same hash value.  This problem is not
significant when SHS is used as a cryptographic hash function, since it only
reduces the search space by 16\% from the full $2^{160}$ possibilities.  However
when SHS is used for encryption, it may be possible to accumulate these quarter
bits, so that after 2560 blocks (50K) of known plaintext, enough bits have been
accumulated to compute the encryption key.  This is because multiple
input/output pairs are available for a given data block, and each one puts more
constraints on the block until eventually you have the actual value can be
determined.

If a hash function is has the properties given above and no such information is
leaked, it can serve to create a strong encryption algorithm, but a serious
weakness in the encryption algorithm is not necessarily a serious weakness in
the hash function.  To date noone has ever demonstrated such a weakness, and
there are a number of similar ``what if'' arguments which can be used against
most encryption schemes.  For example if it were possible to build a quantum
computer then it could be used to break many of the most popular public-key
encryption schemes in use today.  The reason that these schemes aren't being
abandoned is that it is very unlikely that any computer of this form will be
built, and that if someone does manage it then the implications will be far
more serious than just the breaking of a number of encryption schemes.

%Footnote [1]: Most of this analysis was contributed by Stephan Neuhaus,
%              <neuhaus@informatik.uni-kl.de>


\subsection{Deletion of SFS Volumes}

Truly deleting data from magnetic media is very difficult.  The problem lies in
the fact that when data is written to the medium, the write head sets the
polarity of most, but not all, of the magnetic domains.  This is partially due
to the inability of the writing device to write in exactly the same location
each time, and partially due to the variations in media sensitivity and field
strength over time and among devices.

In general terms, when a one is written to disk, the media records a one, and
when a zero is written, the media records a zero.  However the actual effect is
closer to obtaining something like 0.95 when a zero is overwritten with a one,
and 1.05 when a one is overwritten with a one.  Normal disk circuitry is set up
so that both these values are read as ones, but using specialized circuitry it
is possible to work out what previous `layers' contained (in fact on some
systems it may be possible to recover previous data with a simple software
modification to the hardware control code).

This problem is further complicated by the fact that the heads might not pass
exactly over the same track position when data is rewritten, leaving a trace of
the old data still intact.  Current-generation drives reduce this problem
somewhat as track and linear densities have now become so high that the
traditional optical methods of extracting information from the disk platters
has become much more difficult, and in some cases impossible, as the linear bit
cell is below the optical diffraction limit for visible light.  While some data
patterns can still be discerned, recognizing others would be limited to some
subset of patterns.

Despite this small respite, when all the above factors are combined it turns
out that each track on a piece of media contains an image of everything ever
written to it, but that the contribution from each `layer' gets progressively
smaller the further back it was made.  Using techniques like low energy
electron scattering, ferrofluid with optical tracers, or related methods,
followed by the same signal-processing technology which is used to clean up 
satellite images, low-level recorded speech, and other data, it is possible to 
recover previous data with a remarkable degree of accuracy, to a level limited 
only by the sensitivity of the equipment and the amount of expertise of the 
organisation attempting the recovery.  Intelligence organisations have a {\em lot} 
of expertise in this field.

The basic concept behind the overwriting scheme used by SFS is to flip each
magnetic domain on the disk back and forth as much as possible (this is the
basic idea behind degaussing - magnetic media are limited in their ability to
store high-frequency oscillations, so we try to flip the bits as rapidly as
possible).  This means that the disk head should be run at the highest possible
frequency, and the same pattern should not be written twice in a row.  If the
data was encoded directly, that would mean a an alternating pattern of ones and
zeroes.  However, disks always use a NRZI encoding scheme in which a 1 bit
signifies an inversion, making the desired pattern a series of one bits.  This
leads to a further complication as all disks use some form of run-length
limited (RLL) encoding, so that the adjacent ones won't be written.  This
encoding is used so that transitions aren't placed too closely together, or too
far apart, which would mean the drive would lose track of where it was in the
data.

The basic limitation on disks is the proximity of 1 bits.  Floppies (which are
a more primitive form of the technology used in hard disks) like to keep the 1
bits 4$\mu$s apart.  However they can't be kept too far apart or the read clock
loses synchronisation.  This ``too far'' figure depends a lot on the technology
in the drive, it doesn't depend on the magnetic media much (unlike the ``too
close'' figure, which depends a lot on the media involved).  The first
single-density encoding wrote one user data bit, then one ``1'' clock bit, taking
a total of 8$\mu$s.  This was called FM, since a 1 bit was encoded as two
transitions (1 wavelength) per 8 us, while a 0 bit was encoded as one
transition (1/2 wavelength).

Then it was discovered that it was possible to have two 0 bits between adjacent
1s.  The resulting encoding of 4 bits into 5 was called group code recording
(GCR) which was (0,2) RLL.  This allowed 4 user data bits to be written in
$5 * 4\mu s = 20 \mu s$, for an effective time per user bit of 5 $\mu$s, which was a big
improvement over 8 $\mu$s.

But GCR encoding was somewhat complex.  A different approach was taken in
modified FM (MFM), which suppressed the 1 clock bit except between adjacent
0's.  Thus, 0000 turned into 0(1)0(1)0(1)0 (where the ()s are the inserted
clock bits), 1111 turned into 1(0)1(0)1(0)1, and 1010 turned into
1(0)0(0)1(0)0.  The maximum time between 1 bits was now three 0 bits.  However,
there was at least one 0 bit, so it became possible to clock the bits at
2 $\mu$s/bit and not have more than one 1 bit every 4 $\mu$s.  This achieved one user bit
per 4 $\mu$s, a result which was better than GCR and obtainable with simpler
circuitry.  As can be seen, the effective data rate depends on the bit rate
(which has been 4 $\mu$s, 4 $\mu$s and 2 $\mu$s in these examples) and the encoding rate, a
measure of the encoding efficiency. The encoding rates have been 1/2, 4/5 and
1/2.

There is a (2,7) RLL code with rate 1/2, meaning that 1 user bit goes to 2
encoded bits (although the mapping involves multi-bit groups and is somewhat
complex), but there are always at least two 0 bits between 1 bits, so 1 bits
happen at most every 3 bit times.  This allows the clock to be reduced to
1.3333 $\mu$s (this 2/1.33333 = 6/4 = 3/2 is the source of the added capacity gain
of RLL hard drive controllers over equivalent MFM controllers).  The time per
user bit is now 2.6666 = 2 2/3 $\mu$s.

However, the faster clock is finicky.  It is also possible to use a (1,7) RLL
encoding.  Since this is (1,x), the clock rate is 2 $\mu$s/bit, but the encoding
efficiency improves from 1/2 to 2/3.  This allows 2 effective user bits per 6
$\mu$s, or 3 $\mu$s per user bit.  For hard drives, it is easier to increase the clock
rate by an extra 9/8 than to fiddle with a clock 50\% faster, so this is very
popular with more recent disk drives.

The three most common RLL codes are (1,3) RLL (usually known as MFM), (2,7) RLL
(the original ``RLL'' format), and (1,7) RLL (which is popular in newer drives).
The origins of these codes are explained in more details below.  Fortunately,
each of these three have commonly-used encoding tables.  A knowledge of these
tables can be used to design overwrite patterns with lots of transitions after
being encoded with whatever encoding technique the drive uses.

For MFM, the patterns to write to produce this are 0000 and 1111.  So a couple
of rounds of this should be included in the overwrite pattern. MFM drives are
the oldest, lowest-density drives around (this is especially true for the
very-low-density floppy drives).  As such, they're the easiest to recover data
from with modern equipment and we need to take the most care with them.

For (1,7) RLL, the patterns to write are 0011 and 1100, or 0x33 and 0xCC when
expressed as bytes.  To provide some security against bit misalignment, the
values 0x99 and 0x66 should be included as well (although drive manufacturers
like to keep things byte-aligned, so this sort of bit misalignment is unlikely
to happen).

For (2,7) RLL drives, three patterns are necessary, and the problem of byte
endianness question rears its head.  The previous two cases are not
significantly affected by shifting the bytes around, but this one is.
Fortunately, thanks to the strong influence of IBM mainframe drives, everything
seems to be uniformly big-endian within bytes (that is, the most significant
bit is written to the disk first).

For (2,7) RLL using the standard tables:

\begin{center}
\begin{tabular}{l c l}
  10   & $\rightarrow$ & 0100\\
  11   & $\rightarrow$ & 1000\\
  000  & $\rightarrow$ & 00100\\
  010  & $\rightarrow$ & 100100\\
  011  & $\rightarrow$ & 001000\\
  0010 & $\rightarrow$ & 00100100\\
  0011 & $\rightarrow$ & 00001000\\
\end{tabular}
\end{center}

the bit patterns 100100100\dots, 010010010\dots and 001001001\dots will encode into
the maximum-frequency encoded strings 010010010010\dots (0x49, 0x24, 0x92),
10010010\-0100\dots (0x92, 0x49, 0x24), and 001001001001\dots (0x24, 0x92, 0x49).
Writing all three of these patterns will cover all the bases.

For the latest crop of high-density drives which use methods like Partial-
Response Maximum-Likelihood (PRML) methods which may be roughly equated to the
trellis encoding done by V.32 modems in that it is effective but
computationally expensive, all we can do is write a variety of random patterns, 
because the processing inside the drive is too complex to second-guess.  
Fortunately, these drives push the limits of the magnetic media much more than 
the old MFM drives ever did by encoding data with much smaller magnetic 
domains, closer to the physical capacity of the magnetic media.  If these 
drives require sophisticated signal processing just to read the most recently 
written data, reading overwritten layers is also correspondingly more
difficult.  A good scrubbing with random data will do about as well as can be
expected.

To deal with all these types of drives in one overwrite pattern, SFS uses the
following sequence of 30 consecutive writes to erase data:

\begin{enumerate}
   \item  Random
   \item  0000..., MFM encoded to 01010101...
   \item  1111..., MFM encoded to 10101010...
   \item  Random
   \item  010010..., (2,7) RLL encoded to 100100100100...
   \item  100100..., (2,7) RLL encoded to 010010010010...
   \item  001001..., (2,7) RLL encoded to 001001001001...
   \item  Random
   \item  00110011..., (1,7) RLL encoded to 010101010101...
  \item  11001100..., (1,7) RLL encoded to 101010101010...
  \item  01100110..., (1,7) misaligned RLL encoded to 010101010101...
  \item  10011001..., (1,7) misaligned RLL encoded to 101010101010...
  \item  Random
  \item  1111..., MFM encoded to 10101010...
  \item  0000..., MFM encoded to 01010101...
  \item  Random
  \item  001001..., (2,7) RLL encoded to 001001001001...
  \item  100100..., (2,7) RLL encoded to 010010010010...
  \item  010010..., (2,7) RLL encoded to 100100100100...
  \item  Random
  \item  11001100..., (1,7) RLL encoded to 101010101010...
  \item  00110011..., (1,7) RLL encoded to 010101010101...
  \item  10011001..., (1,7) misaligned RLL encoded to 101010101010...
  \item  01100110..., (1,7) misaligned RLL encoded to 010101010101...
  \item  Random
  \item  0000..., MFM encoded to 01010101...
  \item  1111..., MFM encoded to 10101010...
  \item  Random
  \item  Random
  \item  Random
\end{enumerate}

All patterns are repeated twice, once in each order, and MFM is repeated three
times, because MFM drives are generally the lowest density, and thus
particularly easy to examine.  If the device being written to supports cacheing
or buffering of data, SFS will attempt to disable the buffering of data to
ensure physical disk writes are performed (for example by setting the Force
Unit Access bit during SCSI-2 Group 1 write commands).

There is a commonly-held belief that there is a US government standard for
declassifying magnetic media which simply involves overwriting data on it three 
times.  There are in fact a number of standards\footnote{
%Footnote [1]:
               Among others there is the Department of Defense standard DoD
               5200.28-M, Army standard AR 380-19, Navy standards OPNAVINST
               5510.1H and NAVSO P-5239-26, Air Force standard AFSSI-5020, and
               Department of Energy standard DOE 5637.1.
} which contain simple phrases
such as ``Magnetic disks, drums, and other similar rigid storage devices shall
be sanitized by overwriting all storage locations with any character, then the
complement of the character (e.g., binary ones and binary zeros) alternately a
minimum of three times''.  However this simple description is usually
reinterpreted by the appropriate government agencies to a level which often
makes physical destruction of the media and its replacement with new media
preferable to attempting any declassification by overwriting the data (the
(classified) standards for truly declassifying magnetic media probably involve
concentrated acid, furnaces, belt sanders, or any combination of the above\footnote{
%Footnote [2]: 
		The UK Ministry of Defence grinds down disk platters and then
              	supposedly stores the (still-classified) dust for a decade or
                more.  Rumours that they remove programmers brains for storage in 
                order to erase the classified information they contain are
                probably unfounded.
}).

The use of such extreme measures is necessary not only because data recovery 
from the actual tracks itself may (with some effort) be possible, but because 
of factors like intelligent drives which keep so-called ``alternative cylinders'' 
on a disk free to allow dynamic re-allocation of data to one of these tracks in 
case a disk block develops errors.  The original block is no longer accessible 
through any sort of normal access mechanism, and the data on it can't be 
destroyed without going through some unusual contortions which tend to be 
highly hardware-dependant.  Other complications include the use of journaling
filesystems which keep track of previous generations of data, and disk
compression software or hardware which will compress a simple repeated
overwrite pattern to nothing and leave the original data intact on the disk\footnote{
%Footnote [3]:
               \hspace*{5pt} From a posting to the usenet alt.security newsgroup on 1 August
               1994, article-ID $<$31c75s\$pa8@search01.news.aol.com$>$: ``I got fired
               from my job and told to clean my desk, so I immediately went to
               my office and ran Norton WipeDisk on my hard drive, which
               contained much of the work I had done and also my contact list,
               letters, games, and so on.  Unfortunately, I had DoubleSpaced it
               and the files were easily recovered''.
}.
Therefore if ``overwriting all storage locations'' is interpreted to mean
``exposing the entire reading surface to a magnetic field having a strength at
the recording surface greater than the field intensity at which it was
recorded'', the method does have merit.  Unfortunately it is virtually
impossible to get at all storage locations, and simple-minded methods such as
trying to write patterns to the storage allocated to a file in order to erase
it don't even come close to this target.  The overwrite method used by SFS does
come reasonably close by attempting to create a rapidly-fluctuating magnetic
field over all physically accessible sectors which make up a disk volume,
creating the same effect as a degaussing tool used to erase magnetic fields.

Another consideration which needs to be taken into account when trying to erase
data through software is that drives conforming to some of the higher-level 
protocols such as the various SCSI standards are relatively free to interpret 
commands sent to them in whichever way they choose (as long as they still 
conform to the SCSI specification).  Thus some drives, if sent a FORMAT UNIT 
command may return immediately without performing any action, may simply 
perform a read test on the entire disk (the most common option), or may 
actually write data to the disk\footnote{
%Footnote [4]:
               Again it may be possible to bypass this using highly
               hardware-specific methods.  For example Quantum SCSI drives manufactured
               a few years ago could be forced to write data to disk during a
               format by changing the sector filler byte before the format
               command was issued.
} \footnote{
%Footnote [5]: 
		The SCSI-2 standard includes an initialization pattern (IP)
              	option for the FORMAT UNIT command (Section 8.2.1.2), however it
              	is not clear how well this is supported by existing drives.
}.  This is rather common among newer 
drives which can't directly be low-level formatted, unlike older ST-412 and 
ST-506 MFM or RLL drives.  For example trying to format an IDE drive generally 
has little effect---a low-level format generally isn't possible, and the 
standard DOS `format' command simply writes a boot record, FAT, and root 
directory, performs a quick scan for bad sectors, and exits.

Therefore if the data is very sensitive and is stored on floppy disk, it can
best be destroyed by removing the media from the disk liner and burning it.
Disks are relatively cheap, and can easily be replaced.  Permanently destroying
data on fixed disks is less simple, but the multiple-overwrite option used by
SFS at least makes it quite challenging (and expensive) to recover any
information.

%Footnote [1]:  Again it may be possible to bypass this using highly hardware-
%               specific methods.  For example Quantum SCSI drives manufactured
%               a few years ago could be forced to write data to disk during a
%               format by changing the sector filler byte before the format
%               command was issued.

%\documentstyle[a4]{article}

%\begin{document}

%\parindent 0pt
%\parskip 2mm

\section{SFS Disk Volume Layout}

An SFS volume is broken up into two parts, the boot sector which is used to
identify the volume and store assorted status information, and the encrypted
volume itself.  If a program tries to read the boot sector, the SFS driver will
assemble a pseudo-boot sector in memory and return that instead.  If a program
tries to write to the boot sector, the SFS driver will skip the boot sector
while still writing any other sectors which may be requested.

All data on the boot sector, both plaintext and encrypted, is stored in
big-endian format following the convention used by international cryptographic
standards.  Care should be taken to ensure that the proper endianness is
maintained when reading and writing the boot sector on little-endian systems
(the performance of the current implementation was tested against the RS6000
version to confirm that endianness conversion was being done correctly).

In the following discussion a BYTE is an 8-bit quantity, a WORD is a big-
endian 16-bit quantity, and a LONG is a big-endian 32-bit quantity.  There are
no alignment restrictions for the data as stored on disk.  All data is
close-packed with no need for byte-padding for word or longword boundaries.


\subsection{The SFS Volume Header Record}

The boot record is the first sector in a disk volume, and is usually used to
load a bootstrap code block which loads more code which eventually loads DOS or
an operating system.  In the case of SFS there is no bootstrap code as SFS
partitions are not bootable.  Instead, the boot record contains a volume header
record holding identification information which SFS uses when mounting the
volume, and encryption information to allow SFS to decrypt the volume.  The
information is stored as a series of variable-length data packets with fields
in big-endian order.  Although it might be desirable to store the data as
DER-encoded ASN.1 (ISO 8824, ISO 8825), the amount of code necessary to decode
this information is considerable, leading to excessively complicated loaders
for the encrypted volume.  In particular, it would make the use of automatic
volume mounts for the SFS device driver virtually impossible.

The SFS volume header may contain multiple data packets, currently up to six
types are implemented (there is an additional seventh pseudo-packet type which 
is a null data packet).  One packet identifies the volume, one contains 
encryption information needed to en/decrypt the volume, and one contains
filesystem-specific information needed to access the volume.  These packets 
are mandatory.  Finally, a number of optional packets may be used to hold data
such as information for volumes accessible by multiple users, information
needed by the high-speed direct disk access routines in SFS, and information to
control the conditions under which SFS volumes are unmounted.  Each packet 
consists of a packet type identifier followed by the length of the data in the 
packet, and then by the packet data itself.

The volume header is laid out as follows:

\begin{center}
\begin{tabular}{|r|r|l|l|}
\hline
    Offset & Size & Type  &     Description\\
\hline
       0   &  4   & BYTE$[ 4 ]$ & `SFS1' identification string\\
       4   &  2   & WORD      & Information packet 1 ID\\
       6   &  2   & WORD      & Information packet 1 data length\\
       8   & ??   & ????      & Information packet 1 data\\
       n   &  2   & WORD      & Information packet 2 ID\\
     n+2   &  2   & WORD      & Information packet 2 data length\\
     n+4   & ??   & ????      & Information packet 2 data\\
    \dots  & \dots& \dots     & \dots \\
       m   &  2   & WORD      & Information packet n ID\\
     m+2   &  2   & WORD      & Information packet n data length\\
     m+4   & ??   & ????      & Information packet n data\\
\hline
\end{tabular}
\end{center}

It is recommended that the order of the information packets be as follows

\begin{enumerate}
    \item  Volume information
    \item  Encryption information
    \item  Filesystem information
    \item  Other information (such as multiuser access data, direct disk access
        information, and information to control the conditions under which SFS
        volumes are unmounted)
\end{enumerate}

although this is not essential.  However some volume mount software which must 
run with severely limited resources may run into problems if the order of the 
data packets is not as expected.  The remainder of the sector is filled with 
zeroes.  Currently defined packet types are:

\begin{center}
\begin{tabular}{|l|c|l|}
\hline
    Name                 &   Value  &     Information type\\
\hline
    SFS\_PACKET\_NONE      &     0    &     Null packet\\
    SFS\_PACKET\_VOLUMEINFO&     1    &     Volume information\\
    SFS\_PACKET\_ENCRINFO  &     2    &     Encryption information\\
    SFS\_PACKET\_DISK\_BPB &     3    &     Filesystem information\\
    SFS\_PACKET\_MULTIUSER &     4    &     Multiuser access information\\
    SFS\_PACKET\_FASTACCESS&     5    &     Direct disk access information\\
    SFS\_PACKET\_UNMOUNT   &     6    &     Volume unmount information\\
\hline
\end{tabular}
\end{center}


\subsubsection{Packet 0 - Null Packet}

This packet contains no information and is never explicitly set in an SFS
volume header.  However since the remainder of the header after the actual data
is padded out with zeroes, it can be viewed as containing a succession of null
packets.


\subsubsection{Packet 1 - Volume Information Packet}

The volume information packet contains the volume name, the volume creation
time, and the volume serial number.  The packet layout is as follows:

\begin{center}
\begin{tabular}{|c|c|c|l|}
\hline
    Offset & Size  &  Type   &     Description\\
\hline
       0   &   2   &  WORD   &     Volume information packet ID = 1\\
       2   &   2   &  WORD   &     Volume information packet data length\\
       4   &   2   &  WORD   &     Volume name character set identifier\\
       6   &   2   &  WORD   &     Volume name length\\
       8   &  ??   &  BYTE$[ ]$ &     Volume name\\
       n   &   4   &  LONG   &     Volume date, stored as seconds since 1970 GMT\\
     n+4   &   4   &  LONG   &     Volume serial number\\
\hline
\end{tabular}
\end{center}

The character set identifier is one of the following:

\begin{center}
\begin{tabular}{|r|l|}
\hline
    Value &  Type\\
\hline
      0   &  ISO 646 character set\\
      1   &  ISO 8859-1 character set\\
      2   &  ISO 8859-2 charcater set\\
      3   &  ISO 8859-3 character set\\
      4   &  ISO 8859-4 character set\\
      5   &  ISO 8859-5 character set\\
      6   &  ISO 8859-6 character set\\
      7   &  ISO 8859-7 character set\\
      8   &  ISO 8859-8 character set\\
      9   &  ISO 8859-9 character set\\
     ??   &  Reserved for future use\\
\hline
\end{tabular}
\end{center}

The volume name is stored as an octet string immediately following the
character set and length values.  The recommended maximum length for the volume
name field is 100 octets, in order to allow room for the remainder of the
volume header.

The volume creation data is stored as a count of seconds since midnight on 1st
January 1970.  Times are stored relative to GMT, although some operating
systems may have trouble with time zones.

The volume serial number is used mainly for identification purposes and has no
special characteristics, but should be unique across volumes.


\subsubsection{Packet 2 - Encryption Information Packet}

The encryption information packet contains the algorithm identifier of the
encryption algorithm being used, the iteration count used when setting up the
keying information, the initialisation vector (IV) for the disk key, the
encrypted disk key for the volume, and a key check value which may be used to
verify that the correct decryption key has been given. The packet layout is as
follows:

\begin{center}
\begin{tabular}{|c|c|c|l|}
\hline
    Offset & Size &   Type    &    Description\\
\hline
       0   &   2  &   WORD    &    Encryption packet ID = 2\\
       2   &   2  &   WORD    &    Encryption packet data length\\
       4   &   2  &   WORD    &    Encryption algorithm identifier\\
       6   &   2  &   WORD    &    Encryption key setup iteration count\\
       8   &  ??  &   BYTE$[ ]$  &    Disk key IV\\
       n   & 128  &   BYTE$[ ]$  &    Encrypted disk key\\
   n+128   &   2  &   WORD    &    Key check value\\
\hline
\end{tabular}
\end{center}

The encryption algorithm identifier is one of the following:

\begin{center}
\begin{tabular}{|r|l|}
\hline
    Value &  Type\\
\hline
      0   &  MDC/SHS\\
     ??   &  Reserved for future use\\
\hline
\end{tabular}
\end{center}

The MDC/SHS algorithm uses the Secure Hash Standard as the block transformation
in the MDC encryption algorithm.

The size of the disk key IV depends on the encryption algorithm, and is equal
to the block size of the encryption algorithm being used.  In the case of
MDC/SHS this is 160 bits or 20 bytes.

The encrypted disk key contains 128 bytes of cryptographically strong random
information whose derivation is given in the section ``Generating Random
Numbers'' above.  This data may be utilized as required by the encryption
algorithm used for the particular disk volume.  In the case of MDC/SHS the
first 160 bits are used as the master disk IV from which individual sector IV's
are generated as described in the section ``The Use of an Initialization
Vector'', the next 512 bits are used as the en/decryption key, and the remaining
352 bits are ignored.  Other algorithms with larger or smaller key spaces may
use this data differently.

The key check value is provided to give some warning about an incorrect
password, and should be derived from the encryption key in some nontrivial
manner which makes using the key check in an attack no easier than a standard
known plaintext attack.  In the case of MDC/SHS it is the last two octets in
the key data buffer after the key setup operation has been performed (this
value is never used as part of the encryption key, and is the product of 200
iterations of MDC/SHS over the user key).


\subsubsection{Packet 3 - Filesystem Information Packet}

The filesystem information packet contains identification information for the
filesystem contained on the encrypted volume followed by encrypted
filesystem-specific information needed to handle the volume.  This information
is highly system-specific.  The packet layout is as follows:

\begin{center}
\begin{tabular}{|c|c|c|l|}
\hline
    Offset & Size  &  Type   &     Description\\
\hline
       0   &   2   &  WORD   &     Filesystem information packet ID = 3\\
       2   &   2   &  WORD   &     Filesystem information packet data length\\
       4   &   2   &  WORD   &     Filesystem type identifier\\
       6   &  ??   &  BYTE$[ ]$ &     Encrypted filesystem information\\
\hline
\end{tabular}
\end{center}

The filesystem type identifier is one of the following:

\begin{center}
\begin{tabular}{|r|l|}
\hline
    Value &  Type\\
\hline
      0   &  MSDOS FAT filesystem - BPB data\\
     ??   &  Reserved for future use\\
\hline
\end{tabular}
\end{center}

The encrypted MSDOS BPB data record corresponds directly to a standard BIOS
parameter block and is laid out as follows:

\begin{center}
\begin{tabular}{|c|c|c|l|}
\hline
    Offset & Size &   Type   &     Description\\
\hline
       0   &   2  &   WORD   &     Sector size in bytes \\
       2   &   1  &   BYTE   &     Sectors per cluster\\
       3   &   2  &   WORD   &     Number of boot sectors\\
       5   &   1  &   BYTE   &     Number of FAT copies\\
       6   &   2  &   WORD   &     Number of entries in root directory\\
       8   &   2  &   WORD   &     16-bit number of sectors on disk\\
      10   &   1  &   BYTE   &     Media descriptor byte\\
      11   &   2  &   WORD   &     Number of sectors per FAT\\
      13   &   2  &   WORD   &     Number of sectors per track\\
      15   &   2  &   WORD   &     Number of heads\\
      17   &   4  &   LONG   &     Number of hidden sectors\\
      21   &   4  &   LONG   &     32-bit number of sectors on disk\\
\hline
\end{tabular}
\end{center}


\subsubsection{Packet 4 - Multiuser Access Information}

The multiuser access information packet contains an identification number for
the multiuser access data file needed to access the volume.  This value is
checked against an equivalent one in the access data file to ensure that the
data file corresponds to the volume in question.  In addition the presence of
this packet signals to the user software that the volume has multiuser access
enabled.  The packet layout is as follows:

\begin{center}
\begin{tabular}{|c|c|c|l|}
\hline
    Offset & Size &   Type  &      Description\\
\hline
       0   &   2  &   WORD  &      Multiuser access information packet ID = 4\\
       2   &   2  &   WORD  &      Multiuser access information packet data length\\
       4   &   4  &   LONG  &      Multiuser access data file ID\\
\hline
\end{tabular}
\end{center}


\subsubsection{Packet 5 - Direct Disk Access Information}

The direct disk access information packet contains information needed by the
disk access routines in SFS to bypass the normal BIOS disk access methods and
interface directly with the drive.  This leads to a significant speed
improvement with disk types which support this access mode, since SFS can
perform en/decryption operations while waiting for disk I/O to complete.  The
packet layout is as follows:

\begin{center}
\begin{tabular}{|c|c|c|l|}
\hline
    Offset & Size &  Type &      Description\\
\hline
       0   &   2  &  WORD &      Direct access information packet ID = 5\\
       2   &   2  &  WORD &      Direct access information packet data length\\
       4   &   2  &  WORD &      Direct access method\\
       6   &  ??  &  ??   &      Extra information needed for direct access\\
\hline
\end{tabular}
\end{center}

The direct access method is one of the following:

\begin{center}
\begin{tabular}{|c|l|}
\hline
    Value &  Type\\
\hline
      0   &  BIOS disk access (default)\\
      1   &  IDE single-sector programmed I/O access\\
      2   &  SCSI device access via ASPI interface\\
     ??   &  Reserved for future use\\
\hline
\end{tabular}
\end{center}

Some of the high-speed access modes are not used automatically since it is
unsafe to assume that all hardware will support them.  Therefore if the direct
disk access information packet is absent and if BIOS access to the drive is
possible, slower BIOS accesses are used by default.  If it has been determined
by external software that a fast access mode is possible, the software should
write (or update) a direct disk access information packet to reflect this.  If
no access to the drive other than with a fast direct access mode is possible,
then the fast access mode will always be used.


\subsubsection{Packet 6 - Volume Unmount Information}

The volume unmount information packet contains information on conditions under
which SFS volumes are unmounted by the SFS driver.  These may be in response to
external stimuli, or due to internal conditions monitored by the driver.  In
the absence of this packet, the settings used for volume unmounts are the
defaults set by the driver, although these can generally be overridden under
user control.

Currently only one parameter, the auto-unmount timeout value, is defined.  This
contains the default setting for the auto-unmount timer for this volume.  If
no disk access takes place in the specified time, the volume is automatically
unmounted.  A value of 0 in this field indicates that no auto-unmount timer is
to be set (although this is better accomplished by simply deleting the volume
unmount information packet).

The packet layout is as follows:

\begin{center}
\begin{tabular}{|c|c|c|l|}
\hline
    Offset & Size &  Type  &     Description\\
\hline
       0   &   2  &  WORD  &     Volume unmount information packet ID = 6\\
       2   &   2  &  WORD  &     Volume unmount information packet data length\\
       4   &   2  &  WORD  &     Auto-unmount timeout in minutes, 0 = none set\\
\hline
\end{tabular}
\end{center}


\section{SFS Multiuser Access File Layout}

[The following information is preliminary and is bound to change at a moments
 notice]

Associated with each SFS volume which has multiuser access enabled are one or
more database files containing information on each user who has access to that
volume.  This consists of identification information for each user (in the case
of a named database), access control information, and keying information needed
to access the actual volume.

The file is laid out as a simple flat database, which is adequate for its
intended use since a user record is only retrieved once per database access,
and generally the databases will be quite small.

On a system with proper access control this file will be read-only for everyone
but the volume administrator, but unfortunately under some operating systems
and also DOS there is little access control and anyone with access to a sector
editor can change their access rights.  However at this level they can also
change the SFS volume header and the behaviour of mountsfs and the SFS driver,
so little is gained by attempting to plug this hole.  In any case, the basic
access to the volume is cryptographically controlled, and bypassing this goes
beyond simply editing a file.

In the future it may be worthwhile implementing more heavy-duty controls, such 
as using public-key signed access permission tokens from the volume 
administrator to grant access for networked use.

All data in the multiuser access database is stored in big-endian format
following the convention used by international cryptographic standards.  Care
should be taken to ensure that the proper endianness is maintained when reading
and writing the boot sector on little-endian systems (the performance of the
current implementation was tested against the RS6000 version to confirm that
endianness conversion was being done correctly).

In the following discussion a BYTE is an 8-bit quantity, a WORD is a
big-endian 16-bit quantity, and a LONG is a big-endian 32-bit quantity.  There are
no alignment restrictions for the data as stored in the database.  All data is
close-packed with no need for byte-padding for word or longword boundaries.

An access database is broken up into two sections, a header record containing
details on the database as a whole, and one or more individual user records.


\subsection{Database Header}

The database header is laid out as follows:

\begin{center}
\begin{tabular}{|c|c|c|l|}
\hline
    Offset & Size  &  Type     &   Description\\
\hline
       0   &   4   &  BYTE[4]  &   `SFS1' identification string\\
       4   &   2   &  WORD     &   Database type identifier\\
       6   &   4   &  LONG     &   Encrypted volume serial number\\
      10   &   2   &  WORD     &   Number of user records in database\\
      12   &   2   &  WORD     &   Database name character set identifier\\
      14   &   2   &  WORD     &   Database name length\\
      16   &  ??   &  BYTE[]   &   Database name\\
\hline
\end{tabular}
\end{center}

The identification string and database type identifier serve to identify the
general database format.  The type identifier is one of the following:

\begin{center}
\begin{tabular}{|r|l|}
\hline
    Value &  Type\\
\hline
      0   &  Anonymous database containing date, access, keyinfo records\\
      1   &  Named database containing user name, date, access, keyinfo records\\
     ??   &  Reserved for future use\\
\hline
\end{tabular}
\end{center}

An anonymous database contains no identification for any of the user records.
Each keyinfo record is decrypted in turn until a valid record is found, and
this keying information is then used to access the encrypted volume.

A named database contains an identifier for each user record.  Lookup of the
database is performed directly by user name, and then access proceeds as for
the anonymous database.

The encrypted volume serial number is the serial number of the volume this
database is used to access.

[Problem: Should be able to set to zero to keep corresponding volume anonymous]

[Problem: Can lead to trouble if chsfs is used to change volume serial number.
          Perhaps change chsfs to disallow serial number change without also
          changing database serial number]

The database name is the collective name for the database and the users in it.
Example database names might be ``Ancient Illuminated Seers of Bavaria'' or
``Trilateral Commission''.  The name is optional, and may be omitted by
specifying a name with a length of 0 characters.


\subsection{Database Records}

Each record in the database is laid out as follows:

\begin{center}
\begin{tabular}{|c|c|c|l|}
\hline
    Offset & Size &   Type    &    Description\\
\hline
       0   &   2  &   WORD    &    Record type identifier\\
       2   &   2  &   WORD    &    User name character set identifier\\
       4   &   2  &   WORD    &    User name length\\
       6   &  ??  &   BYTE$[ ]$  &    User name\\
       n   &   4  &   LONG    &    Access valid from date\\
     n+4   &   4  &   LONG    &    Access valid to date\\
     n+8   &   4  &   LONG    &    Access rights bitstring\\
    n+12   &  ??  &   BYTE$[ ]$  &    Keying information\\
\hline
\end{tabular}
\end{center}

The record type identifier is one of the following:

\begin{center}
\begin{tabular}{|r|l|}
\hline
    Value &  Type\\
\hline
      0   &  Anonymous record containing date, access, and keyinfo\\
      1   &  Named record containing user name, date, access, and keyinfo\\
     ??   &  Reserved for future use\\
\hline
\end{tabular}
\end{center}

The user name is the name of the user entitled to access the SFS volume the
database corresponds to.  Under multiuser operating systems this will simply
correspond to the userID which the user is known to the operating system under.
Otherwise, the userID can be the user's name or some similar identifying method
capable of distinguishing users.  If the record type is an anonymous record,
the user name fields will not be present.

[Problem: One user with many accounts.  Allow multiple userID's per record?
          This could encourage password sharing though]

The access valid date pair contains the date after which access to the SFS
volume is permitted, and the date at which access rights to the volume expire.
These are stored in the Unix seconds-since-1970 format.

The access rights bitstring contains the access rights the user has to the SFS
volume.

[Problem:~How?  Use RWXRWXRWX?  Or ASN.1-type strings with complex
          FTAM-type rights?  ACL's?]

The keying information is [what?  Same as single-user keying information]


\section{Interfacing with SFS}

The SFS device driver has a sophisticated control interface which allows
complete configurability through the standard DOS IOCTL read and write calls,
which are used to transfer data to and from the control channel of a device.
It is recommended that you consult a system programming guide or your compilers
documentation for more information on making DOS IOCTL calls.  Many compilers
provide standard routines for making these calls.

In the following discussion a BYTE is an 8-bit quantity, a WORD is a
little-endian 16-bit quantity, and a LONG is a little-endian 32-bit quantity.  These 
values will need endianness conversion after being read from the SFS volume 
header.


\subsection{How SFS Identifies Drives}

For floppy disk drives, a value of 0 usually corresponds to drive A: and a
value of 1 usually corresponds to drive B:.  For fixed disk drives accessible
through the system BIOS, a value of 0 with the high bit set (giving an actual 
value of 0x80) usually corresponds to the first physical drive and a value of 1 
with the high bit set (giving an actual value of 0x81) usually corresponds to 
the second physical drive.  These values correspond to physical drives as
accessed through the BIOS, and not logical volumes (SFS can also access drives
not normally accessible through the BIOS).  A single physical drive may contain 
a number of logical volumes.  Since the SFS driver runs below the level at which 
most operating systems operate, it will ignore the DOS drive mappings (in fact 
DOS can't even see the encrypted disk volume) and access the drive directly.  
Thus if there are more than the DOS standard of two floppy drives and two 
physical hard drives connected to the system it should still be able to access 
the drives and present them to the operating system as normal disk volumes.

If the physical drive contains more than one logical volume, the offset of the
start of the logical volume from the start of the physical drive is specified
as a sector or disk block count.  If the physical drive contains only one 
logical volume, this value is set to zero.  Finding the start of the logical 
volume will usually entail parsing the partition table of the physical drive.  
Floppy drives have only one logical volume which is equivalent in size to the 
physical drive, fixed disks have one or more logical volumes on each physical 
volume.  These logical volumes can contain different filesystems and operating 
systems, and not all may be accessible or visible to DOS.  For more information 
on disk organization, refer to a good technical reference which covers disk 
drives, and then spend several days figuring out all the DOS quirks never 
mentioned in any reference.

In the following text the term `drive number' is used to refer to a value which
identifies a particular drive (which depends on the drive type, since BIOS, IDE
and SCSI drive access identifies drives in different ways), the term `physical 
volume' is used to refer to the physical drive being accessed and the term 
`logical volume' is used to refer to the drive as DOS sees it.


\subsubsection{Reading from the SFS Driver}

The SFS driver processes both control channel read and control channel write
requests.  Initially, the drive number of the logical volume being used by the
SFS driver must be found.  This can be done by checking each possible logical
drive as follows:

\begin{verbatim}
    for( driveNumber = 0; driveNumber <= 'Z' - 'A'; driveNumber++ )
        {
        /* Try and read data from the device's control channel */
        data <- ioctlRead( driveNumber );

        /* Check for the SFS identification string */
        if( first 4 bytes of data = 'SFS1' )
            exit loop;
        }
\end{verbatim}

If, at the end of the loop's execution, the driveNumber is greater than the
total number of drives in the system, the SFS driver has not been found,
probably because it isn't present in the system.  Otherwise, driveNumber
contains the drive number of the logical volume which the SFS driver is making
available, and the data packet contains assorted status information about the
SFS volume corresponding to the driveNumber.

The data packet returned by the SFS driver is as follows:

\begin{center}
\begin{tabular}{|c|c|c|p{230pt}|}
\hline
    Offset & Size &   Type     &   Description\\
\hline
       0   &   4  &   BYTE$[ 4 ]$&   `SFS1' identification string\\
       4   &   2  &   WORD     &   SFS unit number, starting from 0\\
       6   &   2  &   WORD     &   Drive the SFS volume is mounted on.
				   Interpretation of this value depends on the
				   disk access mode.\\
       8   &   4  &   LONG     &   Sector or disk block offset of logical volume 
                                   from start of physical volume, or 0 if logical  
                                   volume corresponds to physical volume\\
      12   &   2  &   WORD     &   0 = no disk mounted, 1 = disk mounted, 2 = disk
                                   mounted in non-removable mode\\
      14   &   2  &   WORD     &   0 = disk is read/write, 1 = disk is read-only\\
      16   &   2  &   WORD     &   Quick-unmount hotkey value (high byte = shift\\
           &      &            &   value, low byte = optional keyboard scan code)\\
      18   &   2  &   WORD     &   Auto-unmount time.  -1 = no unmount timer set,\\
           &      &            &   0 = timer set but expired, any other value =\\
           &      &            &   total unmount time in minutes\\
      20   &   2  &   WORD     &   Auto-unmount timeout actual minutes remaining\\
           &      &            &   before the unmount takes place\\
      22   &   2  &   WORD     &   Internal driver check code.  0 = no error, 1 =\\
           &      &            &   driver consistency check failed, 2 = individual\\
           &      &            &   unit consistency check failed.\\
      24   &   4  &   DWORD    &   Last I/O error status.\\
\hline
\end{tabular}
\end{center}

The unit number is used by SFS to handle multiple encrypted drives.  This value
starts at zero for the first unit, and identifies the volume being accessed.

The last I/O error status value provides more details on the last read or write
error encountered by the SFS driver for the drive in question.  This value is
divided into 4 bytes, of which the first contains the access mode in use when
the error occurred, the remaining three contain error information.  The
possible values are as follows:

\begin{center}
\begin{tabular}{|c|c|l|}
\hline
    First byte & Access mode  &  Remaining bytes\\
\hline
        0      &    BIOS      &   Byte 1 = 0\\
               &              &   Byte 2 = 0\\
               &              &   Byte 3 = BIOS error code\\
        1      &    IDE       &   Byte 1 = 0\\
               &              &   Byte 2 = controller status\\
               &              &   Byte 3 = controller error code\\
        2      &    SCSI      &   Byte 1 = sense key\\
               &              &   Byte 2 = auxiliary sense key\\
               &              &   Byte 3 = auxiliary sense key qualifier\\
       ??      &  Reserved    &   Reserved for future use\\
\hline
\end{tabular}
\end{center}

In all cases a value of all ones in bytes 2 and 3 indicate a controller or host
device timeout.

If queried via an IOCTL call on the volume's removability, the SFS driver will
always indicate that the media is removable, even if the volume is on a fixed
disk.  This is because some disk cacheing software won't invalidate a drive's
data, even if the driver signals that the media has changed, unless the volume
is marked as being removable.

This status information can be returned to the user in the form of a status
message or an information dialog if desired (the drive and offset information 
can be used to read the volume name, date, and serial number from the encrypted 
volume).


\subsubsection{Writing to the SFS Driver}

Once the driver has been found, several types of data packet can be sent to the
driver to control its operation.  Each control packet begins with a WORD
containing the magic value `C0' which the SFS driver checks to ensure the
packet is meant for it, and a WORD identifying the data packet type.  The data
packet types are:

\begin{center}
\begin{tabular}{|l|c|l|}
\hline
    Name                   & Value   &    Information type\\
\hline
    PACKET\_SET\_DISKINFO    &   0     &    Disk parameters\\
    PACKET\_SET\_KEYINFO     &   1     &    Keying information\\
    PACKET\_SET\_READONLY    &   2     &    Set disk read-only status\\
    PACKET\_SET\_DRIVENO     &   3     &    Drive number to mount\\
    PACKET\_SET\_MOUNTSTATUS &   4     &    Set mount status\\
    PACKET\_SET\_UNMOUNT     &   5     &    Set/clear quick-unmount\\
                             &         &    hotkey value\\
    PACKET\_SET\_TIMEOUT     &   6     &    Set/clear timed unmount value\\
\hline
\end{tabular}
\end{center}

These packets are explained in more detail below.  All other values are
reserved for future use.

\begin{itemize}
\item Packet 0 - Disk Parameters:

This packet type is used to convey to the SFS driver various pieces of
information about the physical characteristics of the disk drive, including the
sector size, number of heads, total number of sectors, and so on.  The packet
layout is as follows:

\begin{center}
\begin{tabular}{|c|c|c|l|}
\hline
    Offset & Size &   Type    &    Description\\
\hline
       0   &   2  &   WORD    &    Magic value `C0'\\
       2   &   2  &   WORD    &    Packet type 0\\
       4   &   2  &   WORD    &    Sector size in bytes\\
       6   &   1  &   BYTE    &    Sectors per cluster\\
       7   &   2  &   WORD    &    Number of boot sectors\\
       9   &   1  &   BYTE    &    Number of FAT copies\\
      10   &   2  &   WORD    &    Number of entries in root dir\\
      12   &   2  &   WORD    &    Number of sectors on disk\\
      14   &   1  &   BYTE    &    Media descriptor byte\\
      15   &   2  &   WORD    &    Number of sectors per FAT\\
      17   &   2  &   WORD    &    Number of sectors per track\\
      19   &   2  &   WORD    &    Number of heads\\
      21   &   4  &   LONG    &    Number of hidden sectors\\
      25   &   4  &   LONG    &    Number of sectors per disk, 32-bit\\
\hline
\end{tabular}
\end{center}

Sending this data packet to the driver automatically unmounts the encrypted
volume.


\item Packet 1 - Keying information

This packet type sets up the encryption information in the driver.  It contains
the en/decryption key, and the master IV for the encrypted volume.  The packet
layout is as follows:

\begin{center}
\begin{tabular}{|c|c|c|l|}
\hline
    Offset & Size  &  Type       & Description\\
\hline
       0   &   2   &  WORD       & Magic value `C0'\\
       2   &   2   &  WORD       & Packet type 1\\
       4   &  20   &  BYTE$[ 20 ]$ & Master IV for encrypted volume\\
      24   &  64   &  BYTE$[ 64 ]$ & MDC/SHS keying information\\
\hline
\end{tabular}
\end{center}

Sending this data packet to the driver automatically unmounts the encrypted
volume.


\item Packet 2 - Set disk Read-only Status

This packet type sets the read-only status of the disk.  Sending a value of 0
makes the disk read-only.  Sending a value of 1 makes the disk read/write.  The
packet layout is as follows:

\begin{center}
\begin{tabular}{|c|c|c|l|}
\hline
    Offset & Size  &  Type   &     Description\\
\hline
       0   &   2   &  WORD   &     Magic value `C0'\\
       2   &   2   &  WORD   &     Packet type 2\\
       4   &   2   &  WORD   &     Read-only status: 0 = read-only, 1 = read/write\\
\hline
\end{tabular}
\end{center}

Sending this packet type has no effect on the mount status of the volume.


\item Packet 3 - Drive Number to Mount

This packet type sets the drive number and the offset of the logical volume on 
the physical drive, which the driver will en/decrypt.  The packet layout is as 
follows:

\begin{center}
\begin{tabular}{|c|c|c|p{212.4pt}|}
\hline
    Offset & Size  &  Type  &      Description\\
\hline
       0   &   2   &  WORD  &      Magic value `C0'\\
       2   &   2   &  WORD  &      Packet type 3\\
       4   &   2   &  WORD  &      Drive number.  Interpretation of this value
                                   depends on the disk access mode.\\
       6   &   4   &  LONG  &      Sector or disk block offset of logical volume 
				   from start of physical volume, or 0 if logical 
				   volume corresponds to physical volume\\
\hline
\end{tabular}
\end{center}

The drive number contains information allowing the driver to locate the
physical drive on which the SFS volume is located.  The drive access mode and
drive identifying information are encoded into a 16-bit word as follows:

\begin{center}
\begin{tabular}{|l|l|}
\hline
    Bits 15--12 & Disk access mode\\
    Bits 11--0  & Drive identifying information\\
\hline
\end{tabular}
\end{center}

The drive identifying information is laid out as follows:

\begin{center}
\begin{tabular}{|c|l|}
\hline
    Access mode   &  Drive identifying informtion\\
\hline
         0        &  Bits 11--8: 0\\
                  &  Bits 7--0 : BIOS drive number\\
         1        &  Bits 11--8: 0\\
                  &  Bits 7--0 : IDE drive number\\
         2        &  Bits 11--8: SCSI host number\\
                  &  Bits 7--4 : SCSI target ID\\
                  &  Bits 3--0 : SCSI logical unit number\\
\hline
\end{tabular}
\end{center}

Sending this data packet to the driver automatically unmounts the encrypted
volume.


\item Packet 4 - Set mount Status

This packet type sets the mount status of the drive.  A value of 0 unmounts the
drive, a value of 1 mounts the drive.  The packet layout is as follows:

\begin{center}
\begin{tabular}{|c|c|c|l|}
\hline
    Offset & Size  &  Type   &     Description\\
\hline
       0   &   2   &  WORD   &     Magic value `C0'\\
       2   &   2   &  WORD   &     Packet type 4\\
       4   &   2   &  WORD   &     Mount status (0 = unmount, 1 = mount)\\
\hline
\end{tabular}
\end{center}

Unmounting the drive using this packet type also destroys the encryption
information held by the driver and forces all data still held in cache and disk
buffers to be flushed to disk.  This is the only sure way to erase all the
encryption information, as it wipes not only the actual keying information but
also most of the data area used by the driver and any data in cache and disk
buffers.  Merely sending an empty keying information packet will not perform
this task.


\item Packet 5 - Set/clear Quick-Unmount Hotkey Value

This packet is used to set the hotkey value which the SFS driver checks for to
quickly unmount an SFS volume, or to clear the hotkey handling.  The packet
layout is as follows:

\begin{center}
\begin{tabular}{|c|c|c|l|}
\hline
    Offset & Size  &  Type   &     Description\\
\hline
       0   &   2   &  WORD   &     Magic value `C0'\\
       2   &   2   &  WORD   &     Packet type 5\\
       4   &   2   &  WORD   &     Hotkey value (see below)\\
\hline
\end{tabular}
\end{center}

The hotkey which the driver checks for can be any combination of the alt key,
control key, left shift, and right shift (referred to as the shift code), and
an arbitrary keyboard scan code.  The recommended default value is a
combination of the left and right shift keys.  This information is encoded in
the hotkey WORD as follows:

  The high 8 bits contain the shift code.  This contains a set of bitflags
  which specify the shift keys the driver checks for, possibly in addition to
  an optional character code.  The values are:

\begin{center}
\begin{tabular}{|c|c|c|}
\hline
        Shift key   &    Value (binary)   &   Value (hex)\\
\hline
           Alt      &       00001000      &       08\\
         Control    &       00000100      &       04\\
       Left Shift   &       00000010      &       02\\
      Right Shift   &       00000001      &       01\\
\hline
\end{tabular}
\end{center}

  The encoding for the shift code portion of the default left+right shift
  hotkey is therefore 00000011 binary or 03 hex.

  The low 8 bits contain an optional keyboard scan code which the driver will
  check for in addition to the shift code.  This value is used in combination
  with the shift code to specify a key combination to the driver, so that for
  example Alt-Z would be used as the quick-unmount hotkey, although care should
  be taken to ensure that none of a large number of existing special key
  combinations is reused for this.

  If the low 8 bits contain zeroes, the driver does not check for a keyboard
  scan code in addition to a shift code.  The encoding for the keyboard scan
  code portion of the default left+right shift hotkey is therefore 00000000
  binary or 00 hex.

  The overall value is the combination of the shift code and keyboard scan
  code, or 0300 hex for the default hotkey.

Specifying a value of 0 in this word will disable hotkey checking in the
driver.

The hotkey quick-unmount is handled by hooking the int 9h keyboard interrupt.
The interrupt handler which performs this task is installed either when the
driver is loaded (if there is a volume to be mounted or a hotkey code is
specified), or when mountsfs is run (which transmits a quick-unmount control
packet to the driver).  The interrupt handler checks for the programmed hotkey
and unmounts the volume if it is detected.  Subsequent quick-unmount control
packets can be sent to the driver to change the actual hotkey value.  Sending a
value of 0 will disable hotkey checking and deinstall the keyboard interrupt
handler.


\item Packet 6 - Set/clear Timed Unmount Value

This packet is used to set the unmount time after which an SFS volume is
unmounted if no accesses are made to it.  The packet layout is as follows:

\begin{center}
\begin{tabular}{|c|c|c|l|}
\hline
    Offset & Size  &  Type    &    Description\\
\hline
       0   &   2   &  WORD    &    Magic value `C0'\\
       2   &   2   &  WORD    &    Packet type 5\\
       4   &   2   &  WORD    &    Timeout value in minutes before unmount\\
           &       &          &    takes place\\
\hline
\end{tabular}
\end{center}

Specifying a value of 0 as the timeout value will disable the timed unmount
feature in the driver.

The timed unmount is handled by hooking the int 1Ch timer interrupt.  The
interrupt handler which performs this task is installed either when the driver
is loaded (if a timeout value is specified at this time), or when mountsfs is
run and a timeout value is specified (which transmits a timed unmount control
packet to the driver).  The interrupt handler decrements the timer and unmounts
the volume if it expires.  The timer is reset every time a media check packet
is received by the driver, which allows normal DOS accesses to be detected but
doesn't cause false triggering due to device driver control functions.
Subsequent timed unmount control packets can be sent to the driver to change
the actual timeout value for a volume.  Sending a value of 0 will disable the 
timer and, if no more volumes have timers set, deinstall the timer interrupt 
handler.

\end{itemize}

\subsection{Controlling the Driver}

The SFS driver is initialized by sending it a series of control packets giving
the disk information, the keying information, the drive number and logical
volume offset to access, the read/write status, and finally the mount command.
The sequence of operations for mounting a volume is as follows:

\begin{verbatim}
    driveNumber <- find SFS drive number as outlined above;
    ioctlWrite( driveNumber, PACKET_SET_DISKINFO,
                disk information - packet type 0 );
    ioctlWrite( driveNumber, PACKET_SET_KEYINFO,
                keying information - packet type 1 );
    ioctlWrite( driveNumber, PACKET_SET_READONLY,
                read-only status - packet type 2 );
    ioctlWrite( driveNumber, PACKET_SET_DRIVENO,
                drive to mount and vol.offs - packet type 3 );
    ioctlWrite( driveNumber, PACKET_SET_MOUNTSTATUS,
                mount status TRUE - packet type 4 );
\end{verbatim}

To mount a new volume on the same drive:

\begin{verbatim}
    driveNumber <- find SFS drive number as outlined above;
    ioctlWrite( driveNumber, PACKET_SET_KEYINFO,
                keying information - packet type 1 );
    ioctlWrite( driveNumber, PACKET_SET_MOUNTSTATUS,
                mount status TRUE - packet type 4 );
\end{verbatim}

To unmount a volume, erasing the keying information held by the device driver:

\begin{verbatim}
    driveNumber <- find SFS drive number as outlined above;
    ioctlWrite( driverNumber, PACKET_SET_MOUNTSTATUS,
                mount status FALSE - packet type 4 );
\end{verbatim}


\section{Interfacing with mountsfs}

In order to facilitate the use of SFS with other software such as graphical
front-ends, mountsfs can be run in batch mode in which it accepts abbreviated
forms of the usual commands and outputs more complex results to fixed-record
data files instead of the screen.


\subsection{Controlling mountsfs in Batch Mode}

In order to enable these features, the first option given to mountsfs must be
the keyword `batch'.  This allows it to recognise alternative single-letter
forms of the normal commands.  These single-letter options and their equivalent
commands are as follows:

\begin{center}
\begin{tabular}{|c|l|}
\hline
    Letter  &    Full command\\
\hline
      f     &    user file\\
      h     &    hotkey\\
      i     &    info or information\\
      n     &    user name\\
      p     &    password entry\\
      s     &    status\\
      t     &    timeout\\
      u     &    unmount\\
      v     &    volume name\\
\hline
\end{tabular}
\end{center}

The volume name, hotkey, timeout, user name, and user file options must be
followed by the appropriate extra information in the usual format.  The
password entry option should be followed by the password for the encrypted
volume.  For example to mount the volume ``Test'' with a quick-unmount hotkey of
Ctrl-Alt-Z, an auto-unmount timeout of 10 minutes, and a password of ``secret
data'' the command would be:

    {\tt \verb|  |mountsfs batch vTest hCtrlAltZ t10 "psecret data"}

which is equivalent to the usual:

    {\tt \verb|  |mountsfs vol=test hotkey=ctrlAltZ timeout=10}


\subsection{mountsfs Output in Batch Mode}

During normal operation mountsfs may print several lines of information to the
screen giving details on the status of the operation being performed and
details on currently mounted and unmounted SFS volumes.  When mountsfs is run
in batch mode, this information is written to a fixed-record data file named
``sfs.tmp'' instead of to the screen.  The location of the file is given by the
``TMP'' environment variable; if this variable does not exist, the file is
written to the root directory of the C: drive.  If the TMP environment variable
points to a drive without a path, the file is written to the root directory of
that drive.  Thus if the TMP environment variable were set as:

    {\tt \verb|  |set tmp=e:$\backslash$data$\backslash$}

then mountsfs would output the results of the status or info/information
command to the file ``e:$\backslash$data$\backslash$sfs.tmp''.  If the variable were set as:

    {\tt \verb|  |set tmp=a:}

then mountsfs would output its data to the file ``a:$\backslash$sfs.tmp''.

The mountsfs output is written to the file as a series of fixed records.  Each
record is in ASCII text format with individual fields beginning on new lines,
making parsing by external software a simple task.  All control fields are in
uppercase text.  The exact record format is determined by the record type which
is specified in the first line.  In most cases the only record written will be
the result from the mountsfs execution, although the status and
info/information commands produce one or more extra output records which are
detailed in the section ``The Status and Info Commands in Batch Mode'' below.

The mountsfs result record is laid out as follows:

\begin{center}
\begin{tabular}{|c|l|}
\hline
    Field            &   Field contents\\
\hline
    Record type      &   {\tt RESULT}\\
    Success status   &   {\tt TRUE} $or$ {\tt FALSE}\\
    Error message    &   {\em string: error message, or empty string}\\
\hline
\end{tabular}
\end{center}

The record type field identifies the following record as being a mountsfs
result record.

The success status field gives the success status of the mountsfs command.  A
value of ``TRUE'' indicates the command succeeded, a value of ``FALSE'' indicates
the command failed.

The error message field gives the error message generated by mountsfs if the
success status field is set to FALSE, or an empty string if the success status
field is set to TRUE.

A successful mountsfs command result record would be as follows:

\begin{center}
\begin{tabular}{|c|l|}
\hline
    Field              &     Field data\\
\hline
    Record type        &     {\tt RESULT}\\
    Success status     &     {\tt TRUE}\\
    Error message      & \\
\hline
\end{tabular}
\end{center}

A failed mountsfs command result record might be as follows:

\begin{center}
\begin{tabular}{|c|l|}
\hline
    Field              &     Field data\\
\hline
    Record type        &     {\tt RESULT}\\
    Success status     &     {\tt FALSE}\\
    Error message      &     {\tt Cannot find SFS driver}\\
\hline
\end{tabular}
\end{center}


\subsection{The Status and Info Commands in Batch Mode}

Apart from the result record, mountsfs produces two other types of output
records, the output from a status command and the output from an info/
information command.  As with the RESULT record, each record is in ASCII text
format with individual fields beginning on new lines, making parsing by
external software a simple task.  All control fields are in uppercase text.
The exact record format is determined by the record type which is specified in
the first line.  This is either ``STATUS'' for the output from a status command,
or ``INFORMATION'' for the output from an info/information command.

The status command output record is laid out as follows:

\begin{center}
\begin{tabular}{|c|l|}
\hline
    Field              & Field contents\\
\hline
    Record type        & {\tt STATUS}\\
    Unit number        & {\em number: drive unit number}\\
    Mount status       & {\tt TRUE} $or$ {\tt FALSE}\\
    Volume name        & {\em string: volume name, or empty string}\\
    R/W access allowed & {\tt TRUE} $or$ {\tt FALSE}\\
    Drive letter       & {\em character: uppercase driver letter}\\
    Hotkey             & {\em Ctrl, Alt, LeftShift, RightShift = {\tt `c'}, {\tt `a'}, {\tt `l'}, {\tt `r'}}\\
                       & {\em Letter=uppercase letter.}\\
                       & {\em Example: Ctrl-Alt-Z = }{\tt caZ}\\
                       & $or$ {\tt FALSE}\\
    Timeout total time & {\em number: time in minutes, or} {\tt FALSE}\\
    Timeout time left  & {\em number: time in minutes, or} {\tt FALSE}\\
\hline
\end{tabular}
\end{center}

The record type field identifies the following record as being from the output
of the status command.

The unit number field is the drive unit number used by the SFS driver.

The mount status field contains ``TRUE'' if there is a volume mounted, or ``FALSE''
if there are no volumes mounted.  If this field contains the value ``FALSE'' then
the following fields will contain null values (either the value ``FALSE'' or
blank fields as appropriate).

The volume name field contains the name of the volume.  This may contain any
type of character, or may be an empty string if no volume name is present.

The R/W access allowed field contains ``TRUE'' if read/write access to the volume
is allowed, or ``FALSE'' if write access is disabled and the volume is mounted
read-only.

The drive letter field contains the uppercase drive letter which the mounted
SFS volume corresponds to.

The hotkey field contains the quick-unmount hotkey value or ``FALSE'' if none is
set.  The shift keys (if any) are given in lowercase and the ASCII key value
(if any) is given in uppercase.  The shift key codes are Ctrl = `c', Alt = `a',
leftShift = `l', rightShift = `r'.  These are followed by the ASCII key value,
if there is one.  Thus the code for the default quick-unmount hotkey
combination leftShift-rightShift would be ``lr'', and the code for the hotkey
combaination Ctrl-Alt-Z would be ``caZ''.

The timeout total time and timeout time left fields contain either the
auto-unmount total time in minutes and the time left before the unmount takes
place in minutes, or the control string ``FALSE'' if no timeout is set.

A typical status command output record might be as follows:

\begin{center}
\begin{tabular}{|l|l|}
\hline
    Field                &   Field data\\
\hline
    Record type          &   {\tt STATUS}\\
    Unit number          &   {\tt 0}\\
    Mount status         &   {\tt TRUE}\\
    Volume name          &   {\tt Encrypted data}\\
    R/W access allowed   &   {\tt TRUE}\\
    Drive letter         &   {\tt G}\\
    Hotkey               &   {\tt lr}\\
    Timeout total time   &   {\tt 15}\\
    Timeout time left    &   {\tt 4}\\
\hline
\end{tabular}
\end{center}

This would then usually be followed by a RESULT record.

The information command output record is laid out as follows:

\begin{center}
\begin{tabular}{|l|l|}
\hline
    Field              & Field contents\\
\hline
    Record type        & {\tt INFORMATION}\\
    Volume name charset& {\em string: character set name, or} {\tt FALSE}\\
    Volume name        & {\em string: volume name, or empty string}\\
    Volume date        & {\em string: date in YYMMDDHHMMSS format}\\
    Volume ID          & {\em number: volume serial number}\\
    Volume size        & {\em number: volume size in kilobytes, or} {\tt FALSE}\\
    Filesystem type    & {\em string: filesystem type, or} {\tt FALSE}\\
    Mount ID           & {\em string$[8]$: automount ID, or} {\tt FALSE}\\
    Multiuser access   & {\tt TRUE} $or$ {\tt FALSE}\\
    Mount status       & {\em character: uppercase driver letter, or} {\tt FALSE}\\
    Fast access mode   & {\em number: access mode, or} {\tt FALSE}\\
    Current access mode& {\em number: access mode, or} {\tt FALSE}\\
    Timeout            & {\em number: timeout in minutes, or} {\tt FALSE}\\
\hline
\end{tabular}
\end{center}

The record type field identifies the following record as being from the output
of the info/information command.

The volume name character set field contains the name of the character set used
in the volume name.  This is either ``ISO 646'', ``ISO 8859-1'', ``ISO 8859-2'', ``ISO
8859-3'', ``ISO 8859-4'', ``ISO 8859-5'', ``ISO 8859-6'', ``ISO 8859-7'', ``ISO 8859-8'',
``ISO 8859-9'', or ``FALSE'' if the character set is unknown.

The volume name field contains the name of the volume.  This may contain any
character from the character set given in the Volume name character set field,
or may be an empty string if no volume name is present.

The volume date field contains the volume date in YYMMDDHHMMSS (two-digit year,
month, day, hour, minute, second) format.

The volume ID field contains the volume serial number.

The volume size field contains the volume size in kilobytes, or ``FALSE'' if the
volume is removable media.

The filesystem type field contains the name of the filesystem the volume
contains.  This is either ``DOS'', or ``FALSE'' if the filesystem is unknown.

The mount identifier field contains the 8-character mount identifier for the 
volume if the volume is on a fixed disk, or ``FALSE'' if the volume is on a 
removable disk and no mount at system startup is possible.

The multiuser access field contains ``TRUE'' if multiuser access to the volume is
possible or ``FALSE'' if only single-user access is possible.

The mount status field contains the uppercase drive letter which the mounted
SFS volume corresponds to if the volume is mounted, or ``FALSE'' if it is
unmounted.

The fast access mode field contains the fast disk access mode which SFS will
use if possible when accessing this volume, or ``FALSE'' if the default (slower)
access mode will be used.  The current access mode contains the actual access
mode being used to access the volume, or ``FALSE'' if the default (slower) access
mode is being used.

The timeout field contains the default time in minutes before a volume is
auto-unmounted.  This value may be overridden by mountsfs or the SFS driver.

The information command output records corresponding to the volumes given in 
the section ``Mounting an SFS Volume'' above would be as follows:

\begin{center}
\begin{tabular}{|l|l|}
\hline
    Field                &   Field data\\
\hline
    Record type          &   {\tt INFORMATION}\\
    Volume name charset  &   {\tt ISO 646}\\
    Volume name          &   {\tt Data backup}\\
    Volume date          &   {\tt 931101101301}\\
    Volume ID            &   {\tt 1234}\\
    Volume size          &   {\tt FALSE}\\
    Filesystem type      &   {\tt DOS}\\
    Mount ID             &   {\tt FALSE}\\
    Multiuser access     &   {\tt FALSE}\\
    Mount status         &   {\tt FALSE}\\
    Fast access mode     &   {\tt FALSE}\\
    Current access mode  &   {\tt FALSE}\\
    Timeout              &   {\tt FALSE}\\
    Record type          &   {\tt INFORMATION}\\
    Volume name charset  &   {\tt ISO 646}\\
    Volume name          &   {\tt Personal financial records}\\
    Volume date          &   {\tt 930906112219}\\
    Volume ID            &   {\tt 177545}\\
    Volume size          &   {\tt 10000}\\
    Filesystem type      &   {\tt DOS}\\
    Mount ID             &   {\tt 03A12F7B}\\
    Multiuser access     &   {\tt FALSE}\\
    Mount status         &   {\tt E}\\
    Fast access mode     &   {\tt FALSE}\\
    Current access mode  &   {\tt FALSE}\\
    Timeout              &   {\tt 30}\\
    Record type          &   {\tt INFORMATION}\\
    Volume name charset  &   {\tt ISO 646}\\
    Volume name          &   {\tt Encrypted data disk}\\
    Volume date          &   {\tt 930412221700}\\
    Volume ID            &   {\tt 69231461}\\
    Volume size          &   {\tt 42456}\\
    Filesystem type      &   {\tt DOS}\\
    Mount ID             &   {\tt 42DD2536}\\
    Multiuser access     &   {\tt TRUE}\\
    Mount status         &   {\tt FALSE}\\
    Access mode          &   {\tt 1}\\
    Current access mode  &   {\tt 1}\\
    Timeout              &   {\tt 10}\\
\hline
\end{tabular}
\end{center}

This would then usually be followed by a RESULT record.

In order to make use of these records, the controlling program should invoke
mount\-sfs in batch mode with either the `s' status or `i' information command.
mountsfs will run and, if no errors are encountered, write its output to the
output file.  The controlling program can then read the data from the file,
delete the file, and handle the information as appropriate.


\section{Selected Source Code}

This section contains a walkthrough of selected portions of the source code
(mainly the encryption-related parts) to allow the verification of its
correctness and to help people wishing to write SFS-compatible software.

[!!!! Not in the early releases to save space and because I haven't had
      time to add it yet, should be in a later version !!!!]


\section{Future Work}

The following ideas may be incorporated into future versions of SFS if
requested by users.  In addition reasonable requests for other improvements may
also find their way into SFS.

\begin{itemize}

\item Improve error recovery for the encryption process.  This is somewhat
   difficult, and will probably involve keeping a copy of status information
   and the track currently being encrypted on a local volume to allow a restart
   in the case of a power failure.  There are problems inherent in this as it
   involves storing sensitive data in a disk file, and will slow down the
   processing considerably due to the need to write each track to two
   physically separate disk volumes instead of one continuous one.  A partial
   solution is to keep the status information (simply an index of the disk
   section currently being encrypted) in the volume header while mksfs is
   running and provide some sort of restart option if power is lost, although
   what happens if power dies halfway through writing a track is uncertain.

\item A plug-in card which contains a BIOS extension which hooks int 13h and
   encrypts an entire physical disk (not just one disk volume), from the
   master boot record at the start to the very last sector at the end.  This
   means all disk I/O must be done via int 13h and won't work on all systems.
   Cost for the card is estimated to be about \$80-\$100 for the basic version and
   up to \$200 for the full hardware version whose throughput it significantly
   higher than the basic version.

\end{itemize}

\section{Recommended Reading}

In recent years a large number of books and articles on crytography have
appeared.  Many of these are beyond the level of anyone with a casual interest
in the subject, but a good overview is given by Dorothy Dennings (now somewhat
dated) ``Cryptography and Data Security'', published by Addison-Wesley in 1982
(ISBN 0-201-10150-5).  Bruce Schneier's much more recent ``Applied
Cryptography'', published by John Wiley and Sons in 1993 (ISBN 0-471-59756-2),
is probably the best single text on cryptography currently available, and is
recommended reading for anyone wanting more information on the principles
behind SFS.  This book also contains a large list of references to more
information on all manner of encryption algorithms, protocols, and technology.
Voydock and Kent's tutorial ``Security Mechanisms in High-Level Network
Procotols'', published in the ACM Computing Surveys Vol.15, No.2 (June 1983)
provides a good overview of design considerations for encryption systems.

The algorithms and techniques used in SFS are laid down in the following
national and international standards:

\begin{itemize}

\item ANSI X3.106, ``American National Standard for Information Systems --- Data
        Encryption Algorithm --- Modes of Operation''

\item ANSI X9.30 Part 2, ``The Secure Hash Algorithm (SHA)''

\item Australian Standard AS 2805.5.2, ``Electronic Funds Transfer --- Requirements
        for Interface'', Part 5.2, ``Ciphers --- Modes of operation for an n-bit
        block cipher algorithm''

\item ISO 10116:1991, ``Information technology --- Modes of operation for an n-bit
        block cipher algorithm''\footnote{
%Footnote [1]: 
		These publications are available for a horribly high price from
                ISO-affiliated national standards bodies in most countries.
  }

\item ISO 10126-2:1991, ``Banking --- Procedures for message encipherment
        (wholesale) --- DEA Algorithm''

\item NBS FIPS pub. 81, ``DES Modes of Operation'', 1980\footnote{
%Footnote [2]: 
		These publications are available from the National Technical
              	Information Service, Springfield, Virginia 22161.  NTIS will take
              	telephone orders on +1 (703) 487-4650 (8:30 AM to 5:30 PM Eastern
              	Time), fax +1 (703) 321-8547.  For assistance, call +1 (703)
              	487-4679.
  }

\item NIST FIPS pub. 180, ``Secure Hash Standard'', 1993

\end{itemize}

Information on the weaknesses of some cryptosystems are published in a variety
of places.  The largest publicly available sources of information are the
cryptography conference proceedings which are part of the Springer-Verlag
``Lecture Notes in Computer Science'' series.  Eli Biham and Adi Shamir's
``Differential Cryptanalysis of the Data Encryption Standard'', also published by
Springer-Verlag (ISBN 0-387-97930-1), gives information on the resistance to
attack of a number of block ciphers.

Charles Pfleegers book ``Security in Computing'', published by Prentice Hall in
1989 (ISBN 0-13-798943-1) provides a thorough overview of computer security
techniques, including coverage of more obscure areas such as ethical issues.

A good technical coverage of smart cards can be found in ``Smart Card 2000: The
Future of IC Cards'', published by North-Holland in 1988 (ISBN 0-444-70545-7).
The physical and interface characteristics of smart cards are covered in the
following international standard:

\begin{itemize}
\item ISO 7816:1991, ``Identification cards---Integrated circuit card with
        contacts''
\end{itemize}

Finally, James Bamford's ``The Puzzle Palace'', published by Houghton-Mifflin in
1982, is a good source of information on the operation of the NSA, albeit
slightly dated (it predates the widespread availability of encryption for
personal computers, for example).

This list only scratches the surface of the full range of cryptographic and
security literature available - a very detailed bibliography can be found in
the ``Applied Cryptography'' book.


\section{Using SFS}

In general SFS is free for personal (private) use.  However if you find SFS of
use then in order to allow continued development of and enhancements to SFS, in
particular the creation of more user-friendly versions of mksfs and mountsfs,
of versions for other systems, of a low-cost plug-in card containing whole-disk
encryption firmware, and the replacement of the drive I fried testing SFS, a
donation of \$25 would be appreciated.  Alternatively, the donation can be used
to cover the legal costs of those people involved in the current US government
investigation of the PGP encryption program.

Use of SFS in a commercial, government, or institutional environment or for
business purposes is allowed for free for 30 days to allow it to be evaluated.
After this period it should be registered for further use.  The registration
fee covers use of SFS on any one machine at any one time (so you could, for
example, use SFS on a machine at work and keep a copy on your notebook PC for
use while travelling).

If you decide to send a donation or registration, please specify whether you'd
like it to be used for further SFS development work or if it is to go into the
PGP legal fund.  Cheques or money orders can be sent to:

\begin{verbatim}
        Peter Gutmann
        24 Durness Pl.
        Orewa
        Auckland
        New Zealand
\end{verbatim}

I can also be contacted through email at (among others) the following
addresses:

\begin{verbatim}
        pgut1@cs.aukuni.ac.nz
        p.gutmann@cs.auckland.ac.nz
        peterg@kcbbs.gen.nz
\end{verbatim}

with the first address being the preferred one.  Finally, the fastest way to
contact me is by phone between about 10am and 1am at the following number
(remembering that NZ local time is GMT+13, so we're usually about 18-20 hours
ahead of the US and about 12 hours ahead of Europe):

\begin{verbatim}
        +64 9 426-5097
\end{verbatim}


Testimony from one of our satisfied customers:

``I hear this crash and I find a rock, wrapped in paper, next to my living room
  window.  I open up the note and it says `You want it in writing? You got it.
  Next time, use a {\em real} encryption program.  SFS.  We know where you live' ''.

  So why aren't {\em you} using SFS?

Here's what reviewers have been saying about SFS:

``Version 1.1 has several of the advanced features recommended in version 1.0,
  but not all of the ones I'd like to see in version 1.2.  So, it's pretty good
  except when it's not.  Three stars.

  You probably won't use half the features anyway.  I'm a little ticked off
  that it clashes with my most exotic memory-resident programs, but otherwise
  the software runs just fine on my Turbo Rambuster 586.  It will probably run
  like molasses on your 386SX.

  It's great value at \$25, and I recommend you register it, although that's
  easy for me to say because reviewers get freebies''.


\section{Credits}

Thanks to the readers of comp.os.msdos.programmer and comp.os.linux.develop\-ment
and in particular Jouni Kosonen, for their valuable advice with low-level DOS
disk handling and organization, and the users of the Enigma BBS, in particular
Arne Rohde, for helping test-run several early versions of the low-level disk
handling code on various drive configurations and for providing useful advice
on the vagaries of the PC's disk handling.   Thanks also to Steven Altchuler
for performing all sorts of dangerous tests on SFS volumes and for his heroic
efforts in finding the obscure mountsfs bug, and to Vadim Vlasov for help with 
various thorny problems with low-level disk and device driver programming.

Matthias Bruestle (the greatest chemist of all time and space) provided
valuable feedback on assorted problems in mksfs and mountsfs, and risked his
disk drives testing the SFS driver and other SFS software.  At the behest of
the author he also risked imprisonment for Datenvergewaltigung by letting mksfs
near his Linux partition.

Colin Plumb made noises about various problems with CFB-mode encryption,
corrected me when I tried to explain the need for encrypted IV's in email at
3am, and provided much useful feedback on SFS's operation in general, including
the very elegant solution to the CFB-mode encryption problem and help with the
disk overwriting method.

Vesselin Bontchev sent in long lists of suggested improvements to SFS, and
provided an ongoing commentary on features and ideas for the software (it seems
half the functionality provided by SFS is due to his suggestions).  Ralf
Brown's interrupt list provided help with the haze of obscure interrupts used
by mksfs in its quest for random information.

Thanks to John Huttley for the loan of a SCSI controller, Peter Shields for the
loan of a SCSI cable, Stuart Woolford for the loan of a SCSI drive which didn't
work, and John Huttley for the loan of a SCSI drive which did, in order to get
the SCSI interface code for SFS going.

Peter Conrad typset the entire set of SFS docs into \LaTeX\ format.

The Beerdigungsinstitut Utzmann (erstes Erlanger Beerdigungsunternehmen) help\-ed
clean up the casualties from the SFS beta-testing.  Finally, Tony, Geezer,
Bill, Ozzy, Ron, and Grandos helped me stay awake during many long nights of
debugging low-level disk access code and encryption routines.

Various other people have at various times offered help and suggestions on
getting SFS going.  My thanks go to them also.


\section{Warranty}

\subsection{Customer Obligations}

\begin{enumerate}

\item Customer assumes full responsibility that this program meets the
     specifications, capacity, capabilities, and other requirements of said
     customer, and agrees not to bother the author if the program does not
     perform as expected, or performs other than expected, or does not perform
     at all.

\item Customer assumes full responsibility for any deaths or injuries that may
     result from the normal or abnormal operation of this program.  In the
     event of casualties exceeding 1000 persons or property damage in excess of
     \$10 million, customer agrees that he or she has stolen the program and we
     didn't even know he or she had it.

\item Customer agrees not to say bad things about the program or the author to
     anyone claiming to be from ``60 Minutes''.
\end{enumerate}

\subsection{Very Limited Warranty and Conditions of Sale}

\begin{enumerate}
\item For a period of 90 minutes, commencing from the time you first thought
     about getting this program, we warrant that this program may or may not be
     free of any manufacturing defects.  It will be replaced during the
     warranty period upon payment of an amount equal to the original purchase
     price plus \$10.00 for handling.  This warranty is void if the program has
     been examined or run by the user, or if the manual has been read.

\item This program is sold on an {\em as was} basis.  The author makes no warranty
     that it is, in fact, what we say it is in our propaganda, or that it will
     perform any useful function.  We have no obligation whatsoever other than
     to provide you with this fine disclaimer.

\item Some countries do not allow limitations as to how long an implied warranty
     lasts, so we refuse to imply anything.

\item There is an extremely small but nonzero chance that, through a process
     known as ``tunnelling'', this program may spontaneously disappear from its
     present location and reappear at any random place in the universe,
     including your neighbours computer system.  The author will not be
     responsible for any damages or inconvenience that may result.
\end{enumerate}

\subsection{Limitation of Liability}

\begin{enumerate}
\item We have no liability or responsibility to the customer, the customers
     agents, our creditors, your creditors, or anyone else. 
\end{enumerate}

%\end{document}

\end{document}
