%\documentstyle[a4]{article}

%\begin{document}

%\parindent 0pt
%\parskip 2mm

\section{SFS Disk Volume Layout}

An SFS volume is broken up into two parts, the boot sector which is used to
identify the volume and store assorted status information, and the encrypted
volume itself.  If a program tries to read the boot sector, the SFS driver will
assemble a pseudo-boot sector in memory and return that instead.  If a program
tries to write to the boot sector, the SFS driver will skip the boot sector
while still writing any other sectors which may be requested.

All data on the boot sector, both plaintext and encrypted, is stored in
big-endian format following the convention used by international cryptographic
standards.  Care should be taken to ensure that the proper endianness is
maintained when reading and writing the boot sector on little-endian systems
(the performance of the current implementation was tested against the RS6000
version to confirm that endianness conversion was being done correctly).

In the following discussion a BYTE is an 8-bit quantity, a WORD is a big-
endian 16-bit quantity, and a LONG is a big-endian 32-bit quantity.  There are
no alignment restrictions for the data as stored on disk.  All data is
close-packed with no need for byte-padding for word or longword boundaries.


\subsection{The SFS Volume Header Record}

The boot record is the first sector in a disk volume, and is usually used to
load a bootstrap code block which loads more code which eventually loads DOS or
an operating system.  In the case of SFS there is no bootstrap code as SFS
partitions are not bootable.  Instead, the boot record contains a volume header
record holding identification information which SFS uses when mounting the
volume, and encryption information to allow SFS to decrypt the volume.  The
information is stored as a series of variable-length data packets with fields
in big-endian order.  Although it might be desirable to store the data as
DER-encoded ASN.1 (ISO 8824, ISO 8825), the amount of code necessary to decode
this information is considerable, leading to excessively complicated loaders
for the encrypted volume.  In particular, it would make the use of automatic
volume mounts for the SFS device driver virtually impossible.

The SFS volume header may contain multiple data packets, currently up to six
types are implemented (there is an additional seventh pseudo-packet type which 
is a null data packet).  One packet identifies the volume, one contains 
encryption information needed to en/decrypt the volume, and one contains
filesystem-specific information needed to access the volume.  These packets 
are mandatory.  Finally, a number of optional packets may be used to hold data
such as information for volumes accessible by multiple users, information
needed by the high-speed direct disk access routines in SFS, and information to
control the conditions under which SFS volumes are unmounted.  Each packet 
consists of a packet type identifier followed by the length of the data in the 
packet, and then by the packet data itself.

The volume header is laid out as follows:

\begin{center}
\begin{tabular}{|r|r|l|l|}
\hline
    Offset & Size & Type  &     Description\\
\hline
       0   &  4   & BYTE$[ 4 ]$ & `SFS1' identification string\\
       4   &  2   & WORD      & Information packet 1 ID\\
       6   &  2   & WORD      & Information packet 1 data length\\
       8   & ??   & ????      & Information packet 1 data\\
       n   &  2   & WORD      & Information packet 2 ID\\
     n+2   &  2   & WORD      & Information packet 2 data length\\
     n+4   & ??   & ????      & Information packet 2 data\\
    \dots  & \dots& \dots     & \dots \\
       m   &  2   & WORD      & Information packet n ID\\
     m+2   &  2   & WORD      & Information packet n data length\\
     m+4   & ??   & ????      & Information packet n data\\
\hline
\end{tabular}
\end{center}

It is recommended that the order of the information packets be as follows

\begin{enumerate}
    \item  Volume information
    \item  Encryption information
    \item  Filesystem information
    \item  Other information (such as multiuser access data, direct disk access
        information, and information to control the conditions under which SFS
        volumes are unmounted)
\end{enumerate}

although this is not essential.  However some volume mount software which must 
run with severely limited resources may run into problems if the order of the 
data packets is not as expected.  The remainder of the sector is filled with 
zeroes.  Currently defined packet types are:

\begin{center}
\begin{tabular}{|l|c|l|}
\hline
    Name                 &   Value  &     Information type\\
\hline
    SFS\_PACKET\_NONE      &     0    &     Null packet\\
    SFS\_PACKET\_VOLUMEINFO&     1    &     Volume information\\
    SFS\_PACKET\_ENCRINFO  &     2    &     Encryption information\\
    SFS\_PACKET\_DISK\_BPB &     3    &     Filesystem information\\
    SFS\_PACKET\_MULTIUSER &     4    &     Multiuser access information\\
    SFS\_PACKET\_FASTACCESS&     5    &     Direct disk access information\\
    SFS\_PACKET\_UNMOUNT   &     6    &     Volume unmount information\\
\hline
\end{tabular}
\end{center}


\subsubsection{Packet 0 - Null Packet}

This packet contains no information and is never explicitly set in an SFS
volume header.  However since the remainder of the header after the actual data
is padded out with zeroes, it can be viewed as containing a succession of null
packets.


\subsubsection{Packet 1 - Volume Information Packet}

The volume information packet contains the volume name, the volume creation
time, and the volume serial number.  The packet layout is as follows:

\begin{center}
\begin{tabular}{|c|c|c|l|}
\hline
    Offset & Size  &  Type   &     Description\\
\hline
       0   &   2   &  WORD   &     Volume information packet ID = 1\\
       2   &   2   &  WORD   &     Volume information packet data length\\
       4   &   2   &  WORD   &     Volume name character set identifier\\
       6   &   2   &  WORD   &     Volume name length\\
       8   &  ??   &  BYTE$[ ]$ &     Volume name\\
       n   &   4   &  LONG   &     Volume date, stored as seconds since 1970 GMT\\
     n+4   &   4   &  LONG   &     Volume serial number\\
\hline
\end{tabular}
\end{center}

The character set identifier is one of the following:

\begin{center}
\begin{tabular}{|r|l|}
\hline
    Value &  Type\\
\hline
      0   &  ISO 646 character set\\
      1   &  ISO 8859-1 character set\\
      2   &  ISO 8859-2 charcater set\\
      3   &  ISO 8859-3 character set\\
      4   &  ISO 8859-4 character set\\
      5   &  ISO 8859-5 character set\\
      6   &  ISO 8859-6 character set\\
      7   &  ISO 8859-7 character set\\
      8   &  ISO 8859-8 character set\\
      9   &  ISO 8859-9 character set\\
     ??   &  Reserved for future use\\
\hline
\end{tabular}
\end{center}

The volume name is stored as an octet string immediately following the
character set and length values.  The recommended maximum length for the volume
name field is 100 octets, in order to allow room for the remainder of the
volume header.

The volume creation data is stored as a count of seconds since midnight on 1st
January 1970.  Times are stored relative to GMT, although some operating
systems may have trouble with time zones.

The volume serial number is used mainly for identification purposes and has no
special characteristics, but should be unique across volumes.


\subsubsection{Packet 2 - Encryption Information Packet}

The encryption information packet contains the algorithm identifier of the
encryption algorithm being used, the iteration count used when setting up the
keying information, the initialisation vector (IV) for the disk key, the
encrypted disk key for the volume, and a key check value which may be used to
verify that the correct decryption key has been given. The packet layout is as
follows:

\begin{center}
\begin{tabular}{|c|c|c|l|}
\hline
    Offset & Size &   Type    &    Description\\
\hline
       0   &   2  &   WORD    &    Encryption packet ID = 2\\
       2   &   2  &   WORD    &    Encryption packet data length\\
       4   &   2  &   WORD    &    Encryption algorithm identifier\\
       6   &   2  &   WORD    &    Encryption key setup iteration count\\
       8   &  ??  &   BYTE$[ ]$  &    Disk key IV\\
       n   & 128  &   BYTE$[ ]$  &    Encrypted disk key\\
   n+128   &   2  &   WORD    &    Key check value\\
\hline
\end{tabular}
\end{center}

The encryption algorithm identifier is one of the following:

\begin{center}
\begin{tabular}{|r|l|}
\hline
    Value &  Type\\
\hline
      0   &  MDC/SHS\\
     ??   &  Reserved for future use\\
\hline
\end{tabular}
\end{center}

The MDC/SHS algorithm uses the Secure Hash Standard as the block transformation
in the MDC encryption algorithm.

The size of the disk key IV depends on the encryption algorithm, and is equal
to the block size of the encryption algorithm being used.  In the case of
MDC/SHS this is 160 bits or 20 bytes.

The encrypted disk key contains 128 bytes of cryptographically strong random
information whose derivation is given in the section ``Generating Random
Numbers'' above.  This data may be utilized as required by the encryption
algorithm used for the particular disk volume.  In the case of MDC/SHS the
first 160 bits are used as the master disk IV from which individual sector IV's
are generated as described in the section ``The Use of an Initialization
Vector'', the next 512 bits are used as the en/decryption key, and the remaining
352 bits are ignored.  Other algorithms with larger or smaller key spaces may
use this data differently.

The key check value is provided to give some warning about an incorrect
password, and should be derived from the encryption key in some nontrivial
manner which makes using the key check in an attack no easier than a standard
known plaintext attack.  In the case of MDC/SHS it is the last two octets in
the key data buffer after the key setup operation has been performed (this
value is never used as part of the encryption key, and is the product of 200
iterations of MDC/SHS over the user key).


\subsubsection{Packet 3 - Filesystem Information Packet}

The filesystem information packet contains identification information for the
filesystem contained on the encrypted volume followed by encrypted
filesystem-specific information needed to handle the volume.  This information
is highly system-specific.  The packet layout is as follows:

\begin{center}
\begin{tabular}{|c|c|c|l|}
\hline
    Offset & Size  &  Type   &     Description\\
\hline
       0   &   2   &  WORD   &     Filesystem information packet ID = 3\\
       2   &   2   &  WORD   &     Filesystem information packet data length\\
       4   &   2   &  WORD   &     Filesystem type identifier\\
       6   &  ??   &  BYTE$[ ]$ &     Encrypted filesystem information\\
\hline
\end{tabular}
\end{center}

The filesystem type identifier is one of the following:

\begin{center}
\begin{tabular}{|r|l|}
\hline
    Value &  Type\\
\hline
      0   &  MSDOS FAT filesystem - BPB data\\
     ??   &  Reserved for future use\\
\hline
\end{tabular}
\end{center}

The encrypted MSDOS BPB data record corresponds directly to a standard BIOS
parameter block and is laid out as follows:

\begin{center}
\begin{tabular}{|c|c|c|l|}
\hline
    Offset & Size &   Type   &     Description\\
\hline
       0   &   2  &   WORD   &     Sector size in bytes \\
       2   &   1  &   BYTE   &     Sectors per cluster\\
       3   &   2  &   WORD   &     Number of boot sectors\\
       5   &   1  &   BYTE   &     Number of FAT copies\\
       6   &   2  &   WORD   &     Number of entries in root directory\\
       8   &   2  &   WORD   &     16-bit number of sectors on disk\\
      10   &   1  &   BYTE   &     Media descriptor byte\\
      11   &   2  &   WORD   &     Number of sectors per FAT\\
      13   &   2  &   WORD   &     Number of sectors per track\\
      15   &   2  &   WORD   &     Number of heads\\
      17   &   4  &   LONG   &     Number of hidden sectors\\
      21   &   4  &   LONG   &     32-bit number of sectors on disk\\
\hline
\end{tabular}
\end{center}


\subsubsection{Packet 4 - Multiuser Access Information}

The multiuser access information packet contains an identification number for
the multiuser access data file needed to access the volume.  This value is
checked against an equivalent one in the access data file to ensure that the
data file corresponds to the volume in question.  In addition the presence of
this packet signals to the user software that the volume has multiuser access
enabled.  The packet layout is as follows:

\begin{center}
\begin{tabular}{|c|c|c|l|}
\hline
    Offset & Size &   Type  &      Description\\
\hline
       0   &   2  &   WORD  &      Multiuser access information packet ID = 4\\
       2   &   2  &   WORD  &      Multiuser access information packet data length\\
       4   &   4  &   LONG  &      Multiuser access data file ID\\
\hline
\end{tabular}
\end{center}


\subsubsection{Packet 5 - Direct Disk Access Information}

The direct disk access information packet contains information needed by the
disk access routines in SFS to bypass the normal BIOS disk access methods and
interface directly with the drive.  This leads to a significant speed
improvement with disk types which support this access mode, since SFS can
perform en/decryption operations while waiting for disk I/O to complete.  The
packet layout is as follows:

\begin{center}
\begin{tabular}{|c|c|c|l|}
\hline
    Offset & Size &  Type &      Description\\
\hline
       0   &   2  &  WORD &      Direct access information packet ID = 5\\
       2   &   2  &  WORD &      Direct access information packet data length\\
       4   &   2  &  WORD &      Direct access method\\
       6   &  ??  &  ??   &      Extra information needed for direct access\\
\hline
\end{tabular}
\end{center}

The direct access method is one of the following:

\begin{center}
\begin{tabular}{|c|l|}
\hline
    Value &  Type\\
\hline
      0   &  BIOS disk access (default)\\
      1   &  IDE single-sector programmed I/O access\\
      2   &  SCSI device access via ASPI interface\\
     ??   &  Reserved for future use\\
\hline
\end{tabular}
\end{center}

Some of the high-speed access modes are not used automatically since it is
unsafe to assume that all hardware will support them.  Therefore if the direct
disk access information packet is absent and if BIOS access to the drive is
possible, slower BIOS accesses are used by default.  If it has been determined
by external software that a fast access mode is possible, the software should
write (or update) a direct disk access information packet to reflect this.  If
no access to the drive other than with a fast direct access mode is possible,
then the fast access mode will always be used.


\subsubsection{Packet 6 - Volume Unmount Information}

The volume unmount information packet contains information on conditions under
which SFS volumes are unmounted by the SFS driver.  These may be in response to
external stimuli, or due to internal conditions monitored by the driver.  In
the absence of this packet, the settings used for volume unmounts are the
defaults set by the driver, although these can generally be overridden under
user control.

Currently only one parameter, the auto-unmount timeout value, is defined.  This
contains the default setting for the auto-unmount timer for this volume.  If
no disk access takes place in the specified time, the volume is automatically
unmounted.  A value of 0 in this field indicates that no auto-unmount timer is
to be set (although this is better accomplished by simply deleting the volume
unmount information packet).

The packet layout is as follows:

\begin{center}
\begin{tabular}{|c|c|c|l|}
\hline
    Offset & Size &  Type  &     Description\\
\hline
       0   &   2  &  WORD  &     Volume unmount information packet ID = 6\\
       2   &   2  &  WORD  &     Volume unmount information packet data length\\
       4   &   2  &  WORD  &     Auto-unmount timeout in minutes, 0 = none set\\
\hline
\end{tabular}
\end{center}


\section{SFS Multiuser Access File Layout}

[The following information is preliminary and is bound to change at a moments
 notice]

Associated with each SFS volume which has multiuser access enabled are one or
more database files containing information on each user who has access to that
volume.  This consists of identification information for each user (in the case
of a named database), access control information, and keying information needed
to access the actual volume.

The file is laid out as a simple flat database, which is adequate for its
intended use since a user record is only retrieved once per database access,
and generally the databases will be quite small.

On a system with proper access control this file will be read-only for everyone
but the volume administrator, but unfortunately under some operating systems
and also DOS there is little access control and anyone with access to a sector
editor can change their access rights.  However at this level they can also
change the SFS volume header and the behaviour of mountsfs and the SFS driver,
so little is gained by attempting to plug this hole.  In any case, the basic
access to the volume is cryptographically controlled, and bypassing this goes
beyond simply editing a file.

In the future it may be worthwhile implementing more heavy-duty controls, such 
as using public-key signed access permission tokens from the volume 
administrator to grant access for networked use.

All data in the multiuser access database is stored in big-endian format
following the convention used by international cryptographic standards.  Care
should be taken to ensure that the proper endianness is maintained when reading
and writing the boot sector on little-endian systems (the performance of the
current implementation was tested against the RS6000 version to confirm that
endianness conversion was being done correctly).

In the following discussion a BYTE is an 8-bit quantity, a WORD is a
big-endian 16-bit quantity, and a LONG is a big-endian 32-bit quantity.  There are
no alignment restrictions for the data as stored in the database.  All data is
close-packed with no need for byte-padding for word or longword boundaries.

An access database is broken up into two sections, a header record containing
details on the database as a whole, and one or more individual user records.


\subsection{Database Header}

The database header is laid out as follows:

\begin{center}
\begin{tabular}{|c|c|c|l|}
\hline
    Offset & Size  &  Type     &   Description\\
\hline
       0   &   4   &  BYTE[4]  &   `SFS1' identification string\\
       4   &   2   &  WORD     &   Database type identifier\\
       6   &   4   &  LONG     &   Encrypted volume serial number\\
      10   &   2   &  WORD     &   Number of user records in database\\
      12   &   2   &  WORD     &   Database name character set identifier\\
      14   &   2   &  WORD     &   Database name length\\
      16   &  ??   &  BYTE[]   &   Database name\\
\hline
\end{tabular}
\end{center}

The identification string and database type identifier serve to identify the
general database format.  The type identifier is one of the following:

\begin{center}
\begin{tabular}{|r|l|}
\hline
    Value &  Type\\
\hline
      0   &  Anonymous database containing date, access, keyinfo records\\
      1   &  Named database containing user name, date, access, keyinfo records\\
     ??   &  Reserved for future use\\
\hline
\end{tabular}
\end{center}

An anonymous database contains no identification for any of the user records.
Each keyinfo record is decrypted in turn until a valid record is found, and
this keying information is then used to access the encrypted volume.

A named database contains an identifier for each user record.  Lookup of the
database is performed directly by user name, and then access proceeds as for
the anonymous database.

The encrypted volume serial number is the serial number of the volume this
database is used to access.

[Problem: Should be able to set to zero to keep corresponding volume anonymous]

[Problem: Can lead to trouble if chsfs is used to change volume serial number.
          Perhaps change chsfs to disallow serial number change without also
          changing database serial number]

The database name is the collective name for the database and the users in it.
Example database names might be ``Ancient Illuminated Seers of Bavaria'' or
``Trilateral Commission''.  The name is optional, and may be omitted by
specifying a name with a length of 0 characters.


\subsection{Database Records}

Each record in the database is laid out as follows:

\begin{center}
\begin{tabular}{|c|c|c|l|}
\hline
    Offset & Size &   Type    &    Description\\
\hline
       0   &   2  &   WORD    &    Record type identifier\\
       2   &   2  &   WORD    &    User name character set identifier\\
       4   &   2  &   WORD    &    User name length\\
       6   &  ??  &   BYTE$[ ]$  &    User name\\
       n   &   4  &   LONG    &    Access valid from date\\
     n+4   &   4  &   LONG    &    Access valid to date\\
     n+8   &   4  &   LONG    &    Access rights bitstring\\
    n+12   &  ??  &   BYTE$[ ]$  &    Keying information\\
\hline
\end{tabular}
\end{center}

The record type identifier is one of the following:

\begin{center}
\begin{tabular}{|r|l|}
\hline
    Value &  Type\\
\hline
      0   &  Anonymous record containing date, access, and keyinfo\\
      1   &  Named record containing user name, date, access, and keyinfo\\
     ??   &  Reserved for future use\\
\hline
\end{tabular}
\end{center}

The user name is the name of the user entitled to access the SFS volume the
database corresponds to.  Under multiuser operating systems this will simply
correspond to the userID which the user is known to the operating system under.
Otherwise, the userID can be the user's name or some similar identifying method
capable of distinguishing users.  If the record type is an anonymous record,
the user name fields will not be present.

[Problem: One user with many accounts.  Allow multiple userID's per record?
          This could encourage password sharing though]

The access valid date pair contains the date after which access to the SFS
volume is permitted, and the date at which access rights to the volume expire.
These are stored in the Unix seconds-since-1970 format.

The access rights bitstring contains the access rights the user has to the SFS
volume.

[Problem:~How?  Use RWXRWXRWX?  Or ASN.1-type strings with complex
          FTAM-type rights?  ACL's?]

The keying information is [what?  Same as single-user keying information]


\section{Interfacing with SFS}

The SFS device driver has a sophisticated control interface which allows
complete configurability through the standard DOS IOCTL read and write calls,
which are used to transfer data to and from the control channel of a device.
It is recommended that you consult a system programming guide or your compilers
documentation for more information on making DOS IOCTL calls.  Many compilers
provide standard routines for making these calls.

In the following discussion a BYTE is an 8-bit quantity, a WORD is a
little-endian 16-bit quantity, and a LONG is a little-endian 32-bit quantity.  These 
values will need endianness conversion after being read from the SFS volume 
header.


\subsection{How SFS Identifies Drives}

For floppy disk drives, a value of 0 usually corresponds to drive A: and a
value of 1 usually corresponds to drive B:.  For fixed disk drives accessible
through the system BIOS, a value of 0 with the high bit set (giving an actual 
value of 0x80) usually corresponds to the first physical drive and a value of 1 
with the high bit set (giving an actual value of 0x81) usually corresponds to 
the second physical drive.  These values correspond to physical drives as
accessed through the BIOS, and not logical volumes (SFS can also access drives
not normally accessible through the BIOS).  A single physical drive may contain 
a number of logical volumes.  Since the SFS driver runs below the level at which 
most operating systems operate, it will ignore the DOS drive mappings (in fact 
DOS can't even see the encrypted disk volume) and access the drive directly.  
Thus if there are more than the DOS standard of two floppy drives and two 
physical hard drives connected to the system it should still be able to access 
the drives and present them to the operating system as normal disk volumes.

If the physical drive contains more than one logical volume, the offset of the
start of the logical volume from the start of the physical drive is specified
as a sector or disk block count.  If the physical drive contains only one 
logical volume, this value is set to zero.  Finding the start of the logical 
volume will usually entail parsing the partition table of the physical drive.  
Floppy drives have only one logical volume which is equivalent in size to the 
physical drive, fixed disks have one or more logical volumes on each physical 
volume.  These logical volumes can contain different filesystems and operating 
systems, and not all may be accessible or visible to DOS.  For more information 
on disk organization, refer to a good technical reference which covers disk 
drives, and then spend several days figuring out all the DOS quirks never 
mentioned in any reference.

In the following text the term `drive number' is used to refer to a value which
identifies a particular drive (which depends on the drive type, since BIOS, IDE
and SCSI drive access identifies drives in different ways), the term `physical 
volume' is used to refer to the physical drive being accessed and the term 
`logical volume' is used to refer to the drive as DOS sees it.


\subsubsection{Reading from the SFS Driver}

The SFS driver processes both control channel read and control channel write
requests.  Initially, the drive number of the logical volume being used by the
SFS driver must be found.  This can be done by checking each possible logical
drive as follows:

\begin{verbatim}
    for( driveNumber = 0; driveNumber <= 'Z' - 'A'; driveNumber++ )
        {
        /* Try and read data from the device's control channel */
        data <- ioctlRead( driveNumber );

        /* Check for the SFS identification string */
        if( first 4 bytes of data = 'SFS1' )
            exit loop;
        }
\end{verbatim}

If, at the end of the loop's execution, the driveNumber is greater than the
total number of drives in the system, the SFS driver has not been found,
probably because it isn't present in the system.  Otherwise, driveNumber
contains the drive number of the logical volume which the SFS driver is making
available, and the data packet contains assorted status information about the
SFS volume corresponding to the driveNumber.

The data packet returned by the SFS driver is as follows:

\begin{center}
\begin{tabular}{|c|c|c|p{230pt}|}
\hline
    Offset & Size &   Type     &   Description\\
\hline
       0   &   4  &   BYTE$[ 4 ]$&   `SFS1' identification string\\
       4   &   2  &   WORD     &   SFS unit number, starting from 0\\
       6   &   2  &   WORD     &   Drive the SFS volume is mounted on.
				   Interpretation of this value depends on the
				   disk access mode.\\
       8   &   4  &   LONG     &   Sector or disk block offset of logical volume 
                                   from start of physical volume, or 0 if logical  
                                   volume corresponds to physical volume\\
      12   &   2  &   WORD     &   0 = no disk mounted, 1 = disk mounted, 2 = disk
                                   mounted in non-removable mode\\
      14   &   2  &   WORD     &   0 = disk is read/write, 1 = disk is read-only\\
      16   &   2  &   WORD     &   Quick-unmount hotkey value (high byte = shift\\
           &      &            &   value, low byte = optional keyboard scan code)\\
      18   &   2  &   WORD     &   Auto-unmount time.  -1 = no unmount timer set,\\
           &      &            &   0 = timer set but expired, any other value =\\
           &      &            &   total unmount time in minutes\\
      20   &   2  &   WORD     &   Auto-unmount timeout actual minutes remaining\\
           &      &            &   before the unmount takes place\\
      22   &   2  &   WORD     &   Internal driver check code.  0 = no error, 1 =\\
           &      &            &   driver consistency check failed, 2 = individual\\
           &      &            &   unit consistency check failed.\\
      24   &   4  &   DWORD    &   Last I/O error status.\\
\hline
\end{tabular}
\end{center}

The unit number is used by SFS to handle multiple encrypted drives.  This value
starts at zero for the first unit, and identifies the volume being accessed.

The last I/O error status value provides more details on the last read or write
error encountered by the SFS driver for the drive in question.  This value is
divided into 4 bytes, of which the first contains the access mode in use when
the error occurred, the remaining three contain error information.  The
possible values are as follows:

\begin{center}
\begin{tabular}{|c|c|l|}
\hline
    First byte & Access mode  &  Remaining bytes\\
\hline
        0      &    BIOS      &   Byte 1 = 0\\
               &              &   Byte 2 = 0\\
               &              &   Byte 3 = BIOS error code\\
        1      &    IDE       &   Byte 1 = 0\\
               &              &   Byte 2 = controller status\\
               &              &   Byte 3 = controller error code\\
        2      &    SCSI      &   Byte 1 = sense key\\
               &              &   Byte 2 = auxiliary sense key\\
               &              &   Byte 3 = auxiliary sense key qualifier\\
       ??      &  Reserved    &   Reserved for future use\\
\hline
\end{tabular}
\end{center}

In all cases a value of all ones in bytes 2 and 3 indicate a controller or host
device timeout.

If queried via an IOCTL call on the volume's removability, the SFS driver will
always indicate that the media is removable, even if the volume is on a fixed
disk.  This is because some disk cacheing software won't invalidate a drive's
data, even if the driver signals that the media has changed, unless the volume
is marked as being removable.

This status information can be returned to the user in the form of a status
message or an information dialog if desired (the drive and offset information 
can be used to read the volume name, date, and serial number from the encrypted 
volume).


\subsubsection{Writing to the SFS Driver}

Once the driver has been found, several types of data packet can be sent to the
driver to control its operation.  Each control packet begins with a WORD
containing the magic value `C0' which the SFS driver checks to ensure the
packet is meant for it, and a WORD identifying the data packet type.  The data
packet types are:

\begin{center}
\begin{tabular}{|l|c|l|}
\hline
    Name                   & Value   &    Information type\\
\hline
    PACKET\_SET\_DISKINFO    &   0     &    Disk parameters\\
    PACKET\_SET\_KEYINFO     &   1     &    Keying information\\
    PACKET\_SET\_READONLY    &   2     &    Set disk read-only status\\
    PACKET\_SET\_DRIVENO     &   3     &    Drive number to mount\\
    PACKET\_SET\_MOUNTSTATUS &   4     &    Set mount status\\
    PACKET\_SET\_UNMOUNT     &   5     &    Set/clear quick-unmount\\
                             &         &    hotkey value\\
    PACKET\_SET\_TIMEOUT     &   6     &    Set/clear timed unmount value\\
\hline
\end{tabular}
\end{center}

These packets are explained in more detail below.  All other values are
reserved for future use.

\begin{itemize}
\item Packet 0 - Disk Parameters:

This packet type is used to convey to the SFS driver various pieces of
information about the physical characteristics of the disk drive, including the
sector size, number of heads, total number of sectors, and so on.  The packet
layout is as follows:

\begin{center}
\begin{tabular}{|c|c|c|l|}
\hline
    Offset & Size &   Type    &    Description\\
\hline
       0   &   2  &   WORD    &    Magic value `C0'\\
       2   &   2  &   WORD    &    Packet type 0\\
       4   &   2  &   WORD    &    Sector size in bytes\\
       6   &   1  &   BYTE    &    Sectors per cluster\\
       7   &   2  &   WORD    &    Number of boot sectors\\
       9   &   1  &   BYTE    &    Number of FAT copies\\
      10   &   2  &   WORD    &    Number of entries in root dir\\
      12   &   2  &   WORD    &    Number of sectors on disk\\
      14   &   1  &   BYTE    &    Media descriptor byte\\
      15   &   2  &   WORD    &    Number of sectors per FAT\\
      17   &   2  &   WORD    &    Number of sectors per track\\
      19   &   2  &   WORD    &    Number of heads\\
      21   &   4  &   LONG    &    Number of hidden sectors\\
      25   &   4  &   LONG    &    Number of sectors per disk, 32-bit\\
\hline
\end{tabular}
\end{center}

Sending this data packet to the driver automatically unmounts the encrypted
volume.


\item Packet 1 - Keying information

This packet type sets up the encryption information in the driver.  It contains
the en/decryption key, and the master IV for the encrypted volume.  The packet
layout is as follows:

\begin{center}
\begin{tabular}{|c|c|c|l|}
\hline
    Offset & Size  &  Type       & Description\\
\hline
       0   &   2   &  WORD       & Magic value `C0'\\
       2   &   2   &  WORD       & Packet type 1\\
       4   &  20   &  BYTE$[ 20 ]$ & Master IV for encrypted volume\\
      24   &  64   &  BYTE$[ 64 ]$ & MDC/SHS keying information\\
\hline
\end{tabular}
\end{center}

Sending this data packet to the driver automatically unmounts the encrypted
volume.


\item Packet 2 - Set disk Read-only Status

This packet type sets the read-only status of the disk.  Sending a value of 0
makes the disk read-only.  Sending a value of 1 makes the disk read/write.  The
packet layout is as follows:

\begin{center}
\begin{tabular}{|c|c|c|l|}
\hline
    Offset & Size  &  Type   &     Description\\
\hline
       0   &   2   &  WORD   &     Magic value `C0'\\
       2   &   2   &  WORD   &     Packet type 2\\
       4   &   2   &  WORD   &     Read-only status: 0 = read-only, 1 = read/write\\
\hline
\end{tabular}
\end{center}

Sending this packet type has no effect on the mount status of the volume.


\item Packet 3 - Drive Number to Mount

This packet type sets the drive number and the offset of the logical volume on 
the physical drive, which the driver will en/decrypt.  The packet layout is as 
follows:

\begin{center}
\begin{tabular}{|c|c|c|p{212.4pt}|}
\hline
    Offset & Size  &  Type  &      Description\\
\hline
       0   &   2   &  WORD  &      Magic value `C0'\\
       2   &   2   &  WORD  &      Packet type 3\\
       4   &   2   &  WORD  &      Drive number.  Interpretation of this value
                                   depends on the disk access mode.\\
       6   &   4   &  LONG  &      Sector or disk block offset of logical volume 
				   from start of physical volume, or 0 if logical 
				   volume corresponds to physical volume\\
\hline
\end{tabular}
\end{center}

The drive number contains information allowing the driver to locate the
physical drive on which the SFS volume is located.  The drive access mode and
drive identifying information are encoded into a 16-bit word as follows:

\begin{center}
\begin{tabular}{|l|l|}
\hline
    Bits 15--12 & Disk access mode\\
    Bits 11--0  & Drive identifying information\\
\hline
\end{tabular}
\end{center}

The drive identifying information is laid out as follows:

\begin{center}
\begin{tabular}{|c|l|}
\hline
    Access mode   &  Drive identifying informtion\\
\hline
         0        &  Bits 11--8: 0\\
                  &  Bits 7--0 : BIOS drive number\\
         1        &  Bits 11--8: 0\\
                  &  Bits 7--0 : IDE drive number\\
         2        &  Bits 11--8: SCSI host number\\
                  &  Bits 7--4 : SCSI target ID\\
                  &  Bits 3--0 : SCSI logical unit number\\
\hline
\end{tabular}
\end{center}

Sending this data packet to the driver automatically unmounts the encrypted
volume.


\item Packet 4 - Set mount Status

This packet type sets the mount status of the drive.  A value of 0 unmounts the
drive, a value of 1 mounts the drive.  The packet layout is as follows:

\begin{center}
\begin{tabular}{|c|c|c|l|}
\hline
    Offset & Size  &  Type   &     Description\\
\hline
       0   &   2   &  WORD   &     Magic value `C0'\\
       2   &   2   &  WORD   &     Packet type 4\\
       4   &   2   &  WORD   &     Mount status (0 = unmount, 1 = mount)\\
\hline
\end{tabular}
\end{center}

Unmounting the drive using this packet type also destroys the encryption
information held by the driver and forces all data still held in cache and disk
buffers to be flushed to disk.  This is the only sure way to erase all the
encryption information, as it wipes not only the actual keying information but
also most of the data area used by the driver and any data in cache and disk
buffers.  Merely sending an empty keying information packet will not perform
this task.


\item Packet 5 - Set/clear Quick-Unmount Hotkey Value

This packet is used to set the hotkey value which the SFS driver checks for to
quickly unmount an SFS volume, or to clear the hotkey handling.  The packet
layout is as follows:

\begin{center}
\begin{tabular}{|c|c|c|l|}
\hline
    Offset & Size  &  Type   &     Description\\
\hline
       0   &   2   &  WORD   &     Magic value `C0'\\
       2   &   2   &  WORD   &     Packet type 5\\
       4   &   2   &  WORD   &     Hotkey value (see below)\\
\hline
\end{tabular}
\end{center}

The hotkey which the driver checks for can be any combination of the alt key,
control key, left shift, and right shift (referred to as the shift code), and
an arbitrary keyboard scan code.  The recommended default value is a
combination of the left and right shift keys.  This information is encoded in
the hotkey WORD as follows:

  The high 8 bits contain the shift code.  This contains a set of bitflags
  which specify the shift keys the driver checks for, possibly in addition to
  an optional character code.  The values are:

\begin{center}
\begin{tabular}{|c|c|c|}
\hline
        Shift key   &    Value (binary)   &   Value (hex)\\
\hline
           Alt      &       00001000      &       08\\
         Control    &       00000100      &       04\\
       Left Shift   &       00000010      &       02\\
      Right Shift   &       00000001      &       01\\
\hline
\end{tabular}
\end{center}

  The encoding for the shift code portion of the default left+right shift
  hotkey is therefore 00000011 binary or 03 hex.

  The low 8 bits contain an optional keyboard scan code which the driver will
  check for in addition to the shift code.  This value is used in combination
  with the shift code to specify a key combination to the driver, so that for
  example Alt-Z would be used as the quick-unmount hotkey, although care should
  be taken to ensure that none of a large number of existing special key
  combinations is reused for this.

  If the low 8 bits contain zeroes, the driver does not check for a keyboard
  scan code in addition to a shift code.  The encoding for the keyboard scan
  code portion of the default left+right shift hotkey is therefore 00000000
  binary or 00 hex.

  The overall value is the combination of the shift code and keyboard scan
  code, or 0300 hex for the default hotkey.

Specifying a value of 0 in this word will disable hotkey checking in the
driver.

The hotkey quick-unmount is handled by hooking the int 9h keyboard interrupt.
The interrupt handler which performs this task is installed either when the
driver is loaded (if there is a volume to be mounted or a hotkey code is
specified), or when mountsfs is run (which transmits a quick-unmount control
packet to the driver).  The interrupt handler checks for the programmed hotkey
and unmounts the volume if it is detected.  Subsequent quick-unmount control
packets can be sent to the driver to change the actual hotkey value.  Sending a
value of 0 will disable hotkey checking and deinstall the keyboard interrupt
handler.


\item Packet 6 - Set/clear Timed Unmount Value

This packet is used to set the unmount time after which an SFS volume is
unmounted if no accesses are made to it.  The packet layout is as follows:

\begin{center}
\begin{tabular}{|c|c|c|l|}
\hline
    Offset & Size  &  Type    &    Description\\
\hline
       0   &   2   &  WORD    &    Magic value `C0'\\
       2   &   2   &  WORD    &    Packet type 5\\
       4   &   2   &  WORD    &    Timeout value in minutes before unmount\\
           &       &          &    takes place\\
\hline
\end{tabular}
\end{center}

Specifying a value of 0 as the timeout value will disable the timed unmount
feature in the driver.

The timed unmount is handled by hooking the int 1Ch timer interrupt.  The
interrupt handler which performs this task is installed either when the driver
is loaded (if a timeout value is specified at this time), or when mountsfs is
run and a timeout value is specified (which transmits a timed unmount control
packet to the driver).  The interrupt handler decrements the timer and unmounts
the volume if it expires.  The timer is reset every time a media check packet
is received by the driver, which allows normal DOS accesses to be detected but
doesn't cause false triggering due to device driver control functions.
Subsequent timed unmount control packets can be sent to the driver to change
the actual timeout value for a volume.  Sending a value of 0 will disable the 
timer and, if no more volumes have timers set, deinstall the timer interrupt 
handler.

\end{itemize}

\subsection{Controlling the Driver}

The SFS driver is initialized by sending it a series of control packets giving
the disk information, the keying information, the drive number and logical
volume offset to access, the read/write status, and finally the mount command.
The sequence of operations for mounting a volume is as follows:

\begin{verbatim}
    driveNumber <- find SFS drive number as outlined above;
    ioctlWrite( driveNumber, PACKET_SET_DISKINFO,
                disk information - packet type 0 );
    ioctlWrite( driveNumber, PACKET_SET_KEYINFO,
                keying information - packet type 1 );
    ioctlWrite( driveNumber, PACKET_SET_READONLY,
                read-only status - packet type 2 );
    ioctlWrite( driveNumber, PACKET_SET_DRIVENO,
                drive to mount and vol.offs - packet type 3 );
    ioctlWrite( driveNumber, PACKET_SET_MOUNTSTATUS,
                mount status TRUE - packet type 4 );
\end{verbatim}

To mount a new volume on the same drive:

\begin{verbatim}
    driveNumber <- find SFS drive number as outlined above;
    ioctlWrite( driveNumber, PACKET_SET_KEYINFO,
                keying information - packet type 1 );
    ioctlWrite( driveNumber, PACKET_SET_MOUNTSTATUS,
                mount status TRUE - packet type 4 );
\end{verbatim}

To unmount a volume, erasing the keying information held by the device driver:

\begin{verbatim}
    driveNumber <- find SFS drive number as outlined above;
    ioctlWrite( driverNumber, PACKET_SET_MOUNTSTATUS,
                mount status FALSE - packet type 4 );
\end{verbatim}


\section{Interfacing with mountsfs}

In order to facilitate the use of SFS with other software such as graphical
front-ends, mountsfs can be run in batch mode in which it accepts abbreviated
forms of the usual commands and outputs more complex results to fixed-record
data files instead of the screen.


\subsection{Controlling mountsfs in Batch Mode}

In order to enable these features, the first option given to mountsfs must be
the keyword `batch'.  This allows it to recognise alternative single-letter
forms of the normal commands.  These single-letter options and their equivalent
commands are as follows:

\begin{center}
\begin{tabular}{|c|l|}
\hline
    Letter  &    Full command\\
\hline
      f     &    user file\\
      h     &    hotkey\\
      i     &    info or information\\
      n     &    user name\\
      p     &    password entry\\
      s     &    status\\
      t     &    timeout\\
      u     &    unmount\\
      v     &    volume name\\
\hline
\end{tabular}
\end{center}

The volume name, hotkey, timeout, user name, and user file options must be
followed by the appropriate extra information in the usual format.  The
password entry option should be followed by the password for the encrypted
volume.  For example to mount the volume ``Test'' with a quick-unmount hotkey of
Ctrl-Alt-Z, an auto-unmount timeout of 10 minutes, and a password of ``secret
data'' the command would be:

    {\tt \verb|  |mountsfs batch vTest hCtrlAltZ t10 "psecret data"}

which is equivalent to the usual:

    {\tt \verb|  |mountsfs vol=test hotkey=ctrlAltZ timeout=10}


\subsection{mountsfs Output in Batch Mode}

During normal operation mountsfs may print several lines of information to the
screen giving details on the status of the operation being performed and
details on currently mounted and unmounted SFS volumes.  When mountsfs is run
in batch mode, this information is written to a fixed-record data file named
``sfs.tmp'' instead of to the screen.  The location of the file is given by the
``TMP'' environment variable; if this variable does not exist, the file is
written to the root directory of the C: drive.  If the TMP environment variable
points to a drive without a path, the file is written to the root directory of
that drive.  Thus if the TMP environment variable were set as:

    {\tt \verb|  |set tmp=e:$\backslash$data$\backslash$}

then mountsfs would output the results of the status or info/information
command to the file ``e:$\backslash$data$\backslash$sfs.tmp''.  If the variable were set as:

    {\tt \verb|  |set tmp=a:}

then mountsfs would output its data to the file ``a:$\backslash$sfs.tmp''.

The mountsfs output is written to the file as a series of fixed records.  Each
record is in ASCII text format with individual fields beginning on new lines,
making parsing by external software a simple task.  All control fields are in
uppercase text.  The exact record format is determined by the record type which
is specified in the first line.  In most cases the only record written will be
the result from the mountsfs execution, although the status and
info/information commands produce one or more extra output records which are
detailed in the section ``The Status and Info Commands in Batch Mode'' below.

The mountsfs result record is laid out as follows:

\begin{center}
\begin{tabular}{|c|l|}
\hline
    Field            &   Field contents\\
\hline
    Record type      &   {\tt RESULT}\\
    Success status   &   {\tt TRUE} $or$ {\tt FALSE}\\
    Error message    &   {\em string: error message, or empty string}\\
\hline
\end{tabular}
\end{center}

The record type field identifies the following record as being a mountsfs
result record.

The success status field gives the success status of the mountsfs command.  A
value of ``TRUE'' indicates the command succeeded, a value of ``FALSE'' indicates
the command failed.

The error message field gives the error message generated by mountsfs if the
success status field is set to FALSE, or an empty string if the success status
field is set to TRUE.

A successful mountsfs command result record would be as follows:

\begin{center}
\begin{tabular}{|c|l|}
\hline
    Field              &     Field data\\
\hline
    Record type        &     {\tt RESULT}\\
    Success status     &     {\tt TRUE}\\
    Error message      & \\
\hline
\end{tabular}
\end{center}

A failed mountsfs command result record might be as follows:

\begin{center}
\begin{tabular}{|c|l|}
\hline
    Field              &     Field data\\
\hline
    Record type        &     {\tt RESULT}\\
    Success status     &     {\tt FALSE}\\
    Error message      &     {\tt Cannot find SFS driver}\\
\hline
\end{tabular}
\end{center}


\subsection{The Status and Info Commands in Batch Mode}

Apart from the result record, mountsfs produces two other types of output
records, the output from a status command and the output from an info/
information command.  As with the RESULT record, each record is in ASCII text
format with individual fields beginning on new lines, making parsing by
external software a simple task.  All control fields are in uppercase text.
The exact record format is determined by the record type which is specified in
the first line.  This is either ``STATUS'' for the output from a status command,
or ``INFORMATION'' for the output from an info/information command.

The status command output record is laid out as follows:

\begin{center}
\begin{tabular}{|c|l|}
\hline
    Field              & Field contents\\
\hline
    Record type        & {\tt STATUS}\\
    Unit number        & {\em number: drive unit number}\\
    Mount status       & {\tt TRUE} $or$ {\tt FALSE}\\
    Volume name        & {\em string: volume name, or empty string}\\
    R/W access allowed & {\tt TRUE} $or$ {\tt FALSE}\\
    Drive letter       & {\em character: uppercase driver letter}\\
    Hotkey             & {\em Ctrl, Alt, LeftShift, RightShift = {\tt `c'}, {\tt `a'}, {\tt `l'}, {\tt `r'}}\\
                       & {\em Letter=uppercase letter.}\\
                       & {\em Example: Ctrl-Alt-Z = }{\tt caZ}\\
                       & $or$ {\tt FALSE}\\
    Timeout total time & {\em number: time in minutes, or} {\tt FALSE}\\
    Timeout time left  & {\em number: time in minutes, or} {\tt FALSE}\\
\hline
\end{tabular}
\end{center}

The record type field identifies the following record as being from the output
of the status command.

The unit number field is the drive unit number used by the SFS driver.

The mount status field contains ``TRUE'' if there is a volume mounted, or ``FALSE''
if there are no volumes mounted.  If this field contains the value ``FALSE'' then
the following fields will contain null values (either the value ``FALSE'' or
blank fields as appropriate).

The volume name field contains the name of the volume.  This may contain any
type of character, or may be an empty string if no volume name is present.

The R/W access allowed field contains ``TRUE'' if read/write access to the volume
is allowed, or ``FALSE'' if write access is disabled and the volume is mounted
read-only.

The drive letter field contains the uppercase drive letter which the mounted
SFS volume corresponds to.

The hotkey field contains the quick-unmount hotkey value or ``FALSE'' if none is
set.  The shift keys (if any) are given in lowercase and the ASCII key value
(if any) is given in uppercase.  The shift key codes are Ctrl = `c', Alt = `a',
leftShift = `l', rightShift = `r'.  These are followed by the ASCII key value,
if there is one.  Thus the code for the default quick-unmount hotkey
combination leftShift-rightShift would be ``lr'', and the code for the hotkey
combaination Ctrl-Alt-Z would be ``caZ''.

The timeout total time and timeout time left fields contain either the
auto-unmount total time in minutes and the time left before the unmount takes
place in minutes, or the control string ``FALSE'' if no timeout is set.

A typical status command output record might be as follows:

\begin{center}
\begin{tabular}{|l|l|}
\hline
    Field                &   Field data\\
\hline
    Record type          &   {\tt STATUS}\\
    Unit number          &   {\tt 0}\\
    Mount status         &   {\tt TRUE}\\
    Volume name          &   {\tt Encrypted data}\\
    R/W access allowed   &   {\tt TRUE}\\
    Drive letter         &   {\tt G}\\
    Hotkey               &   {\tt lr}\\
    Timeout total time   &   {\tt 15}\\
    Timeout time left    &   {\tt 4}\\
\hline
\end{tabular}
\end{center}

This would then usually be followed by a RESULT record.

The information command output record is laid out as follows:

\begin{center}
\begin{tabular}{|l|l|}
\hline
    Field              & Field contents\\
\hline
    Record type        & {\tt INFORMATION}\\
    Volume name charset& {\em string: character set name, or} {\tt FALSE}\\
    Volume name        & {\em string: volume name, or empty string}\\
    Volume date        & {\em string: date in YYMMDDHHMMSS format}\\
    Volume ID          & {\em number: volume serial number}\\
    Volume size        & {\em number: volume size in kilobytes, or} {\tt FALSE}\\
    Filesystem type    & {\em string: filesystem type, or} {\tt FALSE}\\
    Mount ID           & {\em string$[8]$: automount ID, or} {\tt FALSE}\\
    Multiuser access   & {\tt TRUE} $or$ {\tt FALSE}\\
    Mount status       & {\em character: uppercase driver letter, or} {\tt FALSE}\\
    Fast access mode   & {\em number: access mode, or} {\tt FALSE}\\
    Current access mode& {\em number: access mode, or} {\tt FALSE}\\
    Timeout            & {\em number: timeout in minutes, or} {\tt FALSE}\\
\hline
\end{tabular}
\end{center}

The record type field identifies the following record as being from the output
of the info/information command.

The volume name character set field contains the name of the character set used
in the volume name.  This is either ``ISO 646'', ``ISO 8859-1'', ``ISO 8859-2'', ``ISO
8859-3'', ``ISO 8859-4'', ``ISO 8859-5'', ``ISO 8859-6'', ``ISO 8859-7'', ``ISO 8859-8'',
``ISO 8859-9'', or ``FALSE'' if the character set is unknown.

The volume name field contains the name of the volume.  This may contain any
character from the character set given in the Volume name character set field,
or may be an empty string if no volume name is present.

The volume date field contains the volume date in YYMMDDHHMMSS (two-digit year,
month, day, hour, minute, second) format.

The volume ID field contains the volume serial number.

The volume size field contains the volume size in kilobytes, or ``FALSE'' if the
volume is removable media.

The filesystem type field contains the name of the filesystem the volume
contains.  This is either ``DOS'', or ``FALSE'' if the filesystem is unknown.

The mount identifier field contains the 8-character mount identifier for the 
volume if the volume is on a fixed disk, or ``FALSE'' if the volume is on a 
removable disk and no mount at system startup is possible.

The multiuser access field contains ``TRUE'' if multiuser access to the volume is
possible or ``FALSE'' if only single-user access is possible.

The mount status field contains the uppercase drive letter which the mounted
SFS volume corresponds to if the volume is mounted, or ``FALSE'' if it is
unmounted.

The fast access mode field contains the fast disk access mode which SFS will
use if possible when accessing this volume, or ``FALSE'' if the default (slower)
access mode will be used.  The current access mode contains the actual access
mode being used to access the volume, or ``FALSE'' if the default (slower) access
mode is being used.

The timeout field contains the default time in minutes before a volume is
auto-unmounted.  This value may be overridden by mountsfs or the SFS driver.

The information command output records corresponding to the volumes given in 
the section ``Mounting an SFS Volume'' above would be as follows:

\begin{center}
\begin{tabular}{|l|l|}
\hline
    Field                &   Field data\\
\hline
    Record type          &   {\tt INFORMATION}\\
    Volume name charset  &   {\tt ISO 646}\\
    Volume name          &   {\tt Data backup}\\
    Volume date          &   {\tt 931101101301}\\
    Volume ID            &   {\tt 1234}\\
    Volume size          &   {\tt FALSE}\\
    Filesystem type      &   {\tt DOS}\\
    Mount ID             &   {\tt FALSE}\\
    Multiuser access     &   {\tt FALSE}\\
    Mount status         &   {\tt FALSE}\\
    Fast access mode     &   {\tt FALSE}\\
    Current access mode  &   {\tt FALSE}\\
    Timeout              &   {\tt FALSE}\\
    Record type          &   {\tt INFORMATION}\\
    Volume name charset  &   {\tt ISO 646}\\
    Volume name          &   {\tt Personal financial records}\\
    Volume date          &   {\tt 930906112219}\\
    Volume ID            &   {\tt 177545}\\
    Volume size          &   {\tt 10000}\\
    Filesystem type      &   {\tt DOS}\\
    Mount ID             &   {\tt 03A12F7B}\\
    Multiuser access     &   {\tt FALSE}\\
    Mount status         &   {\tt E}\\
    Fast access mode     &   {\tt FALSE}\\
    Current access mode  &   {\tt FALSE}\\
    Timeout              &   {\tt 30}\\
    Record type          &   {\tt INFORMATION}\\
    Volume name charset  &   {\tt ISO 646}\\
    Volume name          &   {\tt Encrypted data disk}\\
    Volume date          &   {\tt 930412221700}\\
    Volume ID            &   {\tt 69231461}\\
    Volume size          &   {\tt 42456}\\
    Filesystem type      &   {\tt DOS}\\
    Mount ID             &   {\tt 42DD2536}\\
    Multiuser access     &   {\tt TRUE}\\
    Mount status         &   {\tt FALSE}\\
    Access mode          &   {\tt 1}\\
    Current access mode  &   {\tt 1}\\
    Timeout              &   {\tt 10}\\
\hline
\end{tabular}
\end{center}

This would then usually be followed by a RESULT record.

In order to make use of these records, the controlling program should invoke
mount\-sfs in batch mode with either the `s' status or `i' information command.
mountsfs will run and, if no errors are encountered, write its output to the
output file.  The controlling program can then read the data from the file,
delete the file, and handle the information as appropriate.


\section{Selected Source Code}

This section contains a walkthrough of selected portions of the source code
(mainly the encryption-related parts) to allow the verification of its
correctness and to help people wishing to write SFS-compatible software.

[!!!! Not in the early releases to save space and because I haven't had
      time to add it yet, should be in a later version !!!!]


\section{Future Work}

The following ideas may be incorporated into future versions of SFS if
requested by users.  In addition reasonable requests for other improvements may
also find their way into SFS.

\begin{itemize}

\item Improve error recovery for the encryption process.  This is somewhat
   difficult, and will probably involve keeping a copy of status information
   and the track currently being encrypted on a local volume to allow a restart
   in the case of a power failure.  There are problems inherent in this as it
   involves storing sensitive data in a disk file, and will slow down the
   processing considerably due to the need to write each track to two
   physically separate disk volumes instead of one continuous one.  A partial
   solution is to keep the status information (simply an index of the disk
   section currently being encrypted) in the volume header while mksfs is
   running and provide some sort of restart option if power is lost, although
   what happens if power dies halfway through writing a track is uncertain.

\item A plug-in card which contains a BIOS extension which hooks int 13h and
   encrypts an entire physical disk (not just one disk volume), from the
   master boot record at the start to the very last sector at the end.  This
   means all disk I/O must be done via int 13h and won't work on all systems.
   Cost for the card is estimated to be about \$80-\$100 for the basic version and
   up to \$200 for the full hardware version whose throughput it significantly
   higher than the basic version.

\end{itemize}

\section{Recommended Reading}

In recent years a large number of books and articles on crytography have
appeared.  Many of these are beyond the level of anyone with a casual interest
in the subject, but a good overview is given by Dorothy Dennings (now somewhat
dated) ``Cryptography and Data Security'', published by Addison-Wesley in 1982
(ISBN 0-201-10150-5).  Bruce Schneier's much more recent ``Applied
Cryptography'', published by John Wiley and Sons in 1993 (ISBN 0-471-59756-2),
is probably the best single text on cryptography currently available, and is
recommended reading for anyone wanting more information on the principles
behind SFS.  This book also contains a large list of references to more
information on all manner of encryption algorithms, protocols, and technology.
Voydock and Kent's tutorial ``Security Mechanisms in High-Level Network
Procotols'', published in the ACM Computing Surveys Vol.15, No.2 (June 1983)
provides a good overview of design considerations for encryption systems.

The algorithms and techniques used in SFS are laid down in the following
national and international standards:

\begin{itemize}

\item ANSI X3.106, ``American National Standard for Information Systems --- Data
        Encryption Algorithm --- Modes of Operation''

\item ANSI X9.30 Part 2, ``The Secure Hash Algorithm (SHA)''

\item Australian Standard AS 2805.5.2, ``Electronic Funds Transfer --- Requirements
        for Interface'', Part 5.2, ``Ciphers --- Modes of operation for an n-bit
        block cipher algorithm''

\item ISO 10116:1991, ``Information technology --- Modes of operation for an n-bit
        block cipher algorithm''\footnote{
%Footnote [1]: 
		These publications are available for a horribly high price from
                ISO-affiliated national standards bodies in most countries.
  }

\item ISO 10126-2:1991, ``Banking --- Procedures for message encipherment
        (wholesale) --- DEA Algorithm''

\item NBS FIPS pub. 81, ``DES Modes of Operation'', 1980\footnote{
%Footnote [2]: 
		These publications are available from the National Technical
              	Information Service, Springfield, Virginia 22161.  NTIS will take
              	telephone orders on +1 (703) 487-4650 (8:30 AM to 5:30 PM Eastern
              	Time), fax +1 (703) 321-8547.  For assistance, call +1 (703)
              	487-4679.
  }

\item NIST FIPS pub. 180, ``Secure Hash Standard'', 1993

\end{itemize}

Information on the weaknesses of some cryptosystems are published in a variety
of places.  The largest publicly available sources of information are the
cryptography conference proceedings which are part of the Springer-Verlag
``Lecture Notes in Computer Science'' series.  Eli Biham and Adi Shamir's
``Differential Cryptanalysis of the Data Encryption Standard'', also published by
Springer-Verlag (ISBN 0-387-97930-1), gives information on the resistance to
attack of a number of block ciphers.

Charles Pfleegers book ``Security in Computing'', published by Prentice Hall in
1989 (ISBN 0-13-798943-1) provides a thorough overview of computer security
techniques, including coverage of more obscure areas such as ethical issues.

A good technical coverage of smart cards can be found in ``Smart Card 2000: The
Future of IC Cards'', published by North-Holland in 1988 (ISBN 0-444-70545-7).
The physical and interface characteristics of smart cards are covered in the
following international standard:

\begin{itemize}
\item ISO 7816:1991, ``Identification cards---Integrated circuit card with
        contacts''
\end{itemize}

Finally, James Bamford's ``The Puzzle Palace'', published by Houghton-Mifflin in
1982, is a good source of information on the operation of the NSA, albeit
slightly dated (it predates the widespread availability of encryption for
personal computers, for example).

This list only scratches the surface of the full range of cryptographic and
security literature available - a very detailed bibliography can be found in
the ``Applied Cryptography'' book.


\section{Using SFS}

In general SFS is free for personal (private) use.  However if you find SFS of
use then in order to allow continued development of and enhancements to SFS, in
particular the creation of more user-friendly versions of mksfs and mountsfs,
of versions for other systems, of a low-cost plug-in card containing whole-disk
encryption firmware, and the replacement of the drive I fried testing SFS, a
donation of \$25 would be appreciated.  Alternatively, the donation can be used
to cover the legal costs of those people involved in the current US government
investigation of the PGP encryption program.

Use of SFS in a commercial, government, or institutional environment or for
business purposes is allowed for free for 30 days to allow it to be evaluated.
After this period it should be registered for further use.  The registration
fee covers use of SFS on any one machine at any one time (so you could, for
example, use SFS on a machine at work and keep a copy on your notebook PC for
use while travelling).

If you decide to send a donation or registration, please specify whether you'd
like it to be used for further SFS development work or if it is to go into the
PGP legal fund.  Cheques or money orders can be sent to:

\begin{verbatim}
        Peter Gutmann
        24 Durness Pl.
        Orewa
        Auckland
        New Zealand
\end{verbatim}

I can also be contacted through email at (among others) the following
addresses:

\begin{verbatim}
        pgut1@cs.aukuni.ac.nz
        p.gutmann@cs.auckland.ac.nz
        peterg@kcbbs.gen.nz
\end{verbatim}

with the first address being the preferred one.  Finally, the fastest way to
contact me is by phone between about 10am and 1am at the following number
(remembering that NZ local time is GMT+13, so we're usually about 18-20 hours
ahead of the US and about 12 hours ahead of Europe):

\begin{verbatim}
        +64 9 426-5097
\end{verbatim}


Testimony from one of our satisfied customers:

``I hear this crash and I find a rock, wrapped in paper, next to my living room
  window.  I open up the note and it says `You want it in writing? You got it.
  Next time, use a {\em real} encryption program.  SFS.  We know where you live' ''.

  So why aren't {\em you} using SFS?

Here's what reviewers have been saying about SFS:

``Version 1.1 has several of the advanced features recommended in version 1.0,
  but not all of the ones I'd like to see in version 1.2.  So, it's pretty good
  except when it's not.  Three stars.

  You probably won't use half the features anyway.  I'm a little ticked off
  that it clashes with my most exotic memory-resident programs, but otherwise
  the software runs just fine on my Turbo Rambuster 586.  It will probably run
  like molasses on your 386SX.

  It's great value at \$25, and I recommend you register it, although that's
  easy for me to say because reviewers get freebies''.


\section{Credits}

Thanks to the readers of comp.os.msdos.programmer and comp.os.linux.develop\-ment
and in particular Jouni Kosonen, for their valuable advice with low-level DOS
disk handling and organization, and the users of the Enigma BBS, in particular
Arne Rohde, for helping test-run several early versions of the low-level disk
handling code on various drive configurations and for providing useful advice
on the vagaries of the PC's disk handling.   Thanks also to Steven Altchuler
for performing all sorts of dangerous tests on SFS volumes and for his heroic
efforts in finding the obscure mountsfs bug, and to Vadim Vlasov for help with 
various thorny problems with low-level disk and device driver programming.

Matthias Bruestle (the greatest chemist of all time and space) provided
valuable feedback on assorted problems in mksfs and mountsfs, and risked his
disk drives testing the SFS driver and other SFS software.  At the behest of
the author he also risked imprisonment for Datenvergewaltigung by letting mksfs
near his Linux partition.

Colin Plumb made noises about various problems with CFB-mode encryption,
corrected me when I tried to explain the need for encrypted IV's in email at
3am, and provided much useful feedback on SFS's operation in general, including
the very elegant solution to the CFB-mode encryption problem and help with the
disk overwriting method.

Vesselin Bontchev sent in long lists of suggested improvements to SFS, and
provided an ongoing commentary on features and ideas for the software (it seems
half the functionality provided by SFS is due to his suggestions).  Ralf
Brown's interrupt list provided help with the haze of obscure interrupts used
by mksfs in its quest for random information.

Thanks to John Huttley for the loan of a SCSI controller, Peter Shields for the
loan of a SCSI cable, Stuart Woolford for the loan of a SCSI drive which didn't
work, and John Huttley for the loan of a SCSI drive which did, in order to get
the SCSI interface code for SFS going.

Peter Conrad typset the entire set of SFS docs into \LaTeX\ format.

The Beerdigungsinstitut Utzmann (erstes Erlanger Beerdigungsunternehmen) help\-ed
clean up the casualties from the SFS beta-testing.  Finally, Tony, Geezer,
Bill, Ozzy, Ron, and Grandos helped me stay awake during many long nights of
debugging low-level disk access code and encryption routines.

Various other people have at various times offered help and suggestions on
getting SFS going.  My thanks go to them also.


\section{Warranty}

\subsection{Customer Obligations}

\begin{enumerate}

\item Customer assumes full responsibility that this program meets the
     specifications, capacity, capabilities, and other requirements of said
     customer, and agrees not to bother the author if the program does not
     perform as expected, or performs other than expected, or does not perform
     at all.

\item Customer assumes full responsibility for any deaths or injuries that may
     result from the normal or abnormal operation of this program.  In the
     event of casualties exceeding 1000 persons or property damage in excess of
     \$10 million, customer agrees that he or she has stolen the program and we
     didn't even know he or she had it.

\item Customer agrees not to say bad things about the program or the author to
     anyone claiming to be from ``60 Minutes''.
\end{enumerate}

\subsection{Very Limited Warranty and Conditions of Sale}

\begin{enumerate}
\item For a period of 90 minutes, commencing from the time you first thought
     about getting this program, we warrant that this program may or may not be
     free of any manufacturing defects.  It will be replaced during the
     warranty period upon payment of an amount equal to the original purchase
     price plus \$10.00 for handling.  This warranty is void if the program has
     been examined or run by the user, or if the manual has been read.

\item This program is sold on an {\em as was} basis.  The author makes no warranty
     that it is, in fact, what we say it is in our propaganda, or that it will
     perform any useful function.  We have no obligation whatsoever other than
     to provide you with this fine disclaimer.

\item Some countries do not allow limitations as to how long an implied warranty
     lasts, so we refuse to imply anything.

\item There is an extremely small but nonzero chance that, through a process
     known as ``tunnelling'', this program may spontaneously disappear from its
     present location and reappear at any random place in the universe,
     including your neighbours computer system.  The author will not be
     responsible for any damages or inconvenience that may result.
\end{enumerate}

\subsection{Limitation of Liability}

\begin{enumerate}
\item We have no liability or responsibility to the customer, the customers
     agents, our creditors, your creditors, or anyone else. 
\end{enumerate}

%\end{document}
