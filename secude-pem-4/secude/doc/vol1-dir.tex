\section{Directory Services}
\markboth{Directory Services}{Directory Services}
\thispagestyle{myheadings}
\label{ds}
The use of public key cryptography makes it necessary that one knows 
and trusts other's public keys. Public Directories like X.500 Directories
may play a supporting role here. Users and certification authorities
(if they are involved) have the following requirements when they use security
services on the basis of public key cryptography:
\bi
\m Users have the requirement to make their certificates publicly
available and to have access to certificates of others and revocation lists
in order to find out whether a certificate in question is still valid or
has been revoked.
\m Certification authorities have the requirement to make their certificates and revocation
lists publicly available to a distributed community.
\m Certification authorities have the requirement to exchange cross certificates 
   with other certification authorities and make
   them publicly available.
\ei
Therefore the integration of public Directories (in particular X.500 Directories)
into the security infrastructure is a vital part of SecuDE.

A public Directory appears not only as information provider as part of the public security 
infrastructure, but also as user of security services in order to
protect its stored information from unauthorized access and to protect its 
communication. Security policies in case of the globally distributed X.500 Directory
must aim at three major goals:
\be
\m Protect the information base of the Directory.
\m Protect the internal and external communication of the Directory
   via DAP and DSP.
\m Protect the resources of the Directory.
\ee
These needs were the motivation for the development of the {\em Authentication
Framework} (X.509) as part of the X.500 Directory standard where {\em strong authentication} 
methods on the basis of digital signatures and certified key-to-name bindings
are being applied in order to protect the X.500 Directory. The certification procedures
and formats of X.509 are applicable in any context where public key cryptography is used. 
They play a central role in SecuDE. X.509 certificate
formats have been adopted, for instance, in the Internet PEM environment.

\subsection{System Description}
The X.500 Directory is an OSI application which has been developed
jointly by CCITT and ISO. The CCITT version is specified in the
X.500 series of Recommendations; the ISO version is known as
International Standard 9594.

The directory is a system holding information about various entities
(referred to as {\em objects}) and providing users with services for
accessing the information. Examples of entities likely to be
represented in the directory are countries, organizations,
organizational subdivisions, people, etc.
The information held on a particular person may include the
person's name, postal address, telephone number, and electronic mail
address; the information about a particular organization may
comprise that organization's name, address, telephone number, and
telex number.
The information held in the directory is used to facilitate communication
between entities.
The complete set of information to which the directory provides
access is known as the {\em Directory Information Base} (DIB).
\\ [1em]
The directory can be used to support the following services:
\\ [1em]
{\bf Look-up}

The directory supports human users in obtaining, among other things,
telephone and address information of organizations and other users.
\\ [1em]
{\bf Message Handling Support}

The directory can be used in message handling (based on CCITT's
X.400 standard) where it provides the mapping of user-friendly names
(supplied to the directory) onto machine oriented O/R addresses
(returned from the directory). The directory also provides the means
by which two MHS functional entities (two MTAs, or a UA and a MTA)
may establish the identity of each other.
\\ [1em]
{\bf Support of Authentication Process}

The directory meets the needs for authentication and other security
services (e.g. content confidentiality, content integrity) as it is the
place from where communicating peers can obtain
knowledge on each other. The directory provides two approaches to
authentication:
\bi
\m The {\em simple authentication} approach relies on the directory holding
a password for any user that wishes to be able to authenticate itself
to a service.
\m The {\em strong authentication} approach is based on public-key
cryptography; here the directory acts as a repository of {\em public keys}
within {\em certificates}.
\ei
A certificate can be revoked by being added to a
{\em revocation list}.
It is of vital importance that certificates and revocation lists be
easily available to a distributed community; this is achieved by placing
them in a X.500 Directory.
\\ [1ex]
The need for authentication may be uni- or bilateral: In the first case
a user may wish to authenticate itself to an application in order
to be permitted to carry out some action; in the latter case two
communicating entities (applications and/or users) may wish to establish
the identity of each other.
(The directory itself requires its users to
authenticate themselves before enabling them to access confidential
information or to modify some of the directory information.)


\subsubsection{The Directory Model}

The directory is comprised of a set of one or more application
processes known as {\em Directory System Agents} (DSAs). In case
of the directory being composed of more than one DSA, it is said to
be {\em distributed}. In a distributed directory each DSA holds
only a fragment of the total directory information.
\\ [1ex]
Each DSA provides zero, one, or more {\em access points} to the directory
at which users may interact with the directory. A user accesses the
directory by means of an application process known as a {\em Directory
User Agent} (DUA), running in the user's system.
The interaction between a DUA and a DSA is carried out by means of a
directory protocol called {\em Directory Access Protocol} (DAP).
\\ [1ex]
A basic feature of a distributed directory is that a user
should potentially be able to access any directory information
(subject to access control) irrespective of the access point at which
his request originates and without knowledge of the physical location
of the required information. It is required that the distribution
of the directory information among multiple DSAs be hidden from the
user, thereby giving the user the impression that each DSA holds
the whole of the DIB. This is achieved by each DSA (holding
a fragment of the DIB) having some knowledge about the location
of the rest of the DIB.
\\ [1ex]
As a consequence, a DUA need only have direct access to a single DSA to
which it can submit the user's request: If that DSA cannot
satisfy the request, it will be able to locate the DSA holding
the requested information and either return that knowledge to the
requestor or pass on the request to the appropriate DSA. (The former
mode of interaction is known as {\em referral}, the latter one
is known as {\em chaining}.) The interaction between two DSAs is
defined by the {\em Directory System Protocol} (DSP).


\subsubsection{The Information Model}

The information model describes the logical structure of the DIB; in
this context the physical distribution of the DIB across multiple DSAs
is not considered.

The DIB is composed of {\em entries}, each representsing one
(directory) object and containing the information on that object.
An entry is structured as a set of {\em attributes}, each with an
{\em attribute type} and one or more {\em attribute values}.

The DIB is hierarchically structured by arranging its entries
in a tree, known as the {\em Directory Information Tree} (DIT).
The DIT reflects the hierarchical relationship commonly found
among objects. For example, a person works for a department,
which belongs to an organization, which has its head office
in some country. As a consequence, objects representing countries
will be found close to the root of the DIT, whereas entries representing
people or application processes will be located in the leaves of the DIT.

The DIT provides a means of assigning each entry a {\em Distinguished
Name} (DN) which unambiguously identifies that entry; this means that
no two entries may have the same DN.
Naming is accomplished by assigning a {\em Relative Distinguished Name}
(RDN) to each entry. The RDN of an entry is made up of specially
nominated attribute values (the {\em distinguished values}) of the
entry.

The DN of an entry is then defined as being the sequence of RDNs of the
entry and of all its superiors in descending order from the root.

The second hierarchy which rules the DIB is that of the object classes.
An object class represents a family of objects which share certain
characteristics. Every object belongs to at least one object class.
An object class defines which attributes must be present ({\em mandatory}
attributes) and which attributes may be present ({\em optional}
attributes) in an entry of the object class.

Example:

The object class {\em person} defines the attributes {\em commonName}
and {\em surname} as being mandatory and the attributes {\em description},
{\em seeAlso}, {\em telephoneNumber}, and {\em userPassword} as being
optional in an entry of that object class.

An object class may be a {\em subclass} of another object class. In
this case the members of the subclass share all the characteristics
of the {\em superclass} and additional characteristics possessed
by none of the members of the superclass. As a consequence, an object
belonging to a particular object class does not only contain
the mandatory attributes of that object class, but also the
mandatory attributes of all the superclasses of that object class.
(The same applies to the optional attributes.)

\subsection{The Directory as part of the Security Infrastructure}

The Directory also has to serve as a provider of public security information as part 
of the security infrastructure. Therefore, a DSA shall support
\bi
\m storage of attribute type UserCertificate as defined in X.500-88,

\m object class strongAuthenticationUser as defined in X.500-88,

\m storage of attribute type CrosscertificatePair as defined in X.500-88,

\m storage of attribute type CertificateRevocationList as defined in RFC 1422 
  (PEM CRL format),

\m storage of attribute type CaCertificate as defined in X.500-88,

\m object class certificationAuthority which is similarly defined as in X.500-88
  with the difference that the RFC 1422 CertificateRevocationList
  attribute type must be present instead of the X.500-88 defined
  CertificateRevocationList and AuthorityRevocationList.
\ei

\subsubsection{Object Classes ``Strong Authentication User''
and ``Certification Authority''}
\label{ds-oc}

Due to X.500, X.520, X.521, X.509 \cite{cci3} \cite{cci4}:

A {\em certification authority} is represented in the directory
by an object of class ``organization + certi\-fica\-tion\-Au\-tho\-rity'' or
of the class ``organizational\-Unit + certi\-fica\-tion\-Au\-tho\-rity''.
The additive object class ``certification\-Au\-tho\-ri\-ty''
provides the additional attributes
``CACertificate'', ``Certi\-ficate\-Re\-vo\-cation\-List'',
``Autho\-rity\-Re\-vo\-cation\-List'' and, optional,
``Cross\-Certi\-ficate\-Pair'' (cf. X.521, Annex A).

A {\em user} of security oriented services is represented in the directory
by an object of class
``person + strongAuthenticationUser''.
The additive object class ``strong\-Authen\-tica\-tion\-User''
provides one additional attribute:
``UserCertificate''.
(cf. X.521, Annex A).

Within SecuDe a root certification authority
has an additional directory attribute, which is not defined by X.500.
It is called ``Old Certificates'' and is supposed to support users
in verifying old signatures
created by keys which are out of use.
It contains a list of cross certificates in which the root
certification authority crosscertifies its old public signature
keys by its currently valid signature key.

\subsubsection{Certificate Attributes}
\label{ds-ca}

The attribute ``UserCertificate'' ( and ``CACertificate'', respective)
(cf. X.509, Annex G)
is a 1-level certificate. Within SecuDe
it has multiple value:
Each user and each certification authority has {\em two}
unique hierarchy certificates, both issued by the certification
authority at the next level above,
the first one certifying
the public signature key, the other one certifying
the public encryption key.
By evaluating the object identifiers in the algorithm identifiers
of the subjectPublicKeyInfo the respective certificate and its
certified public key is recognized.

``CrossCertificatePair'' (cf. X.509, Annex G)
is an optional attribute of a certification authority.
Within SecuDe
only the ``Forward Certificate'',
which is the first element of the SEQUENCE in this attribute,
is filled.
It has {\em multiple value}
and contains all cross certificates by which other
certification authorities certify the owner's
public signature key.
The distinguishing part of the multiple cross certificate value
is the issuer's name.
The second element of the SEQUENCE,
``Reverse Certificate'', is missing.

\subsubsection{Black Lists}
\label{ds-bl}

A CA must be able to revoke a certificate which it is the issuer
of prior to its expiration time. There are many reasons for a CA
to revoke a certificate:
\bi
\m The user's secret key is assumed to be compromised whereby the
corresponding public component is invalidated.
\m The user's affiliation has changed whereby the distinguished
name contained in the certificate's "subject" field is invalidated.
\m The user is no longer to be certified by the CA.
\m The CA's certificate is assumed to be compromised.
\m The user has violated the CA's security policy.
\ei

A CA can mark a certificate which it issued as "invalid" by adding
it to the list of revoked certificates.

Information relative to certificate revocation is propagated by
means of revocation lists, so-called "black lists".
Revocation lists are made publicly available by being placed in
the directory. Each CA is responsible for maintaining its
revocation lists held within its directory entry.

Two users wishing to authenticate and having learned each others'
certificates from the certification path must check the validity
of the received certificates by determining if any of the
certificates establishing the certification path is contained in the
black list of its respective issuer.

A user who checks a certificate at security level 3 (see~\ref{sl}),
asks the respective ``revocation list'' from the directory entry of the
issuing certification authority.
The directory will return the complete list.
The user verifies the valid time interval as well as
the signature of the list and looks then for
the pair of attributes
{\em Issuer + Serial Number}. If it is in the list,
the certificate is invalid.
Otherwise the user resumes with the verification of the certificate.
Note, that the pair of attributes {\em Issuer + Serial Number}
uniquely and universally identifies a certificate.
\\ [1em]
There are two different formats and semantics for revocation lists:
one defined by the standard (X.509), the other one defined by
PEM (RFC 1422). In the following, the characteristics and deficiencies
of both definitions are explained.

{\bf Certificate Revocation according to X.509}

X.509 defines two revocation lists to be maintained by each CA:
\be
\m In its CRL ("CertificateRevocationList"), a CA includes all user
certificates it issued which have been revoked.
\m In its ARL ("AuthorityRevocationList"), the CA includes all
CA certificates it issued which have been revoked.
\ee
An ARL may
also contain references to CA certificates issued by CAs other
than the list maintainer; this facility enables two CAs to publish,
by bilateral agreement, each others' revocations, that is, each CA
signs a sequence of both its own revocations and those of its
partner CA. This might be helpful to the clients of both CAs as it
reduces the number of their directory accesses when consulting
black lists.

Both lists are certified by their maintaining CA.

ARLs and CRLs are stored differently in the directory: The former list
is held as attribute of type "CertificateRevocationList" within the
directory entry of the responsible CA, the latter as attribute of
type "AuthorityRevocationList". Both attributes are mandatory within
an entry belonging to the object class "Certification Authority"
(the object class "Certification Authority" is used in defining
directory entries for objects which act as certification authorities);
the consequence of this is that both lists must exist in the directory,
even if they are empty.

There is no syntactic difference between these two lists. Their
underlying syntax is defined as follows:

\begin {center}
\begin {tabular}{lll}
CertificateList & \multicolumn{2}{l} {::= {\bf SIGNED SEQUENCE} \{ } \\
  & signature     & AlgorithmIdentifier,        \\
  & issuer        & Name,                       \\
  & lastUpdate    & UTCTime,                    \\
  & revokedCertifiactes   & RevokedCertificates \\ 
 \} &               &                             \\
RevokedCertificates & \multicolumn{2}{l} {::= {\bf SIGNED SEQUENCE OF SEQUENCE} \{ }  \\
  & signature     & AlgorithmIdentifier,        \\
  & issuer        & Name,                       \\
  & userCertificate        & SerialNumber,   \\
  & revocationDate & UTCTime OPTIONAL \\
 \} & & \\
\end {tabular}
\end {center}
\label{fig-ds-blx509}
\stepcounter{Abb}
{\footnotesize Fig.\arabic{Abb}: X.509-type ``Revoked Certificate List''}
\\ [1em] 

The current definition of the CertificateList syntax is incorrect
for two reasons:
\be
\m The type SIGNED SEQUENCE OF SEQUENCE of the "revokedCertificates"
component is syntactically incorrect: As the latter SEQUENCE
comprises an algorithm identifier, SEQUENCE OF SEQUENCE identifies
several (different) algorithm identifiers to compute a single
signature. It is unclear which algorithm identifier will be used
by SIGNED.
\m The CertificateList syntax prescribes exactly two signatures
which are applied in a nested manner: the innermost signature is
applied to the sequence of revoked certificates, where each revoked
certificate is defined as a sequence of several information, among
them an issuer and a serial number which identify a certificate
universally. The outer signature applies to the revocation list as a
whole.
\ee
This makes no sense as it would enable any CA to revoke certificates
signed by any other CA: If a compromised CA wanted to revoke a CA
certificate which it is not the issuer of, it would simply have to
add the issuer and serial number of that certificate to the sequence
of entries in its ARL and then sign that sequence. A directory user
consulting the ARL of the compromised CA would then not be able to
determine whether the CA certificate in question has indeed been revoked
by its issuer; in order to be on the safe side the directory user would
have to access the issuer's ARL.
For this reson, each individual revoked certificate is to be signed by its
issuer before being added to a black list (at least in the case of an
ARL): It is a SIGNED SEQUENCE.

This deficiency of the CertificateList syntax as specified in the
standard is the subject of a Defect Report (Defect Report 9594/033)
where it is classified as an error.


There are two possible solutions for correcting the faulty type
SIGNED SEQUENCE OF SEQUENCE: change it to either SEQUENCE OF SIGNED
SEQUENCE or SET OF SIGNED SEQUENCE. If one considers the certificate
list an UNORDERED list of revoked certificates, SET OF SIGNED SEQUENCE
is appropriate.

However, the SEQUENCE OF SIGNED SEQUENCE solution should be given
preference for the following reason: The SIGNED macro defines a
signature to be applied to the BER code of ToBeSigned.
If the ToBeSigned contains a Set-of type, its BER code may be
varying, as a Set-of type does not specify an order in which
its components shall be encoded. A revocation list represents a
SIGNED quantity. After a revocation list whose revoked certificates
are arranged in a SET has been exchanged several times among
applications in a distributed environment, its BER code of
ToBeSigned will very probably be different from that which was
originally signed by the issuing CA. As a consequence, that
revocation list could not be verified, though it would not have
been manipulated.

In order to enable the validation of a SIGNED type (containing a
Set-of type as a component) in a distributed environment, a
distinguished encoding is required. According to X.509, a
distinguished encoding of a SIGNED data value is obtained by
applying the Basic Encoding Rules defined in X.209 with the restriction
that the components of a Set-of type (that is, the revoked certificates)
shall be encoded in ascending order of their octet value.

In order to save a CA from having to order the entries of
its revocation list according to their octet value, it would be
much more convenient for a CA to arrange its revoked certificates
in a SEQUENCE rather than in a SET, that is, to use the type
SEQUENCE OF SIGNED SEQUENCE for the "revokedCertificates" component
of a black list. The revoked certificates could then be ordered
according to their revocation date.
\\ [1em]
Redundancy in case of the X.509-CertificateList syntax

The mechanism for certificate revocation described in the current
version of X.509 (1988) is not correctly reflected by the
accompanying format for CertificateRevocationList: The format for
CRLs as specified in X.509 associates an issuer distinguished name
which each revoked certificate, even though the text states that a
CRL contains entries for only a single issuer. As the issuer of a CRL
always corresponds to the issuer of each individual revoked certificate
contained in that CRL, the name of the issuing CA need appear only
once in the list: serial numbers in the subsequent entries are to be
interpreted with respect to the issuer whose name appears in the list's
outer SEQUENCE.

The inclusion of an issuer name with each revoked certificate is
reasonable for authority revocation only where the replication of this
field provides a mechanism for a CA to announce in its ARL the revocation
of a CA certificate it has not issued.

Not only the provision of an issuer name with each revoked certificate
in a CRL, but also the appearance of an algorithm identifier in each
CRL entry is superfluous: If the revoked certificates contained in a CRL
are meant to be SIGNED quantities, the algorithm identifier associated
with their signature will always correspond to the algorithm identifier
which appears in the list's outer SEQUENCE.
Moreover, there is no need for the revoked certificates in a CRL being
signed; the signature applied to the CRL's outer SEQUENCE is sufficient,
whereas the signatures applied to each revoked certificate just make the
size of a CRL grow immensely, with no further benefit derived from them.

The inconsistency between the description of CRL management in the
text and the accompanying format for CRLs as specified in X.509 is
the subject of a Defect Report (Defect Report 9594/057). There it is
suggested to define two syntaxes for supporting two mechanisms
(the CertificateRevocationList mechanism and the AuthorityRevocationList
mechanism) with different requirements: a "UserCertificateList" syntax
for CRLs and an "AuthorityCertificateList" syntax for ARLs.
The "UserCertificateList" syntax removes the superfluous issuer name
and algorithm identifier from CRL entries.


{\bf Certificate Revocation according to PEM}

In contrast to X.509, PEM does not distinguish between a
CertificateRevocationList mechanism and an AuthorityRevocationList
mechanism: In order to simplify the management of revocation lists,
it uses CRLs for revocation both of user and of CA certificates.

The format of a revocation list for use in the PEM environment is a
deviation from the format specified in X.509 (1988), though it is
derived from it.

PEM defines its CertificateRevocationList syntax as follows (see RFC
1422):

\begin {center}
\begin {tabular}{lll}
CertificateRevocationList & \multicolumn{2}{l} {::= {\bf SIGNED SEQUENCE} \{ } \\
  & signature     & AlgorithmIdentifier,        \\
  & issuer        & Name,                       \\
  & lastUpdate    & UTCTime,                    \\
  & nextUpdate    & UTCTime                     \\
  & revokedCertifiactes          & {\bf SEQUENCE OF} CRLEntry OPTIONAL, \\
\}  &               &                             \\
CRLEntry & \multicolumn{2}{l} {::= {\bf SEQUENCE} \{ }  \\
  & userCertificate        & SerialNumber,   \\
  & revocationDate & UTCTime \\
 \} & & \\
\end {tabular}
\end {center}
\label{fig-ds-blpem}
\stepcounter{Abb}
{\footnotesize Fig.\arabic{Abb}: RFC 1422-type ``Revoked Certificate List''}
\\ [1em]
 
As reflected by this syntax, PEM does not see a need to have a
revocation list issued by other than the issuer of the certificates
contained in that revocation list. Therefore, the PEM format does not
provide a facility to include in a revocation list certificates from
other issuers: The "revokedCertificates" component of a PEM
revocation list is a SEQUENCE of ordered pairs, in which the first
element is the serial number of the revoked certificate and the
second element is the time and date of the revocation for that
certificate. The serial number is to interpreted with respect to
the issuer whose name appears only once in the list's outer SEQUENCE.

There is only one signature applied to a PEM revocation list; the
signature's associated algorithm identifier is specified in the
"signature" field in the list's outer SEQUENCE.

As an enhancement to the CertificateList format defined in the
standard, the CRL format adopted by PEM provides a "next update"
field containing time and date values which specify the next scheduled
date of issue of a CRL. By means of the "next update" field a CA
informs the directory users of when next to access its
"CertificateRevocationList" directory attribute in order to
maintain the most current version of its revocation list.

It is a CA's responsibility to issue a new CRL when the next
scheduled issue date arrives, even if there are no changes in the list
of entries. (In that case, only the values of the "last update" and
"next update" fields are to be changed.)

The omission of the "next update" field (as in the case of X.509)
presents a problem as it does not allow a directory user to determine
whether a given CRL is the most current list available from its issuing
CA or whether it is already "out of date". As a consequence, a directory
user must access this attribute constantly in order to have any confidence
in its currency.

\subsubsection{``Old Certificates'' of Root Certification Authority}
\label{ds-roc}

{\bf Definition of Attribute ``OldCertificates''}
\label{ds-doc}

A recipient of a digital signature is able to verify the originator
certificate only if he knows a
public key of one, at least of the highest,
of the issuing certification authorities
(i.e. he needs a table of trusted public keys on his smartcard).
Changing keys might make old signatures unverifyable.

In order to allow users to verify old signatures
the root certification authority
crosscertifies all its old public keys by its currently valid key.
These cross-certificates are made public as directory attribute
called ``old certificates'', which is not defined in X.500.

\begin {center}
\begin {tabular}{ll}
OldCertificates    & ::= {\small\bf ATTRIBUTE WITH ATTRIBUTE-SYNTAX} \\
		   & OldCertificateList                              \\
		   &                                                 \\
OldCertificateList & ::=  {\small\bf SEQUENCE OF SEQUENCE} \{        \\
		   & serialNumber     CertificateSerialNumber,       \\
		   & crossCertificate Certificate \}
\end {tabular}
\end {center}

\label{fig-ds-doc}
\stepcounter{Abb}
{\footnotesize Fig.\arabic{Abb}: Directory attribute ``OldCertificates''}

{\bf Values of the Table}
\label{ds-vt}

This is an example of a root certification authority
with an currently valid signature key number 5.
Up to date there were only five old signature keys with the numbers 0-4.

\begin {center}
\begin {tabular}{|c|c|}
\hline
{\em Serial Number} & {\em Cross Certificate} \\
\hline
5000000 & $CA_{S4}$\ka$CA_{P5}$\kz  \\ \hline
4000000 & $CA_{S5}$\ka$CA_{P4}$\kz  \\ \hline
3000000 & $CA_{S5}$\ka$CA_{P3}$\kz  \\ \hline
2000000 & $CA_{S5}$\ka$CA_{P2}$\kz  \\ \hline
1000000 & $CA_{S5}$\ka$CA_{P1}$\kz  \\ \hline
0000000 & $CA_{S5}$\ka$CA_{P0}$\kz  \\ \hline
\end {tabular}
\end {center}
 
\label{fig-ds-vt}
\stepcounter{Abb}
{\footnotesize Fig.\arabic{Abb}: Directory table ``Old Certificates''}
\\
{\footnotesize $CA_{S5}$\ka$CA_{P3}$\kz denotes a cross certificate
by which the root CA certifies its old public key No. 3 by its
currently valid
signature key No.5, etc.}

{\bf The Query Algorithm of the Table}
\label{ds-qa}

Key entry is the left column ``Serial Number''.
A recipient of an originator certificate recognizes
that the highest certificate is old by the key number of the certificate
serial number compared to the key number of the public root-CA-key
on his smartcard.
He would read top-down through the list until
the originator root certificate serial number is greater than or equal with
the table's serial number. He would append the cross certificate of
that respective line to the originator certificate
and is now able to verify the extended originator certificate
completely.

Note the first line of the table: It enables users to update
their smartcards after a change of certification keys of the root-CA.

This algorithm would lead to no result,
if the serial number of the first line in the directory table
is older (i.e. smaller) than the serial number of {\em PKROOT-new}
on the recipient's smartcard (see paragraph \ref{pkroot}).
This indicates an error either on the smartcard
or in the directory table.
Either case can only be resolved by the certification authority.
Also, if a user has stored a very old root key on his smartcard
(serial no of PKRoot-new + 1 {\em smaller than} serial no
of first line of directory table),
this algorithm would have no result.
In this case, the user should ask for an actual value
of the root keys, possibly from his user certification authority.

{\bf Updating the Table}
\label{ds-ut}

Whenever a root certification authority changes its
signature keys, its first step is to update the directory attribute
``Old Certificates''.
In every line the cross certificate in the right column is replaced
by a new cross certificate which certifies the same old key,
however by its new key.
The table gets an additional first line.

