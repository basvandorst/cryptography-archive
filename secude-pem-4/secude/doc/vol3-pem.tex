\section{Privacy Enhanced Mail (PEM)}
\markboth{Privacy Enhanced Mail}{Privacy Enhanced Mail}
\thispagestyle{myheadings}
\label{isw-pem}

\subsection{Secure X.400 MHS}
\label{isw-mhs}

There are three places, at which security services for X.400 message handling
can be applied.
One place is to secure authentic associations between message handling
system agents such as message transfer agents, user agents, and message
stores.
This is defined in X.411-1988 \cite{cci2}
and will be described in paragraph \ref{isw-x400as} below.
The second place is to apply encryption and signature functions
on the message content and place information about the resulting bit strings
into the message envelope.
This is also defined in X.411-1988
and will be described in paragraph \ref{isw-x400env} below.

%\begin{figure}
\begin{center}
\makebox[4.305in][l]{
  \vbox to 5.291in{
    \vfill
    \special{psfile=vol1-fig18.ps}
  }
  \vspace{-\baselineskip}
}
\end{center}
\label{fig-isw-mhs-1}
\stepcounter{Abb}
{\footnotesize Fig.\arabic{Abb}:
Authentication of associations between MHS agents}
%\end{figure}
\\ [1em]
A third logical place would be to define security oriented
body parts. There is, however, no such definition in X.420-1988.
On the other hand, by the definition of {\em externally defined
body parts} in X.420-1988,
other communication applications such as {\em EDIFACT}
or {\em ODA/ODIF} are free to define document oriented security features
and include them into an X.400 message.
This is not considered within SecuDe.
Instead, there is a definition of an MHS-protocol transparent
security function. There are
encryption and signature functions
to be applied on IA5 texts, and a printable encoding
of the resulting signature and encryption bitstrings in order to include
them into an IA5 text.
This has been done by several projects in different ways.
SecuDe follows the
``Privacy Enhanced Mail'' project of the US-Internet,
which published its definitions
in a series of Internet RFCs 1421 - 1423 \cite{rfc1}.
This is described in the next paragraph \ref{isw-pem} below.
%\begin{figure}
\begin{center}
\makebox[4.125in][l]{
  \vbox to 7.383in{
    \vfill
    \special{psfile=vol1-fig19.ps}
  }
  \vspace{-\baselineskip}
}
\end{center}
\label{fig-isw-mhs-2}
\stepcounter{Abb}
{\footnotesize Fig.\arabic{Abb}:
X.400 P1 oriented security}
%\end{figure}

\newpage
%\begin{figure}
\begin{center}
\makebox[3.694in][l]{
  \vbox to 3.513in{
    \vfill
    \special{psfile=vol1-fig20.ps}
  }
  \vspace{-\baselineskip}
}
\end{center}
\label{fig-isw-mhs-3}
\stepcounter{Abb}
{\footnotesize Fig.\arabic{Abb}:
MHS protocol transparent security}
%\end{figure}

\subsubsection{Secure X.400 Messages (envelope oriented)}
\label{isw-x400env}

Currently, SecuDe considers two security services:
{\em message confidentiality} and {\em message integrity}.
These are described in more details below.

Security is a new feature of X.400-1988 \cite{cci2},
security issues are not considered in the older MHS recommendations
X.400-1984 \cite{cci1}.
In X.400-1988,
all security {\em services} for message handling
are designed to encounter certain security {\em threats}
(X.402 \cite{cci2}, 10., 10.2 and table D-1).
They are realized by security {\em elements}, which are
combinations of protocol data units
(X.402, 10.3 and table E-1).
Message oriented security elements reside
inside of the message envelope.
(Note that association oriented security elements
reside as parameters of the bind operation in the association protocol,
as is outlined in paragraph \ref{isw-x400as} below.)
In order to be compatible with the X.400-1984 recommendations
the security elements are embedded into an extension
mechanism which is used for other new services as well,
like the support of distribution lists.
See also figure 18 on page \pageref{fig-isw-mhs-2} above.

{\bf Message Envelope Extensions}
\label{isw-mee}

X.411 \cite{cci2}
defines the following envelope extension elements,
which are designed for security purposes:

\begin{itemize}
\item originator certificate (per message),
\item message token (per recipient),
\item content confidentiality algorithm identifier (per message),
\item content integrity check (per recipient),
\item message origin authentication check (per message),
\item message security label (per message),
\item proof of submission request (per message),
\item proof of delivery request (per message).
\end{itemize}

An X.400-1988 message envelope distinguishes between
per-message and per-recipi\-ent envelope elements.
From the extension elements above, message token and
con\-tent integrity check are per-recipient,
the others are per-message envelope elements.
The terms above are explained in X.411,  8.2.1.1.1.25 -- 32
(8.2 Submission Port, but the same parameters are used
at the Delivery Port (8.3) and the Transfer Port (12.2)).
Additionally,
certificate, message token, security label, and algorithm identifier
are explained in more details in X.411, 8.5.7 -- 8.5.10
(8.5: Common Parameter Types).
The protocol data units of the envelope extension elements are defined in a
section of X.411, 9: MTS Abstract Syntax Definition, figure 2,
Extension Fields.
Every extension element is a sequence of three elements: an integer which
indicates the type of the extension element, a bitstring which indicates
the criticality (for submission, for transfer, for delivery)
of the extension element,
and the value which is the specific extension element itself:
{\small
\begin {center}
\begin {tabular}{llll}
\multicolumn{4}{l} {ExtensionField ::= SEQUENCE \{ } \\
& & & \\
type           & [0] & INTEGER,        & -- -- ExtensionType \\
criticality    & [1] & BitString  \{   & for-submission (0) , \\
	       &     &                 & for-transfer   (1) , \\
	       &     &                 & for-delivery   (2) \} , \\
value          & [2] &  ANY defined by & type \\
 \} & & &
\end {tabular}
\end {center}
}
\label{fig-isw-mee-1}
\stepcounter{Abb}
{\footnotesize Fig.\arabic{Abb}:
X.400 Message envelope extension element format
\\[1ex]
The type-integers of the security relevant extension
elements are given in brackets below:
originator certificate (15),
message token (16),
content confidentiality algorithm identifier (17),
content integrity check (18),
message origin authentication check (19),
message security label (20),
proof of submission request (21),
proof of delivery request (22).
}
\\[1em]
The envelope extension element {\em originator certificate}
is used to convey the originator's public key in a trustworthy way.
The originator certificate consists of the
originator's certificate and the certification path between
the originator's certifiation authority and its Root CA,
i.e. its value is of type ``Certificates''.
The originator certificate as an envelope extension element is used
for validation of signatures inside of security oriented envelope
extension elements (see below).
Its protocol data unit is defined in X.411,
section 9 (MTS Abstract Syntax Definition),
Extension Fields:

{\small
\begin {center}
\begin {tabular}{lll}
\multicolumn{3}{l} {originator-certificate :== SEQUENCE \{ } \\
	    &     & \\
type        & [0] INTEGER "15", & \\
criticality & [1] BitString "2", & -- -- for delivery \\
value       & [2] Certificates & \\
 \}         &     &
\end {tabular}
\end {center}
}
\label{fig-isw-mee-2}
\stepcounter{Abb}
{\footnotesize Fig.\arabic{Abb}:
X.400 originator certificate}
\\[1em]
The protocol data unit of ``Certificates'' (note the plural) is a sequence
(in the context above, an implicit sequence), which comprises at least the
one "Certificate" (note the singular) of the originator, and optionally
more ``Certificate''s as a representation of the certification path of
the certification authorities concerned. Certificates of the
certification path can include sets of cross-referencing certificates.
This concept of certificates
is described in X.509, section 7 (Obtaining a User's Public Key) \cite{cci4}.
The protocol data units of both, Certificates and Certificate is defined in
X.509, Annex G.
\\[1em]
The envelope extension element {\em message security label}
is used for both, message integrity and confidentiality services.
Message security labels are explained at several places of X.411,
the most detailed of which is at 8.5.9 (8: MTS Abstract Service
Definition, 8.5: Common Parameter Types). Other places are
8.2.1.1.1.30, 8.4.1.1.7. Its data unit format is defined in
X.411, section 9 (MTS Abstract Syntax Definition), figure 2:

\begin {center}
\begin {tabular}{lll}
\multicolumn{3}{l} {message-security-label :== SEQUENCE \{ } \\
	    &     & \\
type        & [0] INTEGER "20", & \\
criticality & [1] BitString "2", & -- -- for delivery \\
value       & [2] SecurityLabel & \\
 \}         &     & \\
	    &     & \\
\multicolumn{3}{l} {SecurityLabel ::= SET \{ }  \\
	    &     & \\
security-policy-identifier & SecurityPolicyIdentifier & OPTIONAL, \\
security-classification    & SecurityClassification   & OPTIONAL, \\
privacy-mark               & PrivacyMark              & OPTIONAL, \\
security-categories        & SecurityCategories       & OPTIONAL  \\
 \}         &     &
\end {tabular}
\end {center}

\label{fig-isw-mee-3}
\stepcounter{Abb}
{\footnotesize Fig.\arabic{Abb}:
X.400 Message security label}
\\[1em]
Security labels are assigned to an object, e.g. to a message (other examples:
to an MTA, to a UA, to an MTA-MTA association, etc.), in line with a
security policy in force. The security {\em policy} identifier
is an object identifier.
The security {\em classification} is an enumerate type which expresses
one out of 6 hierarchically ordered levels
(unmarked, unclassified, restricted, confidential, secret, top secret).
The {\em privacy mark} is a printable string, which says
something like ``confidential'' or ``authentic''
or anything which is defined by a security policy in force.
One possible list of privacy marks shall be assigned to any
security classification.
Finally, the security {\em categories} give further
restrictions within the context of security classification
and/or privacy mark.

To give an example for a very simple security policy,
let the following type of labels be assigned to messages.
Two classifications {\em unmarked} and {\em restricted} are defined.
{\em Unmarked} induces the absence of any other label subfield,
while {\em restricted} leaves a choice for the privacy mark
as either "confidential" or "authentic".
Security categories are not specified.
A security label could have one of three values:
\begin {center}
\begin {tabular}{lccc}
\multicolumn{4}{l} {{\em Example SecurityLabel} ::= SET \{ }  \\
 & & & \\
security-policy-identifier:
 & \multicolumn{3}{l} {object id  \pf ``simple example policy''}  \\
security-classification:    & unmarked   & restricted & restricted \\
privacy-mark                & missing    & ``confidential'' & ``authentic'' \\
security-categories         & missing    & missing    & missing  \\
 \} & & &
\end {tabular}
\end {center}

\label{fig-isw-mee-4}
\stepcounter{Abb}
{\footnotesize Fig.\arabic{Abb}:
X.400 Example for a message security label}
\\[1em]
Another envelope extension element, which is used both
for message integrity and confidentiality services is
the {\em message token}.
It has a rather complicated format.
Its basic idea, however, is very simple.
It is designed to carry two portions of information,
the first portion signed for its integrity,
the second portion encrypted for its confidentiality.
It also contains a recipient name, because it is intended
to be on a per-recipient base.
This can be exploited by the confidentiality service,
in that it contains a content confidentiality key subfield:
the same symmetric (i.e. DES) key used to
encrypt the message content
is transferred to every recipient in a different message token
within the same message.

\begin {center}
\begin {tabular}{lll}
\multicolumn{3}{l} {message-token :== SEQUENCE \{ } \\
	    &     & \\
type        & [0] INTEGER "16", & \\
criticality & -- -- missing,    & \\
value       & [2] MessageToken  & \\
 \}         &     & \\
	    &     & \\
\multicolumn{3}{l} {MessageToken ::= Token ::= SEQUENCE \{ }  \\
	    &     & \\
token-type-identifier & [0] OBJECT IDENTIFIER, & -- -- ``asymmetric'' \\
token                 & [1] AsymmetricToken & \\
 \}         &     & \\
	    &     & \\
\multicolumn{3}{l} {AsymmetricToken ::= SIGNED SEQUENCE \{ }  \\
	    &     & \\
signature-algorithm-id. & AlgorithmIdentifier, & \\
recipient-name & \multicolumn{2}{l} {ORAddressAndOrDirectoryName,} \\
time        & UTCTime,      & \\
signed-data & [0] TokenData & OPTIONAL, \\
encryption-algorithm-id. & [1] AlgorithmIdentifier & OPTIONAL, \\
encrypted-data & \multicolumn{2}{l} {[2] ENCRYPTED TokenData OPTIONAL} \\
 \}         &     &
\end {tabular}
\end {center}

\begin {center}
\begin {tabular}{lll}
\multicolumn{3}{l} {TokenData ::= SEQUENCE \{ }  \\
	    &     & \\
type  & [0] INTEGER, & -- -- 2 for ``signed'' \\
      &              & -- -- 3 for ``encrypted'' \\
value & \multicolumn{2}{l} {[1] MessageTokenSigned(or Encrypted)Data} \\
 \}         &     & \\
	    &     & \\
\multicolumn{3}{l} {MessageTokenSignedData ::= SEQUENCE \{ }  \\
	    &     & \\
\multicolumn{3}{l} {content-confidentiality-algorithm-identifier} \\
	    & \multicolumn{2}{l} {[0] Algorithm Identifier OPTIONAL,} \\
content-integrity-check     &  \multicolumn{2}{l} {[1] ContentIntegrityCheck OPTIONAL} \\
message-security-label      &  \multicolumn{2}{l} {[2] SecurityLabel OPTIONAL} \\
proof-of-delivery-request   & \multicolumn{2}{l} {[3] ProofOfDeliveryRequest OPTIONAL} \\
message-sequence-number     &  \multicolumn{2}{l} {[4] INTEGER OPTIONAL} \\
\multicolumn{3}{l} { \} -- -- signed by the originator of the message } \\
	    &     & \\
\multicolumn{3}{l} {MessageTokenEncryptedData ::= SEQUENCE \{ }  \\
	    &     & \\
content-confidentiality-key & \multicolumn{2}{l} {[0] BIT STRING OPTIONAL} \\
content-integrity-check     & \multicolumn{2}{l} {[1] ContentIntegrityCheck OPTIONAL} \\
message-security-label      & \multicolumn{2}{l} {[2] SecurityLabel OPTIONAL} \\
content-integrity-key       & \multicolumn{2}{l} {[3] BIT STRING OPTIONAL} \\
message-sequence-number     & \multicolumn{2}{l} {[4] INTEGER OPTIONAL} \\
\multicolumn{3}{l} { \} -- -- encrypted for the recipient of the message } \\
\end {tabular}
\end {center}

\label{fig-isw-mee-5}
\stepcounter{Abb}
{\footnotesize Fig.\arabic{Abb}:
X.400 Message token
\\[1ex]
Mainly consisting of the two portions ``signed data'' and ``encrypted data''.
The ``recipient name'' and the encrypted ``content confidentiality key''
(or ``content integrity key'', resp.)
allow for encryption (or integrity, resp.) on a per-recipient base.
}
\\[1em]
The envelope extension element
{\em content confidentiality algorithm identifier}
is related to an encryption algorithm
by which the content of a confidential message is encrypted.

\begin {center}
\begin {tabular}{lll}
\multicolumn{3}{l} {content-confidentiality-algorithm-identifier::= SEQUENCE \{ } \\
	    &     & \\
type        & [0] INTEGER "17",       & \\
criticality & -- -- missing,          & \\
value       & [2] AlgorithmIdentifier & \\
 \}         &     &
\end {tabular}
\end {center}
\label{fig-isw-mee-6}
\stepcounter{Abb}
{\footnotesize Fig.\arabic{Abb}:
X.400 Content confidentiality algorithm identifier}
\\[1em]
The envelope extension element
{\em message origin authentication check}
is used for message integrity services.
It contains a signature of the algorithm identifier,
by which the signature is performed,
of the message content and,
optionally, of a content identifier and a security label.

\begin {center}
\begin {tabular}{lll}
\multicolumn{3}{l} {message-origin-authentication-check ::= SEQUENCE \{ } \\
	    &     & \\
type        & [0] INTEGER "19", & \\
criticality & [1] BitString "2", & -- -- for delivery \\
value       & \multicolumn{2}{l} {[2] MessageOriginAuthenticationCheck} \\
 \}         &     & \\
	    &     & \\
\multicolumn{3}{l} {MessageOriginAuthenticationCheck ::= SIGNATURE SEQUENCE \{ } \\
	    &     & \\
algorithm-identifier   & AlgorithmIdentifier, & \\
content                & Content, & \\
content-identifier     & PrintableString      & OPTIONAL, \\
message-security-label & SecurityLabel        & OPTIONAL \\
 \multicolumn{3}{l} { \} -- -- signed by the originator of the message}
\end {tabular}
\end {center}
\label{fig-isw-mee-7}
\stepcounter{Abb}
{\footnotesize Fig.\arabic{Abb}:
X.400 Message origin authentication check
\\[1ex]
In the ContentIntegrityCheck, the optional fields above are missing.
The use of the security label subfield is strongly recommended
in order to link it securely with the message content.
}

{\bf Message Confidentiality Service}
\label{isw-mcs}

X.402, 10.2 \cite{cci2}
defines one service which ensures the confidentiality of a message:
{\em Content Confidentiality}.
X.402, table 14, assigns
the envelope extension elements
{\em content confidentiality algorithm identifier} and
{\em message token} to this service.
A message token contains also a content confidentiality algorithm identifier
as one of its signed subfields.
For the message confidentiality service,
it is recommended to use the following envelope extension elements:

\begin{itemize}
\item originator certificate,
\item per-recipient message token
\begin{itemize}
\item mandatory subfields including:
\item recipient name;
\item signed data including:
\item content confidentiality algorithm identifier,
\item message security label;
\item encryped data including:
\item content confidentiality key.
\end{itemize}
\end{itemize}

When originating a message,
a confidentiality service of this shape would
generate one DES-key and
perform one encryption and one signature action.
When accepting a message,
one decryption and one verification action would be performed;
however, more verifications could be required by the originator certificate.

One message token per recipient should be included into the
extended message envelope.
Each message token should contain its individual recipient name.
The encrypted data of all message tokens of one message
should contain the same
symmetric (DES) content confidentiality key,
however encrypted individually with the recipient's
asymmetric public encryption key.
The other subfields of the encrypted data may be missing.

The signed data of all message tokens of one message should be equal,
signed by the originator's secret signature key.
They should contain the content confidentiality algorithm identifier
(for example ``DES in ECB mode'' without initialization vector),
and also a security label.
The other subfields of the signed data may be missing.

The originator certificate with respect to the message originator's
public verification key is also put as an
extension element into the message envelope.
This is, because the message token contains signed data.

In addition to the originator certificate and the message token,
it is not necessary to have any other security oriented
extension envelope elements. In particular
it is sufficient to have the content confidentiality algorithm identifier
and the security label contained {\em inside} the message token.

An application programming interface for X.400 confidentiality services
should specify C-structures
for an originator certificate,
for a message token,
for a security label,
for an algorithm identifier,
and for a confidentiality key.
It should also provide functions, which fill these C-structures.
The C-structures contain data elements for signed and encrypted data and keys,
and the functions are able to handle these data elements,
i.e. to sign, verify, encrypt and decrypt the subfields concerned.

{\bf Message Integrity Service}
\label{isw-mis}

X.402, 10.2 \cite{cci2}
defines three services which prove the integrity of a message:
{\em Message Origin Authentication},
{\em Content Integrity}, and
{\em Message Sequence Integrity}.
X.402, table 14, assigns
the envelope extension elements
{\em message origin authentication check} and
{\em message token} to the service
{\em Message Origin Authentication} and
the envelope extension elements
{\em content integrity check},
{\em message origin authentication check}, and
{\em message token} to the service
{\em Content Integrity}.
The service {\em Message Sequence Integrity} is realized by
a message sequence number, which is signed inside
the envelope extension element
{\em message token}.
The per-recipient base
of the content integrity check will not be exploited in
SecuDe, because asymmetric encryption
is used for the signature mechanism.
In this case, the message origin authentication is the more
general service, because its envelope extension element
``message origin authentication check'' contains
a signature applied to all subfields
of the ``content integrity check'',
and additionally to a content identifier and to a security label.
For the message integrity service,
it is recommended to use the following envelope extension elements:

\begin{itemize}
\item originator certificate,
\item message security label,
\item message origin authentication check
\begin{itemize}
\item signature including:
\item content,
\item message security label;
\end{itemize}
\item optionally, per-recipient message token
\begin{itemize}
\item signed data including:
\item message sequence number.
\end{itemize}
\end{itemize}

When originating a message,
an integrity service of this shape would perform
no encryption, but two signature actions.
When accepting a message,
no decryption but two verification actions would be performed;
however, more verifications could be required by the originator certificate.

The ``security label'' is an independent envelope extension element,
and a signed subfield of the message token, too.
However, if it is contained in the signature
of the message origin authentication check
(what is recommended), then
it is not necessary to convey it in the message token signed data,
it should better be one extra envelope extension element
per message (not per recipient like the message token).

The ``message token'', which is another envelope extension element
contains in its signed data
the subfield ``message sequence number''.
If the integrity of a {\em sequence of messages} is to be proved,
then the message token
shall be conveyed as an extra envelope extension element, as well.
In this case, its subfields ``encryption algorithm identifier''
and ``encrypted data'' shall be missing, while
its subfield ``signed data'' contains the ``message sequence number'' only.
Note however, that the message token is always on a per-recipient base,
such that there must be as many message tokens as there are
message recipients, although they all contain the same signature.

The originator certificate with respect to the message originator's
public verification key is also put as an
extension element into the message envelope.
This is, because the
message origin authentication check
(and also the message token, if it is included)
contains signed data.

In addition to the originator certificate,
the message security label,
and the message origin authentication check
(and perhaps the message token for message sequence integrity),
it is not necessary to have any other security oriented
extension envelope elements.

An application programming interface for X.400 integrity services
should specify C-structures for an originator certificate,
a security label,
a message origin authentication check and a message token.
It should also provide functions, which fill these C-structures.
The C-structures contain data elements for signed and encrypted data and keys,
and the functions are able to handle these data elements,
i.e. to sign, verify, encrypt and decrypt the subfields concerned.

{\bf Combined Message Confidentiality and Integrity Service}
\label{isw-mcis}

For the combined message confidentiality and integrity service,
the following envelope extension elements
are recommended:

\begin{itemize}
\item originator certificate,
\item message security label (identical with message token labels),
\item per-recipient message token
\begin{itemize}
\item mandatory subfields including:
\item recipient name;
\item signed data including:
\item content confidentiality algorithm identifier,
\item message security label,
\item optionally, message sequence number;
\item encryped data including:
\item content confidentiality key;
\end{itemize}
\item message origin authentication check
\begin{itemize}
\item signature including:
\item content,
\item message security label.
\end{itemize}
\end{itemize}

When originating a message,
a combined confidentiality and integrity service of this shape would
generate one DES-key and
perform one encryption and two signature actions.
When accepting a message,
one decryption and two verification actions would be performed;
however, more verifications could be required by the originator certificate.

The message content is encrypted, but the signature of the
message origin authentication check is applied to the
clear text of the content,
i.e. the integrity service is performed first, the
confidentiality service thereafter.
When accepting a message, the services are performed
in reverse order: first the message is decrypted,
then its integrity is verified.
This is, because the signature of an encrypted text does
not prove a message origin, because the originator might
claim that he has no knowledge of the message content:
everybody could have been able to encrypt the content.
This way, however,
the integrity of a message cannot be verified
without the recipient's private decryption key.
Therefore, no intermediate message transfer agent,
nor the message store are able to produce a proof of delivery.
They can produce unproved delivery reports only.
This seems to be a serious deficiency of the X.400-1988 security model.

\subsubsection{Secure X.400 Associations}
\label{isw-x400as}

{\bf Security Oriented Bind Paramteres}
\label{isw-sbp}

Security is a new feature of X.400-1988 \cite{cci2},
security issues are not considered in the older MHS recommendations
X.400-1984 \cite{cci1}.
In X.400-1988,
all security {\em services} for message handling
are designed to encounter certain security {\em threats}
(X.402 \cite{cci2}, 10., 10.2 and table 13).
They are realized by security {\em elements}, which are
combinations of protocol data units
(X.402, 10.3 and table 14).
As was outlined in the previous paragraph \ref{isw-x400env},
message oriented security elements reside
inside of the message envelope.

Message handling systems comprise interacting agents such as
message transfer agents (MTA), user agents (UA) and message stores (MS).
Interactive operations between these objects can only be invoked
in the context of an established association.
The MTS-bind, and MTA-bind, respectively,
establish an association with the message transfer system (MTS),
and among MTAs, respectively.
Associations can perform security services
like the proof of identity (authenticity) of association partners,
or a confidential or authentic message exchange between association partners.
This is outlined in detail in X.411, paragraphs 8.1.1.1 (MTS-bind)
and 12.1.1.1 (MTA-bind).
See also figure 17, page \pageref{fig-isw-mhs-1} above.

Association oriented {\em security elements} reside as
{\em arguments}, {\em results} and {\em errors}
of the MTS- and MTA-bind operation in the association protocol (ACSE).
ACSE was only developed by ISO in the period since 1984.
Generally, ACSE is part of an X.400-1988 implementations.
However,
in order to be compatible with the X.400-1984 recommendations
X.400-1988 can use the Reliable Transfer Service (RTS)
in a so called ``X.410-1984 mode'', in which the ACSE
is ``bypassed''. In this case, the new security elements
for the associations between message handling agents
such as transfer agents, user agents, and message stores,
can not be activated.

As to associations between message handling agents
such as transfer agents, user agents, and message stores,
the X.400-1988 recommendations offer parameters for
the Association Control (ACSE) bind operation,
that allow for secure exchange of security labels,
tokens of signed and encrypted data,
and certificates.
The initiator of an association
can offer a set of security labels,
which is called the {\em security context}.
The responder of an association
can either accept the security context and send a result set,
or indicate an {\em unacceptable security context error}
and refuse the association.
This way, partner agents agree on the security
elements to be exchanged thereafter.
For example,
they might agree to exchange confidential and/or authentic messages only,
to encrypt all operation arguments between them,
or to perform traffic padding between messages,
or anything else which
is defined by a security policy in force.
This way, associations can automatically perform
security services on behalf of the users of a message handling system.
\\[1em]
\begin {center}
\begin {tabular}{llll}
\multicolumn{4}{l} {SecurityContext ::= SET OF SecurityLabel} \\
 & & &
\end {tabular}
\end {center}
\label{fig-isw-sbp-1}
\stepcounter{Abb}
{\footnotesize Fig.\arabic{Abb}:
X.400 Security context of MTS- and MTA-bind
\\[1ex]
The security label is of exactly the same format as the message
security label discussed in paragraph \ref{isw-mee} above.
}
\\[1em]
The initiator of an association
can include an {\em initiator credentials} in his bind argument set.
The responder of an association
would include a {\em responder credentials} in his bind result set,
or indicate an {\em authentication error} and refuse the association.
Simple credentials are passwords as they are normally used
by X.400-1984 associations. They are defined for X.400-1988
associations as well.
Within SecuDe, however, only {\em strong authentication}
is considered,
because passwords are not protected against unauthorized disclosure,
while they are transferred in plain text.
Both, initiator and responder strong credentials are of the same format:

\begin {center}
\begin {tabular}{llll}
\multicolumn{4}{l} {StrongCredentials ::= SET \{ } \\
& & & \\
bind-token     & [0] & Token           & OPTIONAL, \\
certificate    & [1] & Certificates    & OPTIONAL  \\
 \} & & &
\end {tabular}
\end {center}
\label{fig-isw-sbp-2}
\stepcounter{Abb}
{\footnotesize Fig.\arabic{Abb}:
X.400 Strong credentials of MTS- and MTA-bind
\\[1ex]
``Token'' conveys mainly a signed random number.
``Certificates'' has the same format as the message originator certificate.
}
\\[1em]
The {\em bind token} has a similar base format
as the message token for the message oriented security services,
in that it contains
a time stamp, a recipient name,
and some additional signed and encrypted subfields
(see paragraph \ref{isw-mee} above).
However, the bind token signed data are different from
the message token signed data:

\begin {center}
\begin {tabular}{llll}
\multicolumn{4}{l} {bind-token-signed-data ::=  } \\
\multicolumn{4}{l} {TokenData ::= SEQUENCE \{   } \\
& & & \\
type   & [0] & INTEGER,   & -- -- 1 for ``bind-token''  \\
value  & [1] & BitString  & -- -- Random Number    \\
 \} & & &
\end {tabular}
\end {center}
\label{fig-isw-sbp-3}
\stepcounter{Abb}
{\footnotesize Fig.\arabic{Abb}:
X.400 Bind token signed data
\\[1ex]
The bind token signed data are included in the signed sequence
of the asymmetric token. The signed random number proves
the authenticity of the association partner.
}
\\[1em]
The bind token encrypted data are similar to
the message token signed data,
of which the confidentiality key subfield is of only interest.

The certificates, which is the second subfield
of a strong credentials can be used by the initiator
to convey a verified copy of the public-asymmetric-encryption key
in order to enable the responder to compute the responder bind token.
It can also be used to convey the public-asymmetric-verification key
in order to enable the partner to verify the bind token.
However, in a secure message handling environment,
it is expected, that directly interacting agents have or have access to
their partner's public keys. Therefore, it is recommended
to omit the certificates subfield from the strong credentials element.

{\bf Strong Authentication}
\label{isw-sa}

For strong authentication between MTA and MTA,
and between MTS (MTA) and MTS-user (UA or MS),
it is recommended to exchange the following security elements:

\begin{itemize}
\item {\bf MTS-/MTA-bind argument:}
\item initiator name,
\item strong credentials:
\begin{itemize}
\item asymmetric token (signed sequence):
\item signature algorithm identifier,
\item recipient name,
\item signed data: {\em random number};
\item {\em no} encryption data;
\end{itemize}
\item {\em no} initiator certificates;
\item security context:
\begin{itemize}
\item one security label, demanding this form of strong authentication.
\end{itemize}
\end{itemize}

\begin{itemize}
\item {\bf MTS-/MTA-bind result:}
\item responder name,
\item strong credentials:
\begin{itemize}
\item asymmetric token (signed sequence):
\item signature algorithm identifier,
\item recipient name,
\item signed data: {\em initiator's random number};
\item {\em no} encryption data;
\end{itemize}
\item {\em no} responder certificates;
\end{itemize}

The responder bind token signed data should contain
exactly the same random number as sent by the initiator,
this way making a replay of the responder's result impossible.
The time stamp is supposed to encounter replay of the
initiator's argument. Additionally, the random number
could be computed to be increasing with the number
of associations between two partners; this, too, can be checked
by a responder against replay of an initiator's argument.
The initiator name and the reponder name in combination with
the respective signature of the bind token
proves the identity of the partners to one another,
provided they have or have access to the public keys of one another.

{\bf Confidential Association}
\label{isw-ca}

If an acceptable security context (i.e. a security label)
requires a confidential association,
the bind token encrypted data of the initiator of the association
convey one symmetric DES-key.
After the association has been established,
both partners possess this DES-key,
which is used by both partners in order to
encrypt and decrypt all parameters of the subsequent operations
of this association, such that only DES-encrypted
parameters are visible above the ACSE sublayer.
There is no advantage in security, but a disadvantage of
effectivity from an operational point of view,
if instead of one shared DES-key each partner generates
his own DES-key for encryption, and uses the partner's DES-key for decryption.
However, one DES-key should be {\em valid for only one association
between two partners} and should {\em never be reused}.


\subsection{Internet PEM (RFC 1421 - 1424)}
The acronym {\em PEM} is an abbreviation for
``Privacy Enhanced Mail''.
It is defined by a
related set of RFCs
(RFC 1421, 1422, 1423, 1424)
titled ``Privacy Enhancement for Internet Electronic Mail''.

The basic idea of {\em PEM} is to define security services
for the protection of texts
which are document oriented (in contrast to transfer protocol oriented)
in that they are transparent to the mail transfer systems.
See also figure 19 on page \pageref{fig-isw-mhs-3} above. \\
{\em RFC 1421}, paragraph 1 ({\em Executive Summary}), says:

\begin{quote}
This RFC defines message encipherment and authentication
procedures, in order to provide privacy enhanced mail (PEM)
services for electronic mail transfer in the Internet.
$\ldots$
Privacy enhancement services (confidentiality, authentication,
message integrity assurance, and non-repudiation of origin)
are offered through the use of
end-to-end cryptography between originator and recipient
processes at or above the User Agent level.
No special processing requirements are imposed on the
Message Transfer System at endpoints or at intermediate relay sites.
This approach allows privacy enhancement facilities to be incorporated
selectively on a site-by-site or user-by-user basis without impact
on other Internet entities.
Interoperability among heterogeneous components and mail transport
facilities is supported.
\end{quote}

In {\em RFC 1421}, paragraph 4 ({\em Processing of Messages}),
the approach (4.3.2) is described as quoted below:

\begin{quote}
Our approach to supporting PEM across an
environment in which intermediate conversion may occur
defines an encoding for mail
which is uniformly representable across the set of PEM UAs
regardless of their systems' native character sets.
This encoded form is used
(for specified PEM message types)
to represent mail text in transit from originator to recipient,
but the encoding is not applied to enclosing MTS headers
or to encapsulated headers
inserted to carry control information between PEM UAs.
The encoding's characteristics are such that the transformations
anticipated between originator and recipient UAs will not prevent
an encoded message from being decoded properly at its destination.
\end{quote}

As a consequence of
this property the {\em PEM}-Services are independent of a transport
protocol and are called
{\em document oriented} in SecuDe.
In summary, the outbound message is subjected to the following
composition of transformations:

\begin{displaymath}
Transmit\_Form=Encode(Encrypt(Canonicalize(Local\_Form)))
\end{displaymath}

`Encrypt' means cryptographic encryption.
`Encode' means the printable representation of an arbitrary octet string
(also referred to as `zoning').
An encapsulated header portion of so called
``encapsulated header fields''
is added to the transformed text.
It contains encryption and signature control fields inserted in plaintext.
The encapsulated header fields are
Proc-Type, DEK-Info, Originator-ID,
Certificate, Issuer-Certificate, MIC-Info,
Recipient-ID, Key-Info
(see paragraphs 4.3.2.5, {\em Summary of Transformations},
and 4.4, {\em Encapsulation Mechanism},
of {\em RFC 1421}).

An example of
a privacy enhanced mail including encapsulated header fields and an 
encrypted and encoded
message body, is given in Figure 3, {\em Example Encapsulated Message
(Asymmetric Case)} of {\em RFC 1421}.
{\small
\begin{verbatim}
-----BEGIN PRIVACY-ENHANCED MESSAGE-----
Proc-Type: 4,ENCRYPTED
Content-Domain: RFC822
DEK-Info: DES-CBC,54FDF243C0F93E9F
Originator-Certificate: 
 MIIBwDCCAWoCAQEwDQYJKoZIhvcNAQECBQAwXjELMAkGA1UEBhMCREUxOzA5BgNV
 BAoTMkdlc2VsbHNjaGFmdCBmdWVyIE1hdGhlbWF0aWsgdW5kIERhdGVudmVyYXJi
 ZWl0dW5nMRIwEAYDVQQHEwlEYXJtc3RhZHQwHhcNOTMwMzEyMTcyNTE2WhcNOTQw
 MzEyMTcyNTE2WjB7MQswCQYDVQQGEwJERTE7MDkGA1UEChMyR2VzZWxsc2NoYWZ0
 IGZ1ZXIgTWF0aGVtYXRpayB1bmQgRGF0ZW52ZXJhcmJlaXR1bmcxEjAQBgNVBAcT
 CURhcm1zdGFkdDEbMBkGA1UEAxMSV29sZmdhbmcgU2NobmVpZGVyMFkwCgYEVQgB
 AQICAgADSwAwSAJBAIGZfQqqb9R2HJ+1P3vsFRGJBy75faHOdOUSwn8VqIyR3zQR
 /J2LjUhGRgfk20DYfXy5NymLyb7hdPM5F0IF+ecCAwEAATANBgkqhkiG9w0BAQIF
 AANBABkhNPDPgenCPR9NTOO5LIMz3Pot0p7E0+ES6+DYnwTxqKWtystk4c7ql1qA
 t0EkylUzV0Q/TOrz2LuZBZ3ZIC0=
Key-Info: RSA,
 SrBlHtVHEhuGMOmEJbV598jZX6P1DDziAr5DX9B5ATINFMfj/Ql0vs0JcvFDysk2
 wGBvCMZPj4hjNDO5N3UI7g==
Issuer-Certificate: 
 MIIBjzCCATkCAQIwDQYJKoZIhvcNAQECBQAwSjELMAkGA1UEBhMCREUxOzA5BgNV
 BAoTMkdlc2VsbHNjaGFmdCBmdWVyIE1hdGhlbWF0aWsgdW5kIERhdGVudmVyYXJi
 ZWl0dW5nMB4XDTkzMDMxMjE3MDUwNVoXDTk0MDMxMjE3MDUwNVowXjELMAkGA1UE
 BhMCREUxOzA5BgNVBAoTMkdlc2VsbHNjaGFmdCBmdWVyIE1hdGhlbWF0aWsgdW5k
 IERhdGVudmVyYXJiZWl0dW5nMRIwEAYDVQQHEwlEYXJtc3RhZHQwWTAKBgRVCAEB
 AgICAANLADBIAkEAhh4fPTepNuoG39J/gkg3gjiHYxzs5qEjiYVxR5UB9uRp2sBt
 VrhwMdTKXYkR0lkJgPWmLZWQcUwHq8yWkPnKmQIDAQABMA0GCSqGSIb3DQEBAgUA
 A0EAQMqLY9AwuynacHMa0G+jOT9WKwLBD9CO/f83oGparUOMz09smziq4ojqP31F
 BIOpLyCYqLsN/0/LFeRQeyoKlw==
MIC-Info: RSA-MD5,RSA,
 9v02VRNRUHD9OKK90b63DtoLAK+tIQRdAJ10m7MBLHAgatRJpqRWFraH4NAU3/64
 igVsfUf72dcaTAm+COjDwauAs8fx1gyh
Recipient-ID-Asymmetric: 
 MEYxCzAJBgNVBAYTAlVTMSQwIgYDVQQKExtUcnVzdGVkIEluZm9ybWF0aW9uIFN5
 c3RlbXMxETAPBgNVBAsTCEdsZW53b29k
 ,02
Key-Info: RSA,
 omNyyAx6SwzP8bbAQhCU8h7SSW5kqgmdt5ttViid67Dm1Dme0jlKH/glD1jKJzgX
 mWlDAx9uDCwd2ViWDuOP6w==

aQwmGMjuJ96iM9t9EDsJSTYY3p7wLkZLLt07/VaqM37DpChGB9quI25tvVYM9Pvl
aw385ZAr6ghQ2Ip0JYO+hsiTAy/BUezhwxrdZKXy+2N/XxAS6p2sNnC825xAoYOS
Gc94JvfT3VhHYbWcsesmPCmlTUDCnE3SW11sCUjG612dmPWiPdzM1xE3IAw2bQSu
R1bHgD6M7AfGT8KxIlD3pV8BFSPe2VynDwfbRIdZ2OlxoJqlwNIgvN74wXpS0DOc
FYpJOWvABYOjriMS+yYV6UX4CjXxpMrkosC2pyjIki9YMQ6/oNgGWRKBxiwvUnSj
iukMS93uBZA=
-----END PRIVACY-ENHANCED MESSAGE-----
\end{verbatim}
}

\label{fig-isw-pem}
\stepcounter{Abb}
{\footnotesize Fig.\arabic{Abb}:
Example of an RFC 1421-type Encapsulated Message}
\\[1em]
A C-language structure called {\em PemInfo}
contains the encryption and signature control information
and is utilized by the {\em pem\_*} interface functions
in order to process a message text.

{\small
\begin{verbatim}
typedef struct {
   Boolean       confidential; /* TRUE if PEM shall be encrypted  */
   Boolean       clear;        /* TRUE if PEM shall be unencoded  */
   Key          *encryptKEY;   /* plain DES-key                   */
   Certificates *origcert;     /* originator certificates         */
   AlgId        *signAI;       /* signature algorithm id          */
   RecpList     *recplist;     /* list of recipients' information */
} PemInfo;

typedef struct {
   KeyInfo   *key;          /*in PemInfo:encryptKEY: plain DES-key*/
   KeyRef     keyref;       /*in PemInfo:encryptKEY: unused       */
   PSESel    *pse_sel;      /*in PemInfo:encryptKEY: unused       */
   } Key;

typedef struct {
   AlgId      *subjectAI;   /*in PemInfo:encryptKEY: plain DES-IV */
   BitString  *subjectkey;  /*in PemInfo:encryptKEY: plain DES-key*/
   } KeyInfo;

typedef struct reclist {
   Certificate  *recpcert;       /*recipient's user certificate   */
   OctetString  *key;            /*RSA-encrypted DES-key          */
   struct reclist *next;
   } RecpList;
\end{verbatim}
}

\label{fig-isw-pem2}
\stepcounter{Abb}
{\footnotesize Fig.\arabic{Abb}: C-Structure PemInfo}
\\[1em]
The {\em pem-create} functions map
a plain user text and a related {\em PemInfo} onto
a privacy enhanced mail text.
Inversely, the {\em pem-scan} functions map
a privacy enhanced mail text onto
a plain user text and a related {\em PemInfo}
and verify the message integrity.

