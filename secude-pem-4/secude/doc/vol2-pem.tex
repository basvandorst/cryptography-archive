\subsection{Privacy Enhanced Mail Functions}

\subsubsection{pem\_pse\_open}
\nm{3X}{pem\_pse\_open}{Open PSE}
\label{pem_pse_open}
\hl{Synopsis}
\#include $<$pem.h$>$ 

int {\bf pem\_pse\_open} {\em (ca\_dir, psepath, pin)} \\
char {\em *ca\_dir}, {\em *psepath}, {\em *pin};
\hl{Input-Parameter}
\parname  {\em ca\_dir}
\pardescript {For open a CA PSE {\em *ca\_dir} must be the CA directory
		    where database files are stored. Can be relative to HOME}

\parname  {\em psepath}
\pardescript {The PSE path. Can be relative to HOME.}

\parname  {\em pin}
\pardescript {The PIN of the PSE. Ignored in case of a Smartcard PSE.}
\hl{Output-Parameter}
\parname  {\em Returncode}
\pardescript {An identifier for the PSE or -1 in case of an error.} \\
\hl{Description}

pem\_pse\_open tries to open the PSE {\em *psepath} with the given PIN or
the PIN entered on the SC Terminal.
\\
The returned PSE identifier is used for later access to the PSE.




\subsubsection{pem\_pse\_close}
\nm{3X}{pem\_pse\_close}{Close PSE}
\label{pem_pse_close}
\hl{Synopsis}
\#include $<$pem.h$>$ 

RC {\bf pem\_pse\_close} {\em (pse)} \\
int {\em pse};
\hl{Input-Parameter}
\parname  {\em pse}
\pardescript {The PSE identifier.}

\hl{Output-Parameter}
\parname  {\em Returncode RC}
\pardescript {Returncode 0 indicates success, -1 indicates an error.} \\
\hl{Description}

pem\_pse\_close closes the PSE identified by {\em pse}. The cache and the 
PIN is cleared.










 

\subsubsection{pem\_get\_Certificate}
\nm{3X}{pem\_get\_Certificate}{Search for a Certificate}
\label{pem_get_Certificate}
\hl{Synopsis}
\#include $<$pem.h$>$ 

Certificate {\bf *pem\_get\_Certificate} {\em (pse, type, certificate)} \\
int {\em pse}; \\
KeyType {\em type}; \\
Certificate {\em *certificate};
\hl{Input-Parameter}
\parname  {\em pse}
\pardescript {The PSE identifier.}

\parname  {\em type}
\pardescript {SIGNATURE or ENCRYPTION.}

\parname  {\em certificate}
\pardescript {A certificate where a subject name or an issuer name - serial
                    number combination is stored.}
\hl{Output-Parameter}
\parname  {\em Returncode}
\pardescript {The searched Certificate or a null pointer.} \\
\hl{Description}

If the given certificate is signed a copy is returned.
pem\_get\_Certificate searches on PSE and X500 directory for a certificate
which has the same subject or the same issuer and serial number as the 
given certificate and returns it. If no certificate found or in case
of an error the null pointer is returned.







 

\subsubsection{pem\_fill\_Certificate}
\nm{3X}{pem\_fill\_Certificate}{Search for a Certificate}
\label{pem_fill_Certificate}
\hl{Synopsis}
\#include $<$pem.h$>$ 

RC {\bf *pem\_fill\_Certificate} {\em (pse, type, certificate)} \\
int {\em pse}; \\
KeyType {\em type}; \\
Certificate {\em *certificate};
\hl{Input-Parameter}
\parname  {\em pse}
\pardescript {The PSE identifier.}

\parname  {\em type}
\pardescript {SIGNATURE or ENCRYPTION.}

\parname  {\em certificate}
\pardescript {A certificate where a subject name or an issuer name - serial
                    number combination is stored.}
\hl{Output-Parameter}
\parname  {\em Returncode RC}
\pardescript {The search result.} \\
\hl{Description}

If the given certificate is signed, nothing is done.
pem\_fill\_Certificate searches on PSE and X500 directory for a certificate
which has the same subject or the same issuer and serial number as the 
given certificate. If a certificate is found, the given certificate is replaced
and 1 is returned. Otherwise 0 is returned or -1 in case of an error.







 

\subsubsection{pem\_look\_for\_Certificate}
\nm{3X}{pem\_look\_for\_Certificate}{Search for a Certificate}
\label{pem_look_for_Certificate}
\hl{Synopsis}
\#include $<$pem.h$>$ 

RC {\bf pem\_look\_for\_Certificate} {\em (pse, type, certificate)} \\
int {\em pse}; \\
KeyType {\em type}; \\
Certificate {\em *certificate};
\hl{Input-Parameter}
\parname  {\em pse}
\pardescript {The PSE identifier.}

\parname  {\em type}
\pardescript {SIGNATURE or ENCRYPTION.}

\parname  {\em certificate}
\pardescript {A certificate where a subject name or an issuer name - serial
                    number combination is stored.}
\hl{Output-Parameter}
\parname  {\em Returncode RC}
\pardescript {The search result.} \\
\hl{Description}

pem\_look\_for\_Certificate searches for the TBS of {\em certificate} in
the own certificate and the PKList. The following return codes are possible:

\begin{tabular}{rrl}
-1 &:& error \\
0 &:& not found \\
1 &:& A TBS to this certificate found in PKList \\
2 &:& The given certificate is your own certificate \\
3 &:& The given certificate is a signed version of your own prototype certificate \\
\end{tabular}







 

\subsubsection{pem\_accept\_certification\_reply}
\nm{3X}{pem\_accept\_certification\_reply}{Install signed prototype certificate}
\label{pem_accept_certification_reply}
\hl{Synopsis}
\#include $<$pem.h$>$ 

RC {\bf pem\_accept\_certification\_reply} {\em (pse, local)} \\
int {\em pse};\\
PemMessageLocal {\em *local};
\hl{Input-Parameter}
\parname  {\em pse}
\pardescript {The PSE identifier.}

\parname  {\em local}
\pardescript {A pem message in local form.}
\hl{Output-Parameter}
\parname  {\em Returncode RC}
\pardescript {negative if erraneous, 0 if OK.} \\
\hl{Description}

pem\_accept\_certification\_reply checks if the certificates of the given
Message represents a certification reply for the own prototype
certificate and updates the PSE objects
Cert, FCPath and PKRoot.








 

\subsubsection{pem\_pse\_store\_crlset}
\nm{3X}{pem\_pse\_store\_crlset}{Store PEM CRLs on PSE}
\label{pem_pse_store_crlset}
\hl{Synopsis}
\#include $<$pem.h$>$ 

RC {\bf pem\_pse\_store\_crlset} {\em (pse, set\_of\_pemcrlwithcerts)} \\
int {\em pse};\\
SET\_OF\_CRLWithCertificates {\em *set\_of\_pemcrlwithcerts};
\hl{Input-Parameter}
\parname  {\em pse}
\pardescript {The PSE identifier.}

\parname  {\em set\_of\_pemcrlwithcerts}
\pardescript {A set of PEM CRLs with certificates.}
\hl{Output-Parameter}
\parname  {\em Returncode}
\pardescript {negative if erraneous, 0 if OK.} \\
\hl{Description}

pem\_pse\_store\_crlset stores the CRL components of the given set in the
PSE object CrlSet.







 

\subsubsection{pem\_cadb\_store\_crlset}
\nm{3X}{pem\_cadb\_store\_crlset}{Store PEM CRLs on CA database}
\label{pem_cadb_store_crlset}
\hl{Synopsis}
\#include $<$pem.h$>$ 

RC {\bf pem\_cadb\_store\_crlset} {\em (pse, set\_of\_pemcrlwithcerts)} \\
int {\em pse};\\
SET\_OF\_CRLWithCertificates {\em *set\_of\_pemcrlwithcerts};
\hl{Input-Parameter}
\parname  {\em pse}
\pardescript {The PSE identifier.}

\parname  {\em set\_of\_pemcrlwithcerts}
\pardescript {A set of PEM CRLs with certificates.}
\hl{Output-Parameter}
\parname  {\em Returncode RC}
\pardescript {negative if erraneous, 0 if OK.} \\
\hl{Description}

pem\_pse\_store\_crlset stores the given PEM CRLs in the
database. {\em pse} must identify a CA PSE.







\subsubsection{pem\_certify}
\nm{3X}{pem\_certify}{Certify a PEM message}
\label{pem_certify}
\hl{Synopsis}
\#include $<$pem.h$>$ 

PemMessageLocal *{\bf pem\_certify} {\em (pse, local\_mess)} \\
int {\em pse};\\
PemMessageLocal {\em *local\_mess};
\hl{Input-Parameter}
\parname  {\em pse}
\pardescript {The PSE identifier.}

\parname  {\em local\_mess}
\pardescript {A PEM message in local form.}
\hl{Output-Parameter}
\parname  {\em Returncode }
\pardescript {The resulting certified PEM message.} \\
\hl{Description}

pem\_certify makes a copy of the given PEM message,
signs the originator certificate and adds FCPath and PKRoot as
issuer certificates.













\subsubsection{pem\_validate}
\nm{3X}{pem\_validate}{Validate a PEM message}
\label{pem_validate}
\hl{Synopsis}
\#include $<$pem.h$>$ 

PemMessageLocal *{\bf pem\_validate} {\em (pse, local\_mess)} \\
int {\em pse};\\
PemMessageLocal {\em *local\_mess};
\hl{Input-Parameter}
\parname  {\em pse}
\pardescript {The PSE identifier.}

\parname  {\em local\_mess}
\pardescript {A PEM message in local form.}
\hl{Output-Parameter}
\parname  {\em Returncode }
\pardescript {The resulting validated PEM message.} \\
\hl{Description}

pem\_validate validates the given message and,
if neccessary, decrypts the body.
The validation result  and the clear body are set
in the returned PEM message.











\subsubsection{pem\_validateSet}
\nm{3X}{pem\_validateSet}{Validate a set of PEM messages}
\label{pem_validateSet}
\hl{Synopsis}
\#include $<$pem.h$>$ 

PemMessageLocal *{\bf pem\_validateSet} {\em (pse, message)} \\
int {\em pse};\\
SET\_OF\_PemMessageLocal {\em *message};
\hl{Input-Parameter}
\parname  {\em pse}
\pardescript {The PSE identifier.}

\parname  {\em message}
\pardescript {A set of PEM messages in local form.}
\hl{Output-Parameter}
\parname  {\em Returncode }
\pardescript {The resulting validated PEM messages.} \\
\hl{Description}

pem\_validateSet validates the given messagess and,
if neccessary, decrypts the bodys.
The validation result  and the clear body are set
in the returned PEM messages.




 

\subsubsection{pem\_reply\_crl\_requestSet}
\nm{3X}{pem\_reply\_crl\_requestSet}{Create a reply to CRL requests}
\label{pem_reply_crl_requestSet}
\hl{Synopsis}
\#include $<$pem.h$>$ 

OctetString *{\bf pem\_reply\_crl\_requestSet} {\em (pse, local\_mess)} \\
int {\em pse};\\
SET\_OF\_PemMessageLocal {\em *local\_mess};
\hl{Input-Parameter}
\parname  {\em pse}
\pardescript {The PSE identifier.}

\parname  {\em local\_mess}
\pardescript {PEM CRL-RR messages in local form.}
\hl{Output-Parameter}
\parname  {\em Returncode }
\pardescript {The resulting PEM CRL messages.} \\
\hl{Description}

pem\_reply\_crl\_requestSet creates a PEM CRL message according to each
request given in {\em *local\_mess}.







 

\subsubsection{pem\_crl}
\nm{3X}{pem\_crl}{Create a CRL message}
\label{pem_crl}
\hl{Synopsis}
\#include $<$pem.h$>$ 

PemMessageLocal *{\bf pem\_crl} {\em (pse, issuers)} \\
int {\em pse};\\
SET\_OF\_NAME {\em *issuers};
\hl{Input-Parameter}
\parname  {\em pse}
\pardescript {The PSE identifier.}

\parname  {\em issuers}
\pardescript {Issuers to add PEM CRLs for.}
\hl{Output-Parameter}
\parname  {\em Returncode }
\pardescript {The resulting PEM CRL message.} \\
\hl{Description}

pem\_crl creates a PEM CRL message with PEM CRLs of the given issuers.







 

\subsubsection{pem\_crl\_rr}
\nm{3X}{pem\_crl\_rr}{Create a CRL-RR message}
\label{pem_crl_rr}
\hl{Synopsis}
\#include $<$pem.h$>$ 

PemMessageLocal *{\bf pem\_crl\_rr} {\em (pse, issuers)} \\
int {\em pse};\\
SET\_OF\_NAME {\em *issuers};
\hl{Input-Parameter}
\parname  {\em pse}
\pardescript {The PSE identifier.}

\parname  {\em issuers}
\pardescript {Issuers to add PEM CRL requests for.}
\hl{Output-Parameter}
\parname  {\em Returncode }
\pardescript {The resulting PEM CRL-RETRIEVAL-REQUEST message.} \\
\hl{Description}

pem\_crl\_rr creates a PEM CRL-RETRIEVAL-REQUEST message with requests for the given issuers.







 

\subsubsection{pem\_enhance}
\nm{3X}{pem\_enhance}{Create a PEM message}
\label{pem_enhance}
\hl{Synopsis}
\#include $<$pem.h$>$ 

PemMessageLocal *{\bf pem\_enhance} {\em (pse, proc\_type, text, no\_of\_certs, recipients)} \\
int {\em pse};\\
PEM\_Proc\_Types {\em proc\_type};\\
OctetString {\em *text};\\
int {\em no\_of\_certs};\\
SET\_OF\_NAME {\em *recipients};
\hl{Input-Parameter}
\parname  {\em pse}
\pardescript {The PSE identifier.}

\parname  {\em proc\_type}
\pardescript {The Proc-Type (PEM\_MCC, PEM\_MCO or PEM\_ENC).}

\parname  {\em text}
\pardescript {The body of the message.}

\parname  {\em no\_of\_certs}
\pardescript {The number of inserted certificates. 0 means Originator-Id-Asymmetric,
                           1 means only Originator-Certificate.}

\parname  {\em recipients}
\pardescript {The recipients of the PEM message if Proc-Type is ENCRYPTED.
              In case of an empty element the own KeyInfo field is added}
\hl{Output-Parameter}
\parname  {\em Returncode }
\pardescript {The resulting PEM CRL-RETRIEVAL-REQUEST message.} \\
\hl{Description}

pem\_enhance creates a PEM message in local form.





 

\subsubsection{pem\_check\_name}
\nm{3X}{pem\_check\_name}{Check for aliases}
\label{pem_check_name}
\hl{Synopsis}
\#include $<$pem.h$>$ 

RC *{\bf pem\_check\_name} {\em (pse, name)} \\
int {\em pse};\\
NAME {\em *name};
\hl{Input-Parameter}
\parname  {\em pse}
\pardescript {The PSE identifier.}

\parname  {\em name}
\pardescript {A name in alias, Name or DName form.}
\hl{Output-Parameter}
\parname  {\em Returncode }
\pardescript {negative if it's impossible to create a DName, 0 if OK.} \\
\hl{Description}

pem\_check\_name tries to create the empty components
of the NAME structure.







 

\subsubsection{pem\_forward}
\nm{3X}{pem\_forward}{Forward a PEM message}
\label{pem_forward}
\hl{Synopsis}
\#include $<$pem.h$>$ 

PemMessageLocal *{\bf pem\_forward} {\em (pse, message, proc\_type, recipients)} \\
int {\em pse};\\
PemMessageLocal {\em *message};\\
PEM\_Proc\_Types {\em proc\_type};\\
SET\_OF\_NAME {\em *recipients};
\hl{Input-Parameter}
\parname  {\em pse}
\pardescript {The PSE identifier.}

\parname  {\em message}
\pardescript {The message in local form.}

\parname  {\em proc\_type}
\pardescript {The new Proc-Type (PEM\_MCC, PEM\_MCO or PEM\_ENC).}

\parname  {\em recipients}
\pardescript {The recipients of the PEM message if the new Proc-Type is ENCRYPTED.}
\hl{Output-Parameter}
\parname  {\em Returncode }
\pardescript {The resulting message.} \\
\hl{Description}

pem\_forward changes the Proc-Type of a PEM message. 
This operation does not influence the possibility of validation.
The original message must be decryptable.
If the new Proc-Type is PEM\_ENC new recipients can be specified.
If the new or old
Proc-Type isn't PEM\_MCC, PEM\_MCO or PEM\_ENC the message is only copied.





\subsubsection{pem\_forwardSet}
\nm{3X}{pem\_forwardSet}{Forward a message set}
\label{pem_forwardSet}
\hl{Synopsis}
\#include $<$pem.h$>$ 

PemMessageLocal *{\bf pem\_forwardSet} {\em (pse, message, proc\_type, recipients)} \\
int {\em pse};\\
PemMessageLocal {\em *message};\\
PEM\_Proc\_Types {\em proc\_type};\\
SET\_OF\_NAME {\em *recipients};
\hl{Input-Parameter}
\parname  {\em pse}
\pardescript {The PSE identifier.}

\parname  {\em message}
\pardescript {The message set in local form.}

\parname  {\em proc\_type}
\pardescript {The new Proc-Type (PEM\_MCC, PEM\_MCO or PEM\_ENC).}

\parname  {\em recipients}
\pardescript {The recipients of the PEM message if the new Proc-Type is ENCRYPTED.}
\hl{Output-Parameter}
\parname  {\em Returncode }
\pardescript {The resulting message set.} \\
\hl{Description}

pem\_forwardSet changes the  Proc-Types of the message set. 
This operation does not influence the possibility of validation.
The original messages must be decryptable.
If the new Proc-Type is PEM\_ENC new recipients can be specified.
If the new or old
Proc-Type isn't PEM\_MCC, PEM\_MCO or PEM\_ENC the message is only copied.







\subsubsection{pem\_proctypes}
\nm{3X}{pem\_proctypes}{Count Proc-Types}
\label{pem_proctypes}
\hl{Synopsis}
\#include $<$pem.h$>$ 

int *{\bf pem\_proctypes} {\em (local)} \\
PemMessageLocal {\em *local};
\hl{Input-Parameter}
\parname  {\em local}
\pardescript {The message set in local form.}
\hl{Output-Parameter}
\parname  {\em Returnvalue }
\pardescript {The Proc-Type counts.} \\
\hl{Description}

pem\_proctypes counts how often each Proc-Type is found in a message set. 
the result is returned as an integer array with the enumeration PEM\_Proc\_Types
as index. If no PEM\_MCC, PEM\_MCO or PEM\_ENC is found the first element is set to 
-1. The element with index NO\_PEM is the number of clear messages.






\subsubsection{pem\_LocalSet2CanonSet}
\nm{3X}{pem\_LocalSet2CanonSet}{Convert a message set from local to canon form}
\label{pem_LocalSet2CanonSet}
\hl{Synopsis}
\#include $<$pem.h$>$ 

SET\_OF\_PemMessageCanon *{\bf pem\_LocalSet2CanonSet} {\em (local)} \\
SET\_OF\_PemMessageLocal {\em *local};
\hl{Input-Parameter}
\parname  {\em local}
\pardescript {The message set in local form.}
\hl{Output-Parameter}
\parname  {\em Returnvalue }
\pardescript {The message set in canon form.} \\
\hl{Description}

pem\_LocalSet2CanonSet converts a message set from a local to a canon form
using RFC and ASN.1 encoding for PEM message fields.







\subsubsection{pem\_Local2Canon}
\nm{3X}{pem\_Local2Canon}{Convert a message from local to canon form}
\label{pem_Local2Canon}
\hl{Synopsis}
\#include $<$pem.h$>$ 

PemMessageCanon *{\bf pem\_Local2Canon} {\em (local)} \\
PemMessageLocal {\em *local};
\hl{Input-Parameter}
\parname  {\em local}
\pardescript {The message in local form.}
\hl{Output-Parameter}
\parname  {\em Returnvalue }
\pardescript {The message in canon form.} \\
\hl{Description}

pem\_Local2Canon converts a message from a local to a canon form
using RFC and ASN.1 encoding for PEM message fields.





\subsubsection{pem\_CanonSet2LocalSet}
\nm{3X}{pem\_CanonSet2LocalSet}{Convert a message set from canon to local form}
\label{pem_CanonSet2LocalSet}
\hl{Synopsis}
\#include $<$pem.h$>$ 

SET\_OF\_PemMessageLocal *{\bf pem\_CanonSet2LocalSet} {\em (canon)} \\
SET\_OF\_PemMessageCanon {\em *canon};
\hl{Input-Parameter}
\parname  {\em canon}
\pardescript {The message set in canon form.}
\hl{Output-Parameter}
\parname  {\em Returnvalue }
\pardescript {The message set in local form.} \\
\hl{Description}

pem\_CanonSet2LocalSet converts a message set from a canon to a local form
using RFC and ASN.1 decoding for PEM message fields.







\subsubsection{pem\_Canon2Local}
\nm{3X}{pem\_Canon2Local}{Convert a message from canon to local form}
\label{pem_Canon2Local}
\hl{Synopsis}
\#include $<$pem.h$>$ 

PemMessageLocal *{\bf pem\_Canon2Local} {\em (canon)} \\
PemMessageCanon {\em *canon};
\hl{Input-Parameter}
\parname  {\em canon}
\pardescript {The message in canon form.}
\hl{Output-Parameter}
\parname  {\em Returnvalue }
\pardescript {The message in local form.} \\
\hl{Description}

pem\_Canon2Local converts a message from a canon to a local form
using RFC and ASN.1 decoding for PEM message fields.








\subsubsection{pem\_Local2Clear}
\nm{3X}{pem\_Local2Clear}{Extract the clear body}
\label{pem_Local2Clear}
\hl{Synopsis}
\#include $<$pem.h$>$ 

OctetString *{\bf pem\_Local2Clear} {\em (local, insert\_boundaries)} \\
PemMessageLocal {\em *local};\\
Boolean {\em insert\_boundaries};
\hl{Input-Parameter}
\parname  {\em local}
\pardescript {The message in local form.}
\hl{Output-Parameter}
\parname  {\em Returnvalue }
\pardescript {The clear body.} \\
\hl{Description}

pem\_Local2Clear returns the clear text of the given message encaplsulated
by validation information (if {\em insert\_boundaries} is positiv).









\subsubsection{pem\_LocalSet2Clear}
\nm{3X}{pem\_LocalSet2Clear}{Extract the clear bodies}
\label{pem_LocalSet2Clear}
\hl{Synopsis}
\#include $<$pem.h$>$ 

OctetString *{\bf pem\_LocalSet2Clear} {\em (local, insert\_boundaries)} \\
SET\_OF\_PemMessageLocal {\em *local};\\
Boolean {\em insert\_boundaries};
\hl{Input-Parameter}
\parname  {\em local}
\pardescript {The message set in local form.}
\hl{Output-Parameter}
\parname  {\em Returnvalue }
\pardescript {The clear body.} \\
\hl{Description}

pem\_LocalSet2Clear returns the concatinated clear texts of the given message set,
each encaplsulated
by validation information (if {\em insert\_boundaries} is positiv).









\subsubsection{pem\_mic\_clear\_bodys}
\nm{3X}{pem\_mic\_clear\_bodys}{Extract the clear bodies of PEM\_MCC and PEM\_MCO messages}
\label{pem_mic_clear_bodys}
\hl{Synopsis}
\#include $<$pem.h$>$ 

OctetString *{\bf pem\_mic\_clear\_bodys} {\em (local)} \\
SET\_OF\_PemMessageLocal {\em *local};
\hl{Input-Parameter}
\parname  {\em local}
\pardescript {The message set in local form.}
\hl{Output-Parameter}
\parname  {\em Returnvalue }
\pardescript {The clear bodys of PEM\_MCC and PEM\_MCO messages.} \\
\hl{Description}

pem\_mic\_clear\_bodys returns the concatinated clear texts of all  PEM\_MCC and PEM\_MCO messages
of the given message set.







\subsubsection{pem\_build\_one}
\nm{3X}{pem\_build\_one}{Build a readable text from a message in canon form}
\label{pem_build_one}
\hl{Synopsis}
\#include $<$pem.h$>$ 

OctetString *{\bf pem\_build\_one} {\em (message)} \\
PemMessageCanon {\em *message};
\hl{Input-Parameter}
\parname  {\em message}
\pardescript {The message in local form.}
\hl{Output-Parameter}
\parname  {\em Returnvalue }
\pardescript {The message as OctetString.} \\
\hl{Description}

pem\_build\_one builds a readable text from a message in canon form.









\subsubsection{pem\_build}
\nm{3X}{pem\_build}{Build a readable text from a message set in canon form}
\label{pem_build}
\hl{Synopsis}
\#include $<$pem.h$>$ 

OctetString *{\bf pem\_build} {\em (message)} \\
SET\_OF\_PemMessageCanon {\em *message};
\hl{Input-Parameter}
\parname  {\em message}
\pardescript {The message set in local form.}
\hl{Output-Parameter}
\parname  {\em Returnvalue }
\pardescript {The message set as OctetString.} \\
\hl{Description}

pem\_build builds a readable text from a message set in canon form.







\subsubsection{pem\_parse}
\nm{3X}{pem\_parse}{Parse a text for PEM message parts}
\label{pem_parse}
\hl{Synopsis}
\#include $<$pem.h$>$ 

SET\_OF\_PemMessageCanon *{\bf pem\_parse} {\em (text)} \\
OctetString {\em *text};
\hl{Input-Parameter}
\parname  {\em text}
\pardescript {The message text.}
\hl{Output-Parameter}
\parname  {\em Returnvalue }
\pardescript {The message in canon form.} \\
\hl{Description}

pem\_parse parses a text for PEM message parts. Each part of clear or
PEM message text is stored as an element of a list. A PEM message
is additionally parsed for its header field parameters.





