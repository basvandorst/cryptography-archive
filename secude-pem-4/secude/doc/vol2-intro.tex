\nm{1}{INTRO}{Introduction to SecuDE}
\label{intro1}
\hl{Description}
{\bf General Introduction}
\\ [1em]
The contribution of SecuDE to public domain is part of our effort to 
facilitate the open, authentic and privacy-protected electronic 
telecooperation between people through using public key cryptography and openly 
available security technology. Providing authenticity and protecting the 
privacy is an increasing concern of everyone as electronic information 
storage and exchange is rapidly growing. This can be done only with publicly known
security technology of which the effectiveness is understandable and 
comprehensible through public control. Security must only rely on the secrecy
of keys, not on the secrecy of methods. 
\\ [1em]
SecuDE (Security Development Environment) is a security toolkit for
Unix systems. It comprises a library of security functions and a
number of utilities with the following functionality:
\bi
\m basic cryptographic functions like RSA, DSA, DES, various hash functions, 
   including the RFC 1423 defined algorithm suite, 
   OIW defined algorithms, and DSS,

\m security functions for {\em data confidentiality}, {\em data integrity}, 
   {\em origin authentication} and {\em non-repudation of origin} purposes 
   on the basis of digital signatures and symmetric encryption,

\m X.509 key certification functions, handling of certification pathes, 
   cross-certification, certificate revocation,

\m utilities and functions for the operation of certification authorities 
   (CA) and interaction between certifying CAs and certified users,

\m utilities to sign, verify, encrypt, decrypt and hash files,

\m Internet PEM processing according to RFC 1421 - 1424 (see~\ref{pem})

\m processing of RFC 1422-defined certificate revocation lists,

\m secure access to public X.500 security attributes for the storage 
   and retrieval of certificates, cross-certificates and revocation 
   lists (integrated secured DUA, based on ICR1-QUIPU, using strong 
   authentication and signed DAP operations),

\m all necessary ASN.1 encoding/decoding (based on Isode-ICR1)

\m integrity-protected and confidentiality-protected storage of all security 
   relevant information of a user (secret keys, verification keys, certifiactes
   etc.) in a so called {\em Personal Security Environment} (PSE).
\ei
 
SecuDE provides two different PSE realizations,
\bi 
\m a smartcard environment (SC-PSE), 
\m a DES-encrypted Unix directory (laxly called software-PSE or SW-PSE),
\ei 
both only accessible through the usage of PINs (Personal Identification Numbers).
Smartcards require the purchase of a particular smartcard environment where 
RSA and DES cryptography is done in the smartcard reader (information available 
on request).
The PSE typically contains the user's secret and public key (the latter
contained in an X.509 certificate), the public root key which the user
trusts, the user's distinguished name, the user's login name, and the 
forward certification path to the user's root CA. In addition, the PSE 
allows to securely store other public keys after their validation 
(allowing henceforth to trust them like the root key without verifying 
them again), and certificate revocation lists (CRLs) from any other CA.
\\ [1em]
{\bf SecuDE PEM}

A PEM implementation is part of SecuDE. It provides a PEM filter
which transforms any input stream into a PEM formatted output stream 
and vice versa, and which should be capable of being  easily 
integrated into Mail-UAs or CA tools.SecuDE-PEM realizes all formats 
and procedures defined in the Internet Specifications RFC 1421-1424 
except that it only supports asymmetric key management. It is
possible to securely cache other's certificates and CRLs as this 
is part of the general SecuDE functionality.

SecuDE-PEM supports the certification and CRL procedures defined in RFC 
1424 and is integrated into the SecuDE CA functionality. As an additional 
functionality which goes beyond RFC 1421 - 1424, SecuDE-PEM may be 
configured with an integrated X.500 DUA which allows, for instance, 
automatic retrieval of certificates and CRLs during the PEM de-enhancement 
process.  
\\ [1em]
{\bf X.500 Package}

Secured X.500 DUA

Concerning X.500 Directory access, SecuDE can be configured in three 
alternatives:
\be
\m Without X.500 DUA functionality. In this case, the SecuDE package
   is self-contained and needs no additional software.

\m With integrated X.500 DUA, which operates on the basis of simple
   authentication (i.e. with passwords) and unprotected DAP operations.
   SecuDE uses ICR1-QUIPU library functions for this DUA functionality.
   These library functions are not included in the SecuDE package, and
   a standard ICR1-QUIPU installation is additionally required (the 
   Isode/QUIPU libraries libisode.a and libdsap.a are needed to be able 
   to bind the application programs).

\m With integrated secured X.500 DUA, which operates on the basis of
   strong authentication (i.e. with digital signatures) and signed DAP
   operations.
   SecuDE uses modified (i.e. security-enhanced) ICR1-QUIPU library 
   functions for this DUA functionality. These library functions are not 
   included in the SecuDE package, and a security-enhanced ICR1-QUIPU 
   installation is additionally required (the security-enhanced 
   Isode/QUIPU libraries libisode.a and libdsap.a are needed to be able 
   to bind the application programs).
\ee

Alternative 3 requires a security-enhanced version of Quipu which can
be obtained as a separate distribution from GMD. 
\\ [1em]
Secured X.500 DSA

The security-enhanced version of Quipu also contains software for a secured 
X.500 DSA. ``Secured'' means the ability to perform strong authentication 
during association setup between DUA and DSA, and subsequent signed 
operations at the DAP level. The DAP operations into which we have 
incorporated strong authentication are those assigned by the standard 
(X.511) for that purpose, i.e. Bind, Read, Compare, Search, List, AddEntry,
RemoveEntry, ModifyEntry, and ModifyRDN. We have provided both SIGNED 
arguments and SIGNED results.

This DSA can securely interwork with the secured DUA component of SecuDE 
(alternative 3 above).
\\ [1em]

{\large\bf SecuDE Utilities Summary}
\addtotocsmall{SecuDE Utilities Summary}
\\[1em]
{\bf User Utilities}

SecuDE provides a number 
of utilities to maintain the user's PSE, to hash, sign, verify, 
encrypt and decrypt files, to perform PEM transformations, to
operate certification authorities, and to perform key and certificate
management and distribution functions. All utilities require that
a user has a PSE which is the user's security database (a user may have
several PSEs; all utilities allow to specify a particular PSE using
parameter {\bf -p} {\em psename}). Sectool(1) is the window-oriented
interactive tool to maintain a PSE, access X.500 Directories, and
exchange information between the Directory and the PSE.
Psemaint(1) is the line-oriented equivalent of sectool(1).   
\\ [1em]
{\bf Utilities for Certification Authorities (CA)}
 
For the generation of certificates and CRLs certification authorities 
(CAs) are needed.
SecuDE basicly provides two methods for the exchange of security
information between SecuDE users and external CAs.
\be
\m A file interface of ASN.1 encoded objects is provided. These
   files can be exchanged by any means.
\m Procedures and formats according to PEM RFC 1424 are provided.
\ee
Key generation and certification can be done with
\be
\m User-generated keys, whereby the pubic component is
   subsequently transmitted to the CA for certification and
   retransmitted as certificate to the user (PEM
   procedures allow this method only),
\m CA-generated keys, whereby subsequently a whole PSE
   containing all keys and certificates is transmitted from
   the CA to the user (this requires that also the CA operates
   with SecuDE tools).
\ee
In addition, tools for setting up a CA hierarchy with SecuDE means 
only, and for the operation of CAs are provided.
This tool set is described in KM(1).
\\ [1em]
{\large\bf Access to Public Directories}
\addtotocsmall{Access to Public Directories}

SecuDE provides a set of functions to retrieve and update security 
attributes in Directories, like UserCertificate, CaCertificate, 
CrossCertificatePair, CertificateRevocationList etc.
\\ [1em]
{\bf Global X.500 Directory and Local afdb Directory}
\addtotocsmall{Global X.500 Directory and Local afdb Directory}
 
DUA functionality for the access to the X.500 Directory is based on
QUIPU-ICR1 software and requires a full ISODE-ICR1 installation on the system. 
X.500 DUA functionality is integrated into many SecuDE utilities
and functions. For instance, it is part of sectool(1) and psemaint(1)
which allows the transfer of security information between the user's
PSE and public Directories and vice versa.

As it stands at the moment the use of the global X.500 Directory is not
very widespread. Participants of the X.500 Directory are often
academic and research organizations. Also, in larger organizations 
the nature of publicly needed security information is not global
but local. To securely communicate with people from the same organization,
access to the X.500 Directory is not needed. It would be sufficient
to retrieve the necessary security information from a local database.

In order to support such a kind of functionality, SecuDE provides a
local Directory which is called the afdb Directory (afdb: authentication
framework data base). It resides in the Unix directory \$TOP/.af-db.
he afdb Directory stores, like the X.500 Directory, certificates, 
cross certificate pairs and certificate revocation lists on the basis 
of X.501 distinguished names. Each entry in the afdb Directory is a 
subdirectory of \$TOP/.af-db with the printable DName as (Unix) directory 
name (see below on distinguished names).

Note that in order to use the word ``directory'' unambiguously the word
Directory (with capital D) means an X.500 or afdb Directory and the word
directory (with small d) means a Unix directory throughout the whole
SecuDE documentation. 
\\ [1em]
{\bf Local Names and Distinguished Names}
\addtotocsmall{Local Names and Distinguished Names}

This section provides a brief introduction to the concept of the global
unique name scheme introduced through X.500 Distinguished Names,
and its realization and application in SecuDE.
  
SecuDE uses the X.501 defined name scheme in order to identify subjects who
process or own keys and certificates, e.g. human users, organizations like 
CAs and
particular system components like X.500 DSAs. Each subject is uniquely
identified by a Distinguished Name. 
{\em DistinguishedName} is defined in correspondence to CCITT X.501, Annex 
B, as
a sequence of relative distinguished names which in turn are sets
of \{attribute type / attribute value\} pairs:
{\small
\bvtab
DistinguishedName ::= \6 {\bf SEQUENCE OF} RelativeDistinguishedName                  
\\ [1ex]
RelativeDistinguishedName ::= \6 {\bf SET OF} AttributeValueAssertion                 
\\ [1ex]
AttributeValueAssertion ::= \6 {\bf SEQUENCE} \{                                      
\\
                               \7 attributeType \3 OBJECT IDENTIFIER,                 
\\
                               \7 attributeValue \3 ANY DEFINED BY 
attributeType      \\
                            \6 \} \\
\evtab                             
}
SecuDE handles two representations of distinguished names. One is the {\em 
DName}
data type which is a C-structure corresponding to the above ASN.1 notation 
(see INTRO(3X)). 
The second which is used at the SecuDE utility interface is the {\em Name} 
data type.
This is a printable character string representation
of a distinguished name in the form
\bc
{\small keyword=value $+$ keyword=value ... , keyword=value $+$ keyword = 
value ..., ...}
\ec
keyword is a SecuDE-defined name for an attribute type. 
The ',' delimiter separates the RDN members of the DN, i. e. each 
keyword=value $+$ keyword=value ... expression denotes one relative 
distinguished name.
The '$+$' delimiter separates the attribute value assertions within one RDN
(note that within a single RDN a particular attribute type must not
appear more than once).
The following keywords, representing X.520 defined attributes,  are 
possible:
{\small
\bvtab
C   \2 countryName \\
O   \2 organizationName  \\
OU  \2 organizationalUnitName \\
S   \2 surname \\
CN  \2 commonName \\
L   \2 localityName \\
SP  \2 stateOrProvinceName \\
ST  \2 streetAddress \\
T   \2 title \\
SN  \2 serialNumber \\
BC  \2 businessCategory \\
D   \2 description
\evtab
}
RDNs must be ordered top down. Letter cases are not significant for names 
(except in case of the country code which must be a sequence of two capital
 letters), as well as spaces may be used inside and outside the names.

Example: 

{\small
C=DE, O=GMD, OU=I2 $+$ CN=Institut fuer TeleKooperationsTechnik, CN=Wolfgang 
Schneider}

is a possible printable character string representation of a DName as it might
be used at the utility interface of SecuDE. 
\\ [1em]
{\bf Local Name Aliasing}
\addtotocsmall{Name Aliasing}

Since Distinguished Names are pretty unhandy to deal with in the
day-to-day work, SecuDE provides aliasing functionality. Alias names
can be defined on a per-installation or on a per-user basis. Alias
files are integrity-protected, i.e. alias files are signed when they are
generated or changed, and they ar validated when they are read for
alias name transformations. Per-installation alias files must be signed
by a system administrator; per-user alias files are signed by the
user himself. 

In addition it is possible to store the user aliases on a per PSE basis
in the PSE-object AliasList (where they are integrity-protected through
PSE means). This is necessary if the user is connected to different
certification trees with more than one PSE.

The installation alias file is in \$TOP/.af-alias. \$TOP
is the top level directory of the SecuDE installation. The alias file
has the system administrator's signature in \$(TOP)/.af-alias.sig and his user
certificate and forward certification path in \$(TOP)/.af-alias.ctf.
In order to enable a user to verify the system administrator's signature
it is necessary that all users and the system administrator share the
same public root key.

The user alias file is either in \$HOME/.af-alias or in the PSE-object
AliasList. The alias transforming routines check for the existence of
the PSE-object AliasList first.
 
SecuDE distinguishes three types of alias names: local names (i.e.
login names), Internet mail addresses, and X.400 OR-Names. Each 
X.500 Distinguished Name may have several aliases of the above types.

Each utility which requires a {\em Name} parameter (i.e. the printable
character string representation of a Distinguished Name) maps the given 
parameter through the alias mechanism, i.e. each utility allows to
use alias names instead of X.500 Distinguished Names.
\\ [1em]
{\bf Cryptographic Algorithms}
\addtotocsmall{Cryptographic Algorithms}

SecuDE has a number of cryptographic algorithms implemented for different
purposes, for instance symmetric and asymmetric encryption algorithms
for the purpose of confidentiality and privacy, signature algorithms
for the purpose of authentication and non-repudiation, and hash
algorithms which are used for integrity purposes.
In SecuDE, cryptographic algorithms are identified through names or
through X.509 algorithm identifiers. An implemented algorithm may
have more than one Name. It is, however, uniquely identified through its
algorithm identifier. 

If a utilities allows to specify a cryptographic algorithm, an algorithm
name is required. See a more detailled description in the introduction 
of section 3 (subroutines) about the use of cryptographic algorithms
and their properties and parameters.
\\ [1em]
{\bf Common Parameters}
\addtotocsmall{Common Parameters}

There is set of parameters which are common to many SecuDE utilities:
\\ [1em]
\parname{{\bf -h}}
\pardescript{Print a helptext about the command to stderr.}

\parname{{\bf -p} {\em psename}}
\pardescript{{\em psename} denotes the name of the PSE which is in effect
for the utility. Each user may have more than one PSE. The use of a
particular PSE establishes the security context of the requested action.
For instance, a particular PSE may link the user to a particular
certification tree (and to a particular root key)}

\parname{{\bf -c} {\em cadir}}
\pardescript{This parameter does two things. It indicates that the utility
is called by a CA, and it denotes the name of the PSE which is in effect
for the utility (Per default a CA has the PSE {\em cadir}/.capse). If
both parameters {\bf -c} {\em cadir} and {\bf -p} {\em psename} are used,
it denotes the PSE {\em cadir}/{\em psename}.}

\parname{{\bf -k} {\em keysize}}
\pardescript{Keysize of asymmetric keys to be generated.}

\parname{{\bf -vVW}}
\pardescript{Options {\bf -v}, {\bf -V} and {\bf -W} set three
verbose levels, -v being the least and -W being the most verbose. Not
more than -v is necessary for normal operations. In general -v
causes that in case of signature verifications the structure
VerificationResult is printed, and that in case of errors the error
stack is printed.}

\parname{{\bf -z}}
\pardescript{Enable checking of malloc/free behaviour. Each address obtained
through malloc(3), calloc(3) or realloc(3) is stored in an internal list
which is examined whenn addresses are freed. Inconsistancies are reported
to stderr.} 

\parname{{\bf -i} {\em inputfile}}
\pardescript{Inputfile}

\parname{{\bf -o} {\em outputfile}}
\pardescript{Outputfile}

In case that Directories are being accessed:

\parname{{\bf -D}}
\pardescript{Enter the generated information (certificates or CRLs) into
the Directory}

In case that X.500 Directories are being accessed:

\parname{{\bf -d} {\em dsaname}}
\pardescript{{\em dsaname} denotes the name of the DSA (Directory System Agent) 
to be contacted for this purpose. {\em dsaname} is a name which must be contained in the 
{\em dsaptailor} file of the system as {\em dsa\_address} parameter. If {\bf -d} {\em dsaname} is
not given, the default DSA, which is the first {\em dsa\_address} parameter in the {\em dsaptailor} 
file, is accessed.}

\parname{{\bf -A} {\em authlevel}}
\pardescript{{\em authlevel} determines the level of authentication used when binding to the DSA. 
{\em authlevel} may have the values STRONG, SIMPLE or NOAUTH.
If {\bf -A} {\em authlevel} is omitted, the requested 
level of authentication is taken from the environment variable AUTHLEVEL, and is set to {\em no 
authentication}, if no such variable exists.
In case of simple authentication being used, the caller's distinguished name and
X.500 password are retrieved from the PSE objects Name and QuipuPWD, respectively.
In case of strong authentication being used, subsequent DAP operations are also signed.
STRONG requires that SecuDE is installed with the STRONG option.}

\parname{{\bf -t} {\em dsaptailorfile}}
\pardescript{{\em dsaptailorfile} is the Quipu tailorfile for the DUA side. Default: .../etc/dsaptailor of your local ICR1 installation.}

\hl{Getting Started}
\addtotocsmall{Getting Started}
Though SecuDE is basically a library of security functions for C-programmers, it
provides some utilities which allow users to encrypt and decrypt files (ENCRYPT(1), DECRYPT(1)),
sign and verify files (SIGN(1), VERIFY(1)), create and interpret PEM messages (PEM(1)).
After installation of SecuDE, it is necessary to provide each user with a PSE containing 
his personal secret keys for encryption and signatures, the corresponding certified public keys,
the public key of the root certification authority, and the forward certification path, i.e.
a PSE containing the AF-PSE-objects described in AF\_PSE\_OBJECTS(4).

Type

\begin{center}
secude -h
\end{center}
 
to get a list of available SecuDE utilities. Type

\begin{center}
secude -q
\end{center}

to get a quick reference guide of the available utilities to stderr. 

It is not necessary that a user gets certificates and the like from external or
third parties before he is able to do anything. He just has to create a PSE
with
\begin{center}
psecreate {\em distinguished-name}
\end{center} 
and can start signing and encrypting files and exchanging keys with others. The user's
distinguished name and the name of his default PSE can be defined in .cshrc
in the environment variable MYDNAME and PSE.

On a larger scale, however, it is advisable to connect to an existing certification
tree or to create an own one. To do the latter, a tree of certification authorities, at least consisting of one root 
certification authority, must be created, where a user gets his certificates from. 
This can be done using cacreate(1) and the commands described in KM(1), or using
the command pem(1) to perform PEM certification procedures instead.
After the creation of at least
a root certification authority and the user PSEs, the users are able to
use the utilities mentioned above.
 
A typical sequence of actions to get an own certification tree is described in KM(1).
An alternative could be the following (In this example all 
is done under a single Unix user ID. Normally, the different roles (CAs and users) are
played under different Unix user IDs or even on different systems):
\be
\m Create a root certification authority using {\em cacreate -c RootCA Name}.
\m If an additional level of CAs is needed below the root CA, create one or more 
   certification authorities below the root
   using {\em gen\_ca -c RootCA} at the root CA site and {\em inst\_ca CAFile} at the site
   of the CA to be created. CAFile is the name of a file which contains the CA-directory
   (of the CA to be created) as compressed tar-file. {\em gen\_ca} asks for the name of 
   CAFile and for the name of the CA-directory intercatively.
\m Repeat step 2. for each CA if additional levels of CAs are needed.
\m Create one ore more user PSEs under a CA using {\em gen\_pse -c CADir} at the CA site
   and {\em inst\_pse PSEFile} at the user site. CADir is the CA directory of the issueing
   CA of the user. PSEFile is the name of the file which contains the PSE (of the user)
   as compressed tar-file. 
\ee
See KM(1) for the detailled parameterization of those KM utilities.
 
Readers are also advised, in order to get an example of how to use KM(1), to examine the shell 
script {\em create\_TestTree} which is contained in SecuDE's bin directory. 
{\em create\_TestTree} creates a tree of one root CA, two second level CAs and two
user PSEs under each second level CA for test purposes. Furthermore, it can enter
the signature and encryption certificates of each CA and each user into a directory, which 
may either be a X.500 directory or the local substitute provided by SecuDE. The test 
tree can be deleted with {\em delete\_TestTree}.

\qr{1}{Quick Reference Guide}
{\scriptsize
\begin{verbatim}
SecuDE 4.3                                                                (c)1993 GMD Darmstadt


General User Utilities:
=======================

sign:  Sign Files


Description:

'sign' signs the given <files>. It uses algorithms and keys according
to the parameter -k and -a (default: SKnew/SignSK). For each
file in <files> it produces file.sig (containing the signature) and, if
-C is given, file.ctf (containing the user certificate and the forward
certification path).


usage:

sign [-CvzVWUfh] [-p <pse>] [-c <cadir>] [-k <key>] [-a <alg>] [-H <hashinput>] [<files>]

with:

-C               Produce .ctf files containing user certificate and forward certification
                 path for each file to be signed. Otherwise, produce only .sig files
-v               verbose
-V               Verbose
-W               Grand Verbose (for tests only)
-U               Show time used for cryptographic algorithms
-z               Control malloc/free behaviour
-h               Write this help text
-f               do not check validity time frame of signature certificate
-p <psename>     PSE name (default: Environment variable PSE or .pse)
-c <cadir>       name of CA-directory (default: Environment variable CADIR or .ca)
-k <key>         PSE-object or key reference of signature key. Default: SKnew/SignSK
-a <signalg>     Signature algorithm. Default: md5WithRsaTimedate (RSA) or dsaWithSHA (DSS),
                 depending on the signature key
-H <hashinput>   PSE-object or key reference of hash input (sqmodn only)
<files>          Filenames

----------------------------------------------------------------------------------------------------

verify: Verify Signatures of Files


Description:

'verify' verifies the given <files>. It uses algorithms and keys according
to the parameter -k and -a (default: Cert/SignCert). For each
file in <files> it expects file.sig (containing the signature) and
optionally file.ctf (containing the user certificate and the forward
certification path). If file.ctf does not exist, the verification will
only succeed if the file was signed by oneself.


usage:

verify [-DRvzFVWTUh] [-p <pse>] [-c <cadir>] [-d <dsa>] [-t <dsaptailor>] [-A <authlevel>]
       [-k <key>] [-f <fcpath>] [<files>]

with:

-D               Retrieve missing certificates from the Directory (X.500 or AF_DB)
-F               Consider own FCPath as trusted
-R               Consult certificate revocation lists for all cerificates which
                 are in the certification path
-v               verbose
-V               Verbose
-W               Grand Verbose (for tests only)
-z               Control malloc/free behaviour
-T               Perform each public key RSA operation in the smartcard  terminal
                 instead with the software in the workstation (the latter is default)
-U               Show time used for cryptographic algorithms
-h               Write this help text
-p <psename>     PSE name (default: Environment variable PSE or .pse)
-c <cadir>       name of CA-directory (default: Environment variable CADIR or .ca)
-d <dsa>         name of the DSA to be accessed for retrieving certificates
-t <dsaptailor>  Location of dsaptailor file (default: dsaptailor in the ISODE ETCDIR directory)
                 and certificate revocation lists
-A <authlevel>   Level of authentication used for binding to the X.500 Directory
                 It may be SIMPLE or STRONG (default: environment variable AUTHLEVEL, or NONE, if
                 this does not exist). STRONG implies the use of signed DAP operations
-k <key>         PSE-object (containing either a certificate or a key) or key reference 
                 of verification key. Default: Cert/SignCert
-f <fcpath>      name of PSE-object which contains the Forward Certification Path
<files>          Filenames

----------------------------------------------------------------------------------------------------

encrypt:  Encrypt Files


Description:

'encrypt' encrypts the given <files>. It uses algorithms and keys according
to the parameter -k or -w or -r or -w. Not more than one of these parameters
must be given (default: key and alg from PSE-object Cert/EncCert). For each
file in <files> it encrypts from file to file.enc and removes file.
'encrypt' also works as a filter from stdin to stdout.


usage:

encrypt [-DnxvVWXYzTU] [-k <key>] [-r <recipient>] [-w <pw>] [-e <deskey>] [-E <encalg>] [-p <pse>] 
        [-c <cadir>] [-d <dsa name>] [-t <dsaptailor>] [-A <auth-level>] [<files>]

with:

(not more than one of the parameters -k, -e, -w and -r must be given. If none of these four
 parameters is given, the file is symmetrically (default: desCBC) encrypted with a newly 
 generated DES key, and the DES key is asymmetrically encrypted with the own encryption key
 (Cert/EncCert). This is called the hybrid method.)

-k <object/ref>  PSE-object (containing either a certificate or a key) or key reference 
                 of encryption key. Default: Cert/EncCert. If this key is an asymmetric
                 key, the hybrid method is used.
-e <key>         DES key for desCBC encryption.
-x               If given, <key> of parameter -e must be a 16 character string
                 denoting the key in a [0-9, A-F] notation. Otherwise it must be
                 an 8 character string comprising the key itself.
-w               8 character password which is transformed into a DES key with a
                 one-way function, for desCBC encryption.
-r               Intended recipient (alias allowed). Search encryption key of this recipient
                 in EKList/PKList. If not given, encrypt with own encryption key. This
                 implies the hybrid method
-n               Don't use the hybrid method in case of an asymmetric key. Do asymmetric
                 encryption of the whole file (not recommendable for larger files)
-D               Search public encryption key in the Directory if it cannot
                 be found in EKList/PKList of the PSE
-d <dsa name>    Name of the DSA to be accessed for retrieving the public encryption key
-t <dsaptailor>  Location of dsaptailor file (default: dsaptailor in the ISODE ETCDIR directory)
-A <auth_level>  Level of authentication in case of X.500 Directory access. <auth-level>
                 may be SIMPLE or STRONG (default: environment variable AUTHLEVEL, or NONE, if
                 this does not exist). STRONG implies the use of signed DAP operations
-E <encalg>      Use algorithm encalg instead of desCBC for the file encryption. <encalg>
                 must be a symmetric algorithm
-v               verbose
-V               Verbose
-W               Grand Verbose (for tests only)
-z               Control malloc/free behaviour
-T               Perform each public key RSA operation in the smartcard  terminal
                 instead of employing the software in the workstation (the latter is the default)
-U               Show time used for cryptographic algorithms
-h               Write this help text
-p <psename>     PSE name (default: Environment variable PSE or .pse)
-c <cadir>       name of CA-directory (default: Environment variable CADIR or .ca)
-X               Read random number generator seed from PSE-object Random
-Y               Init random number generator seed through keyboard input
<files>          Filenames

----------------------------------------------------------------------------------------------------

decrypt:  Decrypt Files


Description:

'decrypt' decrypts the given <files>. It uses algorithms and keys according
to the parameter -k or -w or -r or -w. Not more than one of these parameters
must be given (default: key and alg from PSE-object SKnew/DecSKnew). For each
file in <files> it decrypts from file.enc to file and removes file.enc.
'decrypt' also works as a filter from stdin to stdout.


usage:

decrypt [-nxvVWzU] [-k <key>] [-w <pw>] [-e <deskey>] [-E <encalg>] [-p <pse>] [-c <cadir>] [<files>]

with:

(not more than one of the parameters -k, -e and -w must be given. If none of these three
 parameters is given, the file is symmetrically (default: desCBC) decrypted with the key from 
 <file>.sig which will be asymmetrically decrypted before with the own decryption key
 (SKnew/DecSKnew). This is called the hybrid method.)

-k <object/ref>  PSE-object (containing a KeyInfo) or key reference 
                 of decryption key. Default: SKnew/DecSKnew. If this key is an asymmetric
                 key, the hybrid method is used.
-e <key>         DES key for desCBC decryption.
-x               If given, <key> of parameter -e must be a 16 character string
                 denoting the key in a [0-9, A-F] notation. Otherwise it must be
                 an 8 character string comprising the key itself.
-w               8 character password which is transformed into a DES key with a
                 one-way function, for desCBC decryption.
-n               Don't use the hybrid method in case of an asymmetric key. Do asymmetric
                 decryption of the whole file
-E <decalg>      Use algorithm decalg instead of desCBC for the file decryption. <decalg>
                 must be a symmetric algorithm
-v               verbose
-V               Verbose
-W               Grand Verbose (for tests only)
-z               Control malloc/free behaviour
-U               Show time used for cryptographic algorithms
-h               Write this help text
-p <psename>     PSE name (default: Environment variable PSE or .pse)
-c <cadir>       name of CA-directory (default: Environment variable CADIR or .ca)
-X               Read random number generator seed from PSE-object Random
-Y               Init random number generator seed through keyboard input
<files>          Filenames

----------------------------------------------------------------------------------------------------

pem  Privacy Enhancement for Internet Electronic Mail

usage:

pem [ scan | MIC-CLEAR | MIC-ONLY | ENCRYPTED | CRL | CLR-RR | CERTIFY ]
    [-i <inputfile>] [-o <outputfile>] [-p <psename>] [-c <cadir>] [-m|M 1..200] [-u <update>] 
    [-r <name1 ...> ] [-y <name1 ...> ] [-CnfFNOhvzVWRDTIXYZlL] [-d <dsaname>] [-t <dsaptailor>]
    [-H <mic-alg>] [-S <micenc-alg>] [-E <msgenc-alg>] [-K <dekenc-alg>]

with:

scan             read PEM any Proc-Type, write clear body and/or update
                 PSE and/or CA-database according to -u (default)
MIC-CLEAR        read text file, write PEM Proc-Type MIC-CLEAR
MIC-ONLY         read text file, write PEM Proc-Type MIC-ONLY
ENCRYPTED        read text file, write PEM Proc Type ENCRYPTED according to -r
CRL              write PEM Proc-Type CRL according to -y
CRL-RR           write PEM Proc-Type CRL-RETRIEVAL-REQUEST according to -y
CERTIFY          read PEM Proc-Type MIC-CLEAR or MIC-ONLY, check whether it is certification
                 request, sign Prototype-certificate, write certification reply
-i <inputfile>   inputfile (default: stdin)
-o <outputfile>  outputfile (default: stdout)
-p <psename>     PSE name (default: .pse)
-c <cadir>       name of CA-directory (default: .ca)
-N               use of non-PEM conformant algorithms allowed
-O               RFC 1422 DName subordination not required
-F               consider own FCPath as trusted
-f               do not check validity time frames
-R               consult CRLs during validation process
-D               retrieve missing certificates or CRLs from the Directory (X.500 or .af-db)
-d <dsaname>     name of the DSA to be initially accessed (default: locally configured DSA)
-t <dsaptailor>  Location of dsaptailor file (default: dsaptailor in the ISODE ETCDIR directory)
-A <authlevel>   level of authentication used for binding to the X.500 Directory
-Z <device>      Input device ( default is /dev/tty )
-h               write this help text
-v               verbose
-V               Verbose
-W               Grand Verbose (for tests only)
-z               enable memory checking
-T               verification of signature is to be done by the smartcard terminal
-X               Read random number generator seed from PSE-object Random
-Y               Init random number generator seed through keyboard input

Options/parameters for de-enhancement only:

-m <level>       depth of multi PEM body, which is to be de-enhanced (only if pem scan)
-M <level>       depth of multi PEM body, up to which is to be de-enhanced (only if pem scan)
-u <update>      mode for updating the PSE or CA-database after scanning a PEM-msg
                 (ask, yes, no, cadb, pse (default: ask))
-I               store scanned issuer certificates as trusted key in PKList (if update mode set)
-y <issuers>     DNames or alias-names of issuers of CRLs or CRL-RRs (only if pem crl or pem clr-rr)

Options/parameters for enhancement only:

-C               generate PEM-header with Originator-Certificate and all Issuer-Certificates
                 (default: generate PEM-header with Originator-ID-Asymmetric)
-r <recipients>  DNames or alias-names of recipients (only if pem encrypted)
-n               don't insert Key-Info header field for originator (only if pem encrypted)
-l               assume <inputfile> as RFC822-formatted. Enhance only the body (after the first blank
                 line). In case of ENCRYPTED, extract recipient names from "To:" and "Cc:" header 
                 fields.
-L               like -l. Ask for the correct name of unknown recipients.
-H <mic-alg>     MIC algorithm (default: RSA-MD5)
-S <micenc-alg>  MIC encryption algorithm (default: RSA)
-E <msgenc-alg>  Message encryption algorithm (default: DES-CBC) (only if pem encrypted)
-K <dekenc-alg>  DEK encryption algorithm (default: RSA) (only if pem encrypted)

----------------------------------------------------------------------------------------------------

hsh:  Hash Filter


Description:

'hsh' reads <file> and writes its hash value to <hash>. It uses the algorithm
given with parameter -a <alg>. <alg> is the name of an algorithm of type HASH.


usage:

hsh [-vhz] [-a <alg>] [-H <hashinput>] [-p <pse>] [-c <cadir>] [<file> [<hash>] ]

with:

-a <alg>         Name of a hash algorithm (default: md5)
-H <hashinput>   PSE-object or key reference of hash input (sqmodn only)
-p <psename>     PSE name, if <hashinput> is PSE-object (sqmodn only)
-c <cadir>       CA directory, if <hashinput> is PSE-object (sqmodn only)
-v               verbose
-h               Write this help text
-z               control malloc/free behaviour
<file>           Filename of file to be hashed. Stdin, if omitted
<hash>           File where hash value shall be written. Stdout, if omitted

----------------------------------------------------------------------------------------------------

encode  Encode File to RFC 1421 or [0-9,A-F] ASCII Represenation

usage:

encode [-rxhvzVW] [ -i <n> ] [ file [encodedfile] ]

with:

-r               RFC 1421 encoding style (default)
-x               [0-9,A-F] encoding style
-i <n>           insert newlines after n characters (default 64)
-h               write this help text
-z               control malloc/free behaviour
-v               verbose
-V               Verbose
-W               Grand Verbose (for testing only)

----------------------------------------------------------------------------------------------------

decode  decode RFC 1421 or [0-9,A-F] encoded ASCII file to file

usage:

decode [-rxhvzVW] [ encodedfile [ file ] ]

with:

-r               RFC 1421 encoding style (default)
-x               [0-9,A-F] encoding style
-h               write this help text
-v               verbose
-V               Verbose
-W               Grand Verbose (for testing only)
-z               control malloc/free behaviour

----------------------------------------------------------------------------------------------------

algs  Information about algorithms

usage:

algs [-hvzUGV] [-a] [<name>] [-s] [<keyword>] [-k] [<k1> <k2> ... ] [-l] [<quantity>]

with:

-a <name>        Selects a single algorithm with name <name>
-s <keyword>     Selects groups of algorithms. Either one of the AlgTypes (SYM_ENC, ASYM_ENC,
                 HASH, SIG) which selects all algorithms of the given type, or a string
                 which is contained in an algorithm name.
-U               Show performance times of algorithms for sign, verify, encrypt, decrypt
                 and hash, depending on the algorithm type:
                 Signature algorithms: Total, asymmetric encryption and hash time for
                           signing and verifying a given quantity with different
                           keysizes (parameter -k). Asymmetric keys are generated
                           and stored in PSE $(TOP)/lib/.testkeys if they do not exist.
                 Asymetric Encryption algorithms: Encryption and decryption of a block
                           with different keysizes (parameter -k). Asymmetric keys are generated
                           and stored in PSE $(TOP)/lib/.testkeys if they do not exist.
                 Symmetric Encryption algorithms: Encryption of a 100 K quantity
                 Hash algorithms: Hashing of a given quantity
-G               Show key generation times of asymmetric algorithms with
                 different keysizes (parameter -k). Needs time!
-k <k1> <k2> ... Use keysizes k1, k2, ... (default: 512, 640, 756, 1024)
-l <quantity>    Quantity to be signed or hashed in K bytes (default: 100)
-v               verbose
-V               Verbose
-z               Control malloc/free behaviour
-h               Write this help text

----------------------------------------------------------------------------------------------------

secxlock  Locks the local X display using strong authentication with your PSE

usage:

secxlock [-h] [-p <pse>] [-c <cadir>]

with:

-p <psename>     PSE name (default: Environment variable PSE or .pse)
-c <cadir>       name of CA-directory (default: Environment variable CADIR or .ca)
-h               Write this help text

----------------------------------------------------------------------------------------------------


Utilities to Create and Maintain your Personal Security Environment (PSE)
=========================================================================

psecreate: Create User PSE


Description:

'psecreate'  creates a User PSE with one or two asymmetric keypairs on it,
whose public keys are held within self-signed prototype certificates.


usage:

psecreate [-hqvzVWXY] [-p <pse>] [-s <signalg>] [-k <keysize>] [-e <encalg>] [-k <keysize>] [Name]

with:

-p <pse>         PSE name (default: Environment variable PSE or .pse)
-s <signalg>     Signature algorithm (default: rsa)
-k <keysize>     Keysize of RSA signature key
-e <encalg>      Encryption algorithm (default: rsa)
-k <keysize>     Keysize of RSA encryption key
-q               create PSE that contains two RSA keypairs (default: one RSA keypair only)
-h               write this help text
-z               control malloc/free behaviour
-v               verbose
-V               Verbose
-W               Grand Verbose (for testing only)
-X               Read random number generator seed from PSE-object Random
-Y               Init random number generator seed through keyboard input
<Name>           Intended owner of the generated User PSE

----------------------------------------------------------------------------------------------------

sectool: Maintain your PSE
aliastool: Maintain your aliases (Alias sub-tool of sectool)
directorytool: Access Directory (Directory sub-tool of sectool)


Description:

'sectool' is, like psemaint, a maintenance program which can be used by both
certification authority administrators and users for the purpose
of maintaining their PSEs. This includes moving information (e.g. keys,
certificates, revocation lists etc.) from Unix files or a X.500 Directory
into the PSE and vice versa, generating keys, changing PINs, displaying
the content of the PSE, and maintaining the user's aliases. In contrast
to psemaint, which is line-oriented, sectool is an OpenWindows tool.


usage:

sectool [-tADvVWh] [-p <pse>] [-c <cadir>] [-d <dsa name>]
aliastool [-tDvVWh] [-p <pse>] [-c <cadir>] [-d <dsa name>]
directorytool [-tADvVWh] [-p <pse>] [-c <cadir>] [-d <dsa name>]
with:

-p <psename>        PSE name (default: environment variable PSE or .pse)
-c <cadir>          Name of CA-directory (default: environment variable CADIR or .ca)
-t                  control malloc/free behaviour
-v                  verbose
-V                  Verbose
-W                  Grand Verbose (for testing only)
-d <dsa name>       Name of the DSA to be initially accessed (default: locally configured DSA)
-A <authlevel>      Level of authentication used for X.500 Directory access
                    <authlevel> may have one of the values 'SIMPLE' or 'STRONG'
                    (default: environment variable AUTHLEVEL or 'No authentication')
                    STRONG implies the use of signed DAP operations

----------------------------------------------------------------------------------------------------

psemaint: Maintain PSE


Description:

'psemaint' is a maintenance program which can be used by both
certification authority administrators and users for the purpose
of maintaining their PSEs. This includes moving information (e.g. keys,
certificates, revocation lists etc.) from Unix files or a X.500 Directory
into the PSE and vice versa, generating keys, changing PINs and displaying
the content of the PSE.


usage:

psemaint [-hvzACFORDTVW] [-p <pse>] [-c <cadir>] [-a <issueralg>] [-f <notbefore>] [-l <notafter>]
         [-i <inputfile>] [-d <dsa name>] [-t <dsaptailor>] [-A <authlevel>] [cmd]
with:

-p <psename>        PSE name (default: environment variable PSE or .pse)
-c <cadir>          Name of CA-directory (default: environment variable CADIR or .ca)
-i <inputfile>      Scriptfile containing the commands to be executed by 'psemaint'
-a <issueralg>      CA's signature algorithm (default: md2WithRsaEncryption)
-f <notbefore>      First date on which the certificate is valid
                    (evaluated by 'certify' command within 'psemaint')
-l <notafter>       Last date on which the certificate is valid
                    (evaluated by 'certify' command within 'psemaint')
-F                  consider own FCPath as trusted
-O                  RFC 1422 DName subordination not required
-R                  consult PEM revocation lists during verification
-C                  show list of commands available with 'psemaint'
-D                  access Directory (X.500 or AF_DB)
-T                  perform each public key RSA operation in the smartcard  terminal
                    instead of employing the software in the workstation (the latter is the default)
-h                  write this help text
-z                  control malloc/free behaviour
-v                  verbose
-V                  Verbose
-W                  Grand Verbose (for testing only)
-X                  Read random number generator seed from PSE-object Random
-Y                  Init random number generator seed through keyboard input
-d <dsa name>       Name of the DSA to be initially accessed (default: locally configured DSA)
-t <dsaptailor>     Location of dsaptailor file (default: dsaptailor in the ISODE ETCDIR directory)
-A <authlevel>      Level of authentication used for X.500 Directory access
                    <authlevel> may have one of the values 'SIMPLE' or 'STRONG'
                    (default: environment variable AUTHLEVEL or 'No authentication')
                    STRONG implies the use of signed DAP operations
<cmd>               Single command that shall be executed by 'psemaint'
                    (otherwise, commands can be provided interactively
                    or are read from file <inputfile> (see option -i))

----------------------------------------------------------------------------------------------------

instpkroot: Install Public Root Information on PSE


Description:

'instpkroot' reads file <pkroot> or stdin, if <pkroot> is omitted,
and installs its content as PSE object PKRoot on the indicated PSE.
A PKRoot information that already exists on the target PSE will be overwritten.


usage:

instpkroot [-hvzVW] [-p <pse>] [-c <cadir>] [pkroot]

with:

-p <psename>     PSE name (default: environment variable PSE or .pse)
-c <cadir>       Name of CA-directory (default: environment variable CADIR or .ca)
-z               control malloc/free behaviour
-h               write this help text
-v               verbose
-V               Verbose
-W               Grand Verbose (for testing only)
<pkroot>         File containing public root information (or stdin, if omitted)

----------------------------------------------------------------------------------------------------

instfcpath: Install Forward Certification Path on PSE


Description:

'instfcpath' reads file <fcpath> or stdin, if <fcpath> is omitted,
and installs its content as PSE object FCPath on the indicated PSE.
A FCPath information that already exists on the target PSE will be overwritten.


usage:

instfcpath [-hvzVW] [-p <pse>] [-c <cadir>] [fcpath]

with:

-p <psename>     PSE name (default: environment variable PSE or .pse)
-c <cadir>       Name of CA-directory (default: environment variable CADIR or .ca)
-z               control malloc/free behaviour
-h               write this help text
-v               verbose
-V               Verbose
-W               Grand Verbose (for testing only)
<fcpath>         File containing FCPath (or stdin, if omitted)

----------------------------------------------------------------------------------------------------

instcert: Install Certificate on PSE


Description:

'instcert' reads file <cert> or stdin, if <cert> is omitted, and installs its
content on the PSE.


usage:

instcert [-aehrvzVWHD] [-p <pse>] [-c <cadir>] [-d <dsa name>] [-t <dsaptailor>] [-A <authlevel>] 
         [cert]

with:

-p <psename>     PSE name (default: environment variable PSE or .pse)
-c <cadir>       Name of CA-directory (default: environment variable CADIR or .ca)
-z               control malloc/free behaviour
-e               consider certificate as ENCRYPTION certificate (default: SIGNATURE certificate)
-h               write this help text
-v               verbose
-V               Verbose
-W               Grand Verbose (for testing only)
-D               store certificate in Directory (X.500 or AF-DB)
-r               replace existing certificate in AF-DB Directory
-H               install certificate as hierarchy certificate
                 (default: add certificate to set of cross certificates)
-d <dsa name>    Name of the DSA to be initially accessed (default: locally configured DSA)
-t <dsaptailor>  Location of dsaptailor file (default: dsaptailor in the ISODE ETCDIR directory)
-a               store certificate as cACertificate attribute value (default: userCertificate)
-A <authlevel>   Level of authentication used for X.500 Directory access
                 <authlevel> may have one of the values 'SIMPLE' or 'STRONG'
                 (default: environment variable AUTHLEVEL or 'No authentication')
                 STRONG implies the use of signed DAP operations
<cert>           File containing the certificate (or stdin, if omitted)

----------------------------------------------------------------------------------------------------

genkey: Generate Key and Prototype Certificate


Description:

'genkey' generates an asymmetric keypair and installs the secret component on the PSE.
The public component of the keypair is wrapped into a self-signed prototype certificate
which is stored on the PSE and written to the file <proto> or stdout, if <proto> is omitted.


usage:

genkey [-hrvzVWXY] [-p <pse>] [-c <cadir>] [-a <issueralg>] [-s <signalg>] [-k <keysize>]
       [-e <encalg>] [-k <keysize>] [proto]

with:

-p <pse>           PSE name (default: environment variable PSE or .pse)
-c <cadir>         Name of CA-directory (default: environment variable CADIR or .ca)
-a <issueralg>     Issuer algorithm associated with the signature of the prototype certificate
                   (default: md2WithRsaEncryption)
-s <signalg>       Signature algorithm (default: rsa)
-k <keysize>       Keysize of RSA signature key
-e <encalg>        Encryption algorithm (default: rsa)
-k <keysize>       Keysize of RSA encryption key
-r                 replace a previously generated secret key
-h                 write this help text
-z                 control malloc/free behaviour
-v                 verbose
-V                 Verbose
-W                 Grand Verbose (for testing only)
-X                 Read random number generator seed from PSE-object Random
-Y                 Init random number generator seed through keyboard input
<proto>            File containing the resulting prototype certificate (or stdout, if omitted)

----------------------------------------------------------------------------------------------------

getkey: Build Prototype Certificate


Description:

'getkey' creates a prototype certificate from a public key previously
stored on the PSE, and writes its content to file <proto> or stdout,
if <proto> is omitted.


usage:

getkey [-ehsvzVW] [-p <pse>] [-c <cadir>] [proto]

with:

-p <pse>           PSE name (default: Environment variable PSE or .pse)
-c <cadir>         Name of CA-directory (default: Environment variable CADIR or .ca)
-s                 build prototype certificate from public signature key (default)
-e                 build prototype certificate from public encryption key
-h                 write this help text
-z                 control malloc/free behaviour
-v                 verbose
-V                 Verbose
-W                 Grand Verbose (for testing only)
<proto>            File containing the resulting prototype certificate (or stdout, if omitted)

----------------------------------------------------------------------------------------------------

pkadd: Download Public Key into Cache


Description:

'pkadd' retrieves the certificates of owner 'Name' from the Directory
and downloads the requested public key information into the cache of
trusted public keys (PKList or EKList) in the indicated PSE.


usage:

pkadd [-aehivzVW] [-p <pse>] [-c <cadir>] [-d <dsa name>] [-t <dsaptailor>] [-A <authlevel>] [Name]

with:

-p <psename>     PSE name (default: environment variable PSE or .pse)
-c <cadir>       Name of CA-directory (default: environment variable CADIR or .ca)
-e               consider ENCRYPTION certificates only
                 (default: consider SIGNATURE certificates only)
-h               write this help text
-i               let user specify certificate whose public key is to be downloaded
-t               control malloc/free behaviour
-v               verbose
-V               Verbose
-W               Grand Verbose (for testing only)
-d <dsa name>    Name of the DSA to be initially accessed (default: locally configured DSA)
-t <dsaptailor>  Location of dsaptailor file (default: dsaptailor in the ISODE ETCDIR directory)
-a               read cACertificate attribute (default: userCertificate)
-A <authlevel>   Level of authentication used for X.500 Directory access
                 <authlevel> may have one of the values 'SIMPLE' or 'STRONG'
                 (default: environment variable AUTHLEVEL or 'No authentication')
                 STRONG implies the use of signed DAP operations
<Name>           Owner of the certificate whose public key is downloaded into the cache

----------------------------------------------------------------------------------------------------

pkdel  Remove Public Key from Cache


Description:

'pkdel' deletes entries from the cache of trusted public keys (PKList or
EKList). It either deletes all entries of the given <owner>, or the one entry
that is uniquely identified by its <issuer> and <serial> combination.


usage:

pkdel [-ehvzVW] [-p <pse>] [-c <cadir>] [-o <owner>] [-i <issuer>] [-n <serial>]

with:

-p <psename>     PSE name (default: Environment variable PSE or .pse)
-c <cadir>       Name of CA-directory (default: Environment variable CADIR or .ca)
-o <owner>       Owner of public key
-i <issuer>      Issuer of public key
-n <serial>      Serial number of public key
-e               remove public key from cache of public ENCRYPTION keys (EKList)
-z               control malloc/free behaviour
-h               write this help text
-v               verbose
-V               Verbose
-W               Grand Verbose (for testing only)

----------------------------------------------------------------------------------------------------

pklist: Print Cache of Trusted Public Keys


Description:

'pklist' prints out the content of the cache of trusted public keys
(PKList or EKList) of the indicated PSE.


usage:

pklist [-ehvzVW] [-p <pse>] [-c <cadir>]

with:

-p <psename>     PSE name (default: environment variable PSE or .pse)
-c <cadir>       Name of CA-directory (default: environment variable CADIR or .ca)
-e               print cache of trusted public ENCRYPTION keys (EKList)
                 (default: PKList)
-z               control malloc/free behaviour
-h               write this help text
-v               verbose
-V               Verbose
-W               Grand Verbose (for testing only)

----------------------------------------------------------------------------------------------------

showdir: Retrieve and Show Security Attributes from Directory


Description:

'showdir' reads a security attribute from the directory entry of 'Name'
and prints its contents in an appropriate format.


usage:

showdir [-aehvzVW] [-p <pse>] [-c <cadir>] [-o <attributeType>] [-d <dsa name>] [-t <dsaptailor>]
        [-A <authlevel>] [Owner's Name].

with:

-p <psename>        PSE name (default: environment variable PSE or .pse)
-c <cadir>          Name of CA-directory (default: environment variable CADIR or .ca)
-e                  consider ENCRYPTION certificates only
-o <attributeType>  Attribute whose value is requested (default: certificate)
                    supported attribute types:
                    'cert' (certificate),
                    'cross' (cross certificate pair), and
                    'rev' (PEM revocation list)
-h                  write this help text
-z                  control malloc/free behaviour
-v                  verbose
-V                  Verbose
-W                  Grand Verbose (for testing only)
-d <dsa name>       Name of the DSA to be initially accessed (default: locally configured DSA)
-t <dsaptailor>     Location of dsaptailor file (default: dsaptailor in the ISODE ETCDIR directory)
-a                  read cACertificate attribute (default: userCertificate)
-A <authlevel>      Level of authentication used for X.500 Directory access
                    <authlevel> may have one of the values 'SIMPLE' or 'STRONG'
                    (default: environment variable AUTHLEVEL or 'No authentication')
                    STRONG implies the use of signed DAP operations
<Name>              Name of directory entry to be accessed

----------------------------------------------------------------------------------------------------

show  Show ASN.1-coded SecuDE Object in Suitable Form

usage:

show [-hvzVW] [file (containing ASN.1 code)]

with:

-h               write this help text
-v               verbose
-V               Verbose
-W               Grand Verbose (for testing only)
-z               control malloc/free behaviour

----------------------------------------------------------------------------------------------------


Utilities for the Operation of Certification Authorities (X.509)
================================================================

cacreate: Create CA (CA command)


Description:

'cacreate' creates a CA directory containing the CA PSE with one or two
asymmetric keypairs,  self-signed prototype certificate(s), an empty PEM
revocation list, and the SerialNumber on it, and the CA database.


usage:

cacreate [-hrqvzDVWXY] [-p <pse>] [-c <cadir>] [-a <issueralg>] [-s <signalg>] [-k <keysize>]
         [-e <encalg>] [-k <keysize>] [-n <serial>] [-u <nextupdate>] [-f <notbefore>] 
         [-l <notafter>] [-d <dsa name>] [-t <dsaptailor>] [-A <authlevel>] [CA-Name]

with:

-p <psename>       PSE name (default: environment variable CAPSE or .capse)
-c <cadir>         Name of CA-directory (default: environment variable CADIR or .ca)
-a <issueralg>     Issuer algorithm associated with the signature of the prototype certificate(s)
                   (default: md2WithRsaEncryption)
-s <signalg>       Signature algorithm (default: rsa)
-k <keysize>       Keysize of RSA signature key
-e <encalg>        Encryption algorithm (default: rsa)
-k <keysize>       Keysize of RSA encryption key
-n <serial>        Initial value of the serial number to be used by the CA. <serial> must be a
                   string from [0..9,A..F] representing the hexadecimal notation of the number.
-r                 create Root-CA, which means that if the generated PSE holds two keypairs,
                   the corresponding certificates are both signed by the signature key
                   (default: the encryption certificate is signed by the secret decryption key)
-D                 store self-signed certificate(s) in Directory (X.500 or .af-db)
-u <nextupdate>    Time and date of next scheduled CRL update (default: current date + one year)
-f <notbefore>     First date on which self-signed certificate is valid (default: current date).
                   (is only evaluated if option -r was supplied)
-l <notafter>      Last date on which self-signed certificate is valid (default: notbefore + one year)
                   (is only evaluated if option -r was supplied)
-q                 create PSE that contains two RSA keypairs (default: one RSA keypair only)
-t                 control malloc/free behaviour
-h                 write this help text
-v                 verbose
-V                 Verbose
-W                 Grand Verbose (for testing only)
-X                 Read random number generator seed from PSE-object Random
-Y                 Init random number generator seed through keyboard input
-d <dsa name>      Name of the DSA to be initially accessed (default: locally configured DSA)
-t <dsaptailor>    Location of dsaptailor file (default: dsaptailor in the ISODE ETCDIR directory)
-A <authlevel>     Level of authentication used for X.500 Directory access
                   <authlevel> may have one of the values 'SIMPLE' or 'STRONG'
                   (default: environment variable AUTHLEVEL or 'No authentication')
                   STRONG implies the use of signed DAP operations
<CA-Name>          Intended owner of the generated CA PSE

----------------------------------------------------------------------------------------------------

certify: Certify Public Key (CA command)


Description:

'certify' reads a prototype certificate from file <proto> or stdin, if
<proto> is omitted, and transforms it into a 'valid' certificate.
It replaces the 'issuer' and 'serialnumber' fields of the prototype
certificate by its CA values (taken from its CA PSE), and replaces
the signature appended to the prototype certificate by its own signature.
The resulting certificate is written to file <cert> or stdout, if <cert>
is omitted.


usage:

certify [-hvzVW] [-p <pse>] [-c <cadir>] [-a <issueralg>] [-f <notbefore>] [-l <notafter>]
        [proto [cert]]

with:

-p <psename>       PSE name (default: environment variable CAPSE or .capse)
-c <cadir>         Name of CA-directory (default: environment variable CADIR or .ca)
-a <issueralg>     CA's signature algorithm (default: md2WithRsaEncryption)
-f <notbefore>     First date on which the certificate is valid
-l <notafter>      Last date on which the certificate is valid
-z                 control malloc/free behaviour
-h                 write this help text
-v                 verbose
-V                 Verbose
-W                 Grand Verbose (for testing only)
<proto>            File containing the prototype certificate (or stdin, if omitted)
<cert>             File containing the resulting certificate (or stdout, if omitted)

----------------------------------------------------------------------------------------------------

revoke: Revoke one or more Certificates (CA command)


usage:

revoke [-hvzVWD] [-p <pse>] [-c <cadir>] [-a <issueralg>] [-u <nextupdate>] [-d <dsa name>]
       [-t <dsaptailor>] [-A <authlevel>]

with:

-p <psename>     PSE name (default: .capse)
-c <cadir>       name of CA-directory (default: .ca)
-a <issueralg>   Issuer algorithm associated with the signature of the PEM Crl
-u <nextupdate>    Time and date of next scheduled update of PEM revocation list
-z               enable memory checking
-h               write this help text
-v               verbose
-V               Verbose
-W               Grand Verbose (for testing only)
-D               store updated revocation list in Directory (X.500 or .af-db)
-d <dsa name>    Name of the DSA to be initially accessed (default: locally configured DSA)
-t <dsaptailor>  Location of dsaptailor file (default: dsaptailor in the ISODE ETCDIR directory)
-A <authlevel>   level of authentication used for binding to the X.500 Directory

----------------------------------------------------------------------------------------------------

getpkroot  Extract Public Root Information from CA PSE (CA command)


Description:

'getpkroot' extracts the the PSE object PKRoot from the CA's PSE
and writes its content to the file <pkroot> or stdout, if <pkroot>
is omitted.


usage:

getpkroot [-hvzVW] [-p <pse>] [-c <cadir>] [pkroot]

with:

-p <psename>     PSE name (default: environment variable CAPSE or .capse)
-c <cadir>       Name of CA-directory (default: environment variable CADIR or .ca)
-z               control malloc/free behaviour
-h               write this help text
-v               verbose
-V               Verbose
-W               Grand Verbose (for testing only)
<pkroot>         File containing public root information (or stdout, if omitted)

----------------------------------------------------------------------------------------------------

getfcpath: Extract Forward Certification Path from CA PSE (CA command)


Description:

'getfcpath' adds the CA's hierarchy and cross certificates to the CA's
own forward certification path (FCPath), and writes the extended FCPath
to the file <fcpath> or stdout, if <fcpath> is omitted.


usage:

getfcpath [-hvzVW] [-p <pse>] [-c <cadir>] [fcpath]

with:

-p <psename>     PSE name (default: environment variable CAPSE or .capse)
-c <cadir>       Name of CA-directory (default: environment variable CADIR or .ca)
-z               control malloc/free behaviour
-h               write this help text
-v               verbose
-V               Verbose
-W               Grand Verbose (for testing only)
<fcpath>         File containing the extended FCPath (or stdout, if omitted)

----------------------------------------------------------------------------------------------------

gen_pse  Creating and updating of PSE's

usage:

gen_pse            [-i script][-c cadir][-p capse][-H home][-u userunixname]
                   [-a issueralg][-s signalg][-e encalg][-k keysize]
                   [-f notbefore][-l notafter][-x nameprefix]
                   [-P subjectpse][-C caname][-g serialnumber]
                   [-d dsaname] [-t <dsaptailor>] [-n][-vrDqzXYh][namesuffix]
                   
-i <script>        Name of a script file where these options can be set for more than one creation
-c <cadir>         Name of CA-directory (default: environment variable CADIR or .ca)
-p <capse>         CA's PSE name (default: environment variable CAPSE or .capse)
-H <home>          Path of all home-directories
-u <userunixname>  Unixname of the owner of a PSE to create/update
-a <issueralg>     Algorithm to sign certificates with
-s <signalg>       Algorithm of signature key to create/update
-e <encalg>        Algorithm of encryption key to create/update
-k <keysize>       Keysize of key to generate
-f <notbefore>     First date on which the certificate is valid
-l <notafter>      Last date on which the certificate is valid
-x <nameprefix>    First part of the name associated to the PSE
-P <subjectpse>    Name of PSE to create/update (default: environment variable PSE or .pse)
-C <caname>        Create a CA with CA-directory name <caname>
-g <serialnumber>  If a CA is created a serialnumber to start with can be specified
-d <dsaname>       Name of the dsa
-t <dsaptailor>    Location of dsaptailor file (default: dsaptailor in the ISODE ETCDIR directory)
-n                 Read the name of the dsa from PSE
-v                 Verbose
-r                 Replace an existing PSE in case of creation
-D                 Store generated certificates in X500 directory
-q                 Create two different key pairs for signature and encryption
-z                 Check malloc/free behaviour
-X                 Read random number generator seed from PSE-object Random
-Y                 Init random number generator seed through keyboard input
-h                 Write this help text
                   
<namesuffix>      Second part of the name associated to the PSE

----------------------------------------------------------------------------------------------------


Test
====


SecuDE-4.3 						 (c)1993 GMD Darmstadt 
 
create_TestTree 	 Create a test certification tree and test users 
 
usage: 
 
create_TestTree [-v] [-p] [-D] [-z] [-q] 
 
with: 
 
-v 	 verbose 
-p 	 use PEM RFC 1424 certification procedures, 
 	 otherwise use KM utilities for certification 
-D 	 enter certificates into Directory (X.500 or .af-db), 
-z 	 Enable checking of malloc/free behaviour 
-q 	 create PSEs with separate key pairs for SIGNATURE/ENCRYPTION purposes, 
 	 otherwise create 'one key pair' PSEs
 
This shell-script creates, for test purposes, the following tree of 
certification authorities and users: 
 
 			 Root-CA 
 		 / 			 \ 
 	 ORG-1-CA 			 ORG-2-CA 
 	 / 	 \ 			 / 	 \ 
 ORG-1-User-1 ORG-1-User-2 	 ORG-2-User-1 ORG-2-User-2 
 
 
It creates the directory TestTree under the home directory and the following directories 
under TestTree: 
 
Root-CA 	 CA directory of the root ca, CA-PSE is .capse, PIN = test 
ORG-1-CA 	 CA directory of the ca ORG-1-CA under Root-CA, CA-PSE is .capse, PIN = test 
ORG-2-CA 	 CA directory of the ca ORG-2-CA under Root-CA, CA-PSE is .capse, PIN = test 
ORG-1-User-1 	 PSE of user ORG-1-User-1 under ca ORG-1-CA, PIN = test 
ORG-1-User-2 	 PSE of user ORG-1-User-2 under ca ORG-1-CA, PIN = test 
ORG-2-User-1 	 PSE of user ORG-2-User-1 under ca ORG-2-CA, PIN = test 
ORG-2-User-2 	 PSE of user ORG-2-User-2 under ca ORG-2-CA, PIN = test
 
create_test_tree needs about 100 sec on a Sun SPARC10-41


----------------------------------------------------------------------------------------------------
\end{verbatim}
}

\nm{3X}{INTRO}{Introduction to security functions}
\label{intro3}
\hl{Description}
\addtotocsmall{SecuDE Library Functions Summary}
The SecuDE library functions are grouped into five major collections which 
can be identified by the prefex of the function name (sec\_, af\_,
aux\_, pem\_ and e\_/d\_).

%\begin{figure}
\framebox[3.1cm][c]{\rule[-0.5cm]{0cm}{0.9cm}X.500}
\hspace*{0.39cm}
\framebox[3.1cm][c]{\rule[-0.5cm]{0cm}{0.9cm}PEM}
\hspace*{0.4cm}
\framebox[2.9cm][c]{\rule[-0.5cm]{0cm}{0.9cm}Key Managm.}
\hspace*{0.4cm}
\framebox[3.1cm][c]{\rule[-0.5cm]{0cm}{0.9cm}Other Applic.}
\vs{0.3cm}
{\bf
$|$ -- X500-IF -- -- $|$ \hspace*{0.2cm} $|$ -- PEM-IF -- -- $|$ \hspace*{0.3cm} $|$ -- KM-IF -- -- $|$
}
\vs{0.3cm}
\framebox[3.1cm][c]{\rule[-0.5cm]{0cm}{0.9cm}X.500 Support}
\hspace*{0.39cm}
\framebox[3.1cm][c]{\rule[-0.5cm]{0cm}{0.9cm}PEM Support}
\hspace*{0.4cm}
\framebox[2.9cm][c]{\rule[-0.5cm]{0cm}{0.9cm}KM Support}
\vs{0.3cm}
{\bf
$|$ -- -- -- -- -- -- -- -- -- -- -- -- -- -- -- -- -- AF-IF -- -- -- -- -- -- -- -- -- -- -- -- -- -- -- -- $|$
%$|$ -- -- -- -- -- -- -- -- -- -- -- -- AF-IF -- -- -- -- -- -- -- -- -- -- -- $|$
}
\vs{0.3cm}
\framebox[5.5cm][c]{\rule[-0.5cm]{0cm}{0.9cm}Local Certificate Handling}
\hspace*{0.4cm}
\framebox[4.10cm][c]{\rule[-0.5cm]{0cm}{0.9cm}Directory Access}
\hspace*{0.4cm}
\framebox[3.1cm][c]{\rule[-0.5cm]{0cm}{0.9cm}Auxil. Functions}
\vs{0.3cm}
{\bf
$|$ -- -- -- -- -- -- -- -- -- -- -- -- -- -- -- -- Secure-IF -- -- -- -- -- -- -- -- -- -- -- -- -- -- -- $|$
}
\vs{0.3cm}
\framebox[13.8cm][c]{\rule[-0.5cm]{0cm}{0.9cm}Personal Security Environment and Cryptography (technology dependant)}
\vs{0.3cm}
{\bf 
$|$ -- -- -- SCA-IF -- -- -- $|$
}
\vs{0.3cm}
\framebox[4.2cm][c]{\rule[-0.5cm]{0cm}{0.9cm}SC-Environment}
\hspace*{4.05cm}
\framebox[2.5cm][c]{\rule[-0.5cm]{0cm}{0.9cm}Crypto-SW}
\hspace*{0.2cm}
\framebox[2.5cm][c]{\rule[-0.5cm]{0cm}{0.9cm}SW-PSE}
%\vs{0.5cm}
%\caption{Interface Structure}
%\label{bild1}
\\[1em]
\stepcounter{Abb}

{\footnotesize Fig.\arabic{Abb}:
General SecuDE structure}
%\end{figure}
\\ [1em]
The following modules are currently realized (Fig. 1):
\bi
\m {\bf Personal Security Environment and Cryptography (Secure)}. 
   The cryptographic functionality and the PSE functionality
   is strongly coupled in SecuDE. Keys which are being
   used by the cryptographic functions are normally addressed
   through references to PSE-objects. In the smartcard realization
   the PSE is a separated device where cryptographic functions
   are performed. The Secure module provides
   the basic cryptographic functions for the generation and 
   verification of digital signatures, and encryption and decryption 
   of data, and the basic PSE access functions. No higher level
   functionality like certificate processing is available in this module.
   Two PSE realizations are availabe in this module, one of them
   is a particular smartcard realization. Secure-IF, the interface of 
   this module, however, hides different technologies from
   the applications, i.e. Secure-IF claims to be an abstract security
   interface from the application's point of view which allows various
   PSE realizations including smartcard environments without affecting 
   the application's structure.
   The description of the {\em Smartcard Application Interface} (SCA-IF) 
   used in SecuDE is available on request.
   Particularly, the Personal Security Environment and Cryptography 
   module comprises
   \bi
   \m RSA, DSA and DES as basic cryptosystems,
   \m key generation,
   \m different message digest algorithms (SqmodN, MD2, MD4, MD5,
      SHA), 
   \m different signature and encryption algorithms (e.g. PKCS \#1),
   \m routines to access the PSE, handling (create, delete, read, write)
      of PSE objects.
   \ei
   The functions realizing this module are prefixed with sec\_.
\m {\bf Authentication Framework and Certification (AF)}. This module
   adds X.509 certification functionality to SecuDE.
   Both local (i.e. PSE-located) certificates and directory-located
   certificates can be addressed. Together with the Secure module, the
   following functionality is provided:
   \bi
   \m sign, verify, encrypt, and decrypt information using PSE-based or 
      directory-based keys, certificates, certification pathes and 
      certificate revocation lists,
   \m maintain a PSE with the objects defined in the AF 
      module, i.e. maintain certificates, crosscertificate pairs,
      the forward certification path, the public root key,
      certificate revocation lists, and lists of trusted public keys.
   \m check PSE for validity and consistency,
   \ei
   Functions for the support of the X.509 authentication framework are
   prefixed with af\_.
\m {\bf Key Management Support (KM)}. 
   This module provides functions for the ge\-ne\-ra\-tion and distribution
   of keys, certificates,
   cross-certificates, and certification pathes, provides additional
   functionality for the 
   implementation of a Certification Authority (CA), and supports the
   interaction between user and CA
   as well as between CA's.

   Key management support functions are also prefixed with af\_. 
\m {\bf Privacy Enhanced Mail Support (PEM)}. This module covers functions 
   which realize the Internet Specifications RFC 1421 - 1423. 
   The basic idea of PEM is to define document oriented message 
   encipherment and authentication procedures for the protection of 
   messages through the use of 
   end-to-end cryptography between originator and recipient with no special
   processing
   requirements imposed on the message transfer system (in contrast to
   transfer 
   protocol oriented security functionality as with X.400). This makes them
   transparent to the mail 
   transfer systems and applicable for local security services, too. 
   Particularly, this module provides the following functions:
   \bi
   \m enhance messages, i.e. create MIC-CLEAR, MIC-ONLY and
      ENCRYPTED message formats,
   \m de-enhance messages, i.e. scan MIC-CLEAR, MIC-ONLY and
      ENCRYPTED message formats, and perform the necessary
      validation steps (including CRL checks)
   \m create Certification Request, Certification Reply, CRL Storage
      Request and CRL Storage Reply message formats,
   \m scan Certification Request, Certification Reply, CRL Storage
      Request and CRL Storage Reply message formats, and perform
      appropriate actions.
   \ei
   Functions for PEM support are prefixed with pem\_.
\m {\bf ASN.1 Encoding/Decoding Functions}
   ASN.1 encoding and decoding functions are available 
   for all SecuDE information objects which are intended for external
   use (i.e. which are either stored externally, for instance in the
   PSE, or exchanged via communication protocols). 
   The encoding/decoding functions
   transform from the internal C-structures of the objects to
   corresponding (and in many cases standardized) ASN.1 codes
   and vice-versa. The encoding functions generally apply distinguished encoding 
   (DER) according to X.509 clause 8.7 in order to be able to process 
   digitally signed objects.

   ASN.1 encoding functions are prefixed with e\_, ASN.1 decoding functions 
   are prefixed with d\_.
\m {\bf Auxiliary Functions}
   A number of auxiliary functions are available for general functions,
   for instance
   \bi
   \m handling of distinguished names, algorithm identifiers, object
      identifiers and similar basic elements,
   \m UTCTime handling functions,
   \m error handling,
   \m print routines
   \m alias naming functions
   \ei
   Auxiliary functions are prefixed with aux\_.
\ei 

{\bf Keys, Algorithm Identifiers and Object Identifiers}
\addtotocsmall{Keys, Algorithm Identifiers and Object Identifiers}
\\ [1em]
All keys are processed in a type {\em KeyInfo} (see SEC(5)) which holds a 
bit representation of the 
key plus its algorithm identifier. An algorithm identifier
is a pair of object identifier (which identifies the algorithm) and 
algorithm dependant parameter set of that
algorithm. An object identifier is a structured set of nonnegative integers 
which represent a node or leaf
in a tree of objects (i.e. of algorithms in our case).
{\small
\bvtab
\4 typdef struct \{ \\
\6	AlgId     \2 *subjectAI; \\
\6      BitString \2 subjectkey; \\
\4 \} KeyInfo; \\
\evtab
}
with
{\small
\bvtab
\4 typedef struct \{ \\
\6      ObjId  \2  *objid; \3 /* object identifier */ \\
\6      char   \2  *parm;  \3 /* alg-specific parameters */ \\
\4 \} AlgId; \\
\evtab
}
{\small
\bvtab
\4 typedef struct \{ \\
\6      int           \3  oid\_nelem;     \3 /* number of integers */ \\
\6      unsigned int  \3  *oid\_elements; \3 /* array of integers  */ \\
\4 \} ObjId; \\
\evtab
}
{\small
\bvtab
\4 typedef struct \{ \\
\6      unsigned int \3   nbits;  \3 /* number of bits */ \\
\6      char         \3  *bits;   \3 /* bitstring */ \\
\4 \} BitString; \\
\evtab
}
Thus each key is linked to the algorithm to be used. The algorithm 
identifier is assigned
during key generation. 
\\ [1em]
In case of RSA keys, the BitString {\em subjectkey} consists of the DER 
code (see {\em Encoding of PSE Objects}) of the type {\em KeyBits}.
{\small
\btab
\2      typedef struct \{ \\
\3	        OctetString \2 part1; \\
\3      	OctetString \2 part2; \\
\3      	OctetString \2 part3; \\
\3      	OctetString \2 part4; \\
\2      \} KeyBits; \\
\etab
}
Part3 and part4 are unused in case of RSA keys.
Part1 and part2 contain the prime numbers $p$ and $q$ in case of secret keys, and the 
modulus $m$ and the public exponent $e$ in case of public keys. OctetString is a similar
type like BitString dealing with octets instead of bits. 
\\ [1em]
In case of DSAkeys, the BitString {\em subjectkey} consists of the DER 
code  of the type {\em KeyBits}, too.
Part1 to part4 contain $y$, $p$, $q$ and $g$ in case of secret DSA keys, and  
$x$, $p$, $q$ and $g$ in case of public DSA keys. 
\\ [1em]
Keys (i.e. {\em KeyInfo's}) are supposed to reside either in PSEs or in security boxes
which are separate devices attached to the Unix system. Both functionality can be modelled by
the sec\_* functions. If keys have to be passed or identified via
parameters of the sec\_* functions, this is normally done with the type 
{\em Key}:
{\small
\btab
\2 typedef struct \{ \\
\3      KeyInfo  \1 *key; \\
\3      KeyRef   \1  keyref; \\
\3      PSESel    \1 *pse\_sel; \\
\2 \} Key;
\etab
}
This parameter type allows either to pass the key in {\em key}, or to refer 
to a key reference
{\em keyref} in a security box (modelled in SecuDE through a per user key 
pool), 
or to refer to a PSE object via {\em pse\_sel} (see {\em PSE Objects} 
below). 
Only one of the three parts of this type must be non zero. 
\\ [1em]
{\bf Cryptographic Algorithms}
\addtotocsmall{Cryptographic Algorithms} 
\\ [1em]
In most cases algorithms are identified by algorithm identifiers, in some 
cases by names (i.e.
NULL terminated character strings being used in the pem environment). 
In addition, addresses of predefined AlgId structures
are also available as externals (through including {\em secure.h}). The 
following algorithms are available :
\bc
{\scriptsize
\begin{tabular}{|l|l|l|c|l|} \hline
{\bf Name}      & {\bf AlgId *}      &  {\bf Parameter}  & {\bf Object Identifier}    & {\bf Type}   \\ \hline
desECB          & desECB             &  --               & \{ 1 3 14 3 2 6 \}         & SYM\_ENC \\
desCBC          & desCBC             &  IV zeroes        & \{ 1 3 14 3 2 7 \}         & SYM\_ENC \\
desECB3         & desEDE             &  --               & \{ 1 3 14 3 2 17 \}        & SYM\_ENC \\
desCBC3         & desCBC3            &  IV zeroes        & \{ 1 3 36 3 1 11 \}        & SYM\_ENC \\
desCBC\_pad     & desCBC\_pad        &  IV zeroes        & \{ 1 3 36 3 1 5 \}         & SYM\_ENC \\
desCBC3\_pad    & desCBC3\_pad       &  IV zeroes        & \{ 1 3 36 3 1 13 \}        & SYM\_ENC \\
DES-ECB         & desECB             &  --               & \{ 1 3 14 3 2 6 \}         & SYM\_ENC \\
DES-CBC         & desCBC             &  IV zeroes        & \{ 1 3 14 3 2 7 \}         & SYM\_ENC \\
DES-EDE         & desEDE             &  --               & \{ 1 3 14 3 2 17 \}        & SYM\_ENC \\ IDEA            & idea               &  --               & \{ 1 3 36 3 1 30 \}        & SYM\_ENC \\
rsa             & rsa                &  Keysize 512      & \{ 2 5 8 1 1 \}            & ASYM\_ENC \\
RSA             & rsaEncryption      &  --               & \{ 1 2 840 113549 1 1 1 \} & ASYM\_ENC \\
dsa             & dsa                &  --               & \{ 1 3 14 3 2 12 \}        & ASYM\_ENC \\
dsaSK           & dsaSK              &  Keysize 512      & \{ 1 3 36 3 1 20 \}        & ASYM\_ENC \\
NIST-DSA        & dsa                &  --               & \{ 1 3 14 3 2 12 \}        & ASYM\_ENC \\
sqmodn          & sqmodn             &  Blocksize 512    & \{ 2 5 8 2 1 \}            & HASH \\
md2             & md2                &  --               & \{ 1 2 840 113549 2 2 \}   & HASH \\
md4             & md4                &  --               & \{ 1 2 840 113549 2 4 \}   & HASH \\
md5             & md5                &  --               & \{ 1 2 840 113549 2 5 \}   & HASH \\
sha             & sha                &  --               & \{ 1 3 14 3 2 18 \}        & HASH \\
RSA-MD2         & md2                &  --               & \{ 1 2 840 113549 2 2 \}   & HASH \\
RSA-MD5         & md5                &  --               & \{ 1 2 840 113549 2 5 \}   & HASH \\
NIST-SHA        & sha                &  --               & \{ 1 3 14 3 2 18 \}        & HASH \\
sqmodnWithRsa   & sqmodnWithRsa      &  Keysize 512      & \{ 2 5 8 3 1 \}            & SIG \\
md2WithRsa      & md2WithRsa         &  --               & \{ 1 3 14 7 2 3 1 \}       & SIG \\
md4WithRsa      & md4WithRsa         &  --               & \{ 1 3 14 3 2 2 \}         & SIG \\
md5WithRsa      & md5WithRsa         &  --               & \{ 1 3 14 3 2 3 \}         & SIG \\
md2WithRsaEncryption & md2WithRsaEncryption &  --        & \{ 1 2 840 113549 1 1 2 \} & SIG \\
md4WithRsaEncryption & md4WithRsaEncryption &  --        & \{ 1 3 14 3 2 4 \}         & SIG \\
md5WithRsaEncryption & md5WithRsaEncryption &  --        & \{ 1 2 840 113549 1 1 4 \} & SIG \\
md2WithRsaTimeDate   & md2WithRsaTimeDate   &  --        & \{ 1 3 36 3 1 22 \}        & SIG \\
md4WithRsaTimeDate   & md4WithRsaTimedate   &  --        & \{ 1 3 36 3 1 24 \}        & SIG \\
md5WithRsaTimeDate   & md5WithRsaTimedate   &  --        & \{ 1 3 36 3 1 25 \}        & SIG \\
dsaWithSHA      & dsaWithSHA         &  --               & \{ 1 3 14 3 2 13 \}        & SIG \\
dhKeyAgreement  & dhKeyAgreement     &  DH parms         & \{ 1 2 840 113549 1 3 1 \} & KEY\_AGREEMENT \\
dhWithCommonModulus  & dhWithCommonModulus  &  --        & \{ 1 3 14 3 2 16 \}   & KEY\_AGREEMENT \\
\hline
\end{tabular}
}
\ec

\underline{Note:} Some algorithms appear more than once because they can be used under 
different names.
The algorithm identifier uniquely identifies the algorithm. A -- in the parameter field 
indicates an ASN.1 NULL value as parameter. Parameter values shown for algorithms
with parameters are default values. 

{\large\bf DES Algorithms:} 
 
The length of the encryption result is always a multiple of 64 bits. 
 
In case of {\bf desCBC\_pad} and {\bf desCBC3\_pad} 
algorithm, additional encoding which allows 
the decrypting routine ({\em sec\_de\-crypt}/{\em af\_decrypt}) to restore 
the plaintext length is added 
by the encryption routine ({\em sec\_encrypt}/{\em af\_encrypt}) prior to 
encryption. 
If $n$ is the plaintext length,
8 - $n$ $mod$ 8 octets are padded by the encryption routine with
the last octet of the last block containing the number of 
valid octets of that block (between 0 and 7), i.e. if the plaintext 
consists 
of a multiple of eight octets, one extra block is added with the last octet 
set to zero. 
The decryption routine expects this padding. 
 
In case of {\bf desCBC} algorithm, 
padding as defined in RFC 1423 is applied prior to encrypting
which again enables the decryption routine to restore the plaintext length:
\begin{quote}
   The input to the DES CBC encryption process shall be padded to a
   multiple of 8 octets, in the following manner.  Let n be the length
   in octets of the input.  Pad the input by appending 8-(n mod 8)
   octets to the end of the message, each having the value 8-(n mod 8),
   the number of octets being added.  In hexadecimal, the possible
   paddings are:  01, 0202, 030303, 04040404, 0505050505, 060606060606,
   07070707070707, and 0808080808080808.  All input is padded with 1 to
   8 octets to produce a multiple of 8 octets in length.  The padding
   can be removed unambiguously after decryption.
\end{quote} 
In all other cases,
zeroes are padded prior to encrypting until an eight-octet boundary is 
reached, i.e.
no octets are padded if the plaintext length is a multiple of eight octets.
Measures which allow to restore the original plaintext length must
be taken by the application in this case.
\\ [1em]
The algorithms desEDE, desCBC3 and desCBC3\_pad imply each block
to be encrypted with a double key {\em \{k1, k2\}} in the encryption-
decryption-encryption mode,
i. e. each block is encrypted with k1, afterwards decrypted with k2, and 
then encrypted with k1.
Decryption of a block is done vice versa, i. e. decryption with k1, then 
encryption with k2
and then decryption with k1.
\\ [1em]
{\large\bf RSA Algorithms:}

{\em rsa} is the RSA algorithm as defined in X.509. {\em rsaEncryption}, 
{\em md2WithRsaEncryption}
and {\em md5With\-RsaEncryption} are the algorithms defined in PKCS \#1. 
They use a specific
padding method for the RSA input value. {\em md2WithRsa}, {\em md4WithRsa} 
and {\em md5WithRsa}
are defined by OIW Stable Agreements. Here, the RSA input value is left-
padded with zeroes
prior to RSA encryption. {\em md4WithRsaEncryption} is defined by OIW, but 
uses the
PKCS padding method.

The algorithms {\em md2WithRsaTimeDate}, {\em md4WithRsaTimeDate} and {\em 
md5WithRsaTimeDate}
are SecuDE-defined. They are basicly the same as the corresponding PKCS \#1 
signature algorithms
with the following differences:
\be
\m The rsa input block is coded with BlockType 05 instead of 01.
\m The 13 octets after the BlockType contain the date and time of 
signature, encoded
   as GMT UTCTime, instead of 13 0xFF padding octets as defined in PKCS 
\#1.
\ee

Interpretation of Integer Values

The ASN.1 basic encoding rules [ISO8825] together with the X.509 defined 
DER rules
define the encoding of integer values
which is used throughout this specification. To recall from this
definition the value of an integer will be represented with the minimum
possible number of octets.  The highest significant octet comes first
followed by the next significant octet until the least significant
octet as last one. Each octet consists of 8 bits of which the high
significant bit is the first bit and the least significant bit is the
last one.

The definition of {\em subjectPublicKey} (from X.509 Authentication 
Framework) is
a BitString comprising the ASN.1 encoded type of

\bc
	SEQUENCE \{ INTEGER, INTEGER \}
\ec
\noindent
where the first part represents the modulus and the second the public 
exponent.
For example the following encoding represents the ASN.1 encoding of such a
sequence.

\btab
	\1 SEQUENCE (30) L 47 $=$ \{            \\
	\2      INTEGER (02) L 40 $=$           \\
	\3              {\tt 72430266 95995A0F BB148995 393AC2F8 }\\
	\3              {\tt 6092F33A BAEC832B 42FF0926 78C0271F }\\
	\3              {\tt D94479AD 957E5682 6EB3E228 3C1FC338 }\\
	\3              {\tt 06F1E775 32F9812D 21ABEB13 F412A851 }\\
	\2      INTEGER (02) L 03 $=$           \\
	\3              {\tt 010001    }              \\
	\1 \}
\etab
\noindent
Note: SecuDE considers the components of incoming public keys to be 
unsigned integers;
the sign of the integer will be ignored. One leading zero octet will be 
accepted, too.
Public keys produced by SecuDE will be encoded as SEQUENCE of signed 
integers, i.e.
SecuDE will produce a leading zero octet if the most significant bit of a 
component
is one.
\\ [1em]
Encryption

In order to encrypt arbitrary text the input data to encryption must be
segmented in data blocks of equal length.  The length of every data
block is determined by the $keysize$, i.e. the amount of bits to
represent the modulus $N$.  The largest number $L$ of octets containing
fewer bits than the modulus is then defined as

\begin{eqnarray*}
	L	& = & (keysize-1)/8 \\
	keysize & = & int ( log_2 (N) + 1 )
\end{eqnarray*}

Each data block is interpreted as an unsigned integer of length $L$ octets
according to the rules of ASN.1 encoding of integers, except that leading
zero octets may occur and that the value is unsigned.
The last data block, if incomplete, is padded with zero octets.
Additional data blocks of padding are not used.

The data blocks which represent the integer values 0 or 1 will be handled 
in
a special way since the regular RSA computation would not change these 
values.
In order to avoid these data in the encrypted data block the value
$2^{keysize-1}$ will be added before encryption.

The result $Y$ of encrypting one data block $X$, (after the possible 
modification for
the values 0 or 1), is computed with the public key $e$
\begin{eqnarray*}
	Y = X^e ~(mod ~N)
\end{eqnarray*}
\noindent
and gives at most $keysize$ bits. The encrypted data block is extended
with leading zero bits to build a BIT~STRING of size $keybits$. The 
contents
of the BIT~STRING represents the (unsigned) integer value of $Y$.
The result of encryption is the concatenated bitstring of all encrypted 
data
blocks.
\\ [1em]
Decryption

The decryption of arbitrary text uses data blocks of size $keysize$ bits.
These data blocks are interpreted as integer values $Y$ with the highest
significant bit first.
The result of decrypting one data block is computed with the private key 
$d$
\begin{eqnarray*}
	X = (Y^d ~(mod ~N)) ~(mod ~2^{8L})
\end{eqnarray*}
\noindent
The resulting (unsigned) integer value $X$ is transformed into an 
OCTET~STRING
of size $L$ octets with the assumption that the originating
block had this size.  This includes stripping off any increment for
possible integer values 0 or 1.  The padding which had been added by
the encryption will not be discarded.

The result of decryption is the concatenated octetstring of all decrypted
data blocks.
\\ [1em]
{\bf Signatures} 

The signatures are produced with 2 steps of processing on the input data.
The first step is to produce a hash value of the input data. The signature
itself is then produced with encrypting this hash value with the private
key. It will be assumed, that the resulting hash value corresponds to the
$keysize$ of the signature key, i.e. the hash value will be treated $mod 
~N$
the public modulus.

The length of the resulting signature will be extended to a multiple of
8 bits by prepending leading zero bits, but arbitrary lengthes of
signatures will be accepted for verification.
\\ [1em]
{\bf Verification}

In order to verify a given signature $S$ of a text, the first step is to
produce the hash value $H$ of the input data as it was described in the 
last section.
Then the signature value will be computed with the signers public key $e$ 
\begin{eqnarray*}
	V = S^e ~(mod ~N)
\end{eqnarray*}
\noindent
and compared with the hash value $H$. The verification succeeds, if
\begin{eqnarray*}
	V \equiv H ~(mod ~N).
\end{eqnarray*}
\\ [1em]
{\large\bf DSA Algorithms}

    A DSA public key is a BitString
    which contains the ASN.1 DER code of
    \bvtab
    \4  SEQUENCE \{ \\
    \4  \2       key INTEGER, - - y (public key)   \\
    \4  \2       params DSAParameters OPTIONAL  \\
    \4  \} \\
    \evtab
    with
    \bvtab
    DSAParameters ::= SEQUENCE \{ \\
    \4  \2       prime1 [0] INTEGER, - - p \\
    \4  \2       prime2 [1] INTEGER, - - q \\
    \4  \2       base [2] INTEGER, - - g \\
    \4  \} \\
    \evtab
    
    A DSA secret key is a BitString
    which contains the ASN.1 DER code of
    \bvtab
    \4  SEQUENCE \{ \\
    \4  \2       key INTEGER, - - x (secret key)   \\
    \4  \2       params DSAParameters OPTIONAL  \\
    \4  \} \\
    \evtab
    with
    \bvtab
    DSAParameters ::= SEQUENCE \{ \\
    \4  \2       prime1 [0] INTEGER, - - p \\
    \4  \2       prime2 [1] INTEGER, - - q \\
    \4  \2       base [2] INTEGER, - - g \\
    \4  \} \\
    \evtab


{\bf Square-mod-N}

\noindent
The input data is segmented into data blocks of equal length~~$L$
\begin{eqnarray*}
	 L	& = & (keysize-1)/16 \\
	keysize & = & int(log_2(N)+1)
\end{eqnarray*}
\noindent
Each octet of a data block is split into 2 sedecimal digits comprising
the original octet. Each sedecimal digit is prepended with the sedecimal 
digit
$0xF$ and gives one octet for the hash block $B$. The length of the hash 
block
is then $2L$.
The possible free bits remaining in the hash block - up to keysize - are 
left zero.
The last block, if incomplete, will be padded with binary ones.
The so produced integer value is added to the last hash result $H_i$
and then squared ($mod ~N$) giving a new hash result, the starting value
for the first block is assumed to be zero.
\begin{eqnarray*}
	H_0 & = & 0	\\
	H_i & = & (( H_{i-1} \oplus B_i )^2) ~(mod ~N), i = 1, \ldots
\end{eqnarray*}
The add operation $\oplus$ represents the boolean function exclusive or.
\\ [1em]
{\bf PSE Objects}
\addtotocsmall{PSE Objects} 
\\ [1em]
Some of the structures handled as parameters of the security functions (e. 
g. keys and certificates)
are supposed to reside in the Personal Security Environment (PSE), e.g. in 
smartcards.

PSE objects are handled at the parameter interface of the sec\_* functions 
by means of the type
{\em PSESel} (PSE selector):
{\small
\btab
\2 typedef struct \{ \\
\3      char  \1  *app\_name; \\
\3      char  \1  *pin;      \\
\3      struct \{            \\
\4         char *name;       \\
\4         char *pin;        \\
\3      \}    \1 object;     \\
\3      int    \1  dev;      \\
\3      int    \1  app\_id;   \\
\2 \} PSESel;
\etab
}
This structure allows to identify a PSE object by an application name (e.g. 
in the
case of multifunctional smartcards) and an object name. In case of the SW-
PSE, the
application name is the name of the directory which comprises the PSE (see 
below).

In addition, this structure allows to provide a PIN (Personal Identification 
Number)
both as global PIN for the whole application (or PSE) or as object-specific 
PIN.
In case of the SW-PSE, this PIN is a password which is hashed into a DES 
key. If no
PIN is given when requesting the access to a PSE object, and the requested 
object
is PIN-protected, the PIN will be asked from /dev/tty (in case of the SW-
PSE) or
at the auxiliary device (depending on the smartcard environment). If 
specific
smartcard environments are implemented, the passing of PIN's via sec\_* 
parameters
may be impossible.

PSE objects known by the sec\_* functions are {\em 
KeyInfo} which holds a key
plus algorithm identifier, and {\em PSEToc} which holds the table of 
contents of the
PSE. For more information see SEC-PSE-OBJECTS(4).

PSE objects known by the af\_* functions are, for instance, {\em Certificate} 
which holds an X.509 certificate,
{\em FCPath} which holds a forward certification path (in the sense of 
X.509), {\em PKRoot} which
holds the root verification key (i. e. the public key of the highest 
certification authority),
and {\em PKList} which holds a chained list of public keys. For more 
information see
AF-PSE-OBJECTS(4). \\ [1em] 
{\bf Software Realization of the PSE (SW-PSE)} \\ [1em]
The model of the software realization of the PSE was developed under the 
assumption that
it is necessary to store all security relevant and personal information in 
a multi-user
environment in standard 
background storage media which belong to the file system of the 
corresponding
operating system and which are principally accessible by everyone who
has access to the system. Only the standard access control mechanisms 
provided by the operating system are assumed to be available. But even if 
the
operating system is ``secure'', security problems remain, eg. the problem
of privileged people (system administrators, maintenance) who in general 
have
access to all data. Therefore the only possibility is to use encryption 
techniques
with keys which exist in the mind of the user only, but which do not exist 
in
the system for longer time.

A software PSE is a Unix directory (not to be confused with an X.500 
directory).
It is not necessary to take the multifunctionality of a smartcard into 
consideration.
Each application has its own directory. The name of the directory is the 
name of the
application (e.g. {\em .pse}). If the name of the application does not 
begin with a slash,
the PSE is located in the HOME directory of the user, otherwise the name is 
considered 
as the fully qualified pathname of the PSE.

A file name within the PSE with the suffix {\em .sf} indicates that the
content of the file is encrypted in the following way:
\be
\m A PIN or password which is entered by the user via the keyboard is 
mapped onto a 
   DES key by means of a one-way function (the terms PIN and password are 
used 
   synonymously in this context).
\m The content of {\em object} is stored in the file {\em object}.sf in 
DES-encrypted 
   form using this key (.sf stands for {\em secret file}). This is also 
called ``PIN-encrypted''
   or ``PIN-protected'' in the following.
\m The PIN itself is DES-encrypted with this key, too, and stored either in 
the file {\em object}.pw, 
   if the PIN is object specific, or in the file pse.pw, if the PIN is the 
one for the whole PSE
   (.pw stands for {\em password}).
\m Before reading {\em object} one can recognize PIN-protection by the 
suffix {\em .sf}, verify
   the correctness of the PIN entered by the user through hashing and 
encrypting this PIN and 
   comparing the result with the content of the file {\em object}.pw or 
pse.pw resp., and decrypt the file
   {\em object}.sf afterwards.
\ee
The software realization of the PSE consists of: 
\bi
\m The PSE PIN (optional). \\
   This is the PIN which is valid for the whole PSE. It is stored
   in encrypted form in the file {\em pse.pw}, if it exists, otherwise
   the file {\em pse.pw} does not exist.
\m The table of contents (Toc). \\
   If {\em pse.pw} exists,
   the Toc is contained in encrypted form in the file {\em Toc.sf}, 
   otherwise it is contained in clear in the file {\em Toc}.
\m The PSE owner's name. \\
   If {\em pse.pw} exists,
   the owner's name is contained in encrypted form in the file {\em 
Name.sf}, 
   otherwise it is contained in clear in the file {\em Name}.
\m An arbitrary number of files containing the data objects and optionally 
their 
   object-specific PINs (in encrypted form). 
   Each data object is stored in one file. If the name of the file is {\em 
object} 
   with no {\em .sf} suffix, the object is not PIN protected and stored in 
clear 
   in this file. If the name of the file is {\em object.sf}, the object is 
   PIN-protected and stored in encrypted form in this file.
   If, in this case, a file {\em object.pw} exists, it contains the 
corresponding PIN.
   Otherwise, i.e. if the name of the object
   file has the suffix {\em .sf}, but no corresponding {\em .pw} file 
exists, the data object
   is encrypted with the PSE-PIN. 
\ei
The internal structure and bit representation of the data objects is 
application dependent. 
However, in order to achieve a system independant bit representation, 
predefined PSE objects 
which are necessary when using SEC- and AF-functions are encoded by 
applying
the {\em Basic Enoding Rules for ASN.1} (BER) with X.509 DER restrictions 
to the ASN.1 definitions 
of those objects. This is called DER-code in the following. In addition,
corresponding C-structures are defined for these objects. For each PSE 
object both encoding 
and decoding functions exist which perform
the mapping between the program-internal C-structure and the corresponding 
DER-code and vice versa.
See in E\_SEC(3X) and E\_AF(3X) for the available encoding and decoding 
functions and AF-PSE-Objects(4)
and SEC-PSE-Objects(4) for the corresponding ASN.1-structures and C-
structures. The mechanism
of DER-encoding is used for other purposes, too, e.g. to produce exchange 
formates for communication
protocols (see E\_X400(3X)).

The internal structure and bit representation of the Toc is described in 
SEC-PSE-OBJECTS(4). 
The internal structure and bit representation of the owner's name (Name) 
and other
PSE objects for the use of the AF-functions are described in AF-PSE-
OBJECTS(4).
\\ [1em]
This is a pretty simple model which largely prevents the user from being 
eavesdropped by
others. It does not prevent modifications of the user's PSE by unauthorized
persons, but it provides means to the user to recognize such modifications, 
and to check
the consistancy of his PSE. There is no way, however, to restore a 
forgotten PIN from
the stored PSE data.
\\ [1em]
Example of a software PSE ({\em object1} without PIN-protection, 
{\em object2} object-PIN protected, {\em object3} PSE-PIN protected): 
\\[1em]
{\small
\parbox[t]{4cm}{
{\bf Filename}
}
\parbox[t]{10.8cm}{
{\bf Content}
} \\
\parbox[t]{4cm}{
pse.pw
}
\parbox[t]{10.8cm}{
Encrypted PSE-PIN (of the entire PSE).
} \\
\parbox[t]{4cm}{
Toc.sf
}
\parbox[t]{10.8cm}{
Table of Contents of the PSE, PIN-protected (PSE-PIN).
} \\
\parbox[t]{4cm}{
Name.sf
}
\parbox[t]{10.8cm}{
PSE owner's name, PIN-protected (PSE-PIN).
} \\
\parbox[t]{4cm}{
{\em object1}
}
\parbox[t]{10.8cm}{
Object 1, application-specific, without PIN-protection.
} \\
\parbox[t]{4cm}{
{\em object2}.sf
}
\parbox[t]{10.8cm}{
Object 2, application-specific, PIN-protected (object-PIN).
} \\
\parbox[t]{4cm}{
{\em object2}.pw
}
\parbox[t]{10.8cm}{
Encrypted PIN for {\em object2}.
} \\
\parbox[t]{4cm}{
{\em object3}
}
\parbox[t]{10.8cm}{
Object 3, application-specific, PIN-protected (PSE-PIN).
}
} \\ [1em]
In addition to the PSE, a key pool is provided where keys (which means the 
ASN.1 DER code
of KeyInfo) can be stored and restored by means of key references. 
Technically, the key pool is also a
PSE with the only difference that it has the fixed name .key\_pool in the 
home directory of the user and that a user-specific encryption key is used 
which is compiled into the
software. 
\\ [1em]
{\bf Distinguished Names} 
\\ [1em]
See INTRO(1) for a detailled explanation of distinguished names and
the two representations of them handled by SecuDE.
The functions aux\_Name2DName() and aux\_DName2Name do the mapping
between these two distinguished name representations.
\\ [1em]
{\bf Functions for the support of Internet Privacy Enhanced Mail (PEM)} 
\\ [1em]
The basic idea of {\em PEM} is to define security services
for the protection of texts
which are document oriented (in contrast to transfer protocol oriented)
in that they are transparent to the mail transfer systems.
In summary, an outbound message is subjected to the following
composition of transformations:

\begin{displaymath}
Transmit\_Form=Encode(Encrypt(Canonicalize(Local\_Form)))
\end{displaymath}

``Encrypt'' means cryptographic encryption.
``Encode'' means the printable representation of an arbitrary octet string
(also referred to as ``zoning'').
An encapsulated header portion of so called ``encapsulated header fields''
is added to the transformed text.
It contains encryption and signature control fields inserted in plaintext.
The supported encapsulated header fields are
Proc-Type, DEK-Info, Originator-ID-Asymmetric,
Originator-Certificate, Issuer-Certificate, MIC-Info,
Recipient-ID-Asymmetric, Key-Info (see {\em RFC 1423: 4.6 Summary of
Encapsulated Header Fields}).

A C-language structure called {\em PemInfo}
contains the encryption and signature control informations
and is utilized by the {\em pem\_*} interface functions
in order to process a message text.

{\small
\bvtab
typedef struct \{ \\
\1   Boolean     \4  confidential;\3 /* TRUE iff PEM shall be encrypted */ 
\\
\1   Boolean     \4  clear;       \3 /* TRUE iff PEM shall be not encoded 
*/ \\
\1   Key         \4 *encryptKEY;  \3 /* plain DES-key */ \\
\1   Certificates\4 *origcert;    \3 /* originator certificates */ \\
\1   AlgId       \4 *signAI;      \3 /* signature algorithm id */ \\
\1   RecpList    \4 *recplist;    \3  /* list of recipients' informations 
*/ \\
  \} PemInfo; \\ \\

typedef struct \{ \\
\1   KeyInfo     \4 *key;         \3 /* in PemInfo:encryptKEY: plain DES-
key */ \\
\1   KeyRef      \4  keyref;      \3 /* in PemInfo:encryptKEY: unused */ \\
\1   PSESel      \4 *pse\_sel;    \3 /* in PemInfo:encryptKEY: unused */ \\
  \} Key; \\ \\

typedef struct \{  \\
\1   AlgId       \4 *subjectAI;   \3 /* in PemInfo:encryptKEY: plain DES-IV 
*/ \\
\1   BitString   \4 subjectkey;  \3 /* in PemInfo:encryptKEY: plain DES-key 
*/ \\
   \} KeyInfo; \\ \\

typedef struct reclist \{ \\
\1   Certificate \4 *recpcert;    \3 /* recipient's user certificate */ \\
\1   OctetString \4 *key;         \3 /* RSA-encrypted DES-key */ \\
\1   struct reclist \4 *next;     \3 /* RSA-encrypted DES-key */ \\
   \} RecpList;
\etab
}
The {\em pem\_create} functions map
a plain user text and a related {\em PemInfo} onto
a privacy enhanced mail text.
Inversely, the {\em pem\_scan} functions map
a privacy enhanced mail text onto
a plain user text and a related {\em PemInfo}
and verifies the message integrity.
\\ [1em]
{\bf Include Files}
\\ [1em]
All include files which are indicated in the following subroutine 
descriptions can be found
in the directory \$TOP/secude/src/include, where \$TOP is installation-
dependant, e.g. \$TOP=/usr/local.
Application programs which use SecuDE subroutines should be compiled 
therefore with
option -I\$TOP/secude/src/include.
\\ [1em]
{\bf Binding Application Programs}
\addtotocsmall{Binding Application Programs}
\\ [1em]
The SecuDE library is \$TOP/secude/lib/libsecude.a. 
In addition, SecuDE provides the library \$TOP/secude/lib/libisode.a which 
contains a collection
of necessary ISODE subroutine objects. 
When binding application programs with SecuDE subroutines one should use 
therefore options
-L\$TOP/secude/lib -lsecude -lisode.  
 
At sites which have a full ISODE installation, the original libisode.a can 
be used 
instead of \$TOP/secude/lib/libisode.a when binding application programs.

\hl{Errors}
\addtotoc{Errors}
In case of an error ({\em Return Value} -1, if return parameter type  RC, 
or NULL pointer else),
the error cause is indicated in the global variable {\em err\_stack} (data 
type ErrStack).
\bvtab
\4 extern struct ErrStack \{                 \\
\5        int             \3 e\_number;      \\
\5        char            \3 *e\_text;       \\
\5        char            \3 *e\_addr;       \\
\5        Struct\_No      \3  e\_addrtype;   \\
\5        char	          \3 *e\_proc;       \\
\5        struct ErrStack \3 *next;	     \\
\4 \} *err\_stack;
\evtab
err\_stack is a pointer to a chained list of errors which represents a 
trace of all errors
which occured inside the called routine. The first element in the list is 
th most recent error.
err\_stack can be printed using {\em aux\_fprint\_error()} and freed using 
{\em aux\_free\_error()}
(see aux\_error(3)). 
 
{\em err\_stack}\pf e\_proc is the name of the routine where the error 
occured. 
 
If {\em err\_stack}\pf e\_addr is not NULL, it contains the address of a 
structure which caused
the error, for instance the address of a Certificate which could not be 
verified. 
{\em err\_stack}\pf e\_addrtype is of type Struc\_No and indicates the type 
of the address. 
 
{\em err\_stack}\pf e\_text contains an additional explanatory text 
description of the error situation.
\\ [1em]
{\em err\_stack}\pf e\_number can have the following values: \\ [1em]
\firstbox  {1 EALGID}
\secondbox  {Invalid Algorithm Identifier} \\
\firstbox  {2 EAPP}
\secondbox  {Cannot select application} \\
\firstbox  {3 EAPPNAME}
\secondbox  {Application name does not exist} \\
\firstbox  {4 ECREATEAPP}
\secondbox  {Cannot create application (e.g. application name already 
exists)} \\
\firstbox  {5 EOBJ}
\secondbox  {Cannot select object} \\
\firstbox  {6 EOBJNAME}
\secondbox  {Object does not exist} \\
\firstbox  {7 ECREATEOBJ}
\secondbox  {Cannot create object (e.g. object already exists)} \\
\firstbox  {8 EPIN}
\secondbox  {Invalid PIN} \\
\firstbox  {9 EVERIFICATION}
\secondbox  {Verification unsuccessful} \\
\firstbox  {10 ESYSTEM}
\secondbox  {System call failed inside routine} \\
\firstbox  {11 EINVALID}
\secondbox  {Invalid argument} \\
\firstbox  {12 EDAMAGE}
\secondbox  {Toc of PSE not readable or PSE inconsistent} \\
\firstbox  {13 EMALLOC}
\secondbox  {Unable to allocate new memory} \\
\firstbox  {14 EDECRYPT}
\secondbox  {Wrong state of sec\_decrypt} \\
\firstbox  {15 EENCRYPT}
\secondbox  {Wrong state of sec\_encrypt} \\
\firstbox  {16 EHASH}
\secondbox  {Wrong state of sec\_hash} \\
\firstbox  {17 EENCODE}
\secondbox  {ASN.1-encoding error} \\
\firstbox  {18 EDECODE}
\secondbox  {ASN.1-decoding error} \\
\firstbox  {19 ESIGN}
\secondbox  {Wrong state of sec\_sign} \\
\firstbox  {20 EVERIFY}
\secondbox  {Wrong state of sec\_verify} \\ 
\firstbox  {100 EOPEN}
\secondbox  {PSE is already open} \\
\firstbox  {101 EVALIDITY}
\secondbox  {Invalid validity date of certificate} \\
\firstbox  {102 EPK}
\secondbox  {PK already exists in PKList} \\
\firstbox  {103 ENAME}
\secondbox  {Name already exists in PKList} \\
\firstbox  {104 ENOPK}
\secondbox  {PK not found in PKList} \\
\firstbox  {105 ENONAME}
\secondbox  {Name not found in PKList} \\
\firstbox  {106 EROOTKEY}
\secondbox  {Highest verification key not available} \\
\firstbox  {107 ENODIR}
\secondbox  {Directory service does not respond} \\
\firstbox  {108 ENAMEDIR}
\secondbox  {No object assigned to this name found in directory} \\
\firstbox  {109 EASCDIR}
\secondbox  {Directory access rights not sufficient for requested 
operation} \\
\firstbox  {110 EATTRDIR}
\secondbox  {No directory entry of requested attribute type found in 
directory} \\
\firstbox {201 ESPACE}
\secondbox {Malloc cannot allocate space} \\
\firstbox {202 EPOINTER}
\secondbox {A pointer wasn't allocated by calling function} \\
\firstbox {203 EINTEGER}
\secondbox {Unexpected integer value} \\
\firstbox {204 ERETURN}
\secondbox {Unexpected return value of a called function} \\
\firstbox {205 EMSGBUF}
\secondbox {Syntax Error of msgbuf} \\
\firstbox {206 EMIC}
\secondbox {Verification of MIC unsuccessful} \\
\firstbox {207 ECTFOWNER}
\secondbox {Certificate owner is not signer of PEM text} \\
\firstbox {208 EMYNAME}
\secondbox {``my name'' missing in list of Recipient-IDs} \\
\firstbox {209 EENCRMIC}
\secondbox {MIC couldn't be encrypted} \\
\firstbox {210 EENCRBODY}
\secondbox {PEM body couldn't be encrypted} \\
\firstbox {211 EDECRMIC}
\secondbox {MIC couldn't be decrypted} \\
\firstbox {212 EDECRBODY}
\secondbox {PEM body couldn't be decrypted} \\
\firstbox {213 EPEMBOUND}
\secondbox {PEM encapsulation boundary lines inconsistent} \\
\firstbox  {401 ELABEL}
\secondbox  {Security label inconsistent or invalid} \\
\firstbox  {402 EPOLICY}
\secondbox  {Unknown security policy} \\
\firstbox  {403 ENOTOKEN}
\secondbox  {According to message security label
	       no message token required;
	       or message sequence number missing on input} \\
\firstbox  {404 ERECIPNAME}
\secondbox  {Recipient name missing or invalid} \\
\firstbox  {405 ERECIPKEY}
\secondbox  {Recipient public key not available} \\
\firstbox  {406 ESECCONTEXT}
\secondbox  {Unacceptable security context} \\
\firstbox  {407 EMAC}
\secondbox  {Message authentication check (verification) failed} \\
\firstbox  {408 ETOKENCRYPT}
\secondbox  {Cannot decrypt (message or bind) token} \\
\firstbox  {409 ETOKSIGN}
\secondbox  {(Message or bind) token verification failed} \\
\firstbox  {410 ETOKRANDOM}
\secondbox  {Unacceptable random number in bind token} \\
\firstbox  {411 ETOKTIME}
\secondbox  {Unacceptable (message or bind) token time} \\
\firstbox  {412 ECREDCERT}
\secondbox  {Credential certificates failure} \\
\firstbox  {413 ECREDKEY}
\secondbox  {Unverifiable partner public key in
	       bind token of credentials} \\

\hl{Flaws and anticipated changes}
A certain amount of control over the security operations of the SecuDE
functions is done via global variables. This is for historic reasons.
It is not very well structured
and bears a risk of unexpected behaviour and side effects. A later version of SecuDE
will introduce the concept of a  ``security context'' which will have 
to be established prior to security operations. A security-context-id will have to be passed
to the security functions as parameter. This will be a considerable change
to parts of the API.

The error handling of SecuDE is sometimes not satifying. Error messages and codes
are sometimes nothing saying or misleading. Expect to find a major reshuffle
of error codes and error messages in later versions.

\nm{4}{INTRO}{Introduction to File and Object Formats}
\label{intro4}
\hl{Description}
In order to achieve a system independant bit representation, all security 
objects 
which are stored on files or exchanged via communication protocols are 
defined as ASN.1 structures and encoded by applying
the {\em Basic Enoding Rules for ASN.1} (BER) with X.509 DER restrictions 
to these 
ASN.1 structures. This is simply called DER code in the following. 

Within a C-program those objects have corresponding C-structure 
representations.

The following section describes all objects of the PSE, presenting their 
ASN.1 definitions
and the corresponding C-structures, and all objects which are exchanged via 
the X.400-88
protocols.

The available encoding and decoding functions which provide the mapping 
between the 
program-internal C-structures and the corresponding ASN.1 DER-code and vice 
versa are described
in E\_SEC(3X), E\_AF(3X), and E\_X400(3X).

\nm{5}{INTRO}{Introduction to miscellaneous facilities}
\label{intro5}
\hl{Description}
This section mainly contains all C type and data definitions which are 
contained
in the include files $<$af.h$>$, $<$pem.h$>$ and$<$secure.h$>$.

