\subsection{Authentication Framework Functions}         

\subsubsection{af\_pse\_open}
\nm{3X}{af\_pse\_open}{Open PSE with AF objects}
\label{af_pse_open}
\hl{Synopsis}
\#include $<$af.h$>$

PSESel {\bf *af\_pse\_open} {\em (object, createflag)} \\
ObjId {\em *object} \\
Boolean {\em createflag}
\hl{Description}
{\em af\_pse\_open()} opens the PSE identified by the global structure {\em AF\_pse}.
It calls sec\_open({\em pse\_sel}) (see sec\_open(3X)) 
with {\em pse\_sel} filled from the global structure {\em AF\_pse}.
The main purpose is to check the existence of the PSE or PSE object and
to provide access to it through PIN check.
{\em object} is the object identifier of the AF object to be opened.
If {\em object} is NULL, the entire PSE is opened by setting pse\_sel\pf object.name
to NULL prior to the call of sec\_open().
If {\em object} is not NULL and {\em createflag} is TRUE, the object identified by {\em object}
is created if it does not exist.
 
In case of success, {\em af\_pse\_open} returns the address of the PSESel structure which came back from
the call to sec\_open, and {\em AF\_pse.pin}, {\em AF\_pse.app\_id} and, in case of
{\em object} is not NULL, {\em AF\_pse.object[..].pin} are set to those values returned
by sec\_open(). The memory of the returned PSESel structure can be released using free().
If {\em af\_pse\_open()} is not successful, i.e. if the object does not exist or the
PIN check fails, NULL is returned.
\\ [1em]
If the PSE or object is already open, sec\_open (which does the PIN check) is not called, 
and only the corresponding PSESel structure is returned, i.e. the PSE or an object
can be opened more than once, but PIN checking is done only the first time. 
{\em AF\_pse.pin} is used as indicator whether the PSE is already open. If {\em AF\_pse.pin} is not NULL,
the PSE is considered to be already open. {\em af\_pse\_open()} sets {\em AF\_pse.pin} to the value returned
by sec\_open, which is either the address of the PIN if the connected PSE returns the PIN (the
SW-PSE returns the PIN, smartcard based PSEs probably not), or a pointer to a NULL terminated
string if the PSE or object does not require a PIN or the PSE does not return the PIN.
AF\_pse.pin is reset to NULL by {\em af\_pse\_close()}.
 
Similarly, {\em AF\_pse.object[..].pin} of the object identified by {\em object} is used as
indicator whether the object is already open or not.
\\ [1em]
All af\_pse routines which access PSE objects
open the corresponding object using af\_pse\_open(), but they do not
close the object. This allows to access several PSE objects within one program (or process)
and do PIN checking only once when accessing the first time. If PIN checking is required explicitly,
af\_pse\_close() should be used before accessing an object.

\subsubsection{af\_pse\_create}
\nm{3X}{af\_pse\_create}{Create PSE with AF objects}
\label{af_pse_create}
\hl{Synopsis}
\#include $<$af.h$>$

PSESel {\bf *af\_pse\_create} {\em (object)} \\
ObjId {\em *object}
\hl{Description}
{\em af\_pse\_create()} creates the PSE identified by the global structure {\em AF\_pse}.
It calls sec\_create({\em pse\_sel}) (see sec\_create(3X)) 
with {\em pse\_sel} filled from the global structure {\em AF\_pse}.
{\em object} is the object identifier of the AF object to be created.
If {\em object} is NULL, the entire PSE is created by setting pse\_sel\pf object.name
to NULL prior to the call of sec\_create().
{\em af\_pse\_create()} returns NULL if the object or PSE to be created already exists.
 
In case of success, {\em af\_pse\_create} returns the address of the PSESel structure which came back from
the call to sec\_create, and {\em AF\_pse.pin}, {\em AF\_pse.app\_id} and, in case of
{\em object} is not NULL, {\em AF\_pse.object[..].pin} are set to those values returned
by sec\_create(). The memory of the returned PSESel structure can be released using free().

\subsubsection{af\_pse\_close}
\nm{3X}{af\_pse\_close}{Close PSE with AF objects}
\label{af_sc_close}
\hl{Synopsis}
\#include $<$af.h$>$ 

RC {\bf af\_pse\_close} {\em (object)} \\
ObjId {\em *object}
\hl{Description}
af\_pse\_close calls sec\_close({\em pse\_sel}) (see sec\_close(3X)) 
with {\em pse\_sel} filled from the global structure {\em AF\_pse}. If the PSE
or the object identified by {\em object} is not open (see af\_pse\_open()), -1 is returned.
If {\em object} is NULL, {\em af\_pse\_close()} resets the global variable {\em AF\_pse.pin} to NULL.
Otherwise {\em AF\_pse.object[..].pin} is reset to NULL. A subsequent {\em af\_pse\_open()}
to the same PSE or object will do a PIN check again.

\subsubsection{af\_pse\_get\_Name}
\nm{3X}{af\_pse\_get\_Name}{Get Name from PSE}
\label{af_get_Name}
\hl{Synopsis}
\#include $<$af.h$>$

DName *{\bf af\_pse\_get\_Name} {\em ()} \\
\hl{Description}
af\_pse\_get\_Name returns the distinguished name of the owner of the PSE. 
This DName is extracted from the certificate in PSE object Cert (in case of
a PSE holding one key pair) or SignCert (in case of a PSE holding two key pairs).
\\ [1em]
A call of af\_pse\_open() is not required before calling af\_pse\_get\_Name()
(see af\_pse\_open(3X)).

\subsubsection{af\_pse\_get\_Certificate}
\nm{3X}{af\_pse\_get\_Certificate}{Get Certificate from PSE}
\label{af_get_Certificate}
\hl{Synopsis}
\#include $<$af.h$>$

Certificate *{\bf af\_pse\_get\_Certificate} {\em (type, issuer, serial)} \\
KeyType {\em type}; \\
DName	{\em *issuer}; \\
OctetString	{\em *serial};
\hl{Parameter}
\parname  {{\em type}}
\pardescript  {Type of the certificate: \\
                    SIGNATURE: user certificate of the signature key \\
                    ENCRYPTION: user certificate of the encryption key.} 

\parname  {{\em issuer}}
\pardescript  {name of the issuer of a requested cross
		certificate}

\parname  {{\em serial}}
\pardescript  {serial number of a requested cross
		certificate.} 

\parname  {{\em Return Value}}
\pardescript  {Address of Certificate or NULL pointer.}
\hl{Description}
af\_pse\_get\_Certificate returns the address of a Certificate which contains
either a user certificate of the public signature key 
or a user certificate of the public encryption key, depending on {\em type}. 

If {\em issuer} is given, i.e. {\em issuer} is not NULL, a cross certificate issued
by {\em issuer} is searched from the corresponding certificate set in PSE objects 
{\em SignCSet} or {\em EncCSet} and returned in the return value. 
If {\em serial} is additionally given, a cross
certificate with that serial number is searched. Otherwise, if {\em issuer}
is NULL, the hierarchy certificate from PSE objects {\em SignCert} or {\em EncCert}
is returned.

If the PSE from which the certificate is retrieved is restricted to hold a single RSA key pair
for both signature and encryption purposes, however, the requested certificate is returned
from PSE objects {\em CSet} or {\em Cert}, depending on whether a cross certificate or a hierarchy
certificate is requested. In this case, {\em type} will not be evaluated.

NULL is returned if the requested certificate is not found.
\\ [1em]
A call of af\_pse\_open() is not required before calling af\_pse\_get\_Certificate()
(see af\_pse\_open(3X)).

\subsubsection{af\_pse\_get\_CertificateSet}
\nm{3X}{af\_pse\_get\_CertificateSet}{Get CertificateSet from PSE}
\label{af_get_CertificateSet}
\hl{Synopsis}
\#include $<$af.h$>$

SET\_OF\_Certificate *{\bf af\_pse\_get\_CertificateSet} {\em (type)} \\
KeyType {\em type}; \\\
\hl{Parameter}
\parname  {{\em type}}
\pardescript  {Type of the certificate: \\
                    SIGNATURE: user certificate of the signature key \\
                    ENCRYPTION: user certificate of the encryption key.} 

\parname  {{\em Return Value}}
\pardescript  {Address of CertificateSet or NULL pointer.}
\hl{Description}
af\_pse\_get\_CertificateSet returns the address of a CertificateSet which contains
either all cross certificates of the public signature key 
or all cross certificates of the public encryption key, depending on {\em type}.

The certificate set is stored in the PSE object
{\em SignCSet} or {\em EncCSet}, respectively.

If the PSE from which the certificate set is retrieved is restricted to hold a single RSA key pair
for both signature and encryption purposes, however, the requested certificate set is returned
from PSE object {\em CSet}. In that case, {\em type} will not be evaluated.
\\ [1em]
A call of af\_pse\_open() is not required before calling af\_pse\_get\_CertificateSet()
(see af\_pse\_open(3X)).

\subsubsection{af\_pse\_get\_CertificatePairSet}
\nm{3X}{af\_pse\_get\_CertificatePairSet}{Get Set of Cross Certificate Pairs from PSE}
\label{af_get_CertificatePairSet}
\hl{Synopsis}
\#include $<$af.h$>$

SET\_OF\_CertificatePair *{\bf af\_pse\_get\_CertificatePairSet} {\em ()} \\
\hl{Description}
af\_pse\_get\_CertificatePairSet returns the address of a set which contains all 
cross certificate pairs by which the certification authority owning the PSE and other
certification authorities certify each other's public signature keys.

Each cross certificate pair is composed of two components: one forward and one
reverse certificate. Forward and reverse certificates are related to each other in the 
following way: The owner (subject) of the forward certificate (the certification authority
which is the owner of the PSE) is the issuer of the reverse certificate belonging to the same
pair (and vice versa).

The set of cross certificate pairs is stored in the PSE object {\em CrossCSet}.
\\ [1em]
A call of af\_pse\_open() is not required before calling af\_pse\_get\_CertificatePairSet()
(see af\_pse\_open(3X)).

\subsubsection{af\_pse\_get\_FCPath}
\nm{3X}{af\_pse\_get\_FCPath}{Get forward certification path from PSE}
\label{af_get_FCPath}
\hl{Synopsis}
\#include $<$af.h$>$ 

FCPath *{\bf af\_pse\_get\_FCPath} {\em (name)} \\
DName *{\em name};
\hl{Parameter}
\parname  {{\em name}}
\pardescript  {DName of the certification authority of an intended recipient or NULL pointer.}

\parname  {{\em Return Value}}
\pardescript  {Address of FCPath or NULL pointer.}
\hl{Description}
af\_pse\_get\_FCPath returns the address of an FCPath containing the forward certification path.

The forward certification path is stored on the PSE object
{\em FCPath}.

If {\em name} is the NULL pointer, the complete forward certification path is returned. Otherwise
the DName of an issuer can be specified in {\em name}. In this case a reduced forward 
certification path is returned:
\be
\m The level where the certificate of issuer {\em name} is found contains only this certificate.
\m Below this level all cross certificates are omitted.
\m Above this level all certificates are omitted.
\ee
If {\em name} is specified but not found in the PSE, the NULL pointer is returned.
\\ [1em]
A call of af\_pse\_open() is not required before calling af\_pse\_get\_FCPath()
(see af\_pse\_open(3X)).

\subsubsection{af\_pse\_get\_Certificates}
\nm{3X}{af\_pse\_get\_Certificates}{Get Certificates from PSE}
\label{af_get_Certificates}
\hl{Synopsis}
\#include $<$af.h$>$ 

Certificates *{\bf af\_pse\_get\_Certificates} {\em (type, name)} \\
KeyType {\em type}; \\
DName *{\em name};
\hl{Parameter}
\parname  {{\em type}}
\pardescript  {Type of the user certificate of Certificates: \\
                    SIGNATURE: user certificate of the signature key \\
                    ENCRYPTION: user certificate of the encryption key.} 

\parname  {{\em name}}
\pardescript  {DName of the certification authority of an intended recipient or NULL pointer.}

\parname  {{\em Return Value}}
\pardescript  {Address of Certificates or NULL pointer.}
\hl{Description}
af\_pse\_get\_Certificates returns the address of a Certificates containing an user certificate and the 
related forward 
certification path. The user certificate is either the certificate of the signature key,
or the certificate of the encryption key, depending on {\em type}.

User certificate and forward certification path are both stored
in the PSE objects {\em SignCert} or {\em EncCert}, and {\em FCPath}. If the PSE from which the originator certificate 
is retrieved is restricted to hold a single RSA key pair
for both signature and encryption purposes, however, the user certificate is stored in PSE object {\em Cert}; in this case, 
{\em type} will not be evaluated.

If {\em name} is the NULL pointer, Certificates contains the complete forward certification path. Otherwise
the DName of an issuer can be specified in {\em name}. In this case Certificates containes a reduced forward 
certification path:
\be
\m If {\em name} is the name of the issuer of the user certificate, no forward certification path
   is returned. 
\m Otherwise, the level (of the forward certification path) where the certificate of issuer {\em name} is found
   contains only this certificate.
\m Below this level all cross certificates are omitted.
\m Above this level all certificates are omitted.
\ee
If {\em name} is specified but not found in the PSE, the NULL pointer is returned.
\\ [1em]
A call of af\_pse\_open() is not required before calling af\_pse\_get\_Certificates()
(see af\_pse\_open(3X)).


\subsubsection{af\_pse\_get\_PKRoot}
\nm{3X}{af\_pse\_get\_PKRoot}{Get top level verification key from PSE}
\label{af_get_PKRoot}
\hl{Synopsis}
\#include $<$af.h$>$

PKRoot *{\bf af\_pse\_get\_PKRoot} ()
\hl{Parameter}
\parname  {{\em Return Value}}
\pardescript  {Address of PKRoot or NULL pointer.}
\hl{Description}
af\_pse\_get\_PKRoot returns the address of PKRoot. This contains DName, public key, and a corresponding
serial number of the top level certification authority (highest verification key).
The corresponding serial number is the lowest serial number of all certificates to be 
verified with this key.

Both public key and serial number are available in an actual version and in a version which was
actual before.

The top level verification keys are stored in the PSE object
{\em PKRoot}.
\\ [1em]
A call of af\_pse\_open() is not required before calling af\_pse\_get\_PKRoot()
(see af\_pse\_open(3X)).


\subsubsection{af\_pse\_get\_PKList}
\nm{3X}{af\_pse\_get\_PKList}{Get list of public verification or encryption keys from PSE}
\label{af_get_PKList}
\hl{Synopsis}
\#include $<$af.h$>$

PKList *{\bf af\_pse\_get\_PKList} {\em (type)} \\
KeyType {\em type};
\hl{Parameter}
\parname  {{\em type}}
\pardescript  {Type of the list: \\
                    SIGNATURE: List of known public signature verification keys of communication partners. \\
                    ENCRYPTION: List of known public encryption keys of
                                communication partners.}

\parname  {{\em Return Value}}
\pardescript  {Address of PKList or NULL pointer.}
\hl{Description}
af\_pse\_get\_PKList returns the address of PKList, containing either a chained list of all known
public verification keys or a list of all known public encryption keys of communication partners,
depending on {\em type}.
Each list element is a ToBeSigned structure from the certificate which is issued for this public key.
 
The list is stored on the PSE object {\em PKList} or {\em EKList}, respectively.

If the PSE from which the PKList is retrieved is restricted to hold a single RSA key pair
for both signature and encryption purposes, however, the list is in any case retrieved from PSE oject {\em PKList},
no matter what the value of {\em type} is.
\\ [1em]
A call of af\_pse\_open() is not required before calling af\_pse\_get\_PKList()
(see af\_pse\_open(3X)).


\subsubsection{af\_pse\_get\_PCAList}
\nm{3X}{af\_pse\_get\_PCAList}{Get list of recognized PCA keys from PSE}
\label{af_pse_get_PCAList}
\hl{Synopsis}
\#include $<$af.h$>$

PKList *{\bf af\_pse\_get\_PCAList} {\em ()}
\hl{Parameter}
\parname  {{\em Return Value}}
\pardescript  {Address of PKList or NULL pointer.}
\hl{Description}
af\_pse\_get\_PCAList returns the address of PKList, containing a chained list of all known
Policy Certification Authority (PCA) keys.
Each list element is a ToBeSigned structure from the certificate which is issued for this public key.
 
The list is stored on the PSE object {\em PCAList}.
\\ [1em]
A call of af\_pse\_open() is not required before calling af\_pse\_get\_PCAList()
(see af\_pse\_open(3X)).


\subsubsection{af\_pse\_get\_CrlSet}
\nm{3X}{af\_pse\_get\_CrlSet}{Get set of Revocation Lists from PSE}
\label{af_get_CrlSet}
\hl{Synopsis}
\#include $<$af.h$>$

CrlSet *{\bf af\_pse\_get\_CrlSet} {\em ()} \\
\hl{Description}
af\_pse\_get\_CrlSet returns the address of the set of revocation lists of all
certification authorities known to the owner of the PSE.

Each member of this set represents a revocation list which has been retrieved from 
the Directory (X.500) and stored in the set using a simplified format: It comprises the distinguished
name of the issuer of a revocation list, a {\em nextUpdate} field specifying when the next
revocation list of that issuer is scheduled to be issued, and the serial numbers of all revoked
certificates contained in that revocation list. The {\em nextUpdate} field of a specific 
revocation list informs the owner of the PSE of when next to access the 
Directory in order to maintain the most current revocation list available from a given
certification authority. 

The set of revocation lists is stored in the PSE object {\em CrlSet}.
\\ [1em]
A call of af\_pse\_open() is not required before calling af\_pse\_get\_CrlSet()
(see af\_pse\_open(3X)).


\subsubsection{af\_pse\_get\_SerialNumber}
\nm{3X}{af\_pse\_get\_SerialNumber}{Read actual value of serial number from PSE}
\label{af_get_SerialNumber}
\hl{Synopsis}
\#include $<$af.h$>$

OctetString *{\bf af\_pse\_get\_SerialNumber} {\em ()} \\
\hl{Description}
af\_pse\_get\_SerialNumber returns the content of PSE object SerialNumber, which is used by certification authorities only.

The returned value represents the serial number which will be assigned to the certificate 
that will next be issued by the CA.
\\ [1em]
A call of af\_pse\_open() is not required before calling af\_pse\_get\_SerialNumber()
(see af\_pse\_open(3X)).


\subsubsection{af\_pse\_get\_AliasList}
\nm{3X}{af\_pse\_get\_AliasList}{Get AliasList from PSE}
\label{af_get_AliasList}
\hl{Synopsis}
\#include $<$af.h$>$

AliasList *{\bf af\_pse\_get\_AliasList} {\em ()} \\
\hl{Description}
af\_pse\_get\_AliasList returns the content of PSE object AliasList (which
comprises the alias database of the owner of the PSE).
\\ [1em]
A call of af\_pse\_open() is not required before calling af\_pse\_get\_AliasList()
(see af\_pse\_open(3X)).


\subsubsection{af\_pse\_get\_QuipuPWD}
\nm{3X}{af\_pse\_get\_QuipuPWD}{Read X.500 Password from PSE}
\label{af_get_QuipuPWD}
\hl{Synopsis}
\#include $<$af.h$>$

char *{\bf af\_pse\_get\_QuipuPWD} {\em ()} \\
\hl{Description}
af\_pse\_get\_QuipuPWD returns the content of PSE object QuipuPWD (which
comprises the X.500 password of the PSE owner, used for simple authentication to the
X.500 Directory).
\\ [1em]
A call of af\_pse\_open() is not required before calling af\_pse\_get\_QuipuPWD()
(see af\_pse\_open(3X)).


\subsubsection{af\_pse\_update\_Certificate}
\nm{3X}{af\_pse\_update\_Certificate}{Update Certificate in the PSE}
\label{af_update_Certificate}
\hl{Synopsis}
\#include $<$af.h$>$

RC {\bf af\_pse\_update\_Certificate} {\em (type, cert, hierarchy)} \\
KeyType {\em type}; \\
Certificate *{\em cert}; \\
Boolean {\em hierarchy};
\hl{Parameter}
\parname  {{\em type}}
\pardescript  {Type of the certificate: \\
                    SIGNATURE: user certificate of the signature key \\
                    ENCRYPTION: user certificate of the encryption key.} 

\parname  {{\em cert}}
\pardescript  {user certificate.} \\
\parname  {{\em hierarchy}}
\pardescript  {TRUE if cert is hierarchy certificate, FALSE if cert is cross certificate.} \\
\parname  {{\em Return Value}}
\pardescript  {0 or -1.}
\hl{Description}
af\_pse\_update\_Certificate stores Certificate {\em cert} in the PSE.
\\ [1em]
{\em Cert} contains either a user certificate of the public signature key 
or a user certificate of the public encryption key, depending on {\em type}.
\\ [1em]
The certificate is stored in the PSE object {\em SignCert} or 
{\em EncCert} respectively, if it is a hierarchy certificate, i.e. if it is a certificate
issued by a CA of the own CA tree. This is indicated by {\em hierarchy} $=$ TRUE. 
\\ [1em]
Otherwise, if {\em hierarchy} $=$ FALSE, {\em cert} is a cross certificate of the user's
public signature or encryption key. It is stored in this case in the PSE object {\em SignCSet} or 
{\em EncCSet} respectively.
If {\em cert} does not exist in the appropriate certificate set,
it is added to the set.
\\ [1em]
If the PSE where the certificate is to be stored is restricted to hold a single RSA key pair
for both signature and encryption purposes, however, the certificate is stored in PSE object {\em CSet} or {\em Cert},
depending on whether the certificate is considered a cross certificate or a hierarchy certificate. In that case, {\em type} will not
be evaluated.
\\ [1em]
A call of af\_pse\_open() is not required before calling af\_pse\_update\_Certificate()
(see af\_pse\_open(3X)).

\subsubsection{af\_pse\_update\_FCPath}
\nm{3X}{af\_pse\_update\_FCPath}{Update forward certification path in the PSE}
\label{af_update_FCPath}
\hl{Synopsis}
\#include $<$af.h$>$ 

RC {\bf af\_pse\_update\_FCPath} {\em (fcpath)} \\
FCPath *{\em fcpath};
\hl{Parameter}
\parname  {{\em fcpath}}
\pardescript  {Forward Certification Path.}

\parname  {{\em Return Value}}
\pardescript  {0 or -1.}
\hl{Description}
af\_pse\_update\_FCPath stores FCPath {\em fcpath} in the PSE object
{\em FCPath}.
\\ [1em]
A call of af\_pse\_open() is not required before calling af\_pse\_update\_FCPath()
(see af\_pse\_open(3X)).

\subsubsection{af\_pse\_update\_PKRoot}
\nm{3X}{af\_pse\_update\_PKRoot}{Update top level verification key in the PSE}
\label{af_update_PKRoot}

\hl{Synopsis}
\#include $<$af.h$>$ 

RC {\bf af\_pse\_update\_PKRoot} {\em (pkroot)} \\
PKRoot *{\em pkroot};
\hl{Parameter}
\parname  {{\em pkroot}}
\pardescript  {Top level verification keys (new/old).}

\parname  {{\em Return Value}}
\pardescript  {0 or -1.}

\hl{Description}
af\_pse\_update\_PKRoot stores PKRoot {\em pkroot} in the PSE object {\em PKRoot}.
\\ [1em]
A call of af\_pse\_open() is not required before calling af\_pse\_update\_PKRoot()
(see af\_pse\_open(3X)).

\subsubsection{af\_pse\_update\_PKList}
\nm{3X}{af\_pse\_update\_PKList}{Update list of public verification or encryption keys in the PSE}
\label{af_pse_update_PKList}
\hl{Synopsis}
\#include $<$af.h$>$

RC {\bf af\_pse\_update\_PKList} {\em (type, list)} \\
KeyType {\em type}; \\
PKList *{\em list};
\hl{Parameter}
\parname  {{\em type}}
\pardescript  {Type of the list: \\
                    SIGNATURE: List of known public signature verification keys of communication partners. \\
                    ENCRYPTION: List of known public encryption keys of
                                communication partners.}

\parname  {{\em list}}
\pardescript  {Chained list of ToBeSigned structures containing the public keys.}

\parname  {{\em Return Value}}
\pardescript  {0 or -1.}
\hl{Description}
af\_pse\_update\_PKList stores {\em list}, which is either a PKList or an EKList, 
on the PSE. {\em list} is stored in the PSE object {\em PKList} or {\em EKList}, depending
on {\em type}.
If the PSE where {\em list} is to be stored is restricted to hold a single RSA key pair
for both signature and encryption purposes, however, {\em list} is in any case stored in PSE object {\em PKList}, no
matter what the value of {\em type} is.
\\ [1em]
A call of af\_pse\_open() is not required before calling af\_pse\_update\_PKList()
(see af\_pse\_open(3X)).

\subsubsection{af\_pse\_update\_PCAList}
\nm{3X}{af\_pse\_update\_PCAList}{Update list of recognized PCA keys in the PSE}
\label{af_pse_update_PCAList}
\hl{Synopsis}
\#include $<$af.h$>$

RC {\bf af\_pse\_update\_PCAList} {\em (list)} \\
PKList *{\em pcalist};
\hl{Parameter}
\parname  {{\em pcalist}}
\pardescript  {Chained list of ToBeSigned structures containing the PCA public keys.}

\parname  {{\em Return Value}}
\pardescript  {0 or -1.}
\hl{Description}
af\_pse\_update\_PCAList stores {\em pcalist}, which is a PKList structure of all recognized
PCAs, in the PSE object {\em PCAList}.
\\ [1em]
A call of af\_pse\_open() is not required before calling af\_pse\_update\_PKList()
(see af\_pse\_open(3X)).

\subsubsection{af\_pse\_update\_CrlSet}
\nm{3X}{af\_pse\_update\_CrlSet}{Update set of Revocation Lists in the PSE}
\label{af_update_CrlSet}
\hl{Synopsis}
\#include $<$af.h$>$

RC {\bf af\_pse\_update\_CrlSet} {\em (crlset)} \\
CrlSet *{\em crlset}; \\
\hl{Parameter}
\parname  {{\em crlset}}
\pardescript  {Set of revocation lists.}

\parname  {{\em Return Value}}
\pardescript  {0 or -1.}
\hl{Description}
af\_pse\_update\_CrlSet stores {\em crlset} in the PSE object {\em CrlSet} 
where {\em crlset} is the set of revocation lists of all certification authorities 
known to the owner of the PSE.
\\ [1em]
A call of af\_pse\_open() is not required before calling af\_pse\_update\_CrlSet()
(see af\_pse\_open(3X)).


\subsubsection{af\_pse\_update\_SerialNumber}
\nm{3X}{af\_pse\_update\_SerialNumber}{Update current serial number on the PSE}
\label{af_update_SerialNumber}
\hl{Synopsis}
\#include $<$af.h$>$

RC {\bf af\_pse\_update\_SerialNumber} {\em (serial)} \\
OctetString *{\em serial}; \\
\hl{Parameter}
\parname  {{\em serial}}
\pardescript  {Current serial number.}

\parname  {{\em Return Value}}
\pardescript  {0 or -1.}
\hl{Description}
af\_pse\_update\_SerialNumber stores {\em serial} in the PSE object {\em SerialNumber},
where {\em serial} contains the actual value of the serial number used by the PSE owning certification
authority. 

af\_pse\_update\_SerialNumber is called after a certification authority has issued a
certificate and thereby increased the value of its serial number.
\\ [1em]
A call of af\_pse\_open() is not required before calling af\_pse\_update\_SerialNumber()
(see af\_pse\_open(3X)).


\subsubsection{af\_pse\_add\_CertificatePairSet}
\nm{3X}{af\_pse\_add\_CertificatePairSet}{Add set of cross certificate pairs to the set of cross certificate pairs in the PSE}
\label{af_add_CertificatePairSet}
\hl{Synopsis}
\#include $<$af.h$>$

RC {\bf af\_pse\_add\_CertificatePairSet} {\em (cpairset)} \\
SET\_OF\_CertificatePair *{\em cpairset};
\hl{Parameter}
\parname  {{\em cpairset}}
\pardescript  {Set of cross certificate pairs to be added.}

\parname  {{\em Return Value}}
\pardescript  {0 or -1.}
\hl{Description}
af\_pse\_add\_CertificatePairSet adds the set of cross certificate pairs {\em cpairset}
to the set of cross certificate pairs in the PSE object {\em CrossCSet}. 
If any of the cross certificate pairs in {\em cpairset} already exists in the set of cross
certificate pairs stored as value in the PSE object {\em CrossCSet}, it will not be added there.
If all cross certificate pairs in {\em cpairset} already exist in the required set,
the PSE object {\em CrossCSet} is not updated and -1 is returned; otherwise, the respective 
PSE object is updated. 
\\ [1em]
A call of af\_pse\_open() is not required before calling af\_pse\_add\_CertificatePairSet()
(see af\_pse\_open(3X)).

\subsubsection{af\_pse\_update\_CertificatePairSet}
\nm{3X}{af\_pse\_update\_CertificatePairSet}{Update Set of Cross Certificate Pairs in the PSE}
\label{af_update_CertificatePairSet}
\hl{Synopsis}
\#include $<$af.h$>$

RC {\bf af\_pse\_update\_CertificatePairSet} {\em (cpairset)} \\
SET\_OF\_CertificatePair *{\em cpairset};
\hl{Parameter}
\parname  {{\em cpairset}}
\pardescript  {Set of Cross Certificate Pairs.}

\parname  {{\em Return Value}}
\pardescript  {0 or -1.}
\hl{Description}
af\_pse\_update\_CertificatePairSet stores the set of cross certificate pairs {\em cpairset} 
in the PSE object {\em CrossCSet}.
\\ [1em]
A call of af\_pse\_open() is not required before calling af\_pse\_update\_CertificatePairSet()
(see af\_pse\_open(3X)).


\subsubsection{af\_pse\_update\_AliasList}
\nm{3X}{af\_pse\_update\_AliasList}{Update user-defined aliases in the PSE}
\label{af_update_AliasList}
\hl{Synopsis}
\#include $<$af.h$>$ 

RC {\bf af\_pse\_update\_AliasList} {\em (alist)} \\
AliasList *{\em alist};
\hl{Parameter}
\parname  {{\em alist}}
\pardescript  {List of alias names.}

\parname  {{\em Return Value}}
\pardescript  {0 or -1.}
\hl{Description}

af\_pse\_update\_AliasList selects from the supplied alias list {\em alist} all user-defined aliases, i.e.,
all aliases whose corresponding {\em aliasf} field is set to {\em useralias}. It stores the
resulting alias list in the PSE object {\em AliasList}.
\\ [1em]
A call of af\_pse\_open() is not required before calling af\_pse\_update\_AliasList()
(see af\_pse\_open(3X)).


\subsubsection{af\_pse\_update\_QuipuPWD}
\nm{3X}{af\_pse\_update\_QuipuPWD}{Update X.500 Password in the PSE}
\label{af_update_QuipuPWD}
\hl{Synopsis}
\#include $<$af.h$>$ 

RC {\bf af\_pse\_update\_QuipuPWD} {\em (pwd)} \\
char *{\em pwd};
\hl{Parameter}
\parname  {{\em pwd}}
\pardescript  {X.500 password, used for simple authentication.}

\parname  {{\em Return Value}}
\pardescript  {0 or -1.}
\hl{Description}

af\_pse\_update\_QuipuPWD stores the user's X.500 password, identified by the string {\em pwd}, in the PSE object {\em QuipuPWD}. The characters of {\em pwd} are restricted to those of {\em GraphicString} syntax.  
\\ [1em]
A call of af\_pse\_open() is not required before calling af\_pse\_update\_QuipuPWD()
(see af\_pse\_open(3X)).


\subsubsection{af\_pse\_add\_PK}
\nm{3X}{af\_pse\_add\_PK}{Add public key to the list of public verification or encryption keys in the PSE}
\label{af_add_PK}
\hl{Synopsis}
\#include $<$af.h$>$

RC {\bf af\_pse\_add\_PK} {\em (type, tbs)} \\
KeyType {\em type}; \\
ToBeSigned *{\em tbs};
\hl{Parameter}
\parname  {{\em type}}
\pardescript  {Type of the key: \\
                    SIGNATURE: public signature verification key of a communication partner. \\
                    ENCRYPTION: public encryption key of a 
                                communication partner.}

\parname  {{\em tbs}}
\pardescript  {ToBeSigned part of the certificate to be added.}

\parname  {{\em Return Value}}
\pardescript  {0 or -1.}
\hl{Description}
af\_pse\_add\_PK adds a public key to the list of known 
verification keys (PKList) or to the list of known encryption keys (EKList), depending
on {\em type}.
If the PSE where the public key is to be stored is restricted to hold a single RSA key pair
for both signature and encryption purposes, however, only the PKList will be considered for storing the public key,
no matter what the value of {\em type} is.
PKList and EKList are chained lists of ToBeSigned-structures which contain the
whole ToBeSigned part of the certificate which is issued for this public key.
af\_pse\_add\_PK adds the ToBeSigned-structure specified by {\em tbs} to the list.
It updates the respective PSE object. No update is done and -1 is returned if
either the same issuer/serialnumber combination or the same public key as in {\em tbs}
already exists in the required list.
\\ [1em]
A call of af\_pse\_open() is not required before calling af\_pse\_add\_PK()
(see af\_pse\_open(3X)).


\subsubsection{af\_pse\_add\_PCA}
\nm{3X}{af\_pse\_add\_PCA}{Add a PCA to the list of recognized PCAs in the PSE}
\label{af_add_PCA}
\hl{Synopsis}
\#include $<$af.h$>$

RC {\bf af\_pse\_add\_PCA} {\em (tbs)} \\
ToBeSigned *{\em tbs};
\hl{Parameter}
\parname  {{\em tbs}}
\pardescript  {ToBeSigned part of the PCA's certificate to be added.}

\parname  {{\em Return Value}}
\pardescript  {0 or -1.}
\hl{Description}
af\_pse\_add\_PCA adds a PCA to the list of recognized PCAs. 

The list of recognized PCAs is a PKList structure.
This is a chained list of ToBeSigned-structures which contain the
whole ToBeSigned part of the certificate which is issued for the PCA's public key.
af\_pse\_add\_PCA adds the ToBeSigned-structure specified by {\em tbs} to the list.
It updates the PSE object PCAList. No update is done and -1 is returned if
either the same issuer/serialnumber combination or the same public key as in {\em tbs}
already exists PCAList.
\\ [1em]
A call of af\_pse\_open() is not required before calling af\_pse\_add\_PCA()
(see af\_pse\_open(3X)).


\subsubsection{af\_pse\_add\_CRL}
\nm{3X}{af\_pse\_add\_CRL}{Add Revocation List to the set of Revocation Lists in the PSE}
\label{af_add_CRL}
\hl{Synopsis}
\#include $<$af.h$>$ 

RC {\bf af\_pse\_add\_CRL} {\em (crlpse)} \\
Crl *{\em crlpse}; \\
\hl{Parameter}
\parname  {{\em crlpse}}
\pardescript  {Revocation List to be added.}

\parname  {{\em Return Value}}
\pardescript  {0 or -1.}
\hl{Description}
af\_pse\_add\_CRL adds a revocation list specified by {\em crlpse} to the set
of revocation lists stored in the PSE object {\em CrlSet}. It updates that PSE object. 
If a revocation list with the same issuer as in {\em crlpse} is already contained 
in the required list, it will be overwritten by {\em crlpse} in order to maintain
the most current revocation list available from that issuer.

{\em crlpse} identifies a revocation list with PEM format which has been retrieved from
the Directory (X.500 or .af-db). It has a simplified format in that it only comprises the distinguished
name of the issuer of that revocation list, an indication of its next scheduled update,
as well as the serial numbers of all revoked certificates contained in that revocation list.
 
A revocation list retrieved from the Directory will be added to the set of revocation lists
stored in the PSE only in case of its verification having succeeded. 
\\ [1em]
A call of af\_pse\_open() is not required before calling af\_pse\_add\_CRL()
(see af\_pse\_open(3X)).


\subsubsection{af\_pse\_delete\_CRL}
\nm{3X}{af\_pse\_delete\_CRL}{Remove Revocation List from the set of Revocation Lists in the PSE}
\label{af_delete_CRL}
\hl{Synopsis}
\#include $<$af.h$>$ 

RC {\bf af\_pse\_delete\_CRL} {\em (issuer)} \\
DName *{\em issuer}; \\
\hl{Parameter}
\parname  {{\em issuer}}
\pardescript  {Issuer whose Revocation List is to be removed from PSE object CrlSet.}

\parname  {{\em Return Value}}
\pardescript  {0 or -1.}
\hl{Description}
af\_pse\_delete\_CRL removes the revocation list issued by {\em issuer} from the set
of revocation lists stored in the PSE object {\em CrlSet} and updates that PSE object. 
\\ [1em]
A call of af\_pse\_open() is not required before calling af\_pse\_delete\_CRL()
(see af\_pse\_open(3X)).



\subsubsection{af\_pse\_exchange\_PK}
\nm{3X}{af\_pse\_exchange\_PK}{Exchange entry in the list of trusted public keys (PKList/EKList) from PSE}
\label{af_change_PK}
\hl{Synopsis}
\#include $<$af.h$>$

RC {\bf af\_pse\_exchange\_PK} {\em (type, tbs)} \\
KeyType {\em type}; \\
ToBeSigned *{\em tbs};
\hl{Parameter}
\parname  {{\em type}}
\pardescript  {Type of the key: \\
                    SIGNATURE: public signature verification key of a communication partner. \\
                    ENCRYPTION: public encryption key of a
                                communication partner.} 

\parname  {{\em tbs}}
\pardescript  {ToBeSigned part of the certificate to be exchanged.}

\parname  {{\em Return Value}}
\pardescript  {0 or -1.}
\hl{Description}
af\_pse\_exchange\_PK exchanges a ToBeSigned-entry in the list of known 
verification keys (PKList) or in the list of known encryption keys (EKList), depending
on {\em type}. The entry to be exchanged is identified by the partner name (subject name
contained in {\em tbs}) and, if this is not unique, by the issuer name contained in {\em tbs}.
It updates the respective PSE object.
If the PSE where the ToBeSigned-entry is to be exchanged is restricted to hold a single RSA key pair
for both signature and encryption purposes, however, only the PKList will be considered for exchanging the identified ToBeSigned-entry,
no matter what the value of {\em type} is.
No update is done and -1 is returned if
the entry cannot be found in the required list.
\\ [1em]
A call of af\_pse\_open() is not required before calling af\_pse\_exchange\_PK()
(see af\_pse\_open(3X)).

\subsubsection{af\_pse\_delete\_PK}
\nm{3X}{af\_pse\_delete\_PK}{Delete entry from the list of trusted public keys (PKList/EKList) from PSE}
\label{af_del_PK}
\hl{Synopsis}
\#include $<$af.h$>$ 

RC {\bf af\_pse\_delete\_PK} {\em (type, name, issuer, serial)} \\
KeyType {\em type}; \\
DName *{\em name}; \\
DName *{\em issuer}; \\
OctetString *{\em serial}
\hl{Parameter}
\parname  {{\em type}}
\pardescript  {Type of the key: \\
                    SIGNATURE: public signature verification key of a communication partner. \\
                    ENCRYPTION: public encryption key of a
                                communication partner.}

\parname  {{\em name}}
\pardescript  {DName of the partner.}

\parname  {{\em issuer}}
\pardescript  {DName of the issuer.}

\parname  {{\em serial}}
\pardescript  {Serial number.}

\parname  {{\em Return Value}}
\pardescript  {0 or -1.}
\hl{Description}
af\_pse\_delete\_PK deletes an entry from the list of trusted 
verification keys (PKList) or to the list of trusted encryption keys (EKList), depending
on {\em type}. It updates the respective PSE object. In case of the last entry of the required
list being deleted the respective PSE object is deleted, too.
If the PSE from where the entry is to be removed is restricted to hold a single RSA key pair
for both signature and encryption purposes, however, only the PKList will be considered for deleting the identified entry,
no matter what the value of {\em type} is.

PKList and EKList are chained lists of ToBeSigned-structures which contain the
whole ToBeSigned field of the certificate which belongs to the public key. If 
{\em name} is not NULL and {\em issuer} is NULL, all entries which have the same partner name (or
subject name) are deleted. Otherwise, {\em issuer} and {\em serial} must not
be NULL, and the entry which has the same issuer name and serial number is
deleted.  
\\ [1em]
A call of af\_pse\_open() is not required before calling af\_pse\_delete\_PK()
(see af\_pse\_open(3X)).

\subsubsection{af\_pse\_delete\_PCA}
\nm{3X}{af\_pse\_delete\_PCA}{Delete entry from the list of recognized PCAs from PSE}
\label{af_del_PCA}
\hl{Synopsis}
\#include $<$af.h$>$ 

RC {\bf af\_pse\_delete\_PCA} {\em (dname)} \\
DName *{\em dname};
\hl{Parameter}
\parname  {{\em dname}}
\pardescript  {DName of the PCA.}

\parname  {{\em Return Value}}
\pardescript  {0 or -1.}
\hl{Description}
af\_pse\_delete\_PCA deletes an entry from the list of recognized PCAs. 
It updates the PSE object PCAList. In case of the last entry of PCAList
being deleted the PSE object PCAList is deleted, too.

A PCAList is a structure of type PKList.
This is a chained list of ToBeSigned-structures which contain the
whole ToBeSigned part of the certificate which is issued for the PCA public key.  
\\ [1em]
A call of af\_pse\_open() is not required before calling af\_pse\_delete\_PCA()
(see af\_pse\_open(3X)).

\subsubsection{af\_pse\_get\_owner}
\nm{3X}{af\_pse\_get\_owner}{Search public key in the list of trusted public keys (PKList/EKList) from PSE}
\label{af_search_PK}
\hl{Synopsis}
\#include $<$af.h$>$

DName *{\bf af\_pse\_get\_owner} {\em (type, pk)} \\
KeyType {\em type}; \\
KeyInfo *{\em pk};
\hl{Parameter}
\parname  {{\em type}}
\pardescript  {Type of the key: \\
                    SIGNATURE: public signature verification key of a communication partner. \\
                    ENCRYPTION: public encryption key of a
                                communication partner.}

\parname  {{\em pk}}
\pardescript  {public key (X.509 SubjectPublicKeyInfo).}

\parname  {{\em Return Value}}
\pardescript  {Address of DName or NULL.}
\hl{Description}
af\_pse\_get\_owner searches {\em pk} in the list of known 
verification keys (PKList) or in the list of known encryption keys (EKList), depending
on {\em type}, and returns the name of its owner (address of DName).
\\ [1em]
A call of af\_pse\_open() is not required before calling af\_pse\_get\_owner()
(see af\_pse\_open(3X)).

\subsubsection{af\_pse\_get\_PK}
\nm{3X}{af\_pse\_get\_PK}{Get KeyInfo from the list of trusted public keys (PKList/EKList) from PSE}
\label{af_search_Name}
\hl{Synopsis}
\#include $<$af.h$>$

KeyInfo *{\bf af\_pse\_get\_PK} {\em (type, subject, issuer, serial)} \\
KeyType {\em type}; \\
DName *{\em subject}; \\
DName *{\em issuer}; \\
OctetString *{\em serial};
\hl{Parameter}
\parname  {{\em type}}
\pardescript  {Type of the key: \\
                    SIGNATURE: public signature verification key of a communication partner. \\
                    ENCRYPTION: public encryption key of a
                                communication partner.} 

\parname  {{\em subject}}
\pardescript  {DName of the partner.}

\parname  {{\em issuer}}
\pardescript  {DName of the issuer of the partner.}

\parname  {{\em serial}}
\pardescript  {serial number.}

\parname  {{\em Return Value}}
\pardescript  {Address of KeyInfo or NULL.}
\hl{Description}
af\_pse\_get\_PK returns the PublicKeyInfo of an entry in the list of trusted 
verification keys (PKList) or in the list of trusted encryption keys (EKList), depending
on {\em type}.
If the PSE from which the required PublicKeyInfo is retrieved is restricted to hold a single RSA key pair
for both signature and encryption purposes, however, only the PKList will be considered for returning the required PublicKeyInfo,
no matter what the value of {\em type} is.

Both PKList and EKList may contain multiple entries per subject name, however,
each entry is uniquely identified by the issuer-serialnumber pair.

If {\em subject} is not NULL and {\em issuer} is NULL, PK of the first entry which
matches parameter {\em subject} is returned.

If {\em subject} is not NULL and {\em issuer} is not NULL, PK of that entry which 
matches all three parameters {\em subject}, {\em issuer} and {\em serial} is returned.
 
If {\em subject} is NULL and {\em issuer} is not NULL, PK of that entry which matches the
parameters {\em issuer} and {\em serial} is returned.

NULL is returned if the specified entry cannot be found.
\\ [1em]
A call of af\_pse\_open() is not required before calling af\_pse\_get\_PK()
(see af\_pse\_open(3X)).

\subsubsection{af\_pse\_get\_TBS}
\nm{3X}{af\_pse\_get\_TBS}{Get ToBeSigned from the list of trusted public keys (PKList/EKList) from PSE}
\label{af_search_tbs_Name}
\hl{Synopsis}
\#include $<$af.h$>$

ToBeSigned *{\bf af\_pse\_get\_TBS} {\em (type, subject, issuer, serial)} \\
KeyType {\em type}; \\
DName *{\em subject}; \\
DName *{\em issuer}; \\
OctetString *{\em serial};
\hl{Parameter}
\parname  {{\em type}}
\pardescript  {Type of the key: \\
                    SIGNATURE: public signature verification key of a communication partner. \\
                    ENCRYPTION: public encryption key of a
                                communication partner.}

\parname  {{\em subject}}
\pardescript  {DName of the partner.}

\parname  {{\em issuer}}
\pardescript  {DName of the issuer.}

\parname  {{\em serial}}
\pardescript  {serial number.}

\parname  {{\em Return Value}}
\pardescript  {Address of KeyInfo or NULL.}
\hl{Description}
af\_pse\_get\_TBS returns an entry (which is a ToBeSigned) from the list of trusted 
verification keys (PKList) or in the list of trusted encryption keys (EKList), depending
on {\em type}.
If the PSE from which the required entry is retrieved is restricted to hold a single RSA key pair
for both signature and encryption purposes, however, only the PKList will be considered for returning the requested entry,
no matter what the value of {\em type} is.

Both PKList and EKList may contain multiple entries per subject name, however,
each entry is uniquely identified by the issuer-serialnumber pair.

If {\em subject} is not NULL and {\em issuer} is NULL, the first entry which
matches the parameter {\em subject} is returned.

If {\em subject} is not NULL and {\em issuer} is not NULL, the entry which 
matches all three parameters {\em subject}, {\em issuer} and {\em serial} is returned.
 
If {\em subject} is NULL and {\em issuer} is not NULL, the entry which matches the
parameters {\em issuer} and {\em serial} is returned.

NULL is returned if the specified entry cannot be found.
\\ [1em]
A call of af\_pse\_open() is not required before calling af\_pse\_get\_TBS()
(see af\_pse\_open(3X)).

\subsubsection{af\_encrypt}
\nm{3X}{af\_encrypt}{Encrypt octetstring using AF PSE objects}
\label{af_encrypt}
\hl{Synopsis}
\#include $<$af.h$>$

RC {\bf af\_encrypt} {\em (in\_octets, out\_bits, more, key, dname)} \\
OctetString *{\em in\_octets}; \\
BitString *{\em out\_bits}; \\
More {\em more}; \\
Key *{\em key}; \\
DName *{\em dname};
\hl{Parameter}
\parname  {{\em in\_octets}}
\pardescript  {OctetString of the data to be encrypted.}

\parname  {{\em out\_bits}}
\pardescript  {BitString of the encrypted data. Memory must be provided by the calling program.}

\parname  {{\em more}}
\pardescript  {SEC\_MORE: More data will follow for this encryption process. \\
                  SEC\_END: Last data of this encryption process.}

\parname  {{\em key}}
\pardescript  {Structure which identifies the encryption key.}

\parname  {{\em dname}}
\pardescript  {DName of the intended recipient of the data.}

\parname  {{\em Return Value}}
\pardescript  {Number of encrypted bits in {\em out\_bits}, or -1.}
\hl{Description}
af\_encrypt encrypts the input data of octetstring {\em in\_octets} and returns the encrypted data
in {\em out\_bits}. Several successive calls of af\_encrypt can be chained
to one encryption process by means of {\em more}.

The number of octets in {\em in\_octets} is arbitrary. The number of bits in {\em out\_bits}
corresponds in case of {\em more} $=$ SEC\_MORE to the parameters of the indicated algorithm
(e.g. a multiple of 64 in case of DES or DES3 algorithms and a multiple of keysize
in case of RSA). The remaining bits are collected by af\_encrypt.

af\_encrypt puts the encrypted data into {\em out\_bits}\pf bits at the offset
{\em out\_bits}\pf nbits and returns the number of encrypted bits. In addition, this number is added 
to {\em out\_bits}\pf nbits.
The calling program must provide {\em out\_bits}\pf nbits and {\em out\_bits}\pf bits and sufficient
memory in {\em out\_bits}\pf bits.
The amount of memory should be not less than
{\em in\_octets}\pf noctets $+$ ({\em out\_bits}\pf nbits) $/$ 8 $+$ $n$ with $n = 8$ 
in case of DES or DES3 algorithms and $n = keysize / 8$ in case of RSA.

Encryption key and corresponding algorithm are determined by parameters {\em key} and {\em dname} as follows:
\be
\m If {\em key} is the NULL pointer and {\em dname} is the NULL pointer,
   the own public encryption key, which is part of the user certificate stored in the PSE object
   {\em EncCert}, is used.
   If a single RSA key pair is used for both signature and encryption purposes, however, the own public encryption key
   is retrieved from PSE object {\em Cert}.
\m If {\em key} is the NULL pointer and {\em dname} is not the NULL pointer,
   the public encryption key of {\em dname} is searched in the PSE object {\em EKList}.
   If a single RSA key pair is used for both signature and encryption purposes, however, the public encryption key of 
   {\em dname} is searched in the PSE object {\em PKList}.
\m If {\em key}\pf key is not the NULL pointer, the structure it points to
   contains algorithm identifier and key.
\m If {\em key}\pf key is the NULL pointer and {\em key}\pf keyref $>$ 0,
   key and algorithm identifier which are referenced by this number are used. This is a key
   previously installed under this reference with {\em af\_gen\_key} or 
   {\em af\_put\_EncryptedKey}.
\m If {\em key}\pf key is the NULL pointer and {\em key}\pf keyref $=$ 0,
   {\em key}\pf pse\_sel indicates the PSE object which contains key and algorithm.
\ee

{\bf In case of RSA algorithm:}
\be
\m The encryption key which is of type BitString is supposed to be
   a {\bf public key}. It consists of the substructure described in
   SEC-PSE-OBJECTS(4),
   which in case of a public key consists of the
   modulus and the public exponent. af\_encrypt uses internally the
   decoding function {\em d\_KeyBits} to get the modulus {\em n} and the exponent {\em e}.
\m The octetstring passed through {\em in\_octets} is internally blocked into portions of
   {\em blocksize} bits. These blocks are encrypted into blocks of {\em keysize} bits
   (padded with trailing ZEROES) which in turn are combined into the bitstring {\em out\_bits}, i.e
   the bitstring of the encrypted data consists of more bits than the octetstring of the cleartext data
   according to the relation between {\em blocksize} and {\em keysize}.
   The length of the encrypted data is always a multiple of {\em keysize} bits.
   {\em blocksize} is derived from {\em keysize} as the largest multiple of octets which consists of
   less bits than {\em keysize} (for instance $keysize = 512$ leads to $blocksize = 63$).
\m There is no special mechanism to encode the length of the original plaintext prior
   to encryption. Decrypting encrypted (with {\em af\_encrypt}) data with {\em af\_decrypt}
   leads to a result with a length of a multiple of {\em blocksize} octets, i.e. the decrypted
   plaintext is longer than the original plaintext with trailing zeroes.
\ee
{\bf In case of DES or DES3 algorithm:}
   The length of the encryption result is always a multiple of 64 bits.

   In case of {\bf desCBC\_pad}, {\bf desECB\_pad}, {\bf desCBC3\_pad} or 
   {\bf desECB3\_pad} algorithm, additional encoding which allows 
   {\em af\_decrypt} to restore the plaintext length is added prior
   to encryption in case of {\em more} $=$ SEC\_MORE. If $n$ is the plaintext length,
   8 - $n$ $mod$ 8 octets are padded with
   the last octet of the last block containing the number of 
   valid octets of that block (between 0 and 7), i.e. if the plaintext consists 
   of a multiple of eight octets, one extra block is added with the last octet set to zero.

   In case of {\bf desCBC\_pem} or {\bf desECB\_pem} algorithm, 
   padding as defined in RFC 1424 is applied prior to encrypting
   which again enables {\em af\_decrypt} to restore the plaintext length.

   In case of {\bf desCBC}, {\bf desECB}, {\bf desCBC3} or {\bf desECB3} algorithm,
   trailing zeroes are padded prior to encrypting until an eight-octet boundary is reached, i.e.
   no octets are padded if the plaintext length is a multiple of eight.
   Measures which allow to restore the original plaintext length must
   be taken by the application (of {\em af\_encrypt} and {\em af\_decrypt}).

\subsubsection{af\_decrypt}
\nm{3X}{af\_decrypt}{Decrypt bitstring using AF PSE objects}
\label{af_decrypt}
\hl{Synopsis}
\#include $<$af.h$>$

RC {\bf af\_decrypt} {\em (in\_bits, out\_octets, more, key)} \\
BitString *{\em in\_bits}; \\
OctetString *{\em out\_octets}; \\
More {\em more}; \\
Key *{\em key};
\hl{Parameter}
\parname  {{\em in\_bits}}
\pardescript  {BitString of the data to be decrypted.}

\parname  {{\em out\_octets}}
\pardescript  {OctetString of the decrypted data. Memory must be provided by the calling program.}

\parname  {{\em more}}
\pardescript  {SEC\_MORE: More data will follow for this decryption process. \\
                  SEC\_END: Last data of this decryption process.}

\parname  {{\em key}}
\pardescript  {Structure which identifies the decryption key.}

\parname  {{\em Return Value}}
\pardescript  {Number of decrypted octets in {\em out\_octets}, or -1.}
\hl{Description}
af\_decrypt decrypts the input data of bitstring {\em in\_bits} and returns the decrypted data
in {\em out\_octets}. Several successive calls of af\_decrypt can be chained
to one decryption process by means of {\em more}.

The number of bits in {\em in\_bits} is arbitrary. The number of octets in {\em out\_octets}
corresponds in case of {\em more} $=$ SEC\_MORE to the parameters of the indicated algorithm
(e.g. a multiple of eight in case of DES or DES3 algorithms and a multiple of blocksize/8
in case of RSA). The remaining octets are collected by af\_decrypt.

af\_decrypt puts the decrypted data into {\em out\_octets}\pf octets at the offset
{\em out\_octets}\pf noctets and returns the number of decrypted octets. In addition, this number is added 
to {\em out\_octets}\pf noctets.
The calling program must provide {\em out\_octets}\pf noctets and {\em out\_octets}\pf octets and sufficient
memory in {\em out\_octets}\pf octets.
The amount of memory should be not less than
({\em in\_bits}\pf nbits) $/$ 8 $+$ {\em out\_octets}\pf noctets $+$ $n$ with $n = 8$ 
in case of DES or DES3 algorithms and $n = keysize / 8$ in case of RSA.

Decryption key and corresponding algorithm are determined by parameter {\em key} as follows:
\be
\m If {\em key} is the NULL pointer,
   the own secret decryption key, which is stored in the PSE object
   {\em DecSKnew} or {\em DecSKold}, respectively, is used. 
   In this case, the decryption follows
   the procedure described in Vol. 1, Principles of Security Operations, {\em Operation of
   Encryption Keys}.
   If a single RSA key pair is used for both signature and encryption purposes, however, the own secret key is
   obtained from PSE object {\em SKnew} or {\em SKold}, respectively.
\m If {\em key}\pf key is not the NULL pointer, the structure it points to
   contains algorithm identifier and key.
\m If {\em key}\pf key is the NULL pointer and {\em key}\pf keyref $>$ 0,
   key and algorithm identifier which are referenced by this number are used. This is a key
   previously installed under this reference with {\em af\_gen\_key} or 
   {\em af\_put\_EncryptedKey}.
\m If {\em key}\pf key is the NULL pointer and {\em key}\pf keyref $=$ 0,
   {\em key}\pf pse\_sel indicates the PSE object which contains key and algorithm.
\ee

{\bf In case of RSA algorithm:}
\be
\m The decryption key which is of type BitString is supposed to be
   a {\bf secret key}. It consists of the substructure described in
   SEC-PSE-OBJECTS(4),
   which in case of a secret key consists of the secret
   prime factors. af\_decrypt uses internally the
   decoding function {\em d\_KeyBits} to get the prime factors {\em p} and {\em q}.
\m The bitstring passed through {\em in\_bits} is internally blocked into portions of
   {\em keysize} bits. These blocks are decrypted into blocks of {\em blocksize} bits
   (padded with trailing ZEROES) which in turn are combined into the octetstring {\em out\_octets}, i.e
   the octetstring of the decrypted data consists of less bits than the bitstring of the encrypted data
   according to the relation between {\em blocksize} and {\em keysize}.
   {\em blocksize} is derived from {\em keysize} as the largest multiple of octets which consists of
   less bits than {\em keysize} (for instance $keysize = 512$ leads to $blocksize = 63$).
\m af\_decrypt does not assume any specific padding prior encrypting in order to be able to
   restore the original plaintext length. This must be done by the application before 
   calling  af\_encrypt and after calling af\_decrypt.
\m Since there is no special mechanism to encode the length of the original plaintext prior
   to encryption, decrypting encrypted (with {\em af\_encrypt}) data with {\em af\_decrypt}
   leads to a result which is longer than the original plaintext with trailing zeroes.
\ee
{\bf In case of DES or DES3 algorithm:} \\ [1em]
   While the number of bits in {\em in\_bits} is arbitrary for each single call to
   {\em af\_decrypt}, the number of bits of the input data for the whole
   decryption process (until {\em more} $=$ SEC\_END) must be a multiple of 64.

   In case of {\bf desCBC\_pad}, {\bf desECB\_pad}, {\bf desCBC3\_pad} or 
   {\bf desECB3\_pad} algorithm, {\em af\_decrypt} considers 
   the last octet of the last block as the number of 
   valid octets of that block (between 0 and 7) which enables {\em af\_decrypt}
   to restore the original octetstring. If the last octet of the last block has not
   a value between 0 and 7 after decryption, an error is indicated.

   In case of {\bf desCBC\_pem} or {\bf desECB\_pem} algorithm, {\em af\_decrypt} assumes
   that padding as defined in RFC 1423 has been applied prior to encrypting
   which again enables {\em af\_decrypt}
   to restore the original octetstring.

   In case of {\bf desCBC}, {\bf desECB}, {\bf desCBC3} or {\bf desECB3} algorithm, {\em af\_decrypt} does
   not assume any specific padding prior to encrypting. The decrypting result is a multiple
   of 8 octets, and measures which allow to restore the original plaintext length are 
   assumed to be taken by the application (of {\em af\_encrypt} and {\em af\_decrypt}).

\subsubsection{af\_sign}
\nm{3X}{af\_sign}{sign octetstring with own signature key}
\label{af_sign}
\hl{Synopsis}
\#include $<$af.h$>$

RC {\bf af\_sign} {\em (in\_octets, signature, more)} \\
OctetString *{\em in\_octets}; \\
Signature *{\em signature}; \\
More {\em more};
\hl{Parameter}
\parname  {{\em in\_octets}}
\pardescript  {OctetString of the data to be signed.} 

\parname  {{\em signature}}
\pardescript  {Returned signature. Memory is provided by the called program.}

\parname  {{\em more}}
\pardescript  {SEC\_MORE: More data will follow for this sign process. \\
                  SEC\_END: Last data of this sign process.}

\parname  {{\em Return Value}}
\pardescript  {0 or -1.}
\hl{Description}
In case of {\em more} $=$ SEC\_MORE or SEC\_END af\_sign hashes {\em in\_octets} onto
an internal hash value, using the algorithm as described below.
In case of {\em more} $=$ SEC\_END this hash value is encrypted using key and
algorithm as described below and returned as BitString in {\em signature}\pf signature.
All necessary memory for returned information will be provided by the
called program.

As signature key the own secret signature key, which is stored in the PSE
object {\em SignSK}, is used. If a single RSA key pair is used for both signature and encryption purposes, however,
the own secret signature key is stored in the PSE object {\em SKnew}. 

The algorithm to be used is determined as follows:
\be
\m if {\em signature}\pf signAI\pf objid is not the NULL pointer, the algorithm
   determined by this object identifier is used.
\m otherwise, i.e. if either {\em signature}\pf signAI or {\em signature}\pf signAI\pf objid 
   is the NULL pointer, the algorithm
   associated to the own secret signature key is used.
   In addition, this algorithm identifier is returned in {\em signature}\pf aignAI\pf objid.
   All necessary memory allocation is done by the called program.
\ee

\subsubsection{af\_sign}
\nm{3X}{af\_sign}{sign octetstring with own signature key}
\label{af_sign}
\hl{Synopsis}
\#include $<$af.h$>$

RC {\bf af\_sign} {\em (in\_octets, signature, more)} \\
OctetString *{\em in\_octets}; \\
Signature *{\em signature}; \\
More {\em more};

RC {\bf af\_OctetString2SignedFile} {\em (file, ostr, withcerts, signai)} \\
char *{\em file};
OctetString *{\em ostr}; \\
Boolean {\em withcerts}; \\
AlgId *{\em sigai};

\hl{Parameter}
\parname  {{\em in\_octets}}
\pardescript  {OctetString of the data to be signed.} 

\parname  {{\em signature}}
\pardescript  {Returned signature. Memory is provided by the called program.}

\parname  {{\em more}}
\pardescript  {SEC\_MORE: More data will follow for this sign process. \\
                  SEC\_END: Last data of this sign process.}

\parname  {{\em Return Value}}
\pardescript  {0 or -1.}
\hl{Description}
{\bf af\_sign} is the general purpose sign routine which signs the given
{\em in\_octets} and produces the signature {\em signature}.
In case of {\em more} $=$ SEC\_MORE or SEC\_END af\_sign hashes {\em in\_octets} onto
an internal hash value, using the algorithm as described below.
In case of {\em more} $=$ SEC\_END this hash value is encrypted using key and
algorithm as described below and returned as BitString in {\em signature}\pf signature.
All necessary memory for {\em signature}\pf signature and {\em signature}\pf signAI
will be provided by the called program.

The own secret signature key, which is stored in the PSE
object {\em SKnew} or {\em SignSK}, is used as signature key. 

The algorithm to be used is determined as follows:
\be
\m if {\em signature}\pf signAI\pf objid is not the NULL pointer, the algorithm
   determined by this object identifier is used.
\m otherwise, i.e. if either {\em signature}\pf signAI or {\em signature}\pf signAI\pf objid 
   is the NULL pointer, the algorithm
   associated to the own secret signature key is used.
   In addition, this algorithm identifier is returned in {\em signature}\pf aignAI\pf objid.
   All necessary memory allocation is done by the called program.
\ee

{\bf af\_OctetString2SignedFile} signs the given OctetString {\em ostr}, writes {\em ostr}
to the file {\em file}, and writes the signature as ASN.1-encoded Signature to {\em file}.sig.
If {\em withcerts} is TRUE, the file {\em file}.ctf containing the ASN.1-encoded Certificates
from the PSE (which is the user certificate and the forward certification path) is additionally
written. Files are created, if they don't exist.

\subsubsection{af\_verify}
\nm{3X}{af\_verify}{Verify signature and certificates}
\addtotoc{af\_verify\_Certificates}
\addtotoc{af\_check\_validity}
\addtotoc{af\_SignedFile2OctetString}
\label{af_verify}
\hl{Synopsis}
\#include $<$af.h$>$ 

extern Boolean af\_chk\_crl, af\_access\_directory; \\
extern Verficationresult *verifresult; \\ [1em]
RC {\bf af\_verify} {\em (in\_octets, signature, more, or\_cert, time, pkroot)} \\
OctetString *{\em in\_octets}; \\
Signature *{\em signature}; \\
More {\em more}; \\
Certificates *{\em or\_cert}; \\
UTCTime *{\em time}; \\
PKRoot *{\em pkroot};

RC {\bf af\_verify\_Certificates} {\em (or\_cert, time, pkroot)} \\
Certificates *{\em or\_cert}; \\
UTCTime *{\em time}; \\
PKRoot *{\em pkroot};

RC {\bf af\_check\_validity\_of\_Certificate} {\em (time, certificate)} \\
UTCTime *{\em time}; \\
Certificate *{\em certificate};

RC {\bf af\_check\_validity\_of\_PKRoot} {\em (time, pkroot)} \\
UTCTime *{\em time}; \\
PKRoot *{\em pkroot};

RC {\bf af\_check\_validity\_of\_ToBeSigned} {\em (time, tobesigned)} \\
UTCTime *{\em time}; \\
ToBeSigned *{\em tobesigned};

OctetString {\bf *af\_SignedFile2OctetString} {\em (file)} \\
char *{\em file};

\hl{Parameter}
\parname  {{\em in\_octets}}
\pardescript  {OctetString of the input data for the verification.}

\parname  {{\em signature}}
\pardescript  {Signature to be verified.}

\parname  {{\em more}}
\pardescript  {SEC\_MORE: More input data will follow for this verification process. \\
                  SEC\_END: Last input data of this verification process.}

\parname  {{\em or\_cert}}
\pardescript  {Originator Certificate (in form of an X.509 Certificates).}

\parname  {{\em time}}
\pardescript  {Verification time.}

\parname  {{\em pkroot}}
\pardescript  {Structure containing the top level verification key, or NULL pointer.}

\parname  {{\em Return Value}}
\pardescript  {0 (Verification positive) or -1.}
\hl{Description}
{\bf af\_verify} is the general purpose verification routine which verifies the given
{\em signature} of a given {\em in\_octets}, and the whole certification path given
in {\em or\_cert}. 
In case of {\em more} $=$ SEC\_MORE or SEC\_END, {\bf af\_verify} hashes {\em in\_octets} onto
an internal hash value, using the algorithm as described below.
In case of {\em more} $=$ SEC\_END the signature given in {\em signature}\pf signature
is decrypted with the PK contained in the user certificate which in turn is part of the
originator certificate, and the result is compared
with the previously computed internal hash value. The algorithm to be used is either 
{\em signature}\pf signAI, if {\em signature}\pf signAI\pf objid is not the
NULL pointer, or the algorithm belonging to PK of the user certificate. If both hash values
are equal, the verification strategy described below is performed, i.e. the whole certification path
contained in {\em or\_cert} is verified, with the last uncertified public key
contained in {\em pkroot}. If the verification succeeds finally, a return value
of 0 is returned. Otherwise, a return value of -1 is returned, and the certificate which could
not be verified is indicated in af\_error.cert. In either case, it generates the external
variable {\em verifresult} which provides more information about the verification process (see C. Verification Result below).
{\em verifresult} can be printed using aux\_fprint\_VerificationResult(), and can be released
using aux\_free\_VerificationResult(). If {\em verifresult} is not NULL when af\_verify is
called with more $=$ SEC\_END, af\_verify assumes that this is the verification result of a former call to
af\_verify and releases this address with aux\_free\_VerificationResult.

If {\em time} is not the NULL pointer, it specifies date and time of verification. Otherwise
time and date of verification is the current time.

If {\em pkroot} is not the NULL pointer, it specifies the top level public verification key.
Otherwise the top level verification key is taken from the PSE object {\em PKRoot}. \\ [1em]
{\bf af\_verify\_Certificates} verifies only the the originator certificate {\em or\_cert}.
It produces also the external variable {\em verifresult}.
\\ [1em]
{\bf af\_check\_validity\_of\_Certificate} returns 0 if {\em time} is within the validity interval
of {\em certificate}, otherwise it returns -1.
\\ [1em]
{\bf af\_check\_validity\_of\_PKRoot} returns 0 if {\em time} is within the validity interval
of {\em pkroot}, otherwise it returns -1.
\\ [1em]
{\bf af\_check\_validity\_of\_ToBeSigned} returns 0 if {\em time} is within the validity interval
of {\em tobesigned}, otherwise it returns -1.
\\ [1em]
{\bf af\_SignedFile2OctetString} returns the content of file {\em file} as OctetString
if it can be validated with the signature in {\em file}.sig.
 
If {\em file}.ctf exists, the
content of this file is assumed to be an ASN.1-encoded Certificates (user certificate and
forward certification path) and is being used for the verification. Otherwise the
user certificate Cert/SignCert and the forward certification path FCPath from the PSE
is being used for the verification.

A NULL pointer is returned if {\em file}.sig doesn't exist, or if the verification
fails for any reason.
 
af\_SignedFile2OctetString produces, like af\_verify, the external variable {\em verifresult}. 

\hl{Verification Strategy}

The verification process is ruled by the parameters
{\em or\_cert} and {\em pkroot} and the PSE objects {\em PKList} and {\em FCPath} (if
existent). It is also dependant on the point of time provided by parameter {\em time}, which
specifies the time and date of verification; if there
is no time provided, the verification time is set to the current time.
In addition, the verification process can be controlled by the global boolean variables
{\em af\_chk\_crl}, {\em af\_FCPath\_is\_trusted}, and {\em af\_access\_directory}.
\\ [1em]
{\em pkroot} contains the top level verification key, where top 
level does not necessarily refer to the top level of a certification tree 
(that is, pkroot does not necessarily need to correspond to the
public key of a Root-CA). Instead, {\em top level} refers to that level 
of the certification tree where the recipient wants the verification 
process to terminate; it may identify any of the verifier's points of trust.
{\em top level verification key} then represents the 
public key involved in the last verification step.
If {\em pkroot} is empty, the top level verification key is 
taken from the PSE object {\em PKRoot}.
\\ [1em]
Before the verification process starts, it is checked whether the information stored
in {\em pkroot} has expired (see af\_check\_validity\_of\_PKRoot(3X)). An expired
{\em pkroot} information is indicated by its {\em notafter} field having already
passed.
\\ [1em]
If there is no originator certificate {\em or\_cert} to be evaluated 
by the verifier, the signature contained in {\em signature} is 
validated with the top level verification key derived from {\em pkroot}, 
with no further verification step being applied. In order that the 
verification of {\em signature} be successful, {\em pkroot} must contain the 
originator's public key; the originator who has 
generated the signature may be a user, a CA, or a Root-CA.
If {\em pkroot} is empty, however, the top level verification key is 
taken from the PSE object {\em PKRoot}.
\\ [1em]
If the originator certificate {\em or\_cert} is not empty, it
comprises at least one user certificate and optionally one
forward certification path. If af\_verify was called, the user certificate 
contains the public key required for validating the signature supplied in 
{\em signature}; otherwise, the user certificate identifies the
certificate which is to be validated with the optionally appended
forward certification path.
\\ [1em]
There are two ways of constructing a certification path, starting with the 
originator's certificate and ending with the public key of a CA which the 
verifier trusts: \\
The certification path may be extracted completely from a given certification 
path supplied by the originator (this is discussed in section A). 
However, if the originator has not provided a certification path, or the recipient 
fails to identify a trusted CA within an incoming 
certification path, a Directory (X.500 or .af-db) is accessed in order to build 
(complete) the originator's certification path (this is discussed in detail 
in section B).
\\ [1em]

{\bf A. Construction of a Certification Path from a given Forward Certification Path}
\\ [1em]
If the originator has supplied a FCPath (a detailed discussion on the structure 
of a FCPath is given in Volume 1), the verification
process run by the recipient has to extract an optimal chain of certificates 
from the given FCPath; the chain is optimal if it is as short as possible.
\\ [1em]
The selection of a chain of certificates from a given FCPath is performed
from bottom to top, using a stack of certificates whose initial value
is represented by the user certificate.
\\ [1em]
The PKList and FCPath checks described below are only performed in case that {\em af\_chk\_crl}
is set to FALSE, i.e., no revocation lists are to be consulted. As a further restriction, 
the verifier's PSE object {\em FCPath} is only
considered if {\em af\_FCPath\_is\_trusted} is set to TRUE. Note that the
FCPath check described below will first reduce the verifier's FCPath (stored in PSE object
{\em FCPath}) to a chain of hierarchy certificates (see reduce\_FCPath\_to\_HierarchyPathwhich()) 
which may all be trusted. The FCPath check is useful for optimizing the verification process,
if the originator of the digital signature and its recipient have a common point of trust
within their forward certification paths.
\\ [1em]
If af\_verify was called, the user certificate is considered as follows (if
af\_verify\_Certificates was called, however, the user certificate is considered in the
same manner, except that the PKList and FCPath checks are omitted):
\be
\m Validity \\
It is checked whether the time and date of verification specified by {\em time} lies 
within the validity time frame of the user certificate.
\m CRL Check \\
If {\em af\_chk\_crl} is set to TRUE, the user certificate is checked against the current 
revocation list of its issuer, to ensure that it has not been revoked.
\m PKList \\
If the originator's public signature key held within the user certificate
is trusted and has already been stored in the recipient's PKList (whose
corresponding information was found not to have expired),
the signature is validated with this public key, with there being no stack of 
certificates to be evaluated.
\m FCPath \\
If the user certificate is found in the recipient's PSE object {\em FCPath} (whose
corresponding certificate was found not to have expired), its public
key may be trusted and used for validating the signature, with there being no stack of 
certificates to be evaluated.
\m PKRoot \\
If the issuer of the user certificate corresponds to the CA at the top level
({\em pkroot}), the top level verification key is derived from {\em pkroot}. It is 
used for verifying the user certificate which is the only element of the stack.
\m Cross Check \\
If the issuer of the user certificate has been cross-certified by the own Root CA, the 
reverse component of the relevant cross certificate pair (which may be found in PSE
object {\em CrossCSet}) is appended to the user certificate (added to the stack of certificates), 
and the top level verification key is derived from {\em pkroot} (which then corresponds
to the own Root CA).
\ee
If none of that applies to the user certificate, the incoming forward certification path
is processed from bottom to top, as long as the hierarchy levels being encountered contain
a single certificate only. The processing of the forward certification path stops
at the first hierarchy level that is multi-valued, i.e., that contains more than one
certificate.
Each certificate belonging to a single-valued hierarchy level is considered as described
above, with an additional check on whether it can be verified using the public key held 
within the certificate at the next level above.
\\ [1em]
The following part describes how the certificates of multi-valued FCPath levels are selected
to form part of the chain of certificates which is then submitted to the validation process:
\\ [1em]
Each level of the FCPath contributes exactly one certificate to the stack.
At any level, there are two ways of selecting the certificate which is next to be 
added to the stack: one way shall be called {\em horizontal mode} and refers to the 
set of certificates at that particular level of the FCPath; the other one shall be 
called {\em vertical mode} and refers to the set of certificates at the next level 
above. \\
The certificate that was last added to the stack (starting with the first certificate 
at the first FCPath level that is multi-valued) is considered in the 
following way:
\\ [1em]
Condition A.1: PKList \\
If the public key of the CA representing that level has already been stored in 
the recipient's PKList (whose corresponding information was found not to have expired),
the certificate last added to the stack (owned by that CA) need not be verified 
and is therefore removed from the stack. The CA's public key is used as the 
top level verification key. That completes the construction of the chain of
certificates.
\\ [1em]
Condition A.2: FCPath \\
If the certificate is found in the recipient's PSE object {\em FCPath} (whose corresponding 
information was found not to have expired), its public
key may be trusted and used for validating the signature, with there being no stack of 
certificates to be evaluated.
\\ [1em]
Condition A.3: PKRoot \\
If the issuer of that certificate corresponds to the CA at the top level
({\em pkroot}), the top level verification key is derived from {\em pkroot}.
That completes the construction of the chain of certificates.
\\ [1em]
Condition A.4: Cross Check \\
If the issuer of the certificate being considered has been cross-certified by the own Root CA, the 
reverse component of the relevant cross certificate pair (which may be found in PSE
object {\em CrossCSet}) is added to the stack of certificates, 
and the top level verification key is derived from {\em pkroot} (which then corresponds
to the own Root CA).
\\ [1em]
Condition A.5: Cross or Hierarchy certificate? 
\bi
\m Cross certificate \\
If the certificate last added to the stack is
a cross certificate which does not satisfy any of the conditions A.1 to A.4,
it is removed from the stack, and the next certificate (if existent) within the 
set of certificates at that level is put on top of the stack 
({\em horizontal mode}) and examined in the way described above.
This is repeated until any of the remaining certificates at that level satisfies 
any of the conditions A.1 to A.4, or until a hierarchy certificate is encountered. 
In the latter case, the construction of the stack is continued at the next level 
above ({\em vertical mode}).
\m Hierarchy certificate \\
If the certificate last added to the stack is
a hierarchy certificate, the first certificate in the set of certificates at
the next level above is put on top of the stack and examined in the way
described above ({\em vertical mode}).
\ei
If at a particular level of the FCPath there is no way of selecting
the hierarchy certificate (i.e., the FCPath is incomplete), all certificates at 
that particular level are considered as cross certificates. If they all do not 
satisfy any of the conditions A.1 to A.4, the hierarchy certificate at the previous level 
is removed from the stack, and the construction of the stack is resumed at
the previous level.
It is then examined whether any of the remaining cross certificates at that 
(previous) level satisfies any of the conditions A.1 to A.4 ({\em horizontal mode}).
\\ [1em]
If the hierarchy certificate at the first level of the FCPath has been 
removed from the stack (according to the rule described above), and 
the remaining cross certificates at that level do not satisfy any of the conditions A.1 to A.4, 
the verification process terminates with an error message reporting that 
no chain of certificates could be extracted from the given FCPath, unless
the {\em af\_access\_directory} flag is set (see section B).
\\ [1em]
If a CA trusted by the recipient was identified within an incoming FCPath, the certificates
extracted from the multi-valued hierarchy levels of the forward certification path are
submitted to the following checks:
\bi
\m Each certificate of the constructed stack of certificates is verified using the public 
key held within the
certificate at the next level above, and the highest certificate is  
verified by the top level verification key.
\m If the {\em af\_chk\_crl} flag is set to TRUE, one additional validation step
is required for the certificates establishing the constructed stack of certificates: 
Each certificate is checked against the current revocation list from the 
certificate's issuer to ensure that revoked certificates are not employed. 
This requires at worst one directory access per certificate of a chain. One 
revoked certificate within the certification path discredits all certificates 
below and the user signature as well. If a revoked certificate is encountered,
the verifier is warned, and the verification process terminates with an 
appropriate error message being displayed. \\ [1em]
\ei

{\bf B. Construction of a Certification Path with Directory Support (X.500 or .af-db)}
\\ [1em] 
The recipient may use the information stored in a Directory (X.500 or .af-db) in order to 
construct or complete a certification path from the originator up to one of the recipient's
trusted CAs.
\\ [1em]
A Directory is accessed if the following holds true:
\bi
\m The {\em af\_access\_directory} flag is set;
\m There is no FCPath supplied by the originator, or the FCPath supplied 
by the originator does not contain a certification path that would contain a 
CA trusted by the recipient.
\ei
The originator must supply to the recipient at least its user certificate. An optionally
provided forward certification path is considered as described in section A.
The certificate first submitted to the directory support routine may either be
\bi
\m the user certificate itself (in case that the originator has not provided
its forward certification path, or its forward certification path does not contain a 
CA which the recipient trusts),
\m or the certificate of the highest single-valued level of the originator's FCPath.
\ei
The directory support is carried out as follows:
\\ [1em]
Step 1: Access the directory entry of the issuer of the certificate submitted
to the directory support routine, and return the set of certificates stored in 
its {\em cACertificate} attribute.
\\ [1em]
Step 2: Again, the certification path is built step by step, using a stack of
certificates which in the following is referred to as {\em stack A}.
In addition, a second stack (referred to as {\em stack B}) is managed whose 
elements represent set of certificates. The set of certificates last retrieved 
from the Directory is put on top of this stack. The set of certificates returned 
as result of the first {\em read}-operation (see step 1) forms the initial value 
of stack B.
\\ [1em]
Step 3: A certificate (starting with the first) within the set of certificates
on top of stack B is added to stack A, provided that
\\ [1em]
B.1: it is not yet contained in this stack (this is done in order to avoid
loops in the certification path);

B.2: its validity time frame intersects with the validity time frame of all
certificates contained in this stack;

B.3: it certifies the public key required for the validation of the certificate
at the previous level or the text signature 
(this check has to be performed as certificates held within directory entries 
may refer to different key pairs owned by the same directory object).
\\ [1em]
The certificate last added to stack A is considered in the same way as
described for the case where no directory support is required (see section A): 
\bi
\m If it does not satisfy any of the conditions A.1 to A.4, the {\em cACertificate} attribute 
within the directory entry of its issuer is accessed, and the returned set of 
certificates is put on top of stack B. Then, the procedure described above 
(step 3) is repeated.
\m If it satisfies condition A.1 (PKList), it is removed from stack A, and its 
public key is used as the top level verification key. That completes 
the construction of the certification path.
\m If it satisfies condition A.2 (FCPath), it is removed from stack A, and its 
public key is used as the top level verification key. That completes 
the construction of the certification path.
\m If it satisfies condition A.3 (PKRoot), the top level verification key is 
derived from {\em pkroot}. That completes the construction of the 
certification path.
\m If it satisfies condition A.4 (Cross Check), the 
reverse component of the relevant cross certificate pair (which may be found in PSE
object {\em CrossCSet}) is added to the stack of certificates, and the top level verification key is 
derived from {\em pkroot}. That completes the construction of the 
certification path.
\ei

Step 4: The certificates within the set of certificates on top of stack B
are checked one after the other, until one certificate is found that
satisfies requirements B.1 to B.3 and is therefore added to stack A.
\\ [1em]
Step 5: If none of the certificates within the set of certificates on top
of stack B fulfil requirements B.1 to B.3, it is
examined whether there is a {\em CrossCertificatePair} attribute contained
in the owner's directory entry, where owner refers to the owner ({\em subject})
of the set of certificates which has been completely exploited.
(By taking the {\em CrossCertificatePair} attribute into account, it is ensured
that the certification path under construction terminates at some CA in the
recipient's certification tree, even if the originator belongs to a 
different certification tree.) If the entry under consideration contains such
an attribute, it is checked whether
any of the issuers of the forward certificates are known to the recipient
(PKRoot, PKList, FCPath). If such an issuer is found, its public key represents the
top level verification key; that completes the construction of the 
certification path.

Otherwise (or in case that there is no {\em CrossCertificatePair} attribute 
found within the owner's directory entry), the top elements of the two stacks
are removed from their respective stacks. Then, the procedure described
as step 3 is resumed, its input being the next certificate within the
set of certificates now on top of stack B (for the meaning of {\em next}
see step 4). If there are no further certificates to be considered 
within that set of certificates, the procedure under step 5 is carried out.
\\ [1em]
If all the certificates within the set of certificates representing
the initial value of stack B and even the forward certificates 
retrieved from the {\em CrossCertificatePair} attribute within the
directory entry initially accessed turn out to be inappropriate 
for being members of a certification path, the verification process
terminates with an appropriate error message being displayed.
\\ [1em]
If a chain of certificates has been found, it is validated as described for the
case where no directory support was required. \\ \\
{\bf C. Verification Result}
\\ [1em]
The following C-structures are used for collecting any data generated by the af\_verify
af\_verify\_Certificates routines, monitoring the verification process itself: 

{\small
\btab
\1      typedef struct VerificationResult \{ \\
\2		VerificationStep \3 **verifstep; \\
\2		int \3	             trustedKey; \\
\2		Boolean \3	     success; \\
\2		Boolean \3	     textverified; \\
\2		Name \3		    *top\_name; \\
\2		Validity \3 	    *valid; \\
\2		OctetString \3	    *top\_serial; \\
\1       \} VerificationResult; \\
\etab
}
with: \\ \\
{\small
\btab
\1      typedef struct VerificationStep \{ \\
\2		Certificate \2	*cert; \\
\2		int \2		crlcheck; \\
\2		UTCTime	\2	*date; \\
\2              Validity \2     *valid; \\
\2		int \2		supplied; \\
\2              Boolean \2      policy\_CA; \\
\2              Boolean \2      dn\_subordination\_violation; \\
\1       \} VerificationStep; \\
\etab
}

The {\em verifstep} component within the {\em VerificationResult} structure identifies a 
linked list of elements of type {\em VerificationStep}.
Each element of the linked list
\bi
\m identifies a certificate which forms part of the originator's certification path and
that was either successfully verified or failed to be verified
({\em cert});
\m indicates whether the certificate {\em cert} was checked against its issuer's revocation 
list ({\em crlcheck});
\m provides the revocation date, in case of the certificate {\em cert} having been revoked
({\em date});
\m provides the next scheduled date of issue, in case that the revoction list against which the
certificate has been checked is obsolete ({\em date});
\m provides the validity time frame ({\em valid}) of the certificate {\em cert}, in case
that the creation date of the digital signature lies outside that interval;
\m indicates whether the certificate {\em cert} was supplied within the received certification
path or was retrieved from a Directory (X.500 or .af-db) ({\em supplied});
\m indicates whether {\em cert} was issued by a Policy CA ({\em policy\_CA});
\m indicates whether the name subordination rule defined by RFC 1422 is violated ({\em dn\_subordination\_violation}).
\ei

The routine {\bf aux\_fprint\_VerificationResult} is used for printing the contents of the
{\em VerificationResult} structure in a user-friendly manner. Memory allocated for
a {\em VerificationResult} structure can be freed by the {\bf aux\_free\_VerificationResult}
routine.

\subsubsection{af\_gen\_key}
\nm{3X}{af\_gen\_key}{Generate DES key or RSA key pair}
\label{af_gen_key}
\hl{Synopsis}
\#include $<$af.h$>$

RC {\bf af\_gen\_key} {\em (key, ktype, replace)} \\
Key *{\em key}; \\
KeyType {\em ktype}; \\
Boolean {\em replace};
\hl{Parameter}
\parname  {{\em key}}
\pardescript  {Structure which determines the generated key.}

\parname  {{\em ktype}}
\pardescript{type of the key pair:
		    SIGNATURE: signature key pair \\
		    ENCRYPTION: encryption key pair.}

\parname  {{\em replace}}
\pardescript  {if FALSE, do not overwrite existing key.}

\parname  {{\em Return Value}}
\pardescript  {0 or -1.}
\hl{Description}
af\_gen\_key generates a
\bi
\m  DES key, 
    if the algorithm encryption method 
    of {\em key}\pf key\pf subjectAI (as returned by aux\_ObjId2AlgEnc()) is DES.
 
    A DES key is a 64 bit long BitString.
\m  DES3 key, 
    if the algorithm encryption method 
    of {\em key}\pf key\pf subjectAI (as returned by aux\_ObjId2AlgEnc()) is DES3.
 
    DES3 is triple block encryption in the fashion encryption(k1)-decryption(k2)-encryption(k1).
    A DES3 key is a 128 bit long BitString, the first 64 bits comprising k1 and the
    second 64 bits comprising k2.
\m  RSA key pair, 
    if the algorithm encryption method 
    of {\em key}\pf key\pf subjectAI (as returned by aux\_ObjId2AlgEnc()) is RSA. 

    An RSA public key is a BitString
    which contains the ASN.1 DER code of
    \bvtab
    \4  SEQUENCE \{ \\
    \4  \2       n INTEGER,   \\
    \4  \2       e INTEGER  \\
    \4  \} \\
    \evtab
    $n$ is the modulus and $e$ is the public exponent.
    

    An RSA secret key is a BitString
    which contains the ASN.1 DER code of
    \bvtab
    \4  SEQUENCE \{ \\
    \4  \2       p INTEGER,   \\
    \4  \2       q INTEGER  \\
    \4  \} \\
    \evtab
    $p$ and $q$ are the two prime factors of the modulus.

\m  DSA key pair, 
    if the algorithm encryption method 
    of {\em key}\pf key\pf subjectAI (as returned by aux\_ObjId2AlgEnc()) is DSA. 

    A DSA public key is a BitString
    which contains the ASN.1 DER code of
    \bvtab
    \4  SEQUENCE \{ \\
    \4  \2       key INTEGER, - - y (public key)   \\
    \4  \2       params DSAParameters OPTIONAL  \\
    \4  \} \\
    \evtab
    with
    \bvtab
    DSAParameters ::= SEQUENCE \{ \\
    \4  \2       prime1 [0] INTEGER, - - p \\
    \4  \2       prime2 [1] INTEGER, - - q \\
    \4  \2       base [2] INTEGER, - - g \\
    \4  \} \\
    \evtab
    
    A DSA secret key is a BitString
    which contains the ASN.1 DER code of
    \bvtab
    \4  SEQUENCE \{ \\
    \4  \2       key INTEGER, - - x (secret key)   \\
    \4  \2       params DSAParameters OPTIONAL  \\
    \4  \} \\
    \evtab
    with
    \bvtab
    DSAParameters ::= SEQUENCE \{ \\
    \4  \2       prime1 [0] INTEGER, - - p \\
    \4  \2       prime2 [1] INTEGER, - - q \\
    \4  \2       base [2] INTEGER, - - g \\
    \4  \} \\
    \evtab

\ei

The key is returned as follows in case of a DES key or DES3 key:

If {\em key}\pf keyref $=$ 0 and {\em key}\pf pse\_sel is not the NULL pointer, the generated key
is stored in the PSE object specified by {\em key}\pf pse\_sel. The object will be
created first. 
If the object already exists and {\em replace} is FALSE, {\em af\_gen\_key}
does not overwrite the object and returns -1.
 
If {\em key}\pf keyref $>$ 0, the generated key is stored under this key reference, until it is
deleted with sec\_del\_key. 
If the key reference already exists and {\em replace} is FALSE, {\em af\_gen\_key}
does not overwrite the key and returns -1.
 
If {\em key}\pf keyref $=$ -1, the generated key is stored under a key reference which is generated 
by af\_gen\_key and returned in {\em key}\pf keyref. The key remains there until it is deleted
with sec\_del\_key. \\ [0.5cm]
The keys are returned as follows in case of an RSA Key Pair:

The public key is returned in {\em key}\pf key\pf subjectkey. Memory is provided by the called program.
The secret key is handled according to the parameter {\em key} as follows: 

If {\em key}\pf keyref $= 0$ and {\em key}\pf pse\_sel is the NULL pointer, the
generated secret key is stored in the PSE object {\em SignSK} or
{\em DecSKnew}, depending on {\em ktype}. 
If the PSE where the keypair is stored is restricted to carry one RSA keypair only, 
the generated secret key is stored in the PSE object {\em SKnew}, instead.

If {\em key}\pf keyref $=$ 0 and {\em key}\pf pse\_sel is not the NULL pointer, the generated secret key
is stored in the PSE object specified by {\em key}\pf pse\_sel. The object will be
created first. 
If the object already exists and {\em replace} is FALSE, {\em af\_gen\_key}
does not overwrite the object and returns -1.
 
If {\em key}\pf keyref $>$ 0, the generated secret key is stored under this key reference until it is
deleted with sec\_del\_key. 
If the key reference already exists and {\em replace} is FALSE, {\em af\_gen\_key}
does not overwrite the key and returns -1.
 
If {\em key}\pf keyref $=$ -1, the generated secret key is stored under a key reference which is generated 
by af\_gen\_key and returned in {\em key}\pf keyref. The key remains there until it is deleted
with sec\_del\_key.

\subsubsection{af\_create\_Certificate}
\nm{3X}{af\_create\_Certificate}{Create a prototype certificate from the
public key info.}
\label{af_make_certificate}
\hl{Synopsis}
\#include $<$af.h$>$

Certificate *{\bf af\_create\_Certificate}{\em (keyinfo, sig\_alg, obj\_name, subject, serial)} \\
KeyInfo *{\em keyinfo}; \\
AlgId *{\em sig\_alg}; \\
char *{\em obj\_name}; \\
DName {\em *subject}; \\
OctetString {\em *serial};
\hl{Parameter}
\parname{{\em keyinfo}}
\pardescript{ public key to be certified.}

\parname{{\em sig\_alg}}
\pardescript{ signature algorithm }

\parname{{\em obj\_name}}
\pardescript{ name of PSE object where secret signature key resides }

\parname{{\em subject}}
\pardescript{ the intended owner of the certificate }

\parname{{\em serial}}
\pardescript{ serial number to be assigned to the certificate }

\parname{{\em Return Value}}
\pardescript{ the prototype certificate or NULL.}
\hl{Description}
{\em af\_create\_Certificate} builds a prototype certificate around the
public key identified by {\em keyinfo}. A prototype certificate is a self-signed
certificate of the own public (signature or encryption) key. The 
prototype certificate can be sent to a certification authority, which may check
or exchange some of the data (e.g. the signature algorithm, serialnumber or the validity time frame)
and replace the signature by its own signature, in order to produce a real certificate.
\\ [1em]
The version is set to the default. The serialnumber is taken from {\em serial} or else is set to $0$.
The subject name is taken from parameter {\em subject}. If parameter {\em subject} is NULL, however, it is 
retrieved from PSE object {\em Name}. The issuer name is set to the subject name.
The signature algorithm identifier is set to the
own signature algorithm; however, if the own signature 
key is of type {\em ASYM\_ENC}, the algorithm name associated with the certificate's signature is derived
from parameter {\em sig\_alg}. The validity is set to the interval
from now on for one year. Finally, the signature will be created
with the secret key which resides in the PSE object specified by parameter {\em obj\_name}.
The creation of a prototype certificate according to this procedure enables a
certification authority to verify the prototype certificate with
the public key in {\em subjectPK}, i.e. to check whether the public
key is part of an RSA key pair and the secret key is in the hand of the
creator of the prototype certificate (this is a rather weak proof
which does not prevent intelligent attacks but which may help to prevent
unintentional key mismanagement).
\\ [1em]
The return value contains the pointer to the created prototype certificate or
the NULL pointer in case of errors.
\\ [1em]
All necessary memory of the return value is provided by the called function.
The memory of the returned {\em certificate} my be released with 
{\em aux\_free\_Certificate()}.


\subsubsection{af\_get\_EncryptedKey}
\nm{3X}{af\_get\_EncryptedKey}{Encrypt key}
\label{af_get_EncryptedKey}
\hl{Synopsis}
\#include $<$af.h$>$

RC {\bf af\_get\_EncryptedKey} {\em (encrypted\_key, plain\_key, encryption\_key, name, encryption\_key\_algid)} \\
EncryptedKey *{\em encrypted\_key}; \\
Key *{\em plain\_key}; \\
Key *{\em encryption\_key}; \\
DName *{\em name}; \\
AlgId *{\em encryption\_key\_algid};
\hl{Parameter}
\parname  {{\em encrypted\_key}}
\pardescript  {Encrypted key.} 

\parname  {{\em plain\_key}}
\pardescript  {Structure which identifies the key to be encrypted.} 

\parname  {{\em encryption\_key}}
\pardescript  {Structure which identifies the encryption key.} 

\parname  {{\em name}}
\pardescript  {DName of the intended recipient of the encrypted key.} 

\parname  {{\em encryption\_key\_algid}}
\pardescript  {Algorithm according to which the encryption will be performed.} 

\parname  {{\em Return Value}}
\pardescript  {0 or -1.}

\hl{Description}

The key identified by {\em plain\_key} is encrypted with the key and algorithm identified by
{\em encryption\_key} or {\em name} and returned in 
{\em encrypted\_key}. Memory is provided by the called program. 

If {\em encryption\_key} is the NULL pointer and {\em name} is the NULL pointer,
the own public encryption key is used, which is part of the user certificate stored in the PSE object
{\em EncCert} or {\em Cert} (depending on whether or not a single RSA key pair is used for both signature and
encryption purposes). 

If {\em encryption\_key} is the NULL pointer and {\em name} 
is not the NULL pointer, the public encryption key of {\em name} is searched in 
the PSE object {\em EKList} (or {\em PKList}, depending on whether or not a single RSA key pair is used for both signature and
encryption purposes), and used for the encryption of {\em plain\_key}.

If {\em encryption\_key} is not the NULL pointer, the key identified by this structure
is used for the encryption of {\em plain\_key}. 

The algorithm identified by {\em encryption\_key\_algid} may be used to determine the algorithm
according to which the encryption of {\em plain\_key} shall be performed. {\em encryption\_key\_algid}
supersedes the algorithm identifier inherent to the key used as {\em encryption\_key}. 

This function is intended to support the so called ``Hybrid method'' where information which
has to be transmitted confidentially is encrypted with the DES algorithm while the corresponding DES key
is transmitted in RSA encrypted form. af\_get\_EncryptedKey is useful for the sender
of such a key.

\subsubsection{af\_put\_EncryptedKey}
\nm{3X}{af\_put\_EncryptedKey}{Decrypt key}
\label{af_put_EncryptedKey}

\hl{Synopsis}

\#include $<$af.h$>$ 

RC {\bf af\_put\_EncryptedKey} {\em (encrypted\_key, plain\_key, decryption\_key\_algid)} \\
EncryptedKey *{\em encrypted\_key}; \\
Key *{\em plain\_key}; \\
AlgId *{\em decryption\_key\_algid};
\hl{Parameter}
\parname  {{\em encrypted\_key}}
\pardescript  {Encrypted key.} 

\parname  {{\em plain\_key}}
\pardescript  {Structure which determines the decrypted key.} 

\parname  {{\em decryption\_key\_algid}}
\pardescript  {Algorithm according to which the decryption will be performed.} 

\parname  {{\em Return Value}}
\pardescript  {0 or -1.}

\hl{Description}

The encrypted key given in {\em encrypted\_key} is decrypted using the
the own secret decryption key, which is stored in the PSE object
{\em DecSKnew} or {\em DecSKold}, respectively. If a single RSA key pair is used for both signature and encryption
purposes, however, the own secret key is stored in the PSE object {\em SKnew} or {\em SKold}, respectively.
The decryption follows the procedure described in Vol. 1, Principles of Security Operations, 
{\em Operation of Encryption}.
The decrypted key is handled according to {\em plain\_key} as follows:

If {\em plain\_key}\pf keyref $=$ 0 and {\em plain\_key}\pf pse\_sel is the NULL pointer, the decrypted key is
returned as BitString in {\em plain\_key}\pf key\pf subjectkey. Memory is provided by the called program. 

If {\em plain\_key}\pf keyref $>$ 0, the decrypted key is stored under this key reference until it is
deleted with sec\_del\_key. 

If {\em plain\_key}\pf keyref $=$ -1, the decrypted key is stored under a key reference which is generated 
by af\_put\_EncryptedKey and returned in {\em plain\_key}\pf keyref. The key remains there until it is deleted
with sec\_del\_key. 

The algorithm identified by {\em decryption\_key\_algid} may be used to determine the algorithm
according to which the decryption of {\em encrypted\_key} shall be performed. {\em decryption\_key\_algid}
supersedes the algorithm identifier inherent to the own secret decryption key.

This function is intended to support the so called ``Hybrid method'' where information which
has to be transmitted confidentially is encrypted with the DES algorithm while the corresponding DES key
is transmitted in RSA encrypted form. af\_put\_EncryptedKey is useful for the receiver
of such a key.

\subsubsection{af\_dir\_enter\_Certificate}
\nm{3X}{af\_dir\_enter\_Certificate}{Enter Certificate into Directory}
\addtotoc{af\_afdb\_enter\_Certificate}
\label{af_enter_Certificate}
\hl{Synopsis}
\#include $<$af.h$>$ 

RC {\bf af\_dir\_enter\_Certificate} {\em (cert, type)} \\
Certificate {\em *cert}; \\
CertificateType {\em type}; 

RC {\bf af\_afdb\_enter\_Certificate} {\em (cert, keytype)} \\
Certificate {\em *cert}; \\
KeyType {\em keytype};
\hl{Parameter}
\parname  {{\em cert}}
\pardescript  {The certificate to be stored in the Directory. } 

\parname  {{\em type}}
\pardescript  {Type of certificate to be stored in the X.500 Directory: \\
                    usercert: {\em userCertificate} \\
                    cacert: {\em cACertificate}.} 

\parname  {{\em keytype}}
\pardescript  {Type of the certified key (.af-db Directory only): \\
                    SIGNATURE: {\em signature key certificate} \\
                    ENCRYPTION: {\em encryption key certificate}.} 

\parname  {{\em Return Value}}
\pardescript {0 or -1.}
\hl{Description}
{\em af\_dir\_enter\_Certificate} enters the certificate {\em cert} into the X.500 Directory 
entry associated with the certificate owner. The certificate {\em cert} is added as value
to the {\em userCertificate} attribute or to the {\em cACertificate} attribute,
depending on {\em type}.

The caller of {\em af\_dir\_enter\_Certificate} must be the owner of the certificate 
which is to be entered into the Directory; otherwise, the Directory access 
control will refuse this operation.

In case of the {\em userCertificate} attribute being addressed, the Directory entry
associated with the certificate owner must belong to the object class
{\em strongAuthenticationUser}; in the other case, it must belong to the
object class {\em certificationAuthority}. Otherwise, an attribute error will be
reported.

The return value is 0 if the Directory entry succeeds.

The return value is -1 for one of the following reasons:
\bi
\m the Directory service does not respond (ENODIR);
\m the certificate owner's distinguished name does not identify a Directory object (ENAMEDIR);
\m no attribute of the requested type ({\em userCertificate} or {\em cACertificate}) is found 
in the entry, that is, the Directory entry associated with the certificate owner does not
belong to the object class being required ({\em strongAuthenticationUser} or 
{\em certificationAuthority}) (EATTRDIR);
\m the certificate {\em cert} already exists in the Directory entry associated with 
its owner (ECREATEOBJ);
\m the Directory access control refuses this operation (EACCDIR).
\ei
{\em af\_afdb\_enter\_Certificate} enters the certificate {\em cert} into the local .af-db
Directory entry associated with the certificate owner. The certificate {\em cert} is either 
added to the file {\em SignCSet} or {\em EncCSet}, depending on {\em keytype}; if a single RSA key pair is used for both signature and encryption purposes, however, the certificate is added to the
set of certificates stored in file {\em CSet}. 
The return value is -1 if {\em cert} already exists in the required set of certificates
({\em SignCSet}, {\em EncCSet}, or {\em CSet}). Note that the same
certificate may appear both as member of {\em SignCSet} and {\em EncCSet}.


\subsubsection{af\_dir\_retrieve\_Certificate}
\nm{3X}{af\_dir\_retrieve\_Certificate}{Retrieve Certificate from Directory}
\hl{Synopsis}
\#include $<$af.h$>$

SET\_OF\_Certificate {\bf *af\_dir\_retrieve\_Certificate} {\em (dname, type)} \\
DName {\em *dname}; \\
CertificateType {\em type}; 

SET\_OF\_Certificate {\bf *af\_afdb\_retrieve\_Certificate} {\em (dname, keytype)} \\
DName {\em *dname}; \\
KeyType {\em keytype};
\hl{Parameter}
\parname  {{\em dname}}
\pardescript  {Distinguished name of certificate owner} 

\parname  {{\em type}}
\pardescript  {Type of certificate to be retrieved from the X.500 Directory: \\
                    usercert: {\em userCertificate} \\
                    cacert: {\em cACertificate}.} 

\parname  {{\em keytype}}
\pardescript  {Type of requested certified key (.af-db Directory only): \\
                    SIGNATURE: {\em signature certificate} \\
                    ENCRYPTION: {\em encryption certificate}.} 

\parname  {{\em Return Value}}
\pardescript  {The set of certificates or NULL pointer.}
\hl{Description}
{\em af\_dir\_retrieve\_Certificate} returns a set of certificates stored in the 
X.500 Directory entry identified by {\em dname}. The set of certificates comprises the
value(s) of the {\em userCertificate} attribute or the {\em cACertificate} attribute,
depending on {\em type}. \\
In most cases, the set contains the certificate of the public signature key and the 
certificate of the public encryption key. After this function having returned 
its result, the calling process is in the position to select one certificate 
by comparing the algorithm identifier of the subject public key info.

The return value is the NULL pointer for one of the following reasons:
\bi
\m the Directory service does not respond (ENODIR);
\m the parameter {\em dname} does not identify a Directory object (ENAMEDIR);
\m no attribute of the requested type ({\em userCertificate} or {\em cACertificate})
is found in the Directory entry, that is, the Directory entry identified by {\em dname} 
does not belong to the object class being required ({\em strongAuthenticationUser} or  
{\em certificationAuthority}) (EATTRDIR).
\ei
{\em af\_afdb\_retrieve\_Certificate} returns a set of certificates stored in the 
local .af-db Directory entry identified by {\em dname}.
The returned set of certificates comprises the certificate values stored as file {\em CSet}
within the selected .af-db directory. If no such file exists
in the selected .af-db directory, the set of certificates either comprises the
values of the signature certificates (file {\em SignCSet}) or the values of the encryption certificates
(file {\em EncCSet}), depending on {\em keytype}.



\subsubsection{af\_dir\_delete\_Certificate}
\nm{3X}{af\_dir\_delete\_Certificate}{Remove Certificate from Directory}
\addtotoc{af\_afdb\_delete\_Certificate}
\hl{Synopsis}
\#include $<$af.h$>$ 

RC {\bf af\_dir\_delete\_Certificate} {\em (serial, issuer, type)} \\
OctetString *{\em serial}; \\
DName {\em *issuer}; \\
CertificateType {\em type}; 

RC {\bf af\_afdb\_delete\_Certificate} {\em (serial, issuer, keytype)} \\
OctetString *{\em serial}; \\
DName {\em *issuer}; \\
KeyType {\em keytype};
\hl{Parameter}
\parname  {{\em serial}}
\pardescript  {Serial number of the certificate which is to be removed
from the Directory.} 

\parname  {{\em issuer}}
\pardescript  {Issuer's distinguished name.} 

\parname  {{\em type}}
\pardescript  {Type of certificate to be removed from the X.500 Directory: \\
                    usercert: {\em userCertificate} \\
                    cacert: {\em cACertificate}.}  

\parname  {{\em keytype}}
\pardescript  {Type of certificate to be removed from the Directory (.af-db Directory only): \\
                    SIGNATURE: {\em signature certificate} \\
                    ENCRYPTION: {\em encryption certificate}.} 

\parname  {{\em Return Value}}
\pardescript  {0 or -1.}
\hl{Description}
{\em af\_dir\_delete\_Certificate} removes the certificate which is universally 
identified by its serial number {\em serial} and its issuer {\em issuer} from the 
caller's X.500 Directory entry. The caller's distinguished name is derived from
the global variable {\em directory\_user\_dname} which is used to authenticate the caller to the
Directory System Agent (DSA) initially contacted.
The specified certificate is removed as value from the
{\em userCertificate} attribute or the {\em cACertificate} attribute, depending
on {\em type}.
If the serial number of the certificate which is to be removed from the Directory 
is unique among the serial numbers of all certificates stored in the requested attribute, 
it is sufficient to identify the obsolete certificate by its serial number, that is, 
the issuer's distinguished name may be omitted.

The caller of {\em af\_dir\_delete\_Certificate} must be the owner of the certificate
which is to be removed from the Directory; otherwise, the Directory access control
will refuse this operation.

The return value is 0 if the Directory entry succeeds.

The return value is -1 for one of the following reasons:
\bi
\m the Directory service does not respond (ENODIR);
\m the caller's distinguished name does not identify a Directory object (ENAMEDIR);
\m no attribute of the requested type ({\em userCertificate} or {\em cACertificate})
is found in the Directory entry associated with the caller, that is, the Directory entry
associated with the caller does not belong to the object class being required
({\em strongAuthenticationUser} or {\em certificationAuthority}) (EATTRDIR);
\m the specified certificate does not exist in the Directory entry 
identified by the caller's distinguished name (EOBJ);
\m the Directory access control refuses this operation (EACCDIR).
\ei
{\em af\_afdb\_delete\_Certificate} removes the certificate which is
identified by its serial number {\em serial} and its issuer {\em issuer} from the 
caller's local .af-db Directory entry. The caller's distinguished name is derived from
the global variable {\em directory\_user\_dname}. The identified certificate is either 
removed from the caller's signature certificates (file {\em SignCSet}) or the encryption
certificates (file {\em EncCSet}), depending on {\em keytype}. If a single RSA key pair is used for both signature and encryption purposes, however, {\em keytype} is not evaluated, and the identified certificate is removed from the set of certificates stored in file {\em CSet}.


\subsubsection{af\_dir\_enter\_CertificatePair}
\nm{3X}{af\_dir\_enter\_CertificatePair}{Enter Pair of Cross Certificates into Directory}
\addtotoc{af\_afdb\_enter\_CertificatePair}
\hl{Synopsis}
\#include $<$af.h$>$ 

RC {\bf af\_dir\_enter\_CertificatePair} {\em (cpair)} \\
CertificatePair {\em *cpair};

RC {\bf af\_afdb\_enter\_CertificatePair} {\em (cpair)} \\
CertificatePair {\em *cpair};
\hl{Parameter}
\parname  {{\em cpair}}
\pardescript  {The cross certificate pair to be stored in the Directory. } 

\parname  {{\em Return Value}}
\pardescript {0 or -1.}
\hl{Description}
{\em af\_dir\_enter\_CertificatePair} enters the cross certificate pair {\em cpair} 
into the caller's X.500 Directory entry. The caller's distinguished name is derived from
the global variable {\em directory\_user\_dname} which is used to authenticate the caller to the
Directory System Agent (DSA) initially contacted. At least one of the two components 
(forward or reverse certificate) of the cross certificate pair must be present.

An attribute of type {\em crossCertificatePair} is optional within a Directory entry
belonging to the object class {\em certificationAuthority}. If the caller's Directory entry
already contains an attribute of
type {\em crossCertificatePair}, the certificate pair {\em cpair} is added as value
to that attribute; otherwise, an attribute of type {\em crossCertificatePair} will be
created in the Directory entry, its initial value being specified by {\em cpair}.

The caller's X.500 Directory entry must belong to the object class
{\em certificationAuthority}; otherwise, an attribute error will be reported.

The return value is 0 if the Directory entry succeeds.

The return value is -1 for one of the following reasons:
\bi
\m the Directory service does not respond (ENODIR);
\m the caller's Directory entry does not belong to the object class
{\em certificationAuthority} (EOBJCLASS);
\m the Directory access control refuses this operation (EACCDIR).
\ei
{\em af\_afdb\_enter\_CertificatePair} enters the cross certificate pair {\em cpair} 
into the caller's local .af-db Directory entry. The caller's distinguished name used to
identify the .af-db Directory entry is derived from
the global variable {\em directory\_user\_dname}. At least one of the two components 
(forward or reverse certificate) of the cross certificate pair must be present.



\subsubsection{af\_dir\_retrieve\_CertificatePair}
\nm{3X}{af\_dir\_retrieve\_CertificatePair}{Retrieve Pair of Cross Certificates from Directory}
\addtotoc{af\_afdb\_retrieve\_CertificatePair}

\hl{Synopsis}

\#include $<$af.h$>$ 

SET\_OF\_CertificatePair {\bf *af\_dir\_retrieve\_CertificatePair} {\em (dname)} \\
DName {\em *dname};

SET\_OF\_CertificatePair {\bf *af\_afdb\_retrieve\_CertificatePair} {\em (dname)} \\
DName {\em *dname};
\hl{Parameter}
\parname  {{\em dname}}
\pardescript  {Distinguished name of certification authority which is the owner of the requested
cross certificate pairs} 

\parname  {{\em Return Value}}
\pardescript  {The set of cross certificate pairs or NULL pointer.}
\hl{Description}
{\em af\_dir\_retrieve\_CertificatePair} returns the set of cross certificate pairs 
stored in the X.500 Directory entry identified by {\em dname}. The set of cross certificate pairs 
comprises the value(s) of the (multi-valued) {\em crossCertificatePair} attribute. \\
Each cross certificate pair is composed of one forward certificate and one reverse certificate;
at least one of the two components must be present in a cross certificate pair.

The return value is the NULL pointer for one of the following reasons:
\bi
\m the Directory service does not respond (ENODIR);
\m the parameter {\em dname} does not identify a Directory object (ENAMEDIR);
\m no attribute of type {\em crossCertificatePair} is found in the Directory entry
identified by {\em dname} (EATTRDIR).
\ei
{\em af\_afdb\_retrieve\_CertificatePair} returns the set of cross certificate pairs 
stored in the local .af-db Directory entry identified by {\em dname}. 
The set of cross certificate pairs 
comprises the value of the CrossCertificatePair entry. \\
Each cross certificate pair is composed of one forward certificate and one reverse certificate;
at least one of the two components must be present in a cross certificate pair.



\subsubsection{af\_dir\_delete\_CertificatePair}
\nm{3X}{af\_dir\_delete\_CertificatePair}{Remove Certificate Pair from Directory}
\addtotoc{af\_afdb\_delete\_CertificatePair}

\hl{Synopsis}

\#include $<$af.h$>$

RC {\bf af\_dir\_delete\_CertificatePair} {\em (cpair)} \\
CertificatePair {\em *cpair};

RC {\bf af\_afdb\_delete\_CertificatePair} {\em (cpair)} \\
CertificatePair {\em *cpair};
\hl{Parameter}
\parname  {{\em cpair}}
\pardescript  {The cross certificate pair which is to be removed from the Directory.}  

\parname  {{\em Return Value}}
\pardescript  {0 or -1.}
\hl{Description}
{\em af\_dir\_delete\_CertificatePair} removes the cross certificate pair {\em cpair} 
as value from the {\em crossCertificatePair} attribute within the caller's Directory 
entry. The caller's distinguished name is derived from
the global variable {\em directory\_user\_dname} which is used to authenticate the caller to the
Directory System Agent (DSA) initially contacted. At least one of the two components 
(forward or reverse certificate) of the cross 
certificate pair must be present. Each component is identified by its serial number
and its issuer.

The caller of {\em af\_dir\_delete\_CertificatePair} must be the owner of the Directory
entry from where the cross certificate pair {\em cpair} is to be removed; otherwise, the 
Directory access control will refuse this operation.

If {\em cpair} is the only value of the {\em crossCertificatePair} attribute, the 
{\em crossCertificatePair} attribute will be removed from the Directory entry.

The return value is 0 if the Directory entry succeeds.

The return value is -1 for one of the following reasons:
\bi
\m the Directory service does not respond (ENODIR);
\m the caller's distinguished name does not identify a Directory object (ENAMEDIR);
\m no attribute of type {\em crossCertificatePair} is found in the Directory entry 
associated with the caller (EATTRDIR);
\m the specified cross certificate pair does not exist in the Directory entry 
identified by the caller's distinguished name (EOBJ);
\m the Directory access control refuses this operation (EACCDIR).
\ei
{\em af\_afdb\_delete\_CertificatePair} removes the cross certificate pair {\em cpair} 
from the caller's local .af-db Directory entry. The caller's distinguished name used to
identify the .af-db Directory entry is derived from
the global variable {\em directory\_user\_dname}. At least one of the two components 
(forward or reverse certificate) must be present in order to identify the cross certificate
pair that is to be removed.



\subsubsection{af\_dir\_enter\_CRL}
\nm{3X}{af\_dir\_enter\_CRL}{Enter PEM Revocation List into Directory}
\addtotoc{af\_afdb\_enter\_CRL}

\hl{Synopsis}

\#include $<$af.h$>$

RC {\bf af\_dir\_enter\_CRL} {\em (crl)} \\
CRL {\em *crl};

RC {\bf af\_afdb\_enter\_CRL} {\em (crl)} \\
CRL {\em *crl};
\hl{Parameter}
\parname  {{\em crl}}
\pardescript  {Pointer to a revocation list with PEM syntax which is to replace the existing 
attribute value.} 

\parname  {{\em Return Value}}
\pardescript  {0 or -1.}
\hl{Description}
{\em af\_dir\_enter\_CRL} enters the revocation list {\em crl} into 
the Directory entry associated with the issuer of that revocation list.
The format of the revocation list is that defined by PEM (see RFC 1422).

The caller of {\em af\_dir\_enter\_CRL} must be the issuer of the revocation list
{\em crl}; otherwise, the Directory access control will 
refuse this operation.

The caller of {\em af\_dir\_enter\_CRL} must belong to the object class 
{\em certificationAuthority}.

The return value is 0 if the Directory entry succeeds.

The return value is -1 for one of the following reasons:
\bi
\m the Directory service does not respond (ENODIR);
\m the distinguished name of the issuer of the revocation list does not identify a Directory 
object (ENAMEDIR);
\m the Directory entry associated with the issuer of {\em crl} does not belong to the
object class {\em certificationAuthority} (EATTRDIR);
\m the Directory access control refuses this operation (EACCDIR).
\ei
{\em af\_afdb\_enter\_CRL} enters the revocation list {\em crl} 
into the local .af-db Directory entry associated with the issuer of {\em crl}.



\subsubsection{af\_dir\_retrieve\_CRL}
\nm{3X}{af\_dir\_retrieve\_CRL}{Retrieve PEM Revocation List from Directory}
\addtotoc{af\_afdb\_retrieve\_CRL}

\hl{Synopsis}

\#include $<$af.h$>$

CRL {\bf *af\_dir\_retrieve\_CRL} {\em (dname)} \\
DName {\em *dname};

CRL {\bf *af\_afdb\_retrieve\_CRL} {\em (dname)} \\
DName {\em *dname};
\hl{Parameter}
\parname  {{\em dname}}
\pardescript  {Distinguished name of the certification authority which is the issuer of 
the requested revocation list.} 

\parname  {{\em Return Value}}
\pardescript  {Address of a PEM revocation list or NULL pointer.}
\hl{Description}
{\em af\_dir\_retrieve\_CRL} reads the attribute of type {\em pemCRL} from the Directory entry 
identified by {\em dname}, and interprets its value as a revocation list whose underlying
syntax is that defined by PEM (see RFC 1422).
{\em af\_dir\_retrieve\_CRL} returns the address of that revocation list.

The return value is the NULL pointer for one of the following resons:
\bi
\m the Directory service does not respond (ENODIR);
\m the parameter {\em dname} does not identify a Directory object (ENAMEDIR);
\m no attribute of type {\em pemCRL} is found in the Directory entry, that is, the 
Directory entry identified by {\em dname} does not belong to 
the object class {\em certificationAuthority} (EATTRDIR).
\ei
{\em af\_afdb\_retrieve\_CRL} returns the revocation list 
stored in the local .af-db Directory entry identified by {\em dname}.



\subsubsection{af\_cadb\_add\_user}
\nm{3X}{af\_cadb\_add\_user}{Add User to CA Database}
\label{af_cadb_add_user}

\hl{Synopsis}

\#include $<$af.h$>$

RC {\bf af\_cadb\_add\_user} {\em (name, cadir)} \\
Name {\em *name}; \\
char {\em *cadir};
\hl{Parameter}
\parname  {{\em name}}
\pardescript  {Name of the user who is to be stored in the CA database} 

\parname  {{\em cadir}}
\pardescript  {Name of the CA directory, where the CA database resides}

\parname  {{\em Return Value}}
\pardescript  {0 or -1}
\hl{Description}
af\_cadb\_add\_user inserts the name {\em name} as new entry into the calling CA's {\em user} 
database which resides in the CA directory {\em cadir}.

A  zero return indicates no error.
A -1 return indicates an error.



\subsubsection{af\_cadb\_get\_user}
\nm{3X}{af\_cadb\_get\_user}{Get all Certificates issued to a User}
\label{af_cadb_get_user}
\hl{Synopsis}
\#include $<$af.h$>$

SET\_OF\_IssuedCertificate *{\bf af\_cadb\_get\_user} {\em (name, cadir)} \\
Name {\em *name}; \\
char {\em *cadir};
\hl{Parameter}
\parname  {{\em name}}
\pardescript  {Name of the user whose certificates are requested} 

\parname  {{\em cadir}}
\pardescript  {Name of the CA directory, where the CA database resides}

\parname  {{\em Return Value}}
\pardescript  {Address of a Set of Issued Certificates or NULL pointer}
\hl{Description}
af\_cadb\_get\_user returns the address of a set whose elements represent the certificates 
which have been issued to the user identified by {\em name}. Each member of the set identifies
a certificate by its serial number and indicates the certificate's date of issue.

A NULL pointer return indicates
\bi
\m that there is no entry with key {\em name} stored in the CA's {\em user} database.
\m an error.
\ei



\subsubsection{af\_cadb\_add\_Certificate}
\nm{3X}{af\_cadb\_add\_Certificate}{Add Certificate to CA Database}
\label{af_cadb_add_Certificate}
\hl{Synopsis}
\#include $<$af.h$>$

RC {\bf af\_cadb\_add\_Certificate} {\em (keytype, newcert, cadir)} \\
KeyType     {\em keytype}; \\
Certificate {\em *newcert}; \\
char	    {\em *cadir};
\hl{Parameter}
\parname  {{\em keytype}}
\pardescript  {Type of the certificate: \\
                    SIGNATURE: user certificate of the signature key \\
                    ENCRYPTION: user certificate of the encryption key} 

\parname  {{\em newcert}}
\pardescript  {Certificate which is to be inserted into the CA database}

\parname  {{\em cadir}}
\pardescript  {Name of the CA directory, where the CA database resides} 

\parname  {{\em Return Value}}
\pardescript  {0 or -1}
\hl{Description}
af\_cadb\_add\_Certificate inserts the certificate identified by {\em newcert} into the 
calling CA's database which resides in the CA directory {\em cadir}. 

The certificate identified by {\em newcert} is first inserted into the CA's {\em cert} database;
the key under which the certificate is placed is derived from the certificate's serial number.

It is then added to the set of certificates placed in the CA's {\em user} database under the key
which is derived from the name of the certificate owner.

A zero return indicates no error.
A -1 return indicates an error.



\subsubsection{af\_cadb\_get\_Certificate}
\nm{3X}{af\_cadb\_get\_Certificate}{Get Certificate from CA Database}
\label{af_cadb_get_Certificate}
\hl{Synopsis}
\#include $<$af.h$>$

Certificate *{\bf af\_cadb\_get\_Certificate} {\em (serial, cadir)} \\
OctetString {\em *serial}; \\
char {\em *cadir};
\hl{Parameter}
\parname  {{\em serial}}
\pardescript  {Serial number of requested certificate} 

\parname  {{\em cadir}}
\pardescript  {Name of the CA directory, where the CA database resides}

\parname  {{\em Return Value}}
\pardescript  {Address of a Certificate or NULL pointer}
\hl{Description}
af\_cadb\_get\_Certificate accesses the calling CA's {\em cert} database and returns the address of 
the certificate identified by serial number {\em serial}. 

A NULL pointer return indicates
\bi
\m that the requested certificate is not found in the CA's {\em cert} database.
\m an error.
\ei



\subsubsection{af\_cadb\_list\_user}
\nm{3X}{af\_cadb\_list\_user}{Get Names of all Certified Users}
\label{af_cadb_list_user}
\hl{Synopsis}
\#include $<$af.h$>$

SET\_OF\_Name *{\bf af\_cadb\_list\_user} {\em (cadir)} \\
char {\em *cadir};
\hl{Parameter}
\parname  {{\em cadir}}
\pardescript  {Name of the CA directory, where the CA database resides} 

\parname  {{\em Return Value}}
\pardescript  {Address of a Set of Names or NULL pointer}
\hl{Description}
af\_cadb\_list\_user traverses the calling CA's {\em user} database and returns all keys
stored there. Each key represents the name of a user who has been certified by the
calling CA.

A NULL pointer return indicates
\bi
\m that the CA's {\em user} database is empty.
\m an error.
\ei




\subsubsection{af\_cadb\_add\_CRLWithCertificates}
\nm{3X}{af\_cadb\_add\_CRLWithCertificates}{Add Revocation List to CA Database}
\label{af_cadb_add_CRLWithCertificates}
\hl{Synopsis}
\#include $<$af.h$>$

RC {\bf af\_cadb\_add\_CRLWithCertificates} {\em (crlwithcerts, cadir)} \\
CRLWithCertificates	{\em *crlwithcerts}; \\
char		{\em *cadir};
\hl{Parameter}
\parname  {{\em crlwithcerts}}
\pardescript  {Revocation list with accompanying originator certificate} 

\parname  {{\em cadir}}
\pardescript  {Name of the CA directory, where the CA database resides}

\parname  {{\em Return Value}}
\pardescript  {0 or -1}
\hl{Description}
af\_cadb\_add\_CRLWithCertificates inserts the revocation list with accompanying originator certificate
(identified by {\em crlwithcerts}) into the calling CA's {\em crl} database. 

The key under which the revocation list and accompanying originator certificate are placed is derived 
from the name of the issuer of the revocation list.

af\_cadb\_add\_CRLWithCertificates replaces an existing entry if it has the same key.

This routine supports the CRL-storage request mechanism defined by PEM (RFC 1424).

A  zero return indicates no error.
A -1 return indicates an error.




\subsubsection{af\_cadb\_list\_CRLWithCertificates}
\nm{3X}{af\_cadb\_list\_CRLWithCertificates}{Get all Revocation Lists from CA Database}
\label{af_cadb_list_CRLWithCertificates}
\hl{Synopsis}
\#include $<$af.h$>$

SET\_OF\_CRLWithCertificates *{\bf af\_cadb\_list\_CRLWithCertificates} {\em (cadir)} \\
char	{\em *cadir};
\hl{Parameter}
\parname  {{\em cadir}}
\pardescript  {Name of the CA directory, where the CA database resides} 

\parname  {{\em Return Value}}
\pardescript  {Address of a Set of Revocation Lists or NULL pointer.}
\hl{Description}
af\_cadb\_list\_CRLWithCertificates traverses the calling CA's {\em crl} database and returns all data
stored there. Each datum represents the revocation list of a known CA and is accompanied by
the originator certificate of that CA.

This routine supports the CRL-retrieval reply mechanism defined by PEM (RFC 1424).

A NULL pointer return indicates
\bi
\m that the CA's {\em crl} database is empty.
\m an error.
\ei




\subsubsection{af\_cadb\_get\_CRLWithCertificates}
\nm{3X}{af\_cadb\_get\_CRLWithCertificates}{Get Revocation List from CA Database}
\label{af_cadb_get_CRLWithCertificates}
\hl{Synopsis}
\#include $<$af.h$>$

CRLWithCertificates *{\bf af\_cadb\_get\_CRLWithCertificates} {\em (name, cadir)} \\
Name	{\em *name}; \\
char	{\em *cadir};
\hl{Parameter}
\parname  {{\em name}}
\pardescript  {Name of the issuer of a required revocation list} 

\parname  {{\em cadir}}
\pardescript  {Name of the CA directory, where the CA database resides}

\parname  {{\em Return Value}}
\pardescript  {Address of a Revocation List or NULL pointer.}
\hl{Description}
af\_cadb\_get\_CRLWithCertificates accesses the calling CA's {\em crl} database and returns the 
address of the revocation list issued by the CA with name {\em name}. The requested revocation list
is accompanied by its issuer's originator certificate.

This routine supports the CRL-retrieval reply mechanism defined by PEM (RFC 1424).

A NULL pointer return indicates
\bi
\m that the requested revocation list is not found in the CA's {\em crl} database.
\m an error.
\ei


\subsection{ASN.1 Encoding and Decoding Functions}
\nm{3X}{e\_AF}{ASN.1 Encoding and decoding functions for AF-IF}
\label{Encoding-/Decoding-Routinen}

\hl{Synopsis}

\#include $<$af.h$>$ 

OctetString *{\bf e\_DName} {\em (name)} \\
DName *{\em name};

DName *{\bf d\_DName} {\em (octetstring)} \\
OctetString *{\em octetstring};

OctetString *{\bf e\_Certificate} {\em (certificate)} \\
Certificate *{\em certificate};

Certificate *{\bf d\_Certificate} {\em (octetstring)} \\
OctetString *{\em octetstring};

OctetString *{\bf e\_CertificateSet} {\em (certificateset)} \\
SET\_OF\_Certificate *{\em certificateset};

SET\_OF\_Certificate *{\bf d\_CertificateSet} {\em (octetstring)} \\
OctetString *{\em octetstring};

OctetString *{\bf e\_CertificatePairSet} {\em (cpairset)} \\
SET\_OF\_CertificatePair *{\em cpairset};

SET\_OF\_CertificatePair *{\bf d\_CertificatePairSet} {\em (octetstring)} \\
OctetString *{\em octetstring};

OctetString *{\bf e\_ToBeSigned} {\em (tobesigned)} \\
ToBeSigned *{\em tobesigned};

OctetString *{\bf e\_FCPath} {\em (fcpath)} \\
FCPath *{\em fcpath};

FCPath *{\bf d\_FCPath} {\em (octetstring)} \\
OctetString *{\em octetstring};

OctetString *{\bf e\_Certificates} {\em (certificates)} \\
Certificates *{\em certificates};

Certificates *{\bf d\_Certificates} {\em (octetstring)} \\
OctetString *{\em octetstring};

OctetString *{\bf e\_PKRoot} {\em (pkroot)} \\
PKRoot *{\em pkroot};

PKRoot *{\bf d\_PKRoot} {\em (octetstring)} \\
OctetString *{\em octetstring};

OctetString *{\bf e\_PKList} {\em (pklist)} \\
PKList *{\em pklist};

PKList *{\bf d\_PKList} {\em (octetstring)} \\
OctetString *{\em octetstring};

OctetString *{\bf e\_Attribute} {\em (attr)} \\
Attribute *{\em attr};

OctetString *{\bf e\_AttributeType} {\em (attrtype)} \\
AttributeType *{\em attrtype};

OctetString *{\bf e\_AttributeValueAssertion} {\em (ava)} \\
AttributeValueAssertion *{\em ava};

OctetString *{\bf e\_CRL} {\em (crl)} \\
CRL *{\em crl};

CRL *{\bf d\_CRL} {\em (octetstring)} \\
OctetString *{\bf octetstring};

OctetString *{\bf e\_CRLTBS} {\em (tbs)} \\
CRLTBS *{\em tbs};

OctetString *{\bf e\_CRLEntry} {\em (revcertpem)} \\
CRLEntry *{\em revcertpem};

CRLEntry *{\bf d\_CRLEntry} {\em (octetstring)} \\
OctetString *{\bf octetstring};

OctetString *{\bf e\_CRLEntrySequence} {\em (seq)} \\
SEQUENCE\_OF\_CRLEntry *{\em seq};

SEQUENCE\_OF\_CRLEntry *{\bf d\_CRLEntrySequence} {\em (octetstring)} \\
OctetString *{\bf octetstring};

OctetString *{\bf e\_CRLWithCertificates} {\em (arg)} \\
CRLWithCertificates *{\em arg};

CRLWithCertificates *{\bf d\_CRLWithCertificates} {\em (octetstring)} \\
OctetString *{\bf octetstring};

OctetString *{\bf e\_AliasList} {\em (alist)} \\
AliasList *{\em alist};

AliasList *{\bf d\_AliasList} {\em (octetstring)} \\
OctetString *{\bf octetstring};

OctetString *{\bf e\_OCList} {\em (ocl)} \\
OCList *{\em ocl};

OCList *{\bf d\_OCList} {\em (octetstring)} \\
OctetString *{\bf octetstring};

OctetString *{\bf e\_CrlTBS} {\em (rcltbs)} \\
CrlTBS *{\em rcltbs};

OctetString *{\bf e\_SET\_OF\_IssuedCertificate} {\em (isscertset)} \\
SET\_OF\_IssuedCertificate  *{\em isscertset};

SET\_OF\_IssuedCertificate  *{\bf d\_SET\_OF\_IssuedCertificate} {\em (octetstring)}
OctetString *{\em octetstring};
\hl{Description}     
The functions e\_* (Encode) transform the given structure into an OctetString
and return its address. Memory of the entire structure is provided 
by the called program and can be released
by aux\_free\_OctetString().

The functions d\_* (Decode) transform the given OctetString into the indicated structure and return
a pointer to that structure. Memory of the entire structure is provided 
by the called program and can be released by the respective aux\_free* function (see~\ref{aux_free}).

OctetString consists of the ASN.1 encoding of the corresponding ASN.1 description of the C-structure
(see AF-PSE-OBJECTS(4) and AF-DIR-OBJECTS(4)).

\subsection{PSE Objects}
\nm{4}{AF-PSE-objects}{PSE objects of the AF-interface}
\label{af-sw-cc}

\hl{Synopsis}

\#include $<$af.h$>$

\hl{Description}
     
This section lists PSE objects needed by the af\_* functions. 
Each object is presented both in terms
of C-language structures as they are needed by the interface subroutines and
by the application programmer, and in terms of corresponding ASN.1 structures
which are needed as references for the bit coding of the object.
 
Each AF PSE object is wrapped in the following ASN.1 structure:
{\small
\bvtab
\2 PSEObject ::= \3 {\bf SEQUENCE} \{ \\
\7 objectType \3 OBJECT IDENTIFIER, \\
\7 objectValue \3 ANY DEFINED BY objectType \\
\5 \}
\evtab
}
The following object identifiers are used:
\bc
\begin{tabular}{|l|l|l|} \hline
{\bf Object Name} & {\bf Object Identifier} & {\bf Content} \\ \hline
SignCert       & \{ 1 3 36 2 1 1 \} & certificate of public verification key      \\
EncCert        & \{ 1 3 36 2 1 2 \} & certificate of public encryption key        \\
Cert           & \{ 1 3 36 2 1 3 \} & certificate of public key                    \\
SignCSet       & \{ 1 3 36 2 2 1 \} & cross-certificates of public verification key \\
EncCSet        & \{ 1 3 36 2 2 2 \} & cross-certificates of public encryption key   \\
CSet           & \{ 1 3 36 2 2 3 \} & cross-certificates of public key            \\
CrossCSet      & \{ 1 3 36 2 8 1 \} & cross certificate pairs		     \\
SignSK         & \{ 1 3 36 2 3 1 \} & secret signature key                        \\
DecSKnew       & \{ 1 3 36 2 3 2 \} & secret decryption key (current)             \\
DecSKold       & \{ 1 3 36 2 3 3 \} & secret decryption key (old)                 \\
SKnew          & \{ 1 3 36 2 3 4 \} & secret key (current)                        \\
SKold          & \{ 1 3 36 2 3 5 \} & secret key (old)                            \\
FCPath         & \{ 1 3 36 2 4 1 \} & forward certification path                  \\
PKRoot         & \{ 1 3 36 2 5 1 \} & top level public verification key           \\
PKList         & \{ 1 3 36 2 6 1 \} & list of trusted public verification keys    \\
EKList         & \{ 1 3 36 2 6 2 \} & list of trusted public encryption keys      \\
PCAList        & \{ 1 3 36 2 6 3 \} & list of recognized PCAs      \\
CrlSet         & \{ 1 3 36 2 9 1 \} & list of revocation lists         \\
SerialNumber   & \{ 1 3 36 2 10 1 \} & current serial number (CA only)         \\
EDBKey         & \{ 1 3 36 2 11 1 \} & DSA database encryption key        \\
AliasList      & \{ 1 3 36 2 12 1 \} & user's alias list         \\
QuipuPWD       & \{ 1 3 36 2 13 1 \} & password for X.500 directory access         \\ \hline
\end{tabular}
\ec
The AF PSE objects {\em Cert}, {\em CSet}, {\em SKnew}, and {\em SKold} are used in case
of a PSE which is restricted to hold a single RSA key pair for both signature and encryption purposes.
In this case, there are no objects {\em SignCert}, {\em EncCert}, {\em SignCSet}, {\em EncCSet}, {\em SignSK},
{\em DecSKnew}, and {\em DecSKold} installed on the PSE. \\ [1em]
The object identifier of each object is given in the table above.

The structure of objectValue of each object is given in the sections below.
The whole objects (i.e. the DER-code of PSEObject) can be read/written from/to
the PSE using the functions sec\_read() and sec\_write(). The DER-code
of objectValue can be read/written from/to the PSE using the functions
sec\_read\_PSE() and sec\_write\_PSE().
The transformation between the C-representation and the DER oriented bit coding
is provided by the respective encoding/decoding functions e\_*(3X). 


\subsubsection{SignSK}
\hll{SKnew, SKold, DecSKnew, DecSKold, SignSK}
Secret component of asymmetric key pair.

\underline{C-Structure of the object:} KeyInfo
{\small
\bvtab
\4      typedef struct KeyInfo \{ \\
\6              AlgId        \3 *subjectAI; \\
\6              BitString    \3 subjectkey; \\
\4      \} KeyInfo; \\ \\ \\
with: \\ \\
\4      typedef struct AlgId \{  \\
\6              ObjId  \3 *objid; \\
\6              char         \3 *parm; \\
\4      \} AlgId; \\ \\ \\
\4      typedef struct ObjId \{  \\
\6              int          \3 oid\_nelem; \\
\6              unsigned int \3 *oid\_elements; \\
\4      \} ObjId;  \\  \\ \\
\4      typedef struct BitString \{  \\
\6              unsigned int \3 nbits; \\
\6              char         \3 *bits; \\
\4      \} BitString;  \\
\evtab
}

\underline{ASN.1-Structure of objectValue:}
{\small
\bvtab
\4 SubjectPrivateKeyInfo
\evtab
}

{\em SubjectPrivateKeyInfo} is defined to have the same format as
{\em SubjectPublicKeyInfo}, cf. X.509, Annex G.
However, note the substructure of the private key BIT STRING.

{\small
\bvtab
\1 SubjectPrivateKeyInfo ::= \5 {\bf SEQUENCE} \{  \\
\7 algorithm         \2 AlgorithmIdentifier,    \\
\7 subjectPrivateKey BitString            \\
\7 -- -- \\
\7 -- -- the BitString of subjectPrivateKey   \\
\7 -- -- contains the ASN.1 encoding of a sequence of \\
\7 -- -- the factors p and q of the RSA-modulus n:   \\
\7 -- -- \\
\7 -- -- SEQUENCE \{ \\
\7 -- -- \2       p INTEGER,   \\
\7 -- -- \2       q INTEGER  \\
\7 -- -- \} \\
\6 \} \\
\evtab
\bvtab
\1 AlgorithmIdentifier ::= \5 {\bf SEQUENCE} \{    \\
\7 algorithm  \3 OBJECT IDENTIFIER, \\
\7 parameters \3 ANY DEFINED BY algorithm  \\
\6 \}
\evtab
}

Encoding/Decoding: e\_KeyInfo(), d\_KeyInfo() \\
Used by: af\_sign()


\subsubsection{Cert}
\hll{Cert, SignCert, EncCert}
X.509 Public key certificate (which is a User Certificate or a CA Certificate) 

\underline{C-Structure of the object:}

{\small
\bvtab
\4      typedef struct Certificate \{  \\
\6              OctetString   \4 *tbs\_DERcode; \\
\6              ToBeSigned    \4 *tbs; \\
\6              Signature     \4 *sig; \\
\4      \} Certificate;  \\ \\ \\
with: \\ \\
\4      typedef struct Signature \{  \\
\6              AlgId         \3 *signAI; \\
\6              BitString     \3 signature; \\
\4 \}           Signature;  \\ \\ \\
\4      typedef struct ToBeSigned \{ \\
\6              int           \3 version; \\
\6              OctetString   \3 *serialnumber; \\
\6              AlgId         \3 *signatureAI; \\
\6              DName          \3 *issuer; \\
\6              UTCTime       \3 *notbefore; \\
\6              UTCTime       \3 *notafter; \\
\6              DName          \3 *subject; \\
\6              KeyInfo       \3 *subjectPK; \\
\4      \} ToBeSigned;  \\ \\ \\
\4      typedef struct KeyInfo \{  \\
\6              AlgId         \3 *subjectAI; \\
\6              BitString     \3 subjectkey; \\
\4      \} KeyInfo; \\ \\ \\
\4      typedef struct OctetString \{ \\
\6              unsigned int  \3 noctets; \\
\6              char          \3 *octets; \\
\4      \} OctetString;  \\ \\ \\
\4      typedef char UTCTime;
\evtab
}
and AlgId, BitString as in SKnew.

\underline{ASN.1-Structure of objectValue:}
{\small
\bvtab
\1 Cert ::=    \3 Certificate \\ \\
\1 Certificate ::= \3 {\bf SIGNED SEQUENCE} \{  \\
\5 version [0]          \5 INTEGER DEFAULT 0,          \\
\5 serialNumber         \5 INTEGER,                    \\
\5 signature            \5 AlgorithmIdentifier,        \\
\5 issuer               \5 Name,                       \\
\5 validity             \5 Validity,                   \\
\5 subject              \5 Name,                       \\
\5 subjectPublicKeyInfo \5 SubjectPublicKeyInfo     \\
\4 \}
\evtab
}
 
{\em Certificate} is defined by X.509, Annex G.
{\em Algorithm\-Identifier} is defined in X.509, Annex G,
as a SEQUENCE of an Object Identifier and a parameters type
which depends on the algorithm.
{\em Name} is defined in X.501, Annex C,
as a SEQUENCE of Relative Distinguished Names,
which are sets of Attribute-Value-Assertions.
{\em Validity} is defined in X.509, Annex G,
as a SEQUENCE of the UTCTimes {\em notBefore} and {\em notAfter}.


{\em SubjectPublicKeyInfo} is defined by X.509, Annex G.
The substructure of the public key BitString is defined by
X.509, Annex C, C.3.
\\ [1em]

{\small
\bvtab
\1 SubjectPublicKeyInfo ::= \5 {\bf SEQUENCE} \{  \\
\7 algorithm         \2 AlgorithmIdentifier,    \\
\7 subjectPublicKey  BitString            \\
\7 -- -- \\
\7 -- -- the BitString of subjectPublicKey   \\
\7 -- -- contains the ASN.1 encoding of a sequence of \\
\7 -- -- the RSA-modulus n and the pubRSA-exponent e:   \\
\7 -- -- \\
\7 -- -- SEQUENCE \{ \\
\7 -- -- \2       n INTEGER,   \\
\7 -- -- \2       e INTEGER  \\
\7 -- -- \} \\
\6 \} \\
\evtab
}

See also X.509, Annex G, for the SIGNED MACRO,
which is replaced by the following MACRO:

{\small
\begin {center}
\begin {tabular}{lll}
SIGNED MACRO   ::= & & \\
BEGIN              & & \\
TYPE NOTATION  ::= & type(ToBeSigned)     & \\
VALUE NOTATION ::= & type(VALUE           & \\
		   & SEQUENCE \{          & \\
		   & ToBeSigned,          & \\
		   & AlgorithmIdentifier, & \\
 & \multicolumn{2}{l} {-- -- of the algorithm used to generate}  \\
 & \multicolumn{2}{l} {-- -- the signature}                      \\
		   & BitString \}          & \\
 & \multicolumn{2}{l} {-- -- where the BIT STRING is the result} \\
 & \multicolumn{2}{l} {-- -- of the encryption of the value}     \\
 & \multicolumn{2}{l} {-- -- of the hash of ToBeSigned}          \\
 & ) & \\
SEC\_END & &
\end {tabular}
\end {center}
}
 
Encoding/Decoding: e\_Certificate(), d\_Certificate() \\
Used by: af\_pse\_get\_Certificate(), af\_pse\_get\_Certificates(), af\_pse\_update\_Certificate(), 

\subsubsection{SignCSet}
\hll{CSet, SignCSet, EncCSet}
CrossCertificates issued for the own public signature or encryption key.

\underline{C-Structure of the object:}

{\small
\begin{center}
typedef SET\_OF(Certificate)	SET\_OF\_Certificate;
\end{center}
}

{\em ASN.1-Structure of objectValue:}
{\small
\begin{center}
{\bf SET OF} Certificate
\end{center}
}
Encoding/Decoding: e\_CertificateSet(), d\_CertificateSet()



\subsubsection{CrossCSet}
\hll{CrossCSet}
Cross Certificate Pairs issued for the own public signature key.

\underline{C-Structure of the object:} 
{\small
\bvtab
\4 typedef SET\_OF(CertificatePair)	SET\_OF\_CertificatePair;
\evtab
}


\underline{ASN.1-Structure of objectValue:}
{\small
\bvtab
\4        CrossCSet ::= {\bf SET OF} CertificatePair \\ [0.5cm]
\4        CertificatePair ::= \4 {\bf SEQUENCE} \{ \\
\9                 forward [0] Certificate OPTIONAL, \\
\9                 reverse [1] Certificate OPTIONAL  \\
\8 \}
\evtab
}

Encoding/Decoding: e\_CertificatePairSet(), d\_CertificatePairSet()

\subsubsection{FCPath}
\hll{FCPath}
Own Forward Certification Path. 

\underline{C-Structure of the object:}

{\small
\bvtab
\4      typedef struct FCPath \{  \\
\6              SET\_OF\_Certificate \4 *liste; \\
\6              FCPath               \4 *next\_forwardpath; \\
\4      \} FCPath;  \\ \\ \\
with: \\ \\
\4      typedef struct set\_of\_Certificate \{ \\
\6              Certificate          \4 *element; \\
\6              SET\_OF\_Certificate \4 *next; \\
\4      \} SET\_OF\_Certificate;  \\
\evtab
}


\underline{ASN.1-Structure of objectValue:}
{\small
\bvtab
\2 FCPath ::=                    \6 ForwardCertificationPath \\
\2 ForwardCertificationPath ::=  \6 {\bf SEQUENCE OF} CrossCertificates \\
\2 CrossCertificates ::=         \6 {\bf SET OF} Certificate
\evtab
}

{\em FCPath} is an element of {\em Certificates} (note the plural).
{\em Certificates} is defined in X.509, Annex G,
as a SEQUENCE of a user certificate and a forward certification path.
The forward certification path is a SEQUENCE of cross certificates,
which are SETs of 1-level certificates.
One cross certificate per SET is supposed to be
the hierarchy certificate of the path.
The data of the PSE object {\em FCPath}
can be composed with the data of
{\em Cert} or of
{\em SignCert} in order to build an
n-level originator certificate of ASN.1-type {\em Certificates}.

{\small
\bvtab
\1 Certificates ::= \3 {\bf SEQUENCE} \{   \\
\5 certificate          \4 Certificate,                \\
\5 certificationPath    \4 ForwardCertificationPath OPTIONAL   \\
\4 \}
\evtab
}

Encoding/Decoding: e\_FCPath(), d\_FCPath() \\
Used by: af\_pse\_get\_FCPath(), af\_pse\_update\_FCPath(), af\_pse\_get\_Certificates()

\subsubsection{PKRoot}
\hll{PKRoot}
Public key of the root CA (actual/old version).

\underline{C-Structure of the object:}

{\small
\bvtab
\4      typedef struct PKRoot \{ \\
\6              DName          \3 *ca; \\
\6              struct Serial \3 *new; \\
\6              struct Serial \3 *old; \\
\4      \} PKRoot;  \\ \\ \\
with: \\ \\
\4      struct Serial \{ \\
\6             OctetString    \3 *serial; \\
\6             int            \3 version; \\
\6             KeyInfo        \3 *key; \\
\6             UTCTime        \3 *notbefore; \\
\6             UTCTime        \3 *notafter; \\
\6             Signature      \3 *sig; \\
\6             Validity       \3 *valid; \\
\4      \}; \\
\evtab
}


\underline{ASN.1-Structure of objectValue:}

{\small
\bvtab
\3 PKRoot ::= \3 {\bf SEQUENCE} \{     \\
\7 cA                \2 Name,                 \\
\7 newkey            \2 RootKey,              \\
\7 oldkey            \2 RootKey OPTIONAL,     \\
\6 \}
\evtab

\bvtab
\3 RootKey ::= \3 {\bf SEQUENCE} \{     \\
\5 version [0]          \5 INTEGER DEFAULT 0,          \\
\5 serialNumber         \5 INTEGER,                    \\
\5 publicKey            \5 SubjectPublicKeyInfo        \\
\5 validity             \5 Validity,                   \\
\5 signAlgId            \5 AlgorithmIdentifier,        \\
\5 signBitString        \5 BIT STRING,                 \\
\4 \}
\evtab
}
 
{\em Name} is defined in X.501, Annex C,
as a SEQUENCE of Relative Distinguished Names,
which are sets of Attribute-Value-Assertions.
{\em SubjectPublicKeyInfo} is defined by X.509, Annex G.


Encoding/Decoding: e\_PKRoot(), d\_PKRoot() \\
Used by: af\_pse\_get\_PKRoot(), af\_pse\_update\_PKRoot(), af\_verify()

\subsubsection{PKList}
\hll{PKList, EKList, PCAList}
{\em Content of the object:} Trusted public keys of communication partners or trusted PCAs.
 
\underline{C-Structure of the object:}

{\small
\bvtab
\4      typedef struct set\_of\_ToBeSigned \{ \\
\6              ToBeSigned   \3 *element; \\
\6              PKList       \3 *next; \\
\4      \} PKList; \\ \\ \\
\evtab
}


\underline{ASN.1-Structure of objectValue:}

{\small
\bvtab
\3 PKList ::= \3 {\bf SET OF SEQUENCE} \{  \\
\7 version [0]     \3 INTEGER DEFAULT 0,       \\
\7 serialNumber    \3 INTEGER,                 \\
\7 signature       \3 AlgorithmIdentifier,     \\
\7 issuer          \3 Name,                    \\
\7 validity        \3 Validity,                \\
\7 partner         \3 Name,                    \\
\7 verificationKey \3 SubjectPublicKeyInfo \\
\6 \}
\evtab
}
 
Compare the definition of {\em Certificate} above. A PKList is defined
as a set of {\em ToBeSigned} values of certificates.
 
In case of PKList and EKList, the {\em verificationKey} is supposed to have the value
of the currently valid public verification key of the {\em partner}.
Before adding it to the list, it should be verified through
the full certification path.


Encoding/Decoding: e\_PKList(), d\_PKList() \\
Used by: af\_pse\_get\_PKList(), af\_pse\_update\_PKList(), af\_pse\_add\_PK(), af\_pse\_update\_PK(),
af\_pse\_delete\_PK(), af\_pse\_get\_owner(), af\_pse\_get\_PCAList(), af\_pse\_update\_PCAList(),
af\_pse\_add\_PCA(), af\_pse\_delete\_PCA()


\subsubsection{CrlSet}
\hll{CrlSet}
Set of Revocation Lists of all known certification authorities.
 
\underline{C-Structure of the object:}

{\small
\bvtab
\4      typedef struct set\_of\_Crl \{ \\
\6              Crl   \3 *element; \\
\6              CrlSet   \3 *next; \\
\4      \} CrlSet; \\ \\ \\
with: \\ \\
\4      typedef struct Crl \{ \\
\5              DName          \5 *issuer; \\
\5              UTCTime        \5 *nextUpdate; \\
\5              SEQUENCE\_OF\_CRLEntry     \5 *revcerts; \\
\4      \} Crl;  \\ \\ \\
with: \\ \\
\4      typedef struct CRLEntry \{ \\
\5              OctetString          \5 *serialnumber; \\
\5              UTCTime        \5 *revocationDate; \\
\4      \} CRLEntry;  \\ \\ \\
and: \\ \\
\4      typedef struct sequence\_of\_CRLEntry \{ \\
\5              int          \5 element; \\
\5              SEQUENCE\_OF\_CRLEntry \5 *next; \\
\4      \} SEQUENCE\_OF\_CRLEntry;  \\
\evtab
}


\underline{ASN.1-Structure of objectValue:}

{\small
\bvtab
\3 CrlSet ::= \3 {\bf SET OF SEQUENCE} \{  \\
\7 issuer          \3 Name,                    \\
\7 nextUpdate      \3 UTCTime,                \\
\7 serialset       \3 SET OF INTEGER          \\
\6 \}
\evtab
}
 
Encoding/Decoding: e\_CrlSet(), d\_CrlSet() \\
Used by: af\_pse\_get\_CrlSet(), af\_pse\_update\_CrlSet()


\subsubsection{SerialNumber}
\hll{SerialNumber}
Serial Number used by a certification authority.
 
{\em SerialNumber} holds the value of the serial number which will be assigned to the certificate 
that will next be issued by the certification authority. \\

\underline{C-Structure of the object:} OctetString

In order to support the size-independent processing of serial numbers
which may be at least 128 bits in length (as requested by RFC 1422, 3.3.2) and therefore cannot
be represented as an integer value, serial numbers are internally represented as {\em OctetString} structures. \\

\underline{ASN.1-Structure of objectValue:}

{\small
\bvtab
\3 SerialNumber ::= \4 OCTETSTRING     \\
\evtab
}

Encoding/Decoding: e\_OctetString(), d\_OctetString() \\
Used by: af\_pse\_get\_SerialNumber(), af\_pse\_update\_SerialNumber()


\subsubsection{QuipuPWD}
\hll{QuipuPWD}
{\em Content of the object:} Password used for simple authentication to the X.500 Directory (Quipu). \\

\underline{C-Structure of the object:} char \\


\underline{ASN.1-Structure of objectValue:}

{\small
\bvtab
\3 QuipuPWD ::= \4 GRAPHIC STRING     \\
\evtab
}

Encoding/Decoding: e\_GRAPHICString(), d\_GRAPHICString() \\
Used by: af\_pse\_get\_QuipuPWD(), af\_pse\_update\_QuipuPWD()


\subsubsection{EDBKey}
\hll{EDBKey}
{\em Content of the object:} DES key used for encrypting/decrypting a DSA's EDB database. \\

\underline{C-Structure of the object:} KeyInfo \\


\underline{ASN.1-Structure of objectValue:} 
{\small
\bvtab
\1 {\bf SEQUENCE} \{  \\
\4 algorithm    \2 AlgorithmIdentifier,    \\
\4 Key 		\2 BitString            \\
\1 \} \\
\evtab
}

Encoding/Decoding: e\_KeyInfo(), d\_KeyInfo()


\subsubsection{Name}
\hll{Name}
Distinguished names may optionally be stored as PSE-objects with no pre-defined object names. \\

\underline{C-Structure of the object:} DName

{\small
\bvtab
\2 typedef struct DName DName;                  \\
\2 typedef struct RDName RDName;                \\
\2 typedef struct AVA AVA;                      \\
\2 struct DName \{                              \\
\4        RDName       \3 *element\_IF\_2;      \\
\4        DName        \3 *next;                \\
\2 \}                                           \\ [1em]
\2 struct RDName \{                             \\
\4        AVA          \3 *member\_IF\_0;       \\
\4        RDName       \3 *next;                \\
\2 \};                                          \\ [1em]
\2 struct AVA \{                                \\ 
\4        ObjId  \3 *element\_IF\_0;            \\
\4        PElement     \3 *element\_IF\_1;      \\
\2 \};                                          \\
\evtab
}

\underline{ASN.1-Structure of objectValue:}

{\small
\begin {center}
\begin {tabular}{ll}
Name & ::= DistinguishedName    \\
DistinguishedName & ::= RDNSequence              \\
RDNSequence  & ::= {\bf SEQUENCE OF} RelativeDistinguishedName \\
 & \\
RelativeDistinguishedName & ::= {\bf SET OF} AttributeValueAssertion \\
 & \\
AttributeValueAssertion   &
   ::= {\bf SEQUENCE} \{ AttributeType, AttributeValue \} \\
 & \\
AttributeType  &  ::= {\bf OBJECT IDENTIFIER} \\
AttributeValue &  ::= {\bf ANY} -- -- defined by AttributeType
\end {tabular}
\end {center}
}
The X.500 recommendations on the directory specify
distinguished names for the purpose of unique identification of the directory objects
(X.501, Directory Information Model).
For example, distinguished names are assigned to human users.
Also, issuer and subject of a certificate are identified by their distinguished names.
The format of distinguished names is defined in
X.501, Annex C.
The directory attribute {\em DistinguishedName} is a sequence of {\em Relative DistinguishedName}
which in turn is a set of attributeType/attributeValue pairs.
Within SecuDe
only a limited set of attribute types is used to construct relative distinguished names,
these are:
countryName, organizationalName, organizationalUnitName, commonName, surname,
title, serialNumber, localityName, streetAddress, and businessCategory.
All other attribute types would be ignored. SecuDE uses the X.520 defined object identifiers
for the identification of attribute types.

The corresponding C structure {\em DName} uses the ISODE defined presentation element
PElement to carry the attribute value. 
\\ [1em]
Encoding/Decoding: e\_DName(), d\_DName(),

which perform the transformations between the C-structure {\em DName} and the ASN.1 DER code.
There are other name transformation functions which transform between
the printable representation {\em Name} and the C-structure {\em DName}: \\
aux\_Name2DName(), aux\_DName2Name() (see AUX\_NAME(3X)).


\subsection{Directory Objects}
\nm{4}{AF-Dir-Objects}{Directory Objects of the AF-Interface}
\label{af-dirobj}

\hl{Synopsis}

\#include $<$af.h$>$

\hl{Description}
     
This section lists directory objects needed by the
directory oriented af\_dir\_retrieve\_* and af\_dir\_enter\_* functions.
Each object is presented in terms of C-language structures
as they are needed by the interface subroutines and
by the application programmer.
The bit coding of the objects is given by applying
ASN.1 Basic Encoding Rules to the corresponding ASN.1 structures as they
are defined in Vol. 2, 2.1. Transformation between those two representations
is provided by the respective encoding/decoding functions e\_*(3X).

\subsubsection{Directory Attributes}
\hll{Attribute}
Directory attribute type and values. \\
\underline{C-Structure of the object:}

{\small
\btab
\1      typedef struct Attribute \{ \\
\2              ObjId \2   type; \\
\2              SET\_OF \2      *value; \\
\1       \} Attribute; \\ \\ \\
with:   \\ \\
\1      typedef struct set\_of \{ \\
\2              char  \2          *element; \\
\2              struct set\_of \2 *next; \\
\1      \} SET\_OF; \\
\etab
}

The structure SET\_OF can be used by any ASN.1 type ``SET OF'',
in that the subfield {\em *element} is only a placeholder
of an address of the respective SET element.
In particular, attributes are described by an attribute type
and a set of attribute values. In the following example,
the use of these C-structures is demonstrated with
the attribute type ``Certificate''.

{\small
\btab
\1 /* example use: */   \\ \\
\2      SET\_OF     \2   *attrvalue; \\
\2      Attribute   \2   *acert;     \\
\2      Certificate \2   *cert;      \\ \\
\2      for( attrvalue = acert$->$value;  \\
\3         attrvalue != (SET\_OF *)0;  \\
\3         attrvalue = attrvalue$->$next ) \\
\2         \{ /* work on */ cert = (Certificate *)attrvalue$->$element; \} \\
\etab
}

Encoding (and decoding) of an attribute depends on its attribute type.
Thus, before encoding (and decoding) {\em value}\pf {\em element},
the ObjId {\em type} is evaluated.
\\
Encoding/Decoding: e\_Attribute(), d\_Attribute() \\
Used by: af\_dir\_retrieve\_*(), af\_dir\_enter\_*()

\subsubsection{CRL}
\hll{CRL}
Revocation List with PEM syntax. \\
\underline{C-Structure of the object:}

{\small
\btab
\1      typedef struct CRL \{  \\
\2              OctetString \5  *tbs\_DERcode; \\
\2              CRLTBS  \5   *tbs;        \\
\2              Signature \5    *sig; \\
\1      \} CRL;  \\ \\
with: \\ \\
\1      typedef struct CRLTBS \{ \\
\2              AlgId \5        *signatureAI; \\
\2              DName \5         *issuer; \\
\2              SEQUENCE\_OF\_CRLEntry \5     *revokedCertificates; \\
\2              UTCTime  \5     *lastUpdate; \\
\2              UTCTime  \5     *nextUpdate; \\
\1       \} CRLTBS; \\ \\
\1      typedef struct Signature \{  \\
\2              AlgId \5        *signAI; \\
\2              BitString \5    signature; \\
\1 \}           Signature;  \\ \\
\1      typedef struct CRLEntry \{ \\
\2              OctetString  \5          *serialnumber; \\
\2              UTCTime \5      *revocationDate; \\
\1      \} CRLEntry;  \\ \\
with: \\ \\
\1      typedef struct sequence\_of\_CRLEntry \{ \\
\2              CRLEntry          \5 *element; \\
\2              SEQUENCE\_OF\_CRLEntry \5 *next; \\
\1      \} SEQUENCE\_OF\_CRLEntry;  \\
\etab
}

Below, a search in a list of revoked certificates is demonstrated:

{\small
\btab
\1 /* Search revoked certificates: */ \\ \\
\1      CRL   \4 *crl;    \\
\1      SEQUENCE\_OF\_CRLEntry \4 *seq;  \\
\1      /* First verify crl$->$sig with crl$->$tbs\_DERcode. Then: */ \\
\1      for( seq = crl$->$tbs$->$revokedCertificates; \\
\2        seq != (SEQUENCE\_OF\_CRLEntry *)0; \\
\2        seq = seq$->$next ) \\
\1      \{ /* check seq$->$element$->$serialnumber and seq$->$element$->$revocationDate */ \} \\
\etab
}

Encoding/Decoding: e\_CRL(), d\_CRL() \\
Used by: af\_dir\_retrieve\_CRL(), af\_dir\_enter\_CRL()


\subsection{CA Database Objects}
\nm{4}{CA-Database-objects}{CA Database objects of the AF-interface}
\label{cadb}

\hl{Synopsis}

\#include $<$af.h$>$

\hl{Description}
     
This section lists CA database objects required by the af\_cadb\_* functions. 
Each object is presented in terms of C-language structures as they are needed by the 
interface subroutines and by the application programmer. 
\subsubsection{IssuedCertificate}
\hll{IssuedCertificate}
Serial number and date of issue. \\
\underline{C-Structure of the object:}

{\small
\btab
\1      typedef struct IssuedCertificate \{ \\
\2              OctetString  \2          *serialnumber; \\
\2              UTCTime \2  *date\_of\_issue; \\
\1       \} IssuedCertificate; \\
\etab
}

Encoding/Decoding: e\_IssuedCertificate(), d\_IssuedCertificate()

\subsubsection{CRLWithCertificates}
\hll{CRLWithCertificates}
PEM Revocation List with Originator Certificate. \\
\underline{C-Structure of the object:}

{\small
\btab
\1      typedef struct CRLWithCertificates \{ \\
\2              CRL  \2          *crl; \\
\2              Certificates \2  *certificates; \\
\1       \} CRLWithCertificates; \\
\etab
}

Encoding/Decoding: e\_CRLWithCertificates(), d\_CRLWithCertificates()



\subsection{PSE Selection}
\label{cc_sel}
\nm{5}{PSE-Selection}{PSE-Selection at the AF-Interface}

\hl{Synopsis}

\#include $<$af.h$>$

\hl{Description}
     
The af\_* functions do not provide the possibility
to specify at their parameter interface the names and locations of
PSE objects to be used, as it is possible with the sec\_* functions by
means of the structure PSESel. af\_* functions operate on a predefined
set of 12 objects in a PSE with the predefined name {\em .pse}.
However, all af\_* functions use the global structure AF\_pse. AF\_pse is of
type AFPSESel:

{\small
\bvtab
\4 typedef struct \{ \\
\6        char \1    *app\_name; \\
\6        char \1    *pin; \\
\6        struct \{ \\
\8            char \1 *name; \\
\8            char \1 *pin; \\
\6        \}  \1      object[PSE\_MAXOBJ]; \\
\6        int \1     dev; \\
\6        int \1     app\_id; \\
\4 \} AFPSESel; 
\evtab
}

This global structure can be used for two purposes:
\be
\m AF\_pse contains the application name. This is the
   name of the directory constituting the PSE. The name is interpreted 
   relatively to the HOME directory, if it does not begin with character '/'.
   The application name is initialized with {\em .pse} as default value.
   AF\_pse\pf app\_name can be used to specify another PSE before calling
   an af\_* routine. If multiple PSEs are used simultaneously, the content
   of this global structure must be saved for each PSE and restored
   before a call to an af\_* routine.
\m AF\_pse contains all interactively entered PIN's in order to avoid multiple
   questioning of PIN's in case of multiple access to the same PSE object within
   one program. After a call to af\_sign, for instance, which asks for the PIN
   of the PSE and for the PIN of the PSE object SignSK, if there is any,
   AF\_pse\pf pin contains the PSE PIN and AF\_pse.object[1]\pf pin contains the
   PIN of the PSE object SignSK. A second call to af\_sign within the same
   program will not ask again for those PINs. PIN questionning can be
   enforced again by clearing the PINs in the global structure AF\_pse.
   See AF(5) for the exact definition of AF\_pse.
\ee
This structure can be used by the programmer by declaring \\

{\small
extern AFPSESel AF\_pse;
}
