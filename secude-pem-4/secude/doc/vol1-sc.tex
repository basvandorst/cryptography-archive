\section{Use of Smartcards}
\markboth{Use of Smartcards}{Use of Smartcards}
\thispagestyle{myheadings}
\label{sc}

Within SecuDe smartcards are the basic means for the realization
of the PSE functionality (Personal Security Environment).
Smartcards are understood
as multifunctional processor smartcards \cite{str1}.
The structure of a smartcard is modelled by the international
standard {\em ISO 7816:
Identification Cards} \cite{iso3}. Its {\em Part 4:
Interindustry Commands for International Interchange}
states among others:
``the logical structure of the data contained in the
Integrated Circuits Card'', and: ``the security attributes
defining access rights to the data in the Integrated Circuits Card''.
The standardization of ISO 7816-4 is not finished yet.
Many changes are expected until its planned final issue 1992.
Therefore, within SecuDe
the smartcard data and access rules
are structured due to a simplified model.

\subsection{Structure of Smartcard Data}
\label{sc-struct}
The smartcard data files within SecuDe
are structured like as shown in Fig. 12.
\\[1ex]

The smartcard itself is represented by the root of a file tree,
which is called the master file MF.
On the smartcard reside elementary files and dedicated files.
Dedicated files are roots of more elementary files.
There are three general elementary files, the public elementary file PEF,
the application control file ACF and the internal secret file ISF.
Below the dedicated files there are also working elementary files
which are specific to the dedicated application.
The working elementary files are described below.

\subsection{Access Rights}
\label{sc-ar}

Different rights are given to the local smartcard maintenance
and to the local signature and encryption services.
Private keys will not be passed through the interface
of functions which handle the smartcard information but will
only be {\em used} for computations. Signatures and decrypted
texts will pass the interface instead.
Other information like public keys and certificates will
be readable through the interface.

\begin{center}
\makebox[5.666in][l]{
  \vbox to 4.750in{
    \vfill
    \special{psfile=vol1-fig6.ps}
  }
  \vspace{-\baselineskip}
}
\end{center}
\label{fig-sc-struct}
\stepcounter{Abb}
{\footnotesize Fig.\arabic{Abb}: Data structure of a smartcard}

\subsection{The General Elementary Files of the Master File}
\label{sc-mef}

These files will not be implemented, because a software smartcard
will represent a single application only.

\subsection{The General Elementary Files}
\label{sc-gef}

The {\em public elementary file} contains some project information
and the date of the last update of one of the smartcard files.
The {\em application control file} is not implemented.
The {\em internal secret file} will contain a password
which is used to encrypt the other files on the smartcard.
A one-way-function maps the password on a DES-key, which is used
to encrypt the smartcard files.
The password itself is stored in the DES-key encrypted format.

\subsection{The Working Elementary Files}
\label{sc-wef}

\subsubsection{SignSK}
\label{signsk}

This file contains the user's private signature key and a signature algorithm
identifier. The signature algorithm identifier does not restrict the key
to a certain hash function. In particular, it should either say
``RSA signature key'', or if it gives a hash function like
``RSA with sqare-mod-n signature key'',
the hash function is just a default value.
The private signature key is generated by the user.

\subsubsection{DecSKnew and DecSKold}
\label{decsk}

DecSKnew
contains the currently valid private decryption key of the user.
DecSKold
contains the last valid (but now expired) private decryption key of the user.
The private decryption key is generated by the user.

\begin {center}
\begin {tabular}{|c|c|}
\hline
{\em old key} & {\em new key} \\ \hline
$B_{Sold}$    & $B_{Snew}$    \\ \hline
\end {tabular}
\end {center}
 
\label{fig-decsk}
\stepcounter{Abb}
{\footnotesize Fig.\arabic{Abb}: Private decryption keys ``DecSKnew'' and ``DecSKold''}
\\
{\footnotesize $B_S$ denotes the private (``secret'') key
of user B, in this case for the purpose of decryption}
\\ [1em]
For details of the decryption procedure,
see paragraph \ref{ope-ondk} below.

\subsubsection{SignCert}
\label{signcert}

This file contains the user-certificate of the
user's public verification key. This certificate is also
called {\em hierarchy certificate}, in contrast to croos
certificates (see below).
The originator-certificate, for instance, is a composition of the
user-certificate with the
forward certification path (``FCPath'').
A user can send this certificate to his communication
partners in order to be added to their tables of trusted
partners (``PKList'').
A prototype of the
user-certificate is generated by the user
and sent to his CA, the CA will resend it updated and signed.

\subsubsection{SignCSet}
\label{signcset}

This file contains a set of cross-certificates of the
user's public verification key, i.e. certificates issued
by CAs which do not belong to the user's own certification tree.
A cross-certificate has the same format as a hierarchy certificate.
A prototype of the
cross-certificate is generated by the user
and sent to the other CA, the CA will resend it updated and signed.

\subsubsection{EncCert}
\label{enccert}

This file contains the user-certificate of the
user's public encryption key. This certificate is a
hierarchy certificate (see above).
The encryption recipient certificate (see paragraph \ref{ope-epk})
is a composition of the user-certificate with the
forward certification path (``FCPath'').
A user can send this certificate to his communication
partners in order to be added to their tables of trusted
partners (``EKList'').
A prototype of the
user-certificate is generated by the user
and sent to his CA, the CA will resend it updated and signed.

\subsubsection{EncCSet}
\label{enccset}

This file contains a set of cross-certificates of the
user's public encryption key, i.e. certificates issued
by CAs which do not belong to the user's own certification tree.
A prototype of the
cross-certificate is generated by the user
and sent to the other CA, the CA will resend it updated and signed.

\subsubsection{FCPath}
\label{fcpath}

This file contains the generally n-level forward certification path,
the lowest element of which certifies the signature key of the user-CA.
A user's originator-certificate is a composition of his
user-certificate (either ``SignCertificate'' or ``EncrCertificate'')
with the forward certification path.
The user receives the currently valid forward certification path
from his user-CA (regardless, which element of it has changed).

\subsubsection{PKRoot}
\label{pkroot}

This file contains a pair of public verification keys of the root CA.
The old key has expired by the generation of the currently valid new key.
To each key
the lowest serial number of a certificate which is to be verified by
this key is assigned.
The portion ``key number'' of the respective serial number is
the key number of the respective key.
The user receives the currently valid Root-CA-key from his user-CA.

\begin {center}
\begin {tabular}{|l|c|l|c|l|}
\hline
{\em CA-name} &
{\em serial no old} & {\em PKRoot-old} &
{\em serial no new} & {\em PKRoot-new} \\
\hline
``TTT-D'' &
16000000 & $TTT-D_{P16}$ &
17000000 & $TTT-D_{P17}$ \\ \hline
\end {tabular}
\end {center}
 
\label{fig-pkroot}
\stepcounter{Abb}
{\footnotesize Fig.\arabic{Abb}: Highest public CA-keys ``PKRoot''}
\\
{\footnotesize
In this table the highest CA is supposed to be
``TTT-D'', its currently valid signature key number is 17.
``$TTT-D_{P17}$'' denotes the X.509-type public key info,
which contains TTT-D's public key with key number 17.}

\subsubsection{PKList}
\label{pklist}

This file contains a table of trusted public verification keys of other
users and CAs known to the smartcard's owner.
In order to enable the user to map a partner's verification key
on the respective verification certificate,
all certificate fields are kept in the PKList table.
On choice of the user,
new entries into the table are taken from received certificates.

\begin {center}
\begin {tabular}{|l|l|}
\hline
{\em name} & {\em verification certificate}  \\ \hline
``GMD''                & $GMD_tbs$     \\ \hline
``DFN''                & $DFN_tbs$     \\ \hline
``UBC''                & $UBC_tbs$     \\ \hline
``INRIA''              & $INRIA_tbs$   \\ \hline
``Uschi Viebeg''       & $UV_tbs$      \\ \hline
``Wolfgang Schneider'' & $WS_tbs$      \\ \hline
``Rolf Nausester''     & $RN_tbs$      \\ \hline
``R\"udiger Grimm''    & $RG_tbs$      \\ \hline
\end {tabular}
\end {center}
 
\label{fig-pklist}
\stepcounter{Abb}
{\footnotesize Fig.\arabic{Abb}: Trusted public verification keys
``PKList''}
\\
{\footnotesize
``$GMD_tbs$'' denotes the ASN.1-structure of the ``to-be-signed'' entry
of that verification certificate
which contains the public verification key of GMD, etc.}
\\[1em]
The bit string of the verification key is the search criterion of this table.

\subsubsection{EKList}
\label{eklist}

This file contains a table of trusted public encryption keys of other
users and CAs known to the smartcard's owner.
In order to enable the user to map a partner's encryption key
on the respective encryption certificate,
all certificate fields are kept in the EKList table.
On choice of the user,
new entries into the table are taken from received certificates.

\begin {center}
\begin {tabular}{|l|l|}
\hline
{\em name} & {\em encryption key} \\
\hline
``DFN''                & $DFN_tbs$     \\ \hline
``Uschi Viebeg''       & $UV_tbs$      \\ \hline
``Wolfgang Schneider'' & $WS_tbs$      \\ \hline
``Rolf Nausester''     & $RN_tbs$      \\ \hline
``Rudiger Grimm''      & $RG_tbs$      \\ \hline
\end {tabular}
\end {center}
 
\label{fig-eklist}
\stepcounter{Abb}
{\footnotesize Fig.\arabic{Abb}: Trusted public encryption keys ``EKList''}
\\
{\footnotesize
``$UV_tbs$'' denotes the ASN.1-structure of the ``to-be-signed'' entry
of that encryption certificate 
which contains the public encryption key of ``Uschi Viebeg'', etc.}
\\[1em]
The name of the verification key's owner
is the search criterion of this table.

\subsection{The ``Software PSE''}
\label{sc-swc}

Software which represents the
functionality of a smartcard and which uses
standard storage devices of
a users's computer system such as his PC, WS or mainfraime
in order to store smartcard data
is called a ``software PSE''.
For example,
a software smartcard will be implemented as
a Unix user's subdirectory
which acts as a storage of smartcard data
together with software to access these data.

A symmetric, password-governed encryption technology
is used to encrypt the files on the software smartcard
in order to protect their privacy and integrity.

The design of a software PSE for Unix
uses a reasonably simpler smartcard model than described in the
paragraphs \ref{sc-struct} -- \ref{sc-wef} above.
The ``dedicated file'' of each application will be
realized by a local subdirectory under the user's home directory.
Master file or other applications are not realized.
The internal secret file contains the (encrypted) password
which, by a oneway function, is mapped on a DES-key.
This DES-key is used to encrypt and decrypt all the
working elementary files of the application
(including the internal secret file).
Any working elementary file can be password-protected
by a file-specific password, if the user so whishes.
Access to a software PSE file is granted to any user
who knows the correct password of this file.

Authentication based on digital signatures rely on uncompromised
signature keys. This supposition is weeker for software PSEs
than for those using real smartcards. However, the protection against attacks
on communication outside of the local systems is not dependent
on the security of local key storage,
because keys are not transferred.
With the implementation of software PSEs and
strong security services for communication,
communication becomes at least as secure as the local systems are,
and is in many situations (e.g. in an MTA)
even independent of the security of the local systems.
It should be added, though, that well chosen passwords
can be a reasonable base for a good protection of local environments
(however not of communication).
