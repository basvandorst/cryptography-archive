\section{Secured QUIPU}
This section describes how to configure, generate, and install
a security-enhanced ISO Development Environment that provides a secured version of
QUIPU-ICR1, the OSI Directory. 
\\ [1em]
QUIPU-ICR1 has been enhanced with the security functionality provided by SecuDE-4.3, and 
incorporates both a secured DSA and DUA (DISH). It provides for strongly authenticated DAP access
to the Directory.
\\ [1em]
Strongly authenticated DAP provides for strong authentication of the directory user
to the Directory and data integrity of the operation argument. In addition to this, it 
provides for strong authentication of the responding DSA to the initiating user
(thereby giving the user confidence that he has connected to the intended Directory Service)
and data integrity of the operation result (thereby giving the user confidence in the
integrity of the information returned from the Directory).
\\ [1em]
We have integrated the QUIPU/SecuDE interface hooks into ISODE-ICR1 as a compile option.
The files into which we have inserted the SECUDE compile options do all form part
of the {\em libdsap} library. Therefore, it would have been sufficient to distribute
just them, and not the whole of ISODE. However, as the affected files are dispersed
over many nested directories, we found it easier to distribute the whole of ISODE.
\\ [1em]
In the following, it is assumed that you have already generated the SecuDE-4.3 library,
compiled with the X500 and STRONG flags. 
A description of how this is done is provided by the SecuDE installation guide
({\bf README} file in the top-level directory and {\bf CONFIG.make} file in the config/ directory of your SecuDE installation).


\subsection{Configuration}
Go to the config/ directory of your security-enhanced ISODE installation, edit the makefile 
skeleton of your choice, and go to the {\bf STRONG AUTHENTICATION} section.
\\ [1em]
First, uncomment the following lines:
\bvtab
\1 SECUDETOP  $=$  /usr/local/secude \\
\1 SECUDELIBDIR  $=$  \$(SECUDETOP)/lib \\
\1 LIBSECUDE  $=$  \$(SECUDELIBDIR)/libsecude.a \\
\1 LSECUDE  $=$  -lsecude \\
\1 LM  $=$  -lm \\
\1 LIBSEC  $=$  \$(LSECUDE) \$(LM) \\
\1 SECUDEINCDIR  $=$  \$(SECUDETOP)/src/include \\
\1 DFLAGS  $=$  \$(DSECUDE) \$(DEDB\_ENCRYPT) \\
\1 DSECUDE  $=$  -DSECUDE 
\evtab
Second, assign to the {\bf SECUDETOP} variable the absolute pathname of the top-level directory 
of your SecuDE-4.3 installation.
\\ [1em]
If you are interested in the security feature described in par.~\ref{encryptedb}:
{\em Encryption of your DSA's Database}, you should in addition uncomment the following line:
\bvtab
\1 DEDB\_ENCRYPT  $=$  -DEDB\_ENCRYPT
\evtab
You may uncomment the above line at a later time, if for the time being there is no requirement
for cryptographically protecting your DSA's database.
In case that you uncomment the above line at a later time, you have to switch to the quipu/ directory 
of your ISODE installation, re-compile the files
{\bf dsa.c}, {\bf entry\_dump.c}, and {\bf parse2.c}, and then re-install QUIPU:
\bvtab
\1 cd config \\
\1 touch dsa.c entry\_dump.c parse2.c \\
\1 ./make inst-quipu
\evtab


\subsection{Generation and Installation}
Generate and install the basic system and the QUIPU software in the same manner as described in the
ISODE documentation.


\subsection{New Object Identifiers}
SecuDE-4.3 supports the revoction list format adopted 
by PEM, in order to avoid the deficiencies inherent to the ISO 9594-8 format,
and derive benefit from the {\em next update} indication offered by the PEM format. 
\\ [1em]
For this reason, it is required that the Object Identifier tables in the ETCDIR directory
of your security-enhanced ISODE installation are adapted to this definition of revocation lists.
\\ [1em]
The relevant definitions, which were agreed upon within the European PASSWORD project, have been 
added to the files {\bf oidtable.at.local}, 
{\bf oidtable.gen.local}, and {\bf oidtable.oc.local} in the dsap/ directory of your 
security-enhanced ISODE installation, and will automatically be appended to the Object Identifier 
tables in the corresponding ETCDIR directory when your software is
installed. 


\subsection{Personal Security Environment for your DSA}
Any DSA which supports the SecuDE approach to strong authentication
must own a PSE, where all its security-critical data are securely 
stored.
\\ [1em]
Please observe that your DSA will consider only information (e.g., cached
public keys, revocation lists, cross certificate pairs) which was
stored on its PSE before your DSA started. Any data which are stored on your
DSA's PSE while your DSA is running will not be considered by your DSA - unless
you stop your DSA and then re-start it.
\\ [1em]
A DSA manger has to perform the following actions in order to equip
its DSA with the ability to verify and produce digitally-signed 
information:
\be
\m Generate a PSE on behalf of your DSA (see commands {\em psecreate} or {\em genpse}).
   It is required that the name of your DSA's PSE exactly matches with your DSA's distinguished name.
   \\ [1em]
   Example: Assuming that your DSA's distinguished name is ``C=DE, CN=foobar'', you may generate a PSE
   for your DSA by calling the {\em psecreate} command as follows:
   \bvtab
   \1 {\bf psecreate} -v -p C=DE@CN=foobar ``C=DE, CN=foobar''\\
   \evtab
   However, if your DSA's distinguished name is too long, you may assign to your DSA's PSE
   an arbitrary name. In that case, you have to specify your DSA's PSE by providing the {\bf p} 
   option when starting your DSA:
   \bvtab
   \1 {\bf ros.quipu} {\bf -p} location of DSA's PSE \\
   \evtab

\m Let the authorized CA certify your DSA's public key, and install the resulting certificate
   on your DSA's PSE (see the SecuDE commands {\em getkey}, {\em certify}, and {\em instcert}, or 
   follow the certification request/reply mechanism defined by PEM (RFC 1424), which is realized by
   the SecuDE utility {\em pem}).

\m Initialize your DSA's PKList by downloading any public keys trusted by your DSA
   (see commands {\em psemaint} or {\em pkadd}).
\ee


\subsection{Encryption of your DSA's Database}
\label{encryptedb}
On the basis of GMD's secured X.500 software, you may now protect your DSA's EDB database
by symmetrically encrypting it. Encryption (DES-CBC) is applied to each (MASTER, SLAVE, or
CACHE) EDB file which is part of your DSA's database.
\\ [1em]
The DES key used for encryption/decryption is stored in the object {\bf EDBKey} on your DSA's PSE.
\\ [1em]
Up to now, DSAs could only protect any (sensitive) data by enforcing access control
and authentication policy measures (realized by the {\em accessControlList} and {\em authPolicy} 
attributes, respectively).
However, all information (except the user password) was stored in cleartext in the DSA's 
UNIX database and was therefore insufficiently protected against disclosure.
\\ [1em]
Storing a DSA's database in an encrypted form avoids the risk that any (sensitive) data
held by the DSA may be read by unauthorised users who have duped the system's 
UNIX access control.
\\ [1em]
If you want to protect your DSA's database by encrypting it, you should perform the
actions as listed below:
\be
\m Make sure that the line
\bvtab
\1 DEDB\_ENCRYPT $=$ -DEDB\_ENCRYPT
\evtab
in the file CONFIG.make in the config/ directory of your security-enhanced ISODE installation
has been uncommented. If this is not the case, uncomment it now, switch to the quipu/ directory 
of your ISODE installation, re-compile the files
{\bf dsa.c}, {\bf entry\_dump.c}, and {\bf parse2.c}, and then re-install QUIPU:
\bvtab
\1 cd quipu \\
\1 touch dsa.c entry\_dump.c parse2.c \\
\1 ./make inst-quipu
\evtab

\m Access your DSA's PSE, generate a symmetric key (DES-CBC), and store it in the
PSE object {\bf EDBKey} (see {\em psemaint} command).

\m Encrypt all EDB files belonging to your DSA's database with the symmetric key
generated in 3:
\bvtab
\1 {\bf encrypt} -p C=DE@CN=foobar -k EDBKey $<$ EDB $>$ tmp \\
\1 mv tmp EDB
\evtab

\m On (re-)starting your DSA, provide a {\bf -E} flag to the {\em ros.quipu} command:
\bvtab
\1 {\bf ros.quipu} {\bf -E}
\evtab
\ee
If you are no longer interested in having your DSA's database cryptographically protected,
you simply have to stop your DSA, decrypt its EDB files, and then re-start your DSA as usual.
For decrypting your DSA's EDB files, invoke the {\em decrypt} command as follows:
\bvtab
\1 {\bf decrypt} -p C=DE@CN=foobar -k EDBKey $<$ EDB $>$ tmp \\
\1 mv tmp EDB
\evtab


\subsection{Additional parameters to ros.quipu}
The DSA's signature algorithm can be specified by {\bf -a} {\em sigalg} when calling {\em ros.quipu}: 
\bvtab
\1 {\bf ros.quipu} {\bf -a} {\em sigalg}
\evtab
{\em sigalg} must be a signature algorithm (AlgType SIG, see INTRO(3) on available algorithms) 
which fits to the associated algorithm of the signature key (e.g., the algorithm md5WithRsa cannot 
be used with an ElGamal key). The default for {\em sigalg} is md2WithRsaEncryption.

If you want your DSA to consult revocation lists during the verification of a digital 
signature, you have to provide the {\bf -R} parameter to the {\em ros.quipu} command:
\bvtab
\1 {\bf ros.quipu} {\bf -R}
\evtab
Please observe that any revocation lists which are to be considered
by your DSA must have alredy been stored on the DSA's PSE, before the
DSA starts.
\\ [1em]
If a revocation list stored on your DSA's PSE is out-of-date, you have
to download the most recent version of that revocation list onto your 
DSA's PSE and then re-start your DSA.
\\ [1em]
In order that your DSA is able to strongly authenticate a user who
belongs to a different certification tree, make sure that your DSA's
Root-CA and the authenticating user's Root-CA have cross-certified each other, 
and download the relevant cross-certificate pair onto your DSA's PSE.


\subsection{Authentication Data}
Your authentication data required for simple and strong authentication to the
Directory are securely stored on your PSE. Thus, the {\em username}, {\em password}, {\em certificate}, 
and {\em secret\_key} information of your {\em .quipurc} file is no longer needed.


\subsection{Simple Authentication during Bind}
If you wish that your authentication data required for simple authentication (i.e.,
your distinguished name and password) are retrieved from your PSE, you have to invoke the
DISH command as follows:
\bvtab
\1 {\bf dish} -simple -{\bf pse}name,
\evtab
followed by the location of your PSE. (If your PSE resides
at the default place (i.e., {\bf .pse} in your HOME directory), or {\bf PSE} has been 
defined in your local environment, parameter {\em psename} is not needed.)
\\ [1em]
In case of a certification authority initiating a simple authentication request
to the Directory, the DISH command should have the following form:
\bvtab
\1 {\bf dish} -simple -{\bf ca}dir,
\evtab
followed by the CA directory in which the CA's PSE (.capse) resides.
\\ [1em]
Your X.500 password will automatically be stored on your PSE (PSE object {\bf QuipuPWD})
after you have successfully bound to the Directory for the first time, using simple
authentication.


\subsection{Strong Authentication during Bind}
If you wish to strongly authenticate yourself to the Directory, you have to supply
to the DISH command line the following (additional) parameters:
\bi
\m {\em {\bf str}ong}, indicating that strong authentication is to be performed;
\m {\em {\bf pse}name}, followed by the location of your PSE. (If your PSE resides
at the default place (i.e., {\bf .pse} in your HOME directory), or {\bf PSE} has been 
defined in your local environment, parameter {\em psename} is not needed.)
\ei
In case of a certification authority initiating a strong authentication request
to the Directory, the additional parameters that have to be supplied to the DISH
command are as follows:
\bi
\m {\em {\bf str}ong}, indicating that strong authentication is to be performed;
\m {\em {\bf cad}ir}, followed by the CA directory in which the CA's PSE (.capse) resides.
\ei
To try to strongly bind to any of the DSAs listed in your {\em dsaptailor} file, invoke DISH as
described above, with a {\bf -call} flag; e.g., dish {\bf -call bonsai -strong} will try
to strongly connect to the DSA {\em bonsai} running at the University of London
Computer Centre. If the connection and mutual authentication is successful, an appropriate message
and the prompt {\bf Dish -$>$} will be returned. If the connection or mutual
authentication fails, however, the program will exit with an appropriate error message.
\\ [1em]
If the name following the {\em -call} flag is a private keyword that refers to a DSA, it is
required that keyword and corresponding distinguished name are stored as an entry in your
system-wide alias database {\em .af-alias} in the {\em .af-db} directory of your SecuDE
installation (see par.~\ref{dsaaliases}: {\em Alias Names for DSAs}).
\\ [1em]
The signature algorithm associated with the bind argument can be specified by using the 
{\bf -signalg} {\em algorithm} construction, e.g.: 
\bvtab
\1 {\bf dish} {\bf -signalg} {\em algorithm} -strong
\evtab
{\em algorithm} must be a signature algorithm (AlgType SIG, see INTRO(3) on available algorithms) 
which fits to the associated algorithm of the signature key. The default for {\em algorithm} is md2WithRsaEncryption.


\subsection{SIGNED Operations}
The DAP operations into which we have incorporated strong authentication are those
assigned by X.511 for that purpose, i.e. Bind, Read, Compare, Search, List, AddEntry, 
RemoveEntry, ModifyEntry, and ModifyRDN.
\\ [1em]
If you wish to protect your operation argument by your digital signature, you have
to provide the additional {\bf strong} flag to the command line, e.g.:
\bvtab
\1 {\bf list} @C=DE@o=GMD@l=Darmstadt -{\bf str}ong 
\evtab
The signature algorithm associated with the operation argument can be specified by 
{\bf -signalg} {\em algorithm}, e.g.: 
\bvtab
\1 {\bf list} @C=DE@o=GMD@l=Darmstadt {\bf -signalg} {\em algorithm} -strong
\evtab
{\em algorithm} must be a signature algorithm (AlgType SIG, see INTRO(3) on available algorithms) 
which fits to the associated algorithm of the signature key. The default for {\em algorithm} is md2WithRsaEncryption.

SIGNED operation results are only supported in the case of the Read, Compare, Search, and List
operations.
A user may indicate its request that an operation result be digitally protected by using
the {\bf protecresult} flag, e.g.:
\bvtab
\1 {\bf list} @C=DE@o=GMD@l=Darmstadt -strong -{\bf pro}tecresult
\evtab


\subsection{Directory Authorization}
DSAs that are based on the GMD's secured X.500 will enforce the following authentication policy 
on a per entry basis:
\bi

\m They will refuse a requested operation, if the level of security associated with the
   requestor's credentials is inconsistent with the level of protection requested, e.g.
   simple credentials were supplied while strong credentials were required
   (inappropriate authentication).
   The minimum level of authentication required to bind to your DSA over DAP can be
   specified by assigning to the {\em authentication} option in the {\em quiputailor} file of your 
   security-enhanced ISODE installation one of the values {\em none}, {\em dn}, {\em simple}, 
   {\em protected}, or {\em strong} (see {\em The ISO Development Environment: User's Manual}, 
   Volume 5, section 14.3).

\m They will refuse a requested operation, if the requestor does not have the right
   to carry out the requested operation (insufficient access rights). This is
   enforced by the QUIPU attribute {\em accessControlList}. Unfortunately, the attribute {\em authPolicy}
   could not be considered.
\ei


\subsection{Alias Names for DSAs}
\label{dsaaliases}
According to X.511, the DSA to which a user wants to strongly authenticate itself
must be identified by its complete distinguished name.
\\ [1em]
Make sure that any DSAs that are listed in your {\em dsaptailor} file and to which you may
wish to strongly bind, are represented in your SYSTEM alias database
({\em .af-alias}) (or at least in your own USER alias database) by their symbolic keyword and 
distinguished name.
It is required that the keyword by which a specific DSA is represented in your 
SecuDE alias database, exactly matches with the DSA's keyword in your {\em dsaptailor} file.
\\ [1em]
Example: If you wish to strongly bind for the first time to the DSA which is locally referred
to by the keyword {\em bonsai} in your {\em dsaptailor} file, you should find out the DSA's full
distinguished name (which in this case is 
``C=GB, O=University College London, OU=Computer Science, CN=DSA''),
and then store the pair ({\em bonsai}, ``C=GB, O=University College London, OU=Computer Science, CN=DSA'') in your alias database in the following manner:
\begin{quote}
{\small bonsai : C=GB, O=University College London, OU=Computer Science, CN=DSA}
\end{quote}
(You may fill your own USER alias database by calling the {\em addalias} command available with
{\em psemaint}.)
\\ [1em]
Any subsequent strong authentication requests to {\em bonsai} may be invoked as follows:
\begin{quote}
{\bf dish} -call bonsai -strong,
\end{quote}
with {\em bonsai} being mapped onto ``C=GB, O=University College London, OU=Computer Science, CN=DSA'',
which will be inserted into the BindToken and then sent to the intended DSA.
\\ [1em]
The error message {\em Cannot transform alias ``xyz'' into a Distinguished Name} indicates that you
should add the entry (xyz, Distinguished Name of xyz) to your SecuDE alias database.


\subsection{More Help Installing GMD Secure X.500}
Contact the SecuDE discussion list {\bf secude@darmstadt.gmd.de}. Bug reports (and fixes) are
welcome.
