\subsection{Low Level Cryptography}         


\subsubsection{Low Level Cryptography}
\nm{3X}{Low Level Cryptography}{Multiprecision arithmetics, encryption, signature, hashing}
\label{lowlevel}

\hl{Description}

The following routines are internal SecuDE low level routines for cryptographic and related
operations. They are called from sec\_encrypt(), sec\_decrypt(), sec\_sign(), sec\_verify(),
sec\_hash() and sec\_gen\_key(). Programmers who use these sec\_* routines do not need to know
the low level functions. They are listed here for people with particular interest
in such functions.

\hl{Multi-precision arithmetic operations}

Asymmetric cryptographic algorithms like RSA and DSA require to perform arithmetic operations
on very large integer numbers with up to some thousand binary digits. In SecuDE,
a multi-precision integer number, in the following referred to as long number, is internally stored 
and processed in an array of unsigned integers.
\begin{quote}
typedef unsigned int L\_NUMBER; \\
L\_NUMBER long\_number[MAXLGTH]
\end{quote}
The maximum length of the array is MAXLGTH (which is a parameter set at compile-time). 

The first element (long\_number[0]) indicates the number of
following words which contain the long number, starting with
the least-significant word (the word itself may be ordered in different ways,
depending on the machine type). Routines exist which transform from a SecuDE
long number to an OctetString (ordered most-significant octet first) and vice-versa.
The OctetString representation of a long number can be used to derive the ASN.1 code
with e\_INTEGER() which generates the BER-code of an ASN.1 INTEGER.
A long number of value zero is indicated through long\_number[0] $=$ 0. If long\_number[0] $\neq$ 0,
long\_number[long\_number[0]] must be non-zero.

Following are arithmetic and other functions available for long numbers. None of
these does allocate memory for returned results. This must be provided by the
calling functions through the use of appropriate parameters. Note also that none of
these routines does any parameter checking and that unexpected program terminations
may result from the use of inappropriate or syntactically wrong parameters. 
\\ [1ex]

{\bf \_add} {\em (op1, op2, result)} \\
L\_NUMBER {\em *op1, *op2, *result}; 

$result = op1 + op2$
\\ [1ex]

{\bf \_sub} {\em (op1, op2, result)} \\
L\_NUMBER {\em *op1, *op2, *result}; 

$result = op1 - op2$
\\ [1ex]

{\bf \_mult} {\em (op1, op2, result)} \\
L\_NUMBER {\em *op1, *op2, *result}; 

$result = op1 * op2$
\\ [1ex]

{\bf \_div} {\em (op1, op2, result, rest)} \\
L\_NUMBER {\em *op1, *op2, *result, *rest}; 

$result = op1 / op2$ \\
$rest = op1\ (\ {\rm mod }\ op2 )$
\\ [1ex]

{\bf \_shift} {\em (op, bits, result)} \\
L\_NUMBER {\em *op, *result}; \\
int {\em bits};

$result = op * 2^{bits}$
\\ [1ex]

int {\bf \_comp} {\em (op1, op2)} \\
L\_NUMBER {\em *op1, *op2}; 
\btab
Returns \1 zero \1 if {\em op1} $=$ {\em op2}, \\
        \1 1 \1 if {\em op1} $>$ {\em op2} \\
        \1 -1 \1 if {\em op2} $>$ {\em op1} \\
\etab

{\bf msub} {\em (op1, op2, result, modulus)} \\
L\_NUMBER {\em *op1, *op2, *result, *modulus};

$result = op1 - op2\ (\ {\rm mod }\ modulus )$
\\ [1ex]

{\bf mmult} {\em (op1, op2, result)} \\
L\_NUMBER {\em *op1, *op2, *result}; 

$result = op1 * op2\ (\ {\rm mod }\ modulus )$
\\ [1ex]

{\bf mdiv} {\em (op1, op2, result, modulus)} \\
L\_NUMBER {\em *op1, *op2, *result, *modulus}; 

$result = op1 / op2\ (\ {\rm mod }\ modulus )$
\\ [1ex]

{\bf mexp} {\em (op1, op2, result, modulus)} \\
L\_NUMBER {\em *op1, *op2, *result, *modulus};

$result = {op1}^{op2}\ (\ {\rm mod }\ modulus )$
\\ [1ex]

int {\bf lngtouse} {\em (lnum)} \\
L\_NUMBER {\em *lnum}; 

lngtouse returns the number of significant binary digits of the L\_NUMBER {\em lnum},
i.e.omitting leading zeroes. It returns $-1$ if $lnum=0$ or the long number is syntactically wrong.
\\ [1ex]


int {\bf octetstoln} {\em (octs, lnum, offset, size)} \\
OctetString {\em *octs}; \\
L\_NUMBER {\em *lnum}; \\
int {\em offset, size}; 

octetstoln transforms the OctetString starting at {\em octs} $+$ {\em offset} 
and of length {\em size} to a long number {\em lnum}. The OctetString is
interpreted most-significant octet first.
\\ [1ex]

int {\bf lntoctets} {\em (lnum, octs, size)} \\
L\_NUMBER {\em *lnum}; \\
OctetString {\em *octs}; \\
int {\em size}; 

lntoctets converts the long number {\em lnum} (least-significant word first) 
to an OctetString (most-significant octet first). 
If $size=0$, this OctetString is returned in {\em octs}.
If $size\neq 0$, {\em size} octets of this OctetString are appended to {\em octs}.
if {\em size} exceeds the number of significant octets of {\em lnum}, leading
zero-octets are padded.
\\ [1ex]

int {\bf bitstoln} {\em (bits, lnum, offset, size)} \\
BitString {\em *octs}; \\
L\_NUMBER {\em *lnum}; \\
int {\em offset, size}; 

bitstoln is the corresponding function to octetstoln for a BitString.
\\ [1ex]


int {\bf lntobits} {\em (lnum, bits, size)} \\
L\_NUMBER {\em *lnum}; \\
BitString {\em *bits};\\
int {\em size}; 

lntobits is the corresponding function to lntoctets for a BitString.


\hl{RSA Functions}

int {\bf rsa\_gen\_key} {\em (keysize, skey, pkey)} \\
int {\em keysize}; \\
BitString {\em **skey}; \\
BitString {\em **pkey}; 

rsa\_gen\_key generates an RSA key pair with a modulus of {\em keysize} bits.
Pointers to the ASN.1 encoded secret and public key are returned in {\em *skey} and {\em *pkey}.
See INTRO(3) about the ASN.1 structure of RSA keys. 
\\ [1ex]



int {\bf rsa\_set\_key} {\em (key, keytype)} \\
char {\em *key}; \\
int {\em keytype}; 

rsa\_set\_key stores the given {\em key} in a static
structure for further use through rsa\_encrypt(), rsa\_decrypt(),
rsa\_sign() or rsa\_verify(). {\em key} may be either a secret or a public key.

If {\em keytype} is 0, {\em key} must be a pointer to a KeyBits structure.
Otherwise it must be a pointer to a BitString which contains the ASN.1 
encoded key. rsa\_set\_key casts {\em key} to the appropriate type.
\\ [1ex]


int {\bf rsa\_encrypt} {\em (in, out, more, keysize)} \\
OctetString {\em *in}; \\
BitString {\em *out}; \\
More {\em more}; \\
int {\em keysize}; 

rsa\_encrypt encrypts {\em in} with the public key previously 
set by rsa\_set\_key and appends the result
to {\em out}. In each step of the encryption a block of
$(keysize-1) / 8$ octets is encrypted. If $more =$ SEC\_MORE,
the remaining block is filled by the next call of  rsa\_encrypt.
If $more =$ SEC\_END, it is padded and the key is deleted
after the encryption.
\\ [1ex]


int {\bf rsa\_decrypt} {\em (in, out, more, keysize)} \\
BitString {\em *in}; \\
OctetString {\em *out}; \\
More {\em more}; \\
int {\em keysize}; 

rsa\_decrypt decrypts {\em in} with the secret key previously set by 
rsa\_set\_key and appends the result
to {\em out}. In each step of the decryption a block of
$keysize$ bits is decrypted. If $more =$ SEC\_MORE,
the remaining block is filled by the next call of  rsa\_encrypt.
If $more =$ SEC\_END, it is padded and the key is deleted
after the decryption.
\\ [1ex]


int {\bf rsa\_sign} {\em (hash, sign)} \\
OctetString {\em *hash}; \\
BitString {\em *sign}; 

rsa\_sign creates a signature {\em sign} of {\em hash} with the secret key previously 
set by rsa\_set\_key
and deletes that key afterwards.
\\ [1ex]


int {\bf rsa\_verify} {\em (hash, sign)} \\
OctetString {\em *hash}; \\
BitString {\em *sign}; 

rsa\_verify checks the signature {\em sign} of {\em hash} with the public key previously
set by rsa\_set\_key
and deletes that key afterwards. In case of success $0$ is returned, otherwise
$-1$ is returned.
\\ [1ex]


\hl{DSA Functions}


int {\bf dsa\_gen\_key} {\em (keysize, skey, pkey)} \\
int {\em keysize}; \\
BitString {\em **skey}; \\
BitString {\em **pkey}; 

dsa\_gen\_key generates a DSA key pair with a prime modulus $p$ of {\em keysize} bits.
Pointers to the ASN.1 encoded secret and public key are returned in {\em *skey} and {\em *pkey}.
See INTRO(3) about the ASN.1 structure of DSA keys.
\\ [1ex]

int {\bf dsa\_set\_key} {\em (key, keytype)} \\
char {\em *key}; \\
int {\em keytype}; 

dsa\_set\_key stores the given {\em key} in a static
structure for further use through dsa\_sign() or dsa\_verify().
{\em key} may be a secret or public key. If {\em keytype} is 0, {\em key} must be a pointer to a KeyBits structure, otherwise it must be a pointer to a BitString which
contains the ASN.1 encoded key. dsa\_set\_key casts {\em key} to the appropriate type.
\\ [1ex]


int {\bf dsa\_sign} {\em (hash, sign)} \\
OctetString {\em *hash}; \\
BitString {\em *sign}; 

dsa\_sign creates a signature {\em sign} of {\em hash} with the key previously set by dsa\_set\_key
and deletes that key afterwards.
\\ [1ex]


int {\bf dsa\_verify} {\em (hash, sign)} \\
OctetString {\em *hash}; \\
BitString {\em *sign}; 

dsa\_verify checks the signature {\em sign} of {\em hash} with the key previously set by dsa\_set\_key
and deletes that key afterwards. In case of success, $0$ is returned. Otherwise, $-1$
is returned.
\\ [1ex]


\hl{DES Functions}

int {\bf des\_encrypt} {\em (in, out, more, keyinfo)} \\
OctetString {\em *in}; \\
BitString {\em *out}; \\
More {\em more}; \\
KeyInfo {\em *keyinfo}; 

des\_encrypt encrypts {\em in} with the key given in {\em keyinfo} and appends the result
to {\em out}. The DES mode and padding method modes are derived from the algorithm identifier
contained in {\em keyinfo}.
In each step of the encryption a block of
$64$ bits is encrypted. If $more =$ SEC\_MORE,
the remaining block is filled by the next call of  des\_encrypt.
If $more =$ SEC\_END, it is padded and the key, which was hold in static memory since
the first call, is deleted after the encryption.

A DES key is a BitString of 64 bits length, or of 128 bits length in case of DES 3 mode.
It can be created with sec\_random\_bstr(64).
\\ [1ex]

int {\bf des\_decrypt} {\em (in, out, more, keyinfo)} \\
BitString {\em *in}; \\
OctetString {\em *out}; \\
More {\em more}; \\
KeyInfo {\em *keyinfo}; 

des\_decrypt decrypts {\em in} with the key given in {\em keyinfo} and appends the result
to {\em out}. The DES mode and padding methods are derived from the algorithm identifier
contained in {\em keyinfo}.
In each step of the decryption a block of
$64$ bits is decrypted. If $more =$ SEC\_MORE,
the remaining block is filled by the next call of  des\_decrypt.
If $more =$ SEC\_END, the padding of the last decrypted block is tested and the key, which was hold in static memory since
the first call, is deleted.
\\ [1ex]




\hl{IDEA Functions}



int {\bf idea\_encrypt} {\em (in, out, more, keyinfo)} \\
OctetString {\em *in}; \\
BitString {\em *out}; \\
More {\em more}; \\
KeyInfo {\em *keyinfo}; 

idea\_encrypt encrypts {\em in} with the key given in {\em keyinfo} and appends the result
to {\em out}. 
In each step of the encryption a block of
$64$ bits is encrypted. If $more =$ SEC\_MORE,
the remaining block is filled by the next call of  idea\_encrypt.
If $more =$ SEC\_END, it is padded and the key, which was hold in static memory since
the first call, is deleted after the encryption.

An IDEA key is a BitString of 128 bits length.
It can be created with sec\_random\_bstr(128).
\\ [1ex]


int {\bf idea\_decrypt} {\em (in, out, more, keyinfo)} \\
OctetString {\em *in}; \\
BitString {\em *out}; \\
More {\em more}; \\
KeyInfo {\em *keyinfo}; 

idea\_decrypt decrypts {\em in} with the key given in {\em keyinfo} and appends the result
to {\em out}.
In each step of the decryption a block of
$64$ bits is decrypted. If $more =$ SEC\_MORE,
the remaining block is filled by the next call of  idea\_decrypt.
If $more =$ SEC\_END, the padding of the last decrypted block is tested and the key, which was hold in static memory since
the first call, is deleted.
\\ [1ex]



\hl{Hash Functions}



int {\bf sha\_hash} {\em (in\_octets, hash\_result, more)} \\
OctetString {\em *in\_octets}; \\
OctetString {\em *hash\_result}; \\
More {\em more}; 

sha\_hash hashes {\em in\_octets}. In each step of the hashing a block of
$160$ bits is computed. If $more =$ SEC\_MORE,
the remaining block is filled by the next call of  sha\_hash.
If $more =$ SEC\_END, it is padded and the result of the last hashing is stored in {\em hash\_result}.
\\ [1ex]


int {\bf md*\_hash} {\em (in\_octets, hash\_result, more)} \\
OctetString {\em *in\_octets}; \\
OctetString {\em *hash\_result}; \\
More {\em more}; 

md*\_hash hashes {\em in\_octets}. * may be 2, 4 or 5. In each step of the hashing a block of
$128$ bits is computed. If $more =$ SEC\_MORE,
the remaining block is filled by the next call of  md*\_hash.
If $more =$ SEC\_END, it is padded and the result of the last hashing is stored in {\em hash\_result}.
\\ [1ex]


int {\bf hash\_sqmodn} {\em (in\_octets, hash\_result, more, keysize)} \\
OctetString {\em *in\_octets}; \\
OctetString {\em *hash\_result}; \\
More {\em more}; \\
int {\em keysize};

hash\_sqmodn hashes {\em in\_octets}. For this operation it needs a public RSA key with 
modulus length {\em keysize} previously set by rsa\_set\_key.
In each step of the hashing a block of
$(keysize - 1) / 16$ octets is computed. If $more =$ SEC\_MORE,
the remaining block is filled by the next call of  hash\_sqmodn.
If $more =$ SEC\_END, it is padded and the result of the last hashing is stored in {\em hash\_result}.

