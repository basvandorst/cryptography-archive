\section{Operation of Signature Keys}
\markboth{Operation Signature}{Operation Signature}
\thispagestyle{myheadings}
\label{ops}

\subsection{Verification of a Digital Signature}
\label{ops-vds}

In SecuDe
a digital signature is always part of a quadruplet
({\em text, signatur, creation date, originator certificate}).
The sender of a signature sends always the {\em complete}
originator certificate.
The recipient performs these steps in order to verify a signature
completely:

\begin{enumerate}
\item Select optimal chain of certificates from the originator certificate.
\item Compare signature creation date with
validity time frames of all certificates
of the chain found in step 1.
\item Verify the signature of the text and
verify the signatures of all certificates
of the chain found in step 1.
\end{enumerate}

Step 1 is to read through the {\em certificates} of the
originator certificate from bottom to top.
A chain of certificates is a sequence of certificates,
which fulfills three conditions:
There is one certificate per hierarchy level,
the issuer of any certificate is the owner of the certificate at the
next level above,
and the highest certificate is to be verified by a public key
already known to the verificator.
If, by comparison of the key number of the last certificate
of the chain, the highest certificate is recognized as an old
certificate, the user calls the directory attribute ``Old Certificates''
and appends one cross certificate from the list to the chain.
A chain is optimal, if it is as short as possible.
If, for example, two
users of one certification authority exchange signatures,
the chain of certificates selected in step 1 will comprise only one
certificate, i.e. the user certificate.

\begin {center}
\begin {tabular}{lll}
struct & {\em Returncode} \{ & \\
       & unsigned int  & {\em indication}; /* \\
       &               & 0: OK                \\
       &               & 1: Text Signature NOTOK         \\
       &               & 2: Text Signature EXPIRED       \\
       &               & 3: Certificate Signature NOTOK  \\
       &               & 4: No trusted public key        \\
       &               & 5: Certificate Revoked          \\
       &               & */                    \\
       & Certificate   & *{\em current\_certificate}; /* \\
       &               & currently checked certificate,  \\
       &               & NULL, if indication=0,1 or 4    \\
       &               & */                    \\
       & \}            & \\
\end {tabular}
\end {center}
 
\label{fig-ops-vds}
\stepcounter{Abb}
{\footnotesize Fig.\arabic{Abb}: Returncode of the verification algorithm}
 
\subsection{Change of Signature Keys or of Certificates}
\label{ops-csk}

Within SecuDe
any certification authority is responsible for all services to the users,
in particular for the update service of their smartcards.
This is even so if the changes are initiated by other, e.g. higher,
certification authorities.

There are many reasons for signature or encryption keys to change,
for example if new algorithms are inplemented, or
if keys are lost or suspected to be stolen.
The update procedure of a pair of keys is different
for a user-key and a CA-key (see below).

A change of a key makes a change of its certificate necessary.
Conversely, however, a certificate can be updated with
respect to the same public key, i.e. certificates can be extended.
A change of any CA-certificate
makes all user originator certificates below that CA change:
that is, their smartcard entries {\em FCPath} (see paragraph \ref{fcpath})
are to be updated.

Changes of CA-keys have effects on all users below.
In principle, CAs could change their keys independently of
one another. However, the increasing load of smartcard updating on
the users should be taken into account.
Best is a coordinated change of keys of many CAs within one tree
or subtree, possibly not more often than every couple of years.
The change of a user key, however, is not critical for other users
or for CAs, because that is completely in the hand
of the user concerned. One CA to which the user is attached
must become active on the user's request in that it has to sign the
user certificate.

\subsubsection{Extension of a Certificate (Unchanged Key)}
\label{ops-ec}

Expiring Certificates are replaced by new certificates.
The {\em validity-notBefore} (which marks the creation date of the key)
remains the same.
The expired certificates are not stored, because the new certificates
serve the same purpose; the old certificates remain valid for
old signatures of the past time of validity.

\subsubsection{Change of a User Key}
\label{ops-cuk}

Users create their own pairs of keys.
They also create prototype certificates and ask their CAs
to sign it (cf. paragraph \ref{ca-cuk}).

Changing user keys is simply replacing the old keys by the new keys.
The old keys are not stored.
The user-certificate changes, too.
The old user-certificate is not stored.
However, the forward certification path remains unaffected.
It is the responsibility of a recipient of a signature to
keep the originator certificate of that signature, if he
wishes to be able to verify the signature even at later times, when the
sender's signature key has changed.

A user should inform all his important communication partners
about his new public key.
The best way to do that is to send a signed message
together with the new originator-certificate to all of them.
This allows the partners to update their smartcard lists of trusted
verification keys with respect to that new key.

\subsubsection{Change of a CA Key}
\label{ops-cck}

A CA key is a signature key which is used by a certification authority (CA)
in order to sign certificates for subordinate certification authorities
or for users.
A change of a CA key has effect on one level below and one level above
and on all users who are anywhere below in the certification tree.
Therefore, these elements are to be changed:

\begin{enumerate}
\item the certificates ``immediately above'' the CA;
and its cross-certificates by other CAs;
\item the certificates ``immediately below'' the CA;
and its cross-certificates to other CAs;
\item the originator certificates (i.e. the {\em FCPath} infos) of
all users anywhere below the CA;
\item all smartcard entries ``Trusted Public CA Keys'' ({\em PKList infos}) of
all users anywhere below the CA.
\end{enumerate}

This changes the forward certification path (``FCPath'')
and the list of trusted public verification keys (``PKList'')
of all users below this CA.
It is the duty of every user-CA to provide {\em its}
users with the updated FCPath and PKList,
regardless of which CA has caused the change.

\subsubsection{Change of a Root CA Key}
\label{ops-crk}

The public verification key of a root certification authority is
not certified. It is rather stored in all users' smartcard entries
``Public Root-CA-Keys'' (see paragraph \ref{pkroot}).
Therefore, these elements are to be changed:

\begin{enumerate}
\item the directory attribute ``Old Certificates'';
\item the smartcard entries ``Public Root-CA-Keys''
of all users of this certification tree;
\item the certificates ``immediately below'' the CA;
\item the originator certificates (i.e. the FCPaths) of
of all users of this certification tree.
\end{enumerate}

It is of significant importance, that 1 and 2 are changed {\em first}
and 3 and 4 are changed a {\em certain amount of time thereafter}.

1: The Root CA cross-certifies her newly created public key
with her expiring signature key.
The validity time interval of this cross certificate starts
with the creation date of the newly created key
and ends with its planned expiry.
Remember, that validity values refer to the certified keys.
This cross certificate becomes the new first line of the Root CA's
directory table ``Old Certificates'',
and the new first line's serial number is the old first line's
key number plus one.
Then, in all the other lines (now lines 2 thru last) the Root CA replaces
the certificates' signatures and serial numbers,
whereby other certificate attributes including
certified keys and validity values remain unchanged.
Again, this is consistent, because
validity values refer to the certified keys.
Of course, the table's serial numbers of those lines remain the same, too.

2: Together with step 1,
the Root CA informs all CAs about the change of her keys.
The user-CAs
send the new PKRoot information to all the users they are
directly responsible for.
The users update their smartcard entries {\em PKRoot} (cf \ref{pkroot}).

3 and 4: In complete analogy with the respective steps
during the change of keys of any CA,
all user-CAs
send the new FCPath information to all the users they are
directly responsible for.
The users update their smartcard entries {\em FCPath} (cf. \ref{fcpath}).
If any CA changes her key in line with this procedure,
the users are given new {\em PKList} information (cf. \ref{pklist})
and if necessary new user certificates as well.

It is important, that between steps 1/2 and steps 3/4
there is an suitable interval of time.
In some environments, several months are suitable;
in others which are more strictly organized the time interval may be shorter.
In that time, the root CA may also use her new key for normal communication, 
if the directory table ``Old Certificates'' is available for users
(see ``note'' in paragraph \ref{ds-qa} above).
However, she must use it for the certificates to be created.
Every user in this certification tree is affected twice:
the first time, when he updates his smartcard table {\em PKRoot},
the second time after the time interval between steps 1/2 and 3/4,
when he updates his smartcard table {\em FCPath}.
Other changes like the change of {\em PKList} or of his own user key do not
interfere with this process.

There are three phases of a user smartcard during this process:

\begin{enumerate}
\item before change of the smartcard table {\em PKRoot},
\item after change of the smartcard table {\em PKRoot},
but before change of the smartcard table {\em FCPath},
\item after change of the smartcard table {\em FCPath}.
\end{enumerate}

During all three phases a user can use his smartcard without
any restriction. In order to put light on this dynamic process,
consider a user S to be an originator of a message,
sending it to the recipient R.
Let S and R be located within one certification tree.
S signs the message, while R verifies the signature.
Note, that the change of the {\em PKRoot} info on the smartcard
does not affect the signature process,
while the change of {\em FCPath} info on the smartcard does not
affect the verification process.
As a consequence, any signer would behave the same in phases 1 and 2 above,
while any verifier would behave the same in phases 2 and 3.
\\[1ex]
Let S and R be both in phase 1 or both in phase 3:
\\[1ex]
These situations are obviously trivial.
\\[1ex]
S in phase 1 and R in phase 2:
\\[1ex]
The highest certificate in the originator-certificate sent by S is ``old''.
For verification,
R needs the ``old'' public Root-CA key, which he has stored
in his smartcard table {\em PKRoot} as ``old Root CA key''.
R does recognize that from the serial number of the highest certificate
within the originator-certificate sent by S.
\\[1ex]
S in phase 2 and R in phase 1:
\\[1ex]
Still,
the highest certificate in the originator-certificate sent by S is ``old''.
For verification,
R needs the ``old'' public Root-CA key, which he has still stored
in his smartcard table {\em PKRoot} as ``new Root CA key''.
From the point of R there is no difference from the situation of
yet unchanged Root CA keys.
\\[1ex]
Both S and R in phase 2:
\\[1ex]
Same as S in phase 1 and R in phase 2,
because S behaves equally in phases 1 and 2.
\\[1ex]
S in phase 2 and R in phase 3:
\\[1ex]
Same as both S and E in phase 2,
because R behaves equally in phases 2 and 3.
\\[1ex]
S in phase 3 and R in phase 2:
\\[1ex]
The highest certificate in the originator-certificate sent by S is ``new''.
For verification,
R needs the ``new'' public Root-CA key, which he has already stored
in his smartcard table {\em PKRoot} as ``new Root CA key'',
since R has already advanced to phase 2.
\\[1ex]
In a disciplined user community it does not happen,
that one part of the users are still in phase 1 while the other
part has already advanced to phase 3.
But even then {\em no security risk} would occur.
``Lazy'' users who remain in phase 1, while all the others
have already advanced to phase 3,
would be no danger neither to themselves nor to others.
They would need the help of the directory table ``Old Certificates''
of the Root CA in order to verify phase 3 signatures.
And their own signatures would be valid,
even if produced now, as long as their personal user certificates
are still valid.
\\[1ex]
S in phase 1 and R in phase 3:
\\[1ex]
Same as S in phase 1 and R in phase 2,
because R behaves equally in phases 2 and 3.
\\[1ex]
S in phase 3 and R in phase 1:
\\[1ex]
The highest certificate in the originator-certificate sent by S is ``new''.
For verification,
R needs the ``new'' public Root-CA key, which he has not (yet) stored
in his smartcard table {\em PKRoot}.
This situation is recognized by R,
in that R compares the certificate serial number of the
the highest certificate in the originator-certificate
with the serial numbers of the Root CA keys on his smartcard.
The latter are older, i.e. smaller.
This should stimulate R to call the directory entry of the Root CA
and check the attribute ``Old Certificates''.
That information is sufficient both to verify S's signature
and to update his own smartcard table {\em PKRoot}.
The second step would raise R at least into phase 2.
\\[1ex]
A smartcard in phase 2 is absolutely up-to-date
as far as its verification functionality is concerned.
If the directory table ``Old Certificates'' is available to users,
a smartcard in phase 1 is also able to verify actual signatures,
in that it can use the first line of that table.
However, that ability would be lost after the {\em next}
change of Root CA keys.
A smartcard in phase 1 wouldn't become a security risk
even then, however useless.
It wouldn't be able to verify then up-to-date signatures,
because the directory would have been updated as well.
\\[1ex]
A smartcard in phase 1 or 2 would be verifiable by other smartcards
which are up-to-date (at least in phase 2)
with the help of their PKRoot-old entries.
However, if a smartcard would remain in phase 1 or 2 even after the {\em next}
change of Root CA keys,
its signatures would be verifiable by others
{\em only} with the help of the directory table ``Old Certificates''
just as if it were a very old signature.
A user-CA can prevent from this kind of
(undangerous, but annoying) situations
if she makes sure, that validity time intervals of her user certificates
do not cover two validity time periods of a Root CA key.

A change of Root CA keys does not affect the communication
across certification tree boundaries,
because due to cross certificate links, Root CAs are not involved
in the verification process.

\subsubsection{Change of Root CA}
\label{ops-cr}

This is much the same as the change of the signature key of the root
certification authority. However, some things are special:

The smartcard entries ``Public Root-CA-Keys'' will have
only one value for the ``new'' public key, but no value
for the ``old'' public key.

If the old root CA ceases to exist, there will be a break of
verification service (however no security risk)
during a time, when one part of the users
have already changed their originator certificates while others have
not. This time of change should be made as short as possible.

Steps have to be undertaken to maintain the old directory
table ``Old Certificates''.

More likely an
old root CA might become a subordinate of another root CA,
for example as a process of regional extension of a certification tree
or if two trees are joined.
If the old root CA becomes a subordinate of another root CA,
the verification service will not break, because the old
public root-CA-key is added to the smartcard tables ``Trusted Public CA Keys''.
It is best, if
this process follows exactly the pattern of a change of Root CA keys:
users would receive the new {\em FCPath} information
only several months after the reception of the new
{\em PKRoot} and {\em PKList} information.

\subsubsection{Support of Old Signatures}
\label{ops-sos}

A signature is ``old'', if no public CA key in the
originator certificate is known to the verificator,
and if the key number of the highest certificate
is smaller than the key numbers of the smartcard entry
``Public Root-CA-Keys''.
Old signatures remain verifiable with the help of the directory
attribute ``Old Certificates'' of the root certification authority.
The smartcard table {\em PKRoot} (see paragraph \ref{pkroot})
covers two validity time periods of Root CA keys.
If as a routine CA keys remain for two or three years,
verification of up to four or five years old signatures
would succeed without directory support.
This would make directory support necessary for rare cases only.

\subsection{Marking Certificates ``Invalid'' (Black Lists)}
\label{ops-mci}

\subsubsection{Black Lists}
\label{ops-bl}

A certification authority can mark any of the certificates which
it is the issuer of as ``invalid'' by adding it to the
list of revoked certificates. It makes those black lists
public by placing them into its directory entry.

\subsubsection{Repudiation of Signatures}
\label{ops-ros}

A valid signature remains valid until its certificate is placed
into a black list.
Note that the placement of a certificate in a black list
makes the respective signature {\em retroactively
invalid from the beginning},
because a creation date of a signature is proved by the signature only,
and can therefore be set back into a past valid time interval
by a signature with a revoked private key.

This design of a data origin authentication service
does not offer a long ranging non-repudiation service
without additional operative agreements.
An example of an operative agreement is
to put the full responsibility for any kind of use and misuse of a key
on the key's owner.
This might be too hard in some application environments
(such as banking) and would not prevent from strict
non-repudiation at court.

Black lists still have useful applications,
the restrictions of which are described in the following cases:
\\[1ex]
{\bf Case 1:}
\\[1ex]
A user who does not follow the agreed policy of a certification
authority can be deprived of his right to use his pair of
keys by placing his certificate
into a black list.
All his signatures before the revocation date would remain valid.
This measure would prevent the user from creating signatures
after the revocation date.
It would, however, not prevent the user to masquerade
newly created signatures as old signatures, dated back before
the revocation date.
This could only be done by setting the revocation date
earlier than the related certificate's value of {\em validity-notBefore}.
However, that would make even those user's old signatures invalid,
which used to be valid when they were produced.
\\[1ex]
{\bf Case 2:}
\\[1ex]
If a user reports,that his private key is stolen,
there is only one way to protect him against
back-dated counterfeit signatures:
his certificate must be revoked with revocation date
earlier than the certificate's value of {\em validity-notBefore}.
Unfortunately, the unlucky user will have to re-sign
all his old signatures,
which he wanted to remain valid, with a newly created private key.
\\[1ex]
{\bf Case 3:}
\\[1ex]
A malevolent user might want to repudiate one or more of his old signatures.
He simply reports, that his private key is stolen,
and enjoys the revocation of all his old signatures
while he is supplied with a new key (see case 2 above).
He will, however, be unable to repudiate
those among his old signatures, which are notarized
by a third-party signature.
\\[1ex]
{\bf Case 4:}
\\[1ex]
Fraudulent cooperation between users and notarization authorities
can make old signatures invalid in that they ask for revocation
of their certificates together. Although theoretically possible,
it would hardly be believable in practice.
A notarization authority's professional reputation would be destroyed.
\\[1ex]
{\bf Case 5:}
\\[1ex]
A compromised CA key would be most desastrous, but must be
considered as possible, for example by an insider attack.
This CA's black lists were compromised as a whole.
Black lists of superior CAs could and should, of course,
revoke this CA's certificate.
This would, however, affect all user certificates below
the compromised CA.
All the old signatures would immediately loose their validity,
because a CA key can limitless produce certificates of any
validity time intervals, even of the past.
There is no counter-measure against this,
except to handle CA keys very very carefully.
Users should choose notarization authorities which
are ``far away'' from their own position in the certification tree.


\subsubsection{Long Term Validity of Signatures}
\label{ops-ltv}

A signature is
theoretically absolutely valid only at the moment of its verification.
In that, the time elapsed between creation
and verification of the signature plays no role.
However, the valid verification today does not guarantee,
that tomorrow the signature would also be accepted as valid yesterday,
because it might have been put into a black list meanwhile.
Therefore, consequences of signatures for real actions should be taken
as soon as possible.
If a verificator of a signature has ``forgotten'' the signature,
because its realization is completed,
a later revocation has no effect.

Based on operational cooperation within a community,
there are several steps which can be taken in order to force
signatures to have a longer validity.

\begin{enumerate}
\item Signatures are valid until Revocation.
In reality this means: Realize a signature at the moment of its verification.
\item Signatures are valid at least 1 day (1 week, 2 weeks, 1 month, etc.)
after revocation.
The users (their insurances, CAs, etc.) are liable for misuse.
In reality this means: Realize a signature within the guaranteed
period of time.
\item Long term validity of signatures can be guaranteed by adding
notarization signatures to a signed text.
A multiple signature is as questionable as the probability
of fraudulent cooperation between users and authorities.
\\
In general a valid signature remains valid even if it is revoked later,
if the creation date is proved by other means than the signature itself,
e.g. by a notarization signature.
\item There are high sanctions on placing a certificate into a black list.
\\
This takes into account, that in general users want to be long-term
members of a communicating community.
\end{enumerate}

\subsubsection{Non-Repudiation with Notarization Signature}
\label{ops-not}

The best guarantee for a long term validity of a signature is
to add other third-party signatures to a signed document.
This can be initiated by the users themselves,
but there might be also services of certification authorities.

\subsubsection{Examples}
\label{ops-dfn}

As a simple example, digital signatures could be used for the following services:

\begin{itemize}
\item access control (Login, Directory, FT, RJC);
\item proof of integrity of transmitted data (MHS, Directory, FT);
\item proof of originality of transmitted data (MHS, Directory, FT);
\item mutual authentication (entities MTA, DSA, etc., end users).
\end{itemize}

None of these services needs long term validity.
However, signed documents might be stored for later verification.
Users are also free to initiate notarization signatures for their
signed documents, in order to guarantee their long term validity.

The verification of old signatures should be supported in this case.
A non-repudiation service will could be optional
(by user actions).

Old certificates should be supported by the directory table ``Old Certificates''
in which old public keys are cross-certified by the currently valid CA key.
It is sufficient, if only the root certification authority
maintains this directory attribute.
