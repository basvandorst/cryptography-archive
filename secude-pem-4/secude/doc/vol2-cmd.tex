\nm{1}{algs}{show available cryptographic algorithms}
\label{algs}
\hl{Synopsis}
{\bf algs} [ {\bf -UGvVZh} ] [ {\bf -a} {\em name} ] [ {\bf -s} {\em keyword} ] [ {\bf -k} {\em k1 k2 k3 ...} ] [ {\em quantity} ]
\hl{Description}
{\em algs} prints information about all available cryptographic algorithms, or about all algorithms 
which correspond to the given {\em keyword}, or about the algorithm with the given {\em name}.
{\em keyword} can be a pattern which is contained in the algorithm name (e.g. rsa or md2), or one of the 
following algorithm types:
\bvtab
ASYM\_ENC \4 asymmetric encryption algorithms \\
SYM\_ENC  \4 symmetric encryption algorithms  \\
HASH      \4 hash (message digest) algorithms \\
SIG       \4 signature algorithms  \\
KEY\_AGREEMENT \4 Diffie-Hellman key agreement
\evtab
{\em keyword} is not case-sensitive.

If option {\bf -a} is omitted, only the algorithm names are printed. 
Otherwise, the object identifiers and default parameter values are additionally printed. 

With option {\bf -U} online time measurements are performed with precomputed or newly generated keys. 
In case of asymmetric algorithms, keys with the keysizes {\em k1, k2, k3, ..} are used to 
encrypt/decrypt a corresponding block. If {\bf -k} {\em k1 k2 k3 ...} is omitted, keysizes
512, 640, 768 and 1024 are used.

In case of symmetric algorithms and message digest algorithms, an octetstring of {\em quantity} K
octets is encrypted/decrypted or hashed. If {\em quantity} is omitted, 100 K octets are
encrypted/decrypted or hashed.

In case of signature algorithms, an octetstring of {\em quantity} K octets is signed
and verified with keys of keysizes {\em k1, k2, k3, ...}. If {\em quantity} is omitted, 100 K octets are 
signed/verified. If {\bf -k} {\em k1 k2 k3 ...} is omitted, keysizes 512, 640, 768 and 1024 are used.

algs uses the PSE \$TOP/lib/.testkeys. If this PSE does not exist, it is created. If through
parameter {\bf -k} {\em k1 k2 k3 ...} keysizes are requested for which no keys exist,
they are generated and stored afterwards in \$TOP/lib/.testkeys.

With option {\bf -G} additional time measurements will be made for the key generation (asymmetric 
algorithms only) even if the requested keys exist already.

The figures presented here provide a rough impression about the performance of the algorithms on the 
installation platform. Note that the asymmetric algorithms can be installed on the
basis of C-programs or assembler programs. Time measurements are only possible for the software 
implementations of the algorithms. The performance of the smartcard terminal algorithm
implementation cannot be measured.

An example output of algs produced on a Sun Sparc10-41 workstation is shown below:
{\scriptsize
\begin{verbatim}

SYM_ENC Algorithms
==================
-------------------------------------------------------------------------------------
DES-ECB              OID { 1 3 14 3 2 6 } NULL parameter
                     Encr (100 Kbytes):   0.891  
                     Decr (100 Kbytes):   0.875  
-------------------------------------------------------------------------------------
DES-CBC              OID { 1 3 14 3 2 7 } Parameter DES-IV (default zeros)
                     Encr (100 Kbytes):   0.928  
                     Decr (100 Kbytes):   0.925  
-------------------------------------------------------------------------------------
DES-EDE              OID { 1 3 14 3 2 17 } NULL parameter
                     Encr (100 Kbytes):   2.634  
                     Decr (100 Kbytes):   2.630  
-------------------------------------------------------------------------------------
desCBC3              OID { 1 3 36 3 1 11 } Parameter DES-IV (default zeros)
                     Encr (100 Kbytes):   2.652  
                     Decr (100 Kbytes):   2.673  
-------------------------------------------------------------------------------------
IDEA                 OID { 1 3 36 3 1 30 } NULL parameter
                     Encr (100 Kbytes):   0.903  
                     Decr (100 Kbytes):   0.905  

ASYM_ENC Algorithms (one block encrypted)
=========================================
-------------------------------------------------------------------------------------
RSA                  OID { 1 2 840 113549 1 1 1 } NULL parameter
                     Encr ( 512 bits):   0.014  
                     Decr ( 512 bits):   0.172  
                     KGen ( 512 bits):   9.839  
                     Encr ( 640 bits):   0.020  
                     Decr ( 640 bits):   0.333  
                     KGen ( 640 bits):  27.564  
                     Encr ( 768 bits):   0.028  
                     Decr ( 768 bits):   0.495  
                     KGen ( 768 bits):  25.495  
                     Encr (1024 bits):   0.048  
                     Decr (1024 bits):   1.103  
                     KGen (1024 bits):  86.174  

HASH Algorithms
===============
-------------------------------------------------------------------------------------
RSA-MD2              OID { 1 2 840 113549 2 2 } NULL parameter
                     Hash (100 Kbytes):   1.076  
-------------------------------------------------------------------------------------
RSA-MD5              OID { 1 2 840 113549 2 5 } NULL parameter
                     Hash (100 Kbytes):   0.035  
-------------------------------------------------------------------------------------
NIST-SHA             OID { 1 3 14 3 2 18 } NULL parameter
                     Hash (100 Kbytes):   0.045  
-------------------------------------------------------------------------------------
sqmodn               OID { 2 5 8 2 1 } Parameter Keysize (default 512)
                     Hash (100 Kbytes):   2.663  
-------------------------------------------------------------------------------------
md4                  OID { 1 2 840 113549 2 4 } NULL parameter
                     Hash (100 Kbytes):   0.028  

SIG Algorithms (100 K message signed)
=====================================
-------------------------------------------------------------------------------------
sqmodnWithRsa        OID { 2 5 8 3 1 } Parameter Keysize (default 512)
                     Sign ( 512 bits):   2.993 (total),  0.184 (RSA),  2.735 (SQMODN)  
                     Veri ( 512 bits):   2.752 (total),  0.013 (RSA),  2.737 (SQMODN)  
                     Sign ( 640 bits):   3.592 (total),  0.334 (RSA),  3.256 (SQMODN)  
                     Veri ( 640 bits):   3.273 (total),  0.021 (RSA),  3.250 (SQMODN)  
                     Sign ( 768 bits):   4.232 (total),  0.495 (RSA),  3.736 (SQMODN)  
                     Veri ( 768 bits):   3.745 (total),  0.028 (RSA),  3.716 (SQMODN)  
                     Sign (1024 bits):   6.079 (total),  1.123 (RSA),  4.955 (SQMODN)  
                     Veri (1024 bits):   4.901 (total),  0.048 (RSA),  4.850 (SQMODN)  
-------------------------------------------------------------------------------------
md2WithRsa           OID { 1 3 14 7 2 3 1 } NULL parameter
                     Sign ( 512 bits):   1.310 (total),  0.166 (RSA),  1.142 (MD2)  
                     Veri ( 512 bits):   1.115 (total),  0.013 (RSA),  1.101 (MD2)  
                     Sign ( 640 bits):   1.444 (total),  0.323 (RSA),  1.119 (MD2)  
                     Veri ( 640 bits):   1.149 (total),  0.020 (RSA),  1.127 (MD2)  
                     Sign ( 768 bits):   1.572 (total),  0.474 (RSA),  1.097 (MD2)  
                     Veri ( 768 bits):   1.135 (total),  0.028 (RSA),  1.106 (MD2)  
                     Sign (1024 bits):   2.186 (total),  1.093 (RSA),  1.092 (MD2)  
                     Veri (1024 bits):   1.147 (total),  0.052 (RSA),  1.094 (MD2)  
-------------------------------------------------------------------------------------
md4WithRsa           OID { 1 3 14 3 2 2 } NULL parameter
                     Sign ( 512 bits):   0.212 (total),  0.168 (RSA),  0.043 (MD4)  
                     Veri ( 512 bits):   0.042 (total),  0.013 (RSA),  0.027 (MD4)  
                     Sign ( 640 bits):   0.353 (total),  0.317 (RSA),  0.034 (MD4)  
                     Veri ( 640 bits):   0.048 (total),  0.020 (RSA),  0.027 (MD4)  
                     Sign ( 768 bits):   0.526 (total),  0.497 (RSA),  0.028 (MD4)  
                     Veri ( 768 bits):   0.056 (total),  0.028 (RSA),  0.027 (MD4)  
                     Sign (1024 bits):   1.186 (total),  1.158 (RSA),  0.027 (MD4)  
                     Veri (1024 bits):   0.078 (total),  0.049 (RSA),  0.027 (MD4)  
-------------------------------------------------------------------------------------
md5WithRsa           OID { 1 3 14 3 2 3 } NULL parameter
                     Sign ( 512 bits):   0.218 (total),  0.182 (RSA),  0.035 (MD5)  
                     Veri ( 512 bits):   0.050 (total),  0.013 (RSA),  0.035 (MD5)  
                     Sign ( 640 bits):   0.354 (total),  0.318 (RSA),  0.035 (MD5)  
                     Veri ( 640 bits):   0.056 (total),  0.020 (RSA),  0.035 (MD5)  
                     Sign ( 768 bits):   0.527 (total),  0.492 (RSA),  0.035 (MD5)  
                     Veri ( 768 bits):   0.064 (total),  0.028 (RSA),  0.035 (MD5)  
                     Sign (1024 bits):   1.109 (total),  1.071 (RSA),  0.037 (MD5)  
                     Veri (1024 bits):   0.085 (total),  0.048 (RSA),  0.035 (MD5)  
-------------------------------------------------------------------------------------
dsaWithSHA           OID { 1 3 14 3 2 13 } NULL parameter
                     Sign ( 512 bits):   0.235 (total),  0.197 (DSA),  0.037 (SHA)  
                     Veri ( 512 bits):   0.393 (total),  0.347 (DSA),  0.044 (SHA)  
                     Sign ( 640 bits):   0.293 (total),  0.255 (DSA),  0.037 (SHA)  
                     Veri ( 640 bits):   0.536 (total),  0.498 (DSA),  0.037 (SHA)  
                     Sign ( 768 bits):   0.367 (total),  0.329 (DSA),  0.037 (SHA)  
                     Veri ( 768 bits):   0.747 (total),  0.708 (DSA),  0.037 (SHA)  
                     Sign (1024 bits):   0.622 (total),  0.584 (DSA),  0.037 (SHA)  
                     Veri (1024 bits):   1.235 (total),  1.197 (DSA),  0.037 (SHA)  
-------------------------------------------------------------------------------------
md2WithRsaEncryption OID { 1 2 840 113549 1 1 2 } NULL parameter
                     Sign ( 512 bits):   1.368 (total),  0.186 (RSA),  1.181 (MD2)  
                     Veri ( 512 bits):   1.116 (total),  0.015 (RSA),  1.100 (MD2)  
                     Sign ( 640 bits):   1.431 (total),  0.341 (RSA),  1.089 (MD2)  
                     Veri ( 640 bits):   1.108 (total),  0.021 (RSA),  1.086 (MD2)  
                     Sign ( 768 bits):   1.580 (total),  0.488 (RSA),  1.090 (MD2)  
                     Veri ( 768 bits):   1.128 (total),  0.028 (RSA),  1.099 (MD2)  
                     Sign (1024 bits):   2.195 (total),  1.110 (RSA),  1.084 (MD2)  
                     Veri (1024 bits):   1.136 (total),  0.049 (RSA),  1.086 (MD2)  
-------------------------------------------------------------------------------------
md4WithRsaEncryption OID { 1 3 14 3 2 4 } NULL parameter
                     Sign ( 512 bits):   0.201 (total),  0.172 (RSA),  0.028 (MD4)  
                     Veri ( 512 bits):   0.044 (total),  0.014 (RSA),  0.029 (MD4)  
                     Sign ( 640 bits):   0.376 (total),  0.342 (RSA),  0.033 (MD4)  
                     Veri ( 640 bits):   0.049 (total),  0.020 (RSA),  0.027 (MD4)  
                     Sign ( 768 bits):   0.524 (total),  0.496 (RSA),  0.027 (MD4)  
                     Veri ( 768 bits):   0.057 (total),  0.028 (RSA),  0.028 (MD4)  
                     Sign (1024 bits):   1.149 (total),  1.121 (RSA),  0.027 (MD4)  
                     Veri (1024 bits):   0.077 (total),  0.049 (RSA),  0.027 (MD4)  
-------------------------------------------------------------------------------------
md5WithRsaEncryption OID { 1 2 840 113549 1 1 4 } NULL parameter
                     Sign ( 512 bits):   0.216 (total),  0.181 (RSA),  0.035 (MD5)  
                     Veri ( 512 bits):   0.050 (total),  0.014 (RSA),  0.035 (MD5)  
                     Sign ( 640 bits):   0.363 (total),  0.327 (RSA),  0.035 (MD5)  
                     Veri ( 640 bits):   0.056 (total),  0.020 (RSA),  0.035 (MD5)  
                     Sign ( 768 bits):   0.537 (total),  0.501 (RSA),  0.035 (MD5)  
                     Veri ( 768 bits):   0.064 (total),  0.028 (RSA),  0.035 (MD5)  
                     Sign (1024 bits):   1.152 (total),  1.116 (RSA),  0.035 (MD5)  
                     Veri (1024 bits):   0.086 (total),  0.049 (RSA),  0.035 (MD5)  
-------------------------------------------------------------------------------------
md2WithRsaTimeDate   OID { 1 3 36 3 1 22 } NULL parameter
                     Sign ( 512 bits):   1.277 (total),  0.189 (RSA),  1.087 (MD2)  
                     Veri ( 512 bits):   1.119 (total),  0.029 (RSA),  1.087 (MD2)  
                     Sign ( 640 bits):   1.423 (total),  0.327 (RSA),  1.095 (MD2)  
                     Veri ( 640 bits):   1.136 (total),  0.041 (RSA),  1.094 (MD2)  
                     Sign ( 768 bits):   1.594 (total),  0.503 (RSA),  1.089 (MD2)  
                     Veri ( 768 bits):   1.150 (total),  0.056 (RSA),  1.092 (MD2)  
                     Sign (1024 bits):   2.196 (total),  1.109 (RSA),  1.086 (MD2)  
                     Veri (1024 bits):   1.206 (total),  0.116 (RSA),  1.089 (MD2)  
-------------------------------------------------------------------------------------
md4WithRsaTimeDate   OID { 1 3 36 3 1 24 } NULL parameter
                     Sign ( 512 bits):   0.208 (total),  0.180 (RSA),  0.027 (MD4)  
                     Veri ( 512 bits):   0.057 (total),  0.028 (RSA),  0.027 (MD4)  
                     Sign ( 640 bits):   0.356 (total),  0.328 (RSA),  0.027 (MD4)  
                     Veri ( 640 bits):   0.069 (total),  0.041 (RSA),  0.027 (MD4)  
                     Sign ( 768 bits):   0.534 (total),  0.505 (RSA),  0.027 (MD4)  
                     Veri ( 768 bits):   0.085 (total),  0.056 (RSA),  0.027 (MD4)  
                     Sign (1024 bits):   1.248 (total),  1.219 (RSA),  0.027 (MD4)  
                     Veri (1024 bits):   0.129 (total),  0.100 (RSA),  0.027 (MD4)  
-------------------------------------------------------------------------------------
md5WithRsaTimeDate   OID { 1 3 36 3 1 25 } NULL parameter
                     Sign ( 512 bits):   0.209 (total),  0.173 (RSA),  0.035 (MD5)  
                     Veri ( 512 bits):   0.063 (total),  0.027 (RSA),  0.035 (MD5)  
                     Sign ( 640 bits):   0.372 (total),  0.336 (RSA),  0.035 (MD5)  
                     Veri ( 640 bits):   0.077 (total),  0.041 (RSA),  0.035 (MD5)  
                     Sign ( 768 bits):   0.545 (total),  0.509 (RSA),  0.035 (MD5)  
                     Veri ( 768 bits):   0.093 (total),  0.056 (RSA),  0.035 (MD5)  
                     Sign (1024 bits):   1.145 (total),  1.109 (RSA),  0.035 (MD5)  
                     Veri (1024 bits):   0.134 (total),  0.098 (RSA),  0.035 (MD5)  
\end{verbatim}
}

\nm{1}{psemaint}{maintain PSE}
\label{psemaint}
\hl{Synopsis}
{\bf psemaint} [ {\bf -hzvACFRDTVWXY} ] [ {\bf -c} {\em cadir} ] [ {\bf -p} {\em psename} ] [ {\bf -a} {\em 
issueralg} ] [ {\bf -f} {\em notbefore} ] 
\\ \hspace*{1.59cm} [ {\bf -l} {\em notafter} ] [ {\bf -i} {\em inputfile} ] [ {\bf -d} {\em dsaname} ] [ {\bf -A} {\em authlevel} ] [ {\bf -t} {\em dsaptailor} ][ {\em command} ]
\hl{Description}
{\bf psemaint} is a maintenance program which can be used by both
certification authority administrators and users for the purpose of maintaining their
PSEs. This includes moving information (e.g. keys, certificates, revocation lists etc.) from Unix files
or public X.500 directories into the PSE and vice versa, generating keys, changing PINs and displaying the 
content
of the PSE. If a CA PSE is addressed, the CA database contained in the CA directory is used, too.
In the basic version of SECUDE the PSE is realized by software.

If {\bf -p} {\em psename} is given, the PSE {\em psename} is selected. The default for
{\em psename} is either provided by the environment variable PSE, or is {\em .pse} under the
HOME directory, if no such variable exists.

If {\bf -c} {\em cadir} is given, the PSE {\em psename} of a CA residing in the CA directory {\em cadir}
is selected; in this case, the default for PSE is either provided by the environment
variable CAPSE or is {\em .capse}, if no such variable exists.

{\em notbefore} (option {\bf -f}) and {\em notafter} (option {\bf -l}) are dates of validity in the 
form 'YYMMDD'. They are evaluated by the {\bf certify} command available with {\bf psemaint}. 
If they are omitted, the first date on which a certificate generated with the {\bf certify} command
is valid, is set to the current time, 
and the last date on which the certificate is valid is set to one year hereafter.

If option {\bf -D} is provided, Directory-based information may be stored on the PSE, or
information stored on the PSE may be entered into the Directory.
The Directory may either be a 
X.500 directory or the local substitute provided by 
SecuDE ({\em .af-db}) (see also DirectoryType(5)).

If a X.500 directory is accessed, option {\bf -d} may be used to specify the DSA (Directory System 
Agent) 
(identified by {\em dsaname}) which shall be contacted for the required directory operation; otherwise, the DSA whose name and network address are defined by the {\em dsa\_address} parameter in the system's 
{\em dsaptailor} file will be accessed. 
Parameter {\em authlevel} (option {\bf -A}) may be provided to determine the level of authentication used when binding. If {\bf -A} {\em authlevel} is omitted, the requested 
level of authentication is taken from the environment variable AUTHLEVEL, and is set to {\em no 
authentication}, if no such variable exists.
In case of simple authentication being used, the caller's distinguished name and
X.500 password are retrieved from the PSE objects Name and QuipuPWD, respectively.

If a command available with {\bf psemaint} requires a signature to be performed (e.g. command {\em 
certify}) and the own
signature key is of type {\em ASYM\_ENC}, the algorithm name associated with the signature can be 
specified by {\em issueralg} (option {\bf -a}). {\em issueralg} must be a
signature algorithm (AlgType SIG, see INTRO(3) on available algorithms) which fits to the associated 
algorithm of the signature key (e.g., the algorithm md5WithRsa cannot be used with an ElGamal key).
The default for {\em issueralg} is md2WithRsaEncryption.

If option {\bf -F} is provided, the own FCPath is considered as being trusted.
This is useful for optimizing the verification process (see psemaint command {\em verify}),
if the originator of a digital signature and the PSE owner have a common point of trust
within their forward certification paths, or if the PSE owner wishes to verify signatures
generated by himself.

If option {\bf -R} is provided, PEM revocation lists will be considered by the verification process
(see psemaint command {\em verify}).

If option {\bf -X} is given, the PSE object Random
is used to create random numbers.
If option {\bf -Y} is given, the random number generator
will be initialized by keyboard input.

Option {\bf -v} renders the psemaint command verbose, and options {\bf -V} and {\bf -W} provide some further
logging information. 

Option {\bf -z} may be provided for testing purposes; it requests to control 
the malloc/free behaviour.

In case that a smartcard terminal is used, option {\bf -T} requests to perform any public key 
RSA operation in the smartcard terminal, and not with the software in the workstation 
(the latter is the default).

If {\bf -i} {\em inputfile} is given, commands are read from {\em inputfile}, otherwise
commands are read from stdin.

If {\em command} is given, psemaint executes this single command only and then terminates. 
Otherwise, commands can be executed interactively or from {\em inputfile}.make  

If the required PSE does not exist,
and {\em command} is other than {\em create}, psemaint terminates with an appropriate message.
\\ [1em]
{\bf Creation of a PSE}
\\ [1em]
A new PSE can be created by giving the command 
\bc
{\bf psemaint} [ {\bf -p} {\em psename} ] create
\ec
{\bf psemaint} then requires the distinguished name of the owner of the PSE from /dev/tty.
This name is of the type {\em Name} which is the character string representation
of distinguished names described in INTRO(1).
\\ [1em]
After the owner's name was entered, {\bf psemaint} asks for a PIN (which is a password in case
of the PSE) of the entire PSE, the so called PSE-PIN. If a PIN is given here, 
the table of contents (TOC) of the PSE,
and each object for which no object-specific PIN is given on creation time, are encrypted with an 
encryption
key derived from the given PSE-PIN. If no PIN shall be used, CR must be entered. 
\\ [1em]
{\bf Commands} 
\\ [1em]
The following commands are available. They may be abbreviated as indicated by bold letters.
Commands which are marked as CA commands are for CAs only.
\\ [1em]
\begin{tabular}{ll}
{\bf adda}lias     & Add Alias entry to Alias Database           		       \\
{\bf adde}k        & Add ToBeSigned part of $<$certificate $>$ to EKList    		       \\
{\bf addpc}a       & The ToBeSigned part of $<$certificate $>$ is added to PCAList          \\
{\bf addpk}        & Add ToBeSigned part of $<$certificate $>$ to PKList    		       \\
{\bf alg}s         & Show OIDs, names and parameters of algorithm or algorithm type    \\
{\bf alias2}dname  & Search alias and print corresponding DName                        \\
{\bf aliase}s      & Search aliases containing $<$pattern $>$ and print corresponding DName \\
{\bf authno}ne     & Access the X.500 directory without using any authentication       \\
{\bf authsi}mple   & Access the X.500 directory using SIMPLE authentication            \\
{\bf authst}rong   & Access the X.500 directory using STRONG authentication            \\
{\bf cac}rl        & List revocation lists of all CAs known to the CA (CA command) \\
{\bf cas}erialnumbers & Show serialnumbers and dates of issue for given user (CA command) \\
{\bf cau}sers      & List all users of CA (CA command)                                 \\
{\bf certi}fy      & Certify public key contained in Certificate (CA command)          \\
{\bf cert2k}eyinfo & Write object of type KeyInfo with PK from Certificate             \\
{\bf cert2p}kroot  & Write object of type PKRoot with PK from Certificate              \\
{\bf che}ck        & Check PSE consitency                                              \\
{\bf chp}in        & Change PSE-PIN                                                    \\
{\bf cl}ose        & Close PSE or PSE object                                           \\
{\bf cr}eate       & Create PSE or object                                              \\
{\bf dela}lias     & Remove Alias name from Alias Database               	       \\
{\bf delek}        & Remove ToBeSigned part of certificate from EKList    	       \\
{\bf delet}e       & Delete PSE object                                                 \\
{\bf delk}ey       & Delete key with given keyref                                      \\
{\bf delp}ca       & The ToBeSigned with subject name $<$pca $>$ is deleted from PCAList    \\
{\bf delpe}mcrl    & Remove locally stored revocation list of a given issuer           \\
{\bf delpk}        & Remove ToBeSigned part of certificate from PKList    	       \\
{\bf dh1}          & Phase 1 of DH key agreement generates a $<$private\_key $>$            \\
                   & and writes the public key to file $<$public\_key $>$                   \\
{\bf dh2}          & Phase 2 of DH key agreement reads $<$peers\_public\_key $>$ from file  \\
                   & and creates $<$shared\_key $>$ with $<$private\_key $>$                     \\
{\bf dhi}nit       & The PSE object for DH Parameter is generated with                 \\
                   &  modulus length $<$size $>$ and $<$private\_value\_length $>$               \\
{\bf dn}ame2alias  & Search DNames containing given pattern and show corresp. aliases  \\
{\bf end}          & Exit program                                                      \\
{\bf ent}er	   & Enter object into Directory				       \\
{\bf er}ror        & print error stack                                                 \\
{\bf ex}it         & Exit program                                                      \\
{\bf g}enkey       & Generate key and store in object or under keyref                  \\
{\bf h}elpcmd or {\bf ?} & Show helptext for cmd                                       \\
{\bf in}itcrl      & Create an empty CRL (CA command)                               \\
{\bf is}suedcertificate & Show certificate with given serial number (CA command)        \\
{\bf k}eytoc       & Show table of contents (TOC) of key-pool                          \\
{\bf op}en         & Open PSE or PSE object                                            \\
{\bf ow}ner        & Display name of PSE owner                                         \\
{\bf prol}ong      & Prolong the validity of the own CRL (CA command)                \\ 
\end{tabular}
\begin{tabular}{ll}
{\bf prot}otype    & Create self-signed prototype certificate of own public signature key \\
{\bf q}uit         & Exit program                                                      \\
{\bf rea}d         & Read PSE object into file                                         \\
{\bf rem}ove       & Remove object from Directory				       \\
{\bf ren}ame       & Rename PSE object                                                 \\
{\bf res}eterror   & free error stack                                                  \\
{\bf ret}rieve     & Retrieve object from Directory				       \\
{\bf rev}oke       & Revoke one or more certificates (CA command)                      \\
{\bf setp}arm      & Set parameters of given algorithm                                 \\
{\bf sets}erial    & Set SerialNumber to new value (CA command)                        \\
{\bf sh}ow         & Show PSE object or keyref in suitable form                        \\
{\bf sp}lit        & Split Cross Certificate Pair into its components                  \\
{\bf stri}ng2key   & Generate DES key from string and store in object or under keyref  \\
{\bf toc}          & Show table of contents (TOC)                                      \\
{\bf tog}gle       & Toggle verification/encryption tool from SC to SW and vice versa  \\
{\bf tr}ustpath    & Show owner of the PSE and certification path up to the Root-CA    \\
{\bf verb}ose      & Change verbose level                                              \\
{\bf veri}fy       & Verify digital signatures                                         \\
{\bf w}rite        & Write file into PSE object                                        \\
{\bf x}dump        & xdump object or keyref                                            \\
\end{tabular} \\ [1em]
Most commands require parameters which are interactively asked for when they are missing. 
Parameters are partly keyword-parameters of the form {\em keyword$=$value}, partly positional 
parameters. Keywords
may be abbreviated, either. In case of the PSE, object-names are the names of 
the files on the PSE containing the objects.
\\ [1em]
{\bf Description of the Commands} 
\\ [1em]
{\bf algs} {\em algname} or {\em algtype}
\begin{quote}
Shows names, OIDs and parameters of the given algorithm {\em algname} or of all algorithms
of type {\em algtype}. The algorithm type can be one of
\bvtab
\1 ASYM\_ENC \3 - asymmetric encryption algorithms \\
\1 SYM\_ENC  \3 - symmetric encryption algorithms \\
\1 HASH      \3 - hash algorithms \\
\1 SIG       \3 - signature algorithms \\
\1 KEY\_AGREEMENT   3 - Diffie-Hellman key agreement \\
\evtab
See INTRO(3X) for valid algorithm names.
\end{quote}

{\bf addalias} {\em username}
\begin{quote}
Adds an alias entry with distinguished name {\em username} to the USER or SYSTEM alias file.
\end{quote}

{\bf addek} {\em certificate}
\begin{quote}
Extracts the ToBeSigned from PSE object {\em certificate} (which must be of type Certificate)
and adds it to PSE object EKList.
\end{quote}

{\bf addpca} {\em pcacert}
\begin{quote}
Extracts the ToBeSigned from PSE object {\em pcacert} (which must be of type Certificate)
and adds it to PSE object PCAList.
\end{quote}

{\bf addpk} {\em certificate}
\begin{quote}
Extracts the ToBeSigned from PSE object {\em certificate} (which must be of type Certificate)
and adds it to PSE object PKList.
\end{quote}

{\bf alias2dname} {\em alias}
\begin{quote}
Prints the distinguished name corresponding to the given {\em alias}.
\end{quote}

{\bf aliases} {\em pattern}
\begin{quote}
Searches the alias files \$HOME/.af-alias and \$TOP/secude/.af-alias for alias names
containing pattern and prints one line per hit
\bc
{\em aliasname} - - -$>$ $<${\em distinguished name}$>$
\ec
\end{quote}

{\bf authnone}
\begin{quote}
This command is available only if a X.500 Directory is being used.
It determines that the X.500 Directory shall henceforth be accessed without using any authentication.
(Simple authentication is the default level of authentication.) 
\end{quote}

{\bf authsimple}
\begin{quote}
This command is available only if a X.500 Directory is being used.
It determines that the X.500 Directory shall henceforth be accessed using simple authentication.
(Simple authentication is the default level of authentication.) 
\end{quote}

{\bf authstrong}
\begin{quote}
This command is available only if a X.500 Directory is being used.
It determines that the X.500 Directory shall henceforth be accessed using strong authentication. 
\end{quote}

{\bf cacrl}
\begin{quote}
This command is for CA PSEs only.
It lists the revocation lists of all CAs known to the CA. Each of the revocation lists is optionally
accompanied by its issuer's certification path. 
The information is retrieved from the CA database contained in the CA directory (see also {\em 
cacreate(1)}).
\end{quote}

{\bf caserialnumbers} {\em name}
\begin{quote}
This command is for CA PSEs only.
It lists serial number and date-of-issue of all certificates which the PSE-owning CA has 
issued for the user identified by {\em name} where {\em name} is a distinguished name in 
printable representation (Name form) or an alias name. If {\em name} is of Name form
(containing '=' characters) it must be quoted. The information is retrieved from the 
CA database contained in the CA
directory (see also {\em cacreate(1)}).
\end{quote}

{\bf causers}
\begin{quote}
This command is for CA PSEs only.
It lists all users for whom the PSE-owning CA has issued a certificate. Each user is represented 
by its distinguished name in printable representation. The information is retrieved from the CA database
contained in the CA directory (see also {\em cacreate(1)}).
\end{quote}

{\bf certify} {\em certificate}
\begin{quote}
This command is for CA PSEs only.
It certifies the public key contained in {\em certificate} (which must be of type Certificate).
This command is useful for a CA which requests to cross-certify another CA's public key.
\end{quote}

{\bf cert2keyinfo} {\em certificate}~ {\em keyinfo}
\begin{quote}
Extracts the KeyInfo containing the public key from PSE object {\em certificate} 
(which must be of type Certificate), and writes it to {\em keyinfo}.
\end{quote}

{\bf cert2pkroot} {\em certificate}~ {\em pkroot}
\begin{quote}
Extracts the KeyInfo containing the public key from PSE object {\em certificate} 
(which must be of type Certificate), builds a new object of type PKRoot (containing
only the {\em newkey} member) and writes it to {\em pkroot}.
\end{quote}

{\bf check}
\begin{quote}
Check the consistency of the PSE. The certification pathes SignCert -- $>$ FCPath -- $>$ PKRoot
and EncCert -- $>$ FCPath -- $>$ PKRoot are verified. In addition it is checked whether
the public RSA key in SignCert fits to the secret RSA key in SignSK, and whether the
public RSA key in EncCert fits to the secret RSA key in DecSKnew.
\end{quote}

{\bf chpin}
\begin{quote}
Changes the PSE-PIN. If only CR is entered as PIN, the PIN-protection is removed.
\end{quote}

{\bf close} {\em psename}
\begin{quote}
Close the previously opened PSE.
\end{quote}

{\bf create} {\em object}
\begin{quote}
Create the PSE object {\em object}
of the PSE which is currently open. Asks for PIN's.
\end{quote}

{\bf delalias} {\em alias}
\begin{quote}
Removes the given {\em alias} from either the USER alias file or the SYSTEM alias file. If the
last alias name of an alias entry is removed, the whole alias entry is deleted from the alias file.
\end{quote}

{\bf delek} {\em subject}~ {\em issuer}~ {\em serialnumber}
\begin{quote}
Removes ToBeSigned entries from PSE object EKList.
{\em delek} deletes either all entries of the given {\em subject} (owner) or exactly the one entry identified by 
the
given {\em issuer} and {\em serialnumber} combination. {\em Owner} and {\em Issuer} are distinguished
names in printable form or alias names.
\end{quote}

{\bf delete} {\em object}
\begin{quote}
Delete the the PSE object {\em object}. 
\end{quote}

{\bf delkey} {\em keyref} 
\begin{quote}
Removes the key {\em keyref} from the key-pool.
\end{quote}

{\bf delpca} {\em pcaname} 
\begin{quote}
Removes the PCA with Name {\em pcaname} from the PSE object PCAList.
\end{quote}

{\bf delcrl} {\em issuer}
\begin{quote}
Removes revocation list from PSE object CrlSet.
{\em delcrl} deletes the revocation list (PEM syntax) issued by {\em issuer} from the set of 
locally stored revocation lists. {\em Issuer} is a distinguished name in printable form or an alias name.
\end{quote}

{\bf delpk} {\em subject}~ {\em issuer}~ {\em serialnumber}
\begin{quote}
Removes ToBeSigned entries from PSE object PKList.
{\em delpk} deletes either all entries of the given {\em subject} (owner) or exactly the one entry identified by 
the
given {\em issuer} and {\em serialnumber} combination. {\em Owner} and {\em Issuer} are distinguished
names in printable form or alias names.
\end{quote}

{\bf dh1} temporary {\em size}~ {\em private\_value\_length}~ {\em own\_private\_key}~ {\em own\_public\_key}

{\bf dh1} peer {\em peers\_public\_y}~ {\em own\_private\_key}~ {\em own\_public\_key}

{\bf dh1} default {\em with\_parm}~ {\em own\_private\_key}~ {\em own\_public\_key}
\begin{quote}
Reads the DH Parameter from the default PSE object DHparam, from the file {\em peers\_public\_y} or generates a temporary one with
{\em size} and {\em private\_value\_length}.
Generates a private value $x$ for this DH Parameter.
Stores this value as a KeyInfo in the PSE object referenced by {\em private\_key}. Calculates the corresponding
public value $y$ and writes it as an KeyInfo to the file  {\em public\_key}.
If {\em with\_parm} is {\bf common} (in case of default), for both keys
the AlgId dhWithCommonModulus is used instead of dhKeyAgreement, which means that the DH Parameter
is not part of these keys and the peer has to know it.
\end{quote}

{\bf dh2} {\em private\_key}~ {\em peers\_public\_key}~ {\em agreed\_key}
\begin{quote}
Reads the DH Parameter from PSE object DHparam and reads the private value $x$ from the
PSE object referenced by {\em private\_key}. Gets the public value $y$ of the peer from the file
{\em peers\_public\_key}. Calcutates the agreed value $z$ and stores it as a BitString in the
PSE object {\em agreed\_key}. This BitString has as many bits as the prime modulus $p$.
How to use it e.g. as a DES key is not yet specified.
\end{quote}

{\bf dhinit} {\em size}~ {\em private\_value\_length}
\begin{quote}
Generates a DH Parameter. The prime modulus $p$ has the length $size/8$ octets,
so {\em size} should be dividable by 8.
Optionally the length of the private values $x$ that will be
generated to this Parameter can fixed to {\em private\_value\_length} bits.
The DH Parameter is stored as the PSE object DHparam.
\end{quote}

{\bf dname2alias} {\em pattern}
\begin{quote}
Searches the alias files \$HOME/.af-alias and \$TOP/secude/.af-alias for distinguished names
containing pattern and prints all corresponding alias names (one line per alias)
\bc
$<${\em distinguished name}$>$ - - -$>$  {\em aliasname}
\ec
\end{quote}

{\bf enter} {\em attrtype}~ {\em keytype} 
\begin{quote}
Enter information of type {\em attrtype} into Directory. \\
The Directory may either be a X.500 Directory or the local substitute provided by 
SecuDE ({\em .af-db}). (see also DirectoryType(5)).
{\em attrtype} is {\em {\bf U}serCertificate} or {\em {\bf Ce}rtificate} by default, depending on whether
or not a X.500 Directory shall be accessed. Further possible values for {\em attrtype} are
{\em {\bf CA}Certificate}, {\em {\bf Cr}ossCertificatePair}, and {\em {\bf P}emCertificateRevocationList}.
{\em keytype} may have one of the values {\bf E} or {\bf S}.
 
If the X.500 Directory is accessed and {\em attrtype} is {\em UserCertificate} ({\em CACertificate}), a certificate stored in the PSE 
is added to the {\em userCertificate} ({\em cACertificate}) attribute in
the caller's X.500 directory entry.
If the .af-db Directory is accessed, however, the certificate is stored in the file {\em SignCert} or 
{\em EncCert}, depending on {\em keytype}, in the .af-db subdirectory identified by the distinguished name 
of the certificate owner. (If the caller's PSE is restricted to carry one RSA keypair only, the
certificate is stored as file {\em Cert}, instead.)
The PSE object containing the certificate is asked for interactively. \\
The following objects (CrossCertificatePairs and Revocation Lists) may be entered into 
the Directory by CAs only: \\
If the X.500 Directory is accessed and {\em attrtype} is {\em CrossCertificatePair}, 
a cross certificate pair 
is added to the {\em crossCertificatePair} attribute in the calling CA's X.500 directory entry. 
If the local .af-db Directory is accessed, however, the cross certificate pair is added to the 
file {\em CrossCSet}
in the .af-db subdirectory identified by the CA's distinguished name.
The PSE objects (which must be of type 
Certificate) containing the components of the cross certificate pair ({\em forward} and {\em reverse}
certificate) are asked for interactively; at least one of the two components (forward 
or reverse certificate) of the cross certificate pair must be indicated. \\
If the X.500 Directory is accessed and {\em attrtype} is {\em PemCertificateRevocationList}, 
the CA's PEM revocation list
is retrieved from the CA's {\em crlwithcerts} database and stored in the calling CA's X.500 
directory entry.
If the local .af-db Directory is accessed, however, the CA's revocation list is stored in
the file {\em CRL} in the .af-db subdirectory identified by the CA's distinguished name. 
\end{quote}

{\bf genkey} {\em algname}~ {\em object1/keyref1} {\em object2/keyref2} 
\begin{quote}
Generates a symmetric key or asymmetric key pair and stores it in {\em object1/object2} or 
under {\em keyref1/keyref2}. 
If a PSE object is indicated, the object must not exist, otherwise psemaint will ask for overwriting. 
Parameters associated to the indicated algorithm (i.e. keysize for RSA) may
be changed before using {\bf setparm}. 
 
If {\em algname} indicates a DES algorithm, a 64-bit long BitString is generated as DES key. The
known bad DES keys are avoided. If {\em algname} indicates a DES algorithm in 
Encryption(k1)-Decryption(k2)-Encryption(k1) mode, a 128-bit long BitString is generated with
the first 64 bit comprising k1 and the second 64 bit comprising k2.
 
If {\em algname} indicates an RSA algorithm, two BitStrings are generated, one being
the ASN.1 code of the public key, the other being the ASN.1 code of the secret key.
The public key ist stored in {\em object1} or {\em keyref1}, the secret key is stored in
{\em object2} or {\em keyref2}.
See INTRO(3X) for valid algorithm names.
\end{quote}

{\bf help} or {\bf ?} {\em cmd}
\begin{quote}
Shows helptext for the command {\em cmd}.
\end{quote}

{\bf initcrl} 
\begin{quote}
This command is for CA PSEs only.
It creates an empty PEM revocation list which is stored in the CA database and entered into the Directory. \\
The Directory may either be a X.500 directory or the local substitute provided by 
SecuDE ({\em .af-db}). (The decision whether a X.500 directory or the local substitute 
is used is done at compile time through the variable DX500 in the file CONFIG.make
at \$TOP/secude/config.) \\
The {\em lastUpdate} component of the empty revocation list is set to the current time, whereas the value 
of the {\em nextUpdate} component may individually be specified by the calling CA. The empty
revocation list is written to the {\em crlwithcerts} database in the CA's directory, where an already
existing revocation list of the calling CA will be replaced; in addition, 
it is entered into the directory (X.500 or local substitute, respectively), where it replaces
a revocation list that has already been stored there by the calling CA. 
Furthermore, the empty revocation list is added to the PSE object CrlSet in the CA's PSE, where an
already existing revocation list of this CA will be replaced.
\end{quote}

{\bf issuedcertificate} {\em serial}
\begin{quote}
This command is for CA PSEs only.
It shows the certificate with serial number {\em serial} issued by the PSE-owning CA (see also
{\em caserialnumbers} to obtain a list of all issued serial numbers). The
information is retrieved from the CA database contained in the CA directory (see also {\em cacreate(1)}).
\end{quote}

{\bf keytoc}
\begin{quote}
Show table of contents (toc) of the key-pool. The (user-specific) key-pool keeps all
keys which are addressable via keyref's (see INTRO(3X) for further explanation of the key-pool).
\end{quote}

{\bf open} {\em psename}
\begin{quote}
Open a new PSE. The PSE which is currently open is closed. After the invocation 
of psemaint either the PSE .pse or the PSE
given with option {\bf -p} {\em psename} is open. All succeeding commands refer to this
PSE.
\end{quote}

{\bf owner}
\begin{quote}
Display distinguished name of PSE owner.
\end{quote}

{\bf prototype}
\begin{quote}
Create a self-signed prototype-certificate containing the own public signature key.
\end{quote}

{\bf quit} or {\bf exit} or {\bf end} 
\begin{quote}
Exit program.
\end{quote}

{\bf read} {\em object}~ {\em destination} 
\begin{quote}
Read the content of the given PSE object {\em object}
and write the objectValue into file {\em destination}. Includes all PIN-handling. 
{\em sec\_read\_PSE()} is used to get the objectValue part.
\end{quote}

{\bf remove} {\em attrtype}~ {\em keytype} cert$=${\em serial},{\em issuer}~ for$=${\em serial},{\em 
issuer}~ rev$=${\em serial},{\em issuer}
\begin{quote}
Remove information of type {\em attrtype} from Directory. \\
The Directory may either be a X.500 Directory or the local substitute provided by 
SecuDE ({\em .af-db}). (see also DirectoryType(5)).
{\em attrtype} is {\em {\bf U}serCertificate} or {\em {\bf Ce}rtificate} by default, depending on whether
or not a X.500 Directory is accessed. Further possible values for {\em attrtype} are
{\em {\bf CA}Certificate} and {\em {\bf Cr}ossCertificatePair}.
{\em keytype} may have one of the values {\bf E} or {\bf S}.
 
If the X.500 Directory is accessed and {\em attrtype} is {\em UserCertificate} ({\em CACertificate}), the certificate identified by 
{\em serial} and {\em issuer} is removed from the {\em userCertificate} ({\em cACertificate}) attribute in the caller's X.500 directory entry. \\
If the local substitute is used, however, the file {\em SignCert} or {\em EncCert} 
(depending on {\em keytype}) 
in the .af-db subdirectory identified by the caller's distinguished name is deleted.
(If the caller's PSE is restricted to hold one RSA keypair only, the file {\em Cert} is
removed, instead.)
If {\em attrtype} is {\em CrossCertificatePair}, a cross certificate pair whose components ({\em forward}
and {\em reverse} certificate) are identified by their serial number {\em serial} and issuer
{\em issuer}, is removed from the {\em crossCertificatePair} attribute in the calling CA's 
X.500 directory entry. (A cross certificate pair may be removed from the Directory by CAs only.)
If the local substitute is used, however, the cross certificate pair is removed from the file {\em CrossCSet}
in the .af-db subdirectory defined by the CA's distinguished name.
At least one of the two components ({\em forward} or {\em reverse} certificate) is required for identifying 
the cross certificate pair that is to be removed from the Directory.
\end{quote}

{\bf rename} {\em object}~ {\em newname}
\begin{quote}
Rename the given PSE object {\em object} with {\em newname}. 
\end{quote}

{\bf retrieve} {\em dirname}~ {\em attrtype}~ {\em keytype} {\em update}
\begin{quote}
Retrieve information of type {\em attrtype} from Directory. \\
The Directory may either be a X.500 Directory or the local substitute provided by 
SecuDE ({\em .af-db}). (see also DirectoryType(5)).
{\em attrtype} is {\em {\bf U}serCertificate} or {\em {\bf Ce}rtificate} by default, depending on whether
or not a X.500 Directory shall be accessed. Further possible values for {\em attrtype} are
{\em {\bf CA}Certificate}, {\em {\bf Cr}ossCertificatePair}, and {\em {\bf P}emCertificateRevocationList}.
{\em keytype} may have one of the values {\bf E} or {\bf S}.
 
If the X.500 Directory is accessed and {\em attrtype} is {\em UserCertificate} ({\em CACertificate}), 
the set of certificates 
stored in the X.500 directory entry identified by {\em dirname} is returned; the set of 
certificates comprises the value(s) of the {\em userCertificate} ({\em cACertificate}) attribute. 
If the local .af-db Directory is accessed, however, the returned set of certificates either comprises
the certificate stored in file {\em Cert} (if present), or the certificate stored in file 
{\em SignCert} or {\em EncCert} (depending on {\em keytype}), which is located in 
the .af-db subdirectory identified by {\em dirname}.
If {\em update} is set to TRUE, one certificate among the returned set of certificates may
be selected by the user and stored in its PSE.\\
If the X.500 Directory is accessed and {\em attrtype} is {\em CrossCertificatePair}, 
the set of cross certificate 
pairs stored in the directory entry identified by {\em dirname} is returned; the set of 
cross certificate pairs comprises the value(s) of the {\em crossCertificatePair} attribute. 
If the local .af-db Directory is accessed, however, the set of cross certificate pairs is retrieved
from the file {\em CrossCSet} in the .af-db subdirectory identified by {\em dirname}.
If {\em update} is set to TRUE, the cross certificate pairs are added as entries 
to the PSE object CrossCSet; if that PSE object does not exist, however, it is created. \\
If the X.500 Directory is accessed and {\em attrtype} is {\em PemCertificateRevocationList}, 
the revocation list stored 
in the directory entry identified by {\em dirname} is returned. 
If the local .af-db Directory is accessed, however, the revocation list is retrieved from the
file {\em CRL} in the .af-db subdirectory identified by {\em dirname}.
The format of the revocation list is that defined by PEM (see RFC 1422, Appendix A). 
If {\em update} is set to TRUE, the revocation list is added as entry to the 
PSE object CrlSet; if that PSE object does not exist, however, it is created.
\end{quote}

{\bf revoke} 
\begin{quote}
This command is for CA PSEs only.
It revokes one or more certificates and enters the resulting revocation list into the 
Directory. \\
The directory may either be a X.500 directory or the local substitute provided by 
SecuDE ({\em .af-db}). (The decision whether a X.500 directory or the local substitute 
is used is done at compile time through the variable DX500 in the file CONFIG.make
at \$TOP/secude/config.) \\
The certificates which are to be revoked are identified by their serial numbers which are to be
entered interactively (see also {\em caserialnumbers} and {\em issuedcertificate}). 
Each serial number is submitted to the following check: \\
1. It is checked whether the given serial number is contained in the CA database in the
CA directory of the PSE-owning CA; this check is performed in order to find out whether the 
PSE-owning CA has issued a certificate with the given serial number. \\
2. It is checked whether the given serial number is already contained in the most current version
of the CA's revocation list. The CA's revocation list is retrieved from the {\em crl} file
in the directory of the PSE-owning CA; its format is that specified by PEM. If there is no 
such file in the CA's directory, an empty revocation list will be created. \\
This check is performed in order to find out whether the certificate with the given serial number 
has already been revoked by the PSE-owning CA. \\
Only in case of at least one certificate having been revoked and added to the revocation list, the 
{\em lastUpdate} component of the revocation list is set to the current time, whereas the value 
of the {\em nextUpdate} component may individually be specified by the revoking CA. The extended
revocation list is written back to the {\em crlwithcerts} database in the CA's directory; in addition, 
it is entered into the directory (X.500 or local substitute, respectively). 
Furthermore, the extended revocation list is added to the PSE object CrlSet in the CA's PSE.
\end{quote}

{\bf setparm} {\em algname}
\begin{quote}
Changes parameters of the given algorithm {\em algname}. See INTRO(3X) for valid algorithm names.
\end{quote}

{\bf show} {\em object/keyref}~ [DER] [KEYBITS] [ALL]
\begin{quote}
Prints {\em object} or {\em keyref} in suitable form to stdout. If the given object contains
keys, only the algorithm identifier are showed. The keyword DER can be used to additionally
show the DER-Code of the BitString of the key, while the keyword KEYBITS causes
additionally showing the two INTEGERS if RSA-keys are involved. Keyword ALL means both of them.
\end{quote}


{\bf split} for$=${\em serial},{\em issuer}~ rev$=${\em serial},{\em issuer}
\begin{quote}
Split cross certificate pair into its components ({\em forward} and {\em reverse} certificate)
and store them on PSE. The names of the PSE objects where the {\em forward} and {\em reverse} 
certificate are to be stored are asked for interactively. \\
The cross certificate pair to be split must be contained in the PSE object CrossCSet;
its components ({\em forward} and {\em reverse} certificate) are identified by their serial 
number {\em serial} and issuer {\em issuer}. At least one of the two components 
({\em forward} or {\em reverse} certificate) must be present. \\
No update is done on the PSE object CrossCSet.
\end{quote}

{\bf string2key} {\em string}~ {\em object/keyref} 
\begin{quote}
Generates a DES key from the given string and stores it in {\em object} or under {\em keyref}. The
key is generated using a one-way hash function.
\end{quote}

{\bf toc}
\begin{quote}
Show table of contents (toc). 
\end{quote}

{\bf verify} certificate$=${\em certobj}~ fcpath$=${\em fcpathobj}~ pkroot$=${\em pkrootobj}~ 
\begin{quote}
Verifies the chain Certificate -- FCPath -- PKRoot. {\em certobj}, {\em fcpathobj}, and {\em pkrootobj}
are PSE objects containing a Certificate, a FCPath, and a PKRoot, respectively. The default PSE objects
are SignCert, FCPath and PKRoot, if parameters are omitted. If only 'fcpath$=$' is given,
no FCPath is used for verification. \\
The global variable {\em af\_chk\_crl} determines whether a directory (either X.500 or .af-db, see also 
DirectoryType(5)) is to be accessed
in order to find out if any of the certificates encountered in the verification process is 
contained in the revocation list of its issuer.
The global variable {af\_search\_directory} determines whether a directory (either X.500 or .af-db, see also 
DirectoryType(5)) is to be accessed
in order to establish or complete the certification path required for the verification of {\em certobj}. 
\end{quote}

{\bf write} {\em object}~ {\em source}~ {\em type}
\begin{quote}
Read file {\em source} and write it as objectValue the given PSE object {\em object} (using
{\em sec\_write\_PSE()}). The objectType is taken from {\em type}.
Includes all PIN-handling. If the PSE object does not exist, it is created.
If {\em type} is omitted, the content of {\em source} is written to PSE object
{\em object} using {\em sec\_write()}.
\end{quote}

{\bf xdump} {\em object/keyref} 
\begin{quote}
Prints {\em object} or {\em keyref} in a suitable form to stdout.
\end{quote}

\nm{1}{pem}{Privacy Enhanced Mail (PEM) according to RFC 1421 - 1424}
\label{pem}
\hl{Synopsis}
{\bf pem} [ {\bf scan $|$ mic-clear $|$ mic-only $|$ encrypted $|$ crl $|$ crl-rr $|$ certify } ]
\\ \hspace*{0.95cm} 
[ {\bf -i} {\em inputfile} ] [ {\bf -o} {\em outputfile} ] [ {\bf -p} {\em psename} ] [ {\bf -c} {\em cadir} ] [ {\bf 
-m/M}  {\em depth} ] 
\\ \hspace*{0.95cm}
[ {\bf -Z} {\em device} ] [ {\bf -I} ] [ {\bf -u} {\em update} ] [ {\bf -r} {\em name1} {\em name2} ... ] [ {\bf -y} {\em name1} {\em name2} ... ] [ {\bf 
-DFRCnhvVWXYNOTz} ]
\\ \hspace*{0.95cm}
[ {\bf -H} {\em md-alg} ] [ {\bf -S} {\em micenc-alg} ] [ {\bf -E} {\em msgenc-alg} ] [ 
{\bf -K} {\em dekenc-alg} ] [ {\bf -d} {\em dsaname} ]
\\ \hspace*{0.95cm} [ {\bf -A} {\em authlevel} ] [ {\bf -t} {\em dsaptailor} ]

\hl{Description}
This PEM filter transforms any input text file into a PEM formatted output file 
and vice versa. It can be used both by ordinary users or by certification
authorities. It realizes all formats 
and procedures defined in the Internet Specifications RFC 1421-1424 
except that it only supports asymmetric key management. It is
possible to securely cache other's certificates and CRLs as this 
is part of the general SecuDE functionality. In particuar, it is possible
with {\em pem} to
\bi
   \m enhance messages, i.e. create MIC-CLEAR, MIC-ONLY and
      ENCRYPTED message formats,
   \m de-enhance messages, i.e. scan MIC-CLEAR, MIC-ONLY and
      ENCRYPTED message formats, and perform the necessary
      validation steps (including CRL checks),
   \m create Certification Request, Certification Reply, CRL Storage
      Request and CRL Storage Reply message formats,
   \m scan Certification Request, Certification Reply, CRL Storage
      Request and CRL Storage Reply message formats, and perform
      appropriate actions, for instance updating the PSE or
      the CA-database,
   \m support a CRL storage service.
\ei

Pem requires to use a PSE which holds one keypair for signature and encryption purposes since
MIC encryption and DEK encryption is to be performed with the same keypair. 

For the purpose of
MIC encryption, the own secret key contained in PSE-object SKnew is used. If a scanned
privacy enhancement contains an Originator-Certificate header field, MIC decryption
is done with the public key contained there. If a scanned privacy enhancement contains an 
Originator-ID-Asymmetric header field, the given issuer/serialnumber must be found in
the PSE-object PKList which is the list of the user's trusted keys. MIC decryption
is then done with the public key found in the corresponding entry of PKList.

The per-recipient DEK encryption is done with the public key of the recipient which is retrieved
from the PSE-object PKList. If it is not found there, the local afdb Directory is consulted.
If it is not found there either, and if the option -D is given, it is retrieved from
the X.500 Directory. DEK decryption is done with the own secret key contained in PSE-object
SKnew. 

A keyword which must be the first parameter of {\em pem} controls the PEM transformation mode.
Only the {\bf scan} mode transforms from the PEM format to a cleartext format. All other
modes create PEM formats.  

{\large\bf Scan}

Scan is the default mode. The keyword scan can be omitted. In this mode the textfile {\em inputfile}
(or stdin if {\em inputfile} is omitted) is scanned for privacy enhancements. The term input is
subsequently used to denote the data coming either from {\em inputfile} or from stdin. The input
may contain a single privacy enhancement, enclosed in PEM boundary lines, or nested privacy
enhancements, the latter, however, only with Proc-Types MIC-CLEAR, MIC-ONLY or ENCRYPTED.
Multiple consecutive privacy enhancements are not possible. Depending on the type of the scanned
privacy enhancements and the given {\bf -u} {\em update} parameter, {\em pem scan} does one or
more of the following actions:

\bi
\m write a notification about the MIC verification and other status information to stderr,
\m write the PEM transformation result to {\em outputfile} or stdout, if {\em outputfile} is omitted,
   The term output is subsequently used to denote the data which are written either to
   {\em outputfile} or to stdout.
\m update the user's PSE,
\m update the CA-database.
\ei

A notification about the {\bf MIC verification} is given in any case. The MIC validation will be done
according to the procedure described in af\_verify(3). 

If option {\bf -D} is given, missing certificates will automatically be retrieved from the Directory.
The Directory may either be a X.500 directory or the local substitute provided by 
SecuDE ({\em .af-db}) (see also DirectoryType(5)).

If the X.500 Directory is accessed, parameter {\bf -d} {\em dsaname} allows to specify the 
name of the DSA (Directory System Agent) to be contacted for this purpose. 
{\em dsaname} is a name which must be contained in the 
{\em dsaptailor} file of the system as {\em dsa\_address} parameter. If {\bf -d} {\em dsaname} is
not given, the default DSA, which is the first {\em dsa\_address} parameter in the {\em dsaptailor} 
file, is accessed.
Parameter {\em authlevel} (option {\bf -A}) may be provided to determine the level of authentication used when binding. If {\bf -A} {\em authlevel} is omitted, the requested 
level of authentication is taken from the environment variable AUTHLEVEL, and is set to {\em no 
authentication}, if no such variable exists.
In case of simple authentication being used, the caller's distinguished name and
X.500 password are retrieved from the PSE objects Name and QuipuPWD, respectively.

If option {\bf -F} is provided, the own FCPath is considered as being trusted.
This is useful for optimizing the verification process,
if the originator of a PEM message and its recipient have a common point of trust
within their forward certification paths, or if the caller wishes to evaluate PEM messages
generated by himself.

If option {\bf -R} is given, a check of certificate revocation lists is required for all certificates
which are in the certification path. In case that option {\bf -D} is additionally provided, missing CRLs will
automatically be retrieved from the Directory. In addition, whenever a CRL is consulted, the 
nextUpdate field of the CRL is compared with the current time and date, and in case of nextUpdate
being expired, the actual CRL will be retrieved from the Directory. 

If option {\bf -N}
is given, also non standard algorithms (i.e. algorithms which are not defined by RFC 1423,
but which are suitable for PEM purposes and contained in the SecuDE algorithm suite, for instance
NIST-SHA and NIST-DSA) are accepted. Otherwise, only RFC 1423 defined algorithms are accepted.

If option {\em -O} (capital letter O) is given, RFC 1422 defined DName subordination rules
are not applied during MIC verification. Otherwise only RFC 1422 conformant certification
structures are accepted. 

{\bf Status information} may consists of the following notifications:
\bi
\m a Proc-Type ENCRYPTED has been scanned, but that no appropriate 
   issuer/serialnumber was found in the Recipient-ID-Asymmetric header fields,
\m a Proc-Type ENCRYPTED has been scanned and an appropriate 
   issuer/serialnumber was found, but that the decryption of the key from the corresponding 
   Key-Info field failed,
\m a Proc-Type MIC-CLEAR or MIC-ONLY has been scanned and the message appears to be a
   certification request (which is the case if it contains an Originator-Certificate
   header field with the issuer DName and subject DName being identical). 
   issuer/serialnumber was found, but that the decryption of the key from the corresponding 
   Key-Info field failed,
\m a Proc-Type CRL has been scanned, containing the indicated issuers,
\m a Proc-Type CRL-RETRIEVAL-REQUEST has been scanned, containing the indicated issuers,
\m Detected syntax errors.
\ei

The {\bf PEM transformation result} which is written to the output consists of the
following:
\bi
\m If no PEM boundary lines are found, the input is copied to the output.
\m If the scanned Proc-Type is MIC-CLEAR, MIC-ONLY or ENCRYPTED, all input lines which are 
   outside the PEM boundary lines, and the de-enhanced PEM body are written to the output. 
   This is done even in case that the MIC verification and validation failed. If
   syntax errors have been encountered, or an ENCRYPTED message could not be decrypted,
   nothing is written to the output.
\m If the scanned Proc-Type is CRL-RETRIEVAL-REQUEST, a PEM message of Proc-Type CRL is written
   to the output if the request can at least partly be satisfied with the locally stored 
   information. In case that {\em pem scan} was called by a normal user (indicated through the
   absence of parameter {\bf -c} {\em cadir}), the PSE-object CrlSet is consulted to
   satisfy the request. In case that {\em pem} scan was called by a CA (indicated through the
   presence of parameter {\bf -c} {\em cadir}), the CA-database is consulted to
   satisfy the request. If none of the requested CRLs is available, nothing is written
   to the output.
\ei

The user's (or CA's) {\bf PSE} can be potentially {\bf update}d in the following situations:

\bi
\m An Originator-Certificate has been scanned and successfully validated which is not
   contained in the user's list of trusted keys, PKList. The ToBeSigned part of such
   a certificate can be added to PKList in order to consider it being trusted in the future.
   Whenever {\em pem} adds a new entry to PKList, it checks whether the distinguished
   name of the subject has already an alias name. If not, it requests the user to
   supply at least one alias name.  
\m An Originator-Certificate has been scanned and successfully validated which has the
   own distinguished name as subject name. This is supposed to be a certification reply
   if it is different from what is currently stored in the PSE-object Cert, and if it
   is part of a MIC-CLEAR or MIC-ONLY message. This certificate can be stored in the
   PSE-object Cert as own public key certificate.
\m An Issuer-Certificate has been scanned and successfully validated which is not
   contained in the user's list of trusted keys, PKList. If -I is set, the ToBeSigned part of such
   a certificate can be added to PKList in order to consider it being trusted in the future.
   Whenever {\em pem} adds a new entry to PKList, it checks whether the distinguished
   name of the subject has already an alias name. If not, it requests the user to
   supply at least one alias name.  
\m A Proc-Type CRL has been scanned which can either be a CRL-storage request or be a
   CRL-retrieval reply. The CRLs contained in that PEM message can be stored in the CA-database,
   if {\em pem scan} was called by a CA, or can be added to the PSE-object CrlSet which is the
   user's CRL-cache, or it can be stored in both.
\ei

The {\em CA-database} can be potentially {\bf update}d when a Proc-Type CRL has been scanned
which is supposed to be a CRL-storage request.

Parameter {\bf -u} {\em update} controls whether or not {\em pem} scan does these updates automatically. 
{\em update} may have five values:
\bvtab
{\bf ask:}  \2 pem asks interactively via /dev/tty whether it shall do the update or not. \\
{\bf no:}   \2 pem doesn't do any update of the PSE or CA-database. \\
{\bf yes:}  \2 pem automatically updates both PSE and/or CA-database. \\
{\bf cadb:} \2 pem automatically updates the CA-database. \\
{\bf pse:}  \2 pem automatically updates the PSE. \\
\evtab

With parameter {\bf -Z} the device for inputs can be changed from /dev/tty to {\em device}.

If option {\bf -X} is given, the PSE object Random
is used to create random numbers.
If option {\bf -Y} is given, the random number generator
will be initialized by keyboard input.

{\large\bf Mic-clear}

This mode transforms the input to a MIC-CLEAR PEM message. If option {\bf -C} is given, the created
PEM message contains an Originator-Certificate header field and all Issuer-Certificate header fields
up to the root CA. If option {\bf -C} is omitted, it contains an Originator-ID-Asymmetric header
field.

Parameter {\bf -H} {\em md-alg} and {\bf -S} {\em micenc-alg}
allow to use any of the algorithms available in SecuDE and suitable
for the intended purpose. {\em md-alg} and {\em micenc-alg}
are names of algorithms as they are listed in INTRO(3).

{\bf -H} {\em md-alg}: Algorithm which is used to compute the MIC. RSA-MD5 is the default.
To be conformant to RFC 1423, only RSA-MD2 can be choosen as an alternative. If RFC 1423
conformance is not a requirement, any algorithm can be choosen which has AlgType HASH.

{\bf -S} {\em micenc-alg}: Algorithm which is used to encrypt the MIC. RSA is the default.
To be conformant to RFC 1423, no other algorithm can be choosen. If RFC 1423
conformance is not a requirement, any algorithm can be choosen which has AlgType ASYM\_ENC.

{\large\bf Mic-only}

This mode transforms the input to a MIC-ONLY PEM message. All options described under mic-clear
above apply to mic-only, too.

{\large\bf Encrypted}

This mode transforms the input to an ENCRYPTED PEM message. Parameter {\bf -r} {\em name1 name2 
...}
allows to specify the intended recipients. {\em name1 name2 ...} may be alias names or distinguished
names in their printable representation. In the latter case, the names must be quoted since they
contain shell metacharacters.

In addition to the parameters {\bf -C}, {\bf -H} {\em md-alg} and {\bf -S} {\em micenc-alg} 
described under mic-clear, which also apply in the encrypted case, {\bf -E} {\em msgenc-alg} and
{\bf -K} {\em dekenc-alg} allow to use any of the algorithms available in SecuDE and suitable 
for purpose of hybrid message encryption. {\em msgenc-alg} and {\em dekenc-alg}
are names of algorithms as they are listed in INTRO(3).

{\bf -E} {\em msgenc-alg}: Algorithm which is used to encrypt the message body.
DES-CBC is the default.
To be conformant to RFC 1423, no other algorithm can be choosen. If RFC 1423
conformance is not a requirement, any algorithm can be choosen which has AlgType SYM\_ENC.

{\bf -K} {\em dekenc-alg}: Algorithm which is used to encrypt the DEK (data encryption key).
RSA is the default.
To be conformant to RFC 1423, no other algorithm can be choosen. If RFC 1423
conformance is not a requirement, any algorithm can be choosen which has AlgType ASYM\_ENC.

{\large\bf Crl}

This mode creates a CRL PEM message. Parameter {\bf -y} {\em issuer1 issuer2 ...}
allows to specify a list of issuers. {\em issuer1 issuer2 ...} may be alias names or distinguished
names in their printable representation. In the latter case, the names must be quoted since they
contain shell metacharacters. The given issuer's CRLs are searched in the PSE-object CrlSet, or they
are retrieved from the CA-database, depending on whether {\em pem crl} was called from a normal
user (indicated through the absence of parameter {\bf -c} {\em cadir}) or from a CA (indicated through 
the presence of parameter {\bf -c} {\em cadir}).

{\large\bf Crl-rr}

This mode creates a CRL-RETRIEVAL-REQUEST PEM message. Parameter {\bf -y} {\em issuer1 issuer2 
...}
allows to specify a list of issuers. {\em issuer1 issuer2 ...} may be alias names or distinguished
names in their printable representation. In the latter case, the names must be quoted since they
contain shell metacharacters.

{\large\bf Certify}

This mode is used by a CA in order to satisfy an incoming certification request. It scans the input
which is supposed to be a certification request and writes the certification reply to the output.
The output is identical to the input except that the Originator-Certificate from the input which
is a requestor generated prototype certificate is replaced by a certificate signed by the CA.
{\em pem certify} performs all necessary validation and checking. No output is written if the
Originator-Certificate can't be verified or if it is not a prototype certificate. The use
of {\em pem certify} requires additional out-of-band verification of the prototype certificate
and the presented subject name.

\hl{Options}
\parname {{\bf scan}}
\pardescript {Scan PEM message, write clear body of the message ({\bf default}).} \\
\parname {{\bf mic-clear}}
\pardescript {Read clear body, create Proc-Type MIC-CLEAR.} \\
\parname {{\bf mic-only}}
\pardescript {Read clear body, create Proc-Type MIC-ONLY.} \\
\parname {{\bf encrypted}}
\pardescript {Read clear body, create Proc-Type ENCRYPTED} \\
\parname {{\bf crl}}
\pardescript {Create Proc-Type CRL.} \\
\parname {{\bf crl-rr}}
\pardescript {Create Proc-Type CRL-RETRIEVAL-REQUEST.} \\
\parname {{\bf certify}}
\pardescript {Scan PEM message and create certification reply (for CAs only).} \\
\parname {{\bf -i} {\em inputfile}}
\pardescript {Inputfile (default stdin).} \\
\parname {{\bf -o} {\em outputfile}}
\pardescript {Outputfile (default stdout).} \\
\parname {{\bf -p} {\em psename}}
\pardescript {PSE name (default: {\em .pse}).} \\
\parname {{\bf -c} {\em cadir}}
\pardescript {CA-directory name.} \\
\parname {{\bf -m} {\em depth}}
\pardescript {Depth of multi PEM body, which is to be de-enhanced (if {\bf scan} a nested PEM message). A 
{\em depth} of 
zero identifies the outmost body.} \\
\parname {{\bf -M} {\em depth}}
\pardescript {depth of multi PEM body, up to which is to be de-enhanced (if {\bf scan} a nested PEM 
message).} \\
\parname {{\bf -u} {\em update}}
\pardescript {Controls under which circumstances the PSE or the CA-database will be updated.} \\
\parname {{\bf -r} {\em name1 name2 ... }}
\pardescript {List of intended recipients in case of creating a PEM Proc-Type ENCRYPTED.} \\
\parname {{\bf -y} {\em name1 name2 ... }}
\pardescript {Issuers of CRLs, if crl or crl-rr.} \\
\parname {{\bf -C} }
\pardescript {generate PEM-header with Originator-Certificate and all Issuer-Certificates (default: 
generate PEM-header with Originator-ID-Asymmetric).} \\
\parname {{\bf -n} }
\pardescript {don't insert Key-Info header field for originator (only if pem encrypted).} \\
\parname {{\bf -h} }
\pardescript {write help-text to stderr.} \\
\parname {{\bf -v} }
\pardescript {verbose.} \\
\parname {{\bf -V} }
\pardescript {Verbose.} \\
\parname {{\bf -W} }
\pardescript {Grand Verbose.} \\
\parname {{\bf -R} }
\pardescript {Consult CRLs during the validation process.} \\
\parname {{\bf -F} }
\pardescript {Consider own FCPath as trusted.} \\
\parname {{\bf -D} }
\pardescript {Retrieve certificates and CRLs needed for the MIC verification from Directory.} \\
\parname {{\bf -d} {\em dsaname}}
\pardescript {Name of the DSA to be initially accessed (default: locally configured DSA).} \\
\parname {{\bf -t} {\em dsaptailor}}
\pardescript {Tailorfile for the DUA side. Default: ../etc/dsaptailor of your local ICR1 installation.} \\
\parname {{\bf -A} {\em authlevel}}
\pardescript {Authentication level to be used when accessing the X.500 Directory 
(default: environment variable AUTHLEVEL or no authentication).} \\
\parname {{\bf -H} {\em md-alg}}
\pardescript {Message digest algorithm (default: RSA-MD5).} \\
\parname {{\bf -S} {\em micenc-alg}}
\pardescript {MIC encryption algorithm (default: RSA).} \\
\parname {{\bf -E} {\em msgenc-alg}}
\pardescript {Message encryption algorithm (default: DES-CBC) (only if pem encrypted).} \\
\parname {{\bf -K} {\em dekenc-alg}}
\pardescript {DEK encryption algorithm (default: RSA) (only if pem encrypted).} \\
\parname {{\bf -z} }
\pardescript {Enable control of malloc/free behaviour (needs SecuDE compiled with option MFCHECK)} \\
\parname {{\bf -T} }
\pardescript {Perform public key RSA operations in the SCT (per default this is done in the workstation
even if a smartcard terminal is connected)} \\
\hl{Files}
\btab
\$HOME/.af-alias \4 Per user alias names \\
\$TOP/secude/.af-db/.af-alias \4 Per installation alias names 
\etab
\hl{See Also}
pem\_read(3), pem\_write(3)

\nm{1}{encrypt}{encrypt/decrypt files}
\addtotoc{decrypt}
\label{encrypt}
\hl{Synopsis}
{\bf encrypt} [ {\bf -DnxvVWXYzU} ] [ {\bf -k} {\em key} ] [ {\bf -r} {\em recipient} ] [ {\bf -w} {\em pw} ] [ {\bf -e} {\em deskey} ] [ {\bf 
-E} {\em encalg} ] 
\\ \hspace*{1.59cm} [ {\bf -p} {\em psename} ] [ {\bf -c} {\em cadir} ] [ {\bf -d} {\em dsaname} ] [ {\bf -A} {\em authlevel} ] [ {\bf -t} {\em dsaptailor} ] [ {\em files} ]

{\bf decrypt} [ {\bf -nxvVWXYzU} ] [ {\bf -k} {\em key} ] [ {\bf -w} {\em pw} ] [ {\bf -e} {\em deskey} ] [ {\bf -E} {\em decalg} ] \\
\hspace*{1.59cm} [ {\bf -p} {\em psename} ] [ {\bf -c} {\em cadir} ] [ {\em files} ]
\hl{Description}

{\large\bf Encrypt} encrypts all files listed in {\em files}. It transforms each
file from {\em files} into a corresponding file {\em file}.enc, using 
an algorithm and key which are indicated by one of the options 
{\bf -k}, {\bf -r}, {\bf -w}, or {\bf -e}. Not more than one of those options 
must be given.
The plainfiles are removed after the encryption.

If the use of an asymmetric algorithm is indicated through
these parameters, the so called hybrid method will be applied; i.e. encrypt
generates a DES key, encrypts the file with DES-CBC using this DES-KEY,
and asymmetrically encrypts the DES key. The encrypted DES key is stored as
ASN.1 encoded data type EncryptedKey
in file {\em plainfile}.key which is created if it does not exists. Decrypt
does the reverse process using the given keys if it finds the file {\em plainfile}.key
It is not possible in the hybrid case to use encrypt and decrypt as a filter
which transforms stdin to stdout. Option {\bf -n} causes that in the asymmetric case
the hybrid method is not used, and that straight asymmetric encryption is
used instead (which is time consuming).
 
If none of the parameters {\bf -k}, {\bf -r}, {\bf -w}, or {\bf -e} is given, the own public
encryption key and algorithm which is contained in PSE object {\em Cert} or {\em EncCert} is used
for encryption. 
 
Parameter {\bf -k} {\em key} either denotes a PSE object of type KeyInfo or Certificate, 
or a key reference of type KeyInfo, which contains the encryption key.

The location of the PSE is determined by {\bf -p} {\em psename} and/or {\bf -c} {\em cadir}. 
The default for {\em 
psename} is either provided by the environment variable PSE, or is .pse under the HOME directory, 
if no such variable exists.
 
The default for {\em cadir} is NULL. 

If {\em cadir} is specified,
however, the PSE identified by {\em psename} resides under {\em cadir}; in this case, the default
for {\em psename} is either provided by the environment variable CAPSE, or is .capse, if no such variable
exists. 
 
If option {\bf -r} is given, encryption is done with the public encryption key
of the intended recipient {\em recipient}. {\em recipient} is either a Name (see INTRO(1) on Names)
for which an entry exists in PSE object PKList or EKList, or an alias name. If option {\bf -D}
is provided, the public encryption key is searched for in the X.500 Directory, after it could
neither be found in the PSE nor in the local .af-db Directory.
 
If the X.500 Directory is accessed, parameter {\em dsaname} (option {\bf -d}) allows to specify the name of the DSA (Directory System Agent) which
will initially be contacted. {\em dsaname} is a name which must be provided in the 
{\em dsaptailor} file of the system as {\em dsa\_address} parameter. If {\bf -d} {\em dsaname} is
not given, the default DSA, which is the first {\em dsa\_address} parameter in the {\em dsaptailor} 
file, is contacted.
Parameter {\em authlevel} (option {\bf -A}) may be provided to determine the level of authentication used when binding. If {\bf -A} {\em authlevel} is omitted, the requested 
level of authentication is taken from the environment variable AUTHLEVEL, and is set to {\em no 
authentication}, if no such variable exists.
In case of simple authentication being used, the caller's distinguished name and
X.500 password are retrieved from the PSE objects Name and QuipuPWD, respectively.
 
Parameter {\bf -w} {\em pw} denotes DES-CBC encryption with {\em pw} being a password from which the 
DES key 
is derived through a one-way function.
 
Parameter {\bf -e} {\em deskey} denotes DES-CBC encryption with {\em deskey} comprising the key itself.
If option {bf -x} is given in that case, the DES key is a 16 character string comprising
the key in the [ 0 - 9, A - F ] notation. Otherwise, the key is the bitstring given through
an 8 character long string.

Parameter {\bf -E} {\em encalg} allows to specify an alternative symmetric encryption algorithm.
DES-CBC is the default.

Option {\bf -v} renders the encrypt command verbose, and options {\bf -V} and {\bf -W} provide
some further logging information.

Option {\bf -z} may be provided for testing purposes; it requests to control the malloc/free behaviour.
Option {\bf -U} requests to measure the time used for the cryptographic computations. 
In case that a smartcard terminal is used, option {\bf -T} requests to perform each public key 
RSA operation in the smartcard terminal, and not with the software in the workstation 
(the latter is the default). 
\\ [1em]

{\large\bf Decrypt} decrypts all files which are listed in {\em files} to the corresponding
plainfiles. The filenames in {\em files} should have the .enc suffix (this can be omitted;
however when shell metacharacters are used, the suffixes must not be omitted since
otherwise the shell wouldn't be able to expand the filenames). The .enc files are
removed after the decryption. If decrypt
finds a corresponding {\em file}.key file, it assumes that this contains an asymmetrically
encrypted DES key and applies the hybrid method.
 
Algorithm and key are indicated by one of the options 
{\bf -k}, {\bf -w}, or {\bf -e}. Not more than one of those options 
must be given. If none of the options {\bf -k}, {\bf -w}, or {\bf -d} is given, the own
secret decryption key which is contained in PSE object {\em SKnew} or
{\em DecSKnew} is used for decryption. 

If none of the parameters {\bf -k}, {\bf -w}, or {\bf -e} is given, the own secret
decryption key and algorithm which is contained in PSE object {\em SKnew/SKold} or {\em 
DecSKnew/DecSKold}
is used for the decryption.
 
Parameter {\bf -k} {\em key} denotes a PSE object or a key reference of type KeyInfo
which contains the decryption key.
 
The PSE is indicated through parameter {\bf -p} {\em psename} or {\bf -c} {\em cadir}. If none of them
is given, the default PSE .pse is used.

Parameter {\bf -w} {\em pw} denotes DES-CBC encryption with {\em pw} being a password from which the 
DES key 
is derived through a one-way function.
 
Parameter {\bf -e} {\em deskey} denotes DES CBC decryption with {\em deskey} comprising the key itself.
If option {bf -x} is given in that case, the DES key is a 16 character string comprising
the key in the [ 0 - 9, A - F ] notation. Otherwise, the key is the bitstring given through
an 8 character long string.

Parameter {\bf -E} {\em decalg} allows to specify an alternative symmetric encryption algorithm.
DES-CBC is the default.

Option {\bf -v} renders decrypt verbose, and options {\bf -V} and {\bf -W} provide some further logging
information. For test purposes, option {\bf -z} requests
to control the malloc/free behaviour, and option {\bf -U} requests to measure the time used for the 
cryptographic
algorithms.

If option {\bf -X} is given, the PSE object Random
is used to create random numbers.
If option {\bf -Y} is given, the random number generator
will be initialized by keyboard input.
\\ [1em]

{\bf Original file length encoding}

Since the encrypted file has always a multiple of {\em blocksize} or {\em keysize} octets,
additional encoding of the original file length must be added to the plain file prior
to encryption in order to be able to restore the original file during decryption.
 
Several algorithm identifier already imply a method of additional padding (see INTRO(3X)
about algorithms). If an algorithm is used which does not imply padding, the
following padding methods are applied:
 
In case of DES ({\em blocksize} eight octets), the last octet of the last block of the
decrypted file contains the number of valid octets of that block (between 0 and 7), i.e.
if the original file consists of a multiple of eight octets, one extra block is added
with the last octet set to zero.

In case of RSA, the plain file is blocked into portions of {\em blocksize} octets which
are encrypted into portions of {\em keysize} bits each. The parameter {\em keysize}
belongs to the algorithm while {\em blocksize} is derived from {\em keysize} as
$blocksize = (keysize - 1)/8$. The last octet of the last block of the
decrypted file contains the number of valid octets of that block (between 0 and 255), i.e.
if the original file consists of a multiple of {\em blocksize} octets, one extra block is added
with the last octet set to zero. For this reason the maximum RSA keysize for {\em encrypt}
is 2048.

\hl{Options}

\parname {{\bf -k} {\em key}}
\pardescript {If {\em key} is numeric, it denotes a key reference under which key and 
corresponding algorithm identifier was previously installed. If {\em key} contains 
also non-numeric characters, it denotes the name of a PSE object which contains 
the key and corresponding algorithm identifier. This PSE object must be of type KeyInfo.}  
 
\parname {{\bf -r} {\em recipient}}
\pardescript {{\em user} is the Name of an intended recipient whose public encryption key is used
for encryption. This option applies only for {\em encrypt}. {\em recipient} may be a distinguished
name (in the Name form, must be quoted) or an alias. The corresponding public key is searched
in the PSE-object PKList or EKList and after this in the local afdb Directory. In case that
option {\bf -D} is given, the PK is addiotionally searched in the X.500 Directory.}
i  
\parname {{\bf -w} {\em pw}}
\pardescript {Encryption/decryption with desCBC using a password. The arbitrary long {\em pw} is 
transformed into the DES key using a one-way hash function.
If {\em pw} is omitted, the program prompts for a password from /dev/tty.}
 
\parname {{\bf -e} {\em deskey}}
\pardescript {Encryption/decryption with desCBC. {\em deskey} is an eight character 
ASCII string which comprises the key.}

\parname {{\bf -n}}
\pardescript {Don't use the hybrid method in case that the encryption algorithm is asymmetric.}
 
\parname {{\bf -x}} 
\pardescript {This option applies only in conjunction with option {\bf -e}. In this case,
{\em deskey} is the 16-character hexadecimal representation of the key (i.e. ASCII
characters from [0 - 9] and [a - f] or [A - F]).}

\parname {{\bf -E}} 
\pardescript {This option applies only in conjunction with options {\bf -w} or {\bf -d}. 
In this case, desECB is used instead of desCBC.}

\parname {{\bf -p} {\em psename}}
\pardescript {With this parameter 
the PSE {\em psename} is selected instead of the default PSE {\em .pse}.}

\parname {{\bf -c} {\em cadir}}
\pardescript {{\em cadir} is the CA directory whose PSE is used for encrypt/decrypt.}

\parname {{\bf -v}}
\pardescript {Makes sign and verify verbose.}

\parname {{\bf -V}}
\pardescript {Makes sign and verify Verbose.}

\parname {{\bf -z}}
\pardescript {Enables the control of the malloc/free behaviour. For tests only.}

\parname {{\bf -T}}
\pardescript {Enforces that public key RSA operations are performed in the smartcard terminal
instead with software in the workstation, which is the default. Applies only if smartcards
are used.}

\parname {{\bf -U}}
\pardescript {Enables the measurement of time used for the cryptographic algorithms. For tests only.}

\hl{See Also}
af\_encrypt(3), af\_decrypt(3)

\nm{1}{sign}{sign files / verify signatures of files}
\addtotoc{verify}
\label{sign}
\hl{Synopsis}
{\bf sign} [ {\bf -CvVWUfzh} ] [ {\bf -p} {\em psename} ] [ {\bf -c} {\em cadir} ] [ {\bf -k} {\em key} ] [ {\bf 
-a} {\em issueralg} ] [ {\bf -h} {\em hashinput} ]
\\ \hspace *{1cm} 
 [ {\em files} ]
\\
{\bf verify} [ {\bf -DFRvVWzUh} ] [ {\bf -p} {\em psename} ] [ {\bf -c} {\em cadir} ] [ {\bf 
-k} {\em key} ]
\\ \hspace *{1cm} 
[ {\bf -f} {\em fcpath} ] [ {\bf -d} {\em dsaname} ] [ {\bf -A} {\em authlevel} ] [ {\bf -t} {\em dsaptailor} ] [ {\em files} ]
\hl{Description}
{\bf Sign} generates digital signatures of {\em files}. {\em files} is a list of filenames. For each
{\em file} from the list of files, the signature is stored in file {\em file}.sig.
as ASN.1-encoded data type Signature (as produced by e\_Signature()). If no {\em files} are specified, 
stdin is signed and 
the signature is written to stdout. 

Each signature will be produced using the PSE-object SKnew or SignSK
of the PSE indicated through {\bf -p} {\em psename} or {\bf -c} {\em cadir}.
If {\bf -c} {\em cadir} is given, the PSE .capse of a CA residing in the CA directory {\em cadir}
is selected. \\
If {\bf -p} {\em psename} is given, the PSE {\em psename} is selected. Per default the PSE {\em .pse} is 
selected.

The signature algorithm can be given with {\bf -a} {\em issueralg}. 
{\em issueralg} must be a
signature algorithm (AlgType SIG, see INTRO(3) on available algorithms) which fits to the associated 
algorithm
of the signature key (e.g. the algorithm md5WithRsa cannot be used with an ElGamal key). The *TimeDate 
algorithms
are SecuDE-defined. With these algorithms the time and date when the signature was produced is 
contained
in the signature value and can be restored during the verification.
The default for {\em issueralg} is md5WithRsaTimedate or dsaWithSHA, depending on the signature key. 

If option {\bf -C} is given, for each file the user certificate and the forward certification path related to
the corresponding verification key is written to {\em file}.ctf as ASN.1-encoded data type Certicates 
(note the plural; Certificates is the X.509 Certificates which is a SEQUENCE of the user certificate and
the forward certification path). This allows any user who shares the same public root key to verify
this digital signature. If option {\bf -C} is not given, the digital signature of the file can only be
verified by the signing user itself, using his PSE-object Cert or SignCert. This can be useful to
protect own files against manipulation with no need that others are able to validate this.

The .sig and .ctf files are created if they do not exist.
 
Option {\bf -k} {\em key} allows to specify the signature key to be used. {\em key} may
be the name of a PSE-object or a key reference. In case of option {\bf -k} {\em key} is given,
option {\bf -C} must not be used.

If the selected signature algorithm is sqmodnWithRsa, the hashinput for Sqmodn (modulus) is taken 
from the own verification key which is the public key contained in the PSE-object Cert or SignCert.
If {\bf -h} {\em hashinput} is given, the hashinput for Sqmodn is taken from {\em hashinput}
which may be the name of a PSE-object or a key reference.

If option {\bf -f} is provided, the validity time frame of the signature certificate is not checked.

Option {\bf -v} renders the sign command verbose, and options {\bf -V} and {\bf -W} provide some further
logging information. For test purposes, option {\bf -z} requests
to control the malloc/free behaviour, and option {\bf -U} requests to measure the time used for the 
cryptographic algorithms.
\\ [1em]

{\bf Verify} verifies the digital signatures of {\em files}. For each file from the list of files, 
the signature is expected in {\em file}.sig
as ASN.1-encoded data type Signature (as produced by e\_Signature()). The signature will be verfied using 
the PSE-object Cert or SignCert
of the PSE indicated through {\bf -p} {\em psename} or {\bf -c} {\em cadir}.
If {\bf -c} {\em cadir} is given, the PSE .capse of a CA residing in the CA directory {\em cadir}
is selected. \\
If {\bf -p} {\em psename} is given, the PSE {\em psename} is selected. Per default, the PSE {\em .pse} 
is selected.

{\em verify} uses the verification procedure specified in af\_verify(3).

If option {\bf -D} is given, missing certificates will be automatically retrieved from the Directory.
The Directory may either be a X.500 directory or the local substitute provided by 
SecuDE ({\em .af-db}) (see also DirectoryType(5)).

If the X.500 Directory is accessed, parameter {\em dsaname} (option {\bf -d}) allows to specify the name of the DSA (Directory System Agent) which
will initially be contacted. {\em dsaname} is a name which must be provided in the 
{\em dsaptailor} file of the system as {\em dsa\_address} parameter. If {\bf -d} {\em dsaname} is
not given, the default DSA, which is the first {\em dsa\_address} parameter in the {\em dsaptailor} 
file, is contacted.
Parameter {\em authlevel} (option {\bf -A}) may be provided to determine the level of authentication used when binding. If {\bf -A} {\em authlevel} is omitted, the requested 
level of authentication is taken from the environment variable AUTHLEVEL, and is set to {\em no 
authentication}, if no such variable exists.
In case of simple authentication being used, the caller's distinguished name and
X.500 password are retrieved from the PSE objects Name and QuipuPWD, respectively.

If option {\bf -F} is provided, the own FCPath is considered as being trusted.
This is useful for optimizing the verification process,
if the originator of a digital signature and the PSE owner have a common point of trust
within their forward certification paths, or if the PSE owner wishes to verify signatures
generated by himself.

If option {\bf -R} is provided, a check of certificate revocation lists is required for all certificates
which are in the certification path. In case that option {\bf -D} is additionally provided, missing CRLs will
automatically be retrieved from the Directory. In addition, whenever a CRL is consulted, the 
nextUpdate field of the CRL is compared with the current time and date, and in case of nextUpdate
having expired, the actual CRL will be retrieved from the Directory.

If for a particular file from the list {\em file}.ctf exists and contains an ASN.1 encoded data object Certificates, 
user certificate and forward 
certification path for the verification process are taken from there in order to validate the signature
of {\em file}. The final trusted key for 
the verification process is taken from PSE-objects PKRoot or PKList of the specified PSE.
If the file {\em file}.ctf does not exist, the own user certificate of PSE-object Cert or SignCert and
the own forward certification path of PSE-object FCPath will be used for the verification.

Parameter {\bf -k} {\em key} and {\bf -f} {\em fcpath} allow to specify 
alternative PSE-objects being used for the verification process. This is in general only useful for
test purposes. {\em key} may be the name of a PSE-object containing either a certificate
or a key only, encoded as KeyInfo, or may be a key reference. If {\em key} denotes a key only,
no further certicate chain will be checked.

Option {\bf -v} renders the verify command verbose, and options {\bf -V} and {\bf -W} provide some
further logging information. For test purposes, option {\bf -z} requests
to control the malloc/free behaviour and option {\bf -U} requests to measure the time used for the 
cryptographic
algorithms. In case that a smartcard terminal is used, option {\bf -T} requests to perform each public key 
RSA operation in the smartcard terminal instead with software in the workstation (the latter is the 
default). 

 
\hl{Options}

\parname {{\bf -C}}
\pardescript {The file {\em file}.ctf will be generated, containing the user certificate and the 
forward certification path (as ASN.1-encoded data type Certificates) from the PSE-objects Cert/SignCert
and FCPath of the signer of a file. This parameter applies for sign only.}

\parname {{\bf -D}}
\pardescript {Missing certificates will automatically be retrieved from the Directory (X.500 or .af-db). 
This parameter applies for verify only.}

\parname {{\bf -F}}
\pardescript {Consider own FCPath as trusted. This parameter applies for verify only.}

\parname {{\bf -R}}
\pardescript {Requests that certificate revocation lists will be consulted during
the verification process. This parameter applies for verify only.}

\parname {{\bf -A} {\em authlevel}}
\pardescript {Level of authentication to be used when the X.500 Directory is accessed. 
This parameter applies for verify only.}

\parname {{\bf -d} {\em dsaname}}
\pardescript {Name of the DSA to be accessed for retrieving certificates and certificate revocation lists.}

\parname {{\bf -t} {\em dsaptailor}}
\pardescript {Tailorfile for the DUA side. Default: .../etc/dsaptailor of your local ICR1 installation..}

\parname {{\bf -p} {\em psename}}
\pardescript {The PSE {\em psename} is used instead of the default PSE {\em .pse}.}

\parname {{\bf -c} {\em cadir}}
\pardescript {The PSE .capse contained in the CA directory {\em cadir}
 is used instead of the default PSE {\em .pse}.}

\parname {{\bf -a} {\em signalg}}
\pardescript {The algorithm {\em signalg} is used as signature algorithm.  This parameter applies
for sign only. md5WithRsaTimeDate is the default algorithm. {\em signalg} must be a
signature algorithm (AlgType SIG, see INTRO(3) on available algorithms) which fits to the associated 
algorithm
of the signature key (e.g. the algorithm md5WithRsa cannot be used with an ElGamal key). The *TimeDate 
algorithms
are SecuDE-defined. With these algorithms the time and date when the signature was produced is 
contained
in the signature value and can be restored during the verification.}

\parname {{\bf -h} {\em hashinput}}
\pardescript {If the signature algorithm is sqmodnWithRsa, 
{\em hashinput} denotes a public RSA key which serves as hashinput (per default the own verification 
key is being used for this purpose). If {\em hashinput} is numeric, it denotes a key reference.
Otherwise it denotes a PSE-object which may contain either a certificate or a key only, encoded as 
KeyInfo. This parameter applies for sign only.}

\parname {{\bf -k} {\em key}}
\pardescript {If {\em key} is numeric, it denotes a key reference under which the signature/verification 
key and 
corresponding algorithm identifier was previously installed. If {\em key} contains 
also non-numeric characters, it denotes the name of a PSE object which contains 
the signature/verification key and corresponding algorithm identifier. In the latter case,
the PSE-object may contain either a certificate or a key only, encoded as KeyInfo.}

\parname {{\bf -f} {\em fcpath}}
\pardescript {Name of a PSE object which contains the Forward Certification Path as ASN.1-encoded
data type FCPath (as produced by e\_FCPath()).}

\parname {{\bf -v}}
\pardescript {Makes sign and verify verbose.}

\parname {{\bf -V}}
\pardescript {Makes sign and verify Verbose.}

\parname {{\bf -z}}
\pardescript {Enables the control of the malloc/free behaviour. For tests only.}

\parname {{\bf -T}}
\pardescript {Enforces that public key RSA operations are performed in the smartcard terminal
instead with software in the workstation, which is the default. Applies only if smartcards
are used.}

\parname {{\bf -U}}
\pardescript {Enables the measurement of time used for the cryptographic algorithms. For tests only.}

\hl{See Also}
af\_sign(3), af\_verify(3)

\nm{1}{hsh}{hash files}
\label{hsh}
\hl{Synopsis}
{\bf hsh} [ {\bf -a} {\em alg} ] [ {\bf -k} {\em pk} ] [ {\bf -c} {\em cert} ] [ {\bf -p} {\em psename} ] [ {\em file} ] [ 
{\em hashvalue} ] \\
\hl{Description}
{\bf Hsh} generates an hash-value of {\em file}. The hash-value is stored in file {\em hashvalue}.
If {\em hashvalue} is omitted, 
the hash value is stored in the file {\em file}.hsh. If {\em file} and {\em hashvalue} are omitted, stdin is 
hashed and 
the hash value is written to stdout.

{\em alg} must be a name of an available hash algorithm (see INTRO(3X)). If option
{\em -a} is not given, MD4 is used.

If the algorithm Sqmodn is used, {\bf -k} {\em pk} or {\bf -c} {\em cert}
can be aditionally given. {\em pk} is the object name or key reference of a public RSA key.
{\em cert} is the object name of a certificate which contains a public RSA
key. The modulus of this public RSA key is used for the sqmodn function. If neither -k nor -c
is given, the public key of the PSE object SignCert is used. 

\hl{Options}
\parname {{\bf -a} {\em alg}}
\pardescript {Name of the hash algorithm to be used (default MD5).}

\parname {{\bf -h} {\em pk}}
\pardescript {Object name or key reference of a public RSA key. This is for Sqmodn only.}

\parname {{\bf -c} {\em cert}}
\pardescript {Object name of a certificate which contains a public RSA key. This is for Sqmodn only.}
  
\parname {{\bf -p} {\em psename}}
\pardescript {PSE {\em psename} is used instead of the default PSE {\em .pse}. 
This is for Sqmodn only.}
 
\hl{See Also}
sec\_hash(3)

\nm{1}{sectool}{maintain your PSE}
\label{sectool}
\hl{Synopsis}
{\bf sectool} [ {\bf -zDvVWh} ] [ {\bf -p} {\em psename} ] [ {\bf -c} {\em cadir} ] [ {\bf -d} {\em dsa} ] [{\bf -A} {\em auth\_level}] [{\bf -t} {\em dsaptailor}] \\
{\bf aliastool} [ {\bf -zDvVWh} ] [ {\bf -p} {\em psename} ] [ {\bf -c} {\em cadir} ] [ {\bf -d} {\em dsa} ] [{\bf -A} {\em auth\_level}] [{\bf -t} {\em dsaptailor}] \\
{\bf directorytool} [ {\bf -zDvVWh} ] [ {\bf -p} {\em psename} ] [ {\bf -c} {\em cadir} ] [ {\bf -d} {\em dsa} ] [{\bf -A} {\em auth\_level}] [{\bf -t} {\em dsaptailor}] 
\addtotoc{aliastool}
\addtotoc{directorytool}

\hl{Description}
{\em sectool} is, like psemaint,  a maintenance program which can be used by both
certification authority administrators and users for the purpose
of maintaining their PSEs. This includes moving information (e.g. keys,
certificates, revocation lists etc.) from Unix files or a X.500 Directory
into the PSE and vice versa, generating keys, changing PINs, displaying
the content of the PSE, and maintaining the user's aliases. In contrast
to psemaint, which is line-oriented, sectool is an OpenWindows tool.

{\em Aliastool} is a subtool of sectool which only maintains the alias
files and AliasList PSE object.

{\em Directorytool} is a subtool of sectool which provides a comfortable
interface between X.500 Directories and SecuDE, particularly for the
exchange of security information between X.500 Directories and a PSE.

Sectool is still under development. The version included in SecuDE-4.3
is only a demonstrator version which provides a feeling how it works.
Most functions are still not implemented. SecuDE-5.0 will offer a fully
functional sectool.

\hl{Options}
\parname {{\bf -p} {\em psename}}
\pardescript {PSE {\em psename} is used instead of the default PSE {\em .pse}}. 

\parname {{\bf -c} {\em cadir}}
\pardescript {The PSE .capse contained in the CA directory {\em cadir}
 is used instead of the default PSE {\em .pse}.}

\parname {{\bf -d} {\em dsa}}
\pardescript {Object name of a certificate which contains a public RSA key. This is for Sqmodn only.}
  
\parname {{\bf -A} {\em auth\_level}}
\pardescript {Level of authentication used for X.500 Directory access
                    {\em auth\_level} may have one of the values 'SIMPLE' or 'STRONG'
                    (default: environment variable AUTHLEVEL or 'No authentication')
                    STRONG implies the use of signed DAP operations}

\parname {{\bf -t} {\em dsaptailor}}
\pardescript {Tailorfile for the DUA side. Default: .../etc/dsaptailor of your local ICR1 installation..}

 

\nm{1}{secxlock}{lock the local X display using strong authentication with your PSE}
\label{secxlock}
\hl{Synopsis}
{\bf secxlock} [ {\bf -h} ] [ {\bf -p} {\em psename} ] [ {\bf -c} {\em cadir} ]

\hl{Description}
{\em secxlock}  Locks the local X display using strong authentication with your PSE.

\nm{1}{encode}{encode/decode files}
\addtotoc{decode}
\label{encode}
\hl{Synopsis}
{\bf encode} [ {\bf -r} ] [ {\bf -x} ] [ {\bf -a} ] [ {\bf -i} {\em number} ] [ {\em sourcefile} ] [ {\em destinationfile} ]

{\bf decode} [ {\bf -r} ] [ {\bf -x} ] [ {\bf -a} ] [ {\em sourcefile} ] [ {\em destinationfile} ]
\hl{Description}
{\bf Encode} encodes {\em sourcefile} into {\em destinationfile} according to the encoding scheme
specified by options {\bf -r}, {\bf -x} or {\bf -a}. Only one of those three option may be given.
Encoding means transforming the content of the file (which might be arbitrary binary information)
into a printable ASCII representation.
If {\em destinationfile} is omitted, the program encodes to {\bf stdout}. 
If {\em sourcefile} is also omitted, it encodes from {\bf stdin} to {\bf stdout}.

{\bf Decode} does the reverse process.
 
\hl{Options}
\parname {{\bf -r}}
\pardescript {Encode/decode according to the PEM encoding scheme specified in RFC 1421 (see below).
The encoded file is about one third larger than the source file. This option is the
default value.}

\parname {{\bf -x}}
\pardescript {This encoding/decoding uses a [0 - 9, A - F] ASCII representation for each half-octet
(4 bits). The size of the encoded file is double the size of the source file plus inserted extra NEWLINEs 
(see option -i).}

\parname {{\bf -a}}
\pardescript {Like option -x except that a [ A - P ] ASCII representation is used.}

\parname {{\bf -i} {\em number}} 
\pardescript {NEWLINE characters are inserted into the encoded text every {\em number} characters in 
order to allow
the encoded text to be handled by editors and mail systems. The default {\em number} is 64. A {\em 
number}
of zero avoids NEWLINE insertion. This option applies only for {\em encode}. {\em Decode}
ignores all NEWLINEs.}
\newpage
The PEM encoding scheme maps each 24 bit (three octet) binary group into four octets
printable ASCII in that it maps each 6 bit binary input to one octet ASCII output
according to the following mapping table:
\begin{center}
\begin{tabular} {|cc|cc|cc|cc|} \hline
Value & Encoding & Value & Encoding & Value & Encoding & Value & Encoding \\ \hline
 0 & A & 17 & R & 34 & i & 51 & z \\
 1 & B & 18 & S & 35 & j & 52 & 0 \\
 2 & C & 19 & T & 36 & k & 53 & 1 \\
 3 & D & 20 & U & 37 & l & 54 & 2 \\
 4 & E & 21 & V & 38 & m & 55 & 3 \\
 5 & F & 22 & W & 39 & n & 56 & 4 \\
 6 & G & 23 & X & 40 & o & 57 & 5 \\
 7 & H & 24 & Y & 41 & p & 58 & 6 \\
 8 & I & 25 & Z & 42 & q & 59 & 7 \\
 9 & J & 26 & a & 43 & r & 60 & 8 \\
10 & K & 27 & b & 44 & s & 61 & 9 \\
11 & L & 28 & c & 45 & t & 62 & + \\
12 & M & 29 & d & 46 & u & 63 & / \\
13 & N & 30 & e & 47 & v &    &   \\
14 & O & 31 & f & 48 & w & (pad) & = \\
15 & P & 32 & g & 49 & x &    &   \\
16 & Q & 33 & h & 50 & y &    &   \\ \hline
\end{tabular}
\\ [1em]
Printable Encoding Characters \\
\end{center}
Input quantities less than 24 bits are padded with zeros and mapped
to the padding character '='. \\ [1em]

\hl{See Also}
aux\_enchex(3X)

\nm{1}{km}{Key and certification management and CA provision package}
\label{km}
\hl{Synopsis}
Commands for the CA role:

{\bf cacreate} [ {\bf -hrqzvDVWXY} ] [ {\bf -c} {\em cadir} ] [ {\bf -p} {\em capsename} ] [ {\bf -a} {\em issueralg} ]
             \\ \hspace*{1.6cm} [ {\bf -s} {\em signalg} ] [ {\bf -e} {\em encalg} ] [ {\bf -k} {\em keysize} ]
             \\ \hspace*{1.6cm} [ {\bf -n} {\em serial} ] [ {\bf -u} {\em nextupdate} ] [ {\bf -f} {\em notbefore} ]
	     \\ \hspace*{1.6cm} [ {\bf -l} {\em notafter} ] [ {\bf -d} {\em dsaname} ] [ {\bf -A} {\em authlevel} ] [ {\em Name} ] \\
{\bf getpkroot} [ {\bf -hzvVW} ] [ {\bf -c} {\em cadir} ] [ {\bf -p} {\em capsename} ] [ {\em pkroot} ] \\
{\bf getfcpath} [ {\bf -hzvVW} ] [ {\bf -c} {\em cadir} ] [ {\bf -p} {\em capsename} ] [ {\em fcpath} ] \\
{\bf certify} [ {\bf -hzvVW} ] [ {\bf -c} {\em cadir} ] [ {\bf -p} {\em capsename} ] [ {\bf -a} {\em issueralg} ] \\
\hspace*{1.6cm} [ {\bf -f} {\em notbefore} ] [ {\bf -l} {\em notafter} ] [{\em proto} [{\em cert}]] \\
{\bf gen\_pse} [ {\bf -c} {\em cadir} ] [ {\bf -p} {\em capsename} ] [ {\bf -i} {\em scriptfile} ] [ {\bf -h} {\em 
home} ] [ {\bf -u} {\em unixname} ] \\ 
\hspace*{1.6cm} [ {\bf -a} {\em issueralg} ] [ {\bf -s} {\em signalg} ] [ {\bf -e} {\em encalg} ] [ {\bf -k} {\em 
keysize} ] \\ 
\hspace*{1.6cm} [ {\bf -f} {\em notbefore} ] [ {\bf -l} {\em notafter} ] [ {\bf -x} {\em nameprefix} ]  \\ 
\hspace*{1.6cm} [ {\bf -P} {\em subjectpse} ] [ {\bf -C} {\em subjectca} ] [ {\bf -g} {\em serialnumber} ] \\
\hspace*{1.6cm} [ {\bf -d} {\em dsaname} ] [ {\bf -t} {\em dsaptailor} ] [ {\bf -vrDqznhXY} ] [ {\em namesuffix} ] \\
{\bf revoke} [ {\bf -hzvVWD} ] [ {\bf -c} {\em cadir} ] [ {\bf -p} {\em capsename} ] [ {\bf -a} {\em issueralg} ] 
             \\ \hspace*{1.6cm} [ {\bf -u} {\em nextupdate} ] [ {\bf -d} {\em dsaname} ] [ {\bf -A} {\em authlevel} ] [ {\bf -t} {\em dsaptailor} ] \\
{\bf create\_TestTree} [ {\bf -vPqz} ] \\
{\bf delete\_TestTree}
\\ [1em]
Commands for the user role:

{\bf psecreate} [ {\bf -hqzvVWXY} ] [ {\bf -p} {\em psename} ] [ {\bf -s} {\em signalg} ] 
             \\ \hspace*{1.5cm} [ {\bf -e} {\em encalg} ] [ {\bf -k} {\em keysize} ] [ {\em Name} ] \\ 
{\bf instpkroot} [ {\bf -hzvVW} ] [ {\bf -c} {\em cadir} ] [ {\bf -p} {\em psename} ] [ {\em pkroot} ] \\
{\bf instfcpath} [ {\bf -hzvVW} ] [ {\bf -c} {\em cadir} ] [ {\bf -p} {\em psename} ] [ {\em fcpath} ] \\
{\bf genkey} [ {\bf -hrzvVWXY} ] [ {\bf -c} {\em cadir} ] [ {\bf -p} {\em psename} ] [ {\bf -a} {\em issueralg} ] \\
\hspace*{1.6cm} [ {\bf -s} {\em signalg} ] [ {\bf -e} {\em encalg} ] [ {\bf -k} {\em keysize} ] \\ 
\hspace*{1.6cm} [ {\em proto} ] \\
{\bf getkey} [ {\bf -ehszvVW} ] [ {\bf -c} {\em cadir} ] [ {\bf -p} {\em psename} ] [ {\em proto} ] 
\\
{\bf instcert} [ {\bf -aehzvVWHD} ] [ {\bf -c} {\em cadir} ] [ {\bf -p} {\em psename} ] [ {\bf -d} {\em dsaname} ] [ {\bf -A} {\em authlevel} ] [ {\bf -t} {\em dsaptailor} ] [ {\em cert} ] \\
{\bf pklist} [ {\bf -ehzvVW} ] [ {\bf -c} {\em cadir} ] [ {\bf -p} {\em psename} ] \\
{\bf pkadd} [ {\bf -aehizvVW} ] [ {\bf -c} {\em cadir} ] [ {\bf -p} {\em psename} ] [ {\bf -d} {\em dsaname} ] \\
\hspace*{1.6cm} [ {\bf -A} {\em authlevel} ] [ {\bf -t} {\em dsaptailor} ] {\em Name} \\
{\bf pkdel} [ {\bf -ehzvVW} ] [ {\bf -c} {\em cadir} ] [ {\bf -p} {\em psename} ] [ {\bf -o} {\em owner} ] \\
\hspace*{1.6cm} [ {\bf -i} {\em issuer} ] [ {\bf -n} {\em serial} ] \\
{\bf showdir} [ {\bf -aehzvVW} ] [ {\bf -c} {\em cadir} ] [ {\bf -p} {\em psename} ] [ {\bf -o} {\em attributeType} ] \\ \hspace*{1.6cm} [ {\bf -d} {\em dsaname} ] [ {\bf -A} {\em authlevel} ] [ {\bf -t} {\em dsaptailor} ] {\em Name} \\
{\bf inst\_pse} {\em file} \\
{\bf inst\_ca} {\em file} \\
\hl{Description}
The KM-package provides commands for the establishment of arbitrary Certification Authority (CA) trees 
and for the key and certificate exchange between CAs and users (who may be CAs again). 
It comprises commands for the creation of both CA PSEs and user PSEs
including the generation and exchange of the necessary PSE objects. Recall that a PSE is
a protected area which holds the necessary secret keys, certificates and other security
information objects and which is only accessible by the subject which the PSE owns. A PSE
can be either a smartcard or a DES-protected Unix-directory, or a combination of both (in
case that the smartcard is not capable of storing all necessary information).

A  SecuDE-CA is characterized by the CA directory (which is indicated through {\bf -c} {\em cadir} 
in the KM-commands).
This is a Unix-directory which contains the CA PSE and the CA database. The CA database
stores all certified users, issued certificates, blacklists and the like. It is realized by means of 
Unix ndbm. A CA can be operated by means of the commands of the KM-package and by means of the 
commands {\em sectool(1)} or {\em psemaint(1)} (see there). Sectool(1) is an OpenLook style 
tool which allows the manipulation of the CA PSE and the examination of the CA database, and 
the communication with the directory.

The KM-package provides commands for the CA role and commands for the user role (which can be 
established 
on different systems or through different User-IDs on the same system, for instance). 
An entity plays both the CA role towards the certified users and the user role towards the certifying
CA, if it is neither the highest CA (Root-CA) nor an end user.

The package basicly provides a file interface for the exchange of keys, certificates, other
PSE objects or entire PSEs between users and CAs. The content of these files is the ASN.1 BER data 
representation in case of PSE objects, or a compressed tar-file in case of entire PSEs. The files
can be exchanged by any communication means like file-transfer or electronic-mail, or                
simply file-copy if the communicating user and CA reside in the same (may be distributed)
file-system. They may be subject to processing through {\em sign(1)}, {\em verify(1)}, {\em encrypt(1)},
{\em decrypt(1)}, {\em encode(1)} or {\em uuencode(1)}, {\em decode(1)} or {\em uudecode(1)}
before and after communication, depending on the integrity/confidentiality needs in the
communication between CA and user, or when using SMTP-mail or similar communication means 
which are only capable of transfering printable ASCII text.

The CAs of a CA tree should be created in descending order, i.e. beginning with
the highest CA, because some PSE objects must propagate from top to bottom through the CA-system 
from the top-level CA to the user at the leaf (the forward certification path, for
instance, grows when going down the certification tree).
\\ [1em]
{\bf One or two RSA Key Pairs for Signature and Encryption Purposes}

Each subject's PSE (of both CAs or users) may contain two RSA key pairs, one for signature
purposes and one for encryption purposes. SecuDE allows the separation of these two
functionalities through maintaining two different RSA key pairs.
The signature keys are stored in the PSE objects
SignSK and SignCert, the latter containing the public key as certificate. The encryption
keys are stored in the PSE objects DecSKnew and EncCert. The use of this type of PSEs
must be indicated through use of option {\bf -q} in the corresponding commands.

Some applications, however, require the use of a single key pair for signature
and encryption processes. In PEM, for instance, the same RSA key pair is being used
for MIC computation and DEK exchange. Therefore, the KM-package allows to create
PSEs with a single RSA key pair for both functionalities. This is the default option
of the KM-package (no use of option {\bf -q} in the corresponding commands). 
In this case, only the PSE objects SKnew and Cert are generated.
\\[1em]
{\bf Top-Level CA (Root CA)}

The top-level CA ( or {\bf Root-CA}) can be created using {\em cacreate} (the detailed 
description of the KM commands follows below). Root-CAs use a public keys which are
not further certified. A Root-CA has self-signed certificates of its own public
signature key and encryption key in its PSE objects Cert or SignCert/EncCert.
\\ [1em]
{\bf User-generated or CA-Generated RSA Keys}

Once a Root-CA is created or exists elsewhere, one has two possibilities
to create the next certification level down:
\be
\m The RSA keys are generated at the user site.
\m The RSA keys are generated at the CA site.
\ee
{\bf In case of 1.}, the KM-packaged provides the commands {\em psecreate} or
{\em cacreate}, {\bf getpkroot}, {\em instpkroot}, {\bf getfcpath}, {\em instfcpath},
{\em genkey} or {\em getkey}, {\bf certify} and {\em instcert}. Commands in bold are
to be performed at the CA site while commands in italic are to be performed at the user site.
These commands generate Unix-files comprising the necessary PSE objects, 
which can be exchanged between CA and user
in order to build the user PSE. The Unix-files contain the ASN.1-code
if the PSE objects. Note that it is not necessary to transmit the secret 
RSA keys of the user between user and CA, but that, however, there is a need for an
authentic and uncompromised transfer of the public security information
between CA and user. The Unix-files can be exchanged by any communication means like 
file-transfer or electronic-mail, or simply file-copy if the communicating user and CA 
reside in the same (may be distributed) file-system. They may be subject to processing 
through {\em sign(1)}, {\em verify(1)}, {\em encrypt(1)},
{\em decrypt(1)}, {\em encode(1)} or {\em uuencode(1)}, {\em decode(1)} or {\em uudecode(1)}
before and after communication, depending on the integrity/confidentiality needs in the
communication between CA and user, or when using SMTP-mail or similar communication means 
which are only capable of transfering printable ASCII text. The out-of-band exchange
of transaction keys (or transport keys) between user and CA which are being used
with encrypt/decrypt is outside the scope of the KM-package.

An alternative solution for key distribution in the case that RSA keys are generated
at user sites is offered by PEM (RFC 1424). See {\em pem (1)} for key and certificate
distribution through PEM. 

{\bf In case of 2.}, a CA may generate entire user PSEs by means of the command
{\em gen\_pse}. This command allows to process a script file which contains
a list of users with their distinguished names, alias names and other control
information. The PSEs may be generated as encoded and compressed tar-files which can
be sent by any communication means to the user who can install it using {\em inst\_pse}. 
Gen\_pse allows also to generate
the user PSEs directly in the HOME directories of the users if CA and users reside
in the same system or share the same file-system in a local area network, e.g. NFS.
A third possibility is to generate smartcards with gen\_pse. PSEs which are generated
by gen\_pse are PIN-protected which allows a confidential transmission between
CA and user. 
\\ [1em]
{\bf Updating an Existing Certification Tree}

The KM-package allows also to 
to update an existing certification tree, e.g. if new RSA key pairs shall be used,
if existing keys shall be recertified, or if the existing certification tree shall be 
restructured, for instance if a new Root-CA shall be superordinated to the 
existing Root-CA.
\\ [1em]
{\bf Use of Directories} 
 
Obtaining public security information, like public keys, certificates,
crosscertificates and certificate revocation lists, from other persons or instances is
done either by direct communication between these persons or instances (as it
is the case with PEM, for instance), or by directory access. SecuDE provides the
necessary means to write security information into the directory when this
information is being generated and installed for use, or to read security
information from the directory from time to time to store it in a local cache.
This local cache is realized by means of PSE objects. 
The command {\em instcert}, for instance, allows to enter a user
certificate into a directory. The commands {\em pkadd} and {\em showdir}
retrieve certificates from a directory. All necessary DUA functions are
also available through the commands {\em sectool(1)} and {\em psemaint(1)}.
Sectool(1) is an OpenLook style command to maintain the PSE and to provide
the information flow between local PSE and directory, while psemaint(1) is the 
command-line oriented equivalent.

The directory access can be provided 
by SecuDE, per decision at run time, either through access to public
X.500 directories via DAP (Directory Access Protocol), or through access
to a local substitute. The local substitute is a (Unix) directory
which must be configured at \$TOP/secude/.af-db where \$TOP
is the installation-dependant parent directory of the secude directory. 
 
The .af-db directory contains one subdirectory per user. The name of this
subdirectory is the printable character string form of the distinguished name of the user (i.e.
the Name format which may contain blanks and shell metacharacters). It
contains the signature and encryption certificates (i.e. the public
keys) of that user in the files SignCert and EncCert, respectively. However, if the user owns a single 
RSA key pair for both signature and encryption purposes, the user's certificate is stored in the file
Cert.
If the user is a CA, additional files are provided within the CA's .af-db subdirectory: Crosscertificates 
are stored in the
file CrossCSet, while a certificate revocation list is published in the file PemCRL.
 
The .af-db directory may be shared between different machines by means of NFS, but no other 
mechanisms are supplied to propagate the directory objects. 
 
See also {\bf sectool(1)} and {\bf psemaint(1)} for the access of directory located security objects.
 
{\bf Generating and maintaining PSEs}

The basic purpose of a PSE is to protect private information which is used
in security applications involving cryptographic algorithms. For a detailed 
description of the PSE see INTRO(1). The commands described in this
section build up a PSE. The following table shows which AF PSE objects are created or updated
by which commands:
{\small
\begin{tabular}{|l|l|l|} \hline 
PSE object   & Commands            & Comment                                             \\ \hline
PKRoot       & psecreate,cacreate, & installing public key of top level CA               \\
             & instpkroot,pem      &                                                     \\
FCPath       & instfcpath,pem      & installing own certification path                   \\
SKnew        & psecreate,cacreate  & generating a new secret key (one-keypair PSE only)  \\
Cert         & psecreate,cacreate, & installing certificate of public key as             \\
             & instcert,pem        & hierarchy certificate (one-keypair PSE only)        \\
CSet         & instcert            & installing certificate of public key as             \\
             &                     & crosscertificate (one-keypair PSE only)             \\
SignSK       & psecreate,cacreate  & generating a new secret signature key               \\
             &                     & (two-keypair PSE only)                              \\
DecSKnew     & psecreate,cacreate  & generating a new secret decryption key              \\
             &                     & (two-keypair PSE only)                              \\
SignCert     & psecreate,cacreate, & installing certificate of public signature key as   \\
             & instcert            & hierarchy certificate (two-keypair PSE only)        \\
EncCert      & psecreate,cacreate, & installing certificate of public encryption key as  \\
             & instcert            & hierarchy certificate (two-keypair PSE only)        \\
SignCSet     & instcert            & installing certificate of public signature key as   \\
             &                     & crosscertificate (two-keypair PSE only)             \\
EncCSet      & instcert            & installing certificate of public encryption key as  \\
             &                     & crosscertificate (two-keypair PSE only)             \\
PKList       & pkadd,psemaint,pem  & adding a certificate to the list of trusted         \\
             &                     & public keys (only verification keys in case of      \\
             &                     &  two-keypair PSE)                                   \\
EKList       & pkadd,psemaint      & adding a certificate to the list of trusted         \\
             &                     & public encryption keys (two-keypair PSE only)       \\
PCAList      & psemaint            & adding a certificate to the list of trusted         \\
             &                     & PCAs                                                \\
CrossCSet    & psemaint            & downloading a crosscertificate pair attribute       \\
             &                     & from CA Directory entry                             \\
CrlSet       & cacreate,revoke,    & creating or updating a CRL set                      \\
             & psemaint,pem        &                                                     \\
Random       & psecreate,cacreate  & the seed of a random number generator               \\
             & pem,genkey,gen\_pse &                                                     \\
             & encrypt,decrypt     &                                                     \\
             & psemaint            &                                                     \\
AliasList    & psemaint            & store user alias-list on PSE                        \\
QuipuPWD     & cacreate,psemaint,  & store X.500 password for simple authentication      \\
             & pkadd,showdir,      & on PSE                                              \\
             & instcert            &                                                     \\
EDBKey       & psemaint            & Symmetric encryption key for DSA database encryption \\
SerialNumber & cacreate,certify,   & updating the serial number of the last              \\
             & pem                 & issued certificate (CAs only)                       \\ \hline
\end{tabular}
}

The necessary input data for those commands (except {\em psecreate}, {\em cacreate} and {\em genkey}) 
must be
supplied by the CA (the results of {\em getpkroot}, {\em getfcpath}, or {\em certify}, for instance). 
The commands {\em genkey} and {\em getkey} supply data about the certificate (the so called prototype 
certificate)
which is needed for certification by the CA. 
\\ [1em]
{\bf Commands for the CA role:} 
\\ [1em]
{\bf cacreate} [ {\bf -hrqzvDVWXY} ] [ {\bf -c} {\em cadir} ] [ {\bf -p} {\em capsename} ] [ {\bf -a} {\em issueralg} ]
[ {\bf -s} {\em signalg} ] 
\\ \hspace*{2cm} [ {\bf -k} {\em keysize} ] [ {\bf -e} {\em encalg} ] [ {\bf -k} {\em keysize} ]
[ {\bf -n} {\em serial} ] [ {\bf -u} {\em nextupdate} ] 
\\ \hspace*{2cm} [ {\bf -f} {\em notbefore} ] [ {\bf -l} {\em notafter} ] [ {\bf -d} {\em dsaname} ] [ {\bf -A} {\em authlevel} ] [ {\bf -t} {\em dsaptailor} ] [ {\em Name} ]
\\ [1em]
\addtotoc{cacreate}
{\em cacreate} creates the CA directory {\em cadir} (default: environment variable CADIR or .ca 
under the HOME directory)
which contains:
\bi
\m the prototype PSE of that CA. The PSE name is {\em capsename} (default: environment variable CAPSE or .capse). 
The CA
   PSE contains the PSE object {\em SerialNumber} (this is in contrast to a user PSE). This PSE object 
   is an ASN.1-encoded integer which contains the serial 
   number of the most recently issued certificate.
   This serial number is incremented by each call of {\em certify}, {\em getkey}, and {\em genkey}.
\m a database of that CA, comprising information about all certificates
   ever issued by that CA, and the CA's PEM revocation list. This database is built with 
ndbm(3).
\ei
The CA directory must not exist. 
The distinguished name of the CA is {\em Name}. 
If {\em Name} is omitted, it is requested from 
stdin.
It must be quoted when containing blanks and shell metacharacters (which is normally the case).
We recommend the use of different Name schemes for users and CAs in order to be 
able to distinguish from the syntax of the Name whether a CA or an end user is denoted.
One possibility is to require at least one surname or common name component
for an end user, while a CA is required not to have any of these components. 
 
If option {\bf -q} is not provided, however, the newly created CA's PSE contains one RSA keypair in the PSE 
objects {\bf SKnew} and {\bf Cert}. This type of CA is required within the PEM context, for instance.

Otherwise, if option {\bf -q} is provided, the PSE contains two keypairs 
in the PSE objects {\bf SignSK}/{\bf SignCert} and {\bf DecSKnew}/{\bf EncCert}. {\bf Cert} or 
{\bf SignCert}/{\bf EncCert} are self-signed prototype certificates.
The associated subject public key algorithm name (see intro(3X) about available 
algorithm names and their types) of {\bf Cert} or {\bf SignCert} is {\em signalg} (option {\bf -s}) or {\em rsa} 
by default, 
if omitted. Only algorithms of type {\em SIG} or {\em ASYM\_ENC} are valid. The optional
parameter {\em keysize} immediately following the {\bf -s} option is used to control
the keysize of the signature key.
The associated subject public key algorithm name of {\bf EncCert} (if requested by option {\bf -q})
is {\em encalg} (option {\bf -e}) or {\em rsa} by default, 
if omitted. Only algorithms of type {\em ASYM\_ENC} are valid. The optional
parameter {\em keysize} immediately following the {\bf -e} option is used to control
the keysize of the encryption key.

The prototype certificate Cert/SignCert is installed as PSE object PKRoot.

If option {\bf -r} is provided, the PSE of a Root-CA is created, which means that if the 
generated PSE holds two keypairs, the corresponding certificates are both signed by the signature key
(default: the encryption certificate is signed by the secret decryption key).

In addition, an empty certificate revocation list is created, whose {\em lastUpdate} component is set to the 
current time, whereas the value of the {\em nextUpdate} field may be specified by the {\em nextupdate} parameter (option {\bf -u}).

The empty certificate revocation list is stored in the CA's {\em crlwithcerts} database 
and added to the 
PSE object CrlSet in the CA's PSE.
If option {\bf -D} is given, the empty revocation list is entered into the Directory 
(X.500 or local substitute, respectively).

The algorithm name associated with the signature of the prototype certificates 
(issuer algorithm) can be specified by {\em issueralg} 
(option {\bf -a}). {\em issueralg} must be a
signature algorithm (AlgType SIG, see INTRO(3) on available algorithms) which fits to the associated 
algorithm of the signature key (e.g., the algorithm md5WithRsa cannot be used with an ElGamal key).
The default for {\em issueralg} is md2WithRsaEncryption.
 
The period of validity of the self-signed certificate(s) may be specified by the
{\em notbefore} (option {\bf -f}) and {\em notafter} (option {\bf -l}) times.

If option {\bf -D} is provided, {\em cacreate} enters the prototype certificate(s)
into the Directory. 
The Directory may either be a X.500 directory or the local substitute provided by 
SecuDE ({\em .af-db}) (see also DirectoryType(5)).
If a X.500 Directory is accessed, the prototype certificates are added to the {\em cACertificate} 
attribute in the Root CA's X.500 directory entry.

If a X.500 directory is accessed, parameter {\em dsaname} (option {\bf -d}) allows to specify 
the name of the DSA (Directory System Agent) that
will initially be contacted. {\em dsaname} is a name which must be contained in the 
{\em dsaptailor} file of the system as {\em dsa\_address} parameter. If {\bf -d} {\em dsaname} is
not given, the default DSA, which is the first {\em dsa\_address} parameter in the {\em dsaptailor} 
file, is accessed. 
Parameter {\em authlevel} (option {\bf -A}) may be provided to determine the level of authentication used when binding. If {\bf -A} {\em authlevel} is omitted, the requested 
level of authentication is taken from the environment variable AUTHLEVEL, and is set to {\em no 
authentication}, if no such variable exists.
In case of simple authentication being used, the caller's distinguished name and
X.500 password are retrieved from the PSE objects Name and QuipuPWD, respectively.

Option {\bf -n} is used for specifying the initial value of the serial number to be used by that CA.
 
Option {\bf -z} may be provided for testing purposes; it requests to control the malloc/free behaviour.

Option {\bf -v} renders the cacreate command verbose, and options {\bf -V} and {\bf -W} provide some further logging information.

If option {\bf -X} is given, the PSE object Random
is used to create random numbers.
If option {\bf -Y} is given, the random number generator
will be initialized by keyboard input.
\\ [1em]
{\bf getpkroot} [ {\bf -hzvVW} ] [ {\bf -c} {\em cadir} ] [ {\bf -p} {\em capsename} ] [ {\em pkroot} ]
\addtotoc{getpkroot}

{\em getpkroot} extracts the the PSE object PKRoot from the CA's PSE
and writes it to the file {\em pkroot} or stdout, if {\em pkroot} is omitted. 
This file can be sent to a user of the CA by any means,
who may then install it on his own PSE with the command instpkroot (see below). 
 
The PSE from which the public root information is extracted is identified by {\bf -p} {\em capsename} and/or
{\bf -c} {\em cadir}. 
The default for {\em cadir} is either provided by the environment variable
CADIR or is .ca, if no such variable exists.
The default for {\em capsename} is either provided by the environment variable
CAPSE or is .capse, if no such variable exists.

Option {\bf -z} may be provided for testing purposes; it requests to control the malloc/free behaviour.

Option {\bf -v} renders the getpkroot command verbose, and options {\bf -V} and {\bf -W} provide some further logging information.
\\ [1em]
{\bf getfcpath} [ {\bf -hzvVW} ] [ {\bf -c} {\em cadir} ] [ {\bf -p} {\em capsename} ] [ {\em fcpath} ]
\addtotoc{getfcpath}

{\em getfcpath} reads the CA's PSE object FCPath, adds one further level
to it which comprises the CA's certificates Cert/SignCert and CSet/SignCSet (from the same PSE), and writes the 
extended FCPath
to the file {\em fcpath} or stdout, if {\em fcpath} is omitted. This file can be sent to a user 
affiliated to this
CA, who may then install it on his own PSE with the command instfcpath (see below). 
 
The PSE from which the FCPath is extracted is identified by {\bf -p} {\em capsename} and/or
{\bf -c} {\em cadir}. 
The default for {\em cadir} is either provided by the environment variable
CADIR or is .ca, if no such variable exists.
The default for {\em capsename} is either provided by the environment variable
CAPSE or is .capse, if no such variable exists.
 
Option {\bf -z} may be provided for testing purposes; it requests to control the malloc/free behaviour.

Option {\bf -v} renders the getfcpath command verbose, and options {\bf -V} and {\bf -W} 
provide some further logging information.
\\ [1em]
{\bf certify} [ {\bf -hzvVW} ] [ {\bf -c} {\em cadir} ] [ {\bf -p} {\em capsename} ] [ {\bf -a} {\em issueralg} ] [ {\bf -f} {\em notbefore} ] \\
\hspace*{2cm} [ {\bf -l} {\em notafter} ] [{\em proto} [{\em cert}]]
\addtotoc{certify}

{\em certify} reads a prototype certificate (see {\em genkey} or {\em getkey}) from {\em proto},
replaces the issuer and serial number fields by CA values (taken from the CA's PSE), increments its 
serial number,
replaces the signature field by the CA's signature, and writes the resulting
certificate to {\em cert}. 

Certify reads the prototype certificate
from the file {\em proto} or stdin, if {\em proto} is omitted, and writes the certificate
to the file {\em cert} or stdout, if {\em cert} is omitted. 

The CA's signature 
algorithm can be specified with {\bf -a} {\em issueralg}. {\em issueralg} must be a
signature algorithm (AlgType SIG, see INTRO(3) on available algorithms) which fits to the associated 
algorithm of the signature key (e.g., the algorithm md5WithRsa cannot be used with an ElGamal key).
The default for {\em issueralg} is md2WithRsaEncryption. 

The file {\em cert} can be sent by any means to a user affiliated to this
CA, who can install it on his own PSE with the command instcert (see below). 
 
Before the prototype certificate is transformed into a valid certificate, the signature appended to the prototype certificate is verified with the public key held within the prototype certificate; in this way, it is 
checked whether the public key in {\em subjectPK} is part of an RSA key pair, and whether the corresponding
secret key is owned by the creator of the prototype certificate.

{\em notbefore} (option {\bf -f}) and {\em notafter} (option {\bf -l}) are dates of validity in the 
form 'YYMMDD'. If they are omitted, the first date on which the certificate is valid is set to the current time, 
and the last date on which the certificate is valid is set to one year hereafter.
 
The CA's PSE is identified by {\em capsename} (default: environment variable CAPSE or .capse), which resides in the CA directory {\em cadir}
(default: environment variable CADIR or .ca in the HOME directory).
 
Option {\bf -z} may be provided for testing purposes; it requests to control the malloc/free behaviour.

Option {\bf -v} renders the certify command verbose, and options {\bf -V} and {\bf -W} provide 
some further logging information.
\\ [1em]

{\bf revoke} [ {\bf -hzvVWD} ] [ {\bf -c} {\em cadir} ] [ {\bf -p} {\em capsename} ] [ {\bf -a} {\em issueralg} ] \\
\hspace*{2cm} [ {\bf -u} {\em nextupdate} ] [ {\bf -d} {\em dsaname} ] [ {\bf -t} {\em dsaptailor} ] [ {\bf -A} {\em authlevel} ]
\addtotoc{revoke} 
 
{\em revoke} revokes one or more certificates and enters the resulting revocation list into the 
Directory. \\
 
The certificates which are to be revoked are identified by their serial numbers which are to be
entered interactively (see also {\em caserialnumbers} and {\em issuedcertificate} commands, available
with psemaint). 
Each serial number is submitted to the same check as described for the command {\em revoke}
available with {\em psemaint}. \\
Only in case of at least one certificate having been revoked and added to the revocation list, the 
{\em lastUpdate} component of the revocation list is set to the current time, whereas the value 
of the {\em nextUpdate} component is individually specified by the revoking CA. 

The extended revocation list is written back to the CA's {\em crlwithcerts} database and
added to the PSE object CrlSet in the CA's PSE.

The algorithm name associated with the signature of the revocation list 
(issuer algorithm) can be specified by {\em issueralg} 
(option {\bf -a}). {\em issueralg} must be a
signature algorithm (AlgType SIG, see INTRO(3) on available algorithms) which fits to the associated 
algorithm of the signature key (e.g., the algorithm md5WithRsa cannot be used with an ElGamal key).
The default for {\em issueralg} is md2WithRsaEncryption.

The {\em lastUpdate} component of the extended revocation list is set to the 
current time, whereas the value of the {\em nextUpdate} field may be specified by the {\em nextupdate} parameter (option {\bf -u}), or is else asked for interactively.
 
Option {\bf -D} causes the revocation list to be entered into the Directory.
The Directory may either be a X.500 directory or the local substitute provided by 
SecuDE ({\em .af-db}) (see also DirectoryType(5)).
If a X.500 directory is accessed, parameters {\em dsaname} (option {\bf -d}) and {\em authlevel} 
(option {\bf -A}) may be provided, with the same meaning as with {\em cacreate}.

Option {\bf -z} may be provided for testing purposes; it requests to control the malloc/free behaviour.

Option {\bf -v} renders the revoke command verbose, and options {\bf -V} and {\bf -W} provide 
some further logging information.
\\ [1em]
{\bf gen\_pse} [ {\bf -c} {\em cadir} ] [ {\bf -p} {\em capsename} ] [ {\bf -i} {\em scriptfile} ] [ {\bf -h} {\em 
home} ] [ {\bf -u} {\em unixname} ] \\ 
\hspace*{2cm} [ {\bf -a} {\em issueralg} ] [ {\bf -s} {\em signalg} ] [ {\bf -e} {\em encalg} ] [ {\bf -k} {\em 
keysize} ] \\ 
\hspace*{2cm} [ {\bf -f} {\em notbefore} ] [ {\bf -l} {\em notafter} ] [ {\bf -x} {\em nameprefix} ]  \\ 
\hspace*{2cm} [ {\bf -P} {\em subjectpse} ] [ {\bf -C} {\em subjectca} ] [ {\bf -g} {\em serialnumber} ] \\
\hspace*{2cm} [ {\bf -d} {\em dsaname} ] [ {\bf -t} {\em dsaptailor} ] [ {\bf -vrDqmnXYz} ] [ {\em namesuffix} ]
\addtotoc{gen\_pse}

{\em gen\_pse} can be used by a CA in order to generate complete PSEs for users or underlying CAs,
called subject in the following, at the CA site. {\em Gen\_pse} is able to work on a scriptfile
(parameter {\bf -i} {\em scriptfile}) where a list of subjects and additional necessary
parameters can be specified in order to generate the PSEs for these subjects through one
call of {\em gen\_pse}. All parameters listed after {\bf -i} {\em scriptfile}, and the
options {\bf -d} and {\bf -r} can be defined or redefined in {\em scriptfile} on a per subject
basis. Normal use of {\em gen\_pse} would require only parameter {\bf -i} {\em scriptfile}
and optionally {\bf -c} {\em cadir} and/or {\bf -p} {\em capsename}.

If {\bf -q} is not given, a PSE is restricted to carry one RSA key pair only (for both signature and encryption 
purposes)
and is marked accordingly. In that case, the generated PSE contains the PSE objects Name, SKnew, Cert, 
PKRoot, and FCPath
(the last one only if the issuing CA is not the root CA). This type of PSE is required within the PEM context, 
for instance.
Otherwise, if {\bf -q} is given, each generated PSE contains the PSE objects Name, SignSK, DecSKnew, 
SignCert,
EncCert, PKRoot and FCPath (the last one only if the issuing CA is not the root CA).

If {\bf -h} {\em dir} is not specified, the PSE is produced in the tar-format, compressed
using {\em compress(1)}, encoded to printable ASCII characters using {\em encode(1)}, and
stored in the subdirectory genpse of the CA directory. This file can be transported
to the subject site by any means, e.g. electronic mail, FTAM, ftp or cp, depending on
the local situation, and can be installed at the subject site with {\em inst\_pse}. 
{\em gen\_pse} creates the subdirectory genpse if it does not exist.

If {\bf -h} {\em dir} is specified, {\em dir} is the directory under which all user's
home directories reside (e.g. /home). In this case, the PSE is installed immediately
in the subject's home directory; it is not necessary to use {\em inst\_pse}.
The PSE name is constructed by concatenating {\em dir/unixname/subjectpse}. The
option {\bf -h} {\em dir} is intended to be used in configurations where the CA and
the subjects reside in the same system or local area network of systems with a common
file system (e.g. NFS), for instance if the CA is the system administrator of such
a configuration. Watch for cases where the home directory of a subject is not constructed
as {\em dir/unixname}.

{\em cadir} is the directory of the issuing CA (default: environment variable CADIR or .ca in the HOME
directory).
{\em gen\_pse} requests the CA PIN from /dev/tty.

{\em capsename} is the name of the CA's PSE under {\em cadir} (default: environment variable CAPSE or .capse).

The subject name associated to the new PSE is divided, for the purpose of {\em gen\_pse}, 
into two parts. A nameprefix (e.g. 'C=DE, O=GMD, OU=I2, CN=') is expected from
{\bf -x} {\em nameprefix} or from the scriptfile. A namesuffix (e.g. the common name
or surname, depending on how the nameprefix was chosen) is expected from parameter
{\em namesuffix} or from the scriptfile. Both are concatenated to form the distinguished
name of the subject. The concatenation of nameprefix and namesuffix must result in a name
of the form {\em Name}, i.e. a printable representation of a {\em DName}.

{\em unixname} is the Unix login name of the subject.

{\em subjectpse} is the name of the generated subject's PSE. default is .pse.

If {\em subjectca} is set a CA is generated instead of a PSE where {\em subjectca}
is the name of the generated subject's CA.
In this case a serialnumber can be set with {\bf -g}.

{\em issueralg} denotes the signature algorithm  which is used by the CA to generate
the certificates. If given this is used instead of the algorithm associated to
the CA's signature key. If the CA's signature key has the associated algorithm {\em rsa},
and {\em issueralg} is not given, {\em md2WithRsaEncryption} is used by default.

{\em signalg} and {\em encalg} denote the algorithms associated to the subject's newly
generated signature key pair and encryption key pair respectively. {\em rsa} is the
default in either case. A {\bf -k} {\em keysize} following {\em signalg} or {\em encalg}
denotes the respective keysize. 512 is the default.

{\em notbefore} and {\em notafter} denote the dates of validity of both the generated
signature certificate and encryption certificate. If they are omitted, {\em notbefore} is
set to the current date, and {\em notafter} is set to one year after {\em notbefore}.
{\em notbefore} and {\em notafter} must be given as UTCTime strings (YYMMDDhhmmssZ
or YYMMDDhhmmss+hhmm or YYMMDDhhmmss-hhmm) or like YY:[MM:[DD:[hh:[mm:[ss:]]]]]. 
Example: 921020163015Z is October 20, 1992,
16:30:15 h GMT. 921020173015+0100 denotes the same time as local time CET.

{\em gen\_pse} requests a transport PIN from either /dev/tty or the scriptfile on a 
per subject basis. The transport PIN is used for the generated subject's PSE. The transport
PIN can be changed to the final PIN when installing the generated PSE at the subject's
site with {\em inst\_pse}, or at any time using {\em psemaint}.

Option {\bf -r} replaces an existing PSE in either the subject's home directory (if {\bf -h}
{\em dir} is used) or in {\em cadir}/genpse  (if {\bf -h} {\em dir} is not used). Without
option {\bf -r} existing PSEs are not overwritten.

Option {\bf -D} causes the generated certificates to be stored in the Directory.
The Directory may either be a X.500 directory or the local substitute provided by 
SecuDE ({\em .af-db}) (see also DirectoryType(5)).

If a X.500 Directory is accessed, parameter {\em dsaname} (option {\bf -d}) 
indicates a non-default DSA name for this operation. 
A default DSA name should have been installed by the system administator.

Option {\bf -z} may be provided for testing purposes; it requests to control the malloc/free behaviour.

Option {\bf -v} renders {\em gen\_pse} verbose.

If option {\bf -X} is given, the PSE object Random
is used to create random numbers.
If option {\bf -Y} is given, the random number generator
will be initialized by keyboard input.
\\ [0.5cm]
{\large\bf Format of the Scriptfile}

The scriptfile consists of parameter lines and subject lines (one line per subject
for whom a PSE shall be generated).
Like in a shell script, the character '\#' can be used to indicate a comment until EOL.

{\bf Parameter Lines}

Parameter lines are indicated by a '\$' in column 1.
A parameter line has the format

\${\em parametername} [ = {\em first\_value} [ , {\em second\_value} ]].
A parameter defined in this way is valid for all following subjects until it is redefined.
\\ [1em]
{\bf The following parameters are possible:}
\\ [1em]
\begin{tabular}{|l|l|} 
\hline
{\bf \$onekeypair}                                      &  One key-pair or two different key-pairs \cr
{\bf \$twokeypairs}                                     &  for encryption and signature. \cr                          \hline
{\bf \$issuer\_alg} = {\em alg}                         &  Issuer's signature algorithm  (default: \cr
                                                        &  md2WithRsaEncryption or dsaWithSHA). \cr
{\bf \$subject\_encalg} = {\em alg} [, {\em keysize} ]  &  Subject's signature algorithm (default: rsa, 512). \cr
{\bf \$subject\_sigalg} = {\em alg} [, {\em keysize} ]  &  Subject's encryption algorithm (default: rsa, 512). \cr    \hline
{\bf \$notbefore} = {\em date}                          &  The date and valitity value \cr
{\bf \$notafter} = {\em date}                           &  can be given in format 920812080000+0000 \cr
{\bf \$validity} = {\em validity}                       &  or 92:08:12:08:00:00: like described before. \cr
                                                        &  Default validity is one year from now. \cr                 \hline
{\bf \$enter}                                           &  Enter the certificate into the Directory \cr
{\bf \$notenter}                                        &  Do not enter the certificate into the Directory. \cr
{\bf \$dsa} = {\em dsaname}                             &  Name of the dsa initially approached. \cr                  \hline
{\bf \$subject\_pse} = {\em psename}                    &  Name of the generated subject PSE. \cr
{\bf \$home} [ = {\em pathname} ]                       &  Install the generated PSE directly in the user's home \cr
                                                        &  directory {\em pathname}/{\em username}  \cr
                                                        &  (default pathname: /home). Otherwise write compressed \cr
                                                        &  and encoded tar-file to genpse/{\em username}. \cr
{\bf \$userhome} = {\em name}                           &  Name of the subject's home directory if different \cr
                                                        &  from {\em username}. \cr                                  \hline
{\bf \$user}                                            &  Generate of a user PSE. \cr
{\bf \$ca} [= {\em cadir} [, {\em serialnumber} ] ]     &  Generate a CA with the CA directory {\em cadir} \cr
                                                        &  and the initial serialnumber {\em serialnumber}. \cr      \hline
{\bf \$transportpin} = {\em pin}                        &  The PIN of the generated PSE. \cr                         \hline
{\bf \$notreplace}                                      &  Do not overwrite an existing PSE \cr
{\bf \$replace}                                         &  Overwrite an existing PSE. \cr                            \hline
{\bf \$sc\_pse}                                         &  Generate a smartcard-based PSE. \cr
{\bf \$sw\_pse}                                         &  Generate a SW-PSE. \cr                                    \hline
{\bf \$create}                                          &  Generate a new PSE. \cr
{\bf \$update} = {\em list\_of\_objects}                &  Update the given Objects in an existing PSE \cr           \hline
{\bf \$nameprefix} = {\em prefix}                       &  Prefix of subject's DName. \cr                            \hline
\end{tabular}
\\ [1em]
{\bf Subject Lines}

Subject lines do not begin with a '\$' and consist of three words plus optional
parameter definitions in the \${\em parametername} form.

The first word is the unixname of the subject.
This is added to the list of aliases as alias name of the subject's DN.
The second word is a second alias name which can be used to introduce the subject's
mail-address as alias.
The third word is the namesuffix.
A word which contains spaces or commas must be included in " ".

After the namesuffix additional parameter definitions can be given. They are only
valid for this subject.

The subject line ends with the following EOL.
An EOL included in " " is ignored.
\\ [1em]

An {\bf example} of the scriptfile is given in the following template:
{\small
\begin{verbatim}

# Template of script-file for gen_pse

# Global Parameter Definitions:
# -----------------------------

$onekeypair                          # one key pair is used for encryption and signature
#$twokeypairs                        # two key pairs for encryption and signature

$issuer_alg = md5WithRsaEncryption   # signature algorithm of CA
#$subject_encalg = RSA, 520          # algorithm of subject's encryption algorithm
$subject_sigalg = RSA, 520           # algorithm of subject's signature algorithm

$notbefore = 920812080000+0000       # validity not before of generated certificate
$validity = 02:00:00:                # length of validity period from notbefore

$notenter                            # do not enter generated certificate into directory

$subject_pse = .pse                  # subject's PSE name
$home = /home                        # directory containing all home directories

$user                                # user PSE

$transportpin = uakrus               # PIN of PSE

$replace                             # replace an existing PSE ( creation )

$sw_pse                              # only Software PSE

$update = "pkroot oldpkroot newpkroot fcpath signcert newkey encrcert newkey"
                                     # update some objects on an existing PSE

$nameprefix = "C=DE, O=GMD, CN="     # subject's name prefix

# Subject Lines:
# --------------

surkau  Thomas.Surkau@gmd.de      "Thoma Surkau"       $notreplace $create $home $userhome
schneiw Wolfgang.Schneider@gmd.de "Wolfgang Schneider" $notreplace
\end{verbatim}
}
\vspace{1em}
{\bf create\_TestTree} [ {\bf -vPqz} ] 
\addtotoc{create\_TestTree}

This is a test command for a system administrator who installs SecuDE. {\em create\_TestTree}
creates a tree of one root CA (CA-directory {\em Root-CA}), two subordinate CAs (CA-directories
ORG-1-CA and ORG-2-CA), two users of ORG-1-CA (PSE names ORG-1-User-1 and ORG-2-User-2),
and two users of ORG-2-CA (PSE names ORG-2-User-1 and ORG-2-User-2), including all key generation
and certificates. The CA-directories and PSEs are created in the subdirectory TestTree of the 
home directory. 
These PSEs can be used to test the installation. TestTree must not exist when
create\_TestTree is called.

Option {\bf -P} requests that RFC 1424 defined certification procedures are applied
by using pem(1). Otherwise KM(1) utilities are used for certification.

Option {\bf -q} causes the creation of PSEs with two RSA keypairs.

Option {\bf -z} may be provided for testing purposes; it requests to control the malloc/free behaviour.

Option {\bf -v} renders it verbose. \\ [1em]
{\bf delete\_TestTree}
\addtotoc{delete\_TestTree}

{\em delete\_TestTree} deletes the previously created test tree of CAs and users. It removes
the directory TestTree in the user's home directory.
\\ [1em] 
{\bf Commands for the user role:} 
\\ [1em]
{\bf psecreate} [ {\bf -hqzvVWXY} ] [ {\bf -p} {\em psename} ]
[ {\bf -s} {\em signalg} ] [ {\bf -k} {\em keysize} ] \\ 
\hspace*{2cm} [ {\bf -e} {\em encalg} ] [ {\bf -k} {\em keysize} ] [ '{\em Name}' ]
\addtotoc{psecreate}

{\em psecreate} creates a prototype PSE named {\em psename}. The default for {\em psename} is either
provided by the environment variable PSE, or is .pse under the HOME directory of the user, if there
is no such variable. The PSE directory must not exist.

If option {\bf -q} is not given, the prototype PSE comprises the PSE objects 
\bi
\m SKnew, containing a newly generated secret asymmetric signature and decryption key;
\m Cert, containing the corresponding public key as self-signed prototype certificate;
\m PKRoot, containing the public key from Cert as root key.
\ei
This is a one keypair PSE.
 
If option {\bf -q} is provided, the prototype PSE comprises the PSE objects 
\bi
\m SignSK, containing a newly generated secret asymmetric signature key,
\m SignCert, containing the corresponding public verification key as self-signed prototype certificate,
\m DecSKnew, containing a newly generated secret asymmetric decryption key,
\m EncCert, containing the corresponding public encryption key as self-signed prototype certificate,
\m PKRoot, containing the public key from SignCert as root key.
\ei
This is a two keypair PSE.

{\em Name} represents the distinguished name of the owner of the PSE. 
If {\em Name} is omitted, its value is taken from the environment variable MYDNAME. If there is no
such variable, the distinguished name is requested from 
stdin. 
It must be quoted when containing blanks and shell metacharacters (which is normally the case).
It is recommended to use different Name schemes for users and CAs, in order to be 
able to distinguish from the syntax of the Name whether a CA or an end user is denoted.
One solution would be to require at least one common name component
for an end user, while a CA is required not to have such a component. 
 
The subject public key algorithm name (see intro(3X) about available 
algorithm names and their types) associated with {\bf Cert} or {\bf SignCert} is {\em signalg} 
(option {\bf -s}) or {\em rsa} by default, if omitted. Only algorithms of type {\em SIG} or {\em ASYM\_ENC} are valid. 
The optional parameter {\em keysize} immediately following the {\bf -s} option is used to control
the keysize of the RSA signature key.

The subject public key algorithm name associated with {\bf EncCert} (if requested by option {\bf -q})
is {\em encalg} (option {\bf -e}) or {\em rsa} by default, 
if omitted. Only algorithms of type {\em ASYM\_ENC} are valid. The optional
parameter {\em keysize} immediately following the {\bf -e} option is used to control
the keysize of the RSA encryption key.

The algorithm name associated with the signature of the prototype certificates 
(issuer algorithm) is {\em md2WithRsaEncryption}, if the corresponding key is an RSA key,
or {\em dsaWithSHA}, if the corresponding key is a DSA key. 

Option {\bf -z} may be provided for testing purposes; it requests to control the malloc/free behaviour.

Option {\bf -v} renders the psecreate command verbose, and options {\bf -V} and {\bf -W} 
provide some further logging
information.

If option {\bf -X} is given, the PSE object Random
is used to create random numbers.
If option {\bf -Y} is given, the random number generator
will be initialized by keyboard input.
\\ [1em]
{\bf instpkroot} [ {\bf -hzvVW} ] [ {\bf -c} {\em cadir} ] [ {\bf -p} {\em psename} ] [ {\em pkroot} ]
\addtotoc{instpkroot}

{\em instpkroot} reads file {\em pkroot} or stdin, if {\em pkroot} is omitted, and installs it as
PSE object PKRoot on the PSE. A PKRoot information that already exists on the target PSE will
be overwritten.
 
The location of the target PSE is determined by {\em cadir} and {\em psename}. The default for {\em 
psename} is either provided by the environment variable PSE, or is .pse under the HOME directory, 
if no such variable exists.
 
The default for {\em cadir} is NULL. 

If {\em cadir} is specified,
however, the PSE identified by {\em psename} resides under {\em cadir}; in this case, the default
for {\em psename} is either provided by the environment variable CAPSE, or is .capse, if no such variable
exists.
 
Option {\bf -z} may be provided for testing purposes; it requests to control the malloc/free behaviour.

Option {\bf -v} renders the instpkroot command verbose, and options {\bf -V} and {\bf -W} 
provide some further logging
information.
\\ [1em] 
{\bf instfcpath} [ {\bf -hzvVW} ] [ {\bf -c} {\em cadir} ] [ {\bf -p} {\em psename} ] [ {\em fcpath} ]
\addtotoc{instfcpath}

{\em instfcpath} reads file {\em fcpath} or stdin, if {\em fcpath} is omitted, and installs it as
PSE object FCPath in the PSE. A FCPath information that already exists on the target PSE will
be overwritten.
 
The location of the target PSE is determined by {\em cadir} and {\em psename}. The default for {\em 
psename} is either provided by the environment variable PSE, or is .pse under the HOME directory, if no such
variable exists. 

The default for {\em cadir} is NULL. 

If {\em cadir} is specified,
however, the PSE identified by {\em psename} resides under {\em cadir}; in this case, the default
for {\em psename} is either provided by the environment variable CAPSE, or is .capse, if no such variable
exists.

Before the forward certification path provided is installed on the target PSE, it is checked
whether it comprises a chain of hierarchy certificates that could be verified up the PKRoot 
information already stored on the PSE.
 
Option {\bf -z} may be provided for testing purposes; it requests to control the malloc/free behaviour.

Option {\bf -v} renders the instfcpath command verbose, and options {\bf -V} and {\bf -W} 
provide some further logging information.
\\ [1em]
{\bf genkey} [{\bf -hrtvVW}] [ {\bf -c} {\em cadir} ] [ {\bf -p} {\em psename} ] [ {\bf -a} {\em issueralg} ]
[ {\bf -s} {\em signalg} ] [ {\bf -k} {\em keysize} ] \\
\hspace*{2cm} [ {\bf -e} {\em encalg} ] [ {\bf -k} {\em keysize} ] [ {\em proto} ]
\addtotoc{genkey}

{\em genkey} generates an asymmetric key pair and installs the secret key as PSE object
SignSK (default or option {\bf -s}) or DecSKnew (option {\bf -e}). The public key is written to file
{\em proto} or stdout, if {\em proto} is omitted, in form of a self-signed prototype certificate 
(i.e., the prototype certificate is signed with the corresponding secret key). 
The self-signed prototype certificate is stored as PSE object
SignCert (default or option {\bf -s}) or EncCert (option {\bf -e}).

The location of the PSE where secret key and prototype certificate are stored, is determined by 
{\em cadir} and {\em psename}. The default for {\em 
psename} is either provided by the environment variable PSE, or is .pse under the HOME directory, if no such
variable exists.
 
The default for {\em cadir} is NULL. 

If {\em cadir} is specified,
however, the PSE identified by {\em psename} resides under {\em cadir}; in this case, the default
for {\em psename} is either provided by the environment variable CAPSE, or is .capse, if no such variable
exists.

If the PSE identified by {\em cadir} and/or {\em psename} is restricted to carry one RSA key pair 
only, the secret
key is installed as PSE object SKnew, while the self-signed prototype certificate is stored as PSE object Cert.

In case of a signature key pair being generated, {\bf -s} {\em signalg}
denotes the algorithm associated with the public key information held within the prototype certificate.
In case of an encryption key pair being generated, {\bf -e} {\em encalg}
denotes the algorithm associated with the public key information held within the prototype certificate
(see intro(3X) about available algorithm names and their types).
Only algorithms of type {\em SIG} or {\em ASYM\_ENC} are valid in case of a signature key pair 
being generated, while the algorithms which may be associated with the encryption key pair 
are restricted to those of type {\em ASYM\_ENC}. 
However, if the key pair being generated is used for both signature and encryption purposes
(i.e., the specified PSE is a one keypair PSE), the algorithms associated with
it are restricted to those of type {\em ASYM\_ENC}.
If both option {\bf -s} and option {\bf -e} are omitted, a signature key pair whose
associated algorithm name is rsa will be generated. 

If the generated key pair is of type {\em ASYM\_ENC}, the algorithm name 
associated with the signature of the prototype certificate can be specified by {\em issueralg} 
(option {\bf -a}), or is md2WithRsaEncryption by default. 

The specification of an algorithm via {\bf -s}, {\bf -e} or {\em -a} may optionally
be followed by a {\em keysize} (option {\bf -k}), which controls the keysize of the RSA keys.

{\em genkey} takes all data required for the prototype certificate from the PSE.
 
Option {\bf -r} replaces a previously generated secret key, which has already been installed in 
the PSE, by the newly generated one.

Option {\bf -z} may be provided for testing purposes; it requests to control the malloc/free behaviour.

Option {\bf -v} renders the genkey command verbose, and options {\bf -V} and {\bf -W} 
provide some further logging
information.
\\ [1em]
{\bf getkey} [ {\bf -ehszvVW} ] [ {\bf -c} {\em cadir} ] [ {\bf -p} {\em psename} ] [ {\em proto} ]
\addtotoc{getkey}

Such as {\em genkey}, {\em getkey} creates a prototype certificate and writes its content to file
{\em proto} or stdout, if {\em proto} is omitted. The prototype certificate may then be
signed by a CA (e.g., by using {\em certify}). 

In contrast to {\em genkey}, {\em getkey} does not
generate keys, it just builds a prototype certificate from the public key stored in SignCert
(option {bf -s}) or EncCert (option {\bf -e}). 

The location of the PSE from which the public key information is extracted, is determined by 
{\em cadir} and {\em psename}. The default for {\em 
psename} is either provided by the environment variable PSE, or is .pse under the HOME directory, if no such
variable exists.
 
The default for {\em cadir} is NULL. 

If {\em cadir} is specified,
however, the PSE identified by {\em psename} resides under {\em cadir}; in this case, the default
for {\em psename} is either provided by the environment variable CAPSE, or is .capse, if no such variable
exists.

If the PSE identified by {\em cadir} and/or {\em psename} is restricted to carry 
one RSA key pair only, however, the prototype certificate is built upon the public key contained 
in PSE object Cert.

{\em getkey} is intended to support the cross certification
of an existing key. It can also be used when superordinating a new top-level CA over
an existing Root-CA. In this case, the existing public keys of the previous
Root-CA can be transformed into prototype certificates, in order to be certified by the
new top-level CA.
 
Option {\bf -z} may be provided for testing purposes; it requests to control the malloc/free behaviour.

Option {\bf -v} renders the getkey command verbose, and options {\bf -V} and {\bf -W} provide 
some further logging
information.
\\ [1em] 
{\bf instcert} [ {\bf -aehzvVWHD} ] [ {\bf -c} {\em cadir} ] [ {\bf -p} {\em psename} ] [ {\bf -d} {\em dsaname} ] [ {\bf -A} {\em authlevel} ] [ {\bf -t} {\em dsaptailor} ] \\
\hspace*{2cm} [ {\em cert} ]
\addtotoc{instcert}

{\em instcert} reads file {\em cert} or stdin, if {\em cert} is omitted, 
and installs its content in one of the PSE objects SignCert, EncCert, SignCSet, or EncCSet, depending on
the algorithm identifier of the subject key and option {\bf -H}. If {\bf -H} is given, the
certificate is considered a hierarchy certificate and is installed as PSE object SignCert or
EncCert. Otherwise, it is considered a cross certificate and is installed as (or added to) PSE object SignCSet or EncCSet.

If the algorithm identifier of the subject key indicates a signature algorithm (type {\em SIG}).
the certificate is installed in SignCert or SignCSet.
If the algorithm identifier of the subject key is of type {\em ASYM\_ENC}, the certificate
is installed in SignCert or SignCSet by default; if option {\bf -e} is given, however, it is
installed as EncCert or EncCSet.

The location of the target PSE is determined by {\em cadir} and {\em psename}. The default for {\em 
psename} is either provided by the environment variable PSE, or is .pse under the HOME directory, if no such
variable exists.
 
The default for {\em cadir} is NULL. 

If {\em cadir} is specified,
however, the PSE identified by {\em psename} resides under {\em cadir}; in this case, the default
for {\em psename} is either provided by the environment variable CAPSE, or is .capse, if no such variable
exists.

If the PSE identified by {\em cadir} and {\em psename} is restricted to carry one RSA key pair only, 
however, the certificate is installed as PSE object Cert or CSet, depending on whether or not it is considered a hierarchy certificate (option {\bf -H}).

Before a hierarchy certificate (SignCert or EncCert) is installed, an additional check is 
performed:
A random data block is encrypted - using the certified public key of the hierarchy certificate - and then 
decrypted
- using the secret key of the same keytype (signature or encryption), which has already been stored on the 
PSE 
as object SignSK or DecSKnew, respectively. The hierarchy certificate will
be installed only if the decrypted and the original data are identical, i.e., public and private key fit
together.
If a self-signed prototype certificate has already been installed as object SignCert or EncCert
on the PSE (see {\em genkey}), the following check will be performed before installing the hierarchy 
certificate: 
The certified public key of the hierarchy certificate will be compared with the one contained in the prototype 
certificate. If both public keys are identical, the certified public key of the hierarchy certificate is proved to
fit to the secret key already stored on the PSE, and the hierarchy certificate will replace the self-signed
prototype certificate on the PSE.
If the PSE is restricted to carry one RSA key pair only, however, the PSE objects Cert and SKnew are 
considered for the above consistency checks.
 
If option {\bf -D} was provided, {\em instcert} additionally enters the certificate into the Directory. 
The Directory may either be a X.500 directory or the local substitute provided by 
SecuDE ({\em .af-db}) (see also DirectoryType(5)).

If a X.500 directory is accessed, options {\bf -a}, {\bf -A}, and {\bf -d} may be used:
If option {\bf -a} is provided, the certificate is added to the {\em cACertificate} attribute in
the caller's X.500 directory entry; otherwise, it is added to the {\em userCertificate}
attribute.
Parameters {\em dsaname} (option {\bf -d}) and {\em authlevel} (option {\bf -A}) have the same meaning as with {\em cacreate}.
 
If the {\em .af-db} directory is accessed, the certificate is added to the set of
signature or encryption certificates (file {\em SignCSet} or {\em EncCSet}, respectively)
in the {\em .af-db} subdirectory identified by the caller's distinguished name,
depending on whether or not option {\bf -e} was provided; if the PSE identified by 
{\em cadir} and {\em psename} is restricted to carry one RSA key pair only, 
however, the certificate is added to the set of certificates stored in file {\em CSet} in the same subdirectory.

Option {\bf -z} may be provided for testing purposes; it requests to control the malloc/free behaviour.

Option {\bf -v} renders the instcert command verbose, and options {\bf -V} and {\bf -W} 
provide some further logging
information.
\\ [1em] 
{\bf pkadd} [ {\bf -aehizvVW} ] [ {\bf -c} {\em cadir} ] [ {\bf -p} {\em psename} ] [ {\bf -d} {\em dsaname} ] [ {\bf -A} {\em authlevel} ] {\em Name}
\addtotoc{pkadd}

{\em pkadd} retrieves the signature (by default) or encryption (option {\bf -e}) certificate 
of {\em Name} from the Directory
and adds its ToBeSigned part to the PSE object PKList (by default) or EKList 
(option {\bf -e}).
{\em Name} is a distinguished name in printable form or an alias name.
By default, only those certificates are selected from the returned set whose subject PK algorithm 
is of type {\em SIG} or {\em ASYM\_ENC} (SIGNATURE certificates).
With option {\bf -e}, only those certificates are selected from the returned set whose
subject PK algorithm identifier is of type {\em ASYM\_ENC} (ENCRYPTION certificates).

The location of the target PSE is determined by {\em cadir} and {\em psename}. The default for {\em 
psename} is either provided by the environment variable PSE, or is .pse under the HOME directory, if no such
variable exists.
 
The default for {\em cadir} is NULL. 

If {\em cadir} is specified,
however, the PSE identified by {\em psename} resides under {\em cadir}; in this case, the default
for {\em psename} is either provided by the environment variable CAPSE, or is .capse, if no such variable
exists.

If the PSE identified by {\em cadir} and {\em psename} is restricted to carry 
one RSA key pair only, just those certificates are selected from the returned set whose
subject PK algorithm identifier is of type {\em ASYM\_ENC}; in this case, the
ToBeSigned parts of the selected
certificates are added to the PSE object PKList in any case, even if option {\bf -e} was given. 
 
If a X.500 Directory is accessed, options {\bf -i}, {\bf -a}, {\bf -d}, and {\bf -A} may
be provided:
If {\bf -i} is given and more than one certificate is found for {\em Name}, the user is asked
interactively to specify the certificate whose ToBeSigned part shall be added to his
PKList or EKList, respectively; if option {\bf -i} is omitted, however, the ToBeSigned part
of the first certificate among the returned set of certificates is selected by default.
If option {\bf -a} is provided, the certificates are read from the {\em cACertificate} attribute in
the X.500 directory entry of {\em Name}; otherwise, they are retrieved from the {\em userCertificate} attribute. (If the X.500 directory entry of {\em Name} does not contain the requested attribute,
an appropriate error message is reported to the caller.)
Parameters {\em dsaname} (option {\bf -d}) and {\em authlevel} (option {\bf -A}) have the same meaning as with {\em cacreate}.
 
Option {\bf -z} may be provided for testing purposes; it requests to control the malloc/free behaviour.

Option {\bf -v} renders the pkadd command verbose, and options {\bf -V} and {\bf -W} 
provide some further logging
information.
\\ [1em] 
{\bf pkdel} [ {\bf -ehzvVW} ] [ {\bf -c} {\em cadir} ] [ {\bf -p} {\em psename} ] [ {\bf -o} {\em owner} ] [ {\bf -i} {\em issuer} ] [ 
{\bf -n} {\em serial} ] 
\addtotoc{pkdel}

{\em pkdel} deletes entries from PSE object PKList (by default) or EKList (option {\bf -e}). It
either deletes all entries of the given {\em owner}, or the one entry uniquely identified
by its {\em issuer} and {\em serial} combination. {\em owner} and {\em issuer} are distinguished
names in printable form or alias names.
 
The location of the PSE is determined by {\em cadir} and {\em psename}. The default for {\em 
psename} is either provided by the environment variable PSE, or is .pse under the HOME directory, if no such
variable exists.
 
The default for {\em cadir} is NULL. 

If {\em cadir} is specified,
however, the PSE identified by {\em psename} resides under {\em cadir}; in this case, the default
for {\em psename} is either provided by the environment variable CAPSE, or is .capse, if no such variable
exists.
 
Option {\bf -z} may be provided for testing purposes; it requests to control the malloc/free behaviour.

Option {\bf -v} renders the pkdel command verbose, and options {\bf -V} and {\bf -W} 
provide some further logging
information.
\\ [1em] 
{\bf pklist} [ {\bf -ehzvVW} ] [ {\bf -c} {\em cadir} ] [ {\bf -p} {\em psename} ]
\addtotoc{pklist}

{\em pklist} prints to stdout all entries of the PSE object PKList (by default)
or EKList (option {\bf -e}).
 
The location of the PSE is determined by {\em cadir} and {\em psename}. The default for {\em 
psename} is either provided by the environment variable PSE, or is .pse under the HOME directory, if no such
variable exists.
 
The default for {\em cadir} is NULL. 

If {\em cadir} is specified,
however, the PSE identified by {\em psename} resides under {\em cadir}; in this case, the default
for {\em psename} is either provided by the environment variable CAPSE, or is .capse, if no such variable
exists.
 
Option {\bf -z} may be provided for testing purposes; it requests to control the malloc/free behaviour.

Option {\bf -v} renders the pklist command verbose, and options {\bf -V} and {\bf -W} 
provide some further logging information.
\\ [1em]
{\bf showdir} [ {\bf -aehzvVW} ] [ {\bf -c} {\em cadir} ] [ {\bf -p} {\em psename} ] [ {\bf -o} {\em attributeType} ] [ {\bf -d} {\em dsaname} ] \\
\hspace*{2cm} [ {\bf -A} {\em authlevel} ] [ {\bf -t} {\em dsaptailor} ] {\em Name}
\addtotoc{showdir}

{\em showdir} reads the attribute of type {\em attributeType} (option {\bf -o}) from the directory entry
identified by {\em Name} and prints its contents to stdout in a suitable form.
{\em attributeType} is {\em cert} by default; further possible values for {\em attributeType} are
{\em cross} (for Cross Certificate Pairs) and {\em rev} (for PEM revocation lists).
{\em Name} may be a distinguished name in printable form or an alias name.
The directory may either be a X.500 directory or the local substitute provided by 
SecuDE ({\em .af-db}). (see also DirectoryType(5)).
 
If a X.500 directory is accessed and {\em attributeType} is {\em cert},
the set of certificates stored in the {\em userCertificate} (by default) or 
{\em cACertificate} (option {\bf -a}) attribute within {\em Name}'s X.500 directory entry
is returned, from which the signature (by default) or encryption (option {\bf -e}) certificate(s) 
are selected.
If the local substitute {\em .af-db} is used, however, the signature (by default) or encryption 
(option {\bf -e}) certificate (stored in the file SignCert or EncCert, respectively) in the 
.af-db-subdirectory identified by {\em Name} is printed to stdout. \\
If a X.500 directory is used and {\em attributeType} is {\em cross}, the set of cross certificate 
pairs stored in {\em Name}'s directory entry is printed to stdout; the set of 
cross certificate pairs comprises the value(s) of the {\em crossCertificatePair} attribute. 
If the local substitute is used, however, the set of cross certificate pairs is retrieved from the file CrossCSet in the .af-db-subdirectory identified by {\em Name}. \\
If a X.500 directory is used and {\em attributeType} is {\em rev}, the revocation list stored 
in {\em Name}'s directory entry is printed to stdout. 
If the local substitute is used, however, the revocation list is 
retrieved from the file CRL in the .af-db-subdirectory identified by {\em Name}.
The format of the revocation list is that defined by PEM (see RFC 1422, Appendix A).
 
If a X.500 directory is accessed, options {\bf -d} and {\bf -A} may additionally be
provided, with the same meaning as with {\em cacreate}.

The transformation of {\em Name} into a DName structure requires that you indicate a PSE
to {\em showdir}, in order to be able to verify your own and/or the system alias databases,
which are both signed. Furthermore, your PSE may be required for authenticating you
to the X.500 Directory during the binding.

The location of your PSE is determined by {\em cadir} and {\em psename}. The default for {\em 
psename} is either provided by the environment variable PSE, or is .pse under the HOME directory, if no such
variable exists.
 
The default for {\em cadir} is NULL. 

If {\em cadir} is specified,
however, the PSE identified by {\em psename} resides under {\em cadir}; in this case, the default
for {\em psename} is either provided by the environment variable CAPSE, or is .capse, if no such variable
exists.

Option {\bf -z} may be provided for testing purposes; it requests to control the malloc/free behaviour.

Option {\bf -v} renders the showdir command verbose, and options {\bf -V} and {\bf -W} 
provide some further logging
information.
\\ [1em]
{\bf inst\_pse} {\em file}
\addtotoc{inst\_pse}

{\em inst\_pse} can be used to install a complete subject's PSE which has been previously
generated by a CA using {\em gen\_pse}. {\em inst\_pse} reads the encoded and compressed 
tar-file which comprises the PSE to be installed from {\em file}.
\\ [1em]
{\bf inst\_ca} {\em file}
\addtotoc{inst\_ca}

{\em inst\_ca} can be used to install a complete CA directory which has been previously
generated by a higher CA using {\em gen\_ca}. {\em inst\_ca} reads the encoded and compressed 
tar-file which comprises the CA directory to be installed from {\em file}.
\newpage
\hl{Examples of Different Methods of Key Distribution}

{\bf 1. Generation of the Subject's RSA Keys at the Subject Sites}
\\ [1em]
{\small
Explanation: If the different roles (root CA, ca1, ca2, user) are established under different 
User-IDs or on different systems, one has to separate the following commands at the 
intermediate pipes and to move the corresponding files from one site (or
HOME directory) to the other, for instance by means of file transfer or e-mail (the
latter requires an addional ASCII-encoding of the transmitted files).
To illustrate this, commands which have to
be done by the CA role are in {\bf bold}, and commands which have to be done
by the user role are in {\em italic}.
}
\\ [1em]
To install a two-level CA hierarchy consisting of a root CA on the top level and ca1, ca2
on the second level on the same system and under the same User-ID, one can say
\bvtab
\1 {\bf cacreate} -c .root-ca 'C=DE, O=organization \\
\1 {\bf cacreate} -c .ca1 'C=DE, O=organization, OU=ou1' \\
\1 {\bf cacreate} -c .ca2 'C=DE, O=organization, OU=ou2'
\evtab
to create the CAs with their CA directories .root-ca, .ca1, and .ca2 under the HOME directory
of that user. The PSEs of the CAs are located in the CA directories respectively.
All CA PSEs are prototype PSEs at this moment, i.e. each of them contains an RSA key pair 
(the public key as self-signed prototype certificate) and a PKRoot which consists
of the own public key and issuer name set to the own subject name. 

Now the PSE object PKRoot has to be taken from the root CA and installed in ca1 and ca2 by saying
\bvtab
\1 {\bf getpkroot} -c .root-ca $|$ {\em instpkroot} -c .ca1 \\
\1 {\bf getpkroot} -c .root-ca $|$ {\em instpkroot} -c .ca2
\evtab
An FCPath does not exist on the first two levels of CAs (the users of ca1 and ca2, however,
have FCPaths which have to be installed in the user PSEs; recall Vol. 1 chapter 2 for this).
 
The next step is to obtain the
respective certificates for the public keys from the root CA, and to install those 
certificates in the ca1 and ca2 PSEs.
\bvtab
\1 {\em getkey} -c .ca1 $|$ {\bf certify} -c .root-ca -q $|$ {\em instcert} -c .ca1 -h \\
\1 {\em getkey} -c .ca2 $|$ {\bf certify} -c .root-ca -q $|$ {\em instcert} -c .ca2 -h \\
\evtab

Now the creation of the CA directories .root-ca, .ca1 and .ca2 is complete.
 
To create a PSE of a user of ca1, one says for instance
\bvtab
\1 {\em psecreate} 'C=DE, O=organization, OU=ou1, CN=username' \\
\1 {\bf getpkroot} -c .ca1 $|$ {\em instpkroot} \\
\1 {\bf getfcpath} -c .ca1 $|$ {\em instfcpath} \\
\1 {\em getkey} $|$ {\bf certify} -c .ca1 -q $|$ {\em instcert} -h \\
\evtab
which generates the user PSE .pse under the HOME directory of the user.
\\ [1em]
{\bf 2. Generation of the Subject's RSA Keys at the CA Sites}
\\ [1em]
Utility {\em gen\_pse (1)} can be used to generate entire user PSEs at the CA
site. This is a reasonable alternative, for instance, in the case that smartcards
are generated and distributed through the CA. Say
\bvtab
\1 {\bf gen\_pse} -c .ca1 -i gen\_script  \\
\evtab
to generate user PSEs at CA ca1. gen\_script is the script-file which contains
the necessary information about the users of ca1 and how their PSEs shall be generated.
All SW-PSEs which are generated this way are stored as encoded and compressed tar-files
in the directory .ca1/genpse. They are protected through the transport PIN which can
be specified in the script-file. These tar-files can be transmitted to the users
who may install them in their environment using inst\_pse(1).
\bvtab
\1 {\bf gen\_pse} -c .ca1 -i gen\_script -h /home \\
\evtab
does the same except that SP-PSEs are written directly into /home/{\em username}
where {\em username} is the Unix name of the user as specified in the scriptfile.
This option may be useful if the CA and the users of the CA share the same
file-system (e.g. in a local area network with NFS). Note that gen\_pse
must have the necessary permissions to do so.
\\ [1em]
{\bf 3. Key Distribution using PEM}
\\ [1em]
See {\em pem (1)}.
