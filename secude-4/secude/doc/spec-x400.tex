\speconly{
\section{X400-IF: X.400-1988 Security Interface}
\markboth{X.400 Interface}{X.400 Interface}
\thispagestyle{myheadings}
\label{x4-if}

\subsection{Introduction}
\label{x4_fc}

In this chapter, C-functions and C-structures are specified,
which enable an X.400 programmer to add X.400 security elements
into an existing X.400 implementation.
In the following subsections, four types of functions are presented:

\begin{enumerate}
\item
The {\em x400\_create\_*} functions fill the C-structures,
which represent X.400 security elements, with the relevant
security informations.
\item
The {\em x400\_check\_*} functions verify (received) X.400
security elements which are represented as C-structures.
\item
The {\em x400\_include\_*} functions encode the C-structures,
which represent X.400 security elements, into protocol presentation
elements and include them into the existing message presentation
element.
\item
The {\em x400\_extract\_*} functions extract the security
oriented protocol presentation elements from
the existing message presentation
element and decode them into C-structures
which represent X.400 security elements.
\end{enumerate}

Those functions which {\em include}
the security informations stored in the C-structures
into the protocol presentation elements,
and those functions which {\em extract}
the security informations
from the protocol presentation elements
and store them into the C-structures,
{\em depend on the implementation} of the X.400 product.
This is, because the protocol presentation is an essential part
of any X.400 implementation.
\\
Therefore, this specification uses an abstract type PRESEL,
which is a placeholder for the presentation element type of an
X.400 implementation:
\\[1ex]
\begin{center}
typedef $<$local presentation element type$>$ {\bf PRESEL}
\end{center}
%\newpage
\subsubsection{Get Originator Certificate}
\hll{x400\_get\_OCert}
\label{x4_get_ocert}

This functionality is performed by {\em af\_pse\_get\_Certificates}
of the AF-Module
(\ref{af_get_Certificates}).
}
\subsection{X.400 Security Functions}

\subsubsection{x400\_create\_Label}
\nm{3X}{x400\_create\_Label}{Create a security label}
\label{x4_cr_Lab}
\hl{Synopsis}
\#include $<$x400.h$>$ \\ [1ex]
RC {\bf x400\_create\_Label} {\em (label)} \\
Label {\em *label};
\hl{Parameter}
\parname  {{\em label}}
\pardescript  {Pointer to a Label structure defined by calling program.
	       {\em label\pf policy} is set by the calling function.
	       The other subfields of label,
	       {\em class}, {\em mark} and {\em categs},
	       will be allocated by called function
	       and can/must be freed by calling function.} \\[1ex]
\parname  {{\em Return Value}}
\pardescript  {0 or -1}
\hl{Description}
X400\_create\_Label
creates and fills the subfields
{\em label\pf class}, {\em label\pf mark} and {\em label\pf categs} ,
in dependence of the value of {\em label\pf policy}
(default value is
\\
{\em \{ id-dfn-ssp \} }).
For the simple security policy,
there is a terminal prompt for
``authentic'', ``confidential'', or ``both, authentic and confidential''.
In dependence of the terminal input, {\em class} and {\em mark}
are set. {\em categs} is set to NULL.
\\
The return value is 0 on success, and -1 if the security policy
is unknown.

\subsubsection{x400\_create\_MAC}
\nm{3X}{x400\_create\_MAC}{Create message authentication check}
\label{x4_cr_MAC}
\hl{Synopsis}
\#include $<$x400.h$>$ \\ [1ex]
RC {\bf x400\_create\_MAC} {\em (mac, content, contid, label)} \\
MAC {\em *mac};  \\
PRESEL {\em *content}; \\
char {\em *contid};    \\
Label {\em *label};
\hl{Parameter}
\parname  {{\em mac}}
\pardescript  {Pointer to a MAC structure defined by calling program.
	       Its subfields
	       will be allocated and filled by called function
	       and can/must be freed by calling function.} \\[1ex]
\parname  {{\em content}}
\pardescript  {Input: Pointer to a protocol presentation element
	       that contains the message content.} \\[1ex]
\parname  {{\em contid}}
\pardescript  {Input: Pointer to character string
	       that contains the message content identifier.} \\[1ex]
\parname  {{\em label}}
\pardescript  {Input: Pointer to a Label structure.} \\[1ex]
\parname  {{\em Return Value}}
\pardescript  {0 or -1}
\hl{Description}
X400\_create\_MAC
creates and fills the subfields of {\em mac\pf tbs},
encodes them into {\em mac\pf tbs\_DERcode} and
signs the latter into {\em mac\pf sig}.
This requires the following steps:
\\
Read mac\pf tbs\pf signatureAI from the
originator's PSE object SignCert.
Transform (PRESEL *)content to (OctetString *)mac\pf tbs\pf \-con\-tent.
Assign contid and label to tbs-subfields.
Perform e\_MACTBS on mac\pf tbs and assign the result to mac\pf tbs\_DERcode.
Sign the octetstring mac\pf tbs\_DERcode and place the result into mac\pf sig.
\\
Subfunctions used are e\_MACTBS(), af\_sign().
The return value is 0 on success.
If one of the operations fails,
the respective error code is set, and -1 is returned.

\subsubsection{x400\_create\_MsgToken}
\nm{3X}{x400\_create\_MsgToken}{Create message token}
\label{x4_cr_MTok}
\hl{Synopsis}
\#include $<$x400.h$>$ \\ [1ex]
RC {\bf x400\_create\_MsgToken}
{\em (token, label, recipname, recipcert, mseqnumber)} \\
MsgToken {\em *token};  \\
Label {\em *label}; \\
DName {\em *recipname}; \\
Certificate {\em *recipcert}; \\
int {\em *mseqnumber};    \\
\hl{Parameter}
\parname  {{\em token}}
\pardescript  {Pointer to a MsgToken structure defined by calling program.
	       Its subfields
	       will be allocated and filled by called function
	       and can/must be freed by calling function.} \\[1ex]
\parname  {{\em label}}
\pardescript  {Input: Pointer to a Label structure.} \\[1ex]
\parname  {{\em recipname}}
\pardescript  {Input: Pointer to a DName that contains the
	       name of the intended message recipient.} \\[1ex]
\parname  {{\em recipcert}}
\pardescript  {Input: Pointer to a user certificate of the recipient.
	       It might be set to NULL.
	       However, if given, the certificate owner must be the same
	       as the recipient given in the parameter {\em recipname}.} \\[1ex]
\parname  {{\em mseqnumber}}
\pardescript  {Input: Pointer to an integer.
	       Calling program must assign a message sequence
	       number to it or set it to NULL.
	       It will be assigned to token\pf tbs\pf mseq.} \\[1ex]
\parname  {{\em Return Value}}
\pardescript  {0 or -1}
\hl{Description}
X400\_create\_MsgToken
creates and fills the subfields of {\em token\pf tbs},
encodes them into {\em token\pf tbs\_DERcode} and
signs the latter into {\em token\pf sig}.
This requires the following steps:
\\
Evaluate input parameter label and then fill only those
token subfields which are required by the security policy,
the classification and the privacy mark.
If no recipient name is given, terminate with error code ERECIPNAME;
otherwise assign the token subfields
label (if required), recipient name,
(current) time and signatureAI (from PSE object SignCert).
\\
If ``authentic'' is requested, assign mseqnumber to
token\pf tbs\pf mseq.
If, however, mseqnumber is set to (int *)NULL,
and ``confidential'' is not requested,
terminate with error code ENOTOKEN.
\\
If ``confidential'' is requested,
a structure TokenTBE is to be created and filled.
Create a symmetric key and place it into the
subfield cckey of the TokenTBE,
and the symmetric algorithm identifier is assigned
to token\pf tbs\pf confidAI.
If no recipient certificate is given,
read the recipient public encryption key information from
PSE object EKList.
If this is not available, terminate with error code ERECIPKEY.
If a recipient certificate is given and valid,
read the recipient public encryption key information from
this certificate.
The encryption algorithm is assigned to token\pf tbs\pf encryptionAI.
\\
Perform e\_TokenTBE on the created and filled TokenTBE structure,
encrypt the result and assign the encrypted bitstring
to token\pf tbs\pf encrypted.
Perform e\_Msg\-To\-ken\-TBS on token\pf tbs and assign the result
to token\pf tbs\_DERcode.
Sign the octetstring token\pf tbs\_DERcode and place the result
into token\pf sig.
\\
Subfunctions used are e\_TokenTBE(), e\_MsgTokenTBS(), af\_encrypt(),
and af\_\-sign().
The return value is 0 on success.
If one of the operations fails,
the respective error code is set, and -1 is returned.

\subsubsection{x400\_create\_BindToken}
\nm{3X}{x400\_create\_BindToken}{Create bind token}
\label{x4_cr_BTok}
\hl{Synopsis}
\#include $<$x400.h$>$ \\ [1ex]
RC {\bf x400\_create\_BindToken}
{\em (token, recipname, random)} \\
BindToken {\em *token};  \\
DName {\em *recipname}; \\
BitString {\em *random};    \\
\hl{Parameter}
\parname  {{\em token}}
\pardescript  {Pointer to a BindToken structure defined by calling program.
	       Its subfields
	       will be allocated and filled by called function
	       and can/must be freed by calling function.} \\[1ex]
\parname  {{\em recipname}}
\pardescript  {Input: Pointer to a DName that contains the
	       name of the intended token recipient.} \\[1ex]
\parname  {{\em random}}
\pardescript  {Input: Pointer to a BitString.
	       If set to NULL,
	       this called function will create a random number,
	       otherwise the given random number is used unchanged.} \\[1ex]
\parname  {{\em Return Value}}
\pardescript  {0 or -1}
\hl{Description}
X400\_create\_BindToken
creates and fills the subfields of {\em token\pf tbs},
encodes them into {\em token\pf tbs\_DERcode} and
signs the latter into {\em token\pf sig}.
No encryption service is offered, therefore
token\pf tbs\pf encryptionAI and token\pf tbs\pf encrypted are set to NULL.
The following steps are performed:
\\
If no recipient name is given, terminate with error code ERECIPNAME;
otherwise assign the token subfields
recipient name,
(current) time and signatureAI (from PSE object SignCert).
If the random parameter is set on input,
the called function will assign it unchanged
to token\pf tbs\pf random.
If the random parameter is NULL on input,
the called function will create a random number and assign it
to token\pf tbs\pf random.
\\
Perform e\_BindTokenTBS() on token\pf tbs and assign the result
to token\pf tbs\_DERcode.
Sign the octetstring token\pf tbs\_DERcode and place the result
into token\pf sig.
\\
Subfunctions used are e\_BindTokenTBS() and af\_sign().
The return value is 0 on success.
If one of the operations fails,
the respective error code is set, and -1 is returned.

\subsubsection{x400\_create\_Context}
\nm{3X}{x400\_create\_Context}{Create security context}
\label{x4_cr_Scon}
\hl{Synopsis}
\#include $<$x400.h$>$ \\ [1ex]
RC {\bf x400\_create\_Context} {\em (seccont)} \\
SecurityContext {\em *seccont};
\hl{Parameter}
\parname  {{\em seccont}}
\pardescript  {Pointer to a SecurityContext structure
	       defined by calling program.
	       Its subfields
	       will be allocated and filled by called function
	       and can/must be freed by calling function.} \\[1ex]
\parname  {{\em Return Value}}
\pardescript  {0 or -1}
\hl{Description}
X400\_create\_Context
constructs one simple security label that consists only
the subfield {\em policy} set to \{ id-dfn-ssp \} .
The other subfields are set to NULL.
{\em seccont}\pf {\em next} is also set to NULL.
In later versions, this function shall become more complex
in order to assemble different security labels.
\\
Subfunction used is x400\_create\_Label(),
although this is not necessary for the simple version.
The return value is 0 on success, -1 on error.

\subsubsection{x400\_create\_Credentials}
\nm{3X}{x400\_create\_Credentials}{Create credentials}
\label{x4_cr_Cred}
\hl{Synopsis}
\#include $<$x400.h$>$ \\ [1ex]
RC {\bf x400\_create\_Credentials}
{\em (creds, recip\_name, random, certype, ca\_name)} \\
Credentials {\em *creds};  \\
DName {\em *recip\_name}; \\
BitString {\em *random};    \\
KeyType {\em certype};    \\
DName {\em *ca\_name}; \\
\hl{Parameter}
\parname  {{\em creds}}
\pardescript  {Pointer to a Credentials structure defined by calling program.
	       Its subfields
	       will be allocated and filled by called function
	       and can/must be freed by calling function.} \\[1ex]
\parname  {{\em recip\_name}}
\pardescript  {Input: Pointer to a DName that contains the
	       name of the intended credentials recipient.} \\[1ex]
\parname  {{\em random}}
\pardescript  {Input: Pointer to a BitString.
	       If set to NULL,
	       this called function will create a random number,
	       otherwise the given random number is used unchanged.} \\[1ex]
\parname  {{\em certype}}
\pardescript  {Input: Type of the user certificate of Certificates: \\
	       SIGNATURE: user certificate of the signature key, \\
	       ENCRYPTION: user certificate of the encryption key. \\
	       -1: no Certificates to be included in Credentials structure.} \\[1ex]
\parname  {{\em ca\_name}}
\pardescript  {Input: Pointer to a DName that contains the
	       name of a certification authority above
	       the intended token recipient.} \\[1ex]
\parname  {{\em Return Value}}
\pardescript  {0 or -1}
\hl{Description}
X400\_create\_Credentials
calls X400\_create\_BindToken() and af\_pse\_get\_Certificates()
and assembles the bind token and certificates into the credentials structure.
The error codes of the called functions are used.
\\
Subfunctions used are e\_BindToken() and af\_pse\_get\_Certificates().
The return value is 0 on success.
If one of the operations fails,
the respective error code is set, and -1 is returned.

\subsubsection{Check Originator Certificate}
\hll{x400\_check\_OCert}
\label{x4_check_ocert}

This functionality is performed by {\em af\_verify}
of the AF-Module
\speconly{(\ref{af_verify})}
.

\subsubsection{x400\_check\_Label}
\nm{3X}{x400\_check\_Label}{Check validity of a security label}
\label{x4_ck_Lab}
\hl{Synopsis}
\#include $<$x400.h$>$ \\ [1ex]
RC {\bf x400\_check\_Label} {\em (label)} \\
Label {\em *label};
\hl{Parameter}
\parname  {{\em label}}
\pardescript  {Input: Pointer to a Label structure defined by calling function.
	       All subfields are defined and filled
	       by the calling function.} \\[1ex]
\parname  {{\em Return Value}}
\pardescript  {0 or -1}
\hl{Description}
X400\_check\_Label
checks, if the security policy is known.
In case of the simple security policy (\{ id-dfn-ssp \}),
the classification must be ``unrestricted'', ``restricted''
or ``confidential'';
and the privacy mark must be ``a'', ``b'', or ``c'' only.
Change classification ``confidential'' and privacy mark ``a''
to privacy mark ``b''.
Change classification ``confidential'' and no privacy mark
to privacy mark ``c''.
\\
The return value is 0 on success, and -1 if the label is invalid or the
security policy is unknown.
The error code would be EPOLICY, if the security policy is unknown,
and ELABEL, if the label is not consistent or invalid.

\subsubsection{x400\_check\_MAC}
\nm{3X}{x400\_check\_MAC}{Check message authentication}
\label{x4_ck_MAC}
\hl{Synopsis}
\#include $<$x400.h$>$ \\ [1ex]
RC {\bf x400\_check\_MAC} {\em (mac)} \\
MAC {\em *mac};
\hl{Parameter}
\parname  {{\em mac}}
\pardescript  {Input: Pointer to a MAC structure defined by calling function.
	       mac\pf sig and
	       all subfields of mac\pf tbs are defined and filled
	       by the calling function.
	       mac\pf tbs\_DERcode is optionally NULL.} \\[1ex]
\parname  {{\em Return Value}}
\pardescript  {0 or -1}
\hl{Description}
If {\em mac}\pf {\em tbs\_DERcode} is NULL on input,
x400\_check\_MAC calls x400\_create\_MACTBS() in order to fill it.
The MAC signature is verified
by comparison of {\em mac}\pf {\em tbs\_DERcode}
with {\em mac}\pf {\em sig}.
\\
Subfunctions called are x400\_create\_MACTBS(), af\_verify().
The return value is 0, if the given MAC is valid
and if the signature is  verified.
The return value is -1 otherwise;
the error code would be EMAC
or one inherited by one of the called subfunctions.

\subsubsection{x400\_check\_MsgToken}
\nm{3X}{x400\_check\_MsgToken}{Verify and decrypt message token}
\label{x4_ck_MTok}
\hl{Synopsis}
\#include $<$x400.h$>$ \\ [1ex]
RC {\bf x400\_check\_MsgToken} {\em (token, tokentbe)} \\
MsgToken {\em *token}; \\
TokenTBE {\em *tokentbe};
\hl{Parameter}
\parname  {{\em token}}
\pardescript  {Input: Pointer to a MsgToken structure defined by calling function.
	       All subfields are defined and filled
	       by the calling function.} \\[1ex]
\parname  {{\em tokentbe}}
\pardescript  {Input: Pointer to a TokenTBE structure defined by calling function.
	       Its subfields will be allocated and filled by called
	       function and can/must be freed by calling function.} \\[1ex]
\parname  {{\em Return Value}}
\pardescript  {0 or -1}
\hl{Description}
X400\_check\_MsgToken verifies the token signed data and decrypts
the token encrypted data.
If a security label is included in the token signed data,
it is checked
by a call to x400\_check\_Label().
The validity of the token time is checked.
If token encrypted data are present, they are encrypted
and the decrypted result is put into the TokenTBE parameter.
\\
If included,
the label subfields, the content confidentiality key (from TokenTBE),
and the message sequence number
remain for further checks and actions of the calling function.
\\
Subfunctions called are x400\_check\_Label(), af\_decrypt(), af\_verify().
The return value is 0, if the given MsgToken is valid,
if the encrypted data can be decrypted,
and if the signature is  verified.
The return value is -1 otherwise;
the error code would be ETOKSIGN, ETOKENCRYPT or ETOKTIME
or one inherited by one of the called subfunctions.

\subsubsection{x400\_check\_BindToken}
\nm{3X}{x400\_check\_BindToken}{Verify and decrypt bind token}
\label{x4_ck_BTok}
\hl{Synopsis}
\#include $<$x400.h$>$ \\ [1ex]
RC {\bf x400\_check\_BindToken} {\em (token, tokentbe)} \\
BindToken {\em *token}; \\
TokenTBE {\em *tokentbe};
\hl{Parameter}
\parname  {{\em token}}
\pardescript  {Input: Pointer to a BindToken structure defined by calling function.
	       All subfields are defined and filled
	       by the calling function.} \\[1ex]
\parname  {{\em tokentbe}}
\pardescript  {Input: Pointer to a TokenTBE structure defined by calling function.
	       Its subfields will be allocated and filled by called
	       function and can/must be freed by calling function.} \\[1ex]
\parname  {{\em Return Value}}
\pardescript  {0 or -1}
\hl{Description}
X400\_check\_BindToken verifies the token signed data
(and decrypts the token encrypted data into the TokenTBE parameter,
if a confidentiality service is implemented).
The validity of the token time is checked.
\\
If the bind token was sent by an association responder,
its random number
remains for further checks and actions of the calling function.
\\
The subfunction af\_verify() is called.
The return value is 0, if the given BindToken is valid,
if the encrypted data can be decrypted,
and if the signature is  verified.
The return value is -1 otherwise;
the error code would be ETOKSIGN, ETOKRANDOM or ETOKTIME
or one inherited by one of the called subfunctions.

\subsubsection{x400\_check\_Context}
\nm{3X}{x400\_check\_Context}{Check labels of security context}
\label{x4_ck_Scon}
\hl{Synopsis}
\#include $<$x400.h$>$ \\ [1ex]
RC {\bf x400\_check\_Context} {\em (seccont)} \\
SecurityContext {\em *seccont};
\hl{Parameter}
\parname  {{\em seccont}}
\pardescript  {Input: Pointer to a SecurityContext structure defined by calling function.
	       All subfields are defined and filled
	       by the calling function.} \\[1ex]
\parname  {{\em Return Value}}
\pardescript  {0 or -1}
\hl{Description}
X400\_check\_Context reads through all labels of the security context
and checks, if they make up an acceptable security context.
In case of the simple security policy with no confidentiality
service (\{ id-dfn-ssp \}),
only one label is allowed,
which contains the identifier of the simple security policy only.
\\
All labels' subfields
remain for further checks and actions of the calling function.
\\
The subfunction x400\_check\_Label() is called
(however not necessary for the simple security policy).
The return value is 0, if the given SecurityContext is valid.
The return value is -1 otherwise;
the error code would be ESECCONTEXT.

\subsubsection{x400\_check\_Credentials}
\nm{3X}{x400\_check\_Credentials}{Check sender's bind token
and originator certificate}
\label{x4_ck_Cred}
\hl{Synopsis}
\#include $<$x400.h$>$ \\ [1ex]
RC {\bf x400\_check\_Credentials} {\em (creds)} \\
Credentials {\em *creds};
\hl{Parameter}
\parname  {{\em creds}}
\pardescript  {Input: Pointer to a Credentials structure defined by calling function.
	       All subfields are defined and filled
	       by the calling function.} \\[1ex]
\parname  {{\em Return Value}}
\pardescript  {0 or -1}
\hl{Description}
In its full functionality,
x400\_check\_Credentials would check the sender's bind token in the
context of the sender's originator certificate.
However, this specification is based upon the fact,
that prior to the very first association between interacting X.400 agents,
the originator certificates are mutually exchanged and stored.
Therefore, certificates are not supposed to be included
in the credentials exchanged during association establishment.
In this case,
x400\_check\_Credentials checks only the included bind token
and verifies its signature on the base of locally known public key
informations (e.g. from PSE object PKList).
\\
Subfunctions called are x400\_check\_BindToken() and af\_verify().
The return value is 0, if the given Credentials is valid.
The return value is -1 otherwise;
the error code would be ECREDKEY
or ECREDCERT, if Certificates is checked as well,
or one inherited by one of the called subfunctions.

\subsubsection{x400\_include\_*, x400\_extract\_*}
\nm{3X}{x400\_in/ex}{Security elements included in or extracted from protocol elements}
\label{x4_in_ex}
\hl{Synopsis}
\#include $<$x400.h$>$ \\ [1ex]
RC *{\bf x400\_include\_Ocert} {\em (env, ocert)} \\
PRESEL *{\em env}; \\
Certificates *{\em ocert}; \\ [1ex]
RC *{\bf x400\_extract\_Ocert} {\em (env, ocert)} \\
PRESEL *{\em env}; \\
Certificates *{\em ocert}; \\ [1ex]
RC *{\bf x400\_include\_Label} {\em (env, label)} \\
PRESEL *{\em env}; \\
Label *{\em label}; \\ [1ex]
RC *{\bf x400\_extract\_Label} {\em (env, label)} \\
PRESEL *{\em env}; \\
Label *{\em label}; \\ [1ex]
RC *{\bf x400\_include\_MAC} {\em (env, mac)} \\
PRESEL *{\em env}; \\
MAC *{\em mac}; \\ [1ex]
RC *{\bf x400\_extract\_MAC} {\em (env, mac)} \\
PRESEL *{\em env}; \\
MAC *{\em mac}; \\ [1ex]
RC *{\bf x400\_include\_MsgToken} {\em (env, token)} \\
PRESEL *{\em env}; \\
MsgToken *{\em token}; \\ [1ex]
RC *{\bf x400\_extract\_MsgToken} {\em (env, token)} \\
PRESEL *{\em env}; \\
MsgToken *{\em token}; \\
\hl{Description}
The x400\_include\_* functions above
have {\em two} input parameters:
The first parameter is a pointer to a protocol presentation element
(in terms of the given X.400 implementation,
\speconly{(see introductional remarks of section \ref{x4_fc})}
).
This presentation element
represents the {\em message envelope}
as far as it has been built by the calling function.
The second parameter is a pointer to the security element
which is to be included into the message envelope.
After return from an x400\_include\_* function,
the respective security element is included in the
message envelope presentation element.
The returncode is 0 on success, and -1 on failure.
\\[1ex]
The x400\_extract\_* functions above
have {\em two} input parameters:
The first parameter is a pointer to a protocol presentation element
(in terms of the given X.400 implementation,
\speconly{(see introductional remarks of section \ref{x4_fc})}
).
This presentation element
represents the {\em message envelope}
as far as it is to be parsed by the calling function.
The second parameter is a pointer to the security element structure
which is defined by the calling program.
After return from an x400\_extract\_* function,
the subfields of
the respective security element are filled
with the respective security informations,
that the called function has parsed from the given
message envelope presentation element.
The returncode is 0 on success, and -1 on failure.
\\[1em]
\hl{Synopsis}
\#include $<$x400.h$>$ \\ [1ex]
RC *{\bf x400\_include\_Credentials} {\em (bind\_arg, creds)} \\
PRESEL *{\em bind\_arg}; \\
Credentials *{\em creds}; \\ [1ex]
RC *{\bf x400\_extract\_Credentials} {\em (bind\_arg, creds)} \\
PRESEL *{\em bind\_arg}; \\
Credentials *{\em creds}; \\ [1ex]
RC *{\bf x400\_include\_Context} {\em (bind\_arg, seccont)} \\
PRESEL *{\em bind\_arg}; \\
SecurityContext *{\em seccont}; \\ [1ex]
RC *{\bf x400\_extract\_Context} {\em (bind\_arg, seccont)} \\
PRESEL *{\em bind\_arg}; \\
SecurityContext *{\em seccont}; \\
\hl{Description}
The x400\_include\_* functions above
have {\em two} input parameters:
The first parameter is a pointer to a protocol presentation element
(in terms of the given X.400 implementation,
\speconly{(see introductional remarks of section \ref{x4_fc})}
).
This presentation element
represents the {\em bind argument}
as far as it has been built by the calling function.
In case of the Credentials, the presentation element
represents either the bind {\em argument}
or the bind {\em result}.
The second parameter is a pointer to the security element
which is to be included into the bind argument.
After return from an x400\_include\_* function,
the respective security element is included in the
bind argument presentation element.
The returncode is 0 on success, and -1 on failure.
\\[1ex]
The x400\_extract\_* functions
have {\em two} input parameters:
The first parameter is a pointer to a protocol presentation element
(in terms of the given X.400 implementation,
\speconly{(see introductional remarks of section \ref{x4_fc})}
).
This presentation element
represents the {\em bind argument}
as far as it is to be parsed by the calling function.
In case of the Credentials, the presentation element
represents either the bind {\em argument}
or the bind {\em result}.
The second parameter is a pointer to the security element structure
which is defined by the calling program.
After return from an x400\_extract\_* function,
the subfields of
the respective security element are filled
with the respective security informations,
that the called function has parsed from the given
bind argument presentation element.
The returncode is 0 on success, and -1 on failure.

\subsection{Encoding and Decoding Functions}
\nm{3X}{e\_X400}{Encoding and decoding functions for X400-IF}
\label{x4_encdec}
\hl{Synopsis}
\#include $<$x400.h$>$ \\ [1ex]
OctetString *{\bf e\_MACTBS} {\em (mactbs)} \\
MACTBS *{\em mactbs}; \\ [1ex]
OctetString *{\bf e\_TokenTBE} {\em (toktbe)} \\
TokenTBE *{\em toktbe}; \\ [1ex]
OctetString *{\bf e\_MsgTokenTBS} {\em (mtoktbs)} \\
MsgTokenTBS *{\em mtoktbs}; \\ [1ex]
OctetString *{\bf e\_BindTokenTBS} {\em (btoktbs)} \\
BindTokenTBS *{\em btoktbs}; \\ [1ex]
RC *{\bf x400\_include\_*} {\em (presel, c\_struct)} \\
PRESEL *{\em presel}; \\
$<$C\_STRUCT$>$ *{\em c\_struct}; \\ [1ex]
RC *{\bf x400\_extract\_*} {\em (presel, c\_struct)} \\
PRESEL *{\em presel}; \\
$<$C\_STRUCT$>$ *{\em c\_struct}; \\ [1ex]

\subsection{X.400 Security Elements}
\nm{4}{X.400 Security Elements}{X.400 Security Elements}
\label{x4_se}
\hl{Synopsis}
\#include $<$x400.h$>$ \\ [1ex]
\hl{Description}     
This section describes the security elements of X.400-1988
in terms of objects
which the X.400 security module
has access to in order to perform its functionality.
The objects are stored either in the PSE or in
a directory information base,
or they are created by functions of this module.
Each object is presented in terms
of C-language structures as they are needed by the interface subroutines and
by the application programmer as well. The bit coding of the objects
is given by applying ASN.1 Basic Encoding Rules
to the corresponding ASN.1 structures as they are defined in
\speconly{chapter~\ref{asn1-x400}}
\manonly{Vol. 2, 2.1}
. Transformation between those two representations
is provided by the respective encoding/decoding functions
\speconly{of chapter~\ref{x4_encdec}}
\manonly{e\_*(3X)}
or by locally implemented include/extract functions
\speconly{(see introductional remarks of section \ref{x4_fc})}
.

\subsubsection{Originator Certificate}
\hll{x400\_OCert}
\label{x4_ocert}
{\em Name of the object:} {\bf Certificates} \\
{\em Content of the object:} Originator certificate (of type Certificates). \\
{\em Structure of the object:}

{\small
\btab
\1 typedef struct Certs \{ \\
\2    Certificate \2  *usercertificate; \\
\2    FCPath      \2  *forwardpath; \\
\1    \} Certificates; \\
\etab
}

The {\em X.400 originator certificate} is defined to have the
same structure as {\em Certificates}
(see AF-Module).

\subsubsection{Label}
\hll{x400\_Label}
\label{x4_Label}
{\em Name of the object:} {\bf Label} \\
{\em Content of the object:} {\bf Security label informations} \\
{\em Structure of the object:}

{\small
\btab
\1 typedef struct label \{ \\
\2      ObjId \2  policy; \\
\2      int  \2        *class;  \\
\2      char \2        *mark;   \\
\2      char \2        *categs; /* address of security categories */ \\
\1 \} Label; \\
\etab
}

Encoding/Decoding: x400\_include\_Label(), x400\_extract\_Label() \\
Used by: x400\_create\_Label(), x400\_check\_Label()

\subsubsection{Message Authentication Check}
\hll{x400\_MAC}
\label{x4_MAC}
{\em Name of the object:} {\bf MAC, MIC} \\
{\em Content of the object:} Informations for message authentication check. \\
{\em Structure of the object:}

{\small
\btab
\1 typedef struct mac \{ \\
\2      OctetString \2  *tbs\_DERcode; \\
\2      MACTBS    \2    *tbs; \\
\2      Signature \2    *sig; /* sender's signature */ \\
\1 \} MAC; \\ \\
\1 typedef struct mac MIC; \\ \\
with: \\ \\
\1 typedef struct macTBS \{  \\
\2     AlgId       \2   *signatureAI; \\
\2     OctetString \2   *content; \\
\2     char        \2   *contend\_id; \\
\2     Label       \2   *label; \\
\1 \} MACTBS; \\
\etab
}

Note, that this object can also be used for the message integrity check,
in that contend\_id and label are set to NULL,
and the element is included in the message token signed data.
\\
Encoding/Decoding: e\_MACTBS(), x400\_include\_MAC(), x400\_extract\_MAC() \\
Used by: x400\_create\_MAC(), x400\_check\_MAC()

\subsubsection{Message Token}
\hll{x400\_MTOC}
\label{x4_MTOC}
{\em Name of the object:} {\bf MsgToken} \\
{\em Content of the object:} Informations for message token. \\
{\em Structure of the object:}

{\small
\btab
\1 typedef struct mtoken \{ \\
\2      ObjId \2  *token\_type; /* id-tok-asymmetricToken */ \\
\2      OctetString \2  *tbs\_DERcode; \\
\2      MsgTokenTBS \2  *tbs; \\
\2      Signature \2    *sig; /* sender's signature */ \\
\1 \} MsgToken; \\ \\
with: \\ \\
\1 typedef struct mtokenTBS \{  \\
\2     AlgId     \2   *signatureAI; \\
\2     DName      \2   *recipient;   \\
\2     UTCTime   \2   *time;        \\
\2     AlgId     \2   *confidAI;    \\
\2     MIC       \2   *cic;         \\
\2     Label     \2   *label;       \\
\2     int       \2   *pod\_req;    \\
\2     int       \2   *mseq;        \\
\2     AlgId     \2   *encryptionAI;\\
\2     ENCRYPTED \2   *encrypted; /* from TokenTBE */  \\
\1 \} MsgTokenTBS; \\ \\
and: \\ \\
\1 struct tokenTBE \{ \\
\2     BitString \2   *cckey; \\
\2     MIC       \2   *cic;   \\
\2     Label     \2   *label; \\
\2     BitString \2   *cikey; \\
\2     int       \2   *mseq;  \\
\1 \} TokenTBE; \\
\etab
}

Encoding/Decoding: e\_MsgTokenTBS(), e\_TokenTBE(),
x400\_include\_Msg\-To\-ken(), x400\_extract\_MsgToken() \\
Used by: x400\_create\_MsgToken(), x400\_check\_MsgToken()

\subsubsection{Bind Token}
\hll{x400\_BTOC}
\label{x4_BTOC}
{\em Name of the object:} {\bf BindToken} \\
{\em Content of the object:} Informations for bind token. \\
{\em Structure of the object:}

{\small
\btab
\1 typedef struct btoken \{ \\
\2      ObjId  \2 *token\_type; /* id-tok-asymmetricToken */ \\
\2      OctetString  \2 *tbs\_DERcode; \\
\2      BindTokenTBS \2 *tbs; \\
\2      Signature \2    *sig; /* sender's signature */ \\
\1 \} BindToken; \\ \\
with: \\ \\
\1 typedef struct btokenTBS \{  \\
\2     AlgId     \2   *signatureAI; \\
\2     DName      \2   *recipient;   \\
\2     UTCTime   \2   *time;        \\
\2     BitString \2   *random;      \\
\2     AlgId     \2   *encryptionAI;\\
\2     ENCRYPTED \2   *encrypted; /* from TokenTBE */  \\
\1 \} BindTokenTBS; \\ \\
and: \\ \\
\1 struct tokenTBE \{ \\
\2     BitString \2   *cckey; \\
\2     MIC       \2   *cic;   \\
\2     Label     \2   *label; \\
\2     BitString \2   *cikey; \\
\2     int       \2   *mseq;  \\
\1 \} TokenTBE; \\
\etab
}

Encoding/Decoding: e\_BindTokenTBS(), e\_TokenTBE(),
x400\_include\_Bind\-To\-ken(), x400\_extract\_BindToken() \\
Used by: x400\_create\_BindToken(), x400\_check\_BindToken()

\subsubsection{Security Context}
\hll{x400\_SecCont}
\label{x4_SecCont}
{\em Name of the object:} {\bf SecurityContext} \\
{\em Content of the object:} Set of security labels. \\
{\em Structure of the object:}

{\small
\btab
\1 typedef sec\_cont \{ \\
\2      Label \2  *label; \\
\2      struct sec\_cont \2 *next; \\
\1 \} SecurityContext; \\
\etab
}

In the following example,
a loop over the labels in this C-structure is demonstrated:

{\small
\btab
\1 /* example use: */   \\ \\
\2      SecurityContext \2 *cont; \\
\2      Label           \2 *lab;  \\
\2      for( cont = /* given security context */ ; \\
\3         cont != (SecurityContext *)0 ; \\
\3         cont = cont$->$next ) \\
\2         \{ /* work on */ lab = cont$->$label; \} \\
\etab
}

Encoding/Decoding: x400\_include\_Context(), x400\_extract\_Context() \\
Used by: x400\_create\_Context()

\subsubsection{Strong Credentials}
\hll{x400\_Creds}
\label{x4_Creds}
{\em Name of the object:} {\bf Credentials} \\
{\em Content of the object:} Bind token and originator certificate. \\
{\em Structure of the object:}

{\small
\btab
\1 typedef credents \{ \\
\2      BindToken \2  *btoken; \\
\2      Certificates \2 *ocert; \\
\1 \} Credentials; \\
\etab
}

Encoding/Decoding: x400\_include\_Credentials(),
x400\_extract\_Credentials() \\
Used by: x400\_create\_Credentials()

\speconly{
\subsection{Error Handling}
\label{x4_fehler}
In case of an error ({\em Return Value} -1, if return parameter type  RC, or NULL pointer else),
the error cause is indicated in the global variable {\em af\_error} (data type AF\_Error).
{\em Af\_error.number} can have the following values: \\ [0.5cm]
\parname  {EALGID}
\pardescript  {Invalid Algorithm Identifier} \\
\parname  {EAPP}
\pardescript  {Cannot select application} \\
\parname  {EAPPNAME}
\pardescript  {Application name does not exist} \\
\parname  {ECREATEAPP}
\pardescript  {Cannot create application (e.g. application name already exists)} \\
\parname  {EOBJ}
\pardescript  {Cannot select object} \\
\parname  {EOBJNAME}
\pardescript  {Object does not exists} \\
\parname  {ECREATEOBJ}
\pardescript  {Cannot create object (e.g. object already exists)} \\
\parname  {EPIN}
\pardescript  {Invalid PIN} \\
\parname  {EVERIFICATION}
\pardescript  {Verification unsuccessful} \\
\parname  {ESYSTEM}
\pardescript  {System call failed inside routine} \\
\parname  {EDAMAGE}
\pardescript  {Toc of PSE not readable or PSE inconsistent} \\
\parname  {EMALLOC}
\pardescript  {Unable to allocate new memory} \\
\parname  {EENCRYPT}
\pardescript  {Wrong state of sec\_encrypt} \\
\parname  {EDECRYPT}
\pardescript  {Wrong state of sec\_decrypt} \\
\parname  {EHASH}
\pardescript  {Wrong state of sec\_hash} \\
\parname  {ESIGN}
\pardescript  {Wrong state of sec\_sign} \\
\parname  {EENCODE}
\pardescript  {ASN.1-encoding error} \\
\parname  {EDECODE}
\pardescript  {ASN.1-decoding error} \\
\parname  {EINVALID}
\pardescript  {Invalid argument} \\
\parname  {EVALIDITY}
\pardescript  {Invalid validity date of certificate} \\
\parname  {EPK}
\pardescript  {PK already exists in PKList} \\
\parname  {ENAME}
\pardescript  {Name already exists in PKList} \\
\parname  {ENOPK}
\pardescript  {PK not found in PKList} \\
\parname  {ENONAME}
\pardescript  {Name not found in PKList} \\
\parname  {EROOTKEY}
\pardescript  {Highest verification key not available} \\
\parname  {ENODIR}
\pardescript  {Directory service does not respond} \\
\parname  {ENAMEDIR}
\pardescript  {No object assigned to this name found in directory} \\
\parname  {EASCDIR}
\pardescript  {Directory access rights not sufficient for requested
	       operation} \\
\parname  {EATTRDIR}
\pardescript  {No directory entry of requested attribute type
	       found in directory} \\
\parname  {ELABEL}
\pardescript  {Security label inconsistent or invalid} \\
\parname  {EPOLICY}
\pardescript  {Unknown security policy} \\
\parname  {ENOTOKEN}
\pardescript  {According to message security label
	       no message token required;
	       or message sequence number missing on input} \\
\parname  {ERECIPNAME}
\pardescript  {Recipient name missing or invalid} \\
\parname  {ERECIPKEY}
\pardescript  {Recipient public key not available} \\
\parname  {ESECCONTEXT}
\pardescript  {Unacceptable security context} \\
\parname  {EMAC}
\pardescript  {Message authentication check (verification) failed} \\
\parname  {ETOKENCRYPT}
\pardescript  {Cannot decrypt (message or bind) token} \\
\parname  {ETOKSIGN}
\pardescript  {(Message or bind) token verification failed} \
\parname  {ETOKRANDOM}
\pardescript  {Unacceptable random number in bind token} \\
\parname  {ETOKTIME}
\pardescript  {Unacceptable (message or bind) token time} \\
\parname  {ECREDCERT}
\pardescript  {Credential certificates failure. The certificate of concern
	       is returned in {\em af\_error.cert}} \\
\parname  {ECREDKEY}
\pardescript  {Unverifiable partner public key in
	       bind token of credentials} \\[0.5cm]
In case of {\em af\_error.number} $=$ ESYSTEM,  {\em errno} of the failed system call is given
in {\em af\_error.data}. {\em af\_error.addr} contains an explanatory text (which can be printed
to {\em stderr} with function {\em af\_perror()}).
}
