\section{Signature and Certificate}
\markboth{Signature and Certificate}{Signature and Certificate}
\thispagestyle{myheadings}
\label{sigandcert}

\subsection{Digital Signature and Certification}
\label{sc-dsc}

The signature algorithm employed within
SecuDe
is a composition of a hash function followed by an RSA function
\cite{riv1} \cite{ttt1} \cite{ttt2} \cite{ttt3}.
The key used for signature is the signer's private key,
The signer's public key which is used for the verification of the signature
is certified by a certification authority.
For encryption the DES algorithm is used,
whereby, for the purpose of transfer, the secret DES-key is RSA-encrypted.
For every user,
the pair of RSA-keys used for encryption and decryption is
different from the pair of RSA-keys used for signature and verification.
Identification of algorithms and
the model of certification are following the
``Directory Authentication Framework'', X.509 \cite{cci4}.

\subsection{Elements of a Digital Signature}
\label{sc-eds}

In SecuDe
a Digital Signature goes always within a quadruplet
({\em text, signature, creation date, ori\-gi\-na\-tor cer\-ti\-fi\-cate}).

\subsubsection{Text and Signature}
\label{sc-ts}

The Digital Signature is a sequence of a signature algorithm identifier
and a bitstring containing the result of RSA encryption
of a hashed version of the text
by applying the signer's private RSA key.
The digital signature is a function of the algorithm used,
of the signed text,
and of the secret key (and hence of the owner of the secret key).

\subsubsection{Creation Date}
\label{sc-cd}

Neither the algorithms of asymmetric encryption nor the Directory
Authentication Framework demand
a time or date value for the signature process.
It is an additional request to link a signature with a date
which is being interpreted as
{\em creation date} of the signature.
This specification does not say, {\em where} the date is to be got from.
It just says, there must be a date.
(In MHS-X.400 the date can always be taken from the envelope).
The {\em default}-date is the date of signature verification.

\subsubsection{Certificate and Certification Tree}
\label{sc-cct}

From the algorithmic point of view, a public key certificate is
not a necessary part of a digital signature.
A digital signature can be verified without certificate,
if the public key is available by other means,
for example from a public directory or a local smartcard.
In order to verify a signature up to the verifier's point of trust,
a trustworthy chain of certificates is required.
That is,
the first certificate of the chain certifies the public key of the signature,
each other certificate of the chain certifies the public key
of its preceding certificate,
and the public key of the last certificate of the chain
belongs to a certification authority which the verifier trusts.
A trustworthy chain of certificates between a signer and a verifier
is called a certification path.
The complexity of the structure of all
certification paths between any two users
is comparable with the routing topology of store and forward
mail systems.

Within SecuDe a simple standard path through the
X.509 certification trees is anticipated:
all users either belong to {\em one tree},
or there is one {\em cross certificate} between the paths
between the users and their root certification authorities.

\begin{center}
\makebox[5.541in][l]{
  \vbox to 2.680in{
    \vfill
    \special{psfile=vol1-fig2.ps}
  }
  \vspace{-\baselineskip}
}
\end{center}
\label{fig-sc-cct}
\stepcounter{Abb}
{\footnotesize Fig.\arabic{Abb}: Two certification trees linked by cross certificates}
 
\subsubsection{Originator Certificate (certificates)}
\label{sc-oc}

X.411 defines the originator certificate as
X.509 type {\em certificates}, note the plural.
Within SecuDe an originator certificate
is used as follows:
Every user owns one originator certificate,
the lowest certificate of which is his user certificate and the
highest certificate of which is signed by the root certification
authority of his certification tree.

The user certificate is the lowest certificate of the path
between signer and the root certification authority of his
certification tree. This means:
The owner ({\em subject}) of the user certificate
is the signer of the text.
The {\em issuer} of
the user certificate
is the signer's certification authority.

The levels of the forward certification path are the levels of the
signer's certification tree.
The very certification authority representing one level of the path
between the signer and his root certification authority
is the owner ({\em subject}) of all certificates at this level.
At any level
exactly one certificate is its {\em hierarchy certificate},
i.e. its {\em issuer} is the certification authority at the next level above.
All other certificates are cross certificates, i.e. their issuers
are all different.

The hierarchy certificate is mandatory, the cross certificates are optional.
In the simplest case a SET OF \{certificate\} at one level contains
exactly one certificate.

The hierarchy certificate is identified from
the cross certificates in the SET not by its position, but
by the {\em name of issuer} which is the name of the subject
of every certificate at the next level above.
The purpose of cross certificates is to link different
certification trees or to be short cuts within one certification tree.
\newpage
\begin{figure}
\begin{center}
\makebox[5.166in][l]{
  \vbox to 4.750in{
    \vfill
    \special{psfile=vol1-fig3.ps}
  }
  \vspace{-\baselineskip}
}
\end{center}
\label{fig-sc-oc}
\stepcounter{Abb}
{\footnotesize Fig.\arabic{Abb}: Originator-certificate of X.509-type
``certificates''}
\end{figure}

\subsection{Certificate: Formats and Values}
\label{sc-cfv}

In the {\em Authentication Framework} X.509 \cite{cci4} the type
{\em Certificate} is defined like this:

\begin {center}
\begin {tabular}{lll}
Certificate ::= & {\bf SIGNED SEQUENCE} \{ & \\
  & version [0]   & Version {\bf DEFAULT} 0, \\
  & serialNumber  & CertificateSerialNumber, \\
  & signature     & AlgorithmIdentifier,     \\
  & issuer        & Name,                    \\
  & validity      & Validity,                \\
  & subject       & Name,                    \\
  & subjectPublicKeyInfo & SubjectPublicKeyInfo \} \\
\end {tabular}
\end {center}
\label{fig-sc-cfv}
\stepcounter{Abb}
{\footnotesize Fig.\arabic{Abb}: X.509-type ``Certificate''}
\\[1ex]
{\em Version} and {\em CertificateSerialNumber} (see below)
are of type INTEGER,
{\em issuer} is the issuing certification authority
and {\em subject} is the owner of the certificate.
{\em Validity} is a sequence of two elements of type UTCTime,
{\em not\-Be\-fore} und {\em not\-Af\-ter} (see below).
{\em SubjectPublicKeyInfo} contains
the public key of the {\em subject}, while
the {\em certificate} itself (SIGNED SEQUENCE!)
is signed by its  {\em issuer}.
In X.509 it is not specified, which signature the
{\em AlgorithmIdentifier} of the field {\em signature} is related to.
In SecuDe
it is related to the certificate signature of the {\em issuer},
which is therefore present twice: inside and outside of the
SIGNED SEQUENCE.
This will be considered on creation but not on evaluation of a certificate.

\subsubsection{Name Format}
\label{sc-nf}
 
Issuer and subject of a certificate are identified by their names.
The name format is defined in the Directory Information Model
X.501 \cite{cci3}.
The {\em directory attribute ``Name''}
is a sequence of {\em ``Relative Distingushed Names''}
such as a country name, organizational name or personal name.
Within SecuDe
only a limited set of attribute types is used to construct relative distinguished names,
these are:
countryName, organizationalName, organizationalUnitName, commonName, surname,
title, serialNumber, localityName, streetAddress, and businessCategory.
All other attribute types would be ignored.

The {\em printable representation} of a name follows the pattern
\bc
{\small keyword=value \% keyword=value ... ; keyword=value \% keyword = value ...; ...}
\ec
keyword is a name for an attribute type. 
The ';' delimiter separates the RDN members of the DN, i. e. each 
keyword=value \% keyword=value ... expression denotes one relative distinguished name.
The '\%' delimiter separates the attributes within one RDN.
The following keywords, representing X.520 defined attributes,  are possible:
{\small
\bvtab
C   \2 countryName \\
O   \2 organizationName  \\
OU  \2 organizationalUnitName \\
S   \2 surname \\
CN  \2 commonName \\
L   \2 localityName \\
SP  \2 stateOrProvinceName \\
ST  \2 streetAddress \\
T   \2 title \\
SN  \2 serialNumber \\
BC  \2 businessCategory \\
D   \2 description
\evtab
}
RDNs must be ordered top down. Letter cases are not significant for names, 
as well as spaces may be used inside and outside the names.
\\ [1ex]
One of the authors' name would be represented as
{\small
C=DE; O=gmd; OU=I2 \% CN=Institut fuer TeleKooperationsTechnik; CN=Wolfgang Schneider} 

\subsubsection{Certificate Serial Number}
\label{sc-csn}
 
The data element {\em certificate serial number} of a certificate
contains an integer number. In the directory black lists
this number in combination with the name of the issuer of a certificate
identifies a certificate universally.
As a consequence, a certification authority has to make sure,
that serial numbers of all certificates issued by her are unique.
A combination of time stamp values and random numbers,
or continuously ascending numbers would fulfill this condition.
Within SecuDe
another approach is used:
a certification authority would number its public keys
and it would number all certificates signed by one signature key.
The concatenation of these two numbers makes the certificate serial number.

A certificate serial number has the decimal representation $xyyyyyy$.
$x=0,\ldots$ is any (one or multi digit) number
which is uniquely related to the signature key of the issuing certification
authority used to sign this certificate.
$yyyyyy=0,\ldots,999999$ is a number smaller than 1 million, which
is uniquely related to the certificate,
signed by the certification authority with key $x$.
The first part $x$ is called {\em key number},
the second part $yyyyyy$ is called {\em certificate number} of the
certificate.

For example, the certificate serial number 1083
belongs to the 1083th certificate which was signed
by the 0th signature key of its issuer.
The certificate serial number 27080595
belongs to the 80595th certificate which was signed
by the 27th signature key of its issuer.

\subsubsection{Validity}
\label{sc-v}

The data element {\em validity} consists of the two UTCTime elements
{\em validity-notBefore} and {\em validity-notAfter}.
Outside of this time interval a certified signature is invalid.
However, a certificate, which is signed itself, does not indicate
the date or time of its creation. Therefore, the time of the creation
of a certificate's signature is unknown. It must be clarified,
in which way the validity interval of a certificate is restricted
by the validity interval of its signature, that is, by the validity interval
of its superior certificate.
In SecuDe the {\em validity} has this significance:

\begin{itemize}
\item  validities of certificates within one originator certificate
are not {\em directly} related to one another, {\em but:}
\item A signature is only valid,
if its creation date is within the validity of every certificate
of the originator certificate.
Formally:
\item The validity time frame of a signature is the intersection
of all validity time frames of the originator certificate.
\end{itemize}

\subsubsection{validity-notBefore}
\label{sc-vb}

If a signature's creation date is earlier than
the value of {\em validity-notBefore} of its certificate,
it is to be regarded as invalid.
The value of {\em validity-notBefore} is the creation date
of the certified public key (not of the certificate!).
It does not prevent from a manipulated signature creation date
as long as it id placed later than the validity-notBefore value.
However, this value can be of informational interest.

\subsubsection{validity-notAfter}
\label{sc-va}

If a signature's creation date is later than
the value of {\em validity-notAfter} of its certificate,
it is to be regarded as invalid.
The value of {\em validity-notAfter} marks the day
of expiry of the certificate (not of the signature!).
Note, that signatures created before validity-notAfter remain valid
after the real date validity-notAfter has passed.
A new certificate (validity-notAfter $>$ old validity-notAfter)
may specify the same public key information in order to {\em extend}
the validity of a signature key
as long as the rules of the given security policy are obeyed.
The same is true for certificates of encryption keys.

The element {\em validity-notAfter}
does not prevent from the manipulated backwards dating of a signature,
as long as it is placed earlier than the validity-notAfter value.
It does, however, prevent
from the usage of expired signature keys.
Therefore,
it is strongly recommended to integrate the check of this
value into the signature verification process.
