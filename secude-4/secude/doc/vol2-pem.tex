\subsection{Privacy Enhanced Mail Functions} 

\subsubsection{pem\_create}
\nm{3X}{pem\_create}{Generate a PEM text}
\label{pem_create}
\hl{Synopsis}
\#include $<$pem.h$>$ 

RC {\bf pem\_create} {\em (info, in, pem)} \\
OctetString {\em *in, *pem}; \\
PemInfo {\em *info};
\hl{Input-Parameter}
\parname  {\em info}
\pardescript {Signature and encryption control informations.} \\
\parname  {\em in}
\pardescript {Cleartext created by user, subject to
		  privacy enhanced services.
		  It may contain embedded (multiple) PEMs.}
\hl{Output-Parameter}
\parname  {\em pem}
\pardescript {RFC 1421 specified text between the
		  privacy enhanced message encapsulation boundary lines
		  (included).
		  The OctetString structure of {\em pem} is
		  created by the calling function.
		  However, the character string in {\em pem}
		  is allocated by {\em pem\_create}.} 

\parname  {\em Returncode RC}
\pardescript {negative if erraneous, 0 if OK.}
\hl{Description}
pem\_create evaluates the structure PemInfo (info) and the cleartext (in),
and creates the PEM text (pem):
the resulting text is the RFC 1421 specified text between the
privacy enhanced message encapsulation boundary lines (included).
The signature and encryption control informations
stored in a structure PemInfo serve as input to the pem\_create function.
The storage of these control informations into a structure PemInfo
is supported by pem\_cinfo, which can be performed
before the call to pem\_create.
Alternatively, pem\_cinfo can be called as first step inside of
pem\_create.
pem\_create performs the following steps:

{\small
\parname  {\em pem\_cinfo}
\pardescript {optional: prepares PemInfo for input to the other
		    pem\_c* subroutines. Can also be called prior
		    to pem\_create} 


\parname  {\em pem\_chd}
\pardescript {write any of the encapsulated header titles
		    (called by the other pem\_c* functions)}

\parname  {\em pem\_cinit}
\pardescript {create pem\pf octets and put Pre-EB and Proc-Type header into it} \\
\parname  {\em pem\_cdek}
\pardescript {append DEK-Info header to pem\pf octets} \\
\parname  {\em pem\_csend}
\pardescript {append Originator-ID, Certificate,
		    Issuer-Certificate header} \\
\parname  {\em pem\_sign}
\pardescript {produce MIC: canonicalize msg body {\em in} and apply
		    digital signature to it
		    (use MIC production defined by RFC 1421)}

\parname  {\em pem\_cmic}
\pardescript {(encrypt and) encode MIC and append MIC-Info} \\
\parname  {\em pem\_crec}
\pardescript {append Recipient-ID, Key-Info
		    (recipient keys in structure PemInfo are already encrypted,
		    e.g. by pem\_cinfo)} 


\parname  {\em CR--LF}
\pardescript {append empty line (local form: one extra line feed)} \\
\parname  {\em pem\_cbody}
\pardescript {canonicalize, encrypt and encode msg body (from {\em in})
		    and append} \\
\parname  {\em pem\_cend}
\pardescript {append Post-EB as last line}
}

\subsubsection{pem\_scan}
\nm{3X}{pem\_scan}{Scan a PEM text}
\label{pem_scan}
\hl{Synopsis}
\#include $<$pem.h$>$ 

RC {\bf pem\_scan} {\em (info, set\_of\_pemcrlwithcerts, issuer, cbody, pem, no\_path\_verification)} \\
PemInfo {\em *info}; \\
SET\_OF\_PemCrlWithCerts {\em **set\_of\_pemcrlwithcerts}; \\
SET\_OF\_DName 	{\em **issuer}; \\
OctetString {\em *cbody, *pem}; \\
Boolean {\em *no\_path\_verification};
\hl{Input-Parameter}
\parname  {\em pem}
\pardescript {RFC 1421 specified text between the
		    privacy enhanced message encapsulation boundary lines
		    (included).
		    If an embedded multi-level PEM is to be scanned,
		    the first Pre-EB of the pem OctetString must be
		    the Pre-EB of the embedded PEM to be scanned.
		    pem\_sbody will automatically find the corresponding
		    Post-EB (counting Pre-EBs up, and Post-EBs down).}

\parname  {\em no\_path\_verification}
\pardescript {If set to TRUE the forward certification path is not verified.}
\hl{Output-Parameter}
\parname  {\em info}
\pardescript {Signature and encryption control information.
		    The {\em PemInfo} structure of {\em info}
		    is to be allocated by the calling function.
		    However, the pointers below {\em info}:
		    {\em encryptKEY}, {\em origcert}, {\em signAI},
		    and {\em recplist}, are created by {\em pem\_scan}.
		    The informations stored into {\em info} are a result of
		    scanning the encapsulated header fields in {\em pem}.} 

\parname  {\em cbody}
\pardescript {Cleartext created by {\em pem\_scan} as result
		    of scanning the input parameter {\em pem}
		    and applying the control informations found in the
		    encapsulated header fields on the message body.
		    The OctetString structure of {\em out} is created by
		    the calling function.
		    The character string in {\em out}, however,
		    is allocated by {\em pem\_scan}.} 

\parname  {\em set\_of\_pemcrl...}
\pardescript {Structure to store CRLs in case of Proc-Type CRL.} \\
\parname  {\em issuer}
\pardescript {Structure to store issuer names in case of Proc-Type CRL-RETRIEVAL-REQUEST.} \\
\parname  {\em Returncode RC}
\pardescript {negative if erraneous, 0 if OK.} \\
\hl{Description}

pem\_scan evaluates the PEM text (pem) and fills structure PemInfo (info)
and creates the cleartext (out) by scanning the encapsulated header fields and
decrypting the message body.
The message body is decrypted with the help of a DES-key
which is contained in the recipient-related Key-Info header fields,
however RSA-encrypted.
In order to find ``my'' corresponding Key-Info field,
the input parameter ``my name'' is used
to find ``my'' corresponding Recipient-ID header field.
If the input parameter ``myname'' is not set,
pem\_scan evaluates the caller's encryption certificate from the PSE.
\\
The returncode tells, if the MIC was correct.
pem\_scan performs the following steps:

{\small
\parname  {\em pem\_find\_delim}
\pardescript {find first empty line (CR--LF--CR--LF or, in local form LF--LF)
		    which is interpreted as delimiting line between
		    encapsulated header portion and PEM body.} 

\parname  {\em pem\_get\_header}
\pardescript {scan and interpret the header field} \\
\parname  {\em af\_decrypt}
\pardescript {if encrypted, decrypt the body of the message} 

\parname  {\em aux\_canon}
\pardescript {canonicalize the plaintext msg body} 

\parname  {\em af\_verify}
\pardescript {verify the MIC and the forward certification path} 

\parname  {\em aux\_add\_alias}
\pardescript {store an alias for the sender of the message} 

\parname  {\em af\_pse\_add\_PK}
\pardescript {add the public key into the PSE object {\bf PKList}} 

}
\subsubsection{pem\_get\_header}
\nm{3X}{pem\_get\_header}{Scan a PEM text}
\label{pem_get_header}
\hl{Synopsis}
\#include $<$pem.h$>$ 

RC {\bf pem\_get\_header} {\em (info, set\_of\_pemcrlwithcerts, issuer, signature, buf)} \\
PemInfo {\em *info}; \\
SET\_OF\_PemCrlWithCerts {\em **set\_of\_pemcrlwithcerts}; \\
SET\_OF\_DName 	{\em **issuer}; \\
BitString {\em *signature}; \\
OctetString {\em *buf};
\hl{Input-Parameter}
\parname  {\em buf}
\pardescript {RFC 1421 specified text between the
		    privacy enhanced message encapsulation boundary lines
		    (included).
		    If an embedded multi-level PEM is to be scanned,
		    the first Pre-EB of the pem OctetString must be
		    the Pre-EB of the embedded PEM to be scanned.
		    pem\_sbody will automatically find the corresponding
		    Post-EB (counting Pre-EBs up, and Post-EBs down).}

\hl{Output-Parameter}
\parname  {\em info}
\pardescript {Signature and encryption control information.
		    The {\em PemInfo} structure of {\em info}
		    is to be allocated by the calling function.
		    However, the pointers below {\em info}:
		    {\em encryptKEY}, {\em origcert}, {\em signAI},
		    and {\em recplist}, are created by {\em pem\_scan}.
		    The informations stored into {\em info} are a result of
		    scanning the encapsulated header fields in {\em pem}.} 

\parname  {\em signature}
\pardescript {Signature of the scaned PEM body which is encoded in the {\bf MIC } header field.
                    The memory of the BitString structure
                    {\em signature} is created by the calling function.} 

\parname  {\em set\_of\_pemcrl...}
\pardescript {Structure to store CRLs in case of Proc-Type CRL.} \\
\parname  {\em issuer}
\pardescript {Structure to store issuer names in case of Proc-Type CRL-RETRIEVAL-REQUEST.} \\
\parname  {\em Returncode RC}
\pardescript {negative if erraneous, 0 if OK.} \\
\hl{Description}

pem\_get\_header evaluates the PEM text {\em buf} and fills the structures PemInfo {\em info} and Bitstring {\em signature } or
SET\_OF\_PemCrlWithCerts {\em set\_of\_pemcrlwithcerts} or SET\_OF\_DName {\em issuer} depending
on then Proc-Type of the PEM message.

\subsubsection{pem\_cinfo}
\nm{3X}{pem\_cinfo}{Generate signature and encryption control info for/of PEM text}
\label{pem_cinfo}
\hl{Synopsis}
\#include $<$pem.h$>$ 

RC {\bf pem\_cinfo} {\em (info)} \\
PemInfo {\em *info};
\hl{Description}
{\em pem\_cinfo} supports the collection of signature and
encryption control informations and stores them into
the structure PemInfo {\em info}.
pem\_cinfo is either the first of pem\_create,
or, alternatively, it is performed prior to the call of pem\_create.
That is, because the full set of informations in the structure PemInfo
serves as input to pem\_create and all pem\_c* subroutines.


The parameter {\em info} is defined by the calling function.
Some informations are already stored on input to pem\_cinfo.
pem\_cinfo completes the set of informations.


On input to pem\_cinfo, the subfields {\em confidential} and {\em clear}
must be set. If {\em confidential} = TRUE,
{\em recplist} of {\em info} must be created by calling function, too;
in this {\em recplist},
for every recipient's certificate, at least
the owner's name {\em tbs.subject} must be set on input.


On output, pem\_cinfo will have filled the subfields
{\em encryptKEY} and,  for every recipient in {\em recplist}, {\em key}.
If on input one or more of the elements
{\em origcert, signAI}, or a recipient's public encryption key
in {\em recpcert} (in recplist) were not filled,
pem\_cinfo will provide these informations, too.
For every recipient's certificate, pem\_cinfo tries also to collect the
informations {\em tbs.issuer} and {\em tbs.serialnumber}
from {\em EKList} of the caller's PSE or from a public directory.


pem\_cinfo performs the following steps:

\begin{enumerate}
\item
If {\em origcert} has no value on input,
read the originator-certificate (struct Certificates)
with the help of {\em af\_pse\_get\_Certificates}
and store it in {\em origcert}.
\item
If {\em signAI} has no value on input,
take the public key algorithm identifier
from the user's certificate of {\em origcert}
and store it in {\em signAI}.


If PEM type is not ``ENCRYPTED'',
set {\em info\pf encryptKEY} to {\em NULL} and return.
If confidentiality service is required, resume with the following steps:
\item
Generate a DES-key with the help of
af\_gen\_key and store it as key-reference in
{\em info\pf encryptKEY}.
\item
If in a {\em recpcert} (of {\em recplist})
the public encryption key has no value on input,
search for it in the {\em EKList} of the caller's PSE
with the help of {\em af\_pse\_get\_owner}
and store it in this {\em recpcert} .
If this fails, try to collect the recipient's certificate
from a public directory with the help of {\em af\_dir\_retrieve\_Certificate}.
If this also fails, or if
in a {\em recpcert}
the owner's name (tbs.subject) has no input value,
delete this entry from the recplist.
\item
For every recipient in the {\em recplist},
use his public encryption key in {\em recpcert}
in order to RSA-encrypt the DES-key stored in
{\em info\pf encryptKEY}
(call to {\em af\_get\_EncryptedKey}) and store
the result in {\em key} of the respective {\em recplist} entry.
\end{enumerate}

\subsubsection{pem\_read}
\nm{3X}{pem\_read}{Read and scan a PEM text, and report errors to caller}
\label{pem_read}
\hl{Synopsis}
\#include $<$pem.h$>$ 

RC {\bf pem\_read} {\em (ifname, ofname, depth, verbose, cadir)} \\
char {\em *ifname, *ofname, *cadir}; \\
int {\em depth}; \\
Boolean {\em verbose} \\
\hl{Input-Parameter}
\parname  {\em ifname}
\pardescript {Pathname of the file
		    which contains the PEM text.
		    If {\em ifname} is the NULL pointer,
		    the PEM text is read from stdin.}

\parname  {\em ofname}
\pardescript {Pathname of the file
		    into which the cleartext message body is written.
		    If {\em ofname} is the NULL pointer,
		    the cleartext message body is written to stdout.}

\parname  {\em depth}
\pardescript {Integer greater or equal to zero.
		    $depth = 0$ reads the outmost PEM
		    (first Pre-EB, last Post-EB).
		    $depth > 0$ reads the {\em depth}'th embedded PEM
		    ($(depth+1)$st Pre-EB, $(last-depth)$th Post-EB).
		    $depth < 0$ is treated like $depth = 0$.}

\parname  {\em verbose}
\pardescript {If TRUE, pem\_read tells to stderr what it's doing.} \\
\parname  {\em cadir}
\pardescript {The UNIX directory of the CA.}

\hl{Output-Parameter}
\parname  {\em Returncode RC}
\pardescript {-1 if any error occured, 0 if no error occured.} \\
\hl{Description}
pem\_read reads the input file $<$ifname$>$,
opens the output file $<$ofname$>$,
and opens the caller's PSE ({\em af\_pse\_open}).
Then it skips all characters of the read file before the
$(depth+1)$st Pro-EB
and calls the function {\em pem\_scan}.
In normal case pem\_read writes the (decrypted) PEM body in cleartext into the output file.
If the PEM message if of Proc-Type CRL-RETRIEVAL-REQUEST, a corresponding PEM CRL messsage
containing the requested CRLs is written to the output file.
pem\_read evaluates all possible error codes and reports
them in readable form to stderr, if {\em verbose } is TRUE.
In particular, pem\_read reports very clearly
the result of the message integrity check MIC of the PEM.

\subsubsection{pem\_write}
\nm{3X}{pem\_write}{Create and write a PEM text}
\label{pem_write}
\hl{Synopsis}
\#include $<$pem.h$>$ 

RC {\bf pem\_write} {\em (recips, ifname, ofname, encr, clear, verbose)} \\
RecpList {\em *recips}; \\
char {\em *ofname, *ifname}; \\
Boolean {\em encr, clear, verbose}; \\
\hl{Input-Parameter}
\parname  {\em recips}
\pardescript {List of intended recipients, only evaluated,
		    if PEM type is ``ENCRYPTED''
		    (i.e. if  parameter {\em encr} is set to TRUE).
		    In this case, at least the certificates' owners
		    must be set. If the other entries of {\em recips}
		    are set, too, they would be respected by the
		    PEM creation process. Otherwise, their values
		    will be collected during the
		    PEM creation process.}

\parname  {\em ifname}
\pardescript {Pathname of the UNIX file
		    which contains the cleartext message body.
		    If {\em ifname} is the NULL pointer or ZERO string,
		    the message body is read from stdin.}

\parname  {\em ofname}
\pardescript {Pathname of the UNIX file
		    into which the created PEM is written.
		    If {\em ofname} is the NULL pointer or ZERO string,
		    the PEM is written to stdout.}

\parname  {\em encr}
\pardescript {Boolean, set to TRUE, iff the type of the PEM to be created
		    is intended to be ``ENCRYPTED''.}

\parname  {\em clear}
\pardescript {Boolean, set to TRUE, iff the type of the PEM to be created
		    is intended to be ``MIC-CLEAR''.
		    If {\em encr} = TRUE, {\em clear} is ignored.}

\parname  {\em verbose}
\pardescript {If TRUE, pem\_write tells to stderr what it's doing.} \\
\hl{Output-Parameter}
\parname  {\em Returncode RC}
\pardescript {-1 if any error occured, 0 if no error occured.} \\
\hl{Description}
pem\_write reads the input file $<$ifname$>$,
opens the output file $<$ofname$>$,
and opens the caller's PSE ({\em af\_pse\_open}).
A PemInfo structure is created, filled with the input parameters
to pem\_write, and passed to the called function {\em pem\_cinfo},
which completes the values for PemInfo.
With these informations, {\em pem\_create} is called.
pem\_write writes the created PEM into the output file.
pem\_write evaluates all possible error codes and reports
them in readable form to stderr if {\em verbose} is TRUE.

\subsubsection{pem\_crl\_retrieval\_request}
\nm{3X}{pem\_crl\_retrieval\_request}{Creates a PEM text of Proc-Type CRL-RETRIEVAL-REQUEST, and report errors to caller}
\label{pem_crl_retrieval_request}
\hl{Synopsis}
\#include $<$pem.h$>$ 

RC {\bf pem\_crl\_retrieval\_request} {\em (issuer, pem)} \\
SET\_OF\_DName {\em *issuer}; \\
OctetString {\em *pem}; \\
\hl{Input-Parameter}
\parname  {\em issuer}
\pardescript {A list of issuers.}

\hl{Output-Parameter}
\parname  {\em pem}
\pardescript {RFC 1421 specified text between the
		  privacy enhanced message encapsulation boundary lines
		  (included).
		  The OctetString structure of {\em pem} is
		  created by the calling function.
		  However, the character string in {\em pem}
		  is allocated by {\em pem\_crl\_retrieval\_request}.} 

\parname  {\em Returncode RC}
\pardescript {-1 if any error occured, 0 if no error occured.} \\
\hl{Description}
pem\_crl\_retrieval\_request creates the PEM message pem of Proc-Type CRL-RETRIEVAL-REQUEST.
This message contains the {\bf Proc-Type } field and one {\bf Issuer} field
for each issuer given in the list {\em issuer}. The value of such a field is the
encoded distinguished name of the issuer.

\subsubsection{pem\_crl}
\nm{3X}{pem\_crl}{Creates a PEM text of Proc-Type CRL, and report errors to caller}
\label{pem_crl}
\hl{Synopsis}
\#include $<$pem.h$>$ 

RC {\bf pem\_crl} {\em (issuer, pem, cadir)} \\
SET\_OF\_DName {\em *issuer}; \\
OctetString {\em *pem}; \\
char {\em *cadir}; \\
\hl{Input-Parameter}
\parname  {\em issuer}
\pardescript {A list of issuers.}

\parname  {\em cadir}
\pardescript {The UNIX directory of the CA.}

\hl{Output-Parameter}
\parname  {\em pem}
\pardescript {RFC 1421 specified text between the
		  privacy enhanced message encapsulation boundary lines
		  (included).
		  The OctetString structure of {\em pem} is
		  created by the calling function.
		  However, the character string in {\em pem}
		  is allocated by {\em pem\_crl}.} 

\parname  {\em Returncode RC}
\pardescript {-1 if any error occured, 0 if no error occured.} \\
\hl{Description}
pem\_crl creates the PEM message pem of Proc-Type CRL.
This message contains the {\bf Proc-Type } field and one group of a {\bf CRL} field,
an {\bf Originator-Certificate} field and any {\bf Issuer-Certificate} fields
for each issuer given in the list {\em issuer}. A CRL field stores the certificate
revocation list with a signature of one issuer which can be verified with the following
originator and issuer certificates.

\subsubsection{pem\_certify}
\nm{3X}{pem\_certify}{Scan PEM msg, sign the Originator-Certificate, and create new PEM msg}
\label{pem_certify}
\hl{Synopsis}
\#include $<$pem.h$>$ 

RC {\bf pem\_certify} {\em (ifname, ofname, verbose, cadir)} \\
char {\em *ifname, *ofname, *cadir}; \\
Boolean {\em verbose} \\
\hl{Input-Parameter}
\parname  {\em ifname}
\pardescript {Pathname of the  file
		    which contains the PEM.
		    If {\em ifname} is the NULL pointer,
		    the PEM is read from stdin.}

\parname  {\em ofname}
\pardescript {Pathname of the file
		    into which the cleartext message body is written.
		    If {\em ofname} is the NULL pointer,
		    the cleartext message body is written to stdout.}

\parname  {\em cadir}
\pardescript {The UNIX directory of the CA.}

\parname  {\em verbose}
\pardescript {If TRUE, pem\_certify tells to stderr what it's doing.} \\
\hl{Output-Parameter}
\parname  {\em Returncode RC}
\pardescript {-1 if any error occured, 0 if no error occured.} \\
\hl{Description}
pem\_certify reads the input file $<$ifname$>$,
opens the output file $<$ofname$>$,
opens the caller's PSE ({\em af\_pse\_open})
and calls the function {\em pem\_scan}.
pem\_certify signs the originator certificate which is supposed to be 
a prototype certificate and calls
pem\_create to recreate the PEM message, now with a
signed originator certificate and a forward certification path
and writes it to $<$ofname$>$.
This PEM massage  can be verified now.
pem\_certify evaluates all possible error codes and reports
them in readable form to stderr, if {\em verbose } is TRUE.
In particular, pem\_certify reports very clearly
the result of the message integrity check MIC of the PEM.
