%
% Definition eigener Makros fuer SECUDE
%
% einbetten mit \input{spec-makro}
%
\def\parname{\parbox[t]{3.3cm}}
\def\pardescript{\parbox[t]{11.6cm}}
\def\firstbox{\parbox[t]{5.3cm}}
\def\secondbox{\parbox[t]{9.6cm}}
\def\parba{\parbox[t]{4cm}}
\def\parbe{\parbox[t]{11cm}}
\newcommand{\speconly}[1]{#1}
\newcommand{\manonly}[1]{}
\def\nopagenumbering{
        \pagestyle{empty}
        \pagenumbering{alph}
        \setcounter{page}{0}
        \thispagestyle{empty}
}
\def\bi{\begin{itemize}}
\def\ei{\end{itemize}}
\def\bc{\begin{center}}
\def\ec{\end{center}}
\def\be{\begin{enumerate}}
\def\ee{\end{enumerate}}
\def\bd{\begin{deflist}}
\def\ed{\end{deflist}}
\def\bv{\begin{verbatim}}
\def\ev{\end{verbatim}}
\def\bquote{\begin{quote}}
\def\equote{\end{quote}}
\def\m{\item}
\def\n{\newpage}
\def\ka{\raisebox{.4ex}{{\tiny $<\!\!\!<$}}}
\def\kz{\raisebox{.4ex}{{\tiny $>\!\!\!>$}}}
\def\marker#1{\marginpar{\rule[0cm]{1mm}{#1}}}
\def\pf{\-{--\raisebox{.25ex}{{\scriptsize $>$}}}}
\def\btab{
        \begin{tabbing}
        aaaaaaaa \= aaaaaaaa \= aaaaaaaa \= aaaaaaaa \= aaaaaaaa \= aaaaaaaa \= aaaaaaaa \= aaaaaaaa \= aaaaaaaa \= aaaaaaaa \= \kill
}
\def\bvtab{
        \begin{tabbing}
        aaaa \= aaaa \= aaaa \= aaaa \= aaaa \= aaaa \= aaaa \= aaaa \= aaaa \= aaaa \=aaaa \=aaaa \= \kill
}
\def\etab{\end{tabbing}}
\def\evtab{\end{tabbing}}
\def\1{\>}
\def\2{\> \>}
\def\3{\> \> \>}
\def\4{\> \> \> \>}
\def\5{\> \> \> \> \>}
\def\6{\> \> \> \> \> \>}
\def\7{\> \> \> \> \> \> \>}
\def\8{\> \> \> \> \> \> \> \>}
\def\9{\> \> \> \> \> \> \> \> \>}
\def\blankpage{
        \nopagenumbering
        \newpage
        \btab
        \etab
        \newpage
}
\newcommand{\hll}[1]{}
\newcommand{\hl}[1]{\vs{0.5cm} {\bf #1} \\ [1ex]}
\newcommand{\nm}[3]{#3}
\def\addtotoc#1{}
\def\vs#1{
        \mbox{} \vspace{#1} \\
}
\def\ifnot#1#2{
        \def\firstpar{#1}
        \def\secondpar{#2}
        \ifx\firstpar \secondpar \else \secondpar \\ \fi
}
\def\eline#1{
        \def\fpar{#1}
        \def\spar{}
        \ifx\fpar \spar \else \fpar \\ [1cm]\fi
}
\def\secudetitlepage#1#2{
        \title{ {\Huge\bf  SecuDE} \\ [1.5cm]
                {\Large\bf #1} \\ [0.5cm] 
                {\large \eline{#2}}
                {\large Version 4.1} \\ [2cm]
        }
        \author{
                \authors{}
                GMD \\
                Institut f{\"u}r TeleKooperationsTechnik (I2) \\
                Darmstadt, Germany \\ [5cm] 
        }
        \date{
                \today \\ [1cm]
                {\normalsize Copyright \copyright 1993 by Gesellschaft f{\"u}r Mathematik und Datenverarbeitung (GMD)}
        }
        \nopagenumbering
        \maketitle
        \parindent0em
        \newpage
        \conditions{}
        {\small
        Other volumes available within this documentation: \\ [1ex]
	\ifnot{#1}{Overview}
	\ifnot{#1}{Vol. 1: Principles of Security Operations}
	\ifnot{#1}{Vol. 2: Security Commands, Functions and Interfaces}
 	\ifnot{#1}{Vol. 3: Security Applications' Guide}
        }
        \newpage
	\pagenumbering{roman}
        \preface{}
	\cleardoublepage
}

\def\conditions{
        Copyright \copyright Gesellschaft f{\"u}r Mathematik und Datenverarbeitung (GMD), 1993
        \\ [1cm]
        {\footnotesize
        Permission to use, copy, modify, and distribute this software and its documentation
        for any purpose and without fee is hereby granted, provided that this notice and the
        reference to this notice appearing in each software module be retained unaltered, and 
        that the name of GMD or any contributor shall not be used in advertising or publicity 
        pertaining to distribution of the software without specific written prior permission. 
        It is the responsibility of the users of this software to comply with national or 
        international export and import regulations, or with licence rights from third parties.
        \\ [1em]
        GMD AND ALL CONTRIBUTORS DISCLAIM ALL WARRANTIES WITH REGARD TO THIS SOFTWARE, INCLUDING
        ALL IMPLIED WARRANTIES OF MERCHANTIBILITY AND FITNESS. IN NO EVENT SHALL GMD OR ANY
        CONTRIBUTOR BE LIABLE FOR ANY SPECIAL, INDIRECT OR CONSEQUENTIAL DAMAGES OR ANY DAMAGES
        WHATSOEVER RESULTING FROM LOSS OF USE, DATA OR PROFITS, WHETHER IN AN ACTION OF CONTRACT,
        NEGLIGENCE OR OTHER TORTIOUS ACTION, ARISING OUT OF OR IN CONNECTION WITH THE USE OR
        PERFORMANCE OF THIS SOFTWARE. \\ [8cm]
        }
}

\def\preface{
	{\Large\bf Preface}
	\\ [1cm]
        SecuDE has been developed by GMD for the R \& D community as a research tool to
        promote the construction of trustworthy systems, particularly in the context of OSI.
        The initial sponsorship came from the Deutsches Forschungsnetz (DFN), project 
        {\em Secure DFN}, which was made possible by grants from the Ministry of Research 
        and Technology of Germany. 
        \\ [1em]
	The SecuDE documentation was prepared by R{\"u}diger Grimm, Jan L{\"u}he and 
        Wolfgang Schneider.
        \\ [1em]
	The following people contributed to the software development of SecuDE: 

	R{\"u}diger Grimm, Stephan Kolletzki, Jan L{\"u}he, Rolf-Dieter Nausester, 
        J{\"o}rg Reichelt, Wolfgang Schneider, Thomas Surkau and Ursula Viebeg contributed 
        to SecuDE in general. 
	The arithmetic package and the RSA and DSA software were written by Achim Jung, 
        Wolfgang Bott, Rolf-Dieter Nausester, Thomas Surkau and Stephan Thiele. 
        The STARCOS smartcard package and its integration into SecuDE was done by Levona 
        Eckstein, Helga Parslow, Ursula Viebeg and Zhou Anmin.
	\\ [1em]
        The SecuDE package includes parts of ISODE Release 8.0 ASN.1 and QUIPU tools, 
        Phil Karn's DES Package, MD2, MD4 and MD5 software from RSA Data Security Inc., 
        and a SHA implementation from Peter C. Gutman and Colin Plumb (pgut1@cs.aukuni.ac.nz). 
        \\ [1em]
        Comments are welcome and should be addressed to: 

        \parba {Postal Address:}
        \parbe {
                GMD, Institut f{\"u}r TeleKooperationsTechnik (I2) \\
                Dolivostr. 15 \\
                6100 Darmstadt \\
                Germany \\
        }
        \parba {Telephone:}
        \parbe {
                +49-6151-869-719 or -718
        }
        \parba {Fax:}
        \parbe {
                +49-6151-869-785 \\
        }
        \parba {E-Mail:}
        \parbe {
                Internet-Address: \\
                schneider@darmstadt.gmd.de \\
                luehe@darmstadt.gmd.de \\ [1ex]
                X.400-Address: \\
                C=DE; ADMD=DBP; PRMD=GMD; OU=Darmstadt; S=Schneider
        }
	\newpage
	{\Large\bf How to Get SecuDE}
	\\ [1em]	
        The most recent version of SecuDE can be obtained by 
	\bi
        \m ANONYMOUS FTP from internet address {\em ftp.darmstadt.gmd.de} (141.12.50.1), 
        \m FTAM (User-Id {\em anon}) from X.25 address 262 45050363363.
	\ei 
        cd to the pub/secude directory to find 
	\bi
        \m three .ps.Z files comprising compressed postscript files of the available
           documentation,
        \m the file secude-x.y.all.tar.Z comprising a tar'ed and compressed directory 
           of everything including program sources and documentation sources (latex),
        \m the file secude-x.y.src.tar.Z comprising the program sources only,
        \m the file secude-x.y.doc.tar.Z comprising the documentation sources only.
	\ei
	x.y is currently 4.1.
	\\ [1cm]
	{\Large\bf Electronic Mail Distribution List}
	\\ [1em]
	For electronic mail discussions about SecuDE use the Internet address
	
	{\em secude@darmstadt.gmd.de}
	
	To get added to the list send an informal mail to {\em secude-request@darmstadt.gmd.de}. 
	\\ [1cm]	
	{\Large\bf Restrictions}
	\\ [1em]
	The SecuDE package contains cryptographic software for both authentication and
        encryption purposes. This type of software might be subject to national or
        international export or import regulations. It contains also software which
	implements the RSA algorithm. The RSA algorithm is patented in USA.
	
	Those who obtain the SecuDE software via public network access are advised to get
        acquainted with the respective regulations and licence conditions for that
	environment where they intend to use the software. It is their responsibility
	not to get in conflict with such regulations.
}
