\section{Security Levels}
\markboth{Security Levels}{Security Levels}
\thispagestyle{myheadings}
\label{sl}

\subsection{Security Levels of Signature Verification}
\label{sl-sv}

\begin {center}
\begin {tabular}{|c|l|l|}
\hline
{\em Security} & {\em Short Name} & {\em Service} \\
{\em Level} & & \\ \hline
0 & No Security  & No Service \\ \hline
1 & Simple       & Verification of signature \\
  & Authenticity & without certificate \\ \hline
2 & Certified    & Verification of signature and of \\
  & Authenticity & originator certificate \\ \hline
3 & Certified and& Like 2. Additionally:  \\
  & Unrevoked    & Check of black lists \\
  & Authenticity & \\ \hline
\end {tabular}
\end {center}
 
\label{fig-sl-sv}
\stepcounter{Abb}
{\footnotesize Fig.\arabic{Abb}: Services of security levels}

\subsubsection{Security Level 0: No Security}
\label{sl-sv0}
 
\subsubsection{Security Level 1: Simple Authenticity}
\label{sl-sv1}

Signature verification (or encryption, respectively)
is performed without certificate.
The public key of the partner is known and stored on the user's smartcard
({\em PKList} or {\em EKList}, respectively).
\\[1ex]
This security level is safe against attacks.
However, its usage is limited.
It works between mutually trusted partners.
It also works within
a closed communication network with one trustworthy CA
(cf. Kerberos \cite{ste1}).
Inside of an open network, any pairs or groups of mutually trusted
partners can exchange and verify certified public keys,
and then store them in order to communicate on this security level 1.

\subsubsection{Security Level 2: Certified Authenticity}
\label{sl-sv2}

Signature verification (or encryption, respectively)
includes a check of the consistence of the certificate chain up to
a known and trusted public key.
The known and trusted public key belongs either to
a trusted communication partner and was verified
at an earlier occasion,
or it belongs to a certification authority of oneself's trusted
certification path to the Root CA.
In any case, it is stored on the smartcard table {\em PKList}
or {\em EKList}, respectively.

This security level
 
\begin{itemize}
\item prevents from acceptance of expired keys, but
\item correctly accepts expired but formerly valid certificates;
\item protects against wrongly or incompletely certified keys,
\item does not protect against revoked keys.
\end{itemize}
 
Expired CA certificates are valid as parts of certificate chains,
which certify an old signature within the validity time intervals of all
certificates of the chain.
If the last certificate of a chain is signed by an old signature key
of the Root CA,
a valid cross certificate can be obtained from the directory table
``Old Certificates'' (see paragraph \ref{ds-roc} above).

In this security level, black lists are not involved.

\subsubsection{Security Level 3:
Certified and Unrevoked Authenticity}
\label{sl-sv3}

This security level 3 includes
security level 2 and additionally requires black list inquiries.
This provides a protection against keys,
which have been revoked by their certification authorities,
because they are compromised or have been used against CA policies.
One revoked certificate of a chain of certificates discredits
all certificates below and the user signature as well.

Complete black list consults require
at worst one directory access per certificate of a chain.
The number of directory accesses can be reduced, if one authority
keeps the revoked certificates of other CAs as well.
