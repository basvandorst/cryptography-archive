% cfp2-talk.tex - Computers, Freedom, and Privacy Conference II
% The talk was given 20 March 1992 and The slides are identical
% to those used for a presentation to the IEEE CCIP on 17 March 1992.
% The text was typed in from the conference transcript.  Lance J.
% Hoffman (Ed.), ``The second Conference on Computers Freedom and
% Privacy'' Association For Computing Machinery.''  Note that although
% the proceedings bear ACM copyright notice, the right to use the
% material reproduced here were transferred to them  on the explicit
% condition that this material was being placed in the public domain.

% spelling words: Diffie PCs crypto fax gravetomometer

\input slidetextfmt
\input list
\font\large=cmr17

\def\raggedcenter{\leftskip=0pt plus4em \rightskip=\leftskip
                  \parfillskip=0pt \spaceskip=.3333em \xspaceskip=.5em
                  \pretolerance=9999 \tolerance=9999
                  \hyphenpenalty=9999 \exhyphenpenalty=9999 }

\macromode
\let\talk=\halftext
\def\text#1{} \def\halftext#1{}
\def\figref#1{#1}
\def\ednote#1{#1}
\def\slidenum
{10--\the\slidecount
     \ifnum \contslidecount>0 {\alph\contslidecount}\fi}
\def\halftextout
{\ifvoid \rowbox
         \addframe{\vbox to \textheight
                        {\vfil\hbox to \halftextwidth
                                   {\hfil}\vfil}} \fi
 \addframe{\vbox to \textheight{\vfil\unvbox255\vfil}}}
\def\slideout
{\global\advance\slidecount by 1
 \ifvoid \rowbox \else
        \addframe{\vbox to \textheight
                       {\vfil\hbox to \halftextwidth{\hfil}\vfil}} \fi
 \addframe{\slideborder{\box255}}
 \sendtty{\tab\tab\spc Slide_No._\the\slidecount\spc\spc
                (page_\the\pageno):_\the\slidetitle}
 \global\def\frameout{\contslideout}
 \output{\frameout}}
\endmacromode

\def\theauthor{Whitfield Diffie}
\def\fulltitle{Public Policy Issues in Cryptography}
\def\shorttitle{\fulltitle}
\def\lefthead{}         \def\righthead{{\large \shorttitle}}
\def\leftfoot{{\large \theauthor}}      \def\rightfoot{\datetime}
\def\ll{} \rightlabel{\shorttitle} \def\slidelogo{}
\def\botleftgap{\ \theauthor\ }

%\oneslide\def\slideborder#1{\trim{#1}{\borderheight}{\borderwidth}}%
\sixslides%\slidesonly%

\slide{\title{The Essence of Security}
       \list{\bullets} %\listmar=3em
             \item Recognition of those you know
             \item Introduction to those you don't know
             \item Written Signature
             \item Private Conversation
             \endlist

\vskip 2ex
\para Communication security is the transplantation of these basic
social mechanisms to the telecommunications environment. \endpara}

\halftext{\para The most important security measures in which any of
us engage in our daily business have nothing to do with safes, locks,
guards, or badges: \begdis
 \list{\bullets}
  \item We recognize our colleagues
  \item We affix our signatures to letters, contracts, and records.
  \endlist

\enddis It is also noteworthy that in our communications, both
written and spoken, we take for granted that by sealing the envelope
or closing the door we can achieve privacy in our communications. \endpara

\para The challenge of modern security technology is to transplant
these familiar mechanisms from the traditional world of face-to-face
meetings and pen and ink communications to a world in which digital
electronic communications are the norm and the luxury of personal
encounters or handwritten messages are the exception. \endpara}

\talk{\para I think that if you think about it, you'll realize that the most
important things any of you ever do by way of security
have nothing to do with the guards, the fences, the badges,
the safes, or any of that stuff \figref{(Figure 10-1)}.  Far and
away the most important security measure in anybody's
life is that you recognize the people you know. And  you
recognize the people you work with, and you have a
mechanism for extending that mechanism by introduction,
so you come to know, come to trust people you didn't
know before because people you did know introduce them
to you.  Then you have a transferable form of authentication
that you use constantly.  You have your written signature
and you put your signature on something and the recipient
can show it to a third party and say, ``Here. Look.  Whit
Diffie promised to do that.  See his signature on this letter.''
And then finally, you have the possibility of closing your
door, stepping aside somewhere, and having a private
conversation with somebody. \endpara}

\slide{\title{Essence of Security (Cont'd)}
 \heading{What do you do over:}
 \list{\bullets}
  \item Electronic mail?
  \item Telephone?
  \item Video Conference?
  \item EDI?
  \endlist
 \heading{Essential social mechanisms}}

\talk{\para Well, the question for us comes down to, what do you do
when you move from a world of pen and ink and face-to-face
conversations into a world of electronic mail, telephone,
video conferences, and electronic data interchange \figref{(Figure
10-2)}?  And make no mistake about it, before the end of this
decade, two-person video conference, which for reasons of
cost, has not really appeared yet, is just going to sweep
this country the way fax has, the way PCs have.  It'll make
possible collaborative work over a distance, and it will
transform every corporation's and every social group's way
of operating. \endpara}


\slide{\title{Many Modern Technologies Decrease Privacy}
% Police playing catchup?
 \list{\bullets}
  \item Credit (and other) databases
  \item Hidden video cameras
  \item X-ray, NMR, Gravetomometer
  \item Night vision
  \item Rafter
  \item \hbox{\qquad$\ldots$}
  \endlist}

\talk{\para We are awash in technologies that decrease privacy --- lots
of them developed under the umbrella of the war against
terrorism, the war against drugs, a lot of them, as a matter
of fact, from the Vietnam War \figref{(Figure 10-3)}.  There are lots
that you know; let me mention two or three that you might not
know.  There was a wonderful article in Aviation Week four
months ago or so, sometime last fall, about a
gravetomometer so sensitive that it was able to detect a
pound of cocaine suspended in the middle of a 50-gallon
drum of something like orange juice.  I don't remember
what it was.  You really can't hide things from something
that can detect gravitational anomalies that fine.  Well, the
Vietnam War gave us night vision that would make your
bedroom look as though you were making love on the
beach at Monterey at midday.  The last one's one called
rafter, and it was mentioned in a book called Spy Catcher.
Rafter is a British code name for the technique of listening
to radio receivers, that is, listening to the local oscillators
of radio receivers to figure out what station you are listening
to.  Now people are used to the notion that they might be
monitored when they are transmitting; I think that's
perfectly natural.  But think about the fact that people might
be listening to hear what you are receiving, what things
you're interested in getting. \endpara}

\slide{\title{Crypto Can Increase Privacy}
 \list{\bullets}
  \item Secure phone calls
  \item Secure electronic mail
  \item Electronic voting
  \item Digital cash
  \endlist}
% right bracket and word anonymity to right of voting and cash

\talk{\para Against that I can find very few means of increasing privacy
--- I don't guarantee there aren't others --- certainly crypto
is a technology with the capacity to increase people's
privacy \figref{(Figure 10-4)}.  And I think all these cases, the first
couple in any event are fairly obvious and desperately
needed --- security in telephone communications even more
so, security in electronic mail. \endpara

\para Two far less obvious things have been developing over the
last few years, things that require the provision of actual
anonymity, and at the same time prevent you from making
use of your anonymity to cheat --- these are electronic
elections and electronic money. \endpara}

\slide{\title{Crypto Can Increase Accountability}
 \list{\bullets}
  \item Digital Signatures permit auditing.
  \item Auditing is the basis for investigation.
  \endlist}

\talk{\para It's also true that cryptography has developed capacities
that allow it to increase accountability \figref{(Figure 10-5)}.  In short,
if you look at the previous slide \figref{(Figure 10-4)}, increase in
anonymity where appropriate, here increased
accountability.  Digital signatures give you the capacity to
audit, just the way you audit a classical legger, the same
format at least.  You look at the handwriting of each
individual entry in the legger, you know who did it, you
have accountability for the actions that the legger covers.
Digital signatures give you that capacity in electronic media
and auditing is the basis for investigations.  It is the capacity
of investigations to discover what did happen --- to find
out who was accountable is the essence of people's being
held answerable for their actions. \endpara}

\slide{\title{The Root of the Policy Problem}
 \para Cryptography appears to offer the unprecedented possibility
 that individual or groups can protect information on a large scale
 from society's concerted efforts to obtain it. \endpara}

\talk{\para Let me get to the policy difficulty \figref{(Figure 10-6)}.  I think
cryptography offers, or at least appears to offer, something
that as far as I can see is unprecedented.  I mean, if you get
in the spirit of mathematics --- and I emphasize that
nobody knows for sure --- bit it almost looks as though an
individual might be able to, in a systematic way --- for
example, with a mass-marketed piece of software --- protect
information in such a way that the concerted efforts of
society aren't going to be able to get at it.  I mean, no safe
you can procure has that property, right?  The strongest
safes won't stand an hour against oxygen lances, but
cryptography may be different from that. \endpara}

\slide{\title{\vbox{\hbox{Will Crypto Be Undone}%
                    \hbox{\quad by Implementation}}}
 \list{\bullets}
  \item Tempest
  \item Tamper Resistance
  \item Random Number Generation
  \endlist
 \vskip 1ex
 \para Is cryptography a mathematical science or a military
       science? \endpara}

\talk{\para Before I go any further, I'm going to say that in practice it
might turn out to be not so different from that.  As you try
to implement cryptography, you find that you begin to
think, well, maybe this isn't so much a mathematical
problem any more.  Maybe this is an arms race, and we've
got to develop a better technique and they'll develop a
better technique.  Because of the problem of electromagnetic
radiation out of your equipment, particularly the plain text
signal, the tamper-resistance of your equipment and the
quality of your ability to generate unpredictable numbers
are absolutely crucial \figref{(Figure 10-7)}.  And those are all issues
of physical science.  So I put that bee in your bonnets ---
worry a little when you write your crypto programs and
things.  I've heard some fairly cocksure statements around
her, you know, ``Anybody could implement this,'' and
``How's there any hope of controlling it?,'' etc.  Well, it's
not always that easy. \endpara

\para Now we will turn for just one instant to the previous
slide \figref{(Figure 10-6)} and say, from my point of view, that
this, in fact, has a lot to be said for it.  I understand why the
police don't like it.  But let me suggest that a very large
part, if not the essence, of the distinction between a free
society and a totalitarian society consists of the difference
between being answerable for your actions and being
subject to prior restraint on actions that the society doesn't
approve of. \ednote{(applause)} What this means is, in essence, if
you look at it, the subpoena sort of model. They've got to
come to you and say, ``Whit, show us these records or you
sit in jail for a while.''  And I get to decide, as reporters
unfortunately have to decide, whether I'd rather sit in jail
than show the court the information they want.  What I
think is utterly inappropriate is that they can go to some third
party, the keepers of the disk that I have my stuff on, and
say, ``Either you show us Whit's stuff or you go to jail for a
while.'' And you know, it's not their data, what do they
care? OK. \endpara}

\slide{\title{Roots of the Policy Problem --- 2}
 \list{\bullets}
  \item Telecommunication violates a traditional `locality'
        of human society.
  \item This locality has been a major mechanism of
        individual accountability.
  \item Lack of security in communication has in some ways
        replaced locality.
  \item Can even a `free society' tolerate unrestricted use of
        cryptography?
  \endlist}

\talk{\para I really believe, long-term, that there are some serious
problems here \figref{(Figure 10-8)}.  I think telecommunications
violates a locality property that has been the basis of society.
If I come into town to negotiate with somebody, I really
can't be confident that I won't be noticed going in and out
of town.  If I have to stay over in the hotel, I leave a record
there, etc.  If calls can be made that are somehow completely
anonymous, completely secure, any two people could be
in contact, and the whole way that we understand what
groups are in the society would dissolve.  I think, in a
peculiar way, the lack of security in communication has
taken the place of this locality.  That is, you can call
somebody up and conspire with them by phone, but you
can't be sure that you will not have been overheard.  You
don't have the same confidence that you do if you go sneak
off behind the haystack to encounter this person.  And so I
wonder if society in the sense we know it, would exist in
the presence of absolutely unaccountable communication
between  any pair of people, as much as that notion appeals
to me tremendously. \endpara}

\slide{\title{Then and Now}
 \list{\bullets}
  \item Traditionally: freedom of speech, freedom of
        expression, freedom of information $\ldots$
  \item Currently: information is becoming a commodity and societies
        have always regulated commodities.
  \endlist}

\talk{\para The next point, and I think a very down-to-earth one, is
that traditionally we have these principles of freedom of
speech, freedom of expression, freedom of information
% freedom of association
\figref{(Figure 10-9)}.  But we are moving into a society where
information is a commodity.  And societies have always
regulated commodities, so what possible hope do we have
of keeping the information indefinitely unregulated?
Cryptography is a technology with the potential to make
data unregulatable in many cases, so I think we have to
expect a lot of opposition to a lot of its uses. \endpara}

\slide{\title{Data vs. Matter}
 \para Information is not as dangerous as matter and its transmission
 does not need the same controls.  Information about how to build a
 bomb is dangerous, but it is not a bomb. \endpara}

\talk{\para I think, however, that there's certainly something to be
said for the notion that date are less dangerous than matter.
I mean, it may be dangerous if I stand up here and explain
how an H-bomb works, but it's nothing like as dangerous
as if I brought an H-bomb with me. \ednote{(laughter)} \endpara}

\slide{\title{Can Crypto Be Regulated?}
 \centerline{{\it ``If guns are outlawed, only outlaws will
                  have guns.''}}
 \vskip 2ex
 \list{\bullets}
  \item Crypto is done on standard processors.
  \item Crypto can be done by small programs.
  \item Covert channels are hard to detect.
  \item Enforcement would require drastic measures.
  \endlist}

\talk{\para So now I come to the issue, how well could cryptography
be regulated?  I think this is a case where --- I think John
Barlow quoted this slogan --- if guns are outlawed, only
outlaws will have guns.  I think that actually applies much
better in this case because cryptography grows much more
naturally our of mainstream computer science than
ordnance technology does our of mainstream home
machining.  Virtually everybody does cryptography on
standard microprocessors or digital signal processors unless
they want to go really, really fast.  I mean, people build lots
of special purpose crypto chips but lots of people also do it
on standard processors for the same reason that everybody
does everything on standard processors --- that's the cheap
way to do it.  You can do crypto in very small programs.
Even DES, which is a rather complicated crypto
system, is not a very large program, as programs go.  And I
think now, looking at that slogan, right?, if you pass a law
against using cryptography in some circumstances, all of
the people who are normally law-abiding will abide by it,
but somebody who really wants to get messages through
will build covert channels to carry them.  We discovered in
computer security that even within the rather controlled
environment of computer operating systems, bits move
through covert channels rather freely and they are very
hard to find and very hard to limit.  So, as a result, the
enforcement of any such rules would probably require very
drastic measures. \endpara}

\slide{\title{Roots of the Policy Problem --- 3}
 \list{\bullets}
  \item Restricting crypto threatens to undermine society's
        accountability.
  \item By allowing search warrants access to private data,
        we risk allowing criminals access as well.
  \vskip 1ex
  \item (There is a similar tradeoff in copy protection
        technology.)
  \endlist}

\talk{\para Well, down to the last two points here.  I think that if you
do what seems to be proposed in these various bills that
come up, which is to limit the quality of security technology
available to individuals, that you risk limiting two things
\figref{(Figure 10-12)}.  In the first place, you risk the accountability
of the government and the society itself.  That is to say, that
everybody worries, and I think correctly so, that warrants
may not be necessary for some people.  The fact they can't
present the evidence they gained by their warrant-less
wiretaps doesn't mean they can't then build other evidence
based on them. \endpara

\para The second point is that a lot of the case are going to be
exposing you to the actions of somebody other than the
courts and the sheriff and the FBI, so they're effectively
saying, you have to accept the level of protection we provide
you in whatever way --- you aren't allowed to protect
yourself.  Now it seems to be one thing to say that you
can't have a tommy gun in your house to protect yourself
against burglars, and quite another to say that you
can't have a really strong door.  I think that's a much more
similar analogy.  There's also a very interesting precedent
for this, I think, in the copy protection of things
like digital audiotapes.  In some sense, putting copy
protection in digital audiotapes supports not only the
interest of Columbia Records but it supports the interest of
the black market.  If I'm selling black market audiotapes
which I copied form legitimate ones by building an illegal
--- I don't even know if it's illegal, but you build from
scratch --- a nonrestricted copier.  Now I've copied copy-protected
tapes.  I mean, I don't want my customers copying
their own tapes --- I want them coming to me. \ednote{(laughter)} \endpara}

\slide{\title{What Would Be a Good Compromise?}
 \list{\bullets}
  \item Data searchable by societal action.
  \item Search would always be detectable by the victim.
  \item No evidence this is possible!
  \endlist}

\halftext{\para What would be ideal is a mechanism whereby society
could execute a search warrant if it really needed to, but it could
not do it without you're knowing you had been searched. \endpara}

\talk{\para So, what could we do?  The answer is, I don't know.  I have
in mind a compromise \figref{(Figure 10-13)}, but I don't know
how to do it.  I think if you could guarantee accountability
of the society, that is to say, you could say, data always
could be subject to search warrants, but nobody could ever
do it covertly.  It would be like having a perfect signet ring
seals to go on envelopes so that if your letter was opened
you would know that some warrant had been served and
your letter had been opened.  The trouble is, unfortunately,
I have no idea how to do that.  Thanks for your patience.
\ednote{(applause)} \endpara}

\endapage
\enddoc

