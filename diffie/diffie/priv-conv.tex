% priv-conv.tex - private conversation

\input talk-fmt

\sendtty{\botrighttag}

%\def\text{\halftext}

\newtoks\fulltitle \newtoks\shorttitle
\def\theauthor{Whitfield Diffie}
\fulltitle={Private Conversation: Human Right or Privilege of Power?}
\shorttitle=\fulltitle
\def\coursenumber{ETH EE Colloquium on Electronics and Communication}
\talktitle={\coursenumber\ \classdate\ \the\fulltitle}

\macromode
\def\chapstartslide
{\slide{\title{\vbox{\halign{\ctr{##} \cr
		       Private_Conversation: \cr
		       Human_Right_or_Privilege_of_Power? \cr}}}
       %\title{\the\fulltitle}
       \vskip 2ex
       \centerline{Whitfield_Diffie}
       \vskip 1ex
       \centerline{Sun_Microsystems}
       \vskip 5ex
       \centerline{Colloquium_on_Electronics_and_Communication}
       \centerline{Electrical_Engineering_Department}
       \centerline{Swiss_Federal_Institute_of_Technology,_Zurich}
       \vskip 1ex
       \centerline{21_May_1993}}}

\endmacromode

%\oneslide%\def\slideborder#1{\trim{#1}{\borderheight}{\borderwidth}}%
\sixslides\textwidth=7truein%\slidesonly%

%\input remarkform
%\input byhandform

\class{1}{21 May 1992}
\chapter{1}{Introduction}{INTRO}

\chapstartslide

\slide{\title{New U.S. Government Proposal}
 \list{\bullets}
  \item Encryption for unclassified government traffic
  \item Federal standard is \fw{de-facto} commercial standard
  \item Secret encryption algorithm in tamper resistant chips
  \item Back door allows government to decrypt the traffic
  \endlist}

\endrow


\slide{\title{The Skipjack Algorithm}
        \vskip 3ex
        \list{\bullets}
        \item Conventional Cryptosystem
        \item 64 bit block \& FIPS 81 Modes
        \item Replaces standard DES chips
        \item Eighty bit key
        \item Details secret
        \endlist}


\slide{\title{The `Clipper' Chip}
        \vskip 3ex
        \list{\bullets}
        \item Mykotronx MYK--78
        \item Secret Skipjack algorithm
        \item IV generation feature
        \item Law Enforcement Exploitation Block
        \item Unit key and family key
        \item Tamper resistant
        \endlist}

\slide{\title{The Capstone Chip}
        \vskip 3ex
        \list{\bullets}
        \item Mykotronx MYK--80
        \item Skipjack algorithm
        \item Random number generator
        \item Key exchange mechanism
        \item Digital Signature Standard
        \item Secure Hash Algorithm
        \item To be used for email on Defense Message System
        \endlist}

\slide{\title{The Wiretap Aspects}
        \vskip 3ex
        \list{\bullets}
        \item Family key common to many chips (not all)
        \item Unit key unique to chip
        \item Unit key decrypts LEEF
        \item LEEF contains session key
        \item Unit keys stored by two escrow agents
        \item (Like bypass key on some locks.)
        \endlist}

\slide{\title{Strong Key Negotiation Mechanism}
\list{\Letters.}
  \item A and B are issued certificates of their own public
        keys by a \df{Certifying Authority}.
  \item A and B exchange certificates.
  \item A and B verify eachother's identities by signing challenges,
        negotiating a conventional key in the process.
  \item Communication is encrypted using the negotiated key to
        assure continuing authenticity and privacy.
  \endlist}

\halftext{\para The use of public key permits secure communication
without the online assistance of a central resource. \endpara}

\slide{\title{The Essence of Security}
       \list{\bullets} %\listmar=3em
             \item Recognition of those you know
             \item Introduction to those you don't know
             \item Written Signature
             \item Private Conversation
             \endlist

\vskip 2ex
\para Communication security is the transplantation of these basic
social mechanisms to the telecommunications environment. \endpara}

\halftext{\para The most important security measures in which any of
us engage in our daily business have nothing to do with safes, locks,
guards, or badges: \begdis
 \list{\bullets}
  \item We recognize our colleagues
  \item We affix our signatures to letters, contracts, and records.
  \endlist

\enddis It is also noteworthy that in our communications, both
written and spoken, we take for granted that by sealing the envelope
or closing the door we can achieve privacy in our communications. \endpara

\para The challenge of modern security technology is to transplant
these familiar mechanisms from the traditional world of face-to-face
meetings and pen and ink communications to a world in which digital
electronic communications are the norm and the luxury of personal
encounters or handwritten messages are the exception. \endpara}

\slide{\title{Essence of Security (Cont'd)}
 \heading{What do you do over:}
 \list{\bullets}
  \item Electronic mail?
  \item Telephone?
  \item Video Conference?
  \endlist
 \heading{Essential social mechanisms}}

\slide{\title{Many Modern Technologies Decrease Privacy}
% Police playing catchup?
 \list{\bullets}
  \item Credit (and other) databases
  \item Hidden video cameras
  \item Metal detectors
  \item X-ray, NMR, Gravetomometer
  \item Night vision and micro-seismometers
  \item Rafter
  \item \hbox{\qquad$\ldots$}
  \endlist}

\slide{\title{Cryptography Can Increase Privacy}
 \list{\bullets}
  \item Secure phone calls
  \item Secure electronic mail
  \item Secure diaries
  \vskip 2ex
  \item Electronic Elections
  \item Digital cash
  \endlist}
% right bracket and word anonymity to right of voting and cash

\slide{\title{Cryptography Can Increase Accountability}
 \list{\bullets}
  \item Digital Signatures permit auditing.
  \vskip 5ex
  \item Auditing is the basis for investigation.
  \endlist}

\slide{\title{Communications Technology vs. Privacy}

 \list{\bullets}
  \item Telecommunications gave police and intelligence organizations
	a systematic way of spying on large numbers of people.

  \item Cryptography gives individuals a systematic way of protecting
	their communications from spies.

  \vskip 2ex

  \item (The spies don't like this?)

  \endlist}

\slide{\title{Communications and Police Work}

 \list{\bullets}
  \item Modern communications have dramatically
	eased police work.
  \item Modern wiretaps provide \df{caller ID}.
  \item Never know if caller is recording call ---
	even over a secure phone.  (Easier than
        `\df{federal body wire}.')
  \endlist}

\slide{\title{Impact of Communications Technology (Cont'd)}
 \list{\bullets}
  \item Telecommunication violates a traditional `locality'
        of human society.
  \item This locality has been a major mechanism of
        individual accountability.
  \item The tremendous success of communications intelligence 
	is a result of this lack of locality.
  \item Lack of security in communication has in some ways
        replaced locality.
  \item Association only visible to spies through \df{traffic
	analysis}.
  \endlist}

\slide{\title{Can Cryptography Be Regulated?}
 \centerline{{\it ``If guns are outlawed, only outlaws will
                  have guns.''}}
 \vskip 2ex
 \list{\bullets}
  \item Crypto is done on standard processors.
  \item Crypto can be done by small programs.
  \item Covert channels are hard to detect.
  \item Enforcement would require drastic measures.
  \endlist}

\slide{\title{\vbox{\hbox{Will Crypto Be Undone}%
                    \hbox{\quad by Implementation}}}
 \list{\bullets}
  \item Tempest
  \item Tamper Resistance
  \item Random Number Generation
  \item Reliability
  \endlist
 \vskip 2ex
 \para Is cryptography a mathematical science or a military
       science? \endpara}

\slide{\title{What is the Future of Privacy?}
 \list{\bullets}
  \item American Bill of Rights didn't enumerate a right of
	private conversation.  (Who thought it could
        be prevented?)

  \item Telecommunications promote intimate long term
	business and personal relationships between
        people who never meet in person.

  \item If people are not allowed to protect their
	communications, only the wealthy and powerful
        will have access to privacy.
  \endlist}


\enddoc

