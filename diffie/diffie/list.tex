% list.tex - multiple level indented lists with (possibly) numbered bullets.

\sendtty{loading: /files/home/diffie/cur/list.tex}
\macromode
        % Restore first argument from \listlevel'th one of the latter four.
\def\getit#1#2#3#4#5{\ifcase\listlevel \or #1=#2\or #1=#3\or #1=#4\or #1=#5\fi}

                % The Basic List Functions

\def\list#1{\par \begingroup \curmark={#1}
            \rightskip=0pt plus 10em    % ragged right
            \advance\listlevel by1_
            \itemnumber=0_
            \getit{\leftdent}{\ldenta}{\ldentb}{\ldentc}{\ldentd}
            \advance\leftskip by\leftdent_
            \getit{\rightdent}{\rdenta}{\rdentb}{\rdentc}{\rdentd}
            \advance\rightskip by\rightdent_
            \getit{\parindent}{\parina}{\parinb}{\parinc}{\parind}
            \getit{\parskip}{\parska}{\parskb}{\parskc}{\parskd}}

\def\item{\ifnum \listlevel<1_{\sendtty{List_Item_Not_in_List}} \fi
          \par\noindent\advance\itemnumber by1_
          \llap{\the\curmark\hskip \the\markskip}}

\def\markitem#1{{\curmark={#1}\item}}

\def\endlist % End a list
   {\ifnum \listlevel<1_{\sendtty{Extra_ENDLIST}} \fi
    \par \endgroup}

                % List Marker Utility Mechanisms

% \list is the start of list command.  The various different kinds
% are obtained by following this with a ``declaration:'' e.g.,
% \list{(\numbers)} \item ... \endlist

% Four types of variable markers are given below.  These can be used
% alone or in combination with constant text to make a variety of forms.
% Constant text alone will produce the dashed and bulleted forms, but
% the forms \bullets and \dashes given below avoid tedious and ugly text
% in list headers.

\def\bullets{$\bullet$}
\def\dashes{{\rm ---}}
\def\dots{$\cdot$}
\def\letters{\alph{\the\itemnumber}}
\def\Letters{\Alph{\the\itemnumber}}
\def\numbers{\the\itemnumber}
\def\roman{\expandafter\romannumeral\the\itemnumber}
\def\Roman{\uppercase\expandafter{\romannumeral\the\itemnumber}}

        %% Initialization and dimensions

\newcount\itemnumber
\newcount\listlevel \listlevel=0  % list nesting level
\newdimen\leftdent \leftdent=0pt
\newdimen\ldenta \newdimen\ldentb \newdimen\ldentc \newdimen\ldentd
\ldenta=3em \ldentb=2em \ldentc=1em \ldentd=2em
\newdimen\rightdent \rightdent=0pt
\newdimen\rdenta \newdimen\rdentb \newdimen\rdentc \newdimen\rdentd
\rdenta=1em \rdentb=1em \rdentc=1em \rdentd=1em
\newdimen\parina \newdimen\parinb \newdimen\parinc \newdimen\parind
\parina=0pt \parinb=0pt \parinc=0pt \parind=0pt
\newskip\parska \newskip\parskb \newskip\parskc \newskip\parskd
\parska=0pt \parskb=0pt \parskc=0pt \parskd=0pt
\newdimen\markskip \markskip=.5em
\newtoks\curmark

\endmacromode

 % Previous versions: 2 Jul 83 and 29 Jun 83
 % 20 Sep 83 - Added \dots to list types
 % 19 Oct 83 - List margin now recomputed in each \startpar
 % 19 Dec 83 - Restoration of ``old'' margin moved to right place

% 13 May 90 - \listlevel<0 ==> \listlevel<1 in \endlist
% 28 Apr 92 - moved \alph and \Alph to texinit.

