% slidetext.tex - mixed slides and text in several sized and formats.

\input slideborder

\ig{\input /usr/staff/diffie/tex/macros/pageout} % Using \plainoutput
                                                 % from plain.tex

% The slidetext format mixes slides and text on a page.

%       A---------------------------------------B
%       |                                       |
%       |    Left Head          Right Head      |
%       |   E--------------    --------------F  |
%       |   |   -------    |  |    <title>   |  |
%       |   |  | Slide |   |  | - bullet     |  |
%       |   |  | Title |   |  | - bullet     |  |
%       |   |   -------    |  |              |  |
%       |    --------------    --------------   |
%       |                 1                 2   |
%       |                                       |
%       |   a--------------b  i half text   j   |
%       |   | e----------f |  half text  half   |
%       |   | |          | |  text  half text   |
%       |   | |          | |  half text  half   |
%       |   | g----------h |  text half text    |
%       |   c--------------d  k half text   l   |
%       |         <slideno>                     |
%       |                                       |
%       |   m text text text text text text n   |
%       |   text text text text text text       |
%       |   text text text text text text       |
%       |   text text text text text text       |
%       |   text text text text text text       |
%       |   G text text text                 H  |
%       |                                       |
%       |                                       |
%       |    Left Foot   <pageno>  Right Foot   |
%       |                                       |
%       C---------------------------------------D

\newdimen\paperheight   % height of paper: A to C or B to D.
\newdimen\paperwidth    % width of paper: A to B or C to D.

\newdimen\pageheight    % height of page: E to G or F to H.
\newdimen\pagewidth     % width of page: E to F or G to H.
% A page typically holds several slides.

\newdimen\slideheight   % height of page within slide: e to g or f to h.
\newdimen\slidewidth    % width of page within slide: e to g or f to h.

\newdimen\borderheight  % height of slide border: a to c or b to d.
\newdimen\borderwidth   % width of slide border: a to c or c to d.

\newdimen\textheight    % height of text annotating slides: m to G
\newdimen\textwidth     % width of text annotating slides: m to n

\newdimen\halftextwidth % width of text annotating half slides:
                        % i to j or k to l.

\newdimen\frameheight   % height of frame, whether slide or text
\newdimen\framewidth    % width of frame, whether slide or text

\newdimen\vershim       % vertical distance between slides
\newdimen\horshim       % horizontal  distance between slides
\newdimen\numshim       % distance from bottom of frame to slide number

\newdimen\fullslidewidth \fullslidewidth=7in
\newdimen\fullslideheight \fullslideheight=8.5 in
\newdimen\fullborderwidth \fullborderwidth=7.5in
\newdimen\fullborderheight \fullborderheight=9in
%\newdimen\fullborderwidth \fullborderwidth=\pagewidth
%\newdimen\fullborderheight \fullborderheight=\pageheight
\newdimen\halfslidewidth \halfslidewidth=7in
\newdimen\halfslideheight \halfslideheight=5.5in
\newdimen\halfborderwidth \halfborderwidth=7.5in
\newdimen\halfborderheight \halfborderheight=6in

\newcount\rowlim        % maximum number of rows of slides per page
\newcount\collim        % maximum number of columns of slides per page

\newcount\colnum        % number of the column in which a slide sits
\newcount\rownum        % number of the row in which a slide sits

\newbox\rowbox          % box containing slides being assembled into a row
\newbox\pagebox         % box containing rows being assembled into a page

\newcount\slidecount
\newcount\contslidecount
\newtoks\slidetitle
\newtoks\talktitle

\newcount\oneslidemag
\oneslidemag=2400  % ratio of oneslide size to sixslide size

% A tiny font used to write date of production, slide number, etc.
% at the bottom of pages is specified in each of the styles because
% it would otherwise occur before \magnification and since it
% is in truepts it would cause a magnification error.

\macromode
\def\slidestyleinit
% set all the paramters that are global to a set of slides
{\font\tinyfont=cmr5_at_5truept
 \slidecount=0  % reset slide number
 \colnum=0      % reset column number
 \rownum=0      % reset row number
 \pageno=1
 %? These were previously global, but that doesn't work if they are
 %? variables rather than macros
 \textheight=\borderheight
 \textwidth=\pagewidth
 \halftextwidth=\borderwidth
 \hsize=\pagewidth \vsize=\pageheight}

% Slides only:  omit \text and \halftext and the formatting
% commands \endrow and \endpage.

\def\slidesonly{\long\gdef\text##1{} \long\gdef\halftext##1{}%
                \gdef\endrow{} \gdef\endpage{}}


% One slide per page.
%? At present this slide is in landscape mode.  A portrait mode slide
%? could be defined equally easily.

\def\oneslide   % 7 by 9 border in landscape mode
{\magnification=\oneslidemag
 \vin=.45in % will be magnified by 2.4 to equal 1.08in
 \rowlim=1              \collim=1
 \slideheight=180pt     \slidewidth=240pt
 \borderheight=210pt    \borderwidth=270pt
 \vershim=0pt           \horshim=0pt    \numshim=2pt
 \pageheight=\borderheight      \pagewidth=\borderwidth
 \voffset=-1truein % Figure from actual top left corner of paper
 \hoffset=-1truein
 \paperheight=8.5truein \paperwidth=11truein
 \gdef\botrighttag
      {{\tinyfont \the\talktitle\ ---\ \datetime\ --- \slidenum}}%

 \slidesonly{}
 \let\fullslide=\slide \let\halfslide=\slide
 \gdef\pageout{{\hsize\paperwidth \vsize\paperheight
                \shipout\vbox to \paperheight
                             {\vfill\hbox to \paperwidth
                                         {\hfill\box255\hfill}\vfill}}%
               \advancepageno}
 \gdef\topleftgap{} \gdef\toprightgap{} \def\botleftgap{}
 \landscapemode
 \slidestyleinit}

\def\sixslides
{\vin=.45in
 \rowlim=3 \collim=2
 \slideheight=180truept         \slidewidth=240truept
 \borderheight=210truept        \borderwidth=270truept
 \vershim=15truept      \horshim=10truept       \numshim=2truept
 \pageheight=10in       \pageheight=660truept
 \pagewidth=7.5truein   \pagewidth=550truept
 \voffset=-.3truein     % Change top margin to .5 in for 9.4 in page
 \hoffset=-.5truein     % Change left margin to .5 in for 7.5 in page
 \paperheight=11truein  \paperwidth=8.5truein
 % This \plainoutput differs from the plain.tex version in moving
 % the titles a little further from the page proper.  It should be
 % possible to do this by adjusting \makeheadline and \makefootline
 \gdef\plainoutput{\ifodd\pageno \hoffset=-.4truein \else
                                 \hoffset=-.8truein \fi
                   \shipout\vbox{\makeheadline\vskip2ex
                                 \pagebody\vskip2ex\makefootline}%
                   \advancepageno
                   \ifnum\outputpenalty>-10000 \else\dosupereject\fi}
 \gdef\pageout{\plainoutput}
 \gdef\topleftgap{} \gdef\toprightgap{} \def\botleftgap{}
 \def\endrow{\endarow} \def\endpage{\endapage}
 \let\fullslide=\realfullslide \let\halfslide=\realhalfslide
 \portraitmode
 \slidestyleinit}

\def\fifteenslides
{\vin=.3in % magnification may not work for this
 \rowlim=5 \collim=3
 \slideheight=120truept         \slidewidth=160truept
 \borderheight=135truept        \borderwidth=180truept
 \vershim=5truept       \horshim=7truept        \numshim=2truept
 \pageheight=10in       \pageheight=750truept
 \pagewidth=7.5truein   \pagewidth=554truept
 \voffset=-.5truein     % Change top margin to .5 in for 9.4 in page
 \hoffset=-.6truein     % Change left margin to .5 in for 7.5 in page
 \paperheight=11truein  \paperwidth=8.5truein
 \gdef\pageout{\plainoutput}
 \gdef\topleftgap{} \gdef\toprightgap{} \def\botleftgap{}
 \def\endrow{\endarow} \def\endpage{\endapage}
 \let\fullslide=\realfullslide \let\halfslide=\realhalfslide
 \portraitmode
 \slidestyleinit}

\def\landscapemode
{\def\enddoc{\sendtty{}
             \sendtty{\tab\tab\tab ******************}%
             \sendtty{\tab\tab\tab *    \tab\tab    *}%
             \sendtty{\tab\tab\tab *_LANDSCAPE_MODE_*}%
             \sendtty{\tab\tab\tab *    \tab\tab    *}%
             \sendtty{\tab\tab\tab ******************}%
             \sendtty{}
             \endapage\end}}

\def\portraitmode
{\def\enddoc{\sendtty{}
             \sendtty{\tab\tab\tab **********}%
             \sendtty{\tab\tab\tab *  \tab  *}%
             \sendtty{\tab\tab\tab *PORTRAIT*}%
             \sendtty{\tab\tab\tab *  \tab  *}%
             \sendtty{\tab\tab\tab *\spc\spc MODE \spc\spc*}%
             \sendtty{\tab\tab\tab *  \tab  *}%
             \sendtty{\tab\tab\tab **********}%
             \sendtty{}
             \endapage\end}}

\def\slidenum
{\the\slidecount\ifnum \contslidecount>0 {\alph\contslidecount}\fi}

% Two types of large slides.  Not, perhaps, entirely debugged.
% \fullslide is a full page regardless of the packed mode.
% \halfslide is two thirds of a page in 6 per page mode.

\long\def\realfullslide
#1% text of fullslide
{\endapage
 \def\frameout{\fullslideout}
 \frameheight=\fullslideheight \vsize\frameheight
 \framewidth=\fullslidewidth \hsize\framewidth
 \output{\frameout}
 #1 \par\vskip\botskip\eject
 \endapage}

\def\fullslideout
{{\global\advance\slidecount by 1
  \def\borderheight{\fullborderheight}
  \def\borderwidth{\fullborderwidth}
  \addframe{\slideborder{\box255}}
  \sendtty{\tab\tab Full_Slide_No._\the\slidecount
                    (page_\the\pageno):_\the\slidetitle}}}

\long\def\realhalfslide
#1% text of halfslide
{\def\frameout{\halfslideout}
 \frameheight=\halfslideheight \vsize\frameheight
 \framewidth=\halfslidewidth \hsize\framewidth
 \output{\frameout}
 #1 \par\vskip\botskip\eject}

\def\halfslideout
{{\endarow \ifnum \rownum>1 \endapage \fi
  \global\advance\slidecount by 1
  \def\borderheight{\halfborderheight}
  \def\borderwidth{\halfborderwidth}
  \global\advance\rownum by 1_
  \addframe{\slideborder{\box255}} \endarow}
  \sendtty{\tab\tab Half_Slide_No._\the\slidecount
                (page_\the\pageno):_\the\slidetitle}}

\long\def\halftext
#1% the text
{\def\frameout{\halftextout}
 \frameheight=\textheight \vsize\frameheight
 \framewidth=\halftextwidth \hsize\framewidth
 \output{\frameout}
 #1 \par\vfil\eject}

\def\halftextout
{\addframe{\vbox to \textheight{\vfil\unvbox255\vfil}}}

\long\def\slide
#1% contents of the slide
{\contslidecount=0
 \frameheight=\slideheight \vsize\frameheight
 \framewidth=\slidewidth \hsize\framewidth
 \def\frameout{\slideout}
 \output{\frameout}
  #1 \par\vskip\botskip\eject}

\def\slideout
{\global\advance\slidecount by 1
 \addframe{\slideborder{\box255}}
 \sendtty{\tab\tab\spc Slide_No._\the\slidecount\spc\spc
                (page_\the\pageno):_\the\slidetitle}
 \global\def\frameout{\contslideout}
 \output{\frameout}}

\def\contslideout
{\global\advance\contslidecount by 1
 \addframe{\slideborder{\box255}}
 \sendtty{\tab Overflow_Slide_No._\the\slidecount
               \alph{\the\contslidecount}\spc
               (page_\the\pageno):_\the\slidetitle}}

\long\def\text
#1% the text
{\def\frameout{\textout}
 \frameheight=\textheight \vsize\frameheight
 \framewidth=\textwidth \hsize\framewidth
 \output{\frameout}
 #1 \par\vfill\eject}

\def\textout{\endarow
               \setbox\rowbox\hbox{\vbox to \textheight{\vfil\unvbox255\vfil}}
               \endarow
               \output{{\def\vershim{0pt}\textout}}}

\def\endarow
   {\global\colnum=0
    \ifvoid \rowbox \else
     {\global\advance\rownum by 1
      \global\setbox\pagebox
       \vbox{\unvcopy\pagebox
             \ifvoid \pagebox \else \vskip\vershim \fi
             \hbox to \pagewidth{\hfil \unhbox\rowbox \hfil}}
      \ifnum \rowlim=\rownum{\endapage} \fi} \fi}

% From slide routines developed for three column Scientific American
% article format - Fall 1990

\def\endapage{\endarow \global\rownum=0
              \ifvoid \pagebox \else \global\output{\pageoutput}
                                      \global\hsize=\pagewidth
                                      \global\vsize=\pageheight \fi
              \unvbox\pagebox \par\vfill\eject}

\def\pageoutput{\pageout
                \global\output{\frameout}
                \global\vsize=\frameheight
                \global\hsize=\framewidth}

\def\addframe
#1% frame
{\global\setbox\rowbox
 \hbox{\unhcopy\rowbox \ifvoid \rowbox \else \hskip\horshim \fi #1}
 \global\advance\colnum by 1_
 \ifnum \colnum=\collim {\endarow} \fi}

% Copied in from fifteen-fmt.  Previously used \null\slidetopglue
\topskip=0pt plus 1fil
\newskip\botskip \botskip=0pt plus 1fil
\newskip\titleskip \titleskip=1ex plus .1fil

\def\title
#1% the title
{\centerline{\boxit{#1}}
 \vskip \titleskip
 \slidetitle=\expandafter{#1}}

\def\enddoc{\endapage\end}

%? \commentary occurs in some files and is intended to be used in
%? a slidecues form.  \leftlabel and \rightlabel come from Lynn
%? Price's old versions.  Why they have been defined to do nothing
%? escapes me.
\def\commentary#1{} \def\leftlabel#1{} \def\rightlabel#1{}

% copied in from course-fmt.tex - 6 Dec 90
\def\cit#1#2{}

\newdimen\vin % virtual inch.  This replaces some cases of the old \VU
\let\VU=\vin  % but not all.  Both should probably be replaced
              % by `in' and use of the magnify feature.

\endmacromode

% slidetext.tex --- macros to mix slides and text
% previous version --- spring 1991
% changed \savetitle to a tokenlist --- 14Sep91
% changed \savetitle to \slidetitle --- 26Nov91
% changed \textheight, \textwidth, \halftextwidth to counts --- 26Nov91
% change \slidebotglue to \botskip --- 1 Jan 92
% changed \endoc in \oneslide to avoid extra page --- 9Feb92
% spaces ==> underlines in `LANDSCAPE MODE' in \oneslide --- 9Feb92
% \title altered to include \expandafter --- 8Mar92
% added space before --- in tiny print at bottom of oneslide borders - 9Mar92

% \titleglue ==> \titleskip --- 26 Apr 92
%box255 ==> \unvbox255 in \textout --- 26 Apr 92
% removed extra grouping in definition of slide --- 26 Apr 92
% added \s \ss \sss to produce one, two, or three spaces in
% \sendtty output.  These are defined locally so as not to
% conflict with the German \ss.  If I could think of the rigth
% names I would put them in texinit with \tab.

