% texinit.tex - basic capabilities presumed present by all TeX programs.

        %%% Define the Control Characters not in plain TeX

\catcode'15=10  % !!!! carriage return is a space !!!!
                % i.e., multiple carriage returns don't break paragraphs
                % this declaration is repeated in \endmacromode
\catcode'174=4  % | as allignment tab
\catcode'176=8  % ~ for subscripts

        %%% Macros for Printing the Control Characters

\def\%{\char'45 }%      % Note, the space after 45 is needed! (e.g.\%0)
\chardef\@=`\@
\def \\{{\ty\char'134 }}%
\def \leftbrace{{\ty\char'173 }}%
\def \rightbrace{{\ty\char'176 }}%
\def \dollar{{\ty\char'044 }}%
\def \#{{\ty\char'043 }}%
%% wonder what this is for??
\def \|{\setbox9\hbox{\ty A}%
        \hbox to \wd9
             {\ctr{\lower .3ex\vbox to 2ex
                                   {\hrule width .05em height .8ex depth 0pt
                                    \vfill
                                    \hrule width .05em height .8ex}}}}

        %%% Various New Control Structures

                %% Macro Mode

% In macro mode, all spaces, tabs, and carriage returns are ignored.
% Any desired whitespace must be indicated explicitly by the use of
% underline.  The Boolean conditions:

        \newif\ifmacromode              % if in macro mode
        \newcount\macromodelevel        % recursion level

% are used to indiate whether macro mode is in force and to allow
% macro mode to be called recursively from within itself.  This
% avoids accidents in which macro mode is prematurely turned off,
% for example because a file containing \macromode and \endmacromode
% declarations is \input from within a segment of that is itself
% in macro mode.  Macromode explicitly does not use grouping because
% macros defined in macro mode are not intended to be local.

  \def\macromode
     {\advance \macromodelevel by 1
      \ifmacromode \else
        \catcode '015=9 % make <cr> an ignored character
        \catcode '040=9 % make space an ignored character
        \catcode '011=9 % make tab an ignored character
        \catcode '137=10        % make underline equivalent to space
        \macromodetrue
       \fi}%

  \def\endmacromode
     {\ifmacromode
       \advance \macromodelevel by -1
       \ifnum\macromodelevel<1
        \macromodefalse
        \catcode '015=10        %make carriage return a space
        \catcode '040=10        %make space a space
        \catcode '011=10        %make tab a space
        \catcode '137=12        %make underline an ``other'' character
        \catcode`\_=8           % underline for subscripts
        \fi \else \errmessage{Not in macro mode}\fi}

\macromode

                %% Macro that ignores its argument

\long\def\ig#1{}

                %% Macros for Testing the Emptiness of Arguments

\def\ifnull#1#2#3{\setbox8\hbox{#1}\ifdim_\wd8>0pt_{#3}_\else_{#2}}
\def\ifnonnull#1#2#3{\setbox8\hbox{#1}\ifdim_\wd8>0pt_{#2}_\else_{#3}\fi}

                %% Send to Teletype

% send a message to the teletype.  This function has no effect
% on the document

  \def\sendtty#1{\immediate\write16{#1}}

\endmacromode

        %%% Paragraph Structure Macros

% In order to facilitate an open style of layout in the manuscript,
% carriage return is made equivalent to a space

                \catcode'15=10

% so that double carriage returns don't break paragraphs.  Paragraph
% breaking is done by  explicit markers \para and \endpara.  The
% markers \begdis and \enddis mark the beginnings and ends of displays
% within paragraphs.  These need not be mathematical formulae, but might
% be lists, illustrations, etc.  Paragraph and display declarations must
% be correctly related.  The following macros do not do much that
% affects the document.  The \endpara is equivalent to \para and the
% \enddis does a \noindent.  They mostly issue warnings if the following
% rules are violated:
%       Paragraphs must begin before they end;
%       Displays must begin before they end;
%       Paragraphs do not occur within paragraphs;
%       Paragraphs do not occur within displays;
%       Displays do not occur within displays;

% Boolean conditions

\macromode

        \newif\ifinpar % if in paragraph
        \newif\ifindis % if in display

% keep track of whether we are inside a paragraph, inside a display,
% or neither

% Make \endpar a synonym for TeX's standard \par control sequence.
% The conveniently short sequence \par can now (although it isn't at
% present: Jan 92) be used for some other purpose.

                \let \endpar=\par

% beginning of a paragraph

  \def\para{\ifinpar {\errmessage{Missing endpara}}\else {}\fi \inpartrue}

% end of a paragraph

  \def\endpara{\ifinpar {}\else {\errmessage{Missing para}}\fi
               \inparfalse
               \endpar}

% beginning of a display

  \def\begdis{\ifindis {\errmessage{Missing enddis}}\else {}\fi \indistrue}

% end of a display

  \def\enddis{\ifindis {}\else {\errmessage{Missing begdis}}\fi
              \indisfalse
              \noindent}

\newbox\voidbox
\def\hmode{\unhbox\voidbox}     % Force \TeX into horizontal mode.
\let\vmode=\endpar              % Force \TEX into vertical mode.

\def\endpara{\endpar}           % End of paragraph
\def\para{\hmode}               % Beginning of paragraph

\def\begdis{\vmode}                     % Begin display within paragraph
\def\enddis{\ifvmode{\noindent}\else{}\fi} % End display within paragraph

\def\enddoc{\endpar\vfil\end} % Uniform use of \enddoc replaces \endletter,
                              % etc.  The single \enddoc control sequence
                              % is redefined by each format. WD 21 Oct 82

        %%% Boxes and glue macros

\def\pinch#1{\hbox to 0pt{#1}}

\def\lft#1{#1\hss} \def\rt#1{\hss#1} \def\ctr#1{\hss#1\hss}

\def\llap#1{\pinch{\rt{#1}}} \def\rlap#1{\pinch{\lft{#1}}}
\def\clap#1{\pinch{\ctr{#1}}}

\def\crush#1{\vbox to 0pt{#1}}
\def\tp#1{\vss#1} % vertical analog of \lft
\def\bt#1{#1\vss} % vertical analog of \rt
\def\vctr#1{\vss#1\vss} % vertical analog of \ctr
\def\tlap#1{\crush{\tp{#1}}} % vertical analog of \llap
\def\blap#1{\crush{\bt{#1}}} % vertical analog of \rlap
\def\vclap#1{\crush{\vctr{#1}}} % vertical analog of \clap


                %% Macros for placing white or black borders around boxes

% To border something is to put whitespace around it; to frame something
% is to put blackspace around it.  The most common forms of both border
% and frame have fourfold symetry and the basic functions border and
% frame thus take two arguments: and object and a dimension.  It is
% also convenient to put something into a (white) box of given dimensions
% and this function, trim, takes three arguments: #1 is the object; #2
% is the vertical  dimension; #3  is the horizontal  dimension.
% In full generality it is possible to surround something with white or
% black areas whose dimensions vary with direction.  At present these
% functions are called fourborder and fourframe: #1 is the object; #2 is
% the upper vertical  dimension;  #3  is  the  lower  vertical
% dimension; #4 is the left horizontal dimension; #5 is the right horizontal
% dimension.  Positive arguments place the frame outside the box being
% framed, negative ones place it inside.  Finally there is the special
% case of boxit, which has only the object as argument as its border and
% frame dimensions are fixed.

\def\boxit#1{\frame{\border{#1}{3pt}}{.4pt}}
\def\border#1#2{\fourborder{#1}{#2}{#2}{#2}{#2}}
\def\frame#1#2{\fourframe{#1}{#2}{#2}{#2}{#2}}
\def\trim#1#2#3{\vbox to #2{\vctr{\hbox to #3{\ctr{#1}}}}}
\def\contract#1{\trim{#1}{0pt}{0pt}}
\def\fourborder#1#2#3#4#5
   {\vbox{\vskip #2 \hbox{\hskip #4 #1 \hskip #5} \vskip #3}}

\def\fourframe#1#2#3#4#5
   {\vbox{\ifdim#2>0pt {\hrule height #2 depth 0pt} \else
                       {\hrule height -#2 depth 0pt \vskip #2} \fi
          \hbox{\ifdim#4>0pt {\vrule width #4} \else
                             {\vrule width -#4 \hskip #4} \fi
                #1
                \ifdim#5>0pt {\vrule width #5} \else
                             {\hskip #5\vrule width -#5} \fi}
          \ifdim#3>0pt {\hrule height 0pt depth #3} \else
                       {\vskip #3 \hrule height -#3 depth 0pt} \fi}}

        %%% Figure Macro

% Insert a  figure, which  may be  merely a  \vskip or  an empty  border
% intended to  contain a  paste  in.  In  report  formats, this  may  be
% replaced by a  more complex figure  macro that makes  an entry in  the
% table of contents.
% Arg 1: the caption
% Arg 2: the proper name
% Arg 3: the figure itself

\def\figure#1#2#3\endfig
   {$$\vbox{\halign{\ctr{##}\cr
              #3\cr
                \cr
              #1\cr}}$$\vmode}

        %%% Citation and Reference

% The citation and reference commands take two arguments: a proper  name
% for the  item  cited and  a  ``title'' to  be  used in  the  reference
% construction.  The  \ref  command  must be  followed  by  the  \endref
% command.  The string  in between  the two  may be  either a  reference
% written out explicitly or  (if reffmt has been  loaded) a sequence  of
% \Author, \Title, etc. declarations.

\def\cref#1#2{#1}       % cross reference to a figure, chapter, section, etc.

\def\cit#1#2{\ifnonnull{#2}{\unskip$^{#2}$}{[#1]}\null}
\def\contref#1\endcontref{      % Continuation of reference
    \noindent\hangindent \refmar\hmode\hbox to \refmar{}#1\par}
\def\endref{\unskip\endrfa\ifnonnull{\vref}{\vref}{}\par}
\def\ref#1#2{\resetref\def\endrfa{}
           \noindent\hangindent \refmar
           \hmode\hbox to \refmar
                      {\ifnonnull{#2}{[#2]}
                                 {\ifnonnull{#1}{[#1]}{}}
                       \hfill}}
\def\refmar{1\VU}
\def\resetref{\gdef\vref{}} % This function is redefined by reffmt to one
                            % that resets all the fields of a reference to nil.

        %%% Various Items Usually Expressed by Italic

\def\bk#1{{\it #1}}             % book title
\def\jt#1{{\it #1}}             % journal title
\def\df#1{{\it #1}}             % defined term
\def\em#1{{\it #1}}             % emphasis
\def\fw#1{{\it #1}}             % foreign word

        % The macros for first, second, third, and fourth generate
        % superscripted (mathmode exponents) Roman abbreviations.

\def\st{$^{\rm st}$}    % elvated `st' for writing first as 1st
\def\nd{$^{\rm nd}$}    % elvated `nd' for writing second as 2nd
\def\rd{$^{\rm rd}$}    % elvated `rd' for writing third as 3rd
\def\th{$^{\rm th}$}    % elvated `th' for writing fourth as 4th, etc.


        % \Alph and \alph generate lower and uppercase letters
        % from numbers between 1 and 26.

        % Note that the hyphen in the zero position cannot
        % be a digit if the definition is made in macro mode,
        % since this digit will combine with the argument to
        % form a larger number.

\def\alph
#1% number between 1 and 26
{\ifcase #1 -\or a\or b\or c\or d\or e\or f\or g\or h\or
            i\or j\or k\or l\or m\or n\or o\or p\or q\or
            r\or s\or t\or u\or v\or w\or x\or y\or z\fi}

\def\Alph
#1% number between 1 and 26
{\ifcase #1 -\or A\or B\or C\or D\or E\or F\or G\or H\or
            I\or J\or K\or L\or M\or N\or O\or P\or Q\or
            R\or S\or T\or U\or V\or W\or X\or Y\or Z\fi}


\def\note#1{}           % a manuscript note that does not appear
                        % in the document

        % Used in some memos to head informal sections
\def\heading#1{\par\vskip 4ex\centerline{#1}\vskip 1ex\penalty 1000}

                % date and time macro

\def\ifundefined#1{\expandafter\ifx\csname#1\endcsname\relax}
\ifundefined{monthnum}{\global\let\monthnum=\month} \else
                        {\sendtty{monthnum_redefined}} \fi
\def\month
   {\ifcase \monthnum \or {January}\or {February}\or {March}\or
         {April}\or {May}\or {June}\or {July}\or {August}\or
         {September}\or {October}\or {November}\or {December} \fi}
\newcount\scratch \newcount\minute \newcount\hour
\def\setdatetime
   {\scratch=\time \minute=\the\scratch \hour=\the\scratch
    \divide\hour by 60 \multiply\hour by 60
    \global\advance\minute by-\hour \global\divide\hour by 60}
\setdatetime
\xdef\datetime{\the\day\month\the\year-\the\hour:\ifnum\the\minute<10{0}\fi\the\minute}

        %%% Some Important \TEX parameters

\endmacromode


%\input mathinit

\def\form#1#2{}         % potentially generate a formula number
                        % first argument is the formula's name for
                        % cross reference in the manuscript.
                        % second argument is an overriding formula number
                        % that will replace any systematically generated
                        % number.

%\input fonts.tex

        % Page Layout

%\input pageout % for the moment use plain tex output routine

\headline={\rlap{\lefthead}\hss\clap{\centerhead}\hss\llap{\righthead}}
\footline={\rlap{\leftfoot}\hss\clap{\centerfoot}\hss\llap{\rightfoot}}

\def\lefthead{} \def\centerhead{} \def\righthead{}
\def\leftfoot{} \def\centerfoot{\tenrm\folio} \def\rightfoot{\datetime}

\overfullrule=0pt       % don't mark overfull boxes with black slugs

% \spc and \tab expand to one space and eight spaces respectively
% for use in \sendtty messages to the console and log file

        \xdef\spc{ } \xdef\tab{\spc\spc}
        \xdef\tab{\tab\tab} \xdef\tab{\tab\tab}

        % Fixes for SUN installation - 9 Feb 89

\def\small#1{#1}        \def\medium#1{#1}       \def\large#1{#1}
\let\ty=\tt     % Used in lots of places
                % Why did he change the name of the typewriter font?

\let\implies=\Rightarrow
\def\goesto{\rightarrow}
\let\xor=\oplus

        % Miscellaneous Stylistic Parameters

\parskip=1ex plus 1ex


\sendtty{loaded: /files/home/diffie/tex/macros/texinit.tex}

% 27Aug89 - Defs of \st, \nd, \rd, and \th chaned to use rm
% 20Aug90 - \datetime changed to print minutes correctly
%  6Dec90 - Set \overfullrule to 0pt
%         - Put the tabs for \sendtty in texinit
% 26Dec90 - added right brace after \sendtty in \ifundefined{monthnum}
%           construction.  This fixes the (\end occurred inside a group...)
%           message when texinit is loaded twice.
% 27Dec90 - Redefined \lft, \rt, and \ctr to use \hss.
%           Redefined \llap and \rlap in terms of \pinch, \lft, etc.
%           Added \clap, and changed \headline and \footline to use
%           \llap, \clap, and \rlap.
%           Changed \heading based on the one in impact reports.
%           Removed definitions of TeX80 parameters \trace, \jpar,
%           and \ragged.

% What do I do about \VU? - WD 27Dec90
% What about: \chpar16=1 % default allows uppercase words to be hyphenated

% 11Feb92 - defined \@ by technique in plain.tex

% 27Feb92 - added \spc in with \tab
% 28Apr92 - Added \alph and \Alph
% 11Dec92 - Added \parskip=1ex plus 1ex
% 11Dec92 - Changed heading to skip only 1ex to accomodate parskip
%  8Feb93 - added \jt: journal title.

