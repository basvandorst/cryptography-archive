\parskip 10pt plus 1pt
\def\_{\char'137} % underline char in typewriter text
\def\^{\char'136} %  ^ char in typewriter text
\def\~{\char'176} %  ~ char in typewriter text
\def\pms{\raise.8ex\hbox{$\scriptscriptstyle+$}\kern-1.2ex
\raise.1ex\hbox{$\scriptscriptstyle-$}}
\def\t{\textstyle}
\def\s{\scriptstyle}
\def\oo#1#2{\t{\s#1\over\s#2}}
\def\section#1{\bigskip\bigskip\medskip\line{\bf #1 \hfil}\par}
\centerline{\bf Aberth--Schaefer C++ Module for Range Arithmetic,
version September 1993}
\bigskip
The module has files for generating precise computation demonstration 
programs, and for writing your own precise computation programs using
range arithmetic.

\section{Demonstration programs}
Below are listed all the demo programs with a brief description of what
they do. Most programs compute answers correct to the number of decimal
places that is requested.
A program's upper bound on
the number of correct decimal places is in force either to keep
the display of answers intelligible or to
keep the execution time of the program from becoming too long.
The two programs in the middle group compute rational answers as pairs
of integers, and require no decimal place setting.
$$\vbox{\halign{\tt#\hfil&\quad#\hfil\cr
calc&computes quantities like a simple calculator\cr
func&evaluates a function $f(x)$ at evenly spaced $x$ arguments\cr
triangle&solves triangles for unknown sides and angles\cr
findzero&finds where $n$ functions of $n$ variables are simultaneously zero\cr
equat&solves a system of linear equations\cr
deriv&finds derivatives or partial derivatives of a function\cr
integ&finds the definite integral of a function $f(x)$\cr
dif&solves the initial value problem for a single ODE of order $\leq 9$\cr
difsys&solves the initial value problem for a system of first order ODEs\cr
difbnd&solves the two-point boundary value problem for a system of
linear ODEs\cr
p\_ roots& finds the roots of a polynomial with real coefficients\cr
eigen&finds the eigenvalues and eigenvectors of a real square matrix\cr
\cr
r\_calc& computes rational quantities like a simple calculator\cr
r\_equat&solves a system of rational linear equations for rational answers\cr
\cr
c\_calc& computes complex quantities like a simple calculator\cr
c\_equat&solves a system of complex linear equations for complex answers\cr
cp\_ roots&finds the roots of a polynomial with complex coefficients\cr
c\_eigen&finds the eigenvalues and eigenvectors of a complex square matrix\cr
}}$$

You will need an up-to-date C++ compiler to convert the text files into
executable files that can be run on your computer.  The text files have been
checked with the following compilers: Semantec's ZORTECH C++ 3.1 compiler,
Borland's TURBO C++ 3.0 DOS compiler, and the Free Software Foundation's
gcc 2.2.2 C++ compiler.  (If you use the gcc compiler, all the code files
must have their filetype changed from {\tt cpp} to {\tt cc}.)

The text files use the standard C include files {\tt ctype.h},
{\tt stdlib.h}, and {\tt string.h},
and also the standard C++ include files {\tt iostream.h} and {\tt fstream.h}.

The construction of an executable file for a demo program
can be done with the help of the module file {\tt gen\_ make} which
lists the dependencies. For instance, this file has for
the demo program {\tt calc} the entry
$$\hbox{\tt calc:\ \ range.o calc.o}$$
This indicates that two object files {\tt range} and {\tt calc} are
needed to construct the {\tt calc} executable file.  An object file is
formed by  compiling the corresponding textfile of filetype {\tt cpp}. 
Thus the files {\tt range.cpp}
and {\tt calc.cpp} must be compiled into object files before the {\tt calc}
executable file can be formed.

The demo programs that work problems which may take
many keyboard entries
to define, keep a record of the entries in a like-named file of
filetype {\tt log}. Thus after completing a run with {\tt difsys},
you will find an entry record in {\tt difsys.log}.
(If for some reason, this recording process is inconvenient, it can be
turned off by changing a single {\tt TRUE} value to {\tt FALSE}
in a clearly marked line near the end of the {\tt range.h} header file.)

A record of
keyboard strokes is useful when one wishes to view at leisure
the complete record of a problem's solution, after first seeing
the console display of the solution, which may have flashed by too quickly.
Continuing the {\tt difsys} example, this can be done in MS-DOS or UNIX
by the command 
$$\hbox{\tt difsys < difsys.log > output\_file}$$
using the {\tt log}
file for standard input, and using the {\tt output\_file} for
standard output. The {\tt output\_file} (perhaps named {\tt difsys.out})
then can be examined with any convenient editor program.

The keyboard record is also convenient when 
working a set of similar problems.
After the first problem is solved, the {\tt log} file can be
edited to specify the next problem, and then made standard input.
Using a file as standard input is also a simple way of getting around
an operating system's upper bound on the number of keystrokes in
a keyboard entry.

\section{Range arithmetic}
The accuracy of the demo programs is accomplished by using {\sl range}
arithmetic in place of ordinary
floating-point arithmetic. Range arithmetic
has two advantages over floating-point: (1) the precision
of computation is under the control
of an executing program, and can be changed at any time;
(2) a bound on a number's computational error, the {\sl range}, 
is part of the number's internal representation. The range can
be examined before results are displayed, and when the range shows
that there are fewer correct decimal places than are wanted,
the computation can be repeated at a higher precision.

Range arithmetic is essentially a modification of floating-point
arithmetic, and like floating-point, can take either
a binary or a decimal form. Our representation is decimal, so
consider first the usual form of a number in decimal floating-point:
$$ \hbox{(sign)} .d_1d_2\dots d_n \cdot 10^e$$
The decimal digits $d_1$, $d_2$, \dots, $d_n$ form the number's
{\sl mantissa}, with $d_1$ required to be nonzero (unless the
number itself is zero), and $e$ is the
number's {\sl exponent}. The number $n$ of digits in the mantissa
varies with the floating-point size, for which there are usually
several choices.
Normally $n$ can be changed only by revising a program to change 
the floating-point sizes.

A simple range arithmetic system uses the representation
$$ \hbox{(sign)} .d_1d_2\dots d_n \pms r\cdot 10^e$$
Here the number $n$ of mantissa digits is variable, being allowed to
be as small as 1 and as large as the implementation upper bound, which can
be huge.
The decimal digit $r$ is the range, giving
the maximum error of the number, with the decimal significance of $r$ 
matching the least significant mantissa digit $d_n$. By
adding $r$ to $d_n$, or subtracting $r$ from $d_n$, we form the endpoints
of the {\sl interval} in which the number must lie.
The range gives the mantissa's error bound, which may be found
by taking account of the range's mantissa position and ignoring the exponent.
When this value is multiplied by $10^e$, we obtain the number's error bound.
Below these two values are given for some typical numbers.
$$\vbox{\halign{$#$\hfil&\qquad$#$\hfil&\qquad\hfil$#$\hfil\cr
\hfill\hbox{Ranged number}\hfill&\hfill\hbox{Mantissa error bound}\hfill
&\hbox{\ Number error bound}\cr
\noalign{\smallskip}
+.2355555555\pms3\cdot10^4& .0000000003=3\cdot10^{-10}&3\cdot10^{-6}\cr
-.7244666\pms2\cdot10^{-2}&.0000002=2\cdot10^{-7}&2\cdot10^{-9}\cr
+.8888\pms9\cdot10^1&.0009=9\cdot10^{-4}&9\cdot10^{-3}\cr}}$$
When constants are converted to this range arithmetic
form, only as many mantissa digits are used as are needed to fully
represent the constant, and the range digit is set to zero since
the constant is without error. Thus the constant $2.5$ would obtain the
ranged form
$+.25\pms0\cdot 10^1$. Ranged numbers like this one, with a
range of zero, are called {\sl exact}.

If two exact numbers are added, subtracted, or multiplied,
the result can also be represented as an exact number, but this
often requires more mantissa digits than either operand has.
And a division of one exact number by another often yields
a non-terminating quotient mantissa. So an internal constant,
{\tt PRECISION}, is used to set the maximum mantissa length allowed
for results of the four rational operations \hbox{$+$, $-$, $\times$, $\div$}.
On a typical computation involving just these four operations, starting
with exact input constants of short
mantissa length, the mantissas of intermediate results usually
quickly attain the {\tt PRECISION} length. If a result mantissa could
have been exact, but being longer in length than {\tt PRECISION}, gets
truncated to this length, then it acquires a nonzero range digit of 1,
indicating a maximum error of one unit in the last mantissa place.
The accuracy of the rational operations gets changed simply by
changing the {\tt PRECISION} value.

When a rational operation is performed with one or both operands having
a nonzero range, the result also has a nonzero range,
which is computed taking into account the ranges of both operands.
As a computation proceeds, using numbers with nonzero ranges, the
error bound of successive mantissas generally increases, and, accordingly,
the length of successive mantissas tends to decrease. In one respect
this is helpful, in
that suspect mantissa digits get discarded automatically.
Shorter mantissas mean that rational operations execute faster,
so usually the later part of a computation runs quicker than the earlier
part. If the final results have too few mantissa digits to yield the
required number of correct decimal places, then the whole computation
can be repeated at a higher {\tt PRECISION}.  

\section{The C++ range arithmetic system}
In our description of range arithmetic, we allowed the mantissa of a ranged
number to change in length by a single decimal digit, and used a single
decimal digit for the range.
We presented this type of range arithmetic because it is easy
to describe and easy to understand. For a practical C++ system,
it is necessary to take into account the number of 
decimal digits that will
fit into a C type {\tt integer}, or the {\sl decimal width}
of an {\tt integer}. 
The decimal width is commonly four, but occasionally is larger,
depending on the particular computer and particular C++ compiler used.
Our C++ system assumes a width of four, but this setting can be increased
beyond four (see the last section).
A mantissa is allowed to change in length only
by as many decimal digits as equal the width, and the number of digits
in the range equals the width.

To revise the description of range arithmetic given earlier to
apply to a system with a decimal width of $m$ digits,
just change the interpretation of a mantissa digit $d_i$ and a
range digit $r$, from a single digit to a block of $m$ decimal digits.
The exponent base, which was 10, must be changed to $10^m$, that
is, 1 followed by $m$ zeros. The {\sl References} article [1]
has more technical details. In our examples we continue
using the simple system.

Our C++ system has two basic classes for convenient range arithmetic: 
{\tt rvar} (ranged variable) and {\tt svar} (string variable). 
The header file {\tt range.h} defines both classes,
listing all operations and supplied functions
in the `public' sections. Generally,
each ranged number variable is an {\tt rvar}, and is converted
into {\tt svar} form for console display or printing. A keyboard
entry is converted to an {\tt svar}, which then may be read to
change an {\tt rvar}'s value.

\section{The rvar constructors}
The examples below show the ways of initializing rvars.
$$\hbox{{\tt rvar X = 22, Y = "2.4 e2", Z = X;} \quad or
\quad {\tt rvar X(22), Y("2.4 e2"), Z(X);}}$$
A decimal constant of any length may be assigned to an rvar as long 
as that constant is enclosed by quotes. Here the usual notation
`{\tt e}' or `{\tt E}' to denote a power of 10 may be used. A constant
that gets converted by the compiler to C type {\tt int} or {\tt long}
does not require quotes.

\section{Some precalculated rvar constants}
Certain constants are provided as
rvars, and these can be used to save constructor time. These constant rvars
are {\tt zero}, {\tt one}, {\tt two}, {\tt ten}, {\tt tenth}, {\tt half},
{\tt pi}, and {\tt logten}. Thus \hbox{\tt rvar X = ten}  is equivalent
to \hbox{\tt rvar X = 10}, or to \hbox{\tt rvar X = "10"}.
Note that if an rvar is not initialized, it is set equal
to {\tt zero}, so that {\tt rvar X} is equivalent to \hbox{\tt rvar X = zero}.
\section{The rvar operators}
There are the four rational operators {\tt +},  {\tt -}, {\tt *}, and {\tt /},
a prefix minus, and the exponentiation operator~{\tt \^}.
These are used in the same way as floating-point operators would be
used. 

The range arithmetic comparison operators  
{\tt ==}, {\tt !=}, {\tt >}, {\tt >=}, {\tt <}, and {\tt <=}
compare intervals, and so they differ from similar floating-point
operators which compare numbers without ranges, that is, points.
Here \hbox{\tt X == Y} is counted
true when the {\tt X} interval overlaps or just touchs the {\tt Y} interval,
and so from {\tt X == Y} one can not conclude that {\tt X} is identical
to {\tt Y}. On the other hand, {\tt X > Y} is counted true when
the {\tt X} interval is to the right of the {\tt Y} interval, and
then it is certain that the variable {\tt X} is larger than
the variable {\tt Y}. Similarly for the case {\tt X < Y}.

There is also a range modification operator {\tt \%},
where {\tt X \% E} returns {\tt X} with its range increased
by the maximum absolute value of {\tt E}. 

\section{The rvar functions}
There are the usual arithmetic functions
{\tt abs}, {\tt acos}, {\tt asin}, {\tt atan}, {\tt cos}, {\tt cosh},
{\tt exp}, {\tt log} (equivalent to {\tt ln}), {\tt sin}, {\tt sinh},
{\tt sqrt}, {\tt tan}, and {\tt tanh}. There are an additional three
functions which return exact values (zero ranges):\smallskip
\halign{\indent#\qquad&#\hfil\cr
{\tt mid(X)}&midpoint of the {\tt X} interval, that is, {\tt X} with its range
set to zero\cr
{\tt range(X)}&the range of {\tt X} converted to an exact value\cr
{\tt trunc(X)}& the largest integer less than or equal to {\tt abs(mid(X))} \cr
& \qquad with the same sign as {\tt mid(X)}\cr}

Three functions required internally are made public because
they may be useful in a program:  the function {\tt is\_int(X)}
which returns 1 or 0 ({\tt TRUE} or {\tt FALSE}) according as
{\tt X} is or is not
an exact integer, the function {\tt is\_zero(X)}
which returns 1 or 0 according as {\tt X} is or is not
an exact zero, and the function {\tt div2(X)} which returns {\tt X/2}
by means of a fast division algorithm.

To obtain the larger of two rvar quantities, the right interval endpoints
of the two quantities should be examined. The rvar chosen as larger is
the one with the greater right endpoint.
To make this decision easy, the function {\tt max}, taking two rvar
arguments, is provided.
For instance, \hbox{\tt Z = max(X, Y)} sets rvar {\tt Z} equal to the larger 
of the two rvars {\tt X} or {\tt Y}.
Similarly, the function {\tt min}, also taking two {\tt rvar} arguments,
is provided for obtaining the smaller of two rvar quantities.

The {\tt swap} function, taking two rvar arguments, provides a convenient
way of exchanging values between two rvars. For instance, {\tt swap(X, Y)}
switches the values of rvars {\tt X} and {\tt Y}.

\section{Control of the precision of range arithmetic}
The precision of range arithmetic can be varied from a low
of 12 decimal digits, the default value, up to many thousands of
decimal digits. The higher precision is set, the slower range arithmetic
performs, and the more memory space is needed, so
it is important not to set precision excessively high.

There are several functions which access and alter the precision of
range arithmetic. One of these is \hbox{\tt set\_ precision(n)} which sets
the precision to {\tt n} decimal digits. Actually, precision changes
by multiples of four decimal digits, so the setting will be at
least {\tt n} decimal digits, but possibly a few digits higher.
The function {\tt current\_ precision()} returns the true value,
in decimal digits, of the current precision setting.

A function which indirectly affects precision is
\hbox{\tt test(X, t, n)} which examines rvar {\tt X} to determine if
{\tt X} could be printed to {\tt n} correct decimals in a format
determined by the integer {\tt t}. If {\tt t} is 1, the format is
fixed decimal, e.g., \hbox{\tt 325.5573}; if {\tt t} is 2,
the format is scientific floating-point, e.g., \hbox{\tt 5.3456 E3}.
The results of the test affect the global variable \hbox{\tt test\_ failure}
which always equals 0 as a program begins execution. If {\tt X} meets
the printing specification, \hbox{\tt test\_ failure} is unchanged. 
But if {\tt X} fails by $k$ digits from it, then \hbox{\tt test\_ failure}
is changed to $k$ but only if this represents an increase for 
{\tt test\_ failure}. That is, the {\tt test} function never
decreases {\tt test\_ failure} from an earlier setting.
After a series of {\tt test} calls have been made, {\tt test\_ failure}
is at the maximum encountered deficit.

The function {\tt add\_ precision(q)} increases the number 
of decimal digits of range arithmetic precision by the value
of {\tt test\_ failure} plus the integer argument {\tt q}. The
parameter {\tt test\_ failure} is then reset to zero, so that another series
of {\tt test} calls can be made afterwards. If the argument {\tt q}  is omitted,
it is presumed zero.

\section{The svar constructors}
The examples below illustrate the ways an svar can be initialized.
$$\hbox{{\tt svar A = "abc", B = A;} \quad or \quad
{\tt svar A("abc"), B(A);}}$$
If an svar is not initialized, it is set equal to the svar constant 
{\tt empty} which is the string {\tt ""}.

\section{The svar operators}
The operator {\tt +} is provided for concatenation, along with 
the related operator {\tt +=}. Thus if {\tt S} is an svar, then one can write
$$\hbox{{\tt S = S + "defgh";} \quad or \quad {\tt S += "defgh";}}$$
The two comparison operators {\tt ==} and {\tt !=} are provided, along with
the prefix operator {\tt !} and the member {\tt len()}. Here {\tt !S} returns a 
pointer to the {\tt S} string, and {\tt S.len()} returns the integer length
of the string. The operator {\tt <<} is provided for console display of
an svar.

\section{Program input}
A variety of functions are provided to make it easy to obtain data via
the console keyboard. These functions are all to be used as
complete C++ statements. The function
$$\hbox{\tt svar\_ entry(message\_ init, S)}$$
displays {\tt message\_ init} at the console, and then sets svar {\tt S} equal
to the string of symbols entered at the keyboard, up to, but not including, the
carriage return. We will call such a string a {\sl keyboard line}. For this
function, and for all other input functions described in this section,
if the keyboard line consists of just the single letter `{\tt q}', this
causes an exit from your program.

It is frequently necessary to obtain an integer via the console keyboard, and
to check that the integer received is within prescribed bounds. For this
purpose, the function below is provided:
$$\hbox{\tt int\_ entry(message\_ init, I,
lower\_ bound, upper\_ bound, message\_ lo, message\_ hi)}$$
This function displays {\tt message\_ init} at the console, converts the
keyboard line to an integer, and sets the argument {\tt I} equal to this
integer if the bounds are satisfied. If they are not satisfied, the
appropriate message selected from {\tt message\_ lo} or {\tt message\_ hi}
is displayed, and the entire process is restarted. The function
arguments {\tt message\_ lo} and {\tt message\_ hi} are not required
and may be omitted.

Two functions are provided for general rvar input. They are {\tt entry}
and {\tt evaluate}, and they can be used to change any rvar variable
according to the instructions of a keyboard line, which is allowed
to have the usual arithmetic functions and operators.
The {\tt entry} function
checks the syntax of the entry line, and displays for correction any errors
found.
Suppose that rvar {\tt X} is to be changed via the keyboard. Then 
one may use the line
$$\hbox{\tt entry(message\_ init, X, S)}$$
Here {\tt message\_ init} is displayed at the console, and the
keyboard line is evaluated to 
give {\tt X} its new value.
After {\tt entry} returns, svar {\tt S}
holds the keyboard line string, along with a code
representation of the string which is used in case a recalculation
is necessary later. The recalculation is called by the line
$$\hbox{\tt evaluate(X, S)}$$

It is also possible to obtain a function from the keyboard, a feature used by
several demo programs. It is usually desirable to first display
to the keyboard operator all the allowed function symbols, and this
is done via the statement
$$\hbox{\tt list\_ tokens(Q)}$$
Here any convenient rvar can serve as the argument {\tt Q}. This argument is
not used, and is present only to let the compiler distinquish among
several alternate \hbox{\tt list\_ token} functions.
The conversion of a keyboard line to a function may be done by the line
$$\hbox{\tt entry(message\_ init, X, S, SS)}$$
Here svar {\tt S} records the keyboard line as described
previously. The second
svar {\tt SS} supplies a list of allowed function arguments,
with commas separating arguments. Thus if we intend to obtain a
function $f(x,y)$ by keyboard entry, we need a statement
of the form \hbox{\tt SS = "x,y"} before the {\tt entry} statement.
When the second svar is supplied as an argument to {\tt entry},
no evaluation of the keyboard line takes place. An rvar argument,
such as {\tt X} in our example, 
must be designated, but it is ignored.

To evaluate a function whose representation is held in
svar {\tt S}, with rvar {\tt F} to hold the
function's value, use the statement
$$\hbox{\tt evaluate(F, S, P)}$$
Here the extra argument {\tt P} is an rvar pointer, pointing to an array
of the function arguments, in the order designated by {\tt SS}.
If our function has only a single argument held in rvar {\tt X},
then {\tt P} can equal {\tt \&X}. Otherwise, if our function
has {\tt n} arguments, then statements like the following are needed:
$$\hbox{{\tt rvar P[n]; P[0] = {\sl first argument};}
\dots {\tt \  ;  P[n-1] = {\sl last argument};}}$$

The {\tt entry} method of reading in a function can be used
without the necessity of actually typing in the function at
the console. The function can be designated within a program by
initializing some svar to the function, another svar
to the function's variables, and then {\tt entry}
is called with an empty {\tt message\_init}. Thus if we want to use
the function $e^x\sin yz$ in our program, we can first set 
{\tt S = "exp(x)*sin(y*z)"} and {\tt SS = "x,y,z"}, and then call
$$\hbox{\tt entry("", X, S, SS)}$$
This function would be evaluated in the way described
in the preceding paragraph. Constants can be dealt with in a similar fashion.
Whenever the {\tt entry} function receives an empty {\tt message\_init} string,
it assumes that the {\tt S} argument has been set by the program, and it
does not try to load {\tt S} by console input. 

The {\tt entry} function uses three different interpretations
for the exponentiation operator {\tt \^}, corresponding to different
mathematical usages. The function $x^n$ is defined for all $x$ if
$n$ is a positive integer, and defined for nonzero $x$ if $n$ is a negative
integer or zero, while a function such as $x^{0.4}$ is
undefined when $x$ is negative. This is the usual mathematical usage.
However, the function $x^{\oo13}$
can be defined for negative $x$ by using the relation
$(-x)^{\oo13} = -x^{\oo13}$.
Similarly, the function $x^{\oo25}$ can be defined for negative $x$ by
using the relation $(-x)^{\oo25} = x^{\oo25}$. To accomodate this second
and third usage, whenever the keystroke sequence {\tt ...\^\ ( p / q)}
is encountered, {\tt p} and {\tt q} being integers, if {\tt p} and {\tt q}
are both odd, the second usage is assumed, and if {\tt p} is even and
{\tt q} is odd, the third usage is assumed. Thus the function $x^{\oo13}$
can be specified as {\tt x\^(1/3)}, and the function $x^{\oo25}$ can
be specified as {\tt x\^(2/5)}.

Two parmeters allow program control of the behavior of {\tt evaluate},
both parameters
being set to either {\tt TRUE} or {\tt FALSE}.  One parameter is
$$\hbox{\tt rvar::evaluate\_error\_display}$$
With the default {\tt TRUE} setting, an {\tt evaluate} computation error
leads to an error message, and then a display of 
the point in the controlling svar string where the error occurred.
With the {\tt FALSE} setting, all {\tt evaluate} error messages are blocked
from console display. The second parameter is
$$\hbox{\tt rvar::evaluate\_exit\_on\_error}$$
With the default {\tt TRUE} setting, there is an exit from your program
after an
{\tt evaluate} error. With the {\tt FALSE} setting, an error ends
{\tt evaluate} computation and the function returns,
but there is no exit from the program. 
When the {\tt FALSE} setting is used, the parameter {\tt rvar::evaluate\_error}
should be sampled after
each {\tt evaluate} call, to determine from its {\tt TRUE} or {\tt FALSE}
value whether an error occurred.

\section{Program output}
The string contents of an svar {\tt S} are displayed at the console by
the command {\tt cout << S}. On the other hand, 
an rvar can be displayed only after it is converted to
an svar, and the function {\tt rtos} is provided for making the conversion.
To convert an rvar {\tt X} to an svar {\tt S} one can write
$$\hbox{ \tt S = rtos(X, t, n)}$$
Here if {\tt t} is 1, then {\tt X} is converted to fixed decimal format
with {\tt n} decimal places. If {\tt t} is 2, then {\tt X} is converted
to scientific floating-point format with {\tt n} decimal places.
Note that the range of {\tt X} is always consulted by the {\tt rtos} function,
and if a decimal place cannot be filled accurately as requested, it is
filled with an asterisk instead. Occasionally an extra correct
decimal place is supplied, when the original request
for {\tt n} correct decimal places cannot be fulfilled, but
$\hbox{\tt n}+1$ correct places
are obtainable. (This happens sometimes when {\tt X} is near
the halfway point separating two adjacent $n$ decimal output numbers;
the extra digit is always a~`5'.) If {\tt t} is 3, then {\tt X} is
converted to an integer format; in this case the {\tt n} argument is
unnecessary and may be omitted. If both {\tt t} and {\tt n} are omitted,
the {\tt rtos} function makes the decision as to the best form of output
and number of decimals displayed; the maximum number of decimals
consistent with the range of {\tt X} will be chosen.

\section{Rvar error detection}
Because an rvar defines an interval, this complicates the reporting and
detection of
errors. For instance, suppose
after some computation, the variable {\tt X} gets the value
$+.2\pms3\cdot 10^{-2}$,
defining an interval stretching from $-.001$ to $+.005$. The true value
of {\tt X} lies somewhere within this
interval, and could be positive, negative,
or zero. The division operation of {\tt Y = two / X} cannot be successful
because the divisor interval contains the zero point,
so an error must be reported, yet the trouble may disappear
if higher precision were used. An error message for a possible error is
made distinct from an error message for a definite error by containing
the word `approximation'.
Thus {\tt two / zero} yields the error message ``{\tt division by zero}",
while the operation above, {\tt two / X},  yields
``{\tt division by zero approximation}".
Similarly, {\tt log(zero)} yields the error message
``{\tt log(a) with zero a}",
while {\tt log(X)} yields ``{\tt log(a) with zero approximation a}".

A related difficulty occurs with function evaluation. Continuing with
our example where {\tt X} has become $+.2\pms3\cdot 10^{-2}$,
what should be the outcome when we compute {\tt sqrt(X)}? This is
an ambiguous situation, since we have an error if {\tt X} is actually
negative, but not if {\tt X} is actually positive or zero. We could
halt computation and report the possible error, but this would prevent
treating the more likely case where {\tt X} is actually positive or zero,
but through computation has had its interval widened to contain
negative numbers. In this latter case, it is appropriate for the {\tt sqrt}
function to ignore the negative part of {\tt X}'s interval,
and return a value which bounds {\tt sqrt(X)} when {\tt X}
extends from $0$ to $+.005$. A similar ambiguous situation would be
the evaluation of {\tt asin(Y)} or {\tt acos(Y)}, when the {\tt Y}
interval straddles $+1$ or $-1$, for instance, when {\tt Y} is
$+.10000\pms2\cdot 10^{1}$. 

The rvar functions which may encounter ambiguous situations are controlled
by the parameter
$$\hbox{\tt rvar::strict\_ambiguity\_treatment}$$
The parameter's default is {\tt FALSE}, and with that setting the functions
{\tt sqrt}, {\tt asin}, and {\tt acos} return rvar values in the cases
mentioned above.
If the parameter had been set by the program to {\tt TRUE}, then 
these functions would have issued an error message instead.
If the default setting is not used, the {\tt evaluate} function
described earlier gives an error message when called upon to compute 
various common functions at certain arguments. A typical example is
{\tt sqrt(sin(U))} when {\tt U} is zero, since here {\tt sin(U)}
always supplies to {\tt sqrt} an interval straddling the zero point.

\section{Other defined classes}
The header file {\tt rational.h} lists the operators and functions supplied
for doing rational arithmetic, and the file {\tt rational.cpp} supplies
the code. Whenever rational arithmetic is done, the precision of
range arithmetic is automatically set to maximum and needs no control
statements. The textfiles of the rational demo programs
can be consulted to see how to use the rational
functions and operators. The rational input and output functions have been
made identical to the rvar input and output functions, except for
the replacement of any rvar argument by a rational one. This makes it
easier to recall the names and usages of the functions.

Similarly the header file {\tt complex.h} lists the operators and functions
for doing complex range arithmetic, the file {\tt complex.cpp} has the code,
and the textfiles of the complex demo programs can be consulted for help
in understanding how the complex functions are used. The complex input
and output functions also have been made identical to the rvar input
and output functions, except for the replacement of any rvar argument by
a complex one.

The other classes defined are for vector and matrix arithmetic
of various types, for Taylor series evaluation, for formal interval arithmetic
using a pair of rvars,
and for manipulation of lists and stacks of rvars.

\section{Changing the decimal width}
Calculations with an rvar proceed by treating 4 digits at a time.
In certain cases, it is possible to treat 6 or
even 8 digits at a time, with a corresponding major improvement of
computation speed. (It is inconvenient to allow an odd width
because of difficulties this would make for the
{\tt sqrt} function.)
Whether or not a higher width is possible depends on the type of computer
you are using and your C++ compiler.  The program {\tt digit\_ck} is
provided as a way of deciding.  Simply
compile and run the program to obtain a line of output. 
If improvement is possible, {\tt digit\_ck}
will indicate a change to make in a control line in the range.h file.
\bigskip\bigskip
\centerline{\bf References}
\def\paper#1#2#3#4#5#6#7{\par\smallskip\noindent%
[#1]\ #2,\ #3,\ {\sl#4\/}\ {\bf#5}\ (#6),\ #7.}
\def\book#1#2#3#4{\par\smallskip\noindent
[#1]\ #2,\ {\sl#3\/},\ #4.}
\paper{1}{O.~Aberth and M.J.~Schaefer}{Precise computation
using range arithmetic, via C++}{ACM Trans. Math. Software}{18}{1992}{481--491}
\paper{2}{O.~Aberth and M.J.~Schaefer}{Precise matrix eigenvalues using
range arithmetic}{SIAM J. Matrix Analysis and Applications}
{14}{1993}{235--241}
\book{3}{O.~Aberth}{Precise Numerical Analysis}
{Wm. C. Brown Publishers, Dubuque, Iowa, 1988}
\bye
