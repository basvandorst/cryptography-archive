% --------------- COPYRIGHT ------------------------
%| AIT - Computer Science Program - Thailand        |
%|   see COPYRIGHT notice                           |
%--------------------------------------------------

% --------------- PRODUCT IDENTIFICATION -----------
%| Product              : phi-pgp                   |
%| Date                 : November 1993             |
%| Last update          : 22 Nov 93                 |
%| Conceived by         : O. Nicole                 |
%| Purpose              : Documentation of Phi-pgp  |
%--------------------------------------------------

\documentstyle{article}
\begin{document}
\title{Phi-pgp\\Detailled conception file}
\author{Olivier {\sc Nicole}\\AIT - Computer Science Program - Thailand\\\tt <on@cs.ait.ac.th>}
\renewcommand{\today}{November 1993}
\maketitle
The detailled conception file defines the architecture of programming,
it results from the coding.

This   should  be  the  last    of three    files,  both  other files,
specification file and preliminary  conception file are included  into
the document Phi-pgp, a GNU Emacs interface to PGP.

\tableofcontents

\section{Conception}
Phi-pgp provides users of  Emacs  mail facilities with a   convenient,
robust and consistant access to PGP (Ptretty Good Privacy). 

The interface is convenient as it offers all the features one user can
hope for, and  only thoses features. It  is robust to security  and in
hadeling communication between Emacs  and PGP.  Last it is  consistant
with both Emacs and PGP standard use.

\input module

\end{document}