\section{Product identification}
\begin{verbatim}
;;; --------------- PRODUCT IDENTIFICATION -----------
;;;| Product              : phi-pgp                   |
;;;| Date                 : November 1993             |
;;;| Last update          : 22 Nov 93                 |
;;;| Conceived by         : O. Nicole                 |
;;;| Purpose              : GNU Emacs interface for PGP
;;;--------------------------------------------------

;;; --------------- HISTORY ------------------------
;;;|%c%--------------------------------------------------

;;; --------------- CONSTANTS ----------------------

;;; This are strings genearted by output of PGP

(setq pgp-strconst-enter-pass "\nEnter pass phrase: ")
        				; Prompt requesting for the passphrase
(setq pgp-strconst-bad-pass "\nError:  Bad pass phrase\\.\n")
        				; Message for s bad pass phrase.
(setq pgp-strconst-pass-key "^Key for user ID \".*\"\n")
        				; Message specifying the key which
        				; pass phgrase is about to be read
(setq pgp-strconst-key-id "^Key for user ID:")
        				; Message specifying the key ID used 
        				; for secret of public key
(setq pgp-strconst-signed-mess
      "^File has signature.  Public key is required to check signature.")
        				; Message issued when a signed
        				; message is decrypted.
(setq pgp-strconst-good-sign "^Good signature from user \".*\"\\.$")
        				; Message issued when a valid 
        				; signature is encountred.
(setq pgp-strconst-bad-sign "^WARNING: Can't find the right public key-- can't check signature integrity.$")
        				; Message issued when there is no
        				; public key corresponding to the
        				; signature of the message.
(setq pgp-strconst-packet-sep
      "\n?\nLooking for next packet in '.*'\\.\\.\\.$")
        				; Message issued when there are 
        				; several pgpcoded packet in a
        				; same file.
(setq pgp-strconst-good-pass
      "^Pass phrase is good.  Just a moment\\.\\.\\.\\.\\.\\.")
        				; Message issued just before the
        				; message (no \\n) when the pass
        				; is good,
(setq pgp-strconst-recur-sep "Trying a recursive call to pgp...\n")
        				; Message added when a automatic
        				; recursive call to pgp is generated.
(setq pgp-strconst-crypted-mess
      "^File is encrypted.  Secret key is required to read it. $")
        				; Message issued when decrypt a text.
(setq pgp-strconst-double-bad-pass
      "^You do not have the secret key needed to decrypt this file.$")
        				; Message send when the pass phrase
        				; have been entered incorectely
        				; twice while decrypt.
(setq pgp-strconst-bad-ascii
      "^Error: Transport armor stripping failed for file ")
        				; Message issued when the packet has
        				; been altered during the mailransport
(setq pgp-strconst-begin-ro
  "^This message is marked \"For your eyes only\".  Display now (Y/n)\\? y?$")
        				; Promp message displayed when a
        				; view-only text is decrypted.
(setq pgp-strconst-end-ro "^Done\\.\\.\\.hit any key
\\( +
\\)?$")
        				; Prompt message at the end of a
        				; 'view-only' section.
(setq pgp-strconst-more-ro "-- Hit space for next screen, Enter for new line, 'Q' to quit --
\\( +
\\)?")
        				; Prompt message in the middle of a
        				; 'view-only' section
(setq pgp-strconst-add-keyring
      "^Do you want to add this keyfile to keyring '.*' (y/N)\\? $")
        				; Prompt for adding a key to a keyring
(setq pgp-strconst-key-mess
      "^File contains key(s)\\.  Contents follow\\.\\.\\.\nKey ring: '.*'\n")
        				; Message when the decrypted
        				; mail contains key.
(setq pgp-strconst-end-key-mess "^[0-9]+ key(s) examined\\.$")
        				; Message at the end of a
        				; decrypted mail containing keys.
(setq pgp-strconst-no-key-found "^No keys found in '.*'\\.$")
        				; Message when no valid key is
        				; found while adding key.

;;; --------------- MACROS -------------------------

;;; --------------- STATIC VARIABLES ---------------

(defvar pgp-list-keys nil)
        				;Lists of public and secret keys
        				; extracted from pgp-keyrings
;;;(setq pgp-exit-status nil)
        				; Exit status of the latest pgp
        				; asynchromous process
;;;(setq pgp-buffer-search-point nil)
        				; Last character of *PGP process*
        				; buffer already examinated
(setq pgp-list-completion nil)
        				; Define the list of pgp possible
        				; keys used for a key read
(setq pgp-selected-seckey nil)
        				;The secret key currently in use.
        				; It is set to the user define
        				; pgp-default-seckey or to the first
        				; key of secring. It is changed using
        				; thegp-set-seckey function
(defvar pgp-list-secring-set nil
  "Set non-nil means a list of secret keys have been extracted form secring.")
(defvar pgp-list-pubring-set nil
  "Set non-nil means a list of public keys have been extracted form pubring.")
(defvar pgp-program "pgp"
  "Define the name of the local pgp program, if not in the default PATH, the
name should be absolute.")

;;; --------------- USER VARIABLES -----------------

(defvar pgp-default-seckey nil
  "*User define default secret key to use. If not define, will choose the last
added key of secring.")
(defvar pgp-delete-pass nil
  "*Set non-nil means always delete the pass phrase. Keeping a pass phrase 
associated to a secret key during an Emacs session can be a security hole, 
if the pass phrase is in memory of Emacs when it crashes, it will appear 
in the core file.")
(defvar pgp-self-crypt nil
  "*Set to non-nil means encrypt with user's own key in addition of recipients
keys.")
(defvar pgp-auto-add-pubkey nil
  "*Set to non-nil, try to automatically add the user's public key inside
an encrypted message. No meaning for the signed message, the view-only 
messages of the partially encrypted ones. The key is added at the very
end of the message.")

;;; --------------- END PRODUCT IDENTIFICATION -----
\end{verbatim}
\section{Functions identification}
\subsection{Function {\tt pgp-read-passphrase}}
\leavevmode
\begin{verbatim}
;;; --------------- Function identification ----------
;;;| Function name         : pgp-read-passphrase      |
;;;| Purpose               : Read a pass phrase from the minibuffer
;;;| Calling parameters    :                          |
;;;|  prompt: the string that it is prompt at screen  |
;;;| Return value          : the pass phrase or nil   |
;;;--------------------------------------------------

;;; --------------- Cross calls ----------------------
;;;| Called functions     :                           |
;;;| Calling functions    : pgp-get-passphrase        |
;;;-------------------------------------------------- 

;;; --------------- Function definition ------------

;;;Stolen from rat-pgp.el
(defun pgp-read-passphrase (prompt)
  "Read a pass phrase form the minibuffer. 
Echos a '.' for each character typed. 
RET   end the pass phrase
LFD   end the pass phrase
DEL   delete characters backward
C-k   clear the pass phrase
C-g   abort"

;;; --------------- End function identification ----
\end{verbatim}
\subsection{Function {\tt pgp-get-passphrase}}
\leavevmode
\begin{verbatim}
;;; --------------- Function identification ----------
;;;| Function name         : pgp-get-passphrase       |
;;;| Purpose               : Read the pass phrase for a given user.
;;;| Calling parameters    :                          |
;;;| Return value          : t or nil                 |
;;;--------------------------------------------------

;;; --------------- Cross calls ----------------------
;;;| Called functions     : pgp-get-seckey pgp-read-passphrase
;;;| Calling functions    : pgp-output-filter         |
;;;|  pgp-general-sign, pgp-sign-encrypt              |
;;;--------------------------------------------------

;;; --------------- Function definition ------------

(defun pgp-get-passphrase ()
  "Read the pass phrase for a given user."

;;; --------------- End function identification ----
\end{verbatim}
\subsection{Function {\tt pgp-get-seckey}}
\leavevmode
\begin{verbatim}
;;; --------------- Function identification ----------
;;;| Function name         : pgp-get-seckey           |
;;;| Purpose               : return the actual secret key
;;;| Calling parameters    :                          |
;;;| Return value          : the secret key or nil    |
;;;--------------------------------------------------

;;; --------------- Cross calls ----------------------
;;;| Called functions     : pgp-list-secring, pgp-check-seckey
;;;|  pgp-set-seckey                                  |
;;;| Calling functions    : pgp-get-passphrase, pgp-view-seckey,
;;;|  pgp-general-encrypt, pgp-append-pubkey, pgp-sign-encrypt
;;;--------------------------------------------------

;;; --------------- Function definition ------------

(defun pgp-get-seckey ()
  "Return the actual secret key.
The key is a cons of the key and the pass-phrase if it exists."

;;; --------------- End function identification ----
\end{verbatim}
\subsection{Function {\tt pgp-view-seckey}}
\leavevmode
\begin{verbatim}
;;; --------------- Function identification ----------
;;;| Function name         : pgp-view-seckey          |
;;;| Purpose               : Display the selected secret key
;;;| Calling parameters    :                          |
;;;| Return value          : non significant          |
;;;--------------------------------------------------

;;; --------------- Cross calls ----------------------
;;;| Called functions     : pgp-get-seckey            |
;;;| Calling functions    :                           |
;;;--------------------------------------------------

;;; --------------- Function definition ------------

;;;###autoload
(defun pgp-view-seckey ()
  "Display the selected secret key."
  (interactive)

;;; --------------- End function identification ----
\end{verbatim}
\subsection{Function {\tt pgp-check-seckey}}
\leavevmode
\begin{verbatim}
;;; --------------- Function identification ----------
;;;| Function name         : pgp-check-seckey         |
;;;| Purpose               : Return the first matching secret key from secring
;;;| Calling parameters    :                          |
;;;|  key :the key to check                           |
;;;| list: the list to check the key against          |
;;;| Return value          : (key . pass) or nil      |
;;;--------------------------------------------------

;;; --------------- Cross calls ----------------------
;;;| Called functions     : RECURSIVE                 |
;;;| Calling functions    : pgp-get-seckey pgp-set-seckey
;;;|  pgp-set-var-seckey                              |
;;;--------------------------------------------------

;;; --------------- Function definition ------------

(defun pgp-check-seckey (key list)
  "Return the first matching secret key of PGP secring."

;;; --------------- End function identification ----
\end{verbatim}
\subsection{Function {\tt pgp-set-seckey}}
\leavevmode
\begin{verbatim}
;;; --------------- Function identification ----------
;;;| Function name         : pgp-set-seckey           |
;;;| Purpose               : the selected secret key in use
;;;| Calling parameters    :                          |
;;;| Return value          : (key . pass) or nil      |
;;;--------------------------------------------------

;;; --------------- Cross calls ----------------------
;;;| Called functions     : pgp-list-secring, pgp-check-seckey
;;;|  pgp-read-recipient                              |
;;;| Calling functions    : pgp-get-seckey            |
;;;--------------------------------------------------

;;; --------------- Function definition ------------

;;;###autoload
(defun pgp-set-seckey ()
  "Set the selected secret key in use."
  (interactive)

;;; --------------- End function identification ----
\end{verbatim}
\subsection{Function {\tt pgp-set-var-seckey}}
\leavevmode
\begin{verbatim}
;;; --------------- Function identification ----------
;;;| Function name         : pgp-set-var-seckey       |
;;;| Purpose               : Set the selected secret key in use
;;;|   depending on the key send as argument          |
;;;| Calling parameters    :                          |
;;;|  key: the new secret key to select               |
;;;| Return value          : (key . pass)             |
;;;--------------------------------------------------

;;; --------------- Cross calls ----------------------
;;;| Called functions     : pgp-list-secring, pgp-check-seckey
;;;| Calling functions    : pgp-output-filter         |
;;;--------------------------------------------------

;;; --------------- Function definition ------------

(defun pgp-set-var-seckey (key)
  "Set the selected secret key in use depending on the key send as argument.
The key send as argument is a valid secret key."

;;; --------------- End function identification ----
\end{verbatim}
\subsection{Function {\tt pgp-extract-email-recipient}}
\leavevmode
\begin{verbatim}
;;; --------------- Function identification ----------
;;;| Function name         : pgp-extract-email-recipient
;;;| Purpose               : Extract the list of recipient from a
;;;|            *mail* buffer                         |
;;;| Calling parameters    :                          |
;;;| Return value          :list of recipients or nil |
;;;--------------------------------------------------

;;; --------------- Cross calls ----------------------
;;;| Called functions     : pgp-general-encrypt, pgp-sign-encrypt
;;;| Calling functions    : pgp-extract-recipient-feild
;;;--------------------------------------------------

;;; --------------- Function definition ------------

(defun pgp-extract-email-recipient ()
  "Extract the list of recipient from a *mail* buffer.
Extract recipients of every To:, CC: or BCC: mail-header feilds as 
if one request for message encrypting, the message should be made 
readable for any recipient. The *mail* buffer is considered to be
the current buffer. Return a list of strings."

;;; --------------- End function identification ----
\end{verbatim}
\subsection{Function {\tt pgp-extract-recipient-feild}}
\leavevmode
\begin{verbatim}
;;; --------------- Function identification ----------
;;;| Function name         : pgp-extract-recipient-feild
;;;| Purpose               : Extract the list of recipients for a *mail*
;;;|                 buffer for a specified feild     |
;;;| Calling parameters    :                          |
;;;|  feild: the header feild to search for           |
;;;|  end-header-point: a mark to the end of mail header
;;;| Return value          : list of recipients or nil|
;;;--------------------------------------------------

;;; --------------- Cross calls ----------------------
;;;| Called functions     : pgp-string-to-list        |
;;;| Calling functions    : pgp-extract-email-recipient
;;;--------------------------------------------------

;;; --------------- Function definition ------------

(defun pgp-extract-recipient-feild (feild end-header-point)
  "Extract the list of recipients for a *mail* buffer for a specified feild."

;;; --------------- End function identification ----
\end{verbatim}
\subsection{Function {\tt pgp-string-to-list}}
\leavevmode
\begin{verbatim}
;;; --------------- Function identification ----------
;;;| Function name         : pgp-string-to-list       |
;;;| Purpose               : Make a list of single word strings from a
;;;|               string containing several words    |
;;;| Calling parameters    :                          |
;;;|  entry-str: the entry string                     |
;;;| Return value          : list of words or nil     |
;;;--------------------------------------------------

;;; --------------- Cross calls ----------------------
;;;| Called functions     :                           |
;;;| Calling functions    : pgp-extract-recipient-feild
;;;--------------------------------------------------

;;; --------------- Function definition ------------

(defun pgp-string-to-list (entry-str)
  "Make a list of single word strings from a string containing several words.
The entry string contains several words that are SPACE, NEW-LINE, 
TAB and/or comma separated. The resulting list contains single words, 
the separating characters having been removed."

;;; --------------- End function identification ----
\end{verbatim}
\subsection{Function {\tt pgp-create-output-buffer}}
\leavevmode
\begin{verbatim}
;;; --------------- Function identification ----------
;;;| Function name         : pgp-create-output-buffer |
;;;| Purpose               : Create the *PGP process* buffer
;;;| Calling parameters    :                          |
;;;| Return value          : the buffer               |
;;;--------------------------------------------------

;;; --------------- Cross calls ----------------------
;;;| Called functions     :                           |
;;;| Calling functions    : pgp-list-pubring, pgp-list-secring
;;;|  pgp-general-sign, pgp-general-encrypt, pgp-append-pubkey,
;;;|  pgp-sign-encrypt, pgp-decr-mess, pgp-add-key    |
;;;--------------------------------------------------

;;; --------------- Function definition ------------

(defun pgp-create-output-buffer ()
  "Create the *PGP process* buffer.
Create or clean the buffer used for working and output by pgp
asynchronous processes."

;;; --------------- End function identification ----
\end{verbatim}
\subsection{Function {\tt pgp-create-temp-buffer}}
\leavevmode
\begin{verbatim}
;;; --------------- Function identification ----------
;;;| Function name         : pgp-create-temp-buffer   |
;;;| Purpose               : Create the *PGP temp* buffer
;;;| Calling parameters    :                          |
;;;| Return value          : the buffer               |
;;;--------------------------------------------------

;;; --------------- Cross calls ----------------------
;;;| Called functions     :                           |
;;;| Calling functions    : pgp-encrypt, pgp-encrypt-view
;;;|  pgp-encrypt-region, pgp-append-pubkey, pgp-show-status
;;;--------------------------------------------------

;;; --------------- Function definition ------------

(defun pgp-create-temp-buffer ()
  "Create the *PGP temp* buffer.
Create or clean the buffer used for temporary working by pgp
asynchronous processes."

;;; --------------- End function identification ----
\end{verbatim}
\subsection{Function {\tt pgp-wait-process-exit}}
\leavevmode
\begin{verbatim}
;;; --------------- Function identification ----------
;;;| Function name         : pgp-wait-process-exit    |
;;;| Purpose               : Wait for the termination of a pgg
;;;|                         asychronous process      |
;;;| Calling parameters    :                          |
;;;|  proc: the process to wait for                   |
;;;| Return value          : exit code of the process |
;;;--------------------------------------------------

;;; --------------- Cross calls ----------------------
;;;| Called functions     :                           |
;;;| Calling functions    : pgp-list-pubring, pgp-list-secring
;;;|  pgp-general-sign, pgp-general-encrypt, pgp-append-pubkey,
;;;|  pgp-sign-encrypt, pgp-decrypt, pgp-add-key      |
;;;--------------------------------------------------

;;; --------------- Function definition ------------

(defun pgp-wait-process-exit (proc)
  "Wait for the termination of a pgg asychronous process.
This is an active wait with 0.5 seconds pauses. Return an exit status."

;;; --------------- End function identification ----
\end{verbatim}
\subsection{Function {\tt pgp-output-filter}}
\leavevmode
\begin{verbatim}
;;; --------------- Function identification ----------
;;;| Function name         : pgp-output-filter        |
;;;| Purpose               : Filters and print the output of PGP
;;;| Calling parameters    :                          |
;;;|  proc  : the PGP process                         |
;;;|  string: new string to output                    |
;;;| Return value          : non significant          |
;;;--------------------------------------------------

;;; --------------- Cross calls ----------------------
;;;| Called functions     : pgp-ro-protect, pgp-set-var-seckey,
;;;|  pgp-get-passphrase                              |
;;;| Calling functions    : pgp-list-secring, pgp-general-sign
;;;|  pgp-sign-encrypt, pgp-decrypt                   |
;;;--------------------------------------------------

;;; --------------- Function definition ------------

(defun pgp-output-filter (proc string)
  "Filters and print the output of PGP.
Scan the output for various strings corresponding to pgp questions and
messages. Take actions according to thoses messages. Maintains a point
to the last scan section so the same message will not be dected at each
call of the filter."

;;; --------------- End function identification ----
\end{verbatim}
\subsection{Function {\tt pgp-nil}}
\leavevmode
\begin{verbatim}
;;; --------------- Function identification ----------
;;;| Function name         : pgp-nil                  |
;;;| Purpose               : Do nothing, for null key binding only
;;;| Calling parameters    :                          |
;;;| Return value          : no significant           |
;;;--------------------------------------------------

;;; --------------- Cross calls ----------------------
;;;| Called functions     :                           |
;;;| Calling functions    : pgp-ro-protect            |
;;;--------------------------------------------------

;;; --------------- Function definition ------------

(defun pgp-nil ()
  "Do nothing, for null key binding only."
  (interactive)

;;; --------------- End function identification ----
\end{verbatim}
\subsection{Function {\tt pgp-ro-protect}}
\leavevmode
\begin{verbatim}
;;; --------------- Function identification ----------
;;;| Function name         : pgp-ro-protect           |
;;;| Purpose               : et protection on the current buffer when a
;;;|                         message is 'view-only'   |
;;;| Calling parameters    :                          |
;;;| Return value          : no significant           |
;;;--------------------------------------------------

;;; --------------- Cross calls ----------------------
;;;| Called functions     : pgp-nil                   |
;;;| Calling functions    : pgp-output-filter, pgp-show-status
;;;--------------------------------------------------

;;; --------------- Function definition ------------

(defun pgp-ro-protect ()
  "Set protection on the current buffer when a message is 'view-only'.
This is the best level of protection I have been avble to think of till
now."

;;; --------------- End function identification ----
\end{verbatim}
\subsection{Function {\tt pgp-list-pubring}}
\leavevmode
\begin{verbatim}
;;; --------------- Function identification ----------
;;;| Function name         : pgp-list-pubring         |
;;;| Purpose               : Extract a list of public keys from pgp-pubring
;;;| Calling parameters    :                          |
;;;| Return value          : t or nil                 |
;;;--------------------------------------------------

;;; --------------- Cross calls ----------------------
;;;| Called functions     : pgp-create-output-buffer, |
;;;|  pgp-wait-process-exit                           |
;;;| Calling functions    :pgp-general-encrypt, pgp-sign-encrypt
;;;|  pgp-decr-mess, pgp-add-key                      |
;;;--------------------------------------------------

;;; --------------- Function definition ------------

(defun pgp-list-pubring ()
  "Extract a list of public keys from pgp-pubring.
This function is called only once during a Emacs session, the list is
updated automatically if new public keys are added to the pgp-pubring."

;;; --------------- End function identification ----
\end{verbatim}
\subsection{Function {\tt pgp-list-secring}}
\leavevmode
\begin{verbatim}
;;; --------------- Function identification ----------
;;;| Function name         : pgp-list-secring         |
;;;| Purpose               : Extract a list of secret keys from pgp-secring
;;;| Calling parameters    :                          |
;;;| Return value          : t or nil                 |
;;;--------------------------------------------------

;;; --------------- Cross calls ----------------------
;;;| Called functions     : pgp-create-output-buffer, pgp-wait-process-exit
;;;| Calling functions    : pgp-get-seckey, set-set-seckey,
;;;|  pgp-set-var-seckey, pgp-decr-mess               |
;;;--------------------------------------------------

;;; --------------- Function definition ------------

;;;This one is a bit tricky, cf. PGP doc, run a pgp on the secring.pgp file
(defun pgp-list-secring ()
  "Extract a list of secret keys from pgp-secring.
This function is called only once during a Emacs session."

;;; --------------- End function identification ----
\end{verbatim}
\subsection{Function {\tt pgp-try-completion}}
\leavevmode
\begin{verbatim}
;;; --------------- Function identification ----------
;;;| Function name         : pgp-try-completion       |
;;;| Purpose               : Return the first element of the list matching str
;;;| Calling parameters    :                          |
;;;|  str : the element to match                      |
;;;|  list: the list to match str against             |
;;;| Return value          : matching string or nil   |
;;;--------------------------------------------------

;;; --------------- Cross calls ----------------------
;;;| Called functions     : RECURSIVE                 |
;;;| Calling functions    : pgp-find-first-match, pgp-collection
;;;--------------------------------------------------

;;; --------------- Function definition ------------

(defun pgp-try-completion (str list)
  "Return the first element of the list matching str."

;;; --------------- End function identification ----
\end{verbatim}
\subsection{Function {\tt pgp-all-completion}}
\leavevmode
\begin{verbatim}
;;; --------------- Function identification ----------
;;;| Function name         : pgp-all-completion       |
;;;| Purpose               : Return a list of elements matching str
;;;| Calling parameters    :                          |
;;;|  str : the element to match                      |
;;;|  list: the list to match str against             |
;;;| Return value          : the list or nil          |
;;;--------------------------------------------------

;;; --------------- Cross calls ----------------------
;;;| Called functions     : RECURSIVE                 |
;;;| Calling functions    : pgp-collection            |
;;;--------------------------------------------------

;;; --------------- Function definition ------------

(defun pgp-all-completion (str list)
  "Return a list of elements matching str."

;;; --------------- End function identification ----
\end{verbatim}
\subsection{Function {\tt pgp-set-initial}}
\leavevmode
\begin{verbatim}
;;; --------------- Function identification ----------
;;;| Function name         : pgp-set-initial          |
;;;| Purpose               : Try to find a good entry in the list
;;;|                         matching name            |
;;;| Calling parameters    :                          |
;;;|  name: the name to find matching for             |
;;;| Return value          : a string matching        |
;;;--------------------------------------------------

;;; --------------- Cross calls ----------------------
;;;| Called functions     :                           |
;;;| Calling functions    : pgp-read-recipient        |
;;;--------------------------------------------------

;;; --------------- Function definition ------------

;;; This function is not implemented yet.

;;; Given the email address and the list of public keys, it should try
;;; to guess which pgp key correspond to the given email address. This
;;; should be quite a straight process when the pgp key contains the email
;;; address (but require some works anyway when the email is local but the
;;; address given in the key is fully defined) but what to do when the
;;; pgp public key is only the user name?

(defun pgp-set-initial (name)
  "Try to find a good entry in the list matching name."

;;; --------------- End function identification ----
\end{verbatim}
\subsection{Function {\tt pgp-read-recipient}}
\leavevmode
\begin{verbatim}
;;; --------------- Function identification ----------
;;;| Function name         : pgp-read-recipient       |
;;;| Purpose               : Read from minibuffer a PGP key for a given prompt
;;;| Calling parameters    :                          |
;;;|  prompt: a prompt string to display              |
;;;|  list-coll: a list of all possible choices       |
;;;|  name: to be inserted in the prompt              |
;;;| Return value          : a completed key          |
;;;--------------------------------------------------

;;; --------------- Cross calls ----------------------
;;;| Called functions     : pgp-find-first-match, pgp-find-first-exit
;;;|  pgp-restore-map, pgp-mouse-choose, pgp-set-initial,
;;;|  pgp-collection                                  |
;;;| Calling functions    : pgp-set-seckey, pgp-list-recipient-id,
;;;|  pgp-general-encrypt, pgp-sign-encrypt           |
;;;--------------------------------------------------

;;; --------------- Function definition ------------

(defun pgp-read-recipient (prompt list-coll name)
  "Read from minibuffer a PGP key for a given prompt.
This function uses the PGP keys lists and implements completion:
SPC   find the first matching ID
TAB   find the first matching ID and exit
RET   find the first matching ID and exit
LFD   find the first matching ID and exit
C-g   abort
C-n   find next matching
?     give a list of possible matching ID"

;;; --------------- End function identification ----
\end{verbatim}
\subsection{Function {\tt pgp-restore-map}}
\leavevmode
\begin{verbatim}
;;; --------------- Function identification ----------
;;;| Function name         : pgp-restore-map          |
;;;| Purpose               : Restore minibuffer-local-completion-map
;;;|                         before exiting           |
;;;| Calling parameters    :                          |
;;;| Return value          : non significant          |
;;;--------------------------------------------------

;;; --------------- Cross calls ----------------------
;;;| Called functions     :                           |
;;;| Calling functions    : pgp-read-recipient        |
;;;--------------------------------------------------

;;; --------------- Function definition ------------

(defun pgp-restore-map ()
  "Restore minibuffer-local-completion-map before exiting."
  (interactive)

;;; --------------- End function identification ----
\end{verbatim}
\subsection{Function {\tt pgp-find-first-match}}
\leavevmode
\begin{verbatim}
;;; --------------- Function identification ----------
;;;| Function name         : pgp-find-first-match     |
;;;| Purpose               : Find the first PGP recipient ID that match
;;;|                         the string               |
;;;| Calling parameters    :                          |
;;;| Return value          : a completed key or nil   |
;;;--------------------------------------------------

;;; --------------- Cross calls ----------------------
;;;| Called functions     : pgp-try-completion        |
;;;| Calling functions    : pgp-read-recipient, pgp-find-first-exit
;;;--------------------------------------------------

;;; --------------- Function definition ------------

(defun pgp-find-first-match ()
  "Find the first PGP recipient ID that match the string."
  (interactive)

;;; --------------- End function identification ----
\end{verbatim}
\subsection{Function {\tt pgp-find-first-exit}}
\leavevmode
\begin{verbatim}
;;; --------------- Function identification ----------
;;;| Function name         : pgp-find-first-exit      |
;;;| Purpose               : Find the first PGP recipient ID that match
;;;|                         the string and exit      |
;;;| Calling parameters    :                          |
;;;| Return value          : a completed key or nil   |
;;;--------------------------------------------------

;;; --------------- Cross calls ----------------------
;;;| Called functions     : pgp-find-first-match      |
;;;| Calling functions    : pgp-read-recipient        |
;;;--------------------------------------------------

;;; --------------- Function definition ------------

(defun pgp-find-first-exit ()
  "Find the first PGP recipient ID that match the string and exit."
  (interactive)

;;; --------------- End function identification ----
\end{verbatim}
\subsection{Function {\tt pgp-collection}}
\leavevmode
\begin{verbatim}
;;; --------------- Function identification ----------
;;;| Function name         : pgp-collection           |
;;;| Purpose               : Create the alist collection of matching PGP keys
;;;| Calling parameters    :                          |
;;;|  str : the string to complete                    |
;;;|  func: not used but needed                       |
;;;|  flag: set return a list, nil a single element   |
;;;| Return value          : a list or the first      |
;;;--------------------------------------------------

;;; --------------- Cross calls ----------------------
;;;| Called functions     : pgp-try-completion, pgp-all-completion
;;;| Calling functions    : pgp-read-recipient        |
;;;--------------------------------------------------

;;; --------------- Function definition ------------

;;; It is a programmed completion function as decribed in GNE Emacs Lisp
;;; Reference Manual (17.5.2)
(defun pgp-collection (str func flag)
  "Create the alist collection of matching PGP keys."

;;; --------------- End function identification ----
\end{verbatim}
\subsection{Function {\tt pgp-list-recipient-id}}
\leavevmode
\begin{verbatim}
;;; --------------- Function identification ----------
;;;| Function name         : pgp-list-recipient-id    |
;;;| Purpose               : Make a string of quoted PGP recipient ID
;;;| Calling parameters    :                          |
;;;|  mail-list: list of name to find pgp-key for     |
;;;| Return value          : list of key or nil       |
;;;--------------------------------------------------

;;; --------------- Cross calls ----------------------
;;;| Called functions     : RECURSIVE, pgp-read-recipient
;;;| Calling functions    : pgp-general-encrypt, pgp-sign-encrypt
;;;--------------------------------------------------

;;; --------------- Function definition ------------

(defun pgp-list-recipient-id (mail-list)
  "Make a string of quoted PGP recipient ID."

;;; --------------- End function identification ----
\end{verbatim}
\subsection{Function {\tt pgp-mouse-choose}}
\leavevmode
\begin{verbatim}
;;; --------------- Function identification ----------
;;;| Function name         : pgp-mouse-choose         |
;;;| Purpose               : Click on an alternative in the `*Completions*'
;;;|                         buffer to choose it      |
;;;| Calling parameters    :                          |
;;;|  event: a mouse event                            |
;;;| Return value          : non significant          |
;;;--------------------------------------------------

;;; --------------- Cross calls ----------------------
;;;| Called functions     :                           |
;;;| Calling functions    : pgp-read-recipient        |
;;;--------------------------------------------------

;;; --------------- Function definition ------------

;;;Stolen from mouse.el
(defun pgp-mouse-choose (event)
  "Click on an alternative in the `*Completions*' buffer to choose it."
  (interactive "e")

;;; --------------- End function identification ----
\end{verbatim}
\subsection{Function {\tt pgp-clear-sign}}
\leavevmode
\begin{verbatim}
;;; --------------- Function identification ----------
;;;| Function name         : pgp-clear-sign           |
;;;| Purpose               : Clear-sign a mail message|
;;;| Calling parameters    :                          |
;;;| Return value          : non significant          |
;;;--------------------------------------------------

;;; --------------- Cross calls ----------------------
;;;| Called functions     : pgp-general-sign          |
;;;| Calling functions    :                           |
;;;--------------------------------------------------

;;; --------------- Function definition ------------

;;;###autoload
(defun pgp-clear-sign ()
  "Clear-sign a mail message."
  (interactive)

;;; --------------- End function identification ----
\end{verbatim}
\subsection{Function {\tt pgp-sign}}
\leavevmode
\begin{verbatim}
;;; --------------- Function identification ----------
;;;| Function name         : pgp-sign                 |
;;;| Purpose               : Sign a mail message      |
;;;| Calling parameters    :                          |
;;;| Return value          : non significant          |
;;;--------------------------------------------------

;;; --------------- Cross calls ----------------------
;;;| Called functions     : pgp-general-sign          |
;;;| Calling functions    :                           |
;;;--------------------------------------------------

;;; --------------- Function definition ------------

;;;###autoload
(defun pgp-sign ()
  "Sign a mail message."
  (interactive)

;;; --------------- End function identification ----
\end{verbatim}
\subsection{Function {\tt pgp-general-sign}}
\leavevmode
\begin{verbatim}
;;; --------------- Function identification ----------
;;;| Function name         : pgp-general-sign         |
;;;| Purpose               : General function to sign a mail message
;;;| Calling parameters    :                          |
;;;|  clear: flag to clear-sign or not                |
;;;| Return value          : non significant          |
;;;--------------------------------------------------

;;; --------------- Cross calls ----------------------
;;;| Called functions     : pgp-create-output-buffer, pgp-get-passphrase,
;;;|  pgp-wait-process-exit, pgp-output-filter, pgp-show-status,
;;;| pgp-update-pass                                  |
;;;| Calling functions    : pgp-sign, pgp-clear-sign  |
;;;--------------------------------------------------

;;; --------------- Function definition ------------

(defun pgp-general-sign (clear)
  "General function to sign a mail message.
Clear is a boolean tha means if set to non-nil that a clear-signature should
be generated."

;;; --------------- End function identification ----
\end{verbatim}
\subsection{Function {\tt pgp-encrypt}}
\leavevmode
\begin{verbatim}
;;; --------------- Function identification ----------
;;;| Function name         : pgp-encrypt              |
;;;| Purpose               : Enrypt a mail message    |
;;;| Calling parameters    :                          |
;;;| Return value          : non significant          |
;;;--------------------------------------------------

;;; --------------- Cross calls ----------------------
;;;| Called functions     : pgp-append-pubkey, pgp-general-encrypt,
;;;|  pgp-create-temp-buffer                          |
;;;| Calling functions    :                           |
;;;--------------------------------------------------

;;; --------------- Function definition ------------

;;;###autoload
(defun pgp-encrypt ()
  "Encrypt a mail message."
  (interactive)

;;; --------------- End function identification ----
\end{verbatim}
\subsection{Function {\tt pgp-encrypt-view}}
\leavevmode
\begin{verbatim}
;;; --------------- Function identification ----------
;;;| Function name         : pgp-encrypt-view         |
;;;| Purpose               : Encrypt a mail message for view-only
;;;| Calling parameters    :                          |
;;;| Return value          : non significant          |
;;;--------------------------------------------------

;;; --------------- Cross calls ----------------------
;;;| Called functions     : pgp-create-temp-buffer, pgp-general-encrypt
;;;| Calling functions    :                           |
;;;--------------------------------------------------

;;; --------------- Function definition ------------

;;;###autoload
(defun pgp-encrypt-view ()
  "Encrypt a mail message for view-only."
  (interactive)

;;; --------------- End function identification ----
\end{verbatim}
\subsection{Function {\tt pgp-encrypt-region}}
\leavevmode
\begin{verbatim}
;;; --------------- Function identification ----------
;;;| Function name         : pgp-encrypt-region       |
;;;| Purpose               : Encrypt a mail message for viewthe region
;;;|                  defined between mark and point  |
;;;| Calling parameters    :                          |
;;;|  beg: the beginning of the region                |
;;;|  end: the end of the region                      |
;;;| Return value          : non significant          |
;;;--------------------------------------------------

;;; --------------- Cross calls ----------------------
;;;| Called functions     : pgp-create-temp-buffer, pgp-general-encrypt
;;;| Calling functions    :                           |
;;;--------------------------------------------------

;;; --------------- Function definition ------------

;;;###autoload
(defun pgp-encrypt-region (beg end)
  "Encrypt a mail message for viewthe region defined between mark and point."
  (interactive "r")

;;; --------------- End function identification ----
\end{verbatim}
\subsection{Function {\tt pgp-general-encrypt}}
\leavevmode
\begin{verbatim}
;;; --------------- Function identification ----------
;;;| Function name         : pgp-general-encrypt      |
;;;| Purpose               : General function to encrypt a mail message
;;;| Calling parameters    :                          |
;;;|  view: if set, encrypt for view only             |
;;;|  begreg, endreg: limit of the region to encrypt  |
;;;| Return value          : non significant          |
;;;--------------------------------------------------

;;; --------------- Cross calls ----------------------
;;;| Called functions     : pgp-list-pubring, pgp-list-recipient-id,
;;;|  pgp-extract-email-recipient, pgp-set-key, pgp-read-recipient,
;;;|  pgp-create-output-buffer, pgp-wait-process-exit, pgp-show-status
;;;| Calling functions    : pgp-encrypt, pgp-encrypt-region,
;;;| pgp-encrypt-view                                 |
;;;--------------------------------------------------

;;; --------------- Function definition ------------

(defun pgp-general-encrypt (view begreg endreg)
  "General function to encrypt a mail message.
View is a boolean that if set to non-nil means the message should
be crypted with view-only option. The message is crypted between the 
begreg and endreg."

;;; --------------- End function identification ----
\end{verbatim}
\subsection{Function {\tt pgp-append-pubkey}}
\leavevmode
\begin{verbatim}
;;; --------------- Function identification ----------
;;;| Function name         : pgp-append-pubkey        |
;;;| Purpose               : Append the public key to the mail message
;;;| Calling parameters    :                          |
;;;|  &status: display the status or not              |
;;;| Return value          : t or nil                 |
;;;--------------------------------------------------

;;; --------------- Cross calls ----------------------
;;;| Called functions     : pgp-create-output-buffer, pgp-get-seckey,
;;;|  pgp-wait-process-exit, pgp-show-status, pgp-create-temp-buffer
;;;| Calling functions    : pgp-encrypt               |
;;;--------------------------------------------------

;;; --------------- Function definition ------------

;;;###autoload
(defun pgp-append-pubkey (&optional status)
  "Append the public key to the mail message.
The public key is the one corresponding to the selected secret key. It is
appended at the current point. If status is set non-nil show the status
after the run."
  (interactive)

;;; --------------- End function identification ----
\end{verbatim}
\subsection{Function {\tt pgp-sign-encrypt}}
\leavevmode
\begin{verbatim}
;;; --------------- Function identification ----------
;;;| Function name         : pgp-sign-encrypt         |
;;;| Purpose               : Sign and crypt a mail message
;;;| Calling parameters    :                          |
;;;| Return value          : non significant          |
;;;--------------------------------------------------

;;; --------------- Cross calls ----------------------
;;;| Called functions     : pgp-create-output-buffer, pgp-get-passphrase,
;;;|  pgp-list-pubring, pgp-list-recipient-id, pgp-extract-email-recipient,
;;;|  pgp-get-seckey, pgp-read-recipient, pgp-wait-process-exit,
;;;|  pgp-show-status, pgp-update-pass                |
;;;| Calling functions    :                           |
;;;--------------------------------------------------

;;; --------------- Function definition ------------

;;;###autoload
(defun pgp-sign-encrypt ()
  "Sign and crypt a mail message."
  (interactive)

;;; --------------- End function identification ----
\end{verbatim}
\subsection{Function {\tt pgp-show-status}}
\leavevmode
\begin{verbatim}
;;; --------------- Function identification ----------
;;;| Function name         : pgp-show-status          |
;;;| Purpose               : Display the *PGP output* buffer
;;;| Calling parameters    :                          |
;;;|  str: specify the kind of output                 |
;;;| Return value          : non significant          |
;;;--------------------------------------------------

;;; --------------- Cross calls ----------------------
;;;| Called functions     : pgp-create-temp-buff, pgp-ro-protect,
;;;|  pgp-print-return-message                        |
;;;| Calling functions    : pgp-general-sign, pgp-general-encrypt,
;;;|  pgp-append-pubkey, pgp-sign-encrypt, pgp-decr-mess,
;;;|  pgp-add-key                                     |
;;;--------------------------------------------------

;;; --------------- Function definition ------------

(defun pgp-show-status (str)
  "Display the *PGP output* buffer.
Clean the buffer then display it. In case of view-only message, another
function is used. Str is used to switch the cleaning according to the 
performed operation.
Side effect in case of add key to the pubring: update the internal list."
  
;;; --------------- End function identification ----
\end{verbatim}
\subsection{Function {\tt pgp-update-pass}}
\leavevmode
\begin{verbatim}
;;; --------------- Function identification ----------
;;;| Function name         : pgp-update-pass          |
;;;| Purpose               : Save the pass phrase associated to a
;;;|                         secret key or wipe it out|
;;;| Calling parameters    :                          |
;;;| Return value          : non significant          |
;;;--------------------------------------------------

;;; --------------- Cross calls ----------------------
;;;| Called functions     :                           |
;;;| Calling functions    : pgp-general-sign, pgp-sign-encrypt,
;;;|  pgp-decrypt                                     |
;;;--------------------------------------------------

;;; --------------- Function definition ------------

(defun pgp-update-pass ()
  "Save the pass phrase associated to a secret key or wipe it out.
If pgp-delete-pass is set to non-nil the pass phrase is wiped out after
each run of pgp, else it is kept during a session of Emacs. There is a
security hole if the pass phrase is in memory of Emacs when it crashes,
it will appear in the core file."

;;; --------------- End function identification ----
\end{verbatim}
\subsection{Function {\tt pgp-print-return-message}}
\leavevmode
\begin{verbatim}
;;; --------------- Function identification ----------
;;;| Function name         : pgp-print-return-message |
;;;| Purpose               : Display or return message saying how to restore
;;;|                         windows after PGP command|
;;;| Calling parameters    :                          |
;;;| Return value          : non significant          |
;;;--------------------------------------------------

;;; --------------- Cross calls ----------------------
;;;| Called functions     :                           |
;;;| Calling functions    : pgp-show-status           |
;;;--------------------------------------------------

;;; --------------- Function definition ------------

;;;Stolen from help.el
(defun pgp-print-return-message ()
  "Display or return message saying how to restore windows after PGP command.
Computes a message and print it."

;;; --------------- End function identification ----
\end{verbatim}
\subsection{Function {\tt pgp-decrypt}}
\leavevmode
\begin{verbatim}
;;; --------------- Function identification ----------
;;;| Function name         : pgp-decrypt              |
;;;| Purpose               : Decrypt a pgp message    |
;;;| Calling parameters    :                          |
;;;|  buffer: the buffer to read the message from     |
;;;|  begreg, endreg: the region to decrypt           |
;;;| Return value          : t or nil                 |
;;;--------------------------------------------------

;;; --------------- Cross calls ----------------------
;;;| Called functions     : pgp-wait-process-exit, pgp-update-pass
;;;| Calling functions    : pgp-decr-mess             |
;;;--------------------------------------------------

;;; --------------- Function definition ------------

(defun pgp-decrypt (buffer begreg endreg)
  "Decrypt a pgp message.
The message to decrypt is in buffer between begreg and endreg points, this
allows recursive calls. Return non-nil if the message have been scanned
by pgp."

;;; --------------- End function identification ----
\end{verbatim}
\subsection{Function {\tt pgp-decr-mess}}
\leavevmode
\begin{verbatim}
;;; --------------- Function identification ----------
;;;| Function name         : pgp-decr-mess            |
;;;| Purpose               : Decrypt an Emacs buffer  |
;;;| Calling parameters    :                          |
;;;| Return value          : non significant          |
;;;--------------------------------------------------

;;; --------------- Cross calls ----------------------
;;;| Called functions     : pgp-list-pubring, pgp-list-secring,
;;;|  pgp-create-output-buffer, pgp-decrypt, pgp-show-status
;;;| Calling functions    :                           |
;;;--------------------------------------------------

;;; --------------- Function definition ------------

;;;###autoload
(defun pgp-decr-mess ()
  "Decrypt an Emacs buffer.
It can be called from RMAIL buffer or from the 2 output buffers."
  (interactive)

;;; --------------- End function identification ----
\end{verbatim}
\subsection{Function {\tt pgp-add-key}}
\leavevmode
\begin{verbatim}
;;; --------------- Function identification ----------
;;;| Function name         : pgp-add-key              |
;;;| Purpose               : Extract key blocs and update plublic keyring
;;;| Calling parameters    :                          |
;;;| Return value          : non significant          |
;;;--------------------------------------------------

;;; --------------- Cross calls ----------------------
;;;| Called functions     : pgp-list-pubring, pgp-create-output-buffer,
;;;|  pgp-wait-process-exit, pgp-show-status          |
;;;| Calling functions    :                           |
;;;--------------------------------------------------

;;; --------------- Function definition ------------

;;;###autoload
(defun pgp-add-key ()
  "Extract key blocs and update plublic keyring.
It also update the list of public keys accordingly. It can be called from
RMAIL buffer or from the 2 output buffer."
  (interactive)

;;; --------------- End function identification ----
\end{verbatim}
