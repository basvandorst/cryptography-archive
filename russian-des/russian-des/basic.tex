% basic.tex - this file contains a variety of things that seem
	% so useful one wonders why they are not in plain.tex

		%% Send to Teletype

% send a message to the teletype.  This function has no effect 
% on the document

	\def\sendtty#1{\immediate\write16{#1}}

% \spc and \tab expand to one space and eight spaces respectively
% for use in \sendtty messages to the console and log file

	\xdef\spc{ } \xdef\tab{\spc\spc}
	\xdef\tab{\tab\tab} \xdef\tab{\tab\tab}


% The conditional \ifnull tests whether a string is empty

	\def\ifnull
	#1% string to be tested
	{\setbox\tempbox\hbox{#1}%
	 \expandafter\ifdim\wd\tempbox=0pt\relax}

% and follows the normal syntax of TeX conditionals.  The actual
% test is whether the argument has zero width when put in a box.
% This is not guaranteed to work if the argument can be something
% other than a string.  In particular, it will be fooled by a
% non-empty box of zero width.

% The conditional \ifundefined tests whether a string has a

\def\ifundefined#1{\expandafter\ifx\csname#1\endcsname\relax}

% macro definition and also follows standard conditional syntax.


% The functions \Alph and \alph generate lower and uppercase letters
% from numbers between 1 and 26.  They are analogous to \roman and
% \Roman.

\def\alph
#1% number between 1 and 26
{\ifcase #1 \or a\or b\or c\or d\or e\or f\or g\or h\or 
	    i\or j\or k\or l\or m\or n\or o\or p\or q\or 
	    r\or s\or t\or u\or v\or w\or x\or y\or z\fi}

\def\Alph
#1% number between 1 and 26
{\ifcase #1 \or A\or B\or C\or D\or E\or F\or G\or H\or 
	    I\or J\or K\or L\or M\or N\or O\or P\or Q\or 
	    R\or S\or T\or U\or V\or W\or X\or Y\or Z\fi}

\def\enddoc{\par \vfill \supereject \end}
	% \enddoc can be redefined by a document
	% format to preform some additional action
	% before ending.


	%%% Boxes and glue macros

\def\pinch#1{\hbox to 0pt{#1}}

\def\lft#1{#1\hss} \def\rt#1{\hss#1} \def\ctr#1{\hss#1\hss}

\def\llap#1{\pinch{\rt{#1}}} \def\rlap#1{\pinch{\lft{#1}}} 
\def\clap#1{\pinch{\ctr{#1}}}

\def\crush#1{\vbox to 0pt{#1}}
\def\tp#1{\vss#1} % vertical analog of \lft 
\def\bt#1{#1\vss} % vertical analog of \rt
\def\vctr#1{\vss#1\vss} % vertical analog of \ctr
\def\tlap#1{\crush{\tp{#1}}} % vertical analog of \llap
\def\blap#1{\crush{\bt{#1}}} % vertical analog of \rlap
\def\vclap#1{\crush{\vctr{#1}}} % vertical analog of \clap
\def\contract#1{\vclap{\clap{#1}}}


\newcount\tempcount
\newbox\tempbox

\def\note#1{}		% a manuscript note that does not appear
			% in the document

% At numerous places in the text, the quality is improved by
% giving TeX advice that it should not have needed.  This is
% termed a hint.

		\long\def\hint#1{#1}

% and has no function except to label the hint so it can be 
% found.

		% date and time macro

\def\monthname
   {\ifcase \month \or January\or February\or March\or
	 April\or May\or June\or July\or August\or 
	 September\or October\or November\or December \fi}

\def\setdatetime
   {\tempcount=\time 
    \divide\tempcount by 60 \multiply\tempcount by 60
    \global\advance\time by-\tempcount 
    \xdef\minute{\expandafter\the\time}%
    \advance \time by \tempcount
    \divide \tempcount by 60%
    \xdef \hour{\expandafter\the\tempcount}}
\setdatetime

\xdef\datetime{\the\day\monthname\the\year-\hour:\ifnum\minute<10 0\fi\minute}

		% Default Parameters

\hfuzz=1pt \tolerance=10000 \hbadness=10000 \vbadness=10000

% The \overfullrule parameter marks overfull boxes with an ugly
% black rule.  This is useful in debugging, it is much better
% to forget that it was turned off than to forget that it was
% turned on.

\overfullrule=0pt 	% don't mark overfull boxes with black slugs

% to be integrated

\let\ty=\tt	% Used in lots of place

% \endpar is made a synonym for TeX's standard \par control sequence.
% This leaves the conveniently short sequence \par to be redefined
% to do something more complex at the end of a paragraph.

		\let \endpar=\par 

% The commands \hmode and \vmode are named to suggest exactly what they
% do, force TeX into horizontal mode and vertical mode respectively.

\let\hmode=\leavevmode		% Force \TeX into horizontal mode.
\let\vmode=\endpar		% Force \TEX into vertical mode.

