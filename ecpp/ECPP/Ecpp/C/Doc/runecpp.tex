\documentstyle[fullpage,11pt]{report}
\title{ECPP: the final version}
\author{Fran\c{c}ois Morain}

\begin{document}

\maketitle

\section{The $N-1$ test}

 \subsection{Pratt}

Following some discussion with Ilan Vardi, it appears necessary to end
any certificate with a Pratt structure for the last prime, including
the very small primes ($2, 3, \ldots$). In order to use only one type
of data structure, \verb+(Certif) cert+, the Pratt tree is converted onto a
list of numbers that are stored in \verb+cert.lfacto+. For example,
consider the following tree:

\begin{center}
\setlength{\unitlength}{1mm}
\begin{picture}(50, 50)
\put(25, 50){\makebox(0,0){$(53, 2)$}}
 \put(23, 47){\line(-1, -1){8}}
 \put(13, 37){\makebox(0,0){$(2, 1)$}}
 \put(27, 47){\line( 1, -1){8}}
 \put(37, 37){\makebox(0,0){$(13, 2)$}}
  \put(30, 30){\makebox(0,0){$(2, 1)$}}
  \put(50, 30){\makebox(0,0){$(3, 2)$}}
   \put(50, 20){\makebox(0,0){$(2, 1)$}}
\end{picture}
\end{center}

\section{The $N+1$ test}

\section{Elliptic curves}

\section{Certificates}

 \subsection{Formats}

  \subsubsection{C}
  \subsubsection{Maple}
  \subsubsection{Mathematica}


\end{document}
